This paper discussed the challenges and importance of supporting early safety analysis of IoT systems, which are susceptible to various failures that can impact their functionality and the environment they operate in. Failure propagation within these systems is complex due to their diverse components and distributed nature. To address this, the paper discusses using Failure-Logic Analysis (FLA) to understand how component failures may propagate and affect the system's behavior. However, FLA relies on manually specified rules, which can be error-prone and incomplete. The paper proposes adopting testing methodologies to mitigate the issues with manually specified FLA rules. Potential faults can be observed and identified by subjecting the IoT system to various test cases. By means of the proposed testing techniques, it is possible to support the validation of the correctness of the system's behavior and the effectiveness of the specified rules in capturing fault scenarios. Future plans include the support of all the fault types that can be specified at the level of IoT system modeling. Moreover, we intend to investigate the generalizability of the proposed technique by considering different execution environments than Proteus.