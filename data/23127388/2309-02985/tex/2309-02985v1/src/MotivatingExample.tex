
%\subsection{Motivation example}

Figure \ref{fig:motivating} represents a Smart Irrigation System (SIS) that includes all the building blocks of a typical \iot system, \ie actuators, monitors, and sensors. The  system analyzes the environmental conditions to automatically irrigate the soil using the classical MAPE-K control loop \cite{1160055,Brun2009}. In particular, each node (represented by the dashed line in Figure \ref{fig:motivating}) is composed of different types of sensors, \ie \textit{Moisture}, \textit{Temperature}, and \textit{Humidity}.  


\begin{figure}
    \centering
    \includegraphics[width=0.9\linewidth]{img/SmartIrrigation.pdf}
    
    \caption{The Smart Irrigation System use case.}
    \label{fig:motivating}
\end{figure}

Such sensors collect data at a given node and continuously feed it to the \textit{Computing board}. Based on sensor data, the board decides whether to send a signal straight to the \textit{Irrigation unit} actuator to start or stop the watering process. When the irrigation phase is ended, the \textit{LED} indicators switch from green to red. 
The \textit{Physical gateway} connects each irrigation node to the \textit{Cloud server}, allowing users to remotely control, via \textit{Mobile phone}, the irrigation nodes and analyze sensor data.
Even though the presented system is simple, it represents a real-world application composed of miscellaneous \iot components that can be prone to critical malfunctioning. For instance, the Moisture sensor can send the wrong value, thus causing a waste of water or loss to the farm's production. Similarly, the user can erroneously decide to irrigate the field if the LED is malfunctioning. 
Therefore, failure propagation analysis plays an important role in understanding the system's behavior when it suffers from those faults.  

%\textbf{We need to add the reference to the Felicien thesis} 
An early-safety analysis approach has been proposed  by Ihirwe\cite{thesisFelicien},  by relying on Failure-Logic Analysis (FLA)~\cite{Xing2008} mechanisms. %is an approach that 
FLA allows modelers to specify a component's failure logic behavior to help analyze how failures propagate within a system to anticipate possible misbehaviors. %and take corrective actions. 
To be effective, FLA requires accurate knowledge about how failure may behave within the individual components. This can either be by means of propagation or transformation across components. 

%However, due to a wide range of circumstances, the system failure behaviors anticipated by FLA may not accurately reflect how the system behaves once it is fully operational.

Figure \ref{fig:workflow} depicts the failure analysis workflow underpinning the approach proposed by Ihirwe \cite{thesisFelicien}. 
% \textbf{Davide: Felicien thesis}. 
First, the system is modeled by identifying all the needed components and communication channels. Afterward, the user has to check the system's safety by performing a proper failure propagation analysis. 
It is worth noting that the two phases are typically conducted manually with no or limited degree of automation \cite{8411738,6721820}.

\begin{figure}
    \centering
    \includegraphics[width=0.6\linewidth]{img/TestingWorkflow_V2.pdf}
    \vspace{-4mm}
    \caption{Traditional failure analysis workflow.}
    \label{fig:workflow}
\end{figure}


However, detecting those faults is a daunting task since thoroughly exercising an \iot system requires considering both the system and its environment. Therefore, a task of paramount importance is to detect how failures may propagate using early-safety analysis strategies. Even though several frameworks and techniques are in place~\cite{cristea_building_2022,bures_patriot_2021,215955}, there is a need to verify the correctness of such rules at design time. Thus, the main challenges that need to be addressed when modeling IoT systems while supporting early-safety analysis are as follows:

\begin{itemize}[leftmargin=*]
    \item \textbf{CH1: Detecting fault propagation in \iot systems}
    While fault analysis has been studied in generic software systems \cite{Xing2008}, detecting failures in \iot systems has to consider real-time data that may introduce variability in the conducted analysis. Furthermore, failures that occur at the circuits-level should be considered in the analysis as they cause bugs that impact the source code \cite{10.1145/2858036.2858533,9402092};
    
    \item \textbf{CH2: Verifying the completeness and correctness of fault propagation rules:} Even though fault propagation rules can be specified at the design time, their completeness and correctness cannot be granted \textit{a priori}. %Synthesizing automatically such rules allows the users to overcome these issues even at the early stage of development, \ie the design phase. 
\end{itemize}

%The results show that the most challenging aspects of IoT development are related to the security of the employed TPLs, the device constraints, and handling the failures. 

%The conducted study highlights the need of having in-context advice during the development of the \iot project.  



%can be extremely challenging, further than being of high importance.

%Consider, for instance, Smart Irrigation Systems: they monitor parameters related to weather and soil to automatically irrigate crop fields based on the %the levels of data collected. 



%Assessing the failures in a controlled environment is critical to analyzing the system's behavior and possibly devising countermeasures.

 
 %A farmer can view the data collected in real time. A farmer can also control the entire process from his smartphone, including starting, stopping, as well as calibrating the threshold levels.
 %irrigation system are put at various locations throughout the farm to perceive and analyze data independently.The green LED indicator at the node shows that the system node is not irrigating at the time, whereas the red LED indicates that the node is watering the site.