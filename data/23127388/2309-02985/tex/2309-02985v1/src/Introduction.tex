%IoT systems are subject to different kinds of failures, either being internally generated %from the system 
%or caused by the surrounding environment. Such failures may affect not only the correctness of the system but also the environment in which they operate. 

%Assigned to DAVIDE 
IoT systems may experience different failures, either internally generated or caused by the surrounding environment~\cite{KIRCHHOF2022111087}. Such failures may affect not only the correctness of the system, but also the environment in which it operates. Consider, for instance, Smart Irrigation Systems: they monitor parameters related to weather and soil to irrigate crop fields based on the data collected automatically. A failure affecting the behaviour of those IoT systems may cause a waste of water or loss to the farm's production. Since IoT systems are composed of components of different natures (e.g., temperature/humidity sensors, cloud servers, and irrigation units), studying how failures (e.g., caused by a malfunctioning component) may propagate within a system and impact its behaviour can be highly challenging, further than being of high importance \cite{thesisFelicien,challengesIOTFLA}.

Developing IoT systems is complex due to several reasons. The integration of diverse components, the need to handle real-time data, and the distributed nature of IoT systems are just a few factors contributing to this complexity. Several modeling approaches have been proposed over the last few years to raise the level of abstraction (e.g., \cite{MDE4IoT, CAPS, KIRCHHOF2022111087, MonitorIoT}), promoting the adoption of models for increasing automation and easing analysis. These models help understanding systems behaviour, performance, and potential failure scenarios~\cite{ihirwe2021towards}.

Failure-Logic Analysis (FLA)~\cite{Gallina} is one of the analyses that can be applied to IoT systems. By using FLA, it is possible to define how a component's failure logic shall behave, which can help analyze how failures could potentially spread throughout a system and predict any potential issue. For FLA to work correctly, it is important to have accurate information about how failures may occur within each  component and propagate between components.
FLA relies on the manual specification of rules, that rigorously indicate the different kinds of failures that might occur and how they can propagate throughout the components. Specifying such rules is a strenuous and error-prone process, as identifying all possible fault scenarios and formulating accurate rules is challenging, possibly leading to incomplete or incorrect specifications.

In this paper, we propose to adopt testing methodologies to mitigate the issues related to the \emph{completeness} and \emph{correctness} of the manually specified FLA rules. By systematically introducing failures into an IoT system and running test cases, it is possible to observe how failures propagate. The collected evidence is then used to add, refine, and eliminate FLA rules, better capturing the behavior of the system in failure scenarios.

The paper is organized as follows: In Section \ref{sec:motivatingExample} we provide motivation for this work and present an explanatory example. We describe our approach in Section~\ref{sec:proprosedApproach}. Section~\ref{sec:evaluation} reports a preliminary evaluation of the proposed approach. In Section~\ref{sec:relatedWork}, we discuss related work. Finally, Section~\ref{sec:conclusion} concludes the paper and outlines future work.


