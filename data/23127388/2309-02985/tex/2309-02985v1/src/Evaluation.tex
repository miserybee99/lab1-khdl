
% \subsection{SIS system design and safety analysis}

% \subsubsection{System modeling}


% \subsubsection{SIS Failure Logic Analysis}

% \subsubsection{SIS Fault-Tree Analysis}

%\subsection{SIS model-based testing}
To evaluate our approach, we selected the Irrigation unit component, introduced in Figure \ref{fig:testing_process} and shown in more details in Figure~\ref{fig:isolated}. %Once isolated, 
The component, designed with Proteus, comprises two main circuit streams that decide the activation of two water fans and two LEDs (each circuit controls a water fan and a LED). Once isolated, the input signals of the circuit are generated by the \textit{GEN\_IRRIGATION} stub (red circle in Figure~\ref{fig:isolated}), while the output signals are linked to the \textit{PROBE\_IRRIGATION} probes that record the data produced by the circuits (blue circles in Figure~\ref{fig:isolated}).

We use the base execution described in Figure~\ref{fig:input-output} as a basis to study failure propagation. In this scenario, \textit{both the water fans and LEDs start as turned off, then they are turned on for a fixed timespan by an external request, and finally they are turned off again}. %We instantiate the \textit{injector} accordingly to the scenario, and we saved the base execution outcome for later comparisons against the mutated executions. 
As mutated scenarios, we considered every possible combination of the currently supported failure types (i.e., \texttt{Early}, \texttt{Late}, \texttt{ValueCoarse}, and \texttt{ValueSubtle}, from Table \ref{tab:flatypes}) for the two input ports available in the component. Every combination was repeated 3 times, in order to capture variations in the executions, for a total of 48 different mutated scenarios produced by our \textit{injectors}.
We injected each combination separately, ran the mutated scenario, and compared the results via \textit{detectors} to discover how failures propagate. 

%For instance, to inject a Late failure type on an input port, we mutated the original time series by sending some input data after an acceptable time threshold; the output of each port was then compared to the original output to detect any exceeding delay from the expected output messages. Instead, to inject a ValueCoarse failure type, we mutated the value sent from an input port in order to violate some expected range of values for that port, whereas the discovery was conducted by observing any exceeding differences in values returned by the output ports with respect to the original behavior. 

The experiment's results are summarized in Table \ref{tab:test-results}. The column labeled \emph{IN Failures} shows the combinations of failures that were injected into the two input ports, while the column labeled \emph{OUT Failures} displays the combination of failures that were observed on the two output ports. As each combination of input failures was executed multiple times, it was possible to observe various combinations of failures on the output. For example, when the first and second unit ports were given the failures \texttt{Early} and \texttt{ValueSubtle}, respectively, two possible combinations of output failures were identified: \texttt{Early-Late} and \texttt{Early-NoFailure}.

The \textcolor{OliveGreen}{\texttt{green color}} highlights a failure type that propagates unchanged from the input to the output. The \textcolor{Red}{\texttt{red color}} indicates a failure type that is transformed as a failure of a different type. Finally, the \textcolor{Blue}{\texttt{blue color}} determines failures that are masked by the implementation and thus do not propagate to the output.

\begin{table*}[]
\centering
\caption{Failures Propagation in Irrigation unit experiment.}
\begin{tabular}{c|c}
\hline
\textbf{IN Failures}            &  \textbf{OUT Failures}                                         \\ \hline
\multicolumn{1}{l|}{\texttt{Early - Early}}             &  \multicolumn{1}{l}{\textcolor{OliveGreen}{\texttt{Early}} \texttt{-} \textcolor{OliveGreen}{\texttt{Early}}}                     \\ 

\multicolumn{1}{l|}{\texttt{Early - Late}}              & \multicolumn{1}{l}{\textcolor{OliveGreen}{\texttt{Early}} \texttt{-} \textcolor{OliveGreen}{\texttt{Late}}}                      \\ 

\multicolumn{1}{l|}{\texttt{Early - ValueCoarse}}       & \multicolumn{1}{l}{\textcolor{OliveGreen}{\texttt{Early}} \texttt{-} \textcolor{Red}{\texttt{Late}}, \textcolor{OliveGreen}{\texttt{Early}} \texttt{-} \textcolor{Blue}{\texttt{NoFailure}}}                      \\ 

\multicolumn{1}{l|}{\texttt{Early - ValueSubtle}}       & \multicolumn{1}{l}{\textcolor{OliveGreen}{\texttt{Early}} \texttt{-} \textcolor{Red}{\texttt{Late}}, \textcolor{OliveGreen}{\texttt{Early}} \texttt{-} \textcolor{Blue}{\texttt{NoFailure}}}                      \\ 

\multicolumn{1}{l|}{\texttt{Late - Early}}              & \multicolumn{1}{l}{\textcolor{OliveGreen}{\texttt{Late}} \texttt{-} \textcolor{OliveGreen}{\texttt{Early}}}                      \\ 

\multicolumn{1}{l|}{\texttt{Late - Late}}               & \multicolumn{1}{l}{\textcolor{OliveGreen}{\texttt{Late}} \texttt{-} \textcolor{OliveGreen}{\texttt{Late}}}                      \\ 

\multicolumn{1}{l|}{\texttt{Late - ValueCoarse}}        & \multicolumn{1}{l}{\textcolor{OliveGreen}{\texttt{Late}} \texttt{-} \textcolor{Red}{\texttt{Late}}, \textcolor{OliveGreen}{\texttt{Late}} \texttt{-} \textcolor{Blue}{\texttt{NoFailure}}}                      \\ 

\multicolumn{1}{l|}{\texttt{Late - ValueSubtle}}        & \multicolumn{1}{l}{\textcolor{OliveGreen}{\texttt{Late}} \texttt{-} \textcolor{Red}{\texttt{Late}}, \textcolor{OliveGreen}{\texttt{Late}} \texttt{-} \textcolor{Blue}{\texttt{NoFailure}}}                      \\ 

\multicolumn{1}{l|}{\texttt{ValueCoarse - Early}}       & \multicolumn{1}{l}{\textcolor{Red}{\texttt{Late}} \texttt{-} \textcolor{OliveGreen}{\texttt{Early}}, \textcolor{Blue}{\texttt{NoFailure}} \texttt{-} \textcolor{OliveGreen}{\texttt{Early}}}                      \\ 

\multicolumn{1}{l|}{\texttt{ValueCoarse - Late}}        & \multicolumn{1}{l}{\textcolor{Red}{\texttt{Late}} \texttt{-} \textcolor{OliveGreen}{\texttt{Late}}, \textcolor{Blue}{\texttt{NoFailure}} \texttt{-} \textcolor{OliveGreen}{\texttt{Late}}}                      \\ 

\multicolumn{1}{l|}{\texttt{ValueCoarse - ValueCoarse}} & \multicolumn{1}{l}{\textcolor{Blue}{\texttt{NoFailure}} \texttt{-} \textcolor{Red}{\texttt{Late}}, \textcolor{Red}{\texttt{Late}} \texttt{-} \textcolor{Blue}{\texttt{NoFailure}}, \textcolor{Red}{\texttt{Late}} \texttt{-} \textcolor{Red}{\texttt{Late}}, \textcolor{Blue}{\texttt{NoFailure}} \texttt{-} \textcolor{Blue}{\texttt{NoFailure}}}                      \\ 

\multicolumn{1}{l|}{\texttt{ValueCoarse - ValueSubtle}} & \multicolumn{1}{l}{\textcolor{Blue}{\texttt{NoFailure}} \texttt{-} \textcolor{Red}{\texttt{Late}}, \textcolor{Red}{\texttt{Late}} \texttt{-} \textcolor{Blue}{\texttt{NoFailure}}, \textcolor{Red}{\texttt{Late}} \texttt{-} \textcolor{Red}{\texttt{Late}}, \textcolor{Blue}{\texttt{NoFailure}} \texttt{-} \textcolor{Blue}{\texttt{NoFailure}}}                      \\ 

\multicolumn{1}{l|}{\texttt{ValueSubtle - Early}}       & \multicolumn{1}{l}{\textcolor{Red}{\texttt{Late}} \texttt{-} \textcolor{OliveGreen}{\texttt{Early}}, \textcolor{Blue}{\texttt{NoFailure}} \texttt{-} \textcolor{OliveGreen}{\texttt{Early}}}                      \\ 

\multicolumn{1}{l|}{\texttt{ValueSubtle - Late}}        & \multicolumn{1}{l}{\textcolor{Red}{\texttt{Late}} \texttt{-} \textcolor{OliveGreen}{\texttt{Late}}, \textcolor{Blue}{\texttt{NoFailure}} \texttt{-} \textcolor{OliveGreen}{\texttt{Late}}}                      \\ 

\multicolumn{1}{l|}{\texttt{ValueSubtle - ValueCoarse}} & \multicolumn{1}{l}{\textcolor{Blue}{\texttt{NoFailure}} \texttt{-} \textcolor{Red}{\texttt{Late}}, \textcolor{Blue}{\texttt{NoFailure}} \texttt{-} \textcolor{Blue}{\texttt{NoFailure}}, \textcolor{Red}{\texttt{Late}} \texttt{-} \textcolor{Red}{\texttt{Late}}, \textcolor{Red}{\texttt{Late}} \texttt{-} \textcolor{Blue}{\texttt{NoFailure}}}                      \\ 

\multicolumn{1}{l|}{\texttt{ValueSubtle - ValueSubtle}} & \multicolumn{1}{l}{\textcolor{Blue}{\texttt{NoFailure}} \texttt{-} \textcolor{Red}{\texttt{Late}}, \textcolor{Blue}{\texttt{NoFailure}} \texttt{-} \textcolor{Blue}{\texttt{NoFailure}}, \textcolor{Red}{\texttt{Late}} \texttt{-} \textcolor{Red}{\texttt{Late}}, \textcolor{Red}{\texttt{Late}} \texttt{-} \textcolor{Blue}{\texttt{NoFailure}}}                      \\ 

\end{tabular}
\label{tab:test-results}
\vspace{-3mm}
\end{table*}


Notably, \texttt{Early} and \texttt{Late} failures are always propagated to output ports, whereas \texttt{ValueSubtle} and \texttt{ValueCoarse} produce different outcomes depending either on the position within the time series of the value affected by the mutation or the magnitude of the mutation. In fact, when a wrong voltage value, injected as either a \texttt{ValueSubtle} or a \texttt{ValueCoarse},  occurs at the beginning of the time series, changing the original value to a value near or below 0, the component responsible for turning on the water fans and the LEDs is markedly delayed, resulting in the detection of a \texttt{Late} failure on output ports. 
Instead, when the mutated voltage value is set close to 5 Volts, i.e., the maximum accepted value provided by the injector in input according to the use case, or even higher, no notable changes are detected, resulting in a \texttt{NoFailure}. This is because the Irrigation unit component in Proteus is configured to flatten voltage values up to 5 Volts. Interestingly, depending on the context of the failure and the specific value, it may either mask the failure or transform the failure into a failure of a different kind.

These results contributed to improving the knowledge of the engineers about the fault tolerance of the system. In fact, the engineers' supposed failures would only propagate unchanged through the component, while failure propagation rules show more complicated, sometimes context-dependent, patterns. Further, the component could sometime mask the effect of the same failures. 
%
%For instance, referring to table \ref{tab:IU}, only one of the sample FLA rules proposed (last row), was actually covered by our testing mechanisms. This is mainly due to the fact that our testing approach is yet to support the "commission" and "omission" failure types which actually appear in the previous four rules in table \ref{tab}. Having the testing results as presented in Table \ref{tab:test-results}, would not only provide the user with much more clarity of different failure combinations that could potentially improve the correctness as well as the completeness of the FLA rules as well as the generated fault-trees  
%
%\textbf{FELICIEN}
For instance, by referring to Table \ref{tab:IUFLAtypes}, just one of the sample FLA rules (last row) was actually confirmed by our testing techniques. This is primarily because our testing approach has yet to cover the \texttt{Commission} and \texttt{Omission} failure types, which appear in the prior three rules in Table \ref{tab:IUFLAtypes}. Please note that not all the recommended tests listed in Table \ref{tab:test-results} will necessarily be used in the FLA analysis. However, having these results can provide the user with greater clarity regarding potential failure combinations, which can improve the accuracy and comprehensiveness of the FLA rules and generated fault trees.

%\textbf{LEO: shall we say something about how these discovered rules may fit into the FTA?}

%dire che questo e' dovuto un po' a come implementato componente e un po' a proteus. su altri tool e con altre configurazioni si potrebbe ottenere un risultato piu vicino alla propagazione in input. chiudere con investigazione sui parametri





