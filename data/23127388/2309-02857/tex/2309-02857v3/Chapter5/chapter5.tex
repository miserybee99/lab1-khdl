%\pagenumbering{arabic}

\chapter{Dynamics from Hamilton's principle}





\section{Hamilton's principle}
Let $q_{1}, q_{2},..., q_{n}$ be the independent coordinates of a holonomic dynamical system with $n$ degrees of freedom. Let us suppose that we have a point P in an ($n+1$)-dimensional space whose coordinates are ($q_{1}, q_{2},..., q_{n}, t$). This point is called a representative point of the configuration of the system. Let $P_{0}$ be the representative point of the configuration of the system at time $t=t_{0}$ and $P_{1}$ be the representative point at time $t=t_{1}$. Then as the system moves the representative point describes a curve $C$ joining the points $P_{0}$ and $P_{1}$. This curve is called the trajectory of the system. Let us suppose that we have another curve $C'$ in this space, passing through the points $P_{0}$ and $P_{1}$ and lying infinitely near to the curve $C$.
\begin{figure}
	\centering
	\includegraphics[width=0.4\textwidth]{Roshni1.pdf}\\
	\label{madhu1}
\end{figure}

All along the curve $C$, the equations of motion of the system are satisfied and we call it a dynamically possible curve. The curve $C'$ is geometrically possible but dynamically impossible i.e the values of $q_{1}, q_{2}, ..., q_{n}$ and $\dot{q_{1}}, \dot{q_{2}}, ..., \dot{q_{n}}$ as obtained from the curve $C'$ do not satisfy the equation of motion.

Let
\begin{equation}
S=\int_{t_{0}}^{t_{1}}\left(T+U\right)~dt=\int_{t_{0}}^{t_{1}}\left(T-V\right)~dt=\int_{t_{0}}^{t_{1}}~L~dt,\nonumber
\end{equation} where $T$ is the K.E of the system, $U$ is the force  function from which the external forces are derived and $L$ is the Lagrangian of the system. The function $S$ is called the Hamilton's principle function for the motion of the system. When we consider a case of $(n+1)$ dimensions the integral $S$ in an actual motion of the system is taken along the curve $C$ and in any other geometrically possible motion it is taken along the curve $C'$.

Hamilton's principle states that the principle function $S$ is stationary when the integral is taken along the actual trajectory of the system as compare to all other geometrically possible trajectories differ infinitesimally from the actual trajectory $C$ but having the same terminal points as the actual trajectory.

Of all possible motion of a mechanical system co-terminus in space and time, the actual motion is that for which
\begin{equation}
\delta\int_{t_{0}}^{t_{1}}~L~dt=0.\nonumber
\end{equation}
\section{Verification of Hamilton's principle from D'Alembert's principle}
According to D'Alembert's principle among all motions of the system consistent with the constraints the actual motion is that which satisfies at every moment the equation
\begin{equation}
\sum (m{\ddot{\overrightarrow{r}}}-\overrightarrow{F}). \delta \overrightarrow{r} =0.\nonumber
\end{equation} 
Now, $\sum \overrightarrow{F}.\delta \overrightarrow{r}=\delta U$, $U$ is the work function,
\begin{eqnarray}
 \mbox{and}~~~\sum m{\ddot{\overrightarrow{r}}} \delta \overrightarrow{r}&=&\dfrac{d}{dt}~(\sum m{\dot{\overrightarrow{r}}} \delta \overrightarrow{r})-\sum m\dot{\overrightarrow{r}}.\delta \dot{\overrightarrow{r}}\nonumber\\
 &=&\dfrac{d}{dt} (\sum m{\dot{\overrightarrow{r}}}.\delta \overrightarrow{r})-\delta T.\nonumber
 \end{eqnarray}
\begin{eqnarray}
 \therefore~~\sum \left(m{\ddot{\overrightarrow{r}}}-\overrightarrow{F}\right).\delta \overrightarrow{r}&=&\dfrac{d}{dt} ~\sum ( m{\dot{\overrightarrow{r}}} \delta \overrightarrow{r})-\delta T-\delta U\nonumber\\
 &=&\dfrac{d}{dt} (\sum m{\dot{\overrightarrow{r}}} \delta \overrightarrow{r})-\delta L.\nonumber
 \end{eqnarray}
\begin{eqnarray}
\mbox{ Hence}~~ \sum (m{\ddot{\overrightarrow{r}}}-\overrightarrow{F})\delta \overrightarrow{r}=0\nonumber\\
 \implies \dfrac{d}{dt}(\sum m \dot{\overrightarrow{r}}.\delta \overrightarrow{r})-\delta L=0.\nonumber
 \end{eqnarray} Now integrating both sides w.r.t $t$ from $t_{0}$ to $t_{1}$ we have
\begin{equation}
\int_{t_{0}}^{t_{1}}~\delta L~ dt=\int_{t_{0}}^{t_{1}}\dfrac{d}{dt}\left(\sum m~{\dot{\overrightarrow{r}}}\delta \overrightarrow{r}\right)dt=\sum m \dot{\overrightarrow{r}}\delta \overrightarrow{r}|^{t_{1}}_{t_{0}}=0\nonumber
\end{equation} as $\delta \overrightarrow{r}=0$ when $t=t_{0}, t_{1}.$

 $\therefore \delta\int_{t_{0}}^{t_{1}}Ldt=0$ i.e. $\delta S=0$, which is Hamilton's principle. \\
\section{Hamilton's equations from Hamilton's principle}
Let $(q_{1}, q_{2},...,q_{n})$ and $(q_{1}+\delta q_{1}, q_{2}+\delta q_{2},...,q_{n}+\delta q_{n})$ be the generalized coordinates of the system corresponding to the paths $C$ and $C_{1}$ at the same time $t$ with $\delta q_{i}=\epsilon \eta_{i}(t)$. By construction, $\delta q_{1},...,\delta q_{n}$ are quite arbitrary except that $\delta q_{_{i}}=0$ at $t=t_{0}$ and $t=t_{1}$, but there is no restrictions on $\delta\dot{ q_{i}}$'s. As $p_{i}=\dfrac{\partial L}{\partial \dot{ q_{i}}}$ is linear in $\dot{ q_{i}}$'s so $\delta p_{i}$'s are all arbitrary. Now
\begin{equation}
\delta S=\int_{t_{0}}^{t_{1}} \delta L~dt=\int_{t_{0}}^{t_{1}}\delta \left[\sum p_{i}~\dot{ q_{i}}-H\right]~dt\nonumber
\end{equation}
As $\dot{ q_{i}}=\dfrac{d}{dt}(\delta q_{i})$ (see appendix I) so we have
\begin{equation}
\delta \left(\sum p_{i}~\dot{ q_{i}}\right)=\dfrac{d}{dt}(\sum p_{i}~\delta q_{i})+\sum \dot{ q_{i}} \delta p_{i}-\sum_{i} \dot{p_{i}}~\delta q_{i},\nonumber
\end{equation} and $\delta H=\sum_{i} \dfrac{\partial H}{\partial p_{i}}~\delta p_{i}+\sum_{i}\dfrac{\partial H}{\partial q_{i}}~\delta q_{i}$. Also we have $\delta \dot{ q_{i}}=\dfrac{d}{dt}(\delta q_{i})$. Hence,
\begin{eqnarray}\nonumber
\delta S&=&\int_{t_{0}}^{t_{1}}\dfrac{d}{dt}(\sum p_{i} \delta q_{i})+\sum \dot{ q_{i}}\delta p_{i}-\sum \dot{p_{i}}\delta q_{i}-\sum \dfrac{\partial H}{\partial p_{i}}\delta p_i-\sum \dfrac{\partial H}{\partial q_{i}}\delta q_{i}\\&=&\int_{t_{0}}^{t_{1}}\sum_{i}\left[(q_{i}-\dfrac{\partial H}{\partial p_{i}})\delta p_{i}-(\dot{p_{i}}+\dfrac{\partial H}{\partial q_{i}})\delta q_{i}\right]~dt+\sum p_{i}\delta q_{i}|^{t_{1}}_{t_{0}}\nonumber\\&=&\int_{t_{0}}^{t_{1}}\sum_{i}\left[(\dot{ q_{i}}-\dfrac{\partial H}{\partial p_{i}})\delta p_{i}-(\dot{p_{i}}+\dfrac{\partial H}{\partial q_{i}})\delta q_{i}\right]dt \label{eq2..1}
\end{eqnarray} ~~~~~~~~~~~~~~~~~~~~~~~~~~~~~~~~~~~~~~~~~~~~~~~~~~~($\because \delta q_{i}=0$ at $t=t_{0}$ and $t=t_{1}$).

 If we assume Hamilton's equation of motion then R.H.S will be zero and hence $\delta S=0$. This verifies Hamilton's principle on the assumption of Hamilton's canonical equations.

\vspace{0.0cm}

On the other hand, if $\delta S=0$ for all arbitrary variation $\delta q_{i}$'s and $\delta p_{i}$'s then equation (\ref{eq2..1}) shows that the integral on the R.H.S vanishes for all these arbitrary variations. Thus we get $$\dot{ q_{i}}=\dfrac{\partial H}{\partial p_{i}}, \dot{p_{i}}=-\dfrac{\partial H}{\partial q_{i}}.$$ These are Hamilton's canonical equations. Thus Hamilton's canonical equations can be obtained from Hamilton's principle.\\
\section{Lagrange's equation from Hamilton's principle}
Let $q_{1}, q_{2},...,q_{n}$ be the independent coordinates of the holonomic dynamical system with $n$ degrees of freedom, moving under the action of external forces derived from a force function $U$ and let $L=T+U$ be the Lagrangian function of the system. Then the Hamilton's principle function is\\
$~~~~~~~~~~~~~~~~~~~~~~~~~~~~~~~~~~~~~~~~~~~~~~~~~~~~~S=\int_{t_{0}}^{t_{1}}L dt$,~~~~~~~~ $L=L(q,\dot{q},t)$.\\
Thus $\delta S=\int_{t_{0}}^{t_{1}}\delta L dt=\int_{t_{0}}^{t_{1}}\sum_{i}\left(\dfrac{\partial L}{\partial q_{i}}\delta q_{i}+\dfrac{\partial L}{\partial \dot{ q_{i}}}\delta\dot{q_i}\right)dt$.

 But $\delta \dot{ q_{i}}=\dfrac{d}{dt}(\delta q_{i})$, so
\begin{eqnarray}
\int_{t_{0}}^{t_{1}} \dfrac{\partial L}{\partial \dot{ q_{i}}}\delta \dot{ q_{i}}dt&=&\int_{t_{0}}^{t_{1}}\dfrac{\partial L}{\partial \dot{ q_{i}}}\dfrac{d}{dt}(\delta q_{i})dt\nonumber\\
&=&\int_{t_{0}}^{t_{1}}\dfrac{d}{dt}\left(\dfrac{\partial L}{\partial \dot{ q_{i}}}\delta q_{i}\right)dt-\int_{t_{0}}^{t_{1}}\delta q_{i}\dfrac{d}{dt}\left(\dfrac{\partial L}{\partial \dot{ q_{i}}}\right)dt\nonumber\\
&=&
\dfrac{\partial L}{\partial \dot{ q_{i}}} \delta q_{i}|^{t_{1}}_{t_{0}}-\int_{t_{0}}^{t_{1}}\dfrac{d}{dt}\left(\dfrac{\partial L}{\partial \dot{ q_{i}}}\right)\delta q_{i}dt\nonumber\\
&=&
-\int_{t_{0}}^{t_{1}}\dfrac{d}{dt}\left(\dfrac{\partial L}{\partial \dot{ q_{i}}}\right)\delta q_{i}dt,\nonumber
\end{eqnarray} $~~~~~~~~~~~~~~~~~~~~~~~~~~~~~~~~~~~~~~~~~~~~~~~~~~~~~~~~~~~~\because \delta q_{i}=0$ at $t=t_{0},t_{1}$.\\
$~~~~~~~~~~~~~~~~~~~~~~~~~~~~~~~~~~~~\therefore \delta S= \int_{t_{0}}^{t_{1}}\left[\sum_{i}\dfrac{\partial L}{\partial q_{i}}-\dfrac{d}{dt}\left(\dfrac{\partial L}{\partial \dot{ q_{i}}}\right)\right]\delta q_{i}dt.$\\
For $\delta S=0$, when $S$ is taken along the actual trajectory, it is necessary and sufficient that the integral on the R.H.S vanishes. Since $\delta q_{i}$'s are all arbitrary the necessary and sufficient condition for vanishing of $\delta S$ is that
\begin{equation}
\dfrac{\partial L}{\partial q_{i}}-\dfrac{d}{dt}\left(\dfrac{\partial L}{\partial \dot{ q_{i}}}\right)=0,~~~~~i=1,2,...,n. \nonumber
\end{equation} These are the Lagrange's equations of motion.\\
On the other hand, if we assume Lagrange's equations of motion then $\delta S=0$ when taken actual trajectory of the system as compared to any infinitely near trajectory so that Hamilton's principle is proved.\\
\section{Principle of least action:}
The action of a mechanical system is defined at the instant $t$ by the integral
\begin{equation}
A=\int_{t_{0}}^{t}\sum p\dot{q}dt,\nonumber
\end{equation} where $'p'$ is the generalized momentum corresponding to generalized coordinate $q$.\\
The Principle of least action states that the motion of a scelronomic conservative system between two prescribed configuration occurs in such a way that $\Delta A=0$, for the actual path as compared to the neighbouring paths provided the total energy of the system is the same constant value in the neighbouring motions as in the actual motion. Now, $A=\int_{t_{0}}^{t_{1}}\sum p~\dot{q}~dt$.\\
\begin{eqnarray}
\therefore\Delta A&=&\Delta\int_{t_{0}}^{t}\left(\sum p~\dot{q}\right)dt=\int_{t_{0}}^{t}\left[\delta (\sum p \dot{q})+\dfrac{d}{dt}(\sum p \dot{q})\Delta t\right]dt\nonumber\\
&=&\int_{t_{0}}^{t_{1}}\left[\sum p \delta \dot{q}+\sum \dot{q} \delta p\right]dt+\sum p \dot{q}\Delta t|^{t}_{t_{0}}.\nonumber
\end{eqnarray}
 We have $\Delta q=\delta q+\dot{q}\Delta t$ i.e. $\dot{q}\Delta t=\Delta q-\delta q.$
\begin{eqnarray}
\therefore\Delta  A&=&\sum p (\Delta q-\delta q)|^{t}_{t_{0}}+\int_{t_{0}}^{t}\left[\dfrac{d}{dt}\sum p\delta q-\sum \dot{p}\delta q\right]dt+\int_{t_{0}}^{t}\sum \dot{q}~\delta p ~dt\nonumber\\
&=&
\sum p(\Delta q-\delta q)|^{t}_{t_{0}}+(\sum p ~\delta q)|^{t}_{t_{0}}+\int_{t_{0}}^{t}\sum(\dot{q}~\delta p-\dot{p}~\delta q)dt\nonumber\\
&=&\sum p\Delta q|^{t}_{t_{0}}+\int_{t_{0}}^{t}\sum\left(\frac{\partial H}{\partial p}\delta p+\frac{\partial H}{\partial q}\delta q\right)dt\nonumber\\
&=&\sum p\Delta q|^{t}_{t_{0}}+\int_{t_{0}}^{t}\delta H~dt.\nonumber
\end{eqnarray}
As $\Delta q(t_{0})=\Delta q(t)=0$ and $H$ is the total energy which is conserved. So one gets $\Delta A=0$.\\
\section{Poisson Bracket}
Let $f(q,p,t)$ and $g(q,p,t)$ be two functions of the variables $q_{1},...,q_{n}$ and $p_{1},...,p_{n}$ and $t$. Then the expression
$$\sum_{i=1}^{n}\left(\dfrac{\partial f}{\partial q_{i}}~\dfrac{\partial g}{\partial p_{i}}-\dfrac{\partial f}{\partial p_{i}}~\dfrac{\partial g}{\partial q_{i}}\right)$$ is called the Poisson's bracket of the two quantities $f$ and $g$ with respect to the variables $q$ and $p$ and is denoted by $\{f,g\}_{(q,p)}$. 

\vspace{.25cm}

$\bullet$ \textbf{{Some important properties of Poisson bracket:}}\\
\textbf{~I.} $\{f,g\}=-\{g,f\}$\\
\textbf{II.} $\{f,c\}=0, c$ is any constant\\
\textbf{III.} $\{\sum_{i=1}^{n}a_{i}f_{i},g\}=\sum_{i=1}^{n}a_{i}\{f_{i},g\}$\\
\textbf{IV.} $\{f_{1}f_{2},g\}=f_{1}\{f_{2},g\}+f_{2}\{f_{1},g\}$\\
\textbf{V.} $\dfrac{\partial}{\partial t}\{f,g\}=\{\dfrac{\partial f}{\partial t},g\}+\{f,\dfrac{\partial g}{\partial t}\}$\\
$~~~~\dfrac{d}{dt}\{f,g\}=\{\dfrac{df}{dt},g\}+\{f,\dfrac{dg}{dt}\}$\\
\textbf{VI.} $\{f,q_{j}\}=-\dfrac{\partial f}{\partial p_{j}}, ~\{f,p_{j}\}=\dfrac{\partial f}{\partial q_{j}}$ \\
So $\{q_{i},q_{j}\}=0=\{p_{i},p_{j}\}, \{p_{i},q_{j}\}=-\delta_{ij}=\{q_{i},p_{j}\}$\\
Now if we write $H$ for $f$ then we get\\
$\dot{ q_{j}}=\dfrac{\partial H}{\partial p_{j}}=\{q_{j},H\}, \dot{p_{j}}=-\dfrac{\partial H}{\partial q_{j}}=\{p_{j},H\}$. These are the most symmetrical form of the canonical equations of motion of a mechanical system.\\
\textbf{VII. {Jacobi's identity:}} For any three functions $f,g$ and $h$ which are functions of $q,p,t$ the following identity holds:\\
$~~~~~~~~~~~~~~~~~~~~~~~~~~~~~~~~~~~~~~~~~~~~\{f,\{g,h\}\}+\{g,\{h,f\}\}+\{h,\{f,g\}\}=0$\\
Proof: By definition,
\begin{equation}
\{F,H\}=\sum_{i=1}^{n}\left(\dfrac{\partial F}{\partial q_{i}}~\dfrac{\partial H}{\partial p_{i}}-\dfrac{\partial F}{\partial p_{i}}~\dfrac{\partial H}{\partial q_{i}}\right)= D_{F}(H),\label{eq2..2}
\end{equation} where $D_{F}=\sum_{i=1}^{n}\left(\dfrac{\partial F}{\partial q_{i}}\dfrac{\partial}{\partial p_{i}}-\dfrac{\partial F}{\partial p_{i}}~\dfrac{\partial}{\partial q_{i}}\right)=\sum_{i=1}^{2n}F_{i}\dfrac{\partial}{\partial \xi_{i}}$\\
 with $\xi_{i}=p_{i},~\xi_{n+i}=q_{i},~F_{i}=\dfrac{\partial F}{\partial q_{i}},~F_{n+i}=-\dfrac{\partial F}{\partial p_{i}}$.\\
Now, $\{f,\{g,h\}\}+\{g,\{h,f\}\}=\{f,\{g,h\}\}-\{g,\{f,h\}\}=\{f,\lambda\}-\{g,\mu\}$, where $\lambda=\{g,h\}=D_{g}(h)$ and $\mu=\{f,h\}=D_{f}(h)$.

\begin{eqnarray}\therefore&& \{f,\{g,h\}\}+\{g,\{h,f\}\}+\{h,\{f,g\}\}\nonumber
	\\&=&D_{f}(\lambda)-D_{g}(\mu)=\sum_{j=1}^{2n}f_{j}\dfrac{\partial}{\partial \xi_{j}}(\lambda)-\sum_{i=1}^{2n}g_{i}\dfrac{\partial}{\partial \xi_{i}}(\mu)\nonumber\\
	&=&\sum_{j=1}^{2n}f_{j}\dfrac{\partial}{\partial \xi_{j}}\left(\sum_{i=1}^{2n}g_{i}~\dfrac{\partial h}{\partial \xi_{i}}\right)-\sum_{i=1}^{2n}g_{i}\dfrac{\partial}{\partial \xi_{i}}\left(\sum_{j=1}^{2n}f_{j}\dfrac{\partial h}{\partial \xi_{j}}\right)\nonumber\\
&=&\sum_{i,j=1}^{2n}\sum f_{j} \dfrac{\partial}{\partial \xi_{j}}\left(g_{i}\dfrac{\partial h}{\partial \xi_{i}}\right)-\sum_{i,j=1}^{2n}g_{i}\dfrac{\partial}{\partial \xi_{i}}\left(f_{j}\dfrac{\partial h}{\partial \xi_{j}}\right)\nonumber\\
&=&\sum_{i,j=1}^{2n}\left[f_{j}\left(\dfrac{\partial g}{\partial \xi_{j}}\dfrac{\partial h}{\partial \xi_{i}}+g_{i}\dfrac{\partial^{2}h}{\partial \xi_{i}\partial \xi_{j}}\right)-g_{i}\left(\dfrac{\partial f_{j}}{\partial \xi_{i}}\dfrac{\partial h}{\partial \xi_{j}}+f_{j}\dfrac{\partial^{2}h}{\partial \xi_{i}\partial \xi_{j}}\right)\right]\nonumber\\
&=&\sum_{i,j=1}^{2n}\left[f_{j}\dfrac{\partial g_{i}}{\partial \xi_{j}}\dfrac{\partial h}{\partial \xi_{i}}-g_{j}\dfrac{\partial f_{i}}{\partial \xi_{j}}\dfrac{\partial h}{\partial \xi_{i}}\right]\nonumber\\
&=&\sum_{i,j=1}^{2n}\left[f_{j}\dfrac{\partial g_{i}}{\partial \xi_{j}}\dfrac{\partial h}{\partial \xi_{i}}-g_{i}\dfrac{\partial f_{j}}{\partial \xi_{i}}\dfrac{\partial h}{\partial \xi_{j}}\right]\nonumber\\ 
&&\mbox{(interchanging $i$ and $j$ in the 2nd term. This is possible since $i,j$ are dummy indices)}\nonumber\\
&=&\sum_{i=1}^{2n}\dfrac{\partial h}{\partial \xi_{i}}\sum_{j=1}^{2n}\left(f_{j}\dfrac{\partial g}{\partial \xi_{j}}-g_{j}\dfrac{\partial f_{i}}{\partial \xi_{i}}\right)=\sum_{i=1}^{2n}C_{i}\dfrac{\partial h}{\partial \xi_{j}}\nonumber
\end{eqnarray} where,
$C_{i}=\sum_{j=1}^{2n}\left(f_{j}\dfrac{\partial g}{\partial \xi_{j}}-g_{j}\dfrac{\partial f_{i}}{\partial \xi_{j}}\right)$.\\
Therefore,
\begin{eqnarray}
\{f,\{g,h\}\}+\{g,\{h,f\}\}&=&\sum_{i=1}^{n}C_{i}\dfrac{\partial h}{\partial \xi_{i}}+\sum_{i=n+1}^{2n}C_{i}\dfrac{\partial h}{\partial \xi_{i}}\nonumber\\&=&\sum_{i=1}^{n}\left(A_{i}\dfrac{\partial h}{\partial p_{i}}+B_{i}\dfrac{\partial h}{\partial q_{i}}\right)\label{eq2..*}
\end{eqnarray} where $A$ and $B$ are functions of $f$ and $g$ and their derivatives but is independent explicitly of $h$. So when $h$ is varied $A$ and $B$ remain unchanged. Now putting $h=p_{j}$ in the above equation, the R.H.S becomes $A_{j}$ and L.H.S is
\begin{equation}
\{f,\{g,p_{j}\}\}+\{g,\{p_{j},f\}\}=\{f,\dfrac{\partial g}{\partial q_{j}}\}-\{g,\dfrac{\partial f}{\partial q_{j}}\}=\{f,\dfrac{\partial g}{\partial q_{j}},g\}=\dfrac{\partial}{\partial q_{j}}\{f,g\}\label{eq2..**}
\end{equation}
Similarly putting $h=q_{j}$, the R.H.S of (\ref{eq2..*}) becomes $B_{j}$ and the L.H.S is
\begin{eqnarray}
\{f,\{g,q_{j}\}\}+\{g,\{q_{j},f\}\}&=&\left\{f,-\dfrac{\partial g}{\partial p_{j}}\right\}+\left\{g,\dfrac{\partial f}{\partial p_{j}}\right\}\nonumber\\&=&-\left\{f,\dfrac{\partial g}{\partial p_{j}}\right\}-\left\{\dfrac{\partial f}{\partial p_{j}},g\right\}\nonumber\\&=&-\dfrac{\partial}{\partial p_{j}}\{f,g\}\label{eq2..***}
\end{eqnarray}
Putting the values of $A_{i}$'s and $B_{i}$'s from (\ref{eq2..**}) and (\ref{eq2..***}) in (\ref{eq2..*}) we have,\\
$\{f,\{g,h\}\}+\{g,\{h,f\}\}=\sum_{i=1}^{n}\left[\dfrac{\partial}{\partial q_{i}}\{f,g\}.\dfrac{\partial h}{\partial p_{i}}-\dfrac{\partial}{\partial p_{i}}\{f,g\}\dfrac{\partial h}{\partial q_{i}}\right]=-\{h,\{f,g\}\}$\\
$~~~~~~~~~~~~~~~~~~~~~~~~~~\therefore \{g,\{g,h\}\}+\{g,\{h,f\}\}+\{h,\{f,g\}\}=0$.
\section{Constants of motion}
Let $f(q,p,t)$ be any function of the dynamical variables $q$ and $p$ of a mechanical system defined by the Hamiltonian $H(q,p,t)$. Let us calculate the total derivative of $f$ with respect to $t$ i.e.\\
\begin{eqnarray}
\dfrac{df}{dt}&=&\dfrac{\partial f}{\partial t}+\sum_{i=1}^{n}\left(\dfrac{\partial f}{\partial q_{i}}\dot{ q_{i}}+\dfrac{\partial f}{\partial p_{i}}\dot{p_{i}}\right)\nonumber\\
	&=&\dfrac{\partial f}{\partial t}+\sum_{i=1}^{n}\left(\dfrac{\partial f}{\partial q_{i}}\dfrac{\partial H}{\partial p_{i}}-\dfrac{\partial f}{\partial p_{i}}\dfrac{\partial H}{\partial q_{i}}\right)\nonumber\\
	&=&\dfrac{\partial f}{\partial t}+\{f,H\}\nonumber.
\end{eqnarray} Those functions of the canonical variables which remain constant during the motion are called constants of motion. Thus the condition that $f(q,p,t)$ will be a constant of motion is that
\begin{equation}
\dfrac{df}{dt}=0 ~i.e~\dfrac{\partial f}{\partial t}+\{f,H\}=0.\nonumber
\end{equation} However, if $f$ is explicitly independent of $t$ then the condition simplifies to $\{f,H\}=0$.

\vspace{.25cm}

$\bullet$ \textbf{Note I:} If $f(q,p,t)$ and $g(q,p,t)$ be two constants of motion of a mechanical system then their Poisson bracket is also a constant of motion.
\begin{eqnarray}
\dfrac{d}{dt}\{f,g\}&=&\dfrac{\partial}{\partial t}\{f,g\}+\{\{f,g\},H\}\nonumber\\
&=&\left\{\dfrac{\partial f}{\partial t},g\right\}+\left\{f,\dfrac{\partial g}{\partial t}\right\}+\{f,\{g,H\}\}+\{g,\{H,f\}\}\nonumber\\
&&~~~~~~~~~~~~~~~~~~~~~~~~~~~~~~~~~~~~~~~~~~~~~~~\mbox{(by Jacobi's identity)}\nonumber\\
&=&\left\{\dfrac{\partial f}{\partial t}+\{f,H\},g\right\}+\left\{f,\dfrac{\partial g}{\partial t}+\{g,H\}\right\}\nonumber\\&=&\left\{\dfrac{df}{dt},g\right\}+\left\{f,\dfrac{dg}{dt}\right\}=\{0,g\}+\{f,0\}=0.\nonumber
\end{eqnarray} Thus given any two constants of motion we can generate all constants of motion by considering Poisson's bracket.

\vspace{.25cm}

$\bullet$ \textbf{Note II:} \textbf{Lagrange's bracket:} For any two quantities $f(q,p,t)$ and $g(q,p,t)$ the sum
\begin{equation}
\sum_{i=1}^{n}\left(\dfrac{\partial q_{i}}{\partial f}\dfrac{\partial p_{i}}{\partial g}-\dfrac{\partial p_{i}}{\partial f}\dfrac{\partial g_{i}}{\partial g}\right)\nonumber
\end{equation} is called the Lagrange's bracket of the two quantities $f$ and $g$ with respect to $q$ and $p$ and is denoted by $[f,g]_{q,p}$.\\
$\bullet$ \textbf{{Some properties of Lagrange's bracket:}}\\
{\textbf{I.}} $[f,g]=-[g,f]$\\
\textbf{II.} $[f,q_j]=-\dfrac{\partial p_j}{\partial f}$\\
\textbf{III.} $[f,p_j]=\dfrac{\partial q_j}{\partial f}$\\
\textbf{IV.} $[q_i,q_j]=0=[p_i,p_j]$\\
\textbf{V.} $[q_i,p_j]=\delta_{ij}$.

\vspace{.25cm}

$\bullet$ \textbf{{Connection  between two types of brackets:}}\\
Let $u_{i}(q,p,t) (i=1,2,,,,2n)$ be $2n$ independent functions of the variables $q_{1},q_{2},...,q_{n}$ and $p_{1},p_{2},...,p_{n}$ so that they can be solved to express $q_{1},q_{2},...,q_{n}$ as functions of $u,t$ i.e. $q_{i}=q_{i}(u,t)$ and similarly $p_{i}=p_{i}(u,t)$. Then we shall show that
\begin{equation}
\sum_{i=1}^{2n} \{u_{i},u_{j}\}[u_{i},u_{s}]=\delta_{js}\nonumber
\end{equation}
\textbf{Proof:} We have $\{u_i,u_j\}=\sum_{r=1}^{n}\left(\dfrac{\partial q_{r}}{\partial u_{i}}\dfrac{\partial p_{r}}{\partial u_{j}}-\dfrac{\partial q_{r}}{\partial u_{j}}\dfrac{\partial p_{r}}{\partial u_{i}}\right)$ and $[u_i,u_s]=\sum_{i=1}^{n}\left(\dfrac{\partial u_{i}}{\partial q_{k}}\dfrac{\partial u_{s}}{\partial p_{k}}-\dfrac{\partial u_i}{\partial p_k}\dfrac{\partial u_s}{\partial q_k}\right)$\\
Therefore L.H.S =
\begin{eqnarray}
&&\sum_{i=1}^{2n}\sum_{r=1}^{n}\left(\dfrac{\partial q_{r}}{\partial u_{i}}\dfrac{\partial p_{r}}{\partial u_{j}}-\dfrac{\partial q_{r}}{\partial u_{j}}\dfrac{\partial p_{r}}{\partial u_{i}}\right)\sum_{i=1}^{n}\left(\dfrac{\partial u_{i}}{\partial q_{k}}\dfrac{\partial u_{s}}{\partial p_{k}}-\dfrac{\partial u_i}{\partial p_k}\dfrac{\partial u_s}{\partial q_k}\right)\nonumber\\
&=&\sum_{r=1}^{n}\sum_{k=1}^{n}\dfrac{\partial p_{r}}{\partial u_{j}}\dfrac{\partial u_{s}}{\partial p_{k}}\sum_{i=1}^{2n}\dfrac{\partial p_{r}}{\partial u_{i}}\dfrac{\partial u_{i}}{\partial q_{k}}-\sum_{r=1}^{n}\sum_{k=1}^{n}\dfrac{\partial q_{r}}{\partial u_{j}}\dfrac{\partial u_{s}}{\partial p_{k}}\sum_{i=1}^{2n}\dfrac{\partial p_{r}}{\partial u_{i}}\dfrac{\partial u_i}{\partial p_k}\nonumber\\ &&~~~~~~~~~~~~~~~~~~-\sum_{r}\sum_{k}\dfrac{\partial p_{r}}{\partial u_{j}}\dfrac{\partial u_{s}}{\partial q_{k}}\sum_{i=1}^{2n}\dfrac{\partial q_{r}}{\partial u_{i}}\dfrac{\partial u_i}{\partial p_k}+\sum_{r}\sum_{k}\dfrac{\partial q_{r}}{\partial u_{j}}\dfrac{\partial u_{i}}{\partial p_{k}}
\sum_{i=1}^{2n}\dfrac{\partial p_{r}}{\partial u_{i}}\dfrac{\partial u_{i}}{\partial p_{k}}\nonumber\\
&=&\sum_{r=1}^{n}\dfrac{\partial p_{r}}{\partial u_{j}}\dfrac{\partial u_{s}}{\partial p_{r}}-\sum_{r=1}^{n}\dfrac{\partial q_{r}}{\partial u_{j}}\dfrac{\partial u_s}{\partial p_r}.0-0+\sum_{r}\dfrac{\partial q_{r}}{\partial u_{j}}\dfrac{\partial u_s}{\partial q_r}\nonumber\\
&=&\sum_{r=1}^{n}\left(\dfrac{\partial p_{r}}{\partial u_{j}}\dfrac{\partial u_{s}}{\partial p_{r}}+\dfrac{\partial q_{r}}{\partial u_{j}}\dfrac{\partial u_s}{\partial q_r}\right)=\delta_{js}.\nonumber
\end{eqnarray}
\section{Legendre's (dual) transformation:}
Let $Q(q)$ be a given function of the variables $q_{1},q_{2},...,q_{n}$ and let the set of new variables $p_{1},p_{2},...,p_{n}$ be defined by the transformation $p_{i}=\dfrac{\partial Q}{\partial q_{i}}$. Then a new function $P(p)$ can be constructed so as to define the transformation $q_{i}=\dfrac{\partial P}{\partial p_{i}}$. Further, if $Q$ contains the independent parameter $t$ then $P$ does contain $t$ and $\dfrac{\partial Q}{\partial t}=-\dfrac{\partial P}{\partial t}$. (The variable $q$, which actively participate in the transformation is called active variable while $t$ is called a passive variable.)\\
\textbf{Proof:} Let $Q(q,t)$ be a given function of the independent variables $q_{1},q_{2},...,q_{n}$ and independent parameter $t$. We define a set of new independent variables $p_{1},p_{2},...,p_{n}$ by 
\begin{equation}
p_{i}=\dfrac{\partial Q}{\partial q_{i}}\label{eq2..3}
\end{equation}
Since $p_{i}$'s are all independent, the equation (\ref{eq2..3}) can be inverted to give\\
\begin{equation}
q_{i}=q_{i}(p_{i},t)\label{eq2..4}
\end{equation}
provided the Jacobian of the transformation is not equal to zero. We now construct a function $P$ by\\
\begin{equation}
P=\sum p.q- Q(q,t)\label{eq2..5}.
\end{equation} We eliminate from the R.H.S of (\ref{eq2..5}) the variables $q$'s by the corresponding $p$'s with the help of (\ref{eq2..4}) to express $P$ as a function of $p$'s and $t$. Now consider a variation of (\ref{eq2..5}) due to an infinitesimal variation of the variables involved in that equation i.e.\\
\begin{eqnarray}
&&\delta P=\delta \left(\sum p.q-Q\right)\nonumber\\
&&\mbox{ or,}~~
\sum \dfrac{\partial P}{\partial p}\delta p+\dfrac{\partial P}{\partial t}\delta t=\cancel{\sum p.\delta q}+\sum q.\delta p-\cancel{\sum\frac{\partial Q}{\partial q}}\delta q-\dfrac{\partial Q}{\partial t}\delta t. ~~~(\because p=\dfrac{\partial Q}{\partial q}).\nonumber\\
&& \mbox{Consequently},~~
q=\dfrac{\partial P}{\partial p},~~\dfrac{\partial Q}{\partial t}=-\dfrac{\partial P}{\partial t}.\nonumber
\end{eqnarray}
Legendre's transformation changes a given function $Q$ of a given set of variables $q$ into a new function $P$ of a new set of variables $p$.
\section{Transformation of canonical variables}
Let $q_{1},q_{2},...,q_{n}$ be $n$ independent variables and $t$ be an independent character. We define new variables $Q_{i}$ by means of the transformation\\
\begin{equation}
Q_{i}=Q_{i}(q,t)\label{eq2..6}
\end{equation}
and assume that the new variable $Q_{i}$'s are all independent so that the inverse transformation $q_{i}=q_{i}(Q,t)$ exists. We can associate a definite geometrical picture with the transformation (\ref{eq2..6}). We can think of an $n$- dimensional space in which the variables $q$'s are the axial variables of a set of $n$ rectangular axes. We denote this space by $S_{q}$. Similarly we can think of an $n$-dimensional space $S_{Q}$. To a point of $S_{q}$ corresponds a point $S_{Q}$ and for this reason the transformation (\ref{eq2..6}) is called a point transformation.\\
$~~~~~~~~~~~~~~~~~~~~~~~~~~~~~~~~~~~~~~~~~~~~~~~~$Let us now try to extend the transformation (\ref{eq2..6}) by allowing the variables $(q,p)$ take part in equations of transformation (\ref{eq2..6}). Let us consider the transformation of the type: 
\begin{equation}
Q=Q(q,p,t)~~~~~\mbox{and}~~~~ P=P(q,p,t)\label{eq2..7}
\end{equation}
and assume that the new variables $Q$'s and $P$'s are all independent so that the inverse transformation $q=q(Q,P,t),~~p=p(Q,P,t)$ exists. The canonical forms of the equation of motion will not always be preserved under a general transformation of the type (2). We restrict ourselves to those transformation of the type (\ref{eq2..7}) which produce new canonical variables $Q$ and $P$ so that the transformation changes the canonical equations to
\begin{eqnarray}
\dot{q}=\dfrac{\partial H}{\partial p},~~\dot{p}=-\dfrac{\partial H}{\partial q}\label{eq2..8} \\to\nonumber\\
\dot{Q}=\dfrac{\partial K}{\partial P},~~\dot{P}=-\dfrac{\partial K}{\partial Q}\label{eq2..9}
\end{eqnarray}
where $K(Q,P,t)$ is new Hamiltonian of the system. Transformation (\ref{eq2..7}) is then called a canonical or contact transformation i.e. the equations of canonical transformation. The variables $q,p$ satisfy Hamilton's equation of motion (\ref{eq2..8}) and hence satisfy Hamilton's principle i.e.
\begin{eqnarray}
&&\delta \int_{t_{0}}^{t_{1}}\left(\sum p~\dot{q}-H\right)dt=0\nonumber\\ \mbox{or,}~~&&\delta\int_{t_{0}}^{t_{1}}\left(\sum pdq-Hdt\right)=0\label{eq2..10}
\end{eqnarray}
If $Q$, $P$ are also canonical variables and satisfy equation (\ref{eq2..9}) then the Hamilton's principle takes the form
\begin{equation}
\delta \int_{t_{0}}^{t_{1}}\left(\sum PdQ-Kdt\right)=0\label{eq2..11}
\end{equation}
 where $K$ is the Hamiltonian associated with $Q,p$ at moment $t$. The equations (\ref{eq2..10}) and (\ref{eq2..11}) must simultaneously be valid for one implies the other. Hence
 \begin{equation}
\delta \int_{t_{0}}^{t_{1}}\left[\left(\sum pdq-Hdt\right)-C\left(\sum PdQ-Kdt\right)\right]=0\label{eq2..12}
\end{equation}
 where $C$ is a constant. This means that
 \begin{equation}
\sum pdq-Hdt=C(\sum PdQ-Kdt)+dF\label{eq2..13}
\end{equation}
 where $F$ is some arbitrary function of the variables at time $t$.
 
 \vspace{.25cm}
 
(\textbf{Note:} The simultaneous validity of (\ref{eq2..10}) and (\ref{eq2..11}) does not mean that the integrands in the two integrals are equal but they can differ at most by a total derivative of arbitrary function $F$. The integrand between these two end points of such a difference term is then $\int_{t_{0}}^{t_{1}}\dfrac{\partial F}{\partial t}dt=F(t_{1})-F(t_{0})$ and the variation of these integral is automatically zero for any function $F$ since the variations vanish at the end points.)

\vspace{.25cm}

The function $F$ is called the generating function and $C$ is the valency of the canonical transformation. If $C=1$, the canonical transformation is called univalent. Once $F$ is given, the transformation (\ref{eq2..7}) is completely specified.\\
$~~~~~~~~~~~~~~~~~~~~~~~~~~~~~~~$ In order to effect the transformation between the two sets of canonical variables $F$ must be a function of both the old and the new variables besides the time $t$. The generating function   may thus be a function of $4n$ variables in all. But only $2n$ of these are independent because the two sets of co-ordinates are connected by $2n$ transformation equations (\ref{eq2..7}). The generating function can therefore be written as a function of independent variables in one of the four forms:\\
$F_{1}(q,Q,t),~~F_{2}(q,P,t),~~F_{3}(p,Q,t),~~F_{4}(p,P,t)$\\
The selection of the particular form depends on the specific problem under consideration. If in (\ref{eq2..13}) we put $C=1$ then we have $p=\dfrac{\partial F}{\partial q},~~P=-\dfrac{\partial F}{\partial Q},~~K-H=\dfrac{\partial F}{\partial t}$. So if $F$ is independent of $t$ then $K=H$ i.e. the form of $H$ is conserved. Again given $F$ we can find $p$ and $P$. Thus in the case of contact transformation when $(q,p)\rightarrow(Q,P)$ and $K=H$, we have
$$ \sum pdq-\sum PdQ=dF$$. i.e to say in a contact transformation $\sum pdq-\sum PdQ$ is a perfect differential.

\section{Formula for canonical transformation}

\vspace{.5cm}

\textbf{I.} Let $F=F_{1}(q,Q,t)$ then from (\ref{eq2..13}) putting $C=1$, we have 
\begin{equation}
\sum pdq-\sum PdQ+(K-H)dt=dF_{1}(q,Q,t)=\sum \dfrac{\partial F_{1}}{\partial q}dq+\sum \dfrac{\partial F_{1}}{\partial Q}dQ+\dfrac{\partial F_{1}}{\partial t}dt\label{eq2..14}
\end{equation}
Now equating coefficients of like differentials we have
\begin{eqnarray}
p_{i}&=&\dfrac{\partial F_{1}(q,Q,t)}{\partial q_{i}}\label{eq2..15}\\
P_{i}&=&-\dfrac{\partial F_{1}(q,Q,t)}{\partial Q_{i}}\label{eq2..16}\\
\mbox{and}~~K&=&H+\dfrac{\partial F_{1}}{\partial t}\label{eq2..17}
\end{eqnarray} The $n$ equations (\ref{eq2..15}) can be solved for $n$ $Q$'s. As 
\begin{equation}
	Q_{i}=Q_{i}(q,p,t)\label{eq2..18}
\end{equation} so inserting these values of $Q$ in (\ref{eq2..16}) we have
\begin{equation} 
	P_{i}=P_{i}(q,p,t)\label{eq2..19}
	\end{equation} (\ref{eq2..18}) and (\ref{eq2..19}) are the equations of canonical transformation and $(\ref{eq2..17})$ connects $K$ and $H$.\\
\textbf{II.} Let the generating function be $F=F_{2}(q,p,t)$. From (\ref{eq2..14})\\
\begin{eqnarray}
\sum pdq-\sum PdQ+(K-H)dt=dF_{1}(q,Q,t)\nonumber\\
or,~~\sum p dq+\sum QdP+(K-H)dt=d(F_{1}+\sum PQ)\label{eq2..20}
\end{eqnarray} Now, we express the quantity $F_{1}(q,Q,t)+\sum PQ$ in terms of the variables $q,P,t$ by the relation (\ref{eq2..16}) and write $F_{2}$ for the transformed quantity i.e.
\begin{equation}
F_{2}(q,P,t)=F_{1}(q,Q,t)+\sum PQ\label{eq2..21}
\end{equation} where $P_{i}=-\dfrac{\partial F_{1}}{\partial Q_{i}}$. In terms of $F_{2}$ equation (\ref{eq2..20}) becomes 
\begin{equation}
\sum pdq+\sum QdP+(K-H)dt=dF_{2}(q,P,t)=\sum \dfrac{\partial F_{2}}{\partial q}dq+\sum \dfrac{\partial F_{2}}{\partial P}dP+\dfrac{\partial F_{2}}{\partial t}dt,\nonumber
\end{equation} so the transformation equations become
\begin{eqnarray}
p_{i}&=&\dfrac{\partial F_{2}}{\partial q_{i}}\label{eq2..22}\\
Q_{i}&=&\dfrac{\partial F_{2}}{\partial P_{i}}\label{eq2..23}\\
and,&&\nonumber\\
K&=&H+\dfrac{\partial F_{2}}{\partial t}\label{eq2..24}
\end{eqnarray} (\ref{eq2..22}) can be solved for P to give
\begin{equation}
 P_{i}=P_{i}(q,p,t)\label{eq2..25}
 \end{equation} and then from (\ref{eq2..23}) 
\begin{equation}
Q_{i}=Q_{i}(q,p,t)\label{eq2..26}
\end{equation}
Equations (\ref{eq2..25}) and (\ref{eq2..26}) are the equations of canonical transformation for the generating function of the type $F_{2}$.\\
\textbf{III.} Let the generating function $F=F_{3}(p,Q,t)$\\
We write eq (\ref{eq2..14}) in this case as 
\begin{equation}
-\sum qdp-\sum PdQ+\sum (K-H)dt=d(F_{1}-\sum pdq)\label{eq2..27}
\end{equation} Now express $F_{1}-\sum pq$ in terms of the variables $p,Q,t
$ by (\ref{eq2..15}) and write $F_{3}(p,Q,t)$ as
\begin{equation}
	 F_{3}(p,Q,t)=F_{1}(q,Q,t)-\sum pq\label{eq2..28}
	 \end{equation} where $p_{i}=\dfrac{\partial F_{1}(q,Q,t)}{\partial q_{i}}$. In terms of $F_{3}$ equation (\ref{eq2..27}) now becomes,
\begin{eqnarray}
\sum qdp-\sum PdQ+(K-H)dt=dF_{3}=\sum \dfrac{\partial F_{3}}{\partial p}dp+\sum \dfrac{\partial F_{3}}{\partial Q}dQ+\dfrac{\partial F_{3}}{\partial t}dt\nonumber
\end{eqnarray}
$\therefore$
\begin{eqnarray}
q_{i}&=&-\dfrac{\partial F_{3}}{\partial p_{i}}\label{eq2..29}\\
P_{i}&=&-\dfrac{\partial F_{3}}{\partial Q_{i}}\label{eq2..30}\\
K&=&H+\dfrac{\partial F_{3}}{\partial t}\label{eq2..31}
\end{eqnarray} Equations (\ref{eq2..29}) and (\ref{eq2..30}) constitute the transformation equations for the generating function $F_{3}(p,Q,t)$ and (\ref{eq2..31}) is the relation between $K$ and $H$.\\
\textbf{IV.} $F=F_{4}(p,P,t)$, the we have from (\ref{eq2..14})
\begin{eqnarray}
\sum pdq-\sum PdQ+(K-H)dt=dF_{1}\nonumber\\
or~~~~-\sum qdp+\sum QdP+(K-H)dt=d\left[F_{1}(q,Q,t)+\sum PQ-\sum pq\right]\nonumber
\end{eqnarray} By equation, (\ref{eq2..15}) we can express $Q$ in terms of $p$ and $q$ and then by (\ref{eq2..16}) we can express $Q$ in terms of $P$ and $p$. So by (\ref{eq2..15}) and (\ref{eq2..16}) we can write
\begin{equation}
F_{1}(q,Q,t)+\sum PQ-\sum pq= F_{4}(p,P,t)\nonumber
\end{equation}  Thus we write
\begin{equation}
-\sum qdp+\sum QdP+(K-H)dt=dF_{4}(p,P,t)=\sum \dfrac{\partial F_{4}}{\partial p}dp+\sum \dfrac{\partial F_{4}}{\partial P}dP+\dfrac{\partial F_{4}}{\partial t}dt\nonumber
\end{equation} Hence the transformation equations are
\begin{eqnarray}
q_{i}&=&-\dfrac{\partial F_{4}}{\partial p_{i}}\label{eq2..32}\\
Q_{i}&=&\dfrac{\partial F_{4}}{\partial P_{i}}\label{eq2..33}\\
K&=&H+\dfrac{\partial F_{4}}{\partial t}\label{eq2..34}
\end{eqnarray}
Equations (\ref{eq2..32}) and (\ref{eq2..33}) constitute the canonical transformation and (\ref{eq2..34}) is the relation between $K$ and $H$.

\vspace{.5cm}


$\bullet$ \textbf{Problem I:} Prove that the transformation $Q=p$, $P=-q$ is a contact transformation.

\vspace{.25cm}

\textbf{Solution:} Here $pdq-PdQ=pdq+qdp=d(p,q)$. So it is a contact transformation.

\vspace{.5cm}

$\bullet$ \textbf{Problem II:} $P=\dfrac{1}{2}(p^{2}+q^{2}),~~Q=tan^{-1}\left(\frac{q}{p}\right)$.  So $pdq-PdQ=pdq-\dfrac{1}{2}(p^{2}+q^{2}).\dfrac{1}{1+\frac{q^{2}}{p^{2}}}d(\frac{q}{p})=pdq-\frac{1}{2}\dfrac{p^{2}(pdq-qdp)}{p^{2}}=\frac{1}{2}(pdq+qdp)=d(\frac{1}{2}pq)$.

\vspace{.5cm}

$\bullet$ \textbf{Problem III:} $P=logsinp,~~Q=q~tanp$.

\vspace{.5cm}

$\bullet$ \textbf{Problem IV:} $Q=log(\frac{1}{2}cosp),~~P=q~cot p$.

\vspace{.5cm}

$\bullet$ \textbf{Problem V:} $Q=log(1+\sqrt{q} cosp), ~~P=2(1+\sqrt{q}cos p)\sqrt{q}sin p$.

\vspace{.5cm}

$\bullet$ \textbf{Problem VI:} Show that the transformation $Q=\sqrt{q}cos p,~P=\sqrt{q} sin p$ represents a canonical transformation with valency not equal to unity. Solve for the new Hamiltonian of the system for which $T=\frac{1}{2}m\dot{q}^{2},~~V=\frac{1}{2}\mu {q}^{2}$.

\vspace{.25cm}

\textbf{Solution:} $PdQ=\sqrt{q} sin p\left[\frac{1}{2}q^{-\frac{1}{2}} ~cosp~ dq-sin p~ \sqrt{q}~ dp\right]=\dfrac{1}{2}sin p ~cos p ~dq-q~ sin^{2}p~dp$\\
$\therefore Cpdq-PdQ=(Cp-\frac{1}{2} ~sinp~cosp)dq+q~sin^{2}p~dp=udq+vdp$ (say)\\
$\dfrac{\partial u}{\partial p}=C-\frac{1}{2}(cos^{2}p-sin^{2}p)=C-\frac{1}{2}(1-2sin^{2}p)$\\
$\dfrac{\partial u}{\partial q}=sin^{2}p$. Hence for exactness: $\dfrac{\partial u}{\partial p}=\dfrac{\partial v}{\partial q}\implies C-\frac{1}{2}+sin^{2}p=sin^{2}p\implies C=\frac{1}{2}$\\
$H=T+V=\frac{1}{2}m\dot{q}^{2}+\frac{1}{2}\mu {q}^{2}$.\\
$p=\dfrac{\partial L}{\partial \dot{q}}=\dfrac{\partial T}{\partial \dot{q}}=m\dot{q}$. $\therefore H=\dfrac{1}{2}m\dfrac{p^{2}}{m^{2}}+\dfrac{1}{2}\mu q^{2}=\dfrac{p^{2}}{2m}+\dfrac{1}{2}\mu q^{2}$.
Now, $Q=\sqrt{q}cos p,~~ P=\sqrt{q}sin p. \therefore Q^{2}+P^{2}=q,~~\dfrac{P}{Q}=tan p$. $\therefore K=CH=\dfrac{1}{2}\left[\dfrac{1}{2m}\left(tan^{-1}(\frac{P}{Q})\right)^{2}+\dfrac{1}{2}\mu(Q^{2}+P^{2})^{2}\right]$ is the new Hamiltonian.

\vspace{.5cm}

$\bullet$ \textbf{Problem VII:} The K.E and P.E  of one dimensional harmonic oscillator is $T=\dfrac{1}{2}\dot{q}^{2},~~V=\dfrac{1}{2}\mu^{2}q^{2}.$. Solve the problem using canonical transformation $F_{1}=\dfrac{1}{2}\mu q^{2} cot Q$.

\vspace{.25cm}

\textbf{Solution:} $p=\dfrac{\partial T}{\partial \dot{q}}=\dot{q}$. $\therefore$ Hamiltonian $H=T+V=\dfrac{1}{2}p^{2}+\dfrac{1}{2}\mu q^{2}=\dfrac{1}{2}(\mu q^{2}+p^{2})$\\
The canonical equations for $F_{1}$ are\\
$p_{i}=\dfrac{\partial F_{1}}{\partial q_{i}},~~P_{i}=-\dfrac{\partial F_{1}}{\partial Q_{i}}$.\\
$\therefore p=\mu q cotQ,~~P=\dfrac{1}{2}\mu q^{2}cosec^{2}Q$\\
$\therefore q^{2}=\dfrac{2P}{\mu cosec^{2}Q},~~p^{2}=\mu^{2}\dfrac{2cot^{2}Q}{\mu cosec^{2}Q}=2\mu P cos^{2}Q$\\
$\therefore K=H=\dfrac{1}{2}\left[\mu.2.P.sin^{2}Q+2.\mu.P.cos^{2}Q\right]=\mu P.$\\
$\therefore-\dot{P}=\dfrac{\partial K}{\partial Q}=0,~~\dot{Q}=\dfrac{\partial K}{\partial P}=\mu$\\
i.e. $P=const=\dfrac{E}{\mu}~(say),~~Q=\mu t+\epsilon$.\\
$\therefore q=\dfrac{\sqrt{2E}}{\mu}sin(\mu t+\epsilon),~~p=\sqrt{2E}cos(\mu t+\epsilon)$.
\section{Infinitesimal contact transformation}
Consider the canonical transformation for which the generating function $F_{2}(q,P)=\sum q_{i}P_{i}$, then the transformation equations are
\begin{equation}
p_{i}=\dfrac{\partial F_{2}}{\partial q_{i}}=P_{i},~~Q_{i}=\dfrac{\partial F_{2}}{\partial P_{i}}=q_{i},~~K=H.\nonumber
\end{equation} As the new and old co-ordinates being the same so $F_{2}$ generates the identity transformation.\\
If the new co-ordinates differ from the old ones only by infinitesimals it is called infinitesimal contact transformation. Here only 1st order terms are retained in these infinitesimal. This is represented by $Q_{i}=q_{i}+\delta q_{i},~~P_{i}=p_{i}+\delta p_{i}$, where $\delta q_{i}$, $\delta p_{i}$ are the infinitesimal changes in the co-ordinates. It is clear that in this case the generating function will differ only by an infinitesimal amount from the above identity transformation. Thus we can write,
\begin{equation}
F_{2}=\sum q_{i} P_{i}+\epsilon G(q,P)\nonumber
\end{equation} where $\epsilon$ is some infinitesimal parameter of the transformation. The transformation equations are
\begin{eqnarray}
p_{i}=\dfrac{\partial F}{\partial q_{i}}=P_{i}+\epsilon\dfrac{\partial G}{\partial q_{i}}\nonumber\\
\therefore\delta p_{i}=P_{i}-p_{i}=-\epsilon\dfrac{\partial G}{\partial q_{i}}\label{eq2..35}
\end{eqnarray} Again, $Q_{i}=\dfrac{\partial F}{\partial P_{i}}=q_{i}+\epsilon\dfrac{\partial G}{\partial P_{i}}$. The second term on the R.H.S contains the infinitesimal parameter and since from (\ref{eq2..35}) $P$ differs from $p$ only an infinitesimal so we can replace $P_{i}$ by $p_{i}$ in the derivative $\dfrac{\partial G}{\partial P_{i}}$ to have the result correct up to 1st order. Thus
\begin{equation}
\delta q_{i}=Q_{i}-q_{i}=\epsilon\left(\dfrac{\partial G}{\partial p_{i}}\right)\label{eq2..36}
\end{equation} Now we consider an infinitesimal contact transformation in which $\epsilon$ is the small time interval $dt$ and $G$ is $H(q,p)$, then we have
\begin{eqnarray}
\delta q_{i}=dt.\dfrac{\partial H}{\partial p_{i}}=\dot{ q_{i}}dt=dq_{i}\nonumber\\
\delta p_{i}=-dt\dfrac{\partial H}{\partial q_{i}}=\dot {p_{i}} dt=dp_{i}\nonumber
\end{eqnarray} Thus the transformation changes the co-ordinates and momenta at the time $t$ to the values they have at the time $t+\delta t$. Thus the motion of the system in a time interval $\delta t$ can be described by an infinitesimal contact transformation generated by the Hamiltonian.\\
~~~~~~~~~~~~~~~Extending this result we can say that the system motion in a finite interval from $t_{0}$ to $t$ is represented by a succession of infinitesimal contact transformation. Thus the values of $p$ and $q$ at any time $t$ can be attained from their initial values by a canonical transformation, which is a constant function of time. Thus the Hamiltonian is the generator of motion of the system in time translation.\\
~~~~~~~~~~~~~Now consider the change in some function $u=u(q,p)$ as a result of this transformation. If the infinitesimal canonical transformation is generated by the Hamiltonian, the result for substituting the new variable for the old is to change $u$ from its value at $t$ to the value it has at a time $dt$ later. The change in a function $'u'$ as a result of an infinitesimal canonical transformation is
\begin{eqnarray}
\delta u&=& u(q_{i}+\delta q_{i}, p_{i}+\delta p_{i})-u(q_{i},p_{i})\nonumber\\
&=&\sum_{i}(\delta q_{i}\dfrac{\partial u}{\partial q_{i}}+\delta p_{i}\dfrac{\partial u}{\partial p_{i}})~~~~~\mbox{(upto~~ 1st ~~~order)}\nonumber\\
&=&\epsilon\sum_{i}\left(\dfrac{\partial u}{\partial q_{i}}\dfrac{\partial G}{\partial p_{i}}-\dfrac{\partial u}{\partial p_{i}}\dfrac{\partial G}{\partial q_{i}}\right)~~~~~\mbox{(by (\ref{eq2..35}) ~and~ (\ref{eq2..36}))}\nonumber\\
&=&\epsilon[u,G]\nonumber
\end{eqnarray} If $u$ is replaced by the Hamiltonian then $\delta H=\epsilon [H,G]$. Now, if $G$ be a constant of motion its Poisson bracket with the Hamiltonian is zero. i.e $[H,G]=0$, which implies $\delta H=0$. Thus the constants of motion are the generating functions of the infinitesimal canonical transformation which leave the Hamiltonian invariant.
\section{Invariance of Poisson bracket under Canonical transformation}
The Poisson's bracket of two arbitrary functions $F$ and $G$ with respect to the variables $q$, $p$ is defined as
\begin{equation}
\{F,G\}_{q,p}=\sum_{i}\left(\dfrac{\partial F}{\partial q_{i}}\dfrac{\partial G}{\partial p_{i}}-\dfrac{\partial F}{\partial p_{i}}\dfrac{\partial G}{\partial q_{i}}\right)\label{eq2..37}
\end{equation} Now considering $q_{i}$ and $p_{i}$ as functions of the transformed set of variables $Q_{k}$ and $P_{k}$, the equation (\ref{eq2..37}) can be written as
\begin{eqnarray}
\{F,G\}_{q,p}&=&\sum_{i,k}\left[\dfrac{\partial F}{\partial q_{i}}\left(\dfrac{\partial G}{\partial Q_{k}}\dfrac{\partial Q_{k}}{\partial p_{i}}+\dfrac{\partial G}{\partial P_{k}}\dfrac{\partial P_{k}}{\partial p_{i}}\right)-\dfrac{\partial F}{\partial p_{i}}\left(\dfrac{\partial G}{\partial Q_{k}}\dfrac{\partial Q_{k}}{\partial q_{i}}+\dfrac{\partial G}{\partial P_{k}}\dfrac{\partial P_{k}}{\partial q_{i}}\right)\right]\nonumber\\
&=&\sum_{k}\left[\dfrac{\partial G}{\partial Q_{k}}\{\sum_{i}\left(\dfrac{\partial F}{\partial q_{i}}\dfrac{\partial Q_{k}}{\partial p_{i}}-\dfrac{\partial F}{\partial p_{i}}\dfrac{\partial Q_{k}}{\partial q_{i}}\right)\}+\dfrac{\partial G}{\partial P_{k}}\{\sum_{i}\left(\dfrac{\partial F}{\partial q_{i}}\dfrac{\partial P_{k}}{\partial p_{i}}-\dfrac{\partial F}{\partial p_{i}}\dfrac{\partial P_{k}}{\partial q_{i}}\right)\}\right]\nonumber\\
&=&\sum_{k}\left[\dfrac{\partial G}{\partial Q_{k}}\{F,Q_{k}\}_{q,p}+\dfrac{\partial G}{\partial P_{k}}\{F,P_{k}\}_{q,p}\right]\label{eq2..38}
\end{eqnarray}
We shall use this formula (\ref{eq2..38}) to evaluate $\{F,Q_{k}\}_{q,p}$ and $\{F,P_{k}\}_{q,p}$. Now,
\begin{eqnarray}
\{Q_{k},F\}_{q,p}=\sum_{j}\left\{\dfrac{\partial F}{\partial Q_{j}}\{Q_{k},Q_{j}\}_{q,p}+\dfrac{\partial F}{\partial P_{j}}\{Q_{k},P_{j}\}_{q,p}\right\}\nonumber\\=
\sum_{j}\left(\dfrac{\partial F}{\partial Q_{j}}.0+\dfrac{\partial F}{\partial P_{j}}.\delta_{kj}\right)=\dfrac{\partial F}{\partial P_{k}}\nonumber
\end{eqnarray} $\therefore \{F,Q_{k}\}_{q,p}=-\dfrac{\partial F}{\partial P_{k}}$. \\
Similarly, we find $\{P_{k},F\}_{q,p}=-\dfrac{\partial F}{\partial Q_{k}}$ i.e. $\{F,P_{k}\}_{q,p}=\dfrac{\partial F}{\partial Q_{k}}$. \\
Hence from (\ref{eq2..38})
\begin{equation}
\{F,G\}_{q,p}=\sum_{i}\left(-\dfrac{\partial G}{\partial Q_{i}}\dfrac{\partial F}{\partial P_{i}}+\dfrac{\partial G}{\partial P_{i}}\dfrac{\partial F}{\partial Q_{i}}\right)=\{F,G\}_{(Q,P)}\nonumber
\end{equation} Therefore Poisson bracket remains invariant under canonical transformation.

\vspace{.5cm}

$\bullet$ Show that the Jacobian of any canonical transformation is unity i.e. $J=\dfrac{\partial (Q_{1},Q_{2},...,Q_{n},P_{1},P_{2},...,P_{n})}{\partial (q_{1},q_{2},...,q_{n},p_{1},p_{2},...,p_{n})}=1$

\vspace{.25cm}

\textbf{Proof:} We shall use $F_{2}(q,P,t)$ to give the canonical transformation from $(q,p)\rightarrow(Q,P)$, where $Q_{i}=Q_{i}(q,p,t)$ and $P_{i}=P_{i}(q,p,t)$. The transformation equations are $p_{i}=\dfrac{\partial F_{2}}{\partial q_{i}},~~Q_{i}=\dfrac{\partial F_{2}}{\partial P_{i}}$.\\
Now we can write,
\begin{eqnarray}
J&=&\dfrac{\partial (Q_{1},Q_{2},...,Q_{n},P_{1},P_{2},...,P_{n})}{\partial (q_{1},q_{2},...,q_{n},p_{1},p_{2},...,p_{n})}\nonumber\\
&=&\dfrac{\partial (Q_{1},Q_{2},...,Q_{n},P_{1},P_{2},...,P_{n})}{\partial (q_{1},q_{2},...,q_{n},P_{1},P_{2},...P_{n})}\times \dfrac{\partial (q_{1},q_{2},...,q_{n},P_{1},P_{2},...,P_{n})}{\partial (q_{1},q_{2},...,q_{n},p_{1},p_{2},...,p_{n})}\nonumber\\&=&\dfrac{\partial (Q_{1},Q_{2},...,Q_{n})}{\partial (q_{1},q_{2},...,q_{n})}|_{P=const.}\times \dfrac{\partial (P_{1},P_{2},...,P_{n})}{\partial (p_{1},p_{2},...,p_{n})}|_{q=const.}\label{eq2..39}
\end{eqnarray}
By definition, $
\dfrac{\partial (Q_{1},Q_{2},...,Q_{n})}{\partial (q_{1},q_{2},...,q_{n})}=$ \[
\begin{vmatrix}
\frac{\partial Q_{1}}{\partial q_{1}} & \frac{\partial Q_{2}}{\partial q_{1}}  & \dots & \frac{\partial Q_{n}}{\partial q_{1}} \\ 
\frac{\partial Q_{1}}{\partial q_{2}} & \frac{\partial Q_{2}}{\partial q_{2}}  & \dots & \frac{\partial Q_{n}}{\partial q_{2}} \\
\hdotsfor{4} \\
\frac{\partial Q_{1}}{\partial q_{n}} & \frac{\partial Q_{2}}{\partial q_{n}}  & \dots & \frac{\partial Q_{n}}{\partial q_{n}} 
\end{vmatrix}\]
\[=
\begin{vmatrix}
\frac{\partial^{2}F_{2}}{\partial q_{1} \partial P_{1}} & \frac{\partial^{2} F_{2}}{\partial q_{1} \partial P_{2}}  & \dots & \frac{\partial^{2} F_{2}}{\partial q_{1} \partial P_{n}} \\ 
\frac{\partial^{2}F_{2}}{\partial q_{2} \partial P_{1}} & \frac{\partial^{2} F_{2}}{\partial q_{2}\partial P_{1}}  & \dots & \frac{\partial_{2} F_{2}}{\partial q_{2}\partial P_{n}} \\
\hdotsfor{4} \\
\frac{\partial^{2}F_{2}}{\partial q_{n} \partial P_{1}} & \frac{\partial^{2} F_{2}}{\partial q_{n} \partial P_{2}}  & \dots & \frac{\partial^{2}F_{2}}{\partial q_{n} \partial P_{n}} 
\end{vmatrix}\]
\[=
\begin{vmatrix}
\frac{\partial}{\partial P_{1}} (\frac{\partial F_{2}}{\partial q_{1}}) & \frac{\partial}{\partial P_{2}}(\frac{\partial F_{2}}{\partial q_{1}})  & \dots & \frac{\partial}{\partial P_{n}}(\frac{\partial F_{2}}{\partial q_{1}}) \\ 
\frac{\partial }{\partial P_{1}}(\frac{\partial F_{2}}{\partial q_{2}}) & \frac{\partial}{\partial P_{2}}(\frac{\partial F_{2}}{\partial q_{2}})  & \dots & \frac{\partial}{\partial P_{n}} (\frac{\partial F_{2}}{\partial q_{2}})\\
\hdotsfor{4} \\
\frac{\partial}{\partial P_{1}} (\frac{\partial F_{2}}{\partial q_{n}})& \frac{\partial}{\partial P_{2}}(\frac{\partial F_{2}}{\partial q_{n}})  & \dots & \frac{\partial}{\partial P_{n}}(\frac{\partial F_{2}}{\partial q_{n}}) 
\end{vmatrix}\]
\[=
\begin{vmatrix}
\frac{\partial p_{1}}{\partial P_{1}} & \frac{\partial p_{1}}{\partial P_{2}}  & \dots & \frac{\partial p_{1}}{\partial P_{n}} \\ 
\frac{\partial p_{2}}{\partial P_{1}} & \frac{\partial p_{2}}{\partial P_{2}}  & \dots & \frac{\partial p_{2}}{\partial P_{n}} \\
\hdotsfor{4} \\
\frac{\partial p_{n}}{\partial P_{1}} & \frac{\partial p_{n}}{\partial P_{2}}  & \dots & \frac{\partial p_{n}}{\partial P_{n}} 
\end{vmatrix}\]\\
$$=\frac{\partial(p_1,p_2,...,p_n)}{\partial(P_1,P_2,...,P_n)}$$
Hence from (\ref{eq2..39}) $J=1$. (proved)
\section{Phase space:}
For each degree of freedom of a dynamical system there are two quantities namely $(q_{k},p_{k})$ that assume independent role. Imagine a space of $2m$ dimension in which $p_{1},p_{2},...,p_{m}$ and $q_{1},q_{2},...,q_{m}$ are the co-ordinates associated with a point called representative point. As $q_{k},p_{k}$ change in physical space with time the representative point moves in this $2m$ dimensional phase space.
\section{Liouviell's Theorem}
\textbf{Statement:} The density of an element in the phase space corresponding to a system of particles remains constant during the motion i.e. the total increment in $\rho$ as the state of the system varies is zero.

\vspace{.25cm}

\textbf{Proof:} Suppose that a large no. of identical one dimensional system are present each having a representative point in the $pq$-plane. At a given point in the $pq$-plane and within an area $dp.dq$ there will be many representative points at any instant. Let the density of such points be $\rho(q,p,t)$. By the density we mean the number of points divided by the elementary area as the latter approaches to zero. The no. of representative points moving into $dpdq$ on the left edge is $\rho ~\dot{q}~dp$ per unit time. Thus the  no. moving out of $dpdq$ through its right edge is $\{\rho~\dot{q}+\dfrac{\partial}{\partial q}(\rho~\dot{q})dq\}dp$. Hence the net increase in $\rho$ in the element is 
\begin{equation}
-\left(\{\rho~\dot{q}+\dfrac{\partial}{\partial q}(\rho~\dot{q})dq\}dp\right)+\rho\dot{q}=-\dfrac{\partial}{\partial q}(\rho \dot{q})~dq.dp\label{eq2..40}
\end{equation} In a similar way the net gain due to flow in the vertical direction is
\begin{equation}
-\dfrac{\partial}{\partial p}(\rho \dot{p})~dq.dp\label{eq2..41}
\end{equation} But the sum of these (\ref{eq2..40}) and (\ref{eq2..41}) equals $\left(\dfrac{\partial \rho}{\partial t}\right)dpdq$.
\begin{eqnarray}
\therefore \dfrac{\partial \rho}{\partial t}+\dfrac{\partial}{\partial q}(\rho \dot{q})+\dfrac{\partial}{\partial \rho}(\rho \dot{p})=0\nonumber\\
or, ~~\dfrac{\partial \rho}{\partial t}+\{\dot{q}\dfrac{\partial \rho}{\partial q}+\dot{p}\dfrac{\partial \rho}{\partial p}\}+\rho\{\dfrac{\partial \dot{q}}{\partial q}+\dfrac{\partial \dot{p}}{\partial p}\}=0\nonumber
\end{eqnarray} If $H$ be the Hamiltonian of the system then $\dot{p}=-\frac{\partial H}{\partial q},~\dot{q}=\frac{\partial H}{\partial p}$. So the 3rd term in the R.H.S within curly bracket vanishes identically. Hence we have,
\begin{eqnarray}
\dfrac{\partial \rho}{\partial t}+\{\dot{q}\dfrac{\partial \rho}{\partial q}+\dot{p}\dfrac{\partial \rho}{\partial p}\}=0\nonumber\\
or,~\dfrac{\partial \rho}{\partial t}dt+\dfrac{\partial \rho}{\partial p}\dot{p}dt+\dfrac{\partial \rho}{\partial q}\dot{q}dt=0\nonumber\\
i.e. ~~\dfrac{\partial \rho}{\partial t}=0\nonumber
\end{eqnarray} , Hence the theorem.\\
The result deduced above for a system with 1 degree of freedom may be generalized for a system with $m$ degrees of freedom. If the representative point in phase space be defined by the co-ordinates $q_{1},q_{2},...,q_{m},p_{1},p_{2},...,p_{m}$ the Liouviell's theorem becomes
\begin{equation}
\dfrac{\partial \rho}{\partial t}+\sum_{k=1}^{m}\left(\dot{q_{k}}\dfrac{\partial \rho}{\partial q_{k}}+\dot{p_{k}}\dfrac{\partial \rho}{\partial p_{k}}\right)=0\nonumber
\end{equation} If $H$ be the Hamiltonian of the system then we have
\begin{eqnarray}
\dfrac{\partial \rho}{\partial t}+\sum_{k=1}^{m}\left(\dfrac{\partial H}{\partial p_k}\dfrac{\partial \rho}{\partial q_{k}}-\dfrac{\partial H}{\partial q_{k}}\dfrac{\partial \rho}{\partial p_{k}}\right)=0\nonumber\\
or,~~	\dfrac{\partial \rho}{\partial t}+\{\rho,H\}=0~~ i.e. ~~\dfrac{d\rho}{dt}=0\nonumber
\end{eqnarray} Thus $\rho$ is a constant of motion. The density of the systems in the neighbourhood of some given system in phase space remains constant in time.

\begin{figure}
	\centering
	\includegraphics[width=0.4\textwidth]{pic-2_13.pdf}\\
	\label{liovell_2_13}
\end{figure}


\section{Integral Invariance of Poincare}
The following integrals
\begin{eqnarray}
J_{1}=\int_{S_{2}}\sum_{i}dq_{i}dp_{i}\nonumber\\
J_{2}=\int_{S_{4}}\sum_{i,k}dq_{i}dp_{i}dq_{k}dp_{k}\nonumber\\
---------------\nonumber\\
---------------\nonumber\\
J_{n}=\int_{S_{2n}}dq_{1}dq_{2}...dq_{n}dp_{1}dp_{2}...dp_{n}\nonumber
\end{eqnarray} are invariant under canonical transformation, where $S_{2}$ is a two dimensional surface in phase space, $S_{4}$ is a four dimensional surface in phase space,.... $S_{2n}$ is a $2n$ dimensional surface in phase space, enclosing completely an arbitrary region of space.\\
Proof: Due to complexity we shall show only the invariance of $J_{1}$.\\
Here $S_{2}$ is any two dimensional surface of phase space. Since the point in $2D$ can be represented uniquely by two independent variables $u$ and $v$ (say). We can take on the surface\\
$q_{i}=q_{i}(u,v),~p_{i}=p_{i}(u,v)$.\\
Now,
\begin{eqnarray}
\sum dq_{i}dp_{i}=\sum_{i}\dfrac{\partial (q_{i},p_{i})}{\partial (u,v)}dudv\nonumber\\
=\sum_{i}\left(\dfrac{\partial q_{i}}{\partial u}\dfrac{\partial p_{i}}{\partial v}-\dfrac{\partial q_{i}}{\partial v}\dfrac{\partial p_{i}}{\partial u}\right)dudv\nonumber\\
=[u,v]_{q,p}dudv\nonumber\\
\therefore J_{1}=\int \int_{S_{2}}\sum dq_{i} dp_{i}=\int \int_{S_{2}} [u,v]_{q,p}dudv.\nonumber
\end{eqnarray} To show the invariance of $J_{1}$ under canonical transformation we shall have to show that
\begin{equation}
\int \int_{S_{2}}\sum_{i}\dfrac{\partial (q_{i},p_{i})}{\partial (u,v)}dudv=\int \int_{S_{2}} \sum_{k} \dfrac{\partial (Q_{k},P_{k})}{\partial (u,v)}dudv=\int \int_{S_{2}}[u,v]_{q,p}dudv\nonumber
\end{equation} where $Q_{i}=Q_{i}(q,p,t),~~P_{i}=P_{i}(q,p,t)$ are also canonical variables. As the region of integration is arbitrary so to prove the invariance we shall have to show
\begin{equation}
\sum_{i}\dfrac{\partial (q_{i},p_{i})}{\partial (u,v)} =\sum_{k} \dfrac{\partial (Q_{k},P_{k})}{\partial (u,v)}\label{eq2..42}
\end{equation}
Now for canonical transformation from $(q,p)$ to $(Q,P)$ we take the help of $F_{2}=F_{2}(q,P,t)$. The transformation equations are
\begin{equation}
p_{i}=\dfrac{\partial F_{2}}{\partial q_{i}},~~Q_{i}=\dfrac{\partial F_{2}}{\partial P_{i}}\label{eq2..43}
\end{equation} Thus the L.H.S of (\ref{eq2..42}) is
\[\sum_{i}
\begin{vmatrix}
\dfrac{\partial q_{i}}{\partial u} & \dfrac{\partial p_{i}}{\partial u} \\ 
\dfrac{\partial q_{i}}{\partial v} & \dfrac{\partial p_{i}}{\partial v} 
\end{vmatrix}
\] Using (\ref{eq2..43}) we can write,
\begin{eqnarray}
\dfrac{\partial p_{i}}{\partial u}=\dfrac{\partial}{\partial u}\left(\dfrac{\partial F_{2}}{\partial q_{i}}\right)=\sum_{k} \dfrac{\partial^{2}F_{2}}{\partial P_{k}\partial q_{i}}.\dfrac{\partial P_{k}}{\partial u}+\sum_{k}\dfrac{\partial^{2}F_{2}}{\partial q_{k}\partial q_{i}}.\dfrac{\partial q_{k}}{\partial u}\nonumber\\
and,~~	\dfrac{\partial p_{i}}{\partial v}=\dfrac{\partial}{\partial v}\left(\dfrac{\partial F_{2}}{\partial q_{i}}\right)=\sum_{k} \dfrac{\partial^{2}F_{2}}{\partial P_{k}\partial q_{i}}.\dfrac{\partial P_{k}}{\partial v}+\sum_{k}\dfrac{\partial^{2}F_{2}}{\partial q_{k}\partial q_{i}}.\dfrac{\partial q_{k}}{\partial v}\nonumber
\end{eqnarray} So the L.H.S of (\ref{eq2..42}) becomes
\[
\sum_{i}
\begin{vmatrix}
\dfrac{\partial q_{i}}{\partial u} & \sum_{k} \dfrac{\partial^{2}F_{2}}{\partial P_{k}\partial q_{i}}.\dfrac{\partial P_{k}}{\partial u}+\sum_{k}\dfrac{\partial^{2}F_{2}}{\partial q_{k}\partial q_{i}}.\dfrac{\partial q_{k}}{\partial u} \\ 
\dfrac{\partial q_{i}}{\partial v} & \sum_{k} \dfrac{\partial^{2}F_{2}}{\partial P_{k}\partial q_{i}}.\dfrac{\partial P_{k}}{\partial v}+\sum_{k}\dfrac{\partial^{2}F_{2}}{\partial q_{k}\partial q_{i}}.\dfrac{\partial q_{k}}{\partial v}
\end{vmatrix}
\]
\[=\sum_{i} \sum_{k} \dfrac{\partial^{2}F_{2}}{\partial P_{k}\partial q_{i}}
\begin{vmatrix}
\dfrac{\partial q_{i}}{\partial u} & \dfrac{\partial P_{k}}{\partial u} \\ 
\dfrac{\partial q_{i}}{\partial v} & \dfrac{\partial P_{k}}{\partial v} 
\end{vmatrix}+\sum_{i}\sum_{k} \dfrac{\partial^{2}F_{2}}{\partial q_{k}\partial q_{i}}
\begin{vmatrix}
\dfrac{\partial q_{i}}{\partial u} & \dfrac{\partial q_{k}}{\partial u} \\ 
\dfrac{\partial q_{i}}{\partial v} & \dfrac{\partial q_{k}}{\partial v} 
\end{vmatrix}
\] (The second term on the R.H.S is zero by interchanging $i$ and $k$. We replace this zero term by a similar zero term as follows:)
\[=
\sum_{i} \sum_{k} \dfrac{\partial^{2}F_{2}}{\partial P_{k}\partial q_{i}}
\begin{vmatrix}
\dfrac{\partial q_{i}}{\partial u} & \dfrac{\partial P_{k}}{\partial u} \\ 
\dfrac{\partial q_{i}}{\partial v} & \dfrac{\partial P_{k}}{\partial v} 
\end{vmatrix}+\sum_{i}\sum_{k} \dfrac{\partial^{2}F_{2}}{\partial P_{k}\partial P_{i}}
\begin{vmatrix}
\dfrac{\partial P_{i}}{\partial u} & \dfrac{\partial P_{k}}{\partial u} \\ 
\dfrac{\partial P_{i}}{\partial v} & \dfrac{\partial P_{k}}{\partial v} 
\end{vmatrix}
\]
\[=\sum_{k}
\begin{vmatrix}
\sum_{i}  \dfrac{\partial^{2}F_{2}}{\partial P_{k}\partial q_{i}}	\dfrac{\partial q_{i}}{\partial u}+\sum \dfrac{\partial^{2}F_{2}}{\partial P_{k}\partial P_{i}}\dfrac{\partial P_{i}}{u} &&&\dfrac{\partial P_{k}}{\partial u} \\ 
\sum_{i}  \dfrac{\partial^{2}F_{2}}{\partial P_{k}\partial q_{i}}	\dfrac{\partial q_{i}}{\partial v}+ \sum \dfrac{\partial^{2}F_{2}}{\partial P_{k}\partial P_{i}}\dfrac{\partial P_{i}}{v} &&& \dfrac{\partial P_{k}}{\partial v} 
\end{vmatrix}
\]
\[=
\sum_{k}
\begin{vmatrix}
\dfrac{\partial}{\partial u}\left(\dfrac{\partial F_{2}}{\partial P_{k}}\right) & \dfrac{\partial P_{k}}{\partial u} \\ 
\dfrac{\partial}{\partial v}\left( \dfrac{\partial F_{2}}{\partial P_{k}}\right)& \dfrac{\partial P_{k}}{\partial v}
\end{vmatrix}
\]
\[=
\sum_{k}
\begin{vmatrix}
\dfrac{\partial Q_{k}}{\partial u} & \dfrac{\partial P_{k}}{\partial u} \\ 
\dfrac{\partial Q_{k}}{\partial v} & \dfrac{\partial P_{k}}{\partial v}
\end{vmatrix}=\sum_{k} \dfrac{\partial (Q_{k}, P_{k})}{\partial (u,v)}=[u,v]_{(Q,P)}
\]$~~~~~~~~~~~~~~~~~~~~~~~~~~~~~~~~~~~~~~$ Hence the proof.

