


\chapter{Motion in 2-dimension}


\section{Velocity and Acceleration of a particle moving in a plane curve in different frames of references: a vector treatment}

We shall now derive the components of velocity and acceleration in different frames of references namely (a) Cartesian frame of reference, (b) Polar co-ordinates, (c) Intrinsic co-ordinate system.\\

\subsection{Cartesian frame of references}

\subsubsection{Case I: When the axes are fixed in space:}

Let $P(x,y)$ be the position of the particle at time $t$. So the position vector of the particle is given by

\begin{wrapfigure}[7]{r}{0.34\textwidth}\vspace{-1.2\intextsep}
	\includegraphics[height=4.5 cm , width=5.5 cm ]{f2.pdf}
	\begin{center}\vspace{-\intextsep}
		Fig. 2.1
	\end{center}
\end{wrapfigure}


$\vec{r}=x\vec{i}+y\vec{j}$

where $\vec{i}$ and $\vec{j}$ are constant unit vectors parallel to the axes $OX$ and $OY$ respectively.

$\therefore \vec{v}=\dfrac{\mathrm{d}\vec{r}}{\mathrm{d}t}=\dfrac{\mathrm{d}x}{\mathrm{d}t}\vec{i}+\dfrac{\mathrm{d}y}{\mathrm{d}t}\vec{j}=\dot{x}\vec{i}+\dot{y}\vec{j}$

$\vec{f}=\dfrac{\mathrm{d}\vec{v}}{\mathrm{d}t}=\ddot{x}\vec{i}+\ddot{y}\vec{j}$

\subsubsection{Case II: When the axes are not fixed but are rotating:}

In this case $\vec{i}$ and $\vec{j}$ are not constant vectors, so

$\dfrac{\mathrm{d}\vec{i}}{\mathrm{d}t}=\dfrac{\mathrm{d}\vec{i}}{\mathrm{d}\theta}\dot{\theta}=\dot{\theta}\vec{j}$ ($\because \dfrac{\mathrm{d}\vec{i}}{\mathrm{d}\theta}$ is perpendicular to $\vec{i}$ and is a unit vector, see the appendix I)

Similarly, $\dfrac{\mathrm{d}\vec{j}}{\mathrm{d}t}=-\dot{\theta}\vec{i}$

Therefore $\vec{r}=x\vec{i}+y\vec{j}$
\begin{eqnarray}
\vec{v}=\dfrac{\mathrm{d}\vec{r}}{\mathrm{d}t}&=&\dot{x}\vec{i}+\dot{y}\vec{j}+x\dot{\theta}\vec{j}-y\dot{\theta}\vec{j}\nonumber\\&=&(\dot{x}-y\dot{\theta})\vec{i}+(\dot{y}+x\dot{\theta})\vec{j}\nonumber
\end{eqnarray}

Differentiating again
$\vec{f}=\dfrac{\mathrm{d}\vec{v}}{\mathrm{d}t}=(\ddot{x}-x\dot{\theta}^2-2\dot{y}\dot{\theta}-y\ddot{\theta})\vec{i}+(\ddot{y}-y\dot{\theta}^2+2\dot{x}\dot{\theta}+x\ddot{\theta})\vec{j}$

In particular, if the axes are rotating with constant angular velocity $\omega$, then $\dot{\theta}=\omega$, $\ddot{\theta}=0$, so the velocity and acceleration simplifies to
$$v_x=\dot{x}-y\omega,~~ v_y=\dot{y}+x\omega$$and
$$f_x=\ddot{x}-2\omega\dot{y}-x\omega^2,~~f_y=\ddot{y}+2\dot{x}\omega-y\omega^2$$

\subsection{Polar Co-ordinates}

In polar co-ordinate system let $P(r,\theta)$ be the position of the particle at time $t$. Suppose $\hat{r}$ and $\hat{\theta}$ be the unit vectors along and perpendicular to the radius vector.

Now,  $\vec{r}=r\hat{r}$,  $r=OP$, $\hat{r}$ is the unit vector along the radial direction.

\begin{wrapfigure}[7]{r}{0.34\textwidth}\vspace{-1.9\intextsep}
	\includegraphics[height=4.5 cm , width=5.5 cm ]{f3.pdf}
	\begin{center}\vspace{-1.9\intextsep}
		Fig. 2.2
	\end{center}
\end{wrapfigure}

$\therefore ~\vec{v}=\dfrac{\mathrm{d}\vec{r}}{\mathrm{d}t}=\dfrac{\mathrm{d}r}{\mathrm{d}t}\hat{r}+r\dfrac{\mathrm{d}\hat{r}}{\mathrm{d}t}=\dot{r}\hat{r}+r\dfrac{\mathrm{d}\theta}{\mathrm{d}t}\dfrac{\mathrm{d}\hat{r}}{\mathrm{d}\theta}=\dot{r}\hat{r}+r\dot{\theta}\hat{\theta} $

i.e., radial velocity = $\dot{r}$,

Cross-radial velocity = $r\dot{\theta}$

Differentiating once more, we have, 
\begin{eqnarray}
	\vec{f}=\dfrac{\mathrm{d}\vec{v}}{\mathrm{d}t}&=&
	\ddot{r}\hat{r}+\dot{r}\dfrac{\mathrm{d}\hat{r}}{\mathrm{d}t}+\dfrac{\mathrm{d}(r\dot{\theta})}{\mathrm{d}t}\hat{\theta}+r\dot{\theta}\dfrac{\mathrm{d}\hat{\theta}}{\mathrm{d}t}\nonumber\\&=&
	\ddot{r}\hat{r}+\dot{r}\dot{\theta}\hat{\theta}+\dfrac{\mathrm{d}(r\dot{\theta})}{\mathrm{d}t}\hat{\theta}-r\dot{\theta}^2\hat{r}~~~\left(\because \dfrac{\mathrm{d}\hat{\theta}}{\mathrm{d}t}=\dfrac{\mathrm{d}\theta}{\mathrm{d}t}\dfrac{\mathrm{d}\hat{\theta}}{\mathrm{d}\theta}=-\dot{\theta}\hat{r}\right)\nonumber\\&=&
	(\ddot{r}-r\dot{\theta}^2)\hat{r}+\dfrac{1}{r}\dfrac{\mathrm{d}(r^2\dot{\theta})}{\mathrm{d}t}\hat{\theta}\nonumber
\end{eqnarray}

\subsection{Intrinsic co-ordinate system}

We shall first show the following:

``If a particle moves in a plane curve then its velocity will always be along the tangent to the curve"

\begin{wrapfigure}[7]{r}{0.34\textwidth}\vspace{-1.2\intextsep}
	\includegraphics[height=4.5 cm , width=5.5 cm ]{f4.pdf}
	\begin{center}\vspace{-\intextsep}
		Fig. 2.3
	\end{center}
\end{wrapfigure}

Let $s$ be the arc length along the curve.
\begin{eqnarray}
	\vec{v}&=&\dfrac{\mathrm{d}\vec{r}}{\mathrm{d}t}\nonumber\\&=&\dfrac{\mathrm{d}\vec{r}}{\mathrm{d}s}\dot{s}=\dot{s}\vec{t}\nonumber
\end{eqnarray}

Now,\begin{eqnarray}
	\vec{t}=\dfrac{\mathrm{d}\vec{r}}{\mathrm{d}s}&=&\dfrac{\mathrm{d}}{\mathrm{d}s}(r\hat{r})=\dfrac{\mathrm{d}r}{\mathrm{d}s}\hat{r}+r\dfrac{\mathrm{d}\theta}{\mathrm{d}s}\dfrac{\mathrm{d}\hat{r}}{\mathrm{d}\theta}\nonumber\\&=&\dfrac{\mathrm{d}r}{\mathrm{d}s}\hat{r}+r\dfrac{\mathrm{d}\theta}{\mathrm{d}s}\hat{\theta}\nonumber\\&=&\cos\phi~\hat{r}+\sin\phi~\hat{\theta}\nonumber
\end{eqnarray}
which shows that $\vec{t}$ is a unit vector along the tangent to the curve. Hence velocity is always tangent to the curve.

\begin{eqnarray}
	\therefore~\vec{f}=\dfrac{\mathrm{d}\vec{v}}{\mathrm{d}t}&=&\dfrac{\mathrm{d}}{\mathrm{d}t}(\dot{\vec{t}})=\ddot{s}\vec{t}+\dot{s}\dot{s}\dfrac{\mathrm{d}\vec{t}}{\mathrm{d}s}\nonumber\\&=&\ddot{s}\vec{t}+\dot{s}^2\cdot\kappa\vec{n}~~~\mbox{(by Frenet Formula)}\nonumber\\&=& \ddot{s}\vec{t}+\frac{\dot{s}^2}{\rho}\vec{n}\nonumber
\end{eqnarray}
$\kappa$ is the curvature of the curve at the given point, $\rho$ is the radius of curvature of the curve and $\vec{n}$ is the unit vector along the normal.\\

\section{A particle is acted on by a given force. Determine the differential equation of the path of a particle.}

\subsection{Path of the particle in the Cartesian form}

Let $F_x$ and $F_y$ be the components of the force (per unit mass) along the two Cartesian axes $x$ and $y$. So the equation of motion along the axes are 
$$\ddot{x}=F_x\mbox{~~and~~}\ddot{y}=F_y$$

Now, $\dot{y}=\dfrac{\mathrm{d}y}{\mathrm{d}t}=\dfrac{\mathrm{d}y}{\mathrm{d}x}\dot{x}$

$\therefore~\ddot{y}=\dfrac{\mathrm{d}y}{\mathrm{d}x}\ddot{x}+\dfrac{\mathrm{d}^2y}{\mathrm{d}x^2}\dot{x}^2$

$\therefore~F_y=F_x\dfrac{\mathrm{d}y}{\mathrm{d}x}+\dfrac{\mathrm{d}^2y}{\mathrm{d}x^2}\dot{x}^2$

Also from the equation of motion $\ddot{x}=F_x$, if we multiply both sides by $2\dot{x}$ and integrate then we obtain
$$\dot{x}^2=2\int F_x \mathrm{d}x=\chi(x)~~\mbox{(say)}$$

$\therefore$ The differential equation of the path of the particle be
$$\dfrac{\mathrm{d}^2y}{\mathrm{d}x^2}=\frac{F_y-F_x\dfrac{\mathrm{d}y}{\mathrm{d}x}}{\chi(x)}$$\\

\subsection{Path of the particle in Polar form}

Suppose $F_r$ and $F_\theta$ be the component of the force (per unit mass) along the radial and cross-radial direction. So the equations of motion along these directions are
\begin{eqnarray}
		\ddot{r}-r\dot{\theta}^2&=&-F_r\label{eq2.1}\\\mbox{and}~~~\frac{1}{r}\frac{\mathrm{d}}{\mathrm{d}t}(r^2\dot{\theta})&=&F_\theta
\end{eqnarray}

Let us denote $h=r^2\dot{\theta}$, $u=\dfrac{1}{r}$.

$\therefore F_\theta=u\dfrac{\mathrm{d}h}{\mathrm{d}t}=u\dfrac{\mathrm{d}h}{\mathrm{d}\theta}\dfrac{\mathrm{d}\theta}{\mathrm{d}t}=hu^3\dfrac{\mathrm{d}h}{\mathrm{d}\theta}$

Now, $\dot{r}=-\dfrac{1}{u^2}\dfrac{\mathrm{d}u}{\mathrm{d}t}=-\dfrac{1}{u^2}\dfrac{\mathrm{d}u}{\mathrm{d}\theta}\dot{\theta}=-h\dfrac{\mathrm{d}u}{\mathrm{d}\theta}$
\begin{eqnarray}
	\ddot{r}=\dfrac{\mathrm{d}}{\mathrm{d}\theta}\left(-h\dfrac{\mathrm{d}u}{\mathrm{d}\theta}\right)\dot{\theta}=-\left[h\dfrac{\mathrm{d}^2u}{\mathrm{d}\theta^2}+\dfrac{\mathrm{d}h}{\mathrm{d}\theta}\dfrac{\mathrm{d}u}{\mathrm{d}\theta}\right]hu^2\nonumber\\=-hu^2\left[h\dfrac{\mathrm{d}^2u}{\mathrm{d}\theta^2}+\dfrac{F_\theta}{hu^3}\dfrac{\mathrm{d}u}{\mathrm{d}\theta}\right]\nonumber
\end{eqnarray}

So equation (\ref{eq2.1}) can be written as
\begin{eqnarray}
&&-h^2u^2\dfrac{\mathrm{d}^2u}{\mathrm{d}\theta^2}-\dfrac{F_\theta}{u}\dfrac{\mathrm{d}u}{\mathrm{d}\theta}	-h^2u^3=-F_r\nonumber\\
&\therefore& h^2u^2\left[\dfrac{\mathrm{d}^2u}{\mathrm{d}\theta^2}+u\right]=F_r-\dfrac{F_\theta}{u}\dfrac{\mathrm{d}u}{\mathrm{d}\theta}\nonumber
\end{eqnarray}

This is the differential equation of the path of a particle moving under a force.

In particular, if $F_\theta=0$ and $F_r=F(r)$, a function of the radial co-ordinate  (i.e., central force) then the differential equation of the path of the particle is 
$$h^2u^2\left[\dfrac{\mathrm{d}^2u}{\mathrm{d}\theta^2}+u\right]=F$$


\subsection{Differential equation of the path of a particle in intrinsic co-ordinates}

Let $F_T$ and $F_N$ be the components of the force (per unit mass) along the tangential and normal directions. So the equations of motion be
$$\ddot{s}=F_T~,~~\frac{v^2}{\rho}=F_N$$

$v^2=\rho F_N=F_N\dfrac{\mathrm{d}s}{\mathrm{d}\psi}$

$\dot{s}=\left(F_N\dfrac{\mathrm{d}s}{\mathrm{d}\psi}\right)^{\frac{1}{2}}$

$\implies \dfrac{\mathrm{d}s}{\mathrm{d}\psi}\dot{\psi}=\left(F_N\dfrac{\mathrm{d}s}{\mathrm{d}\psi}\right)^{\frac{1}{2}}$

$\implies \dot{\psi}=\left[\dfrac{F_N}{\frac{\mathrm{d}s}{\mathrm{d}\psi}}\right]^{\frac{1}{2}}$

Now, 
\begin{eqnarray}
	F_T&=&\ddot{s}=\dfrac{\mathrm{d}}{\mathrm{d}t}\left[F_N\dfrac{\mathrm{d}s}{\mathrm{d}\psi}\right]^{\frac{1}{2}}=\dfrac{\mathrm{d}}{\mathrm{d}\psi}\left[F_N\dfrac{\mathrm{d}s}{\mathrm{d}\psi}\right]^{\frac{1}{2}}\dot{\psi}\nonumber\\F_T&=&\frac{1}{2}\left[F_N\dfrac{\mathrm{d}s}{\mathrm{d}\psi}\right]^{-\frac{1}{2}}\left\{\dfrac{\mathrm{d}F_N}{\mathrm{d}\psi}\dfrac{\mathrm{d}s}{\mathrm{d}\psi}+F_N\dfrac{\mathrm{d}^2s}{\mathrm{d}\psi^2}\right\}\left[\dfrac{F_N}{\frac{\mathrm{d}s}{\mathrm{d}\psi}}\right]^{\frac{1}{2}}\nonumber\\&=&\frac{1}{2\frac{\mathrm{d}s}{\mathrm{d}\psi}}\left\{\dfrac{\mathrm{d}F_N}{\mathrm{d}\psi}\dfrac{\mathrm{d}s}{\mathrm{d}\psi}+F_N\dfrac{\mathrm{d}^2s}{\mathrm{d}\psi^2}\right\}\nonumber
\end{eqnarray}

Thus the differential equation of the path of the particle in intrinsic co-ordinates can be written as
$$\dfrac{\mathrm{d}^2s}{\mathrm{d}\psi^2}=\frac{1}{F_N}\left(2F_T-\dfrac{\mathrm{d}F_N}{\mathrm{d}\psi}\right)\dfrac{\mathrm{d}s}{\mathrm{d}\psi}$$

