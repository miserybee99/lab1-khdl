%\pagenumbering{arabic}

\chapter{Lagrangian and Hamiltonian formulation}

 
\section{Constraint system}
 
 In nature, there is no absolute concept of free motion. Every dynamical system is restricted by either a geometrical or a physical condition. Such a condition (or a restriction) is called a constraint and the force which gives rise this constraint is called constraint force. In fact, any force acting on constraint surface is called a constraint force.
 
 Consider a system of particles $P_k~(k=1,2,3,...,n.)$ referred to contain frame of reference. $\overrightarrow{r_k}$ is the radius vector drawn to the k-th particle and $\dot{\overrightarrow{r_k}}=\overrightarrow{v_k}$, is the velocity of the k-th particle (`$.$' indicates differentiation with respect to time).
 
 Let there be some constraints imposed on the particles as to their position and velocity. A system moving under no constraint is called a free system. Analytically, constraints are of the form 
 \begin{equation}
 	f(\overrightarrow{r_k},\dot{\overrightarrow{r_k}},f)=0\label{1}
 \end{equation}   
 (Note that this scalar equation has $6n+1$ variables $\overrightarrow{r_k}=(x_k,y_k,z_k)$)
 
 If the constraint equation is independent of the velocity then it is known as finite or geometric constraint. So $f(\overrightarrow{r_k},t)=0$ is an example of these constraints. Usually, constraints of the type (\ref{1}) are known as differential or kinematic constraints. We shall consider those differential constraints which are linear in velocities i.e.,
 \begin{equation}
 	l_1\dot{\overrightarrow{r_1}}+l_2\dot{\overrightarrow{r_2}}+...+l_n\dot{\overrightarrow{r_n}}+D=0\label{2}
 \end{equation}
 Here $l_1,l_2,...,l_3,D$ are scalar functions of $t$ and $\dot{\overrightarrow{r}}$. 
 
 It is to be noted that a finite constraint imposes restrictions as to the position of the system and time while with a differential constraint the system may occupy any arbitrary position in space and in time $'t'$ but in this position the velocities can not be arbitrary and there will be restriction on the velocities.
 
 A finite constraint like $f(\overrightarrow{r_k},t)=0$ implies a differential constraint which is obtained by termwise differentiation as 
 \begin{equation}
 	\Sigma_{k=1}^{n}\frac{\partial f}{\partial r_k}\dot{\overrightarrow{r_k}} +\frac{\partial f}{\partial t}=0\label{3}
 \end{equation}  
 But this differential constraint is equivalent to the finite constraint $f(\overrightarrow{r_k},t)=c$ (which is the original finite constraint if $c=0$ only). The differential constraints (\ref{3}) is called integrable.
 
 The constraints can be classified as follows :
 
 A) : (Scleronomic or Rheonomic)
 
 B) :(Holonomic or Non-holonomic)
 
 C): (Conservative or Dissipative)
 
 D): (Bilateral or Unilateral)
 
 A constraint which does not depend explicitly on time is called a Scleronomic constraint, otherwise it is called a Rheonomic constraint. So for a scleronomic constraint `$f$',~~$\frac{\partial f}{\partial t}=0$. 
 
 So equation (\ref{3}) becomes $~\Sigma_{k=1}^{n}\frac{\partial f}{\partial r_k}\dot{\overrightarrow{r_k}}=0~$, which is linear and homogeneous in velocities. Now comparing this with the differential constraint (\ref{2}) we see that a differential constraint will be stationary if $D=0$ and the vector $\overrightarrow{l}=(l_1,l_2,...,l_3)$ is not an explicit function of `$t$'. 
 
 \vspace{.5cm}
 
 $\bullet${\textbf{Example}} : If we consider the motion of a simple pendulum of constant length then the position of the bob at any time can be characterized by the constraint
 \begin{equation}
 	x^2+y^2+z^2=l^2\label{4}
 \end{equation}
 (l is the constant length of the pendulum)
 
 This constraint is called a scleronomic constraint. But if the length of the pendulum changes with time then the above constraint Still holds with $l=l(t)$. So such a constraint is called a rheonomic constraint.
 
 If a constraint relation does not contain any term depending on velocity then it is called a holonomic constraint, otherwise it s non-holonomic. The constraint in equation (\ref{4}) is an example of  holonomic constraint.
 
 If a dynamical system preserves the total mechanical energy of the system then we say that the dynamical system is in conservative form (here the constraint forces do not work). The non-conservation of the total energy leads a dynamical system to be dissipative (in this case the constraint forces do work).
 
 In principle, a simple pendulum motion is conservative while if its bob slides on a rough circular track, then the system is dissipative (due to friction there will be a loss of energy).
 
 If on the constraint surface, it is possible for the dynamical system to have forward and backward motion then the constraint is called a bilateral constraint. Here the constraints are expressed mathematically in the form of the equations of the type $f(\overrightarrow{r_k},\dot{\overrightarrow{r_k}},t)=0$. On the otherhand, if at some point on the constraint surface the motion is uni-directional then the constrain is called a unilaleral constraint . Usually, the unilateral constraints are expressed in the form of inequalities of the form : $f(r_k,\dot{r_k},t)\geq0$. In the motion of a simple pendulum the constraint surface is the surface of a sphere and the motion is possible in all directions. So the constraint is bilateral. The motion of a gas confined to a spherical container of radius $R$, the constraint is $|r|\leq R$ and hence it is unilateral.
 
 A particle is constrained to move over a surface. Suppose the surface is given by $f(x,y,z)=0$. Then it is a finite stationary constraint.
 
 If the surface is undergoing deformation or is moving , then the equation of the surface will be of the form $f(x,y,z,t)=0$. This is a non-stationary but finite constraint.
 
 The equation of constraint of two particle connected by a rod of constant length is $|\overrightarrow{r_1}-\overrightarrow{r_2}|=l,$   i.e. $(x_1-x_2)^2+(y_1-y_2)^2+(z_1-z_2)^2=l^2$. This is a holonomic, scleronomic system.
 
 But if the length of the rod be variable the equation of constraint will be
 \begin{equation}
 	(x_1-x_2)^2+(y_1-y_2)^2+(z_1-z_2)^2=[f(t)]^2=l^2\nonumber
 \end{equation} 
 This is a holonomic but rheonomic constraint.
 
 Two particles in a plane are connected by a rod of constant length $'l'$ and are constraint to move in a manner such that the velocity of the middle point of the rod is in the direction of the length of the rod. The constraint equations are $z_1=0=z_2~,~(x_1-x_2)^2+(y_1-y_2)^2=l^2$ and $\dfrac{\dot{x_1}+\dot{x_2}}{x_1-x_2}=\dfrac{\dot{y_1}-\dot{y_2}}{y_1-y_2}$.
 
 This is a non-holonomic system because the last equation is a non-integrable differential constraint.
 
 Suppose two particles are connected by a thread and is expressed as $l^2-|r_1-r_2|^2\geq0$ (unilateral constraint). The condition of equality sign is the condition of taut. So a system with unilateral constraint may be regarded in such a way that a part of the constraint refers to the condition of tout i.e., bilateral constraint and the other part is such that there is no such constraint. Thus a unilateral constraint is either a bilateral constraint or is eliminated altogether. Hence one can use only the bilateral constraint.
 
 \vspace{.5cm}
 
 \section{{{Generalised Coordinates} :}}
 
 \vspace{.25cm}
 
 
 The least number of independent variables (compatible with the constraints) which can characterise the position and configuration of a dynamical system are called the degrees of freedom of the dynamical system. Also these independent variables (coordinate) are termed as generalised coordinates.
 
 It may be noted that the cartesian co-ordinates are not necessarily be the generalised coordinates but they may be functions of them. In fact, the particular set of independent variables which is suitable for describing a dynamical system is called the generalised coordinate system.
 
 For example, if we consider the motion of a particle revolving around a fixed attracting  centre (central force problem) then the polar coordinates namely $(r,\theta,\phi)$ are generalised coordinates and the form of the cartesian coordinates are $x=r\cos\phi\sin\theta~, ~y=r\sin\phi\sin\theta~,~z=r\cos\theta$ .
 
 \vspace{.5cm}
 
 \section{{{Lagrange's equation of motion}} :}
 
 \vspace{.25cm}
 Suppose a system is defined at the instant $'t'$ by $k$ generalised coordinates $q_1,q_2,...,q_k$. These coordinates $q$ define position of each member of the system at the instant and hence if $\overrightarrow{r}$ be the position vector of a material point of mass $m$
 of the system at time $t$ relative to some base point, then 
 \begin{equation}
 	\overrightarrow{r}=\overrightarrow{r}(q_1,q_2,...,q_k;t)=\overrightarrow{r}(q_k;t)=\overrightarrow{r}(\overrightarrow{q},t)~~j=1,2,...k.\nonumber
 \end{equation}
 We assume that these function $\overrightarrow{r}(\overrightarrow{q},t)$ together with their partial derivatives are continuous in a certain region of $q$ and $t$. We shall now consider the following results :
 
 \vspace{.5cm}
 
 $\bullet$ \textbf{{Lemma I}} :
 
 
 
 \begin{equation}
 	\frac{\partial\dot{\overrightarrow{r}}}{\partial\dot{\overrightarrow{q_j}}}= \frac{\partial\overrightarrow{r}}{\partial q_j}\label{5}
 \end{equation}
 \textbf{Proof} :
 
 As  $\overrightarrow{r}=\overrightarrow{r}(\overrightarrow{q};t)$ so $\dot{\overrightarrow{r}}=\frac{d\overrightarrow{r}}{dt}=\frac{\partial\overrightarrow{r}}{\partial t}+ \Sigma_{j=1}^{k}\frac{\partial\overrightarrow{r}}{\partial q_j}\dot{q_j}$;
 
 Now differentiating partially with respect to $\dot{q_j}$ we have $\frac{\partial\dot{\overrightarrow{r}}}{\partial\dot{\overrightarrow{q_j}}}= \frac{\partial\overrightarrow{r}}{\partial q_j}$ .
 
 \vspace{.5cm}
 
 $\bullet$ \textbf{{Lemma II}} : Show that 
 \begin{equation}
 	\frac{d}{dt}\left(\frac{\partial\overrightarrow{r}}{\partial q_j}\right) =\frac{\partial}{\partial q_j}\left(\frac{d\overrightarrow{r}}{dt}\right)\label{6}
 \end{equation}
 \textbf{Proof} : As $\overrightarrow{r}$ is a function of $q_j$ and $t$ so the partial derivative $\frac{\partial\overrightarrow{r}}{\partial q_j}$ is also function of $q_j$'s and $t$. Hence 
 \begin{align}
 	\frac{d}{dt}\left(\frac{\partial\overrightarrow{r}}{\partial q_j}\right) &=	\frac{\partial}{\partial t}\left(\frac{\partial\overrightarrow{r}}{\partial q_j}\right) +\Sigma_{p=1}^{k}\frac{\partial^2\overrightarrow{r}}{\partial q_j\partial q_p}\dot{q_p}\nonumber\\
 	&=\frac{\partial}{\partial q_j}\left(\frac{\partial\overrightarrow{r}}{\partial t}\right) + \frac{\partial}{\partial q_j}\left(\Sigma\frac{\partial\overrightarrow{r}}{\partial q_p}\dot{q_p}\right)\nonumber\\
 	&=\frac{\partial}{\partial q_j}\left[\frac{\partial\overrightarrow{r}}{\partial t}+ \Sigma_p\frac{\partial\overrightarrow{r}}{\partial q_p}\dot{q_p} \right]\nonumber\\
 	&=\frac{\partial}{\partial q_j}\left(\frac{d\overrightarrow{r}}{dt}\right).\nonumber
 \end{align}
 Let $\overrightarrow{F_i}$ be the external force at time t on the particle $m_i$ of the system having position vector $\overrightarrow{r_i}(\overrightarrow{q},t)$. Then according to D'Alembert's principle, the system of forces $(\overrightarrow{F_i}-m_i\ddot{\overrightarrow{r_i}})$  acting at different points of the system, are in equilibrium. Hence, for an arbitrary virtual displacement of the system consistent with the constraint, the total work done will be zero i.e., 
 \begin{equation}
 	(\overrightarrow{F_i}-m_i\ddot{\overrightarrow{r_i}})\delta\overrightarrow{r_i}=0 \label{7}
 \end{equation}
 (note that the virtual work done by the reactions of the constraints due to any arbitrary virtual displacement consistent with the constraint is zero). Let $\delta w$ be the virtual work done by the forces $\overrightarrow{F_i}$ on the virtual displacement $\delta\overrightarrow{r_i}$, so we have 
 \begin{align}
 	\delta w=\sum_{i=1}\overrightarrow{F_i}.\delta\overrightarrow{r_i}&= \sum_{i=1}\overrightarrow{F_i}.\sum_{j}\frac{\partial\overrightarrow{r_i }}{\partial q_j}\delta q_j\nonumber\\
 	&=\sum_{j}\delta q_j\sum_{i}\frac{\partial\overrightarrow{r_i }}{\partial q_j}\dot{\overrightarrow{F_i}}\nonumber\\
 	&=\sum_{j}\delta q_j.Q_j\label{8}
 \end{align} 
 where $Q_j$ is called the generalised force associated with the generalised coordinate $q_j$ with expression 
 \begin{equation}
 	Q_j=\sum_{i}\overrightarrow{F_i}.\frac{\partial\overrightarrow{r_i}}{\partial q_j}= \frac{\partial w}{\partial q_j}\label{9}
 \end{equation} 
 Now,
 \begin{align}
 	\sum_{i}m_i\ddot{\overrightarrow{r_i}}\delta\overrightarrow{r_i} &=\sum_{i}m_i\ddot{\overrightarrow{r_i}}\Sigma_{j}\frac{\partial\overrightarrow{r_i }}{\partial q_j}\delta q_j\nonumber\\
 	&=\sum_{j}\delta q_j\left[\frac{d}{dt}\left(\sum_{i}m_i\dot{\overrightarrow{r_i}} \frac{\partial\overrightarrow{r_i}}{\partial q_j}\right)-\sum_{i}m_i \dot{\overrightarrow{r_i}}\frac{d}{dt}\left(\frac{\partial \overrightarrow{r_i}}{\partial q_j}\right)\right]\nonumber\\
 	&=\sum_{j}\delta q_j\left[\frac{d}{dt}\left(\sum_{i}m_i\dot{\overrightarrow{r_i}} \frac{\partial\dot{\overrightarrow{r_i}}}{\partial\dot{q_j}}\right)-\sum_{i}m_i \dot{\overrightarrow{r_i}}\frac{\partial\dot{\overrightarrow{r_i}}}{\partial q_j}\right]	\label{10}
 \end{align}
 (using lemma I in the 1st term and lemma II in the 2nd term)
 
 Let $T$ is the kinetic energy of the system i.e., 
 \begin{align}
 	T&=\frac{1}{2}\sum_{i}m_i\dot{\overrightarrow{r_i}}^2\nonumber\\
 	\frac{\partial T}{\partial\dot{q_j}}&=\sum_{i}m_i\dot{\overrightarrow{r_i}}\frac{\partial \dot{\overrightarrow{r_i}}}{\partial\dot{q_j}}\nonumber
 \end{align}
 and
 \begin{equation}
 	\frac{\partial T}{\partial {q_j}}=\sum_{i}m_i\dot{\overrightarrow{r_i}}\frac{\partial \dot{\overrightarrow{r_i}}}{\partial {q_j}}\nonumber
 \end{equation}
 Using these partial derivatives in (\ref{10}) we have
 \begin{equation}
 	\sum_{i}m_i\ddot{\overrightarrow{r_i}}\delta\overrightarrow{r_i}=\sum_{j}\delta q_j \left[\frac{d}{dt}\left(\frac{\partial T}{\partial\dot{q_j}}\right)-\frac{\partial T}{\partial q_j} \right]\label{11}
 \end{equation}
 Now substituting (\ref{8}) and (\ref{11}) in (\ref{7}) we have
 \begin{equation}
 	\sum_{j=1}^{K}\delta q_j \left[Q_j-\frac{d}{dt}\left(\frac{\partial T}{\partial\dot{q_j}}\right)+\frac{\partial T}{\partial q_j} \right]=0\label{12}
 \end{equation}
 
 \vspace{.25cm}
 
 \textbf{\underline{Case I}} : {Unconnected holonomic system} :
 
 In this case all the generalised coordinates $q_1,q_2,...,q_k$ are all independent and consequently, they will allow arbitrary independent variations.Hence the coefficients of $\delta q_j$ in (\ref{12}) must vanish separately i.e.,
 \begin{equation}
 	\frac{d}{dt}\left(\frac{\partial T}{\partial\dot{q_j}}\right)-\frac{\partial T}{\partial q_j}= Q_j~,~~j=1,2,...,k\label{13}
 \end{equation}
 These are the $'k'$ second order equations and are known as Lagrange's equations for a holonomic unconnected system. Integration of these equations determine q's as a function of $'t'$. Further, if the external field is a potential field then we can write 
 \begin{align}
 	\delta w=-\delta V(q,t)=-\sum_{j}\frac{\partial V}{\partial q_j}\delta q_j\nonumber\\
 	\therefore\frac{\partial w}{\partial q_j}=-\frac{\partial V}{\partial q_j}=Q_j\nonumber
 \end{align}
 Hence the above Lagrange's equation becomes
 \begin{align}
 	\frac{d}{dt}\left(\frac{\partial T}{\partial\dot{q_j}}\right)-\frac{\partial T}{\partial q_j}=-\frac{\partial V}{\partial q_j}\nonumber\\
 	i.e.,\frac{d}{dt}\left(\frac{\partial T}{\partial\dot{q_j}}\right)-\frac{\partial (T-V)}{\partial q_j}=0\nonumber\\
 	i.e.,\frac{d}{dt}\left(\frac{\partial L}{\partial\dot{q_j}}\right)-\frac{\partial L}{\partial q_j}=0\nonumber
 \end{align}
 where $L=T-V$ is called the Lagrangian function ( or Kinetic potential).
 
 \vspace{.25cm}
 
 \textbf{\underline{Case II}} :  {Connected holonomic system} :
 
 In this case, the number of degrees of freedoms being less than the no. of generalised coordinates i.e., all the coordinates $q_1,q_2,...,q_k$ are not all independent. As the system is holonomic and connected the relations between the coordinates are of the form 
 \begin{equation}
 	f_i(q,t)=0,~i=1,2,...,r<k.\nonumber
 \end{equation}
 Thus $r$ coordinates of the system are not all independent. Now without any loss of generality, we assume that first r-coordinates of the k-coordinates are not independent. For virtual variations of the coordinates at time $t$, we can write 
 \begin{equation}
 	\sum_{j=1}^{k}\frac{\partial f_i}{\partial q_j}\delta q_j=0~,~i=1,2,...,r\nonumber
 \end{equation}
 From this it follows that $r$ of the variations $\delta q_1,\delta q_2,...,\delta q_k$ depend on the remaining variations which are assumed to be independent. We now multiply these equations by unknown arbitrary multipliers $\lambda_i$ and sum over $i$ to get
 \begin{eqnarray}
 	\sum_{i=1}^{r}\lambda_i\left(\sum_{j=1}^{k}\frac{\partial f_i}{\partial q_j}\delta q_j \right)=0\nonumber\\
 	i.e.,\sum_{j=1}^{k}\left(\sum_{i=1}^{r}\lambda_i\frac{\partial f_i}{\partial q_j}\right) \delta q_j=0\nonumber
 \end{eqnarray}
 Adding these to equation (\ref{12}) we get
 \begin{equation}
 	\sum_{j=1}^{k}\left[Q_j+\sum_{i=1}^{r}\lambda_i\frac{\partial f_i}{\partial q_j}-\frac{d}{dt}\left(\frac{\partial T}{\partial\dot{q_j}}\right)+ \frac{\partial T}{\partial q_j} \right]\delta q_j=0\label{14}
 \end{equation}
 It is possible to choose $\lambda_i$'s such that the coefficients of first r-dependent variables $\delta q_1,\delta q_2,...,\delta q_r$ vanish separately. Hence with this choice one gets 
 \begin{equation}
 	\frac{d}{dt}\left(\frac{\partial T}{\partial\dot{q_j}}\right)- \frac{\partial T}{\partial q_j} =Q_j+\sum_{i=1}^{r}\lambda_i\frac{\partial f_i}{\partial q_j}\label{15} 
 \end{equation}
 Thus equation (\ref{14}) reduces to
 \begin{equation}
 	\sum_{j=r+1}^{k}\left[Q_j+\sum_{i=1}^{r}\lambda_i\frac{\partial f_i}{\partial q_j}-\frac{d}{dt}\left(\frac{\partial T}{\partial\dot{q_j}}\right)+ \frac{\partial T}{\partial q_j} \right]\delta q_j=0\label{16}
 \end{equation}
 In the above equation (\ref{16}) the variations $\delta q_j~(j=r+1,...,k)$ are all independent and consequently the coefficients must vanish separately to give 
 \begin{equation}
 	\frac{d}{dt}\left(\frac{\partial T}{\partial\dot{q_j}}\right)- \frac{\partial T}{\partial q_j} =Q_j+\sum_{i=1}^{r}\lambda_i\frac{\partial f_i}{\partial q_j}~,~j=r+1,...,k.\label{17}
 \end{equation}
 Thus combining these two cases, one gets
 \begin{equation}
 	\frac{d}{dt}\left(\frac{\partial T}{\partial\dot{q_j}}\right)- \frac{\partial T}{\partial q_j} =Q_j+\sum_{i=1}^{r}\lambda_i\frac{\partial f_i}{\partial q_j}\label{18}
 \end{equation}
 with $i=1,2,...,r<k$ and $j=1,2,...,k.$
 
 These are the system of  $k$ equations with $k+r$ variables (namely $'k'$ no. of $'q'$s and $'r'$ no of $\lambda$'s). These system of equations (\ref{18}) together with the constraint equations are the Lagrange's equation of connected holonomic system defined by $k$ generalised coordinates with $r$ conditions (constrains). For detail about the Lagrangian formulation see Appendix III.
 
 \vspace{.5cm}
 
 \section{Lagrange's equations in non-holonomic system.}
 
 \vspace{.25cm}
 
 The derivation of Lagrange's equation for holonomic system required that the generalised coordinates be independent. For a non-holonomic system however there will be more generalised coordinates than the no of degrees of freedom. Therefore, $\delta q$'s are no longer independent if one assume a virtual displacement consistent with the constraint.
 
 Let $q_1,q_2,...,q_k$ be the $k$ generalised coordinates of the system subject to $r$ non-integrable differential constraints.
 \begin{equation}
 	\sum_{j=1}^{k}a_{ij}\dot{q_j}+a_{i0}=0~,~i=1,2,...,r<k.\label{19}
 \end{equation} 
 where $a'$s are functions of $q$ and $t$ alone. For a virtual variation at time $t$ one gets
 \begin{equation}
 	\sum a_{ij}\delta q_j=0~,~i=1,2,...,r.\label{20}
 \end{equation}
 ($t$ is not allowed to vary)
 
 Now, multiplying equation (\ref{20}) by arbitrary undetermine multiplier $\lambda_i$ and summing over $i$ results
 \begin{eqnarray}
 	\sum_{i}\lambda_{i}\sum_{j} a_{ij}\delta q_j=0\nonumber\\
 	i.e.,\sum_{j}\left(\sum_{i}\lambda_{i}a_{ij}\right)\delta q_j=0\label{21}
 \end{eqnarray}
 Adding equation (\ref{21}) to equation (\ref{6}) one gets
 \begin{equation}
 	\sum_{j}\left[Q_j+\sum_{i}\lambda_ia_{ij} -\frac{d}{dt}\left(\frac{\partial T}{\partial\dot{q_j}}\right)+ \frac{\partial T}{\partial q_j} \right]\delta q_j=0\label{22}
 \end{equation}
 As before $\lambda_{i}$'s are chosen in such a manner that the coefficients of  $\delta q_1,\delta q_2,,...,\delta q_r$ in (\ref{22}) vanish separately. then the remaining variation $\delta q_{r+1},...,\delta q_k$ in (\ref{22}) are perfectly arbitrary and independent and consequently their coefficient must also vanish separately. As a result one gets the following system of equations;
 \begin{equation}
 	\frac{d}{dt}\left(\frac{\partial T}{\partial\dot{q_j}}\right)- \frac{\partial T}{\partial q_j}= Q_j+\sum_{i}\lambda_ia_{ij},
 	~~i=1,2,...,r;~~~i=1,2,...,k.\label{23}
 \end{equation} 
 These are the $k$ Lagrange's equations of motion of non-holonomic system whose constraints are defined by equation (\ref{19}).
 
 \vspace{.5cm}
 
 $\bullet$ \textbf{{Examples}} :
 
 \vspace{.25cm}
 
 \textbf{1}. \underline{Planetary motion} :
 For a planet moving in an elliptic orbit around the sun the K.E.
 \begin{equation}
 	T=\frac{1}{2}m(\dot{r}^2+r^2\dot{\theta}^2)\nonumber
 \end{equation}
 and $\overrightarrow{F}=-\dfrac{m\mu}{r^3}\overrightarrow{r}$ i.e., $V=-\dfrac{m\mu}{r}$
 
 $\therefore L=T-V=\dfrac{1}{2}m(\dot{r}^2+r^2\dot{\theta}^2)+\dfrac{m\mu}{r}$.
 
 Here $m$ is the mass of the planet and that of sun is $\mu$. The two polar coordinates $(r,\theta)$ are the generalised coordinates here. Then Lagrange's equations of motion are
 \begin{equation}
 	\frac{d}{dt}\left(\frac{\partial L}{\partial\dot{r}}\right)-\frac{\partial L}{\partial r}=0\mbox{~~and~~}\frac{d}{dt}\left(\frac{\partial L}{\partial\dot{\theta}}\right)-\frac{\partial L}{\partial \theta}=0\nonumber
 \end{equation}
 \begin{equation}
 	i.e.,\ddot{r}-r\dot{\theta}^2=-\frac{\mu}{r^2}\mbox{~~and~~}\frac{d}{dt}(r^2\theta)=0,\nonumber
 \end{equation}
 which are the well known radial and cross radial equations of motion.
 
 \vspace{.25cm}
 
 \textbf{2}. The Lagrangian $L$ for the motion of a particle of unit mass is given by 
 \begin{equation}
 	L=\frac{1}{2}(\dot{x}^2+\dot{y}^2+\dot{z}^2)-V+\dot{x}A+\dot{y}B+\dot{z}C\nonumber
 \end{equation}
 where $V,A,B,C$ are functions of $x,y,z$. Show that the equations of motion are 
 \begin{equation}
 	\ddot{x}=-\frac{\partial V}{\partial x}+\dot{y}\left(\frac{\partial B}{\partial x}- \frac{\partial A}{\partial y}\right)-\dot{z}\left(\frac{\partial A}{\partial z}-\frac{\partial C}{\partial x} \right)\nonumber
 \end{equation}
 and similar two equations.
 
 \vspace{.25cm}
 
 \textbf{3}. The Lagrangian of a dynamical system is given by 
 \begin{equation}
 	L=m(\dot{q_4}^2-\dot{q_1}^2-\dot{q_2}^2-\dot{q_3}^2)^{\frac{1}{2}}+e\sum_{k=1}^{4}A_k\dot{q_k}\nonumber
 \end{equation}
 where $q_i$'s $(i=1,2,3,4)$ are the generalised coordinates, $A_i$'s are functions of $q_i$'s only and $e,m$ are constant. Show that the Lagrange's equations of motion are
 \begin{equation}
 	m\frac{d}{dt}(\lambda\dot{q_r})=e\sum_{k=1}^{4}\left(\frac{\partial A_r}{\partial q_k} -\frac{\partial A_k}{\partial q_r} \right)\dot{q_r}~,~r=1,2,3\nonumber
 \end{equation}
 and
 \begin{equation}
 	m\frac{d}{dt}(\lambda\dot{q_4})=e\sum_{k=1}^{4}\left(\frac{\partial A_k}{\partial q_4} -\frac{\partial A_4}{\partial q_k} \right)\dot{q_k}\nonumber
 \end{equation}
 where $\lambda^{-2}=\dot{q_4}^2-\dot{q_1}^2-\dot{q_2}^2-\dot{q_3}^2$.
 
 Also evaluate $p_i=\frac{\partial L}{\partial\dot{q_i}},~~(i=1,2,3,4)$ and show that 
 \begin{equation}
 	p_4=eA_4+\sqrt{\sum_{k=1}^{4}(p_k-eA_k)^2+m^2}.\nonumber
 \end{equation}
 
 \vspace{.5cm}
 
 \section{Expression for K.E. of a dynamical system :}
 
 \vspace{.25cm}
 
 The K.E. of a moving system is by definition
 \begin{equation}
 	T=\frac{1}{2}\Sigma m\dot{\overrightarrow{r}}^2~,~~\mbox{where}~\overrightarrow{r}=\overrightarrow{r}(q,t) \nonumber
 \end{equation}
 Here $q$ stands for $k$ generalised coordinates $q_1,q_2,...,q_k,$ so 
 \begin{equation}
 	\dot{\overrightarrow{r}}=\frac{d\overrightarrow{r}}{dt}=\frac{\partial\overrightarrow{r}}{\partial t}+\sum_{j=1}^{k}\frac{\partial\overrightarrow{r}}{\partial q_j}\dot{q_j}\nonumber
 \end{equation}
 Thus
 \begin{eqnarray}
 	T&=&\frac{1}{2}\sum m\left(\frac{\partial\overrightarrow{r}}{\partial t}+\sum_{j=1}^{k}\frac{\partial\overrightarrow{r}}{\partial q_j}\dot{q_j}\right)^2\nonumber\\
 	&=&\frac{1}{2}\sum m\left[\left(\frac{\partial\overrightarrow{r}}{\partial t}\right)^2+ 2\frac{\partial\overrightarrow{r}}{\partial t}\sum_{j=1}^{k}\frac{\partial\overrightarrow{r}}{\partial q_j}\dot{q_j}+ \left(\sum_{j=1}^{k}\frac{\partial\overrightarrow{r}}{\partial q_j}\dot{q_j}\right)^2 \right]\nonumber\\
 	&=&\frac{1}{2}\sum m\left(\frac{\partial\overrightarrow{r}}{\partial t}\right)^2 + \sum_{j=1}^{k}\left(\sum m\frac{\partial\overrightarrow{r}}{\partial t}\frac{\partial\overrightarrow{r}}{\partial q_j}\right)\dot{q_j} +\frac{1}{2}\sum_{i}\sum_{j}\left(\sum m\frac{\partial\overrightarrow{r}}{\partial q_i}\frac{\partial\overrightarrow{r}}{\partial q_j}\right)\dot{q_i}\dot{q_j}\nonumber
 \end{eqnarray}
 Here the terms $\sum m\frac{\partial\overrightarrow{r}}{\partial t}\frac{\partial\overrightarrow{r}}{\partial q_j}$ and $\sum m\frac{\partial\overrightarrow{r}}{\partial q_i}\frac{\partial\overrightarrow{r}}{\partial q_j}$ are summation over the system (i.e., over $m$ and $r$) and not on $i$ and $j$. So they can be symbolically written as $b_j$ and $a_{ij}$ respectively, and we write 
 \begin{equation}
 	T=\frac{1}{2}\sum_{i,j}a_{ij}\dot{q_i}\dot{q_j}+\sum_{i}b_i\dot{q_i}+\frac{1}{2}C \label{24}
 \end{equation}
 where $a_{ij}=\sum m\frac{\partial\overrightarrow{r}}{\partial q_i}\frac{\partial\overrightarrow{r}}{\partial q_j}=a_{ji}~,~b_i=\sum m\frac{\partial\overrightarrow{r}}{\partial t}\frac{\partial\overrightarrow{r}}{\partial q_j}~,~C=\sum m\left(\frac{\partial\overrightarrow{r}}{\partial t}\right)^2.$
 
 In general, $a,b,c$ are functions of $\overrightarrow{r}$ i.e., $q$ and $t$. But in particular, if the system is scleronomic in nature then $b$ and $c$ will be zero and we have 
 \begin{equation}
 	T=\frac{1}{2}\sum_{i}\sum_{j}a_{ij}\dot{q_i}\dot{q_j}
 \end{equation}
 
 \vspace{.5cm}
 
 $\bullet$ \textbf{{Some results}} :
 
 \vspace{.25cm}
 
 \textbf{I.} If the K.E. is homogeneous and quadratic of the components of velocities then 
 \begin{equation}
 	\sum\dot{q}\frac{\partial T}{\partial\dot{q}}-L=T+V\nonumber
 \end{equation}
 \textbf{Proof :} By Euler's theorem on homogeneous function $\sum\dot{q}\frac{\partial T}{\partial\dot{q}}=2T.$
 
 Hence, L.H.S.$=2T-(T-V)=T+V$.
 
 \vspace{.25cm}
 
 \textbf{II.} Show that $\left(\sum\dot{q}\frac{\partial T}{\partial\dot{q}}-L\right)$ is a constant of motion.
 
 \textbf{Proof} : We know that if $A$ is a constant of motion, then $\frac{dA}{dt}=0$.
 
 Here, 
 \begin{eqnarray}
 	\frac{d}{dt}\left[\sum\dot{q}\frac{\partial T}{\partial\dot{q}}-L \right]&=&\frac{d}{dt}\left[ \sum\dot{q}\frac{\partial (T-V)}{\partial\dot{q}}-L\right]~(V\mbox{is independent of }\dot{q})\nonumber\\
 	&=&\frac{d}{dt}\left[\sum\dot{q}\frac{\partial L}{\partial\dot{q}}-L \right]\nonumber\\
 	&=&\cancel{\sum\ddot{q}\frac{\partial L}{\partial\dot{q}}}+\sum\dot{q}\frac{d}{dt}\left( \frac{\partial L}{\partial\dot{q}}\right)-\cancel{\sum\frac{\partial L}{\partial\dot{q}}\ddot{q}} - \sum\frac{\partial L}{\partial q}\dot{q}\nonumber\\
 	&=&\sum\dot{q} \frac{\partial L}{\partial{q}}- \sum\dot{q}\frac{\partial L}{\partial q}=0,~\mbox{by Lagrange's equation of motion.}\nonumber
 \end{eqnarray}
 Thus in a scleronomic system the total energy of the system remains constant during the motion.
 
 \vspace{.5cm}
 
 $\bullet$\textbf{{Problem}} : Determine the motion of a dynamical system whose Lagrangian $L$ is given by 
 \begin{equation}
 	L=\frac{ma^2}{7^2}(16\dot{q_1}^2+20\dot{q_1}\dot{q_2}+25\dot{q_2}^2) - \frac{mga}{2} \left(\frac{1}{3}{q_1}^2+\frac{5}{6}{q_2}^2\right),\nonumber
 \end{equation}
 given that $\dot{q_1}=0=\dot{q_2}$ and ${q_1}=\beta={q_2}$ at $t=0$.
 
 \vspace{.5cm}
 
 \section{Equation of energy :}
 
 \vspace{.25cm}
 
 We have
 \begin{equation}
 	T=\frac{1}{2}\sum_{i}\sum_{j}a_{ij}\dot{q_i}\dot{q_j}=T(q,\dot{q})\nonumber
 \end{equation}
 We assume that the geometrical equations do not depend on $t$ explicitly. So $T$ is a homogeneous quadratic function of velocity and in the above $a_{ij}=a_{ji}$, is a function of $q$'s only. Now,
 \begin{equation}
 	\frac{dT}{dt}=\sum_{j}\frac{\partial T}{\partial q_j}\dot{q_j}+ \sum_{j}\frac{\partial T}{\partial \dot{q_j}}\ddot{q_j}\nonumber
 \end{equation} 
 The Lagrange's equations of motion are 
 \begin{equation}
 	\frac{d}{dt}\left(\frac{\partial T}{\partial\dot{q_j}}\right)-\frac{\partial T}{\partial q_j}= Q_j~,~~ j=1,2,...,k.\nonumber
 \end{equation}
 Now, multiplying these equations by $\dot{q_j}$ and summing over $j$ we have
 \begin{eqnarray}
 	\sum\left[\dot{q_j}\frac{d}{dt}\left(\frac{\partial T}{\partial\dot{q_j}}\right)-\dot{q_j}\frac{\partial T}{\partial q_j}\right]&=&\sum Q_j\dot{q_j}\nonumber\\
 	i.e.,\sum\left[\frac{d}{dt}\left(\dot{q_j}\frac{\partial T}{\partial\dot{q_j}}\right)-\ddot{q_j} \frac{\partial T}{\partial\dot{q_j}}-\dot{q_j}\frac{\partial T}{\partial q_j} \right]&=&\sum Q_j\dot{q_j}\nonumber\\
 	i.e.,2\frac{dT}{dt}-\frac{dT}{dt}&=&\sum Q_j\dot{q_j}~\mbox{(By Euler's theorem)} \nonumber\\
 	i.e.,\frac{dT}{dt}&=&\sum Q_j\dot{q_j}\label{26}
 \end{eqnarray}
 Thus the rate of change of K.E. of a holonomic scleronomic dynamical system with frictionless bilateral constraints is equal to the rate at which work has been done by the external forces. If the external forces are derived form a force function $U,$ independent of $t$ then 
 \begin{equation}
 	U=U(q)~\mbox{and}~Q_i=\frac{\partial U}{\partial q_i}\nonumber
 \end{equation}
 \begin{equation}
 	\mbox{and~}\sum Q_i\dot{q_i}=\sum_{i}\frac{\partial U}{\partial q_i}.\frac{dq_i}{dt}= \frac{dU}{dt}\nonumber
 \end{equation}
 Thus the above energy equation takes the form $\frac{dT}{dt}=\frac{dU}{dt}$ i.e., $T-U=$Constant i.e.,$T+V=$Constant, where V=-U is the potential energy.
 
 \vspace{.5cm}
 
 \section{Gerneralised momentum and cyclic co-ordinate :}
 
 \vspace{.25cm}
 
 Let $L=L(q,\dot{q},t)$ be the Lagrangian of a mechanical system of $n$ d.f. defined at any instant $'t'$ by $n$ generalised coordinates $q_1,q_2,...,q_n$. The generalised or conjugate momentum $p_i$ associated with the generalised coordinate $q_i$ is defined as 
 \begin{equation}
 	p_i=\frac{\partial L}{\partial\dot{q_i}}\label{27}
 \end{equation}
 In terms of the generalised momentum Lagrange's equations of motion becomes 
 \begin{equation}
 	\dot{p_i}=\frac{\partial L}{\partial{q_i}}\label{28}
 \end{equation}
 A coordinate $q_i$ is said to be cyclic or ignorable when $L$ does not involve this coordinate explicitly. The velocity corresponding to cyclic coordinate may remain in $L$. If $q_i$ be cyclic then $\frac{\partial L}{\partial q_i}=0$ and hence $\dot{p_i}=0$ i.e., $p_i=$constant. Thus generalised momentum associated with cyclic coordinate is a conserved quantity.
 
 \vspace{.5cm}
 
 \section{Lagrangian system with cyclic or ignorable coordinates :}
 
 \vspace{.25cm}
 
 Let us consider a holonomic dynamical system with $k$ d.f. and $q_1,q_2,...,q_k$ be the generalised coordinates of the system. If the external forces are derived from a force function $U$, then the Lagrange's equations of motion are
 \begin{equation}
 	\frac{d}{dt}\left(\frac{\partial L}{\partial\dot{q_i}}\right)-\frac{\partial L}{\partial q_i}=0,~~ i=1,2,...,k\label{29}
 \end{equation}
 Suppose $\mu(<k)$ of these coordinates say $q_1,q_2,...,q_\mu$ are ignorable then we have $\mu$ first integrals of Lagrange's equations in the form 
 \begin{equation}
 	\frac{\partial L}{\partial\dot{q_i}}=\mbox{constant}=\beta_i,~~i=1,2,...,\mu\label{30}
 \end{equation}
 As $L=T-V$ and $T$ is a quadratic function of generalised velocities $\dot{q_1},\dot{q_2},...,\dot{q_k},$ so equation (\ref{30}) are linear in $\dot{q_1},\dot{q_2},...,\dot{q_\mu}$. Therefore from these $\mu-$equations (\ref{30}), we can express $\dot{q_1},\dot{q_2},...,\dot{q_\mu}$ in terms of $\dot{q_{\mu+1}},\dot{q_{\mu+2}},...,\dot{q_k}, \\{q_{\mu+1}},...,{q_k},\beta_1,...,\beta_k$ and $t$ i.e.,
 
 $\dot{q_i}=\dot{q_i}(\dot{q}_{\mu+1},\dot{q}_{\mu+2},...,\dot{q_k},\dot{q}_{\mu+1},...,\dot{q_k},\beta_1,...,\beta_{\mu},t),~~i=1,2,...,\mu.$
 Let us now introduce a function $R$ defined as 
 \begin{equation}
 	R=L-\sum_{i=1}^{\mu}\dot{q_i}\frac{\partial L}{\partial\dot{q_i}}=L-\sum_{i=1}^{\mu} \beta_i\dot{q_i}\nonumber
 \end{equation}
 i.e.,
 \begin{eqnarray}
 	R({q}_{\mu+1},{q}_{\mu+2},...,{q_k},\dot{q}_{\mu+1},...,\dot{q_k},\beta_1,...,\beta_{\mu},t)~~~~~~~~~~~~~~~~~~~~~~~~~~~~~~~~~~~~~~~~\nonumber\\	=L({q}_{\mu+1},{q}_{\mu+2},...,{q_k},\dot{q}_{1},...,\dot{q_k},\beta_1,...,\beta_{\mu},t)-\sum_{i=1}^{\mu}\beta_i\dot{q_i}\label{31}
 \end{eqnarray}
 Consider a virtual variation of $R$ at any instant $t$
 \begin{equation}
 	\delta R=\delta\left(L-\sum_{i=1}^{\mu}\dot{q}_i\beta_i \right)\nonumber
 \end{equation}
 \begin{eqnarray}
 	i.e.,&~&\sum_{i=\mu+1}^{k}\frac{\partial R}{\partial q_i}\delta q_i + \sum_{i=\mu+1}^{k}\frac{\partial R}{\partial\dot{q}_i}\delta\dot{q}_i + \sum_{i=1}^{\mu}\frac{\partial R}{\partial\beta_i}\delta\beta_i\nonumber\\
 	&~&=\sum_{i=\mu+1}^{k}\frac{\partial L}{\partial q_i}\delta q_i +\cancel{ \sum_{i=1}^{\mu}\frac{\partial L}{\partial\dot{q}_i}\delta\dot{q}_i} + \sum_{i=\mu+1}^{k}\frac{\partial L}{\partial\dot{q}_i}\delta\dot{q}_i -\sum_{i=1}^{\mu} \dot{q}_i\delta\beta-\cancel{\sum_{i=1}^{\mu}\beta_i\delta\dot{q}_i}\nonumber\\
 	i.e.,&~&\sum_{i=\mu+1}^{k}\left(\frac{\partial R}{\partial q_i}-\frac{\partial L}{\partial q_i} \right)\delta q_i + \sum_{i=\mu+1}^{k}\left(\frac{\partial R}{\partial\dot{q}_i}-\frac{\partial L}{\partial\dot{q}_i} \right)\delta\dot{q}_i + \sum_{i=1}^{\mu}\left(\frac{\partial R}{\partial\beta_i}+\dot{q}_i\right)\delta\beta_i=0\nonumber
 \end{eqnarray}
 This relation is true for arbitrary variations, therefore,
 \begin{eqnarray}
 	\mbox{therefore,~}\frac{\partial R}{\partial q_i}&=&\frac{\partial L}{\partial q_i},~~i=\mu+1,...,k\nonumber\\
 	\frac{\partial R}{\partial\dot{q}_i}&=&\frac{\partial L}{\partial\dot{q}_i},~~i=\mu+1,...,k\nonumber\\
 	\mbox{and~~}\dot{q}_i&=&-\frac{\partial R}{\partial\beta_i},~~i=1,2,...,\mu\label{32}
 \end{eqnarray}
 Putting these relations in Lagrange's equations of motion i.e.,
 \begin{equation}
 	\frac{d}{dt}\left(\frac{\partial L}{\partial\dot{q_i}}\right)-\frac{\partial L}{\partial q_i}=0,~~ i=\mu+1,...,k\nonumber
 \end{equation}
 we obtain
 \begin{equation}
 	\frac{d}{dt}\left(\frac{\partial R}{\partial\dot{q_i}}\right)-\frac{\partial R}{\partial q_i}=0,~~ i=\mu+1,...,k\label{33}
 \end{equation}
 Thus we have reduced the original Lagrange's equations with $'k'$ degrees of freedom to another set of Lagrange's equations with $k-\mu$ degrees of freedom. These $k-\mu$ equations will determine $q_{\mu+1},...,q_k$ as function of $t$. Subsequently, we can determine $q_1,q_2,...,q_{\mu}$ from the equations 
 \begin{equation}
 	\frac{\partial R}{\partial\beta_i}=-\dot{q}_i~\mbox{~i.e.,~}~q_i=-\int\frac{\partial R}{\partial\beta_i}dt~,~i=1,2,...,\mu\nonumber
 \end{equation} 
 Here the function $R$ defined in equation (\ref{31}) is called the modified Lagrangian function or Routhian.
 
 \vspace{.5cm}
 
 $\bullet$ \textbf{{Problem :}}
 
 \vspace{.25cm}
 
 Discuss the planetary motion in terms of Routhian.
 
 For planetary motion,
 \begin{equation}
 	T=\frac{1}{2}(\dot{r}^2+r^2\dot{\theta}^2)~,~V=-\frac{\mu}{r}\mbox{~~so~~}L=\frac{1}{2}(\dot{r}^2+r^2\dot{\theta}^2)+\frac{\mu}{r}\nonumber
 \end{equation}
 Here $\theta$ is a cyclic coordinate and hence $\frac{\partial L}{\partial\dot{\theta}} =r^2\dot{\theta}=\beta$, a constant.
 
 Thus the Routhian function is given by 
 \begin{equation}
 	R=L-\dot{\theta}\beta=\frac{1}{2}\dot{r}^2-\frac{1}{2}r^2\dot{\theta}^2+\frac{\mu}{r}=\frac{1}{2}\dot{r}^2-\frac{1}{2}\frac{\beta^2}{r^2}+\frac{\mu}{r}\nonumber
 \end{equation}
 By Lagrange's equation of motion for $r$
 \begin{eqnarray}
 	\frac{d}{dt}\left(\frac{\partial R}{\partial\dot{r}}\right)-\frac{\partial R}{\partial r}&=&0\nonumber\\
 	\mbox{i.e.,~~}\ddot{r}-\left(\frac{B^2}{r^3}-\frac{\mu}{r^2}\right)&=&0\nonumber
 \end{eqnarray}
 Integrating once
 \begin{eqnarray}
 	\dot{r}^2&=&-\frac{\beta^2}{r^2}+\frac{2\mu}{r}+h\nonumber\\
 	\therefore\frac{dr}{d\theta}=\frac{\dot{r}}{\dot{\theta}}&=&\pm\frac{r^2}{\beta}\sqrt{h-\frac{\beta^2}{r^2}+\frac{2\mu}{r}}\nonumber\\
 	\mbox{i.e.,~}\pm d\theta=\frac{\left(\frac{\beta}{r^2}\right)dr}{\sqrt{h-\frac{\beta^2}{r^2}+\frac{2\mu}{r}}}&=&\frac{-d\left(\frac{\beta}{r}\right)}{\sqrt{\left(h+\frac{\mu^2}{\beta^2}\right)-\left\{ \frac{\beta}{r}-\frac{\mu}{\beta}\right\}^2}}\nonumber\\
 	&=&\frac{-d\left(\frac{\beta}{r}-\frac{\mu}{\beta}\right)}{\sqrt{\left(h+\frac{\mu^2}{\beta^2}\right)-\left(\frac{\beta}{r}-\frac{\mu}{\beta}\right)^2}}\nonumber\\
 	\mbox{i.e.,~~}\pm(\theta+\alpha)&=&\arccos\left[\frac{\left(\frac{\beta}{r}-\frac{\mu}{\beta}\right)}{\sqrt{\left(h+\frac{\mu^2}{\beta^2}\right)}}\right]\nonumber\\
 	\mbox{i.e.,~~}\frac{\beta}{r}&=&\frac{\mu}{\beta}+\sqrt{h+\frac{\mu^2}{\beta^2}}\cos(\theta+\alpha)\nonumber\\
 	\mbox{i.e.,~~}\frac{\left(\frac{\beta^2}{\mu}\right)}{r}&=&1+\sqrt{1+\frac{\beta^2h}{\mu^2}} \cos(\theta+\alpha)\nonumber
 \end{eqnarray}
 $\rightarrow$The elliptic path of the planet.
 
 \vspace{.25cm}
 
 \textbf{Problem: 1}
 
 The Lagrangian of a dynamical system is 
 \begin{equation}
 	L=\frac{q_1^2}{aq_2+b}+\frac{1}{2}\dot{q}_2^2+2q_2^3+cq_2\nonumber
 \end{equation}
 where $a,b,c$ are given constants. Find an integral giving $q_2$ as a function of $t$.
 
 \vspace{.25cm}
 
 \textbf{Problem: 2}
 
 In a dynamical system with $2$ d.f., the K.E. is given by 
 \begin{equation}
 	L=\frac{\dot{q}_1^2}{2(aq_2+b)}+\frac{1}{2}q_2^2\dot{q}_2^2~,~V=c+dq_2\nonumber
 \end{equation}	
 where $a,b,c$ and $d$ are given constants. Show that the value of $q_2$ in terms of time is given by the equation of the form 
 \begin{equation}
 	(q_2-k)(q_2+2k)^2=h(t-t_0)^2\nonumber
 \end{equation}
 with $h,k,t_0$ as constants.
 
 \vspace{.25cm}
 
 \textbf{Problem: 3}
 
 The energies of a dynamical system with $2$ d.f. are given by
 \begin{equation}
 	L=\frac{\dot{q}_1^2}{2(aq_2+b)}+\frac{1}{2}q_2^2\dot{q}_2^2~,~V=c+dq_2^2\nonumber
 \end{equation} 
 Find $q_1$ and $q_2$ by the method of ignorations of coordinates.
 
 \vspace{.5cm}
 
\section{Lioville's theorem for a dynamical system :}
 
 \vspace{.25cm}
 
 $\bullet$ \textbf{Statement } : If for a dynamical system $T$ and $V$ are of the form 
 \begin{eqnarray}
 	2T&=&\{u_1(q_1)+...+u_n(q_n)\}\{v_1(q_1)\dot{q}_1^2+...+v_n(q_n) \dot{q}_n^2\}\nonumber\\
 	&=&u\sum_{r=1}^{n}v_r(q_r)\dot{q}_r^2~,~~u=\sum_{r=1}^{n}u_r(q_r)\label{34}	
 \end{eqnarray}
 \begin{equation}
 	V=-\phi(\mbox{force function})=\frac{w_1(q_1)+...+w_n(q_n)}{u_1(q_1)+...+u_n(q_n)}=\frac{1}{u}\sum_{r=1}^{n}w_r(q_r)\nonumber\\
 \end{equation}
 where $q_1,q_2,...,q_n$ are independent parameters defining the position of system then the solution of the problem can be obtained by a quadrature.
 
 \textbf{{Proof} :}
 
 Let us make a change of variables from$q_1,q_2,...,q_n$ to $Q_1,Q_2,...,Q_n$ such that 
 \begin{equation}
 	\dot{Q}_r^2=v_r(q_r)\dot{q}_r^2\label{35}
 \end{equation}
 \begin{eqnarray}
 	\mbox{i.e.,~}Q_r&=&\int\sqrt{v_r(q_r)}dq_r~,~r=1,2,...,n.\nonumber\\
 	\mbox{So~}u=\sum u_r(q_r)&=&\sum U_r(Q_r)=U\nonumber\\
 	V=\frac{1}{u}\sum w_r(q_r)&=&\frac{1}{U}\sum W_r(Q_r)\nonumber\\
 	\mbox{Hence,~}T&=&\frac{1}{2}U\sum_{r=1}^{n}\dot{Q}_r^2\nonumber
 \end{eqnarray}
 The Lagrange's equation of motion 
 \begin{equation}
 	\frac{d}{dt}\left(\frac{\partial T}{\partial\dot{Q_r}}\right)-\frac{\partial T}{\partial Q_r}=-\frac{\partial V}{\partial Q_r}~,~r=1,2,...,r\nonumber
 \end{equation}
 now becomes 
 \begin{eqnarray}
 	\frac{d}{dt}(U\dot{Q}_r)-\frac{1}{2}\frac{\partial U}{\partial Q_r}\sum\dot{Q}_r^2&=&-\frac{\partial V}{\partial Q_r}\nonumber\\
 	\mbox{i.e.,}2U\dot{Q}_r\frac{d}{dt}(U\dot{Q}_r)-U\dot{Q}_r\frac{\partial U}{\partial Q_r}\sum\dot{Q}_r^2&=&-2U\dot{Q}_r\frac{\partial V}{\partial Q_r} \nonumber\\
 	\mbox{i.e.,}\frac{d}{dt}(U^2\dot{Q}_r^2)-\dot{Q_r}\left[U\frac{\partial{U}}{\partial{Q_r}}\sum\dot{Q_r}^2-2U\frac{\partial{V}}{\partial{Q_r}}\right]&=&0\nonumber
 \end{eqnarray}
 Form the energy equation : $T+V=h$, a constant.
 
 We have $\dfrac{1}{2}U\sum\dot{Q_r}=h-V$
 
 So using this result in the above equation we obtained
 \begin{eqnarray}
 	\frac{d}{dt}(U^2\dot{Q}_r^2)-\dot{Q_r}\left[\frac{\partial{U}}{\partial{Q_r}}(2h-2V)-2U\frac{\partial{V}}{\partial{Q_r}}\right]&=&0\nonumber\\
 	\mbox{i.e.,}\frac{d}{dt}(U^2\dot{Q}_r^2)-2\dot{Q_r}\left[h\frac{\partial{U}}{\partial{Q_r}}-\frac{\partial}{\partial{Q_r}}\sum{W_r(Q_r)}\right]&=&0\nonumber\\
 	\mbox{i.e.,}\frac{d}{dt}(U^2\dot{Q}_r^2)-2\dot{Q_r}\left[h\frac{d{U_r}}{d{Q_r}}-\frac{d{W_r}}{d{Q_r}}\right]&=&0\nonumber\\
 	\mbox{i.e.,}\frac{d}{dt}\left(U^2\dot{Q}_r^2-2hU_r+2W_r\right)&=&0\nonumber\\
 	\mbox{i.e.,}U^2\dot{Q}_r^2-2hU_r+2W_r&=&\mbox{Constant}\nonumber\\
 	\mbox{i.e.,}\frac{1}{2}u^2v_r(q_r)\dot{q_r}^2&=&hu_r-w_r+\gamma_r~\mbox{(say)}\nonumber\\
 	&=&\chi_r(q_r)\nonumber\\
 	&~&~~~~~~\mbox{(in the old coordinates $q_r$)}\nonumber\\
 	\mbox{i.e.,}\frac{\sqrt{2}}{u}dt&=&\sqrt{\frac{v_r(q_r)}{\chi_r(q_r)}}dq_r\label{36}
 \end{eqnarray}
 Thus we have 
 \begin{eqnarray}
 	\int\sqrt{\frac{v_1(q_1)}{\chi_1(q_1)}}dq_1&=&\int\sqrt{\frac{v_2(q_2)}{\chi_2(q_2)}}dq_2+\beta_2\nonumber\\
 	&=&...\nonumber\\
 	&=&...\nonumber\\
 	&=&\int\sqrt{\frac{v_r(q_r)}{\chi_r(q_r)}}dq_r+\beta_r\nonumber\\
 	&=&...\nonumber\\
 	&=&...\nonumber\\
 	&=&\int\sqrt{\frac{v_n(q_n)}{\chi_n(q_n)}}dq_n+\beta_n\label{37}
 \end{eqnarray}
 Now, multiplying equation (\ref{36}) by $u_r(q_r)$ and summing over `$r$' from $1$ to $n$ we get
 \begin{equation}
 	\frac{2}{u}\sum{u_r}(q_r)dt=\sum_{r=1}^{n}u_r(q_r)\sqrt{\frac{v_r(q_r)}{\chi_r(q_r)}}dq_r\nonumber
 \end{equation}
 Integrating once we have
 \begin{equation}
 	\sqrt{2}t+c=\int\sum{u_r(q_r)}\sqrt{\frac{v_r(q_r)}{\chi_r(q_r)}}dq_r\label{38}
 \end{equation} 
 Equation (\ref{37}) gives the coordinates in terms of any one of them and then equation (\ref{38}) determines all the coordinates as a function of $t$. Thus equation (\ref{37}) and (\ref{38}) give the solution of the problem subject to the condition that the solution contain $2n$ arbitrary constant. Again from the equation 
 \begin{equation}
 	\frac{1}{2}u^2v_r(q_r)\dot{q_r}^2=hu_r-w_r+\gamma_r\nonumber
 \end{equation} 
 we have on addition, 
 \begin{eqnarray}
 	u\frac{u}{2}\sum{v_r}\dot{q_r}^2&=&h\sum{u_r}-\sum{w_r}+\sum{\gamma_r}\nonumber\\
 	\mbox{i.e.,~}u.T&=&h.u-u.V+\sum{\gamma_r}\nonumber\\
 	\mbox{i.e.,~}u(T+V)&=&u.h+\sum{\gamma_r}\nonumber\\
 	\mbox{i.e.,~}\sum{\gamma_r}&=&0.\nonumber
 \end{eqnarray} 
 Hence we have $2n-1$ arbitrary constants in the solutions.
 
 \vspace{.5cm}
 
 $\bullet$ \textbf{{Problem: 1} }
 
 \vspace{.25cm}
 
 Show that the dynamical system for which 
 \begin{equation}
 	2T=q_1q_2(\dot{q_1}^2+\dot{q_2}^2),~V=\frac{1}{q_1}+\frac{1}{q_2}\nonumber
 \end{equation}
 can be expressed as one of Liouville's type.
 
 \vspace{.25cm}
 
 \textbf{{Solution} :}
 \begin{eqnarray}
 	\mbox{~~~~~~~~~~~~~~~~~~~~Let,~} q_1=r_1+r_2,~q_2&=&r_1-r_2\nonumber\\
 	\therefore{2T}&=&(r_1^2-r_2^2)\left[2\dot{r_1}^2+2\dot{r_2}\right]\nonumber\\
 	\mbox{i.e.,}~T&=&(r_1^2-r_2^2)\left(\dot{r_1}^2+\dot{r_2}\right)\nonumber\\
 	V&=&\frac{2r_1}{\dot{r_1}^2-\dot{r_2}}\nonumber
 \end{eqnarray}
 So the dynamical problem is of Liouville's type.
 
 \vspace{.5cm}
 
 $\bullet$ \textbf{{Problem: 2} }
 
 \vspace{.25cm}
 
 The K.E. and P.E. of a dynamical system is given by 
 \begin{equation}
 	T=\frac{1}{2}({q_1}^2+{q_2}^2)(\dot{q_1}^2+\dot{q_2}^2),~V=\frac{1}{{q_1}^2+{q_2}^2}\nonumber
 \end{equation}
 Show by Liouville's theorem that the relation between $q_1$ and $q_2$ is $a^2q_1^2+b^2q_2^2+2abq_1q_2\cos\gamma=\sin^2\gamma$ where $a,~b,~\gamma$ are constants.
 
 \vspace{.25cm}
 
 \textbf{{Solution} :}
 
 \vspace{.25cm}
 
 Here, $u_1=q_1^2,~u_2=q_2^2;~v_1=1,~v_2=1; w_1=1,w_2=0~$ so the dynamical system is a Liouville's type.
 \begin{eqnarray}
 	\mbox{Here,~}\chi_1&=&hu_1-w_1+\gamma=hq_1^2-1+\gamma\nonumber\\
 	\chi_2&=&hu_2-w_2-\gamma=hq_2^2-\gamma\nonumber
 \end{eqnarray}
 
 Hence, the first integral can be written as 
 
 \begin{eqnarray}
 	\int\sqrt{\frac{v_1}{\chi_1}}dq_1&=&\int\sqrt{\frac{v_2}{\chi_2}}dq_2+\beta\nonumber\\
 	\mbox{i.e.,}\int\frac{dq_1}{\sqrt{hq_1^2+\gamma-1}}&=&\int\frac{dq_2}{\sqrt{hq_2^2-\gamma}}+\beta\label{39A}\\
 	\mbox{i.e.,}\int\frac{dq_1}{\sqrt{q_1^2+\frac{\gamma-1}{h}}}&=&\int\frac{dq_2}{\sqrt{q_2^2-\frac{\gamma}{h}}}+\beta\sqrt{h}\nonumber\\
 	\mbox{i.e.,}\ln\left|{q_1+\sqrt{q_1^2+\frac{\gamma-1}{h}}}\right|&=&\ln\left|{q_2+\sqrt{q_2^2-\frac{\gamma}{h}}}\right|+\beta\sqrt{h}\nonumber\\
 	\mbox{i.e.,}\frac{q_1+\sqrt{q_1^2+\frac{\gamma-1}{h}}}{q_2+\sqrt{q_2^2-\frac{\gamma}{h}}}&=&e^{\beta\sqrt{h}}=c\mbox{~(say)}\nonumber\\
 	\mbox{i.e.,}q_1-cq_2&=&\left(c\sqrt{q_2^2-\frac{\gamma}{h}}-\sqrt{q_1^2+\frac{\gamma-1}{h}}\right)\nonumber
 \end{eqnarray}
 
 \begin{eqnarray}
 	\mbox{i.e.,}q_1^2+c^2q_2^2-2cq_1q_2&=&c^2\left(q_2^2-\frac{\gamma}{h}\right)+q_1^2+\frac{\gamma-1}{h}-2c\sqrt{\left(q_2^2-\frac{\gamma}{h}\right)\left(q_1^2+\frac{\gamma-1}{h}\right)}\nonumber\\
 	\mbox{i.e.,}\frac{c^2\gamma}{h}-\frac{\gamma-1}{h}-2cq_1q_2&=&-2c\sqrt{\left(q_2^2-\frac{\gamma}{h}\right)\left(q_1^2+\frac{\gamma-1}{h}\right)}\nonumber\\
 	\mbox{i.e.,}k-2cq_1q_2&=&-2c\sqrt{q_1^2q_2^2-\frac{\gamma{q_1}^2}{h}+\frac{\gamma-1}{h}q_2^2-\frac{\gamma(\gamma-1)}{h^2}},~~~~k=\frac{c^2\gamma-\gamma+1}{h}\nonumber
 \end{eqnarray}
 squaring both sides
 \begin{eqnarray}
 	\mbox{i.e.,~}k^2-2ckq_1q_2+4c^2q_1^2q_2^2&=&4c^2q_1^2q_2^2-4c^2\frac{\gamma{q_1}^2}{h}+4c^2\frac{\gamma-1}{h}q_2^2-4c^2\frac{\gamma(\gamma-1)}{h^2}\nonumber\\
 	\mbox{i.e.,~}4c^2\frac{\gamma{q_1}^2}{h}+4c^2\frac{1-\gamma}{h}q_2^2-4ckq_1q_2&=&4c^2\frac{\gamma(1-\gamma)}{h^2}-k^2\nonumber\\
 	\mbox{i.e.,}a^2q_1^2+b^2q_2^2+2abq_1q_2\cos\gamma&=&\sin^2\gamma\nonumber
 \end{eqnarray}
 where, $a^2=\dfrac{4c^2\gamma}{h},~b^2=\dfrac{4c^2(1-\gamma)}{h},~4ck=4c\left[\dfrac{c^2\gamma}{h}-\dfrac{\gamma-1}{h}\right]=\dfrac{4c}{h}(c^2\gamma-\gamma+1)$
 
 i.e.,$4ck=2\sqrt{\dfrac{c^2\gamma}{h}}2\sqrt{\dfrac{c^2(1-\gamma)}{h}}.\dfrac{c^2\gamma-\gamma+1}{c\sqrt{\gamma(1-\gamma)}}=2ab\cos\gamma$
 
 $\therefore\cos\gamma=\dfrac{c^2\gamma-\gamma+1}{2c\sqrt{\gamma(1-\gamma)}}~\&~\sin^2\gamma=1-\dfrac{(c^2\gamma-\gamma+1)^2}{4c^2\gamma(1-\gamma)}$
 
 If $h<0$ then from (\ref{39A})
 \begin{eqnarray}
 	&~&-\int\frac{dq_1}{\sqrt{(\frac{1-\gamma}{-h})-q_1^2}}=-\int\frac{dq_2}{\sqrt{(\frac{\gamma}{-h})-q_2^2}}+\beta\sqrt{-h}\nonumber\\
 	&~&\mbox{i.e,}~~\cos^{-1}\frac{q_1}{\sqrt{\left(\frac{1-\gamma}{-h}\right)}}=\cos^{-1}\frac{q_2}{\sqrt{\left(\frac{\gamma}{-h}\right)}}+\beta\sqrt{-h}\nonumber\\
 	&~&\mbox{i.e,}~~\cos^{-1}\left[\frac{q_1q_2}{\sqrt{\frac{(1-\gamma)\gamma}{h^2}}}-\sqrt{\left(1-\frac{q_1^2h}{1-\gamma}\right)\left(1-\frac{q_2^2h}{\gamma}\right)}\right]=\pi-\gamma\nonumber
 \end{eqnarray}
 
 
 
 ~~~~~~~~~~~~~~~~~~~~~~~~~~~~~~~~~~~~~~~~~~~~~~~~~~~~~~~~~~~~~~~~~~~~~~~~~~~~~~~~~~~~~~~~~~~~~~~~~~~~~~~~~~~~~~~~~~~~~~~~~~~~~~~~~~~~~~~~~~~~~~~~~~~~~~~~~~~~~~~~~~~~~~~~~~~~~~~~~~~~~~~~~~~~~~~~~~~~~~~~~~~~~~~~~~~~~~~~~~~
 \vspace{.5cm}
 
 $\bullet$ \textbf{{Problem: 3} }
 
 \vspace{.25cm}
 
 The K.E. and P.E. of a particle moving in a plane are given by 
 \begin{equation}
 	2T=\dot{x}^2+\dot{y}^2,~V(r)=-\frac{\mu}{r}-\frac{\mu'}{r'}\nonumber
 \end{equation}
 where $r,r'$ are the distances of the particle from the points $(c,0)$ and $(-c,0)$ respectively. Prove by using Liouville's theorem that the problem can be solved completely.
 \begin{figure}[h!]
 	\centering
 	\includegraphics[width=0.6\textwidth]{Sir_graph_1.pdf}\\
 	\label{fig1}
 \end{figure}
 
 \textbf{{Solution} :}
 
 \vspace{.25cm}
 
 Here $r^2=(x-c)^2+y^2,~{r'}^2=(x-c)^2+y^2$. 
 
 Let, $r=q_1+r_2,~r'=q_1-q_2$. 
 
 Now,
 \begin{eqnarray}
 	V&=&\frac{-\mu(q_1-q_2)-\mu'(q_1+q_2)}{(q_1^2-q_2^2)}\nonumber\\
 	&=&-\frac{-\left[(\mu+\mu')q_1+(\mu'-\mu)q_2\right]}{(q_1^2-q_2^2)}\nonumber
 \end{eqnarray}
 \begin{eqnarray}
 	r^2-r'^2&=&-4cx~~~
 	\mbox{i.e.,~}(q_1-q_2)^2-(q_1+q_2)^2=4cx\nonumber\\
 	\mbox{i.e.,~}x&=&-\frac{q_1q_2}{c}~~~
 	\implies\dot{x}=-\frac{1}{c}(\dot{q_1}q_2+q_1\dot{q_2})\nonumber
 \end{eqnarray}
 Similarly,
 \begin{eqnarray}
 	r^2+r'^2&=&2(x^2+c^2)+2y^2\nonumber\\
 	\mbox{i.e.,~}2(q_1^2+q_2^2)&=&2\left(\frac{q_1^2q_2^2}{c^2}+c^2\right)+2y^2\nonumber\\
 	\therefore{y}&=&\frac{1}{c}\left[(c^2-q_1^2)^{\frac{1}{2}}(q_2^2-c^2)^{\frac{1}{2}}\right]\nonumber
 \end{eqnarray}
 \begin{eqnarray}
 	\therefore\dot{y}&=&\frac{1}{c}\left[-\frac{q_1\dot{q_1}}{\sqrt{c^2-q_1^2}}\sqrt{q_2^2-c^2}+\frac{q_2\dot{q_2}}{\sqrt{q_2^2-c^2}}\sqrt{c^2-q_1^2}\right]\nonumber
 \end{eqnarray}
 \begin{eqnarray}
 	\therefore\dot{x}^2+\dot{y}^2&=&\frac{1}{c^2}(\dot{q_1}q_2+q_1\dot{q_2})^2+\frac{1}{c^2}\left[\frac{q_2^2\dot{q_2}^2(c^2-q_1^2)}{q_2^2-c^2}+\frac{1}{c^2}\frac{q_1^2\dot{q_1}^2(q_2^2-c^2)}{c^2-q_1^2}-\frac{1}{c^2}2q_1q_2\dot{q_1}\dot{q_2}\right]\nonumber\\
 	&=&\frac{1}{c^2}\left[\dot{q_1}^2q_2^2+\dot{q_2}^2{q_1}^2+\frac{q_1^2\dot{q_1}^2(q_2^2-c^2)}{(c^2-q_1^2)}+\frac{q_2^2\dot{q_2}^2(c^2-q_1^2)}{(q_2^2-c^2)}\right]\nonumber\\
 	&=&\frac{1}{c^2}\left[\dot{q_1}^2\left\{\frac{(c^2-q_1^2)q_2^2+q_1^2(q_2^2-c^2)}{(c^2-q_1^2)}\right\}+\dot{q_2}^2\left\{\frac{q_1^2(q_2^2-c^2)+q_2^2(c^2-q_1^2)}{(q_2^2-c^2)}\right\}\right]\nonumber\\
 	&=&\frac{1}{c^2}\left[\frac{\dot{q_1}^2c^2(q_1^2-q_2^2)}{(c^2-q_1^2)}+\frac{\dot{q_2}^2c^2(q_1^2-q_2^2)}{(q_2^2-c^2)}\right]\nonumber\\
 	&=&(q_1^2-q_2^2)\left[\frac{\dot{q_1}^2}{c^2-q_1^2}+\frac{\dot{q_2}^2}{q_2^2-c^2}\right]\nonumber\\
 	\therefore{2T}=\dot{x}^2+\dot{y}^2&=&(q_1^2-q_2^2)\left[\frac{\dot{q_1}^2}{c^2-q_1^2}+\frac{\dot{q_2}^2}{q_2^2-c^2}\right]\nonumber
 \end{eqnarray}
 Hence the problem is of Lioville's type and hence can solved.
 
 \vspace{.5cm}
 
 $\bullet$ \textbf{{Problem: 4} }
 
 \vspace{.25cm}
 
 The K.E. of a particle whose rectangular co-ordinates $(x,y)$ is $\frac{1}{2}(\dot{x}^2+\dot{y}^2)$ and its P.E. is $\frac{A}{x^2}+\frac{A'}{y^2}+\frac{B}{r}+\frac{B'}{r'}+c(x^2+y^2),$  where $A,~B,~A',~B'$ and $C$ are constants $r,~r'$ are the distances of the particle form the points whose co-ordinates are $(\pm{c},0),~c$ being a constant. Show that when the quantities $\frac{1}{2}(r\pm{r'})$ are taken as new variables the system is of Lioville's type.
 
 \vspace{.25cm}
 
 \textbf{{Solution} :}
 
 \vspace{.25cm}
 
 Let $r=q_1+q_2,~r'=q_1-q_2$ so as in previous problem 
 \begin{eqnarray}
 	{2T}&=&(q_1^2-q_2^2)\left[\frac{\dot{q_1}^2}{c^2-q_1^2}+\frac{\dot{q_2}^2}{q_2^2-c^2}\right]\nonumber\\
 	V&=&\frac{A}{x^2}+\frac{A'}{y^2}+\frac{B}{r}+\frac{B'}{r'}+c(x^2+y^2)\nonumber\\
 	&=&\frac{Ac^2}{q_1^2q_2^2}+\frac{A'c^2}{(c^2-q_1^2)(q_2^2-c^2)}+\frac{(B+B')q_1+(B'-B)q_2}{q_1^2-q_2^2}+C(q_1^2+q_2^2-c^2)\nonumber\\
 	&=&\frac{Ac^2(q_1^2-q_2^2)}{q_1^2q_2^2(q_1^2-q_2^2)}+\frac{A'c^2(q_1^2-q_2^2)}{(c^2-q_1^2)(q_2^2-c^2)(q_1^2-q_2^2)}\nonumber\\
 	&~&~~~~~~~~~~~+\frac{(B+B')q_1+(B'-B)q_2}{(q_1^2-q_2^2)}+\frac{c(q_1^2-q_2^2)(q_1^2+q_2^2-c^2)}{(q_1^2-q_2^2)}\nonumber\\
 	&=&\frac{Ac^2\left(\frac{1}{q_2^2}-\frac{1}{q_1^2}\right)}{(q_1^2-q_2^2)}+\frac{A'c^2\left(-\frac{1}{c^2-q_1^2}-\frac{1}{q_2^2-c^2}\right)}{(q_1^2-q_2^2)}+\frac{(B+B')q_1+(B'-B)q_2}{q_1^2-q_2^2}\nonumber\\
 	&~&~~~~~~~~~~~~~~~~~~~~~~~~~~~~~~~~~~~~~~~~~~~~~~~~~~~~~~~~+\frac{c(q_1^4-c^2q_1^2)}{(q_1^2-q_2^2)}+\frac{c(q_2^4-c^2q_2^2)}{(q_1^2-q_2^2)}\nonumber
 \end{eqnarray}
 i.e., $V$ is of the form $\frac{w_1(q_1)+w_2(q_2)}{v_1(q_1)+v_2(q_2)}$ where 
 \begin{equation}
 	w_1(q_1)=-\frac{Ac^2}{q_1^2}+\frac{A'c^2}{q_1^2-c^2}+(B+B')q_1+c(q_1^4-c^2q_1^2)\nonumber
 \end{equation}
 and
 \begin{equation}
 	w_2(q_2)=\frac{Ac^2}{q_2^2}+\frac{A'c^2}{c^2-q_2^2}+(B-B')q_2+c(q_2^4-c^2q_2^2)\nonumber
 \end{equation}
 Hence the problem is Lioville's type.
 
 \vspace{.5cm}
 
\section{Hamiltonian of a Mechanical system: } 
 
 \vspace{.25cm}
 
 
 Suppose a dynamical system with $n$ degrees of freedom is described by the Lagrangian $L(q,\dot{q},t)$. The generalised momentum $p$ associated with the generalised coordinate $q$ is given by 
 \begin{equation}
 	p=\frac{\partial{L}}{\partial\dot{q}}\label{39}
 \end{equation}
 We introduce a function $H$ such that
 \begin{equation}
 	H=\sum_{i=1}^{n}p_i\dot{q}_i-L\label{40}
 \end{equation}
 As $L=T-V$, contains terms of 2nd degree in $\dot{q}$'s, so $p$'s are linear function of  $\dot{q}$'s. Therefore, solving the equation (\ref{39}) for $\dot{q}$'s we can express them as function of ${q}$'s, $p$'s and $t$. Thus $H$ is a function of $q$'s, $p$'s and $t$ i.e., $H=H(q,p,t)$. The function so constructed is called the Hamiltonian of a dynamical system.
 
 \vspace{.5cm}
 
 \section{Hamilton's canonical equations of motion:}
 
 \vspace{.25cm}
 
 Consider a dynamical system described at any instant by the Hamiltonian $H$ so that 
 \begin{equation}
 	H=\sum_{i=1}^{n}p_i\dot{q}_i-L\nonumber
 \end{equation} 
 where $p_i=\frac{\partial{L}}{\partial\dot{q}_i}$.
 
 Let us consider a virtual variation of $H$ at time instant $t$ i.e., we have 
 \begin{eqnarray}
 &~&\delta{H}=\delta\left(\sum_{i=1}^{n}p_i\dot{q}_i\right)-\delta{L}\nonumber\\
 	&~&\mbox{i.e.,}\sum_{i=1}^{n}\frac{\partial{H}}{\partial\dot{q}_i}\delta{q_i}+\sum_{i=1}^{n}\frac{\partial{H}}{\partial{p}_i}\delta{p_i}+\frac{\partial{H}}{\partial{t}}\delta{t}=\cancel{\sum_{i=1}^{n}p_i\delta\dot{q_i}}+\sum_{i=1}^{n}\dot{q}_i\delta{p_i}-\sum_{i=1}^{n}\frac{\partial{L}}{\partial{q}_i}\delta{q_i}-\cancel{\sum_{i=1}^{n}\frac{\partial{L}}{\partial\dot{q}_i}\delta{\dot{q}_i}}-\frac{\partial{L}}{\partial{t}}\delta{t}\nonumber
 \end{eqnarray}
 As $p_i=\frac{\partial{L}}{\partial\dot{q}_i}$ and from Lagrange's equation of motion i.e., $\frac{d}{dt}\left(\frac{\partial{L}}{\partial{\dot{q}_i}}\right)=\frac{\partial{L}}{\partial{q}_i}$ we have $\dot{p}_i=\frac{\partial{L}}{\partial{q}_i}$ so the equation simplifies to 
 \begin{eqnarray}
 	\sum_{i=1}^{n}\frac{\partial{H}}{\partial{q}_i}\delta{q_i}+\sum_{i=1}^{n}\frac{\partial{H}}{\partial{p}_i}\delta{p_i}+\frac{\partial{H}}{\partial{t}}\delta{t}=\sum_{i=1}^{n}\dot{q}_i\delta{p_i}-\sum_{i=1}^{n}\dot{p}_i\delta{q_i}-\frac{\partial{L}}{\partial{t}}\delta{t}\nonumber\\
 	\mbox{i.e.,}\sum_{i=1}^{n}\left(\frac{\partial{H}}{\partial{q_i}}+\dot{p}_i\right)\delta{q}_i+\sum_{i=1}^{n}\left(\frac{\partial{H}}{\partial{p_i}}-\dot{q}_i\right)\delta{p}_i+\left(\frac{\partial{H}}{\partial{t}}+\frac{\partial{L}}{\partial{t}}\right)\delta{t}=0\nonumber
 \end{eqnarray}
 As $q$ and $p$'s are independent so we have 
 \begin{equation}
 	\frac{\partial{H}}{\partial{q}_i}=-\dot{p}_i,~\frac{\partial{H}}{\partial{p}_i}=\dot{q}_i,~~i=1,2,...,n.\nonumber
 \end{equation}
 and $\dfrac{\partial{H}}{\partial{t}}=-\dfrac{\partial{L}}{\partial{t}}$.
 
 The above set of $2n$ equations of 1st order in $2n$ unknown $p_i$'s and $q_i$'s of a system having $n$ degrees of freedom are called the Hamilton's canonical equations of motion.
 
 \vspace{.5cm}
 
 \section{Physical significance of $H$ in case of a Sceleronomic system:}
 
 \vspace{.25cm}
 
 As in a Sceleronomic system $T$ is a homogeneous quadratic in $\dot{q}$'s and $V$ is independent of $\dot{q}$'s so $L=T-V$ is also quadratic in $\dot{q}$'s.
 
 Thus
 \begin{eqnarray}
 	H&=&\sum{p}_i\dot{q}_i-L=\sum\dot{q}_i\frac{\partial{L}}{\partial{\dot{q}_i}}-L=\sum\dot{q}_i\frac{\partial{T}}{\partial{\dot{q}_i}}-L=2T-(T-V)=T+V\nonumber\\
 	&=&\mbox{Total energy of the system.}\nonumber
 \end{eqnarray}
 
 \textbf{\underline{Note-I} :}
 
 As $H=\sum{p}_i\dot{q}_i-L$, so if a coordinate $q_k$ is absent in Lagrangian it also absent in $H$ i.e., cyclic coordinates do not occur in $H$.
 
 \textbf{\underline{Note-II} :}
 
 In a Sceleronomic system
 \begin{eqnarray}
 	\frac{dH}{dt}&=&\sum\left(\frac{\partial{H}}{\partial{q}_i}.\frac{dq_i}{dt}+\frac{\partial{H}}{\partial{p}_i}.\frac{dp_i}{dt}\right)=\sum_{i}\left(\dot{q}_i\frac{\partial{H}}{\partial{q}_i}+\dot{p}_i\frac{\partial{H}}{\partial{p}_i}\right)\nonumber\\
 	&=&\sum_{i}\left(\frac{\partial{H}}{\partial{p}_i}.\frac{\partial{H}}{\partial{q}_i}-\frac{\partial{H}}{\partial{q}_i}.\frac{\partial{H}}{\partial{p}_i}\right)=0\nonumber
 \end{eqnarray}
 So $H$ is constant.
 
 As in a Sceleronomic system $H=T+V$, hence K.E + P.E. is constant for such system. However, if $H$ has explicit time dependence then 
 \begin{equation}
 	\frac{dH}{dt}=\frac{\partial{H}}{\partial{t}}.\nonumber
 \end{equation} 
 
 
 
 \section{Problems}
 
 
 $\bullet$ \textbf{Problem:} If the Hamiltonian of a dynamical system is given by $H=p_{1}q_{1}-p_{2}q_{2}-aq_{1}^{2}+bq_{2}^{2}$ where $a$, $b$ are constants, then show that $\dfrac{p_{2}-bq_{2}}{q_{1}}$ is constant and also find the position of the system.
 
 \vspace{.5cm}
 
 $\bullet$ \textbf{Problem:} If $H=qp^{2}-qp+bp$, be the Hamiltonian of a system then find $q$ and $p$ as function of $t$.
 
 \vspace{.5cm}
 
 $\bullet$ \textbf{Problem:} If all the coordinates of a system are cyclic then prove that the coordinate may be found by integration.
 
 \vspace{.25cm}
 
 \textbf{Solution:} If all the coordinates are ignorable then $H=H(p)$ so $\dot{p_{i}} =-\dfrac{\partial H}{\partial q_{i}}=0$ i.e $p_{i}$'s are constant, and $\dot{q_{i}}=\dfrac{\partial H}{\partial p_{i}}=f_{i}(t)$ ($\because p_{i}$'s are constants). Therefore $q_{i}=\int f_{i}(t)dt + \alpha_{i}$, $i=1,2,...,n$. Further, if the system is scelronomic, then $H=H(p_{i})$ and $\dot{q_{i}}=\dfrac{\partial H}{\partial p_{i}}=K_{i}$ (say) i.e $q_{i}=K_{i}t+\alpha_{i}, i=1,2,...,n$. So generalized coordinates are linear functions of $t$.
 
 \vspace{.5cm}
 
 $\bullet$ \textbf{Problem:} If the K.E. $T(q,\dot{q})$ of a system is a homogeneous quadratic in the velocities $\dot{q}$'s and $T'(q,p)$ is what $T$ becomes when expressed in terms of variables $q$ and $p$, then prove that
 
 (i) $\dot{q_{i}}=\dfrac{\partial T'}{\partial p_{i}}$, (ii) $\dfrac{\partial T}{\partial q_{i}}+\dfrac{\partial T'}{\partial q_{i}}=0$
 
 
 (iii) $T'$ is a homogeneous quadratic in p
 
 
 (iv) $T+T'=\sum p_{i}\dot{q_{i}}$
 
 \vspace{.25cm}
 
 \textbf{Solution:} (i) $\dot{q_{i}}=\dfrac{\partial H}{\partial p_{i}}=\dfrac{\partial (T'+V)}{\partial p_{i}}=\dfrac{\partial T'}{\partial p_{i}}$ ($\because V$ is a function of $q$'s only)
 
 (ii) $\dfrac{\partial H}{\partial q_{i}}=-\dot{p_{i}}=-\dfrac{\partial L}{\partial q_{i}}$. This implies $\dfrac{\partial (T'+V)}{\partial q_{i}}=-\dfrac{\partial(T-V)}{\partial q_{i}}$ i.e $\dfrac{\partial T'}{\partial q_{i}}+\dfrac{\partial T}{\partial q_{i}}=0$.
 
 (iii) $T$ is a homogeneous quadratic function of $\dot{q_{i}}$'s i.e $2T=\sum\dot{q_{i}}\dfrac{\partial T}{\partial \dot{q_{i}}}=\sum \dot{q_{i}}\dfrac{\partial L}{\partial \dot{q_{i}}}=\sum_{i} p_{i}\dot{q_{i}}$.
 Therefore $2T'=\sum p_{i}\dfrac{\partial T'}{\partial p_{i}}$ (using the result of (i)) which shows that $T'$ is a homogeneous quadratic in $p_{i}$'s.
 
 (iv) $H=\sum p_{i}\dot{q_{i}}-L$, i.e $T'+V=\sum p_{i}\dot{q_{i}}-(T-V)$, i.e $T+T'=\sum p_{i}\dot{q_{i}}$.
 
 \vspace{.5cm}
 
 $\bullet$ \textbf{Problem:} If the K.E of a scleronomic system is $T(q,\dot{q_{i}})=\dfrac{1}{2}\sum a_{ij}\dot{q_{i}}\dot{q_{j}}$ and if $T'(q,p)$ is what $T$ becomes when expressed in terms of $q$ and $p$ then show that $2T'+\dfrac{D}{\Delta}=0$ where $\Delta=det(a_{ij})$, 
 \[
 D =
 \begin{vmatrix}
 	a_{11} & a_{12}  & \dots & a_{1n} & p_{1} \\ 
 	a_{21} & a_{22}  & \dots & a_{2n} & p_{2} \\
 	\hdotsfor{5} \\
 	a_{n1} &  a_{n2} & \dots & a_{nn} & p_{n}\\
 	p_{1} & p_{2} &  \dots & p_{n} & 0
 \end{vmatrix}
 \]
 
 \textbf{Proof:} By definition
 $p_{i}=\dfrac{\partial T}{\partial \dot{q_{i}}}=\sum_{i} a_{ij}\dot{q_{j}}$, i.e $\sum_{j=1}^{n} a_{ij}\dot{q_{j}}-p_{i}=0$ or in explicit form
 \begin{eqnarray}
 	a_{11}\dot{q_{1}}+a_{12}\dot{q_{2}}+...+a_{1n}\dot{q_{n}}-p_{1}=0\nonumber\\ 
 	a_{21}\dot{q_{1}}+a_{22}\dot{q_{2}}+...+a_{2n}\dot{q_{n}}-p_{2}=0\nonumber\\ 
 	......	...................................................\nonumber\\
 	a_{n1}\dot{q_{1}}+a_{n2}\dot{q_{2}}+...+a_{nn} \dot{q_{n}}-p_{n}=0\nonumber
 \end{eqnarray}
 Also, $2T'=\sum_{i}p_{i}\dot{q_{i}}$ i.e
 \begin{equation}
 	p_{1}\dot{q_{1}}+p_{2}\dot{q_{2}}+...+p_{n}\dot{q_{n}}-2T'=0.\nonumber\end{equation} Now eliminating $\dot{q_{1}}. $$\dot{q_{2}}$,...,$\dot{q_{n}}$ we have
 \[
 \begin{vmatrix}
 	a_{11} & a_{12}  & \dots & a_{1n} & p_{1} \\ 
 	a_{21} & a_{22}  & \dots & a_{2n} & p_{2} \\
 	\hdotsfor{5} \\
 	a_{n1} &  a_{n2} & \dots & a_{nn} & p_{n}\\
 	p_{1} & p_{2} &  \dots & p_{n} & 2T'
 \end{vmatrix}=0
 \] i.e
 \[
 \begin{vmatrix}
 	a_{11} & a_{12}  & \dots & a_{1n} & p_{1}+0 \\ 
 	a_{21} & a_{22}  & \dots & a_{2n} & p_{2}+0 \\
 	\hdotsfor{5} \\
 	a_{n1} &  a_{n2} & \dots & a_{nn} & p_{n}+0\\
 	p_{1} & p_{2} &  \dots & p_{n} & 0+2T'
 \end{vmatrix}=0\] i.e
 \[D+
 \begin{vmatrix}
 	a_{11} & a_{12}  & \dots & a_{1n} & 0 \\ 
 	a_{21} & a_{22}  & \dots & a_{2n} & 0 \\
 	\hdotsfor{5} \\
 	a_{n1} &  a_{n2} & \dots & a_{nn} & 0\\
 	p_{1} & p_{2} &  \dots & p_{n} & 2T'
 \end{vmatrix}=0\] i.e
 $~~~~~~~~~~~~~~~~~~~~~~~~~~~~~~~~~~~~~~~~~D+2T' \Delta=0$ or, $2T'+\dfrac{D}{\Delta}=0$.
 
 \vspace{.25cm}
 
 $\bullet$ \textbf{Problem:} Solution of Planetary motion using Hamilton's canonical equations:\\
 The K.E of a planet in polar coordinates is 
 \begin{eqnarray}
 	T=\dfrac{m}{2}(\dot{r}^{2}+r^{2}\dot{\theta}^{2})-\dfrac{\mu}{r},~~ V=-\dfrac{\mu}{r}.\nonumber
 \end{eqnarray}
 $~~~~~~~~~~~~~~~~~~~~~~ \therefore H=T+V=\dfrac{m}{2}(\dot{r}^{2}+r^{2}\dot{\theta}^{2})-\dfrac{\mu}{r}$.\\
 Let $p_{r}$, $p_{\theta}$ are the generalized momenta corresponding to $r$ and $\theta$ variables then 
 
 $p_{r}=\dfrac{\partial T}{\partial \dot{r}}=m\dot{r}$,~~ $p_{\theta}=\dfrac{\partial T}{\partial \dot{\theta}}=mr^{2}\dot{\theta}$.
 
 $\therefore \dot{r}= \dfrac{p_{r}}{m},~~ \dot{\theta}=\dfrac{p_{\theta}}{mr^{2}}$.
 
 $H=\dfrac{m}{2}\left(\dfrac{p_{r}^{2}}{m^{2}}+r^{2} \dfrac{p_{\theta}^{2}}{m^{2}r^{4}}\right)-\dfrac{\mu}{r}=\dfrac{1}{2m}\left(p_{r}^{2}+\dfrac{p_{\theta}^{2}}{r^{2}}\right)-\dfrac{\mu}{r}$. 
 
 The Hamilton's canonical equations are
 \begin{eqnarray}
 	\dot{r}=\dfrac{\partial H}{\partial p_{r}}=\dfrac{p_{r}}{m},\nonumber\\
 	\dot{\theta}=\dfrac{\partial H}{\partial p_{\theta}}=\dfrac{p_{\theta}}{mr^{2}}.\nonumber
 \end{eqnarray} 
 
 Also, $\dfrac{\partial H}{\partial r}=\dfrac{-p_{\theta}^{2}}{mr^{3}}+\dfrac{\mu}{r^{2}}=-\dot{p_{r}}$
 
 and $\dfrac{\partial H}{\partial \theta}=0$ i.e $\dot{p_{\theta}}=0$ or $p_{\theta}=\mbox{Constant}=h~(\mbox{say}),$. 
 
 (note that $\theta$ is a cyclic coordinate).
 
 Thus, $r^{2}\dot{\theta}=\dfrac{h}{m}=c$(say) i.e, $\dot{\theta}=\dfrac{c}{r^{2}}$
 
 and $\dot{p_{r}}=\dfrac{h^{2}}{mr^{3}}-\dfrac{\mu}{r^{2}}$. 
 
 i.e, 
 \begin{equation}
 	\dfrac{d^{2}r}{dt^{2}}=\dfrac{h^{2}}{m^{2}r^{3}}-\dfrac{\mu}{mr^{2}}=\dfrac{c^{2}}{r^{3}}-\dfrac{l}{r^{2}}\nonumber
 \end{equation}
 i.e, 
 \begin{equation}
 	m(\ddot{r}-r\dot{\theta}^{2})=-\dfrac{\mu}{r^{2}},\nonumber
 \end{equation} 
 
 which is the radial equation of motion.
 
 
 
 
 
 
 
 
 
 
 
 
 
 
 
 
 
