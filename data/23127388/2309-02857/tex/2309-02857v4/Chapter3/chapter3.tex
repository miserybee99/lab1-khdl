
\chapter{Motion in 3-dimensions}

\section{Components of velocity and acceleration in spherical polar coordinates}

\vspace{0.6cm}

\begin{wrapfigure}[10]{r}{0.4\textwidth}\vspace{-3.5\intextsep}
\centering	\includegraphics[height=5.5 cm , width=5.5 cm ]{f5.pdf}
	\begin{center}\vspace{-\intextsep}
		Fig. 3.1
	\end{center}
\end{wrapfigure}


~~Let $P(r,\theta,\phi)$ be the position of a particle. In the figure $APK$ be the meridian plane and $PT$ be the normal to the meridian plane and PS be in the meridian plane but perpendicular to OP. From the figure $r=OP$, $\angle AOP=\theta$, and $\angle XON=\phi$.\\

Now,
\begin{eqnarray}
	\mbox{velocity of }P&=&\mbox{velocity of }P\mbox{ in the meridian plane}\nonumber\\&&+\mbox{ velocity of }P\perp\mbox{to the meridian plane}\nonumber\\
	&=&\dot{r}\mbox{ along }OP+r\dot{\theta}\mbox{ along }PS+\mbox{velocity of }N\perp\mbox{to the meridian plane}\nonumber\\&=&\dot{r}\mbox{ along }OP+r\dot{\theta}\mbox{ along }PS+r\sin\theta\dot{\phi}\mbox{ along }PT\nonumber	
\end{eqnarray}

Thus $v_r=\dot{r}$, $v_\theta=r\dot{\theta}$ and $v_\phi=r\sin\theta\dot{\phi}$ are the components of the velocity along $OP$, $PS$ and $PT$ respectively.
\begin{eqnarray}
	\mbox{The acceleration of }P&=&\mbox{acceleration of }P\parallel\mbox{ to the }XOY\mbox{plane}\nonumber\\&&+\mbox{ acceleration of }P\perp\mbox{ to the }XOY\mbox{plane}\nonumber\\&=&\mbox{acceleration of }N\mbox{ in }XOY\mbox{plane}+\frac{\mathrm{d}^2}{\mathrm{d}t^2}(NP)\mbox{ along }NP\nonumber\\&=&\left[\frac{\mathrm{d}^2}{\mathrm{d}t^2}(ON)-(ON)\dot{\phi}^2\right]\mbox{ along }ON\nonumber\\&&+\left[\frac{1}{ON}\frac{\mathrm{d}}{\mathrm{d}t}(ON^2\cdot\dot{\phi})\right]\perp\mbox{ to the meridian plane}+\frac{\mathrm{d}^2}{\mathrm{d}t^2}(NP)\mbox{ along }NP\nonumber
\end{eqnarray}

AS $ON=r\sin\theta$, $NP=r\cos\theta$, so,
\begin{eqnarray}
	f_r&=&\sin\theta\left[\frac{\mathrm{d}^2}{\mathrm{d}t^2}(ON)-(ON)\dot{\phi}^2\right]+\cos\theta\frac{\mathrm{d}^2}{\mathrm{d}t^2}(NP)\nonumber\\&=&\sin\theta\left[\frac{\mathrm{d}}{\mathrm{d}t}(\dot{r}\sin\theta+r\cos\theta\dot{\theta})-r\sin\theta\dot{\phi}^2\right]+\cos\theta\frac{\mathrm{d}}{\mathrm{d}t}(\dot{r}\cos\theta-r\sin\theta\dot{\theta})\nonumber\\&=&\ddot{r}-r\dot{\theta}^2-r\sin^2\theta\dot{\phi}^2\nonumber
\end{eqnarray}
\begin{eqnarray}
	f_\theta&=&\cos\theta\left[\frac{\mathrm{d}^2}{\mathrm{d}t^2}(ON)-(ON)\dot{\phi}^2\right]-\sin\theta\frac{\mathrm{d}^2}{\mathrm{d}t^2}(NP)\nonumber\\&=&\cos\theta\left[\frac{\mathrm{d}}{\mathrm{d}t}(\dot{r}\sin\theta+r\cos\theta\dot{\theta})-r\sin\theta\dot{\phi}^2\right]-\sin\theta\frac{\mathrm{d}}{\mathrm{d}t}(\dot{r}\cos\theta-r\sin\theta\dot{\theta})\nonumber\\&=&2\dot{r}\dot{\theta}+r\ddot{\theta}-r\sin\theta\cos\theta\dot{\phi}^2\nonumber
\end{eqnarray}
\begin{eqnarray}
	f_\phi&=&\frac{1}{ON}\frac{\mathrm{d}}{\mathrm{d}t}(ON^2\cdot\dot{\phi})\nonumber\\&=&r\sin\theta\ddot{\phi}+2\dot{r}\sin\theta\dot{\phi}+2r\cos\theta\dot{\theta}\dot{\phi}\nonumber
\end{eqnarray}
\subsection{Vector Method}

\begin{wrapfigure}[20]{r}{0.4\textwidth}
	\centering	\includegraphics[height=5.5 cm , width=5.5 cm ]{f6.pdf}
	\begin{center}
		Fig. 3.2
	\end{center}
\end{wrapfigure}

Let $\hat{r}$ be the unit vector along the radial direction $OP$, $\hat{\theta}$ be the unit vector $\perp$ to $OP$ but in the meridian plane and $\hat{\phi}$ be the unit vector perpendicular to the meridian plane as shown in the figure.
$$\vec{r}=r~\hat{r}$$
$$\therefore~\vec{v}=\frac{\mathrm{d}\vec{r}}{\mathrm{d}t}=\dot{r}\hat{r}+r\frac{\mathrm{d}\hat{r}}{\mathrm{d}t}$$.

Here $\dfrac{\mathrm{d}\hat{r}}{\mathrm{d}t}$ is orthogonal to $\hat{r}$, so it will be in the plane of $\hat{\theta}$ and $\hat{\phi}$. Hence we have
$$\frac{\mathrm{d}\hat{r}}{\mathrm{d}t}=\dot{\theta}\hat{\theta}+\sin\theta\dot{\phi}\hat{\phi}$$
$$\therefore~\vec{v}=\frac{\mathrm{d}\vec{r}}{\mathrm{d}t}=\dot{r}\hat{r}+r\dot{\theta}\hat{\theta}+r\sin\theta\dot{\phi}\hat{\phi}$$
\begin{equation}
	\vec{f}=\frac{\mathrm{d}\vec{v}}{\mathrm{d}t}=\ddot{r}\hat{r}+\dot{r}\frac{\mathrm{d}\hat{r}}{\mathrm{d}t}+\left(\dot{r}\dot{\theta}+r\ddot{\theta}\right)\hat{\theta}+r\dot{\theta}\frac{\mathrm{d}\hat{\theta}}{\mathrm{d}t}+\left(\dot{r}\sin\theta\dot{\phi}+r\cos\theta\dot{\theta}\dot{\phi}+r\sin\theta\ddot{\phi}\right)\hat{\phi}+r\sin\theta\dot{\phi}\frac{\mathrm{d}\hat{\phi}}{\mathrm{d}t}\nonumber
\end{equation}

As before, $\dfrac{\mathrm{d}\hat{\theta}}{\mathrm{d}t}$ and $\dfrac{\mathrm{d}\hat{\phi}}{\mathrm{d}t}$ are respectively perpendicular to $\hat{\theta}$ and $\hat{\phi}$. So they are respectively the linear combination of ($\hat{r}$, $\hat{\phi}$) and ($\hat{r}$, $\hat{\theta}$) as follows:
$$\frac{\mathrm{d}\hat{\theta}}{\mathrm{d}t}=-\dot{\theta}\hat{r}+\cos\theta\dot{\phi}\hat{\phi}$$
$$\frac{\mathrm{d}\hat{\phi}}{\mathrm{d}t}=-\sin\theta\dot{\phi}\hat{r}-\cos\theta\dot{\phi}\hat{\theta}$$
\begin{equation}
	\therefore~\vec{f}=\left[\ddot{r}-r\dot{\theta}^2-r\sin^2\theta\dot{\phi}^2\right]\hat{r}+\left[\frac{1}{r}\frac{\mathrm{d}}{\mathrm{d}t}\left(r^2\dot{\theta}\right)-r\sin\theta\cos\theta\dot{\phi}^2\right]\hat{\theta}+\frac{1}{r\sin\theta}\frac{\mathrm{d}}{\mathrm{d}t}\left(r^2\sin^2\theta\dot{\phi}\right)\hat{\phi}\nonumber
\end{equation}

\begin{wrapfigure}[20]{r}{0.32\textwidth}
	\centering	\includegraphics[height=5.5 cm , width=4.5 cm ]{f7.pdf}
	\begin{center}
		Fig. 3.3
	\end{center}
\end{wrapfigure}

$\bullet$ \textbf{Particular cases:}\\

\textbf{I. Motion of a particle on the surface of a cone}\\

In the figure $PG$ is orthogonal to $OP$. Let $OP=r$, $\theta=\alpha$, a constant

$$\dot{\theta}=0$$

$$v_r=\dot{r},~v_\theta=0,~v_\phi=r\sin\alpha\dot{\phi}$$

$$f_r=\ddot{r}-r\sin^2\alpha\dot{\phi}^2,~f_\theta=-r\sin\theta\cos\theta\dot{\phi}^2,~f_\phi=r\sin\alpha\ddot{\phi}+2\dot{r}\dot{\phi}\sin\alpha$$

\textbf{II. Motion of a particle on  the surface of a sphere}\\

Here $r=a$, a constant.

$$v_r=a,~v_\theta=a\dot{\theta},~v_\phi=a\sin\theta\dot{\phi}$$

$$f_r=-a\dot{\theta}^2-a\sin^2\theta\dot{\phi}^2~f_\theta=a\ddot{\theta}-a\sin\theta\cos\theta\dot{\phi}^2,~f_\phi=a\sin\theta\ddot{\phi}+2a\cos\theta\dot{\theta}\dot{\phi}$$

\section{Components of velocity and acceleration in cylindrical polar co-ordinates}

\begin{wrapfigure}[10]{r}{0.32\textwidth}
	\centering	\includegraphics[height=5 cm , width=4.5 cm ]{f8.pdf}
	\begin{center}
		Fig. 3.4
	\end{center}
\end{wrapfigure}

Let $ON=r$, $\angle XON=\theta$.
\begin{eqnarray}
		\mbox{Velocity of }P&=&\mbox{velocity of }P~\parallel\mbox{to } XOY \mbox{ plane}\nonumber\\&&+\mbox{velocity of }P\perp\mbox{to the } XOY \mbox{ plane}\nonumber\\&=&\mbox{velocity of }N\mbox{ in the } XOY \mbox{ plane}\nonumber\\&&+\mbox{velocity of }P~\parallel\mbox{to } Z \mbox{ axis}\nonumber\\&=&\dot{r}\mbox{ along }ON+r\dot{\theta}\perp \mbox{ to }ON+\dot{z}\parallel \mbox{ to }Z\mbox{ axis}\nonumber
\end{eqnarray}
$$\therefore~~v_r=\dot{r},~v_\theta=r\dot{\theta}, v_z=\dot{z}$$

Similarly $f_r=\ddot{r}-r\dot{\theta}^2$, $f_\theta=\dfrac{1}{r}\dfrac{\mathrm{d}}{\mathrm{d}t}\left(r^2\dot{\theta}\right)$, $f_z=\ddot{z}$.



\textbf{Vector method}\\

$\vec{R}=\overrightarrow{OP}=\overrightarrow{ON}+\overrightarrow{NP}=r\hat{r}+z\hat{z}$
\begin{eqnarray}
\therefore~~\vec{v}&=&\frac{\mathrm{d}\vec{R}}{\mathrm{d}t}=\dot{r}\hat{r}+r\frac{\mathrm{d}\hat{r}}{\mathrm{d}t}+\dot{z}\hat{z}~~(\because~\hat{z}\mbox{ is a fixed direction})\nonumber\\&=&\dot{r}\hat{r}+r\dot{\theta}\hat{\theta}+\dot{z}\hat{z}\nonumber
	\end{eqnarray}
\begin{eqnarray}
	\vec{f}=\frac{\mathrm{d^2}\vec{R}}{\mathrm{d}t^2}&=&\ddot{r}\hat{r}+\dot{r}\frac{\mathrm{d}\hat{r}}{\mathrm{d}t}+\left(\dot{r}\dot{\theta}+r\ddot{\theta}\right)\hat{\theta}+r\dot{\theta}\frac{\mathrm{d}\hat{\theta}}{\mathrm{d}t}+\ddot{z}\hat{z}\nonumber\\&=&\ddot{r}\hat{r}+\dot{r}\dot{\theta}\hat{\theta}+\left(\dot{r}\dot{\theta}+r\ddot{\theta}\right)\hat{\theta}-r\dot{\theta}^2\hat{r}+\ddot{z}\hat{z}\nonumber\\&=&\left(\ddot{r}-r\dot{\theta}^2\right)\hat{r}+\left(2\dot{r}\dot{\theta}+r\ddot{\theta}\right)\hat{\theta}+\ddot{z}\hat{z}\nonumber
\end{eqnarray}

\section{Components of velocity and acceleration along tangent, principal normal and binormal of a particle moving along a space curve}

Let $\vec{r}=\vec{r}(s)$ be the space curve where $s$, the arc length is chosen as parameter
$$\vec{v}=\frac{\mathrm{d}\vec{r}}{\mathrm{d}t}=\frac{\mathrm{d}\vec{r}}{\mathrm{d}s}\dot{s}=\dot{s}\hat{t}$$
where $\hat{t}$ is the unit tangent vector to the space curve. The above result shows that the velocity of the particle is always along the tangent to the curve.
\begin{eqnarray}
	\vec{f}=\frac{\mathrm{d}\vec{v}}{\mathrm{d}t}&=&\ddot{s}\hat{t}+\dot{s}\frac{\mathrm{d}\hat{t}}{\mathrm{d}t}\nonumber\\&=&\ddot{s}\hat{t}+\dot{s}^2\frac{\mathrm{d}\hat{t}}{\mathrm{d}s}\nonumber\\&=&\ddot{s}\hat{t}+\dot{s}^2\kappa\vec{n}\mbox{~~~ (by Serret-Frenet formula)}\nonumber\\&=&\ddot{s}\hat{t}+\frac{\dot{s}^2}{\rho}\vec{n}\nonumber
\end{eqnarray}

Hence the acceleration vector has no component along the binormal direction. Similar to a plane curve the acceleration vector is always in the osculating plane of the curve.\\

\section{Angular velocity vector}

Let arc$PP'=\Delta s$, $\overrightarrow{OP}=\vec{r}$ and $\vec{\epsilon}$ is the unit vector along the axis of rotation. Suppose in small time $\Delta t$, $P$ goes to $P'$ such that $PP'=\Delta s$. Then the angular velocity
$$\Omega=\lim\limits_{\Delta t\rightarrow0}\frac{\Delta\theta}{\Delta t}=\frac{\mathrm{d}\theta}{\mathrm{d}t}=\dot{\theta}$$

Let $\vec{\omega}=\vec{\epsilon}\Omega$, is termed as the angular velocity vector at P about the axis of rotation, extending along the direction of the advanced right handed screw.\\

\begin{wrapfigure}[10]{r}{0.22\textwidth}
	\centering	\includegraphics[height=5 cm , width=3 cm ]{f9.pdf}
	\begin{center}
		Fig. 3.5
	\end{center}
\end{wrapfigure}

Let $\vec{u}$ be the linear velocity at P due to rotation, then
\begin{minipage}{.3\textheight}
\begin{eqnarray}
	\vec{u}&=& NP~\dot{\theta}\frac{\vec{\epsilon}\times\vec{r}}{|\vec{\epsilon}\times\vec{r}|}\nonumber\\&=&\dot{\theta}\left(\vec{\epsilon}\times\vec{r}\right)\nonumber\\&=&\Omega\left(\vec{\epsilon}\times\vec{r}\right)\nonumber\\&=&\vec{\omega}\times\vec{r}\nonumber
\end{eqnarray}
\end{minipage}
\begin{minipage}{.2\textheight}
	\begin{eqnarray}
		|\vec{\epsilon}\times\vec{r}|&=&|\vec{\epsilon}||\vec{r}|\sin\phi\nonumber\\&=&r\sin\phi~=~NP\nonumber
	\end{eqnarray}
\end{minipage}

\bigskip

\section{Moving axes in three dimensions}

Let $P$ be a point whose co-ordinates are $(x,y,z)$ referred to $OX$, $OY$, $OZ$ as axes. Consider $\vec{i}$, $\vec{j}$ and $\vec{k}$ be the unit vectors along the $x$, $y$ and $z$ axes 
$$\overrightarrow{OP}=\vec{r}=\xi\vec{i}+\eta\vec{j}+\zeta\vec{k}$$
\begin{wrapfigure}[7]{r}{0.35\textwidth}
	\centering	\includegraphics[height=5 cm , width=5.3 cm ]{f10.pdf}
	\begin{center}
		Fig. 3.6
	\end{center}
\end{wrapfigure}
\begin{equation}
\therefore~~	\vec{v}=\frac{\mathrm{d}\vec{r}}{\mathrm{d}t}=\left(\dot{\xi}\vec{i}+\dot{\eta}\vec{j}+\dot{\zeta}\vec{k}\right)+\left(\xi\frac{\mathrm{d}\vec{i}}{\mathrm{d}t}+\eta\frac{\mathrm{d}\vec{j}}{\mathrm{d}t}+\zeta\frac{\mathrm{d}\vec{k}}{\mathrm{d}t}\right)\nonumber
\end{equation}

Note that $\dfrac{\mathrm{d}\vec{i}}{\mathrm{d}t}$ is a vector whose components are the components of the velocity of the extremity of the unit vector $\vec{i}$ drawn from $O$. Suppose $\omega_1$, $\omega_2$ and $\omega_3$ be the components of the angular velocity about $OX$, $OY$ and $OZ$ respectively. Then $\dfrac{\mathrm{d}\vec{i}}{\mathrm{d}t}$ is given by
\begin{equation}
	\frac{\mathrm{d}\vec{i}}{\mathrm{d}t}=\begin{array}{|ccc|}
		\vec{i}&\vec{j}&\vec{k}\\\omega_1&\omega_2&\omega_3\\1&0&0
	\end{array}=\omega_3\vec{j}-\omega_2\vec{k}\nonumber
\end{equation}

Similarly $\dfrac{\mathrm{d}\vec{j}}{\mathrm{d}t}=\omega_1\vec{k}-\omega_3\vec{j}$ and $\dfrac{\mathrm{d}\vec{j}}{\mathrm{d}t}=\omega_2\vec{i}-\omega_1\vec{j}$.\\

$\therefore~~v_x=\dot{x}-\omega_3y+\omega_2z$, $v_y=\dot{y}-\omega_1z+\omega_3x$, $v_z=\dot{z}-\omega_2x+\omega_1y$.\\

(Note that we replace $\left(\xi,\eta,\zeta\right)$ by $\left(x,y,z\right)$).\\

$f_x=\dot{v}_x-\omega_3v_y+\omega_2v_z$, $f_y=\dot{v}_y-\omega_1v_z+\omega_3v_x$, $f_z=\dot{v}_z-\omega_2v_x+\omega_1v_y$.\\
\newpage
\section{Applications}
\subsection{Spherical Polar Co-ordinates}

\begin{wrapfigure}[13]{r}{0.35\textwidth}
	\centering	\includegraphics[height=5 cm , width=5.3 cm ]{f11.pdf}
	\begin{center}
		Fig. 3.7
	\end{center}
\end{wrapfigure}

Let the origin $O$ be fixed and the axes are rotating. Let $OP=r$, where $P\left(r,\theta,\phi\right)$ be the spherical polar co-ordinates. Suppose $M$ be the position of $P$ after a rotation in the increasing direction of $\theta$ and $\phi$. So the co-ordinates of $M$ be $\left(r,\theta+\mathrm{d}\theta,\phi+\mathrm{d}\phi\right)$. The angular velocity of the system is equivalent to $\dot{\phi}$ about $OZ$ together with $\dot{\theta}$ about an axis in the direction of $\phi$.. If $\left(\omega_1,\omega_2,\omega_3\right)$ be the component of angular velocity about $r$, $\theta$, $\phi$ direction, then we have
$$\omega_r=\dot{\phi}\cos\theta,~\omega_\theta=\dot{\phi}\sin\theta,~\omega_\phi=\dot{\theta}$$

So referred to $\left(r,\theta,\phi\right)$ as instantaneous axes of rotation, the co-ordinates of P are $(r,0,0)$.Thus if $\left(u_r,u_\theta,u_\phi\right)$ be the component of velocity along $r$, $\theta$ and $\phi$ direction, then
\begin{eqnarray}
	&&\left(u_r,u_\theta,u_\phi\right)=\left(\dot{r},0,0\right)+\begin{array}{|ccc|}
		\hat{r}&\hat{\theta}&\hat{\phi}\\\omega_r&\omega_\theta&\omega_\phi\\r&0&0
	\end{array}\nonumber\\\implies&& u_r=\dot{r},~u_\theta=r\dot{\theta},~u_\phi=r\sin\theta\dot{\phi}\nonumber
\end{eqnarray}

Similarly, for the acceleration vector,
\begin{eqnarray}
	&&\vec{f}=\left(f_r,f_\theta,f_\phi\right)=\left(\dot{u}_r,\dot{u}_\theta,\dot{u}_\phi\right)+\begin{array}{|ccc|}
		\hat{r}&\hat{\theta}&\hat{\phi}\\\omega_r&\omega_\theta&\omega_\phi\\u_r&u_\theta&u_\phi
	\end{array}\nonumber\\\implies&& f_r=\ddot{r}-r\dot{\theta}^2-r\sin^2\theta\dot{\phi}^2,~f_\theta=\frac{1}{r}\frac{\mathrm{d}}{\mathrm{d}t}\left(r^2\dot{\theta}\right)-r\sin\theta\cos\theta\dot{\phi}^2,~f_\phi=\frac{1}{r\sin\theta}\frac{\mathrm{d}}{\mathrm{d}t}\left(r^2\sin^2\theta\dot{\phi}\right)\nonumber
\end{eqnarray}


\subsection{Cylindrical co-ordinates}

\begin{wrapfigure}[10]{r}{0.35\textwidth}
	\centering	\includegraphics[height=5 cm , width=5.3 cm ]{f8.pdf}
	\begin{center}
		Fig. 3.8
	\end{center}
\end{wrapfigure}

In this case if $\hat{r}$, $\hat{\theta}$ and $\hat{z}$ be the unit vectors and $\omega_r$, $\omega_\theta$ and $\omega_z$ be the angular velocities along the axes $r$, $\theta$ and $z$, then
$$\omega_r=0=\omega_\theta,~\omega_z=\dot{\theta}$$

Now referred to $\left(r,\theta,z\right)$ as the instantaneous set of axes, co-ordinates of $P$ are $(r,0,z)$, the components of velocity are
\begin{eqnarray}
	&&\vec{v}=\left(u_r,u_\theta,u_z\right)=\left(\dot{r},0,\dot{z}\right)+\begin{array}{|ccc|}
		\hat{r}&\hat{\theta}&\hat{z}\\\omega_r&\omega_\theta&\omega_z\\r&0&z
	\end{array}\nonumber\\\implies&& u_r=\dot{r},~u_\theta=r\dot{\theta},~u_\phi=\dot{z}\nonumber
\end{eqnarray}

Similarly for acceleration vector,
\begin{eqnarray}
	&&\vec{f}=\left(f_r,f_\theta,f_z\right)=\left(\dot{u}_r,\dot{u}_\theta,\dot{u}_z\right)+\begin{array}{|ccc|}
		\hat{r}&\hat{\theta}&\hat{z}\\\omega_r&\omega_\theta&\omega_z\\u_r&u_\theta&u_z
	\end{array}\nonumber\\\implies&& f_r=\ddot{r}-r\dot{\theta}^2,~f_\theta=\frac{1}{r}\frac{\mathrm{d}}{\mathrm{d}t}\left(r^2\dot{\theta}\right),~f_z=\ddot{z}\nonumber
\end{eqnarray}

\section{Osculating  plane of a space curve}

Let $x=\phi_1(t)$, $y=\phi_2(t)$, $z=\phi_3(t)$ be the parametric equation of a space curve. We assume that $\phi_i(t),i=1,2,3$ are continuous in a certain range of $t$. Further, it is assumed that all points of the curve (within the given range) are arbitrary points so that $\phi'_i(t)$ and $\phi''_i(t)$ exist and continuous.\\

Consider a point $P_0$ on the curve. Suppose $P_1$, $P_2$ are two neighbouring points of $P_0$. Draw a plane through the points $P_0$, $P_1$ and $P_2$. The limiting position of the plane as $P_1$, $P_2\rightarrow P_0$ along the curve is called the osculating plane of the curve at $P_0$.\\

\subsection{Analytic form of the osculating plane}

As points on the space curve are ordinary points so $\phi'_1(t)=0=\phi'_2(t)=\phi'_3(t)$ will not hold simultaneously. Let $t_0$, $t_1$ and $t_2$ be the value of the parameter at the points $P_0$ and the neighbouring points $P_1$ and $P_2$ respectively. Suppose
\begin{equation}\label{eq3.1}
	ax+by+cz+d=0
\end{equation}
represent a plane in the three dimensional space, where $a$, $b$, $c$ and $d$ are constants. Let us define
\begin{equation}
	F(t)=a\phi_1(t)+b\phi_2(t)+c\phi_3(t)+d
\end{equation}

If we suppose that the plane (\ref{eq3.1}) passes through the points $P_0$, $P_1$ and $P_2$ then we have
\begin{equation}
	F(t_0)=0=F(t_1)=F(t_2)
\end{equation}

Note that $F(t)$ is a continuous function of $t$ having second order continuous derivatives. Now applying Role's theorem on $\left[t_0,t_1\right]$ and $\left[t_2,t_0\right]$ (assuming $t_2<t_0<t_1$) we have
\begin{equation}
	F'(\tau_1)=0 \mbox{ ~and ~}F'(\tau_2)=0
\end{equation}
where $t_0<\tau_1<t_1$ and $t_2<\tau_2<t_0$. Further, application  of Rolle's theorem on $F'(t)$ in $\left[\tau_2,\tau_1\right]$, we obtain $F''(\tau_3)=0$, $\tau_2<\tau_3<\tau_1$. Now due to limiting process $(P_1,P_2)\rightarrow P_)$ i.e. $(t_1,t_2)\rightarrow t_0$, we have $(\tau_1,\tau_2)\rightarrow t_0$ and also $\tau_3\rightarrow t_0$. Hence due to the conditioning of $F'(t)$ and $F''(t)$ we must have
\begin{equation}\label{eq3.5}
	F(t_0)=0=F'(t_0)=F''(t_0)
\end{equation}

As $P_0$ is any point on the plane (\ref{eq3.1}) so the result in equation (\ref{eq3.5}) holds for all points on the plane (\ref{eq3.1}) i.e.
\begin{equation}\label{eq3.6}
	F(t)=0=F'(t)=F''(t)
\end{equation}

Hence we have
\begin{eqnarray}
	ax+by+cz+d=0\nonumber\\ax'+by'+cz'~~~~=0\nonumber\\ax''+by''+cz''~~~=0
\end{eqnarray}

Thus solving the above relations one can determine $a$, $b$ and $c$  uniquely. Therefore, the osculating plane through a given point is always unique. We shall now determine the analytic form of the osculating plane through the point $P_0(x_0,y_0,z_0)$ as follows:\\

By notation, $x_0=x(t_0)$, $y_0=y(t_0)$, $z_0=z(t_0)$. As the plane (\ref{eq3.1})  passes through $P_0$ so we have
\begin{equation}\label{eq3.8}
	ax_0+by_0+cz_0+d=0
\end{equation}

Now, eliminating $d$  between  (\ref{eq3.1}) and (\ref{eq3.8})  we have
\begin{equation}\label{eq3.9}
	a(x-x_0)+b(y-y_0)+c(z-z_0)=0
\end{equation}

Also from relations (\ref{eq3.5}) (or (\ref{eq3.6}) ) we write
\begin{eqnarray}
	ax_0'+by_0'+cz_0'=0\\ax_0''+by_0''+cz_0''=0\label{eq3.11}
\end{eqnarray}

So eliminating $a$, $b$ and $c$ from equations (\ref{eq3.9}) - (\ref{eq3.11}) we obtain the equation of the osculating plane as
\begin{equation}
	\begin{array}{|ccc|}
		x-x_0&y-y_0&z-z_0\\x_0'&y_0'&z_0'\\x_0''&y_0''&z_0''
	\end{array}=0
\end{equation}

\subsection{Properties of the osculating plane}

The d.c. of the normal to the plane are proportional to $y_0'z_0''-z_0'y_0''$, $z_0'x_0''-x_0'z_0''$, $x_0'y_0''-y_0'x_0''$.

Now the d.c. of the tangent to the space curve at $P_0(x,y,z)$ are proportional to $\left(x_0',y_0',z_0'\right)$. Now, from the identity
$$x_0'\left(y_0'z_0''-z_0'y_0''\right)+y_0'\left(z_0'x_0''-x_0'z_0''\right)+z_0'\left(x_0'y_0''-y_0'x_0''\right)=0$$
it is evident that the tangent at $P_0$ to the space-curve lies on the osculating plane.\\

Let us now consider a line $L$ passing through $P_0$ having d.c. proportional to $\left(x_0'',y_0'',z_0''\right)$. Again from the identity 
$$x_0''\left(y_0'z_0''-z_0'y_0''\right)+y_0''\left(z_0'x_0''-x_0'z_0''\right)+z_0''\left(x_0'y_0''-y_0'x_0''\right)=0$$
we conclude that the line $L$ also lies on the osculating plane. Now to  find a relation between the tangent to the curve and the line $L$ we shall assume without any loss of generality that the arc length (measured from some fixed point) is taken as the parameter. So now $\left(x_0',y_0',z_0'\right)$ is the d.c. of the tangent (here $'$ stands for differentiation with respect to the arc length). Hence we write
\begin{eqnarray}
	\left(x_0'\right)^2+\left(y_0'\right)^2+\left(z_0'\right)^2=1\nonumber\\\mbox{i.e., }x_0'x_0''+y_0'y_0''+z_0'z_0''=0\nonumber
\end{eqnarray}
which shows that the line $L$ is orthogonal to the tangent line at $P_0$ but both of them lies on the osculating plane. Hence the line $L$ is directed along the principal normal to  the curve at $P_0$. Thus the osculating plane contains both the tangent line and the principal normal to the space curve at $P_0$.\\

\subsection{Interpretation of curvature}

We shall now give an interpretation of the curvature of the space-curve in analogy to a plane curve. Suppose with some fixed origin let $\vec{r}_0$ be the position vector of $P_0$ i.e.,
$$\vec{r}_0=x_0\vec{i}+y_0\vec{j}+z_0\vec{k}$$
where $\left(\vec{i},\vec{j},\vec{k}\right)$ are the unit vectors along the co-ordinate axes fixed in space. Choosing arc length as the parameter 
$$\vec{r}_0'=x_0'\vec{i}+y_0'\vec{j}+z_0'\vec{k}$$

Here $\vec{t}_1$ is the unit vector along the tangent to the space curve at $P_0$. So,
$$\vec{r}_0''=x_0''\vec{i}+y_0''\vec{j}+z_0''\vec{k}$$
is directed along the principal normal.\\

On the other hand, for the plane curve, 
$$\vec{r}_0''=x_0''\vec{i}+y_0''\vec{j}$$

Also, $	\left(x_0'\right)^2+\left(y_0'\right)^2=1$

 i.e., $x_0'x_0''+y_0'y_0'''=0$ (differentiating with respect to $s$)\\
 
 So we write
 \begin{eqnarray}
 	&&\frac{x_0'}{y_0''}=\frac{y_0'}{-x_0'}=\frac{x_0'y_0''-y_0'x_0''}{x_0''^2+y_0''^2}=\frac{\sqrt{x_0'^2+y_0'^2}}{\sqrt{x_0''^2+y_0''^2}}=\frac{1}{\sqrt{x_0''^2+y_0''^2}}\nonumber\\&\implies& x_0'y_0''-y_0'x_0''=\sqrt{x_0''^2+y_0''^2}=|\vec{r}_0''|\nonumber
 \end{eqnarray}

However, for a plane curve in parametric form, the curvature is expressed as
$$\frac{x'y''-y'x''}{\left(x'^2+y'^2\right)^{\frac{3}{2}}}=x'y''-y'x''=|\vec{r}_0''|$$

Hence the magnitude of $\vec{r}_0^{''}$ gives the curvature of a plane curve. Now generalizing this idea to space curve, we define the curvature at a point of a curve in space as the magnitude of the vector
$$\vec{r}_0^{''}=x_0''\vec{i}+y_0''\vec{j}+z_0''\vec{k}$$
i.e., $\kappa=\dfrac{1}{\rho}=|\vec{r}_0^{''}|$ (at $P_0$).

Thus the unit normal vector $\vec{n}$ can be defined as
$$\vec{n}=\rho\vec{r}_0^{''}=\frac{\vec{r}_0^{''}}{|\vec{r}_0^{''}|}$$
Consequently, the unit vector perpendicular to the osculating plane at $P_0$ is $$\vec{b}=\vec{t}_1\times\vec{n}$$
and is along the binormal vector of the curve at $P_0$.\\

\subsection{Curvature of a surface}

We now introduce the notion of curvature of a surface. Let us consider the section of aa surface by a plane parallel to an indefinitely near to tangent plane at any point (say $O$) on it. Clearly, the section will be a conic whose centre lies on the normal to the surface at $O$. This conic is called the \textit{indicatrix}.

\begin{wrapfigure}[13]{r}{0.35\textwidth}
	\centering	\includegraphics[height=5 cm , width=5.3 cm ]{f12.pdf}
	\begin{center}
		Fig. 3.9
	\end{center}
\end{wrapfigure}

Let any plane through the normal $OV$ at $O$ cut the indicatrix along the diameter $QVQ'$. Let $\rho$ be the radius of curvature at $O$ of the normal section. Then $$\rho=\lim\limits_{V\to O} \frac{{QV}^2}{2 OV}$$

Thus the radius of curvature of different normal sections varies as the square  of the diameter of the conic through that section.

Moreover, it is well known for a conic that the sum of the squares of the reciprocal of two perpendicular semi diameters is constant. This implies the sum of the reciprocal of radii of curvature of two perpendicular normal sections is constant. Also the diameter of a conic section has a maximum and a minimum value i.e. radius of curvature for different normal sections has a maximum and a minimum value. The sections having maximum and minimum value of the radius of curvature are called the principal sections and the corresponding radius of curvatures are the principal radius of curvature.\\

\subsection{Euler's Theorem}

\textbf{Statement:}\\
Suppose M be a surface in 3D Euclidian space and P is a point on M. A normal plane through P is a plane passing through P containing the normal vector to M. For each tangent vector $\vec{S}$ to the manifold at P, there exists a normal plane $M_X$ which intersects M in a curved having non-constant curvature $\kappa_X$. Let $\kappa_1=\kappa_{X_1}$ and $\kappa_2=\kappa_{X_2}$ are the largest and smallest possible curvatures then Euler's theorem state that corresponding  $\vec{X}_1$ and $\vec{X}_2$ are orthogonal to each other and if $\theta$ be the angle between $\vec{X}_1$ and $\vec{X}_2$ then $\kappa_X= \kappa_1 cos^2 \theta + \kappa_2~ sin^2\theta $ $i.e.,$ $1/\rho=cos^2 \theta/\rho_1 +sin^2 \theta/\rho_2$,
with $\rho,~ \rho_1$ and $\rho_2$ are the corresponding radii of the curvature.

\textbf{Proof:} We choose $XY$ plane as the tangent plane to the surface at $O$ and $z$ axis is along the normal to the surface at $O$. Suppose $x$ and $y$ axes are taken along the axes of the indicatrix. So upto second order the equation of the surface (i.e. neglecting 3rd and higher order terms) takes the form
$$2z=ax^2+by^2$$
Suppose $\rho_1$ and $\rho_2$ be the principal radii of curvature at $O$ i.e.
$$\rho_1=\lim\limits_{\substack{x\to0\\y\to0}}\frac{x^2}{2z}=\frac{1}{a},~\rho_2=\lim\limits_{\substack{x\to0\\y\to0}}\frac{y^2}{2z}=\frac{1}{b}$$
\begin{wrapfigure}[13]{r}{0.35\textwidth}
	\centering	\includegraphics[height=5 cm , width=5.3 cm ]{f13.pdf}
	\begin{center}
		Fig. 3.10
	\end{center}
\end{wrapfigure}
So the above equation of the surface becomes
$$2z=\frac{x^2}{\rho_1}+\frac{y^2}{\rho_2}$$

If the diameter $VK$ makes an angle $\theta$ with $x$ direction, then 
\begin{eqnarray}
	2z&=&\frac{r^2\cos^2\theta}{\rho_1}+\frac{r^2\sin^2\theta}{\rho_2}\nonumber\\\mbox{i.e. }\frac{2z}{r^2}&=&\frac{\cos^2\theta}{\rho_1}+\frac{\sin^2\theta}{\rho_2}\nonumber
\end{eqnarray}

Hence if $\rho$ be the radius of curvature of the normal section of the surface through the diameter $VK$, then
$$\rho=\lim \frac{VK^2}{2OY}$$
i.e. in the limit $2\rho OV=VK^2=r^2$

i.e. $\dfrac{2z}{r^2}=\dfrac{1}{\rho}$ in the limit.

$\therefore$ $\dfrac{1}{\rho}=\dfrac{\cos^2\theta}{\rho_1}+\dfrac{\sin^2\theta}{\rho_2}$, which is the Euler's result.\\

\subsection{Meusnier's Theorem}

If $P_0$ and $P$ be the radii of curvature of a normal section and an oblique section of a surface through the same tangent line then $\rho=\rho_0\cos\theta$, $\theta$ is the angle between the sections.
\begin{wrapfigure}[16]{r}{0.35\textwidth}
	\centering	\includegraphics[height=6 cm , width=5.3 cm ]{f14.pdf}
	\begin{center}
		Fig. 3.11
	\end{center}
\end{wrapfigure}

\textbf{Proof:} Let $XOY$ plane be taken as the tangent plane to the surface at $O$, $z$ axis is along the normal to the surface at $O$. The $x$ axis is taken along the common tangent.\\

Let $OC=h$. The equation of the indicatrix, when third and higher order terms are neglected is given by
$$z=h,~2h=rx^2+2sxy+ty^2,$$ so that
$\rho_0=\lim\dfrac{CA^2}{2OC}=\dfrac{1}{r}$ (as equation of the normal section is $y=0$.

Equation of $QVQ'$ is $z=h$, $y=h\tan\theta$, so for points of intersection of $QVQ'$ with the surface  we have
$$2h=rx^2+2sxh\tan\theta+th^2\tan^2\theta$$

Note that if $x$, $y$ are 1st order quantities then $h$ (and also $z$) is a second order quantity. So neglecting $h^2$, $hx$ terms we get (upto 2nd order of smallness)
$$2h=rx^2\mbox{~ i.e. ~} QV^2=\frac{2h}{r}$$
$\therefore$ $\rho=\lim\limits_{h\to 0}\dfrac{QV^2}{2OV}=\lim\limits_{h\to0}\dfrac{\frac{2h}{r}}{2h\sec\theta}=\dfrac{\cos\theta}{r}=\rho_0\cos\theta$

\section{Motion of a particle on a fixed smooth surface}

\begin{wrapfigure}[13]{r}{0.35\textwidth}
	\centering	\includegraphics[height=5 cm , width=5.3 cm ]{f15.pdf}
	\begin{center}
		Fig. 3.12
	\end{center}
\end{wrapfigure}

Let $m$ be the mass of the particle $P$ moving on a fixed smooth surface with velocity $v$ at any time $t$. Suppose $PT$, $PN$, $PB$ be the tangent, principal normal and binormal to the path of the particle (on the surface) at $P$. Let $PN_0$ is the normal to the surface at $P$ and $PB_0$ is a tangent line on the surface at $P$ and is perpendicular to $PT$.\\

As $PT$  is perpendicular to $PN_0$, $PN$, $PB_0$, and $PB$ so they are all coplanar. Let $\chi$ be the angle which the osculating plane at $P$ makes with the plane normal to the surface and passing through $PT$. So we have $\angle N_0PN=\chi$.\\

Let $\rho$ be the radius of curvature of the path at $P$. Then the acceleration of the particle at $P$ has components $v\dfrac{\mathrm{d}v}{\mathrm{d}s}$ along $PT$ and $\dfrac{v^2}{\rho}$ along the principal normal $PN$.
$$\frac{v^2}{\rho}\mbox{ along }PN\equiv\frac{v^2}{\rho}\cos\chi\mbox{ along }PN_0+\frac{v^2}{\rho}\sin\chi\mbox{ along }PB_0$$

If $F$, $G$ and $H$ are the components of the external force acting on the particle $P$ along $PT$, $PN_0$ and $PB_0$ and $R$ be the reaction of the surface on the particle along $PN_0$, then the equation of motion of the particle are
$$mv\dfrac{\mathrm{d}v}{\mathrm{d}s}=F~,~~m\frac{v^2}{\rho}\cos\chi=G+R~,~~m\frac{v^2}{\rho}\sin\chi=H$$

Further, if $\rho_0$ be the radius of curvature of the normal section through $PT$, then by Meunier's theorem $\rho=\rho_0\cos\chi$, so that the equations of motion become
$$mv\dfrac{\mathrm{d}v}{\mathrm{d}s}=F~,~~m\frac{v^2}{\rho_0}=G+R~,~~m\frac{v^2}{\rho_0}\tan\chi=H$$

The first equation gives the velocity of the particle at any instant. Knowing $v$, the normal reation of the surface can be determined from the second equation.  The third equation gives the position of the osculating plane and also the differential equation of the path of the particle.\\

\textbf{Corollary:} If the external forces are absent then the equation of motion simplifies to
\begin{eqnarray}
	mv\dfrac{\mathrm{d}v}{\mathrm{d}s}=0\label{eq3.13}\\m\frac{v^2}{\rho_0}=R\label{eq3.14}\\m\frac{v^2}{\rho_0}\tan\chi=0\label{eq3.15}
\end{eqnarray}

The first equation shows that the particle moves on the surface with constant velocity. Equation (\ref{eq3.14}) gives the normal reaction of the surface on the particle. Equation (\ref{eq3.15}) implies $\chi=0$ (assuming $\dfrac{1}{\rho_0}\neq0$ i.e. the surface is not a plane surface) i.e. the osculating plane of the path of the particle contains the normal to the surface. So the path of the particle is a geodesic on the surface. Thus if a particle moves freely on a smooth surface then it describes a geodesic on the surface with constant velocity.\\

\textbf{Note:} The path of a particle on a smooth surface may be geodesic even in the presence of non-zero external forces. In particular, if the external forces be such that $H=0$ then the path of the particle will be a geodesic on the surface. Further, if a particle moves on a rough surface under no other external forces then the path of the particle will also be a geodesic and the velocity of the particle will gradually diminish until it comes to rest.\\

\subsection{Motion of a particle on a smooth surface of revolution}

\begin{wrapfigure}[13]{r}{0.35\textwidth}
	\centering	\includegraphics[height=5 cm , width=4 cm ]{f16.pdf}
	\begin{center}
		Fig. 3.13
	\end{center}
\end{wrapfigure}


Let the axis $oz$ be taken along the axis of revolution and let the equation of meridian section of the surface through the particle $P(r,\theta,z)$ at any time be
$$z=\phi(r),~r=\sqrt{x^2+y^2}=CP$$
Since the surface is smooth the reaction of the surface will be along the normal to the surface. Hence the work done by the reaction in any displacement of the particle on the surface is zero. So the equation of energy gives
\begin{equation}\label{eq3.16}
	\mathrm{d}\left(\frac{1}{2}mv^2\right)=X\mathrm{d}x+Y\mathrm{d}y+Z\mathrm{d}z
\end{equation}
where $m$ is the mass of the particle, $v$ is velocity at any time $t$, $X$, $Y$, $Z$ are the components of the external forces along $x$, $y$, $z$ axes. \\

As the reaction $R$ of the surface is acting along the normal to the surface, so $R$ lies on the meridian plane passing through the particle. Hence the moment of $R$ about the axis of revolution is zero. The components of the velocity of $P$ along $(r,\theta,z)$ direction are $\dot{r}$, $r\dot{\theta}$ and $\dot{z}$  respectively. As $\dot{r}$ and $\dot{z}$ lie on the meridian plane and $r\dot{\theta}$ is perpendicular to the meridian plane.  Hence the angular momentum of $P$ about $oz$ is $mr^2\dot{\theta}$. So by the principle of angular momentum
\begin{equation}\label{eq3.17}
	\frac{\mathrm{d}}{\mathrm{d}t}(mr^2 \dot{\theta})=-Xy+Yx=r(Y\cos\theta-X\sin\theta)
\end{equation}

 Thus equations (\ref{eq3.16}) and (\ref{eq3.17}) will determine the motion of the particle on the surface.\\
 
We now assume that the components of external forces are derived from a force function $U(r,z)$ where $U(r,z)$ is symmetrical about the axis of evolution.  Then we get from (\ref{eq3.16})
\begin{equation}\label{eq3.18}
	\frac{1}{2}mv^2=U+\frac{1}{2}mc.~~\left(\mbox{$c$, the constant of integration}\right)
\end{equation}

As the components of the external force along $r$, $\theta$, $z$ axes are $\dfrac{\partial U}{\partial r}$, $0$, $\dfrac{\partial U}{\partial z}$ respectively so the force lies on the meridian plane passing through the particle. Hence the moment of the external force about the axis of revolution is zero.  So from equation (\ref{eq3.17}) we get
$$\frac{\mathrm{d}}{\mathrm{d} t}\left(mr^2\dot{\theta}\right)=0$$
which an integration gives 
\begin{equation}\label{eq3.19}
	r^2\dot{\theta}=h, ~~\left(\mbox{a constant independent of $t$}\right)
\end{equation}

As the particle moves on the surface $z=\phi(r)$, so
$$\dot{z}=\frac{\mathrm{d}\phi}{\mathrm{d}r}\dot{r}=\phi'(r)\dot{r}.$$

Hence the velocity of the particle is given by
\begin{eqnarray}
	v^2&=&\dot{r}^2+r^2\dot{\theta}^2+z^2\nonumber\\&=&\left[1+\{\phi'(r)\}^2\right]\dot{r}^2+\frac{h^2}{r^2}\nonumber
\end{eqnarray}

So from (\ref{eq3.18}) i.e. $v^2=\dfrac{2U}{m}+c$
we obtain
$$\frac{\mathrm{d}r}{\mathrm{d}t}=\pm\sqrt{\frac{\frac{2U}{m}+c-\frac{h^2}{r^2}}{1+\{\phi'(r)\}^2}}$$
\hfill(the sign on the right hand side is to be chosen suitably)
\begin{equation}\label{eq3.20}
	\mbox{i.e. }\mathrm{d}t=\pm\sqrt{\frac{1+\{\phi'(r)\}^2}{\frac{2U}{m}+c-\frac{h^2}{r^2}}}~\mathrm{d}r
\end{equation}

Also from (\ref{eq3.19}) we get
\begin{equation}\label{eq3.21}
	\mathrm{d}\theta=\frac{h}{r^2}\mathrm{d}t=\pm\frac{h}{r^2}\sqrt{\frac{1+\{\phi'(r)\}^2}{\frac{2U}{m}+c-\frac{h^2}{r^2}}}~\mathrm{d}r
\end{equation}

Integrating (\ref{eq3.20}) and (\ref{eq3.21}) we get $t$ and $\theta$ as functions of $r$. Thus the solution of the problem of the motion of a particle on smooth surface of revolution under the action of external forces which can be derived from a force function symmetrical about the axis of revolution is reduced to two quadratures.\\

\subsection{Motion of a heavy particle on a smooth surface of revolution the axis of which is vertical}

We take the vertical axis of revolution as $z$ axes and use cylindrical co-ordinates $(r,\theta,z)$. If $v$ be the velocity of the particle at any time $t$ then
$$v^2=\dot{r}^2+r^2\dot{\theta}^2+\dot{z}^2$$

The energy equation gives $$v^2=c-2gz$$

$\left(\dfrac{\mathrm{d}}{\mathrm{d}t}\left(\dfrac{1}{2}mv^2\right)=-mg\mathrm{d}z,\mbox{ on integration, }\dfrac{1}{2}mv^2=-mgz+\dfrac{1}{2}mc\right)$

\begin{wrapfigure}[13]{r}{0.35\textwidth}
	\centering	\includegraphics[height=5 cm , width=4 cm ]{f17.pdf}
	\begin{center}
		Fig. 3.14
	\end{center}
\end{wrapfigure}

Since the forces acting on the particle are the force of gravitation acting parallel to the axis (of revolution i.e., $z$ axis) and the normal reaction of the surface intersects the axis of revolution, hence the moment of momentum about the axis is a constant. Thus $r^2\dot{\theta}=h$, a constant. Let the equation of the meridian curve through the particle be $z=\phi(r)$, so that $\dot{z}=\phi'(r)\dot{r}$.\\

\begin{equation}
\therefore~v^2=\left(1+\{\phi'(r)\}^2\right)\dot{r}^2+\dfrac{h^2}{r^2}=c-2gz~.\nonumber
\end{equation}

$\left(1+\{\phi'(r)\}^2\right)\dot{r}^2=c-2gz-\dfrac{h^2}{r^2}$\\

Also $r^4\dot{\theta}^2=h^2$.\\

$\therefore~\dfrac{\left(1+\{\phi'(r)\}^2\right)}{r^4}\left(\dfrac{\mathrm{d}r}{\mathrm{d}\theta}\right)^2=\dfrac{c}{h^2}-\dfrac{2g}{h^2}z-\dfrac{1}{r^2}$\\

This is the differential equation of the path of the particle on the horizontal plane.\\

Let us now find the condition that the path of the particle on the surface is a circle of radius $r_0$ (say).\\

 Let $v_0$ be the velocity of the particle in its circular path at $P$. The acceleration of the particle is $\dfrac{v_0^2}{r_0}$ and it is maintained by its weight $mg$ and the normal reaction of the surface $R$. So we have 
  \begin{wrapfigure}[11]{r}{0.35\textwidth}
 	\centering	\includegraphics[height=4 cm , width=5 cm ]{f18.pdf}
 	\begin{center}
 		Fig. 3.15
 	\end{center}
 \end{wrapfigure}
 $$m\dfrac{v_0^2}{r_0}=R\sin\psi_0,~~mg=R\cos\psi_0~.$$
 
 Hence $\dfrac{v_0^2}{gr_0}=R\tan\psi_0$.\\
 
 But $\tan\psi_0=\dfrac{\mathrm{d}z}{\mathrm{d}r}{\bigg|}_{r=r_0}=\phi'(r_0)$.\\
 
 $\therefore~ \dfrac{v_0^2}{r_0}=\phi'(r_0)\mbox{ i.e., }v_0^2=gr_0\phi'(r_0)$.\\
 
 This is the condition that has to be satisfied to maintain a circular path.\\
 
 \subsection{Determine whether a heavy particle will rise or fall when it is projected horizontally on the surface of revolution}
 
 Let the vertical axis of revolution is taken as $z$ axis and $v_0$ be the initial velocity of projection from a point having $r$, $z$ co-ordinates $r_0$ and $z_0$ respectively. Suppose $r=f(z)$ be the equation of the meridian section of the surface then the energy equation gives
 $$v^2=c-2gz$$
 and the equation of angular momentum gives
 $$r^2\dot{\theta}=h$$
 
 Using initial conditions we have 
 $$v_0^2=c-2gz_0,~h=v_0r_0,~r_0=f(z_0)$$
 
 Thus we get $$v^2=v_0^2+2g(z_0-z)$$
 and $$r^2\dot{\theta}=v_0r_0$$
 
 But $v^2=\dot{r}^2+r^2\dot{\theta}^2+\dot{z}^2=\dot{z}^2\{1+[f'(z)]^2\}+\dfrac{v_0^2r_0^2}{r^2}$\\
 
 Thus we obtain
 \begin{eqnarray}
 	&&\dot{z}^2\{1+[f'(z)]^2\}+\dfrac{v_0^2r_0^2}{\{f(z)\}^2}=v_0^2+2g(z_0-z)\nonumber\\
 	\mbox{i.e., }&&\dot{z}^2\{1+[f'(z)]^2\}=v_0^2\left[1-\left\{\frac{f(z_0)}{f(z)}\right\}^2\right]+2g(z_0-z)=F(z)\mbox{ (say)}\nonumber
 \end{eqnarray}

Hence the particle will rise or fall if $\dot{z}\neq0$ and hence $F(z)>0$ with $F(z_0)=0$.\\

Using mean value theorem of differential calculus on $f(z)$ we obtain
\begin{eqnarray}
	&&\frac{F(z)-F(z_0)}{z-z_0}=F'\{z_0+(z-z_0)\theta\},~~0<\theta<1\nonumber\\
	\mbox{i.e., }&&F(z)=(z-z_0)F'\{z_0+(z-z_0)\theta\}>0\nonumber
\end{eqnarray}

This shows that $z-z_0$ and $F'\{z_0+(z-z_0)\theta\}$ will have same sign, however small $z-z_0$ may be. The particle will rise or fall according as $z-z_0$ is positive or negative i.e., according as $F'\{z_0+(z-z_0)\theta\}$ is positive or negative. Now making $z\to z_0$, the particle will rise or fall according as $F'(z_0)$ is positive or negative.\\

Now, $F'(z)=2\dfrac{\{f(z_0)\}^2}{\{f(z)\}^3}f'(z)v_0^2-2g$\\

So $F'(z_0)=2\left[\dfrac{f'(z_0)}{f(z_0)}v_0^2-g\right]$\\

Hence the particle will rise or fall according as
$$\dfrac{f'(z_0)}{f(z_0)}v_0^2~>\mbox{ or }<g$$
\vspace{4.5cm}

\subsection{Initial radius of curvature}

  \begin{wrapfigure}[11]{r}{0.35\textwidth}
	\centering	\includegraphics[height=4 cm , width=5 cm ]{f19.pdf}
	\begin{center}
		Fig. 3.16
	\end{center}
\end{wrapfigure}

Let a particle be projected horizontally with velocity $v_0$ from a point on the surface whose $r$ and $z$ co-ordinates are $r_0$, $z_0$ respectively. Suppose $\rho$ be the initial radius of curvature of the path and $\rho_0$ be the radius of curvature of the normal section of the surface through the tangent to the path at the pint under consideration. Let $\chi$ be the angle between the normal section at that point and the osculating plane. Let $v$ be the velocity of the particle at that point then we have the equation of motion 
$$m\frac{v^2}{\rho_0}\tan\chi=H$$
where $H$ is the component of the external forces along the tangent to the surface but perpendicular to the tangent to the path.\\

  \begin{wrapfigure}[11]{r}{0.35\textwidth}
	\centering	\includegraphics[height=4 cm , width=5 cm ]{f20.pdf}
	\begin{center}
		Fig. 3.17
	\end{center}
\end{wrapfigure}

In the figure $PG$ is the normal to the surface at $P$ and intersects the $z$ axis at $G$. Hence the radius of curvature of the normal section passing through the tangent to the curve at P will be $\rho_0=PG=\dfrac{r_0}{\sin\phi_0}$.\\

Hence $H=mg\sin\phi_0$, as $H$ is acting along the tangent to the meridian curve.\\

Thus we have
\begin{eqnarray}
&&	m\frac{v^2}{\rho_0}\sin\phi_0\tan\chi=mg\sin\phi_0\nonumber\\
\mbox{i.e., }&&\frac{v^2}{\rho_0}\tan\chi=g ~~(\because\sin\phi_0\neq0)\nonumber\\
\mbox{i.e., }&&\tan\chi=\frac{gr_0}{v^2}\nonumber\\
\therefore&&\rho=\rho_0\cos\chi=\frac{r_0v^2}{\sin\phi_0\sqrt{v^4+g^2r_0^2}}\nonumber
\end{eqnarray}


\subsection{Motion of a heavy particle on the surface of a smooth sphere}
  \begin{wrapfigure}[11]{r}{0.35\textwidth}
	\centering	\includegraphics[height=4 cm , width=4 cm ]{f21.pdf}
	\begin{center}
		Fig. 3.18
	\end{center}
\end{wrapfigure}

We take the centre of the sphere $O$ as origin and $z$ axis vertically downwards. Let $a$ be the radius of the sphere and $P(a,\theta,\phi)$ be the position of the particle at any time $t$. The energy equation gives
\begin{eqnarray}
	&&\frac{1}{2}mv^2=mgz+\frac{1}{2}mc\nonumber\\
	\mbox{i.e., }&&v^2=c+2gz\nonumber\\
	\mbox{i.e., }&&v^2=c+2ga\cos\theta\label{eq3.22}
\end{eqnarray}

The conservation of angular momentum is written as
\begin{eqnarray}
	&&\frac{\mathrm{d}}{\mathrm{d}t}\left(a^2\dot{\phi}\sin^2\theta\right)=0\nonumber\\
	\mbox{i.e. }&&a^2\sin^2\theta\dot{\phi}=h\mbox{  ~~~~~(a constant)}\label{eq3.23}
\end{eqnarray}

The equation of motion along the normal  $PO$ is
\begin{equation}\label{eq3.24}
	m\frac{v^2}{\rho_0}=G+R
\end{equation}
where $G$ is the component of the external force along the inward drawn normal, $R$ is the reaction of the surface and $\rho_0$ is the radius of curvature of the normal section of the surface through a plane passing through the tangent to the path at $P$.\\

As any normal section is a great circle for a sphere so $\rho_0=a$ and $G=-mg\cos\theta$.\\

$\therefore~R=m\dfrac{v^2}{a}+mg\cos\theta=mg\left(\dfrac{c}{ag}+3\cos\theta\right)$.\\

Also $v^2=\dot{r}^2+r^2\dot{\theta}^2+r^2\sin^2\theta\dot{\phi}^2=a^2\dot{\theta}^2+a^2\sin^2\theta\dot{\phi}^2$.\\

Now using equations (\ref{eq3.22}) and (\ref{eq3.23}) we get
$$a^4\sin^2\theta\dot{\theta}^2=a^2\sin^2\theta(c+2ga\cos\theta)-h^2$$

Putting $z=a\cos\theta$, the above equation becomes
\begin{eqnarray}
	a^2\dot{z}^2&=&(a^2-z^2)(c+2gz)-h^2\nonumber\\
	&=&2g(a^2-z^2)(c_0+z)-h^2=F(z)\mbox{ ~~(say)}\nonumber
\end{eqnarray}	
where $c_0=\dfrac{c}{2g}$.\\

Note that, $F(+\infty)=-\mbox{ve}$, $F(a)=-\mbox{ve}$, $F(-a)=-\mbox{ve}$,  $F(-\infty)=+\mbox{ve}$. As for any actual position of the particle $z=z_0$ (say) on the surface of the sphere, we have $\dot{z}^2>0$ i.e. $F(z_0)>0$, so we have 
$$F(a)=-\mbox{ve}, F(z_0)=+\mbox{ve}, F(-a)=-\mbox{ve}, F(-\infty)=+\mbox{ve}$$

Thus there exists three real roots of $F(z)=0$ namely $z_1$, $z_2$ and $-z_3$ within the intervals $(z_0,a)$, $(-a,z_0)$ and $(-\infty,-a)$ respectively. Hence we write
\begin{eqnarray}
	\dot{z}^2=F(z)=2g(z_1-z)(z-z_2)(z_3+z)\nonumber\\\mbox{i.e. }z=\pm\int\sqrt{2g(z_1-z)(z-z_2)(z_3+z)}\mathrm{d}t\nonumber
\end{eqnarray} 

This gives $z$ i.e. $\theta$ in terms of $t$.\\

Note that $\dot{z}$ is real if $z_1>z>z_2$ and $\dot{z}=0$ at $z=z_1$ and $z_2$. Thus the motion of the particle is confined to a zone bounded by the circles $z=z_1$ and $z=z_2$.

Further, from (\ref{eq3.23}), $\dot{\phi}=\dfrac{h}{a^2\sin^z\theta}=\dfrac{h}{(a^2-z^2)}$,

which gives the angular velocity of the meridian plane through the particle about $OZ$. So knowing $\theta$ in terms of $t$, one can obtain $\phi$ as a function of time after integrating once.\\\\

\section{Problems}

{\bf 1. } A particle is projected horizontally with velocity $\sqrt{2ga}$ along the smooth surface of a sphere of radius $a$ at the at the level of the centre. Prove that the motion is confined between two horizontal planes at a distance $\dfrac{1}{2}(\sqrt{5}-1)a$ apart.\\

{\bf Solution: } We choose the centre of the sphere as origin and $z$-axis vertically downward. Let $P(a,\theta,\phi)$ be the position of the particle at time $t$ in spherical polar co-ordinates.\\

\begin{wrapfigure}[8]{r}{0.35\textwidth}
	\centering	\includegraphics[height=4 cm , width=4 cm ]{f22.pdf}
	\begin{center}
		Fig. 3.19
	\end{center}
\end{wrapfigure}

Then the equation of energy gives
\begin{eqnarray}
	\frac{1}{2}mv^2=\frac{1}{2}mc+mgz\nonumber\\\mbox{i.e. }v^2=c+2gz=c+2ag\cos\theta\nonumber
\end{eqnarray}

Initially, $v=\sqrt{2ga}$, $\theta=\dfrac{\pi}{2}$ 

$\Rightarrow c=2ga$. \\

$\therefore v^2=2g(a+z)$ where $z=a\cos\theta$, gives the velocity of the particle at $P$.\\

Now, the equation of motion along the normal to the surface at $P$ is 
$$m\frac{v^2}{\rho_0}=R-mg\cos\theta$$
where $\rho_0$ is the radius of curvature of the normal section of the surface through a plane passing through the tangent to the path of $P$.\\

As any normal section is a great circle in case of a sphere so $\rho_0=a$. Hence we obtain
\begin{equation}
	R=m\left(\frac{v^2}{a}+\frac{gz}{a}\right)=\frac{mg}{a}(3z+2a)
\end{equation}

Thus $R$ vanishes when $z=-\dfrac{2}{3}a$.\\

Again, $v^2=\dot{r}^2+r^2\dot{\theta}^2+r^2\sin^2\theta\dot{\phi}^2=a^2\dot{\theta}^2+a^2\sin^2\theta\dot{\phi}^2$,

so the equation of angular momentum gives
\begin{eqnarray}
&&	a^2\sin^2\theta\dot{\phi}=h=a\sqrt{2ga}\nonumber\\
	\therefore&&a\sin\theta\dot{\phi}=\frac{\sqrt{2ga}}{\sin\theta}\nonumber\\
	\therefore&&v^2=a^2\dot{\theta}^2+\frac{2ga}{\sin^2\theta}=2ga+2gz\nonumber\\
	\therefore&&a^2\dot{\theta}^2=2ga+2gz-\frac{2ga}{1-\frac{z^2}{a^2}}\nonumber\\
		\mbox{i.e., }&&\dot{z}^2=\frac{2g}{a^2}\left[(a^2-z^2)(z+a)-a^3\right]=F(z)\mbox{ (say)}\nonumber
\end{eqnarray}

For maximum and minimum value of $z$, $\dot{z}=0$ and the values are given by
\begin{eqnarray}
	&&(a^2-z^2)(z+a)-a^3=0\nonumber\\\mbox{i.e., }&&z^3-a^2z+az^2=0\nonumber\\
	\implies&& z=0 \mbox{ or }z=\frac{a}{2}\left(-1\pm\sqrt{5}\right)\nonumber
\end{eqnarray}

As $z=-\dfrac{a}{2}\left(1+\sqrt{5}\right)<-a$, which is inadmissible so $\dot{z}=0$ at $z=0$ and $z=\dfrac{a}{2}\left(\sqrt{5}-1\right)$.\\

Thus the particle moves within the region bounded by the planes $z=0$, $z=\dfrac{a}{2}\left(\sqrt{5}-1\right)$. The distance between the plane is $\dfrac{a}{2}\left(\sqrt{5}-1\right)$.\\

\textbf{2.} A particle is projected horizontally under gravity with a velocity $V$ from a point on the inner surface of a smooth sphere at an angular distance $\alpha$ from the lowest point. Prove that the pressure on the surface when it is at angular distance $\theta$ from the lowest point is $$mg\left(3\cos\theta-2\cos\alpha+\frac{V^2}{ag}\right)$$
where $m$ is the mass of the particle and $a$ is the radius of the sphere. Prove that in the subsequent motion the particle will leave the surface if $3\sin\alpha<1$ and $2\dfrac{V^2}{ag}-7\cos\alpha$ lies between $\pm3\sqrt{1-9\sin^2\alpha}$.\\

\begin{wrapfigure}[11]{r}{0.35\textwidth}
	\centering	\includegraphics[height=5 cm , width=4 cm ]{f23.pdf}
	\begin{center}
		Fig. 3.20
	\end{center}
\end{wrapfigure}

\textbf{Solution: }The equation of energy gives
$$v^2=c+2gz=c+2ga\cos\theta$$

So $V^2=c+2ga\cos\alpha$.
\begin{equation}
	\therefore v^2=V^2+2ga(\cos\theta-\cos\alpha)
\end{equation}

The equation of angular momentum is
\begin{eqnarray}
	a^2\sin^2\theta\dot{\phi}=h=a\sin\alpha V\nonumber\\
	\mbox{i.e., }a\sin\theta\dot{\phi}=\frac{V\sin\alpha}{\sin\theta}
\end{eqnarray}

Now, $v^2=a^2\dot{\theta}^2+a^2\sin^2\theta\dot{\phi}^2=a^2\dot{\theta}^2+\dfrac{V^2\sin^2\alpha}{\sin^2\theta}$.\\

The equation of motion along the normal to the surface at $P$ is
\begin{eqnarray}
	m\frac{v^2}{a}&=&R-mg\cos\theta\label{eq3.2.3}\\
	\mbox{i.e., }~R&=&m\left(\frac{v^2}{a}+g\cos\theta\right)\nonumber\\&=&m\left\{\frac{V^2}{a}+2g(\cos\theta-\cos\alpha)+g\cos\theta\right\}\nonumber\\&=&mg\left(\frac{V^2}{ag}+3\cos\theta-2\cos\alpha\right)
\end{eqnarray}

From equation (\ref{eq3.2.3}) it is clear that if $\cos\theta$ is positive i.e., if $0<\theta\leq\dfrac{\pi}{2}$, i.e., the particle is within the lower hemisphere then $R>0$, so that it will not leave the surface of the sphere so long as it is within the region.\\

So, 
\begin{eqnarray}
	v^2&=&a^2\dot{\theta}^2+\frac{V^2\sin^2\alpha}{\sin^2\theta}=V^2+2ag(cos\theta-\cos\alpha)\nonumber\\
	\implies a^2\dot{\theta}^2&=&\frac{V^2}{\sin^2\theta}\left[\cos^2\alpha-\cos^2\theta\right]+2ag(\cos\theta-\cos\alpha)\nonumber\\
	&=&\frac{(\cos\alpha-\cos\theta)}{\sin^2\theta}\left[V^2(\cos\alpha+\cos\theta)-2ag\sin^2\theta\right]\nonumber\\
	&=&\frac{(\cos\alpha-\cos\theta)}{\sin^2\theta}f(cos\theta)\nonumber
\end{eqnarray}
where $f(\cos\theta)=2ga\cos^2\theta+V^2(\cos\alpha+\cos\theta)-2ag$.\\

For maximum and minimum value of $\theta$, $\dot{\theta}=0$ and it gives either $\theta=\alpha$ or $f(\cos\theta)=0$ where\\

$f(\cos0)=V^2(1+\cos\alpha)>0$.\\

$f(\cos\alpha)=2(V^2\cos\alpha-ag\sin^2\alpha)$.\\

$f(\cos\pi)=V^2(\cos\alpha-1)<0$.\\

Now, if $V^2\cos\alpha-ag\sin^2\alpha>0$, then the root of the equation $f(\cos\theta)=0$ will lie between $\cos\alpha$ and $-1$.   Hence in this case the particle projected horizontally at $\theta=\alpha$ will rise. However, if $V^2\cos\alpha-ag\sin^2\alpha<0$, the root of the equation $f(\cos\theta)=0$ will lie between $\cos\alpha$ and $1$. Hence the particle projected horizontally at $\theta=\alpha$ will fall. For $V^2\cos\alpha-ag\sin^2\alpha=0$, the particle will always at $\theta=\alpha$.\\

As $R=mg\left[\dfrac{V^2}{ag}+3\cos\theta-2\cos\alpha\right]$ , so the particle will leave the surface at a point where $R$ vanishes, provided that the particle reaches that point.\\

Now, 
\begin{eqnarray}
&&	f(\cos\theta)=0\nonumber\\\implies&&\cos^2\theta+2n^2(\cos\theta+\cos\alpha)-1, \mbox{ putting }\frac{v^2}{ag}=4n^2\nonumber\\
	\mbox{ i.e., }&&\cos\theta=-n^2\pm\sqrt{n^4+1-2n^2\cos\alpha}\nonumber
\end{eqnarray}

As the lower sign is inadmissible so $\dot{\theta}=0$ at $\theta_1=0$, where
$$\cos\theta=-n^2+\sqrt{n^4+1-2n^2\cos\alpha}$$ 

Now, if $\theta_1>\pi-\alpha$ then 
\begin{eqnarray} 
&&	\cos\theta_1<\cos(\pi-\alpha)=-\cos\alpha\nonumber\\
	\mbox{i.e., }&&-n^2+\sqrt{n^4+1-2n^2\cos\alpha}<-\cos\alpha\nonumber\\
	\mbox{i.e., }&&n^4+1-2n^2\cos\alpha<n^4+\cos^2\alpha-2n^2\cos\alpha\nonumber\\
	\mbox{i.e., }&&\cos^2\alpha>1,\mbox{ which is impossible}\nonumber
	\end{eqnarray}

Hence $\theta_1<\pi-\alpha$.\\

Let $R=0$ at $\theta=\theta_2$, given by
$$\cos\theta_2=\frac{2}{3}\cos\alpha-\frac{4}{3}n^2$$

The particle will leave the surface if $\theta_2<\theta_1$ i.e., $\cos\theta_2>\cos\theta_1$
\begin{eqnarray}
	\implies&&\frac{2}{3}\cos\alpha-\frac{4}{3}n^2>-n^2+\sqrt{n^4+1-2n^2\cos\alpha}\nonumber\\
	\mbox{i.e., }&&2\cos\alpha-n^2>3\sqrt{n^4+1-2n^2\cos\alpha}\nonumber\\
		\mbox{i.e., }&&4\cos^2\alpha+n^4-4n^2\cos\alpha>9\left(n^4+1-2n^2\cos\alpha\right)\nonumber\\
		\mbox{i.e. }&&8n^4-14n^2\cos\alpha-4\cos^2\alpha+9<0\nonumber\\
		\mbox{i.e. }&&\left(4n^2-\frac{7}{2}\cos\alpha\right)^2<\frac{81-81\sin^2\alpha-72}{4}=\frac{9}{4}\left(1-9\sin^2\alpha\right)\label{eq3.3.30}
\end{eqnarray}

The inequality will be true provided $1-9\sin^2\alpha>0$ i.e., $\sin\alpha<\dfrac{1}{3}$.\\

So, if $\sin\alpha<\dfrac{1}{3}$ then $\left|4n^2-\dfrac{7}{2}\cos\alpha\right|<\dfrac{3}{2}\sqrt{1-9\sin^2\alpha}$\\

i.e., $\left|2\dfrac{V^2}{ag}-7\cos\alpha\right|<3\sqrt{1-9\sin^2\alpha}$\\

If $\sin\alpha>\dfrac{1}{3}$, then inequality (\ref{eq3.3.30}) can not be satisfied by any real $v$. This means that the particle will not leave the surface if $\sin\alpha>\dfrac{1}{3}$.\\

{\bf3. } A heavy particle $P$ moves on the smooth inner surface of a fixed sphere of centre $O$ and the angle between OP and downward vertical is denoted by $\theta$. The particle is projected horizontally on the surface from a point at which $\theta=\alpha\left(<\dfrac{\pi}{2}\right)$. Prove that whatever be the velocity of projection $\theta$ will not exceed $(\pi-\alpha)$ in the subsequent motion and that if $\sin\alpha>\dfrac{1}{3}$ the particle will not leave the surface.\\

{\bf Solution: } Similar to the previous problem.\\

{\bf 4. } A particle is projected with a velocity $V$ along the inside of a smooth sphere of radius $a$ along a $\parallel$ whose latitude is $\lambda$. It is repealed from the point upon the sphere whose latitude is $90^o$ by a force per unit mass equal to $\mu$ times the distance from that point. Find the rate of increase of latitude and longitude of the particle when its latitude is $\theta$ and show that its path cuts the equator at 
$$\tan^{-1}\left\{\tan\lambda\left(1+\frac{2\mu a^2}{V^2}\mbox{cosec}\lambda\right)^\frac{1}{2}\right\}$$

{\bf Solution: } Let $O'P=r'$; so that force is $\mu r'$. If $\psi$ be the potential of the force then 
$$-\dfrac{\partial\psi}{\partial r'}=\mu r'\implies\psi=-\frac{1}{2}\mu r'^2+c$$

\begin{wrapfigure}[6]{r}{0.35\textwidth}
	\centering	\includegraphics[height=4 cm , width=4 cm ]{f24.pdf}
	\begin{center}
		Fig. 3.21
	\end{center}
\end{wrapfigure}
 
 Let $\theta$ and $\phi$ be the co-latitude and longitude of the particle.\\
 
 From the energy equation we get $\dfrac{1}{2}v^2+\psi=$ constant, where $ v$  is the velocity of the particle at $P$.\\
 
 Thus we have, $\dfrac{1}{2}v^2-\dfrac{1}{2}\mu r'^2=$ constant.\\
 
 But $r'^2=a^2+a^2-2a^2\cos\theta=2a^2(1-\cos\theta)$.\\

Hence we get,
\begin{eqnarray}
	\dfrac{1}{2}v^2+\mu a^2\cos\theta&=&\mbox{constant=Initial value}\nonumber\\
	&=&\dfrac{1}{2}V^2+\mu a^2\sin\lambda\nonumber\\
	\therefore ~v^2~=~V^2&+&2\mu a^2(\sin\lambda-\cos\theta)\nonumber
\end{eqnarray}

As there is no force perpendicular to $ZOP$, so angular momentum is constant, i.e.,
\begin{eqnarray}
	a^2\sin^2\theta\dot{\phi}&=&\mbox{constant}=Va\cos\lambda\nonumber\\
	\mbox{i.e., }a\sin^2\theta\dot{\phi}&=&V\cos\lambda\nonumber
\end{eqnarray}

This gives the rate of increase of longitude when $\theta$ is known.\\

As,
\begin{eqnarray}
	v^2&=&a^2\dot{\theta}^2+a^2\sin^2\theta\dot{\phi}^2=a^2\dot{\theta}^2+\frac{V^2\cos^2\lambda}{\sin^2\theta}\nonumber\\
	&=&V^2+2\mu a^2(\sin\lambda-\cos\theta)\nonumber\\
	\therefore~a^2\dot{\theta}^2&=&V^2\left(1-\frac{\cos^2\lambda}{\sin^2\theta}\right)+2\mu a^2(\sin\lambda-\cos\theta)\nonumber
\end{eqnarray}

This gives the rate of increase of co-latitude of the particle and hence the latitude of the particle.\\

At the equator, $\theta=\dfrac{\pi}{2}$, so $a\dot{\phi}=V\cos\lambda$ and 
\begin{eqnarray}
	a^2\dot{\theta}^2&=&V^2\sin^2\lambda+2\mu a^2\sin\lambda\nonumber\\&=&V^2\sin^2\lambda\left\{1+\frac{2\mu a^2}{V^2}\mbox{cosec}\lambda\right\}\nonumber\\\therefore~\dfrac{\dot{\theta}^2}{\dot{\phi}^2}&=&\tan^2\lambda\left\{1+\frac{2\mu a^2}{V^2}\mbox{cosec}\lambda\right\}\nonumber
\end{eqnarray}

If the path cuts the equator at an angle $\alpha$, we have,
\begin{eqnarray}
	V\cos\alpha=a\dot{\phi},~V\sin\alpha=a\dot{\theta}~~(\because\theta=\frac{\pi}{2}\mbox{ at the equator})\nonumber\\
	\therefore~\dfrac{\dot{\theta}}{\dot{\phi}}=\tan\alpha=\tan\lambda\left\{1+\frac{2\mu a^2}{V^2}\mbox{cosec}\lambda\right\}^\frac{1}{2}\nonumber
\end{eqnarray}

{\bf 5. } A heavy particle is projected horizontally along the inner surface of a smooth sphere with a velocity due to a fall from the level of the centre to the point of projection. Show that the radius of curvature of its path when it is at an angular distance $\theta$ from the lowest point of the sphere is $\dfrac{a}{\left(1+\frac{1}{4}\sin^2\alpha\cos\alpha\sec^3\theta\right)^\frac{1}{2}}$, where $\alpha$ is the initial value of $\theta$ and $a$ is the radius of the sphere.\\

{\bf Solution: } Let $P(a,\theta,\phi)$ be the position of the particle at time $t$ and $v$ be the velocity at $P$.
$$\therefore~v^2=a^2\dot{\theta}^2+a^2\sin^2\theta\dot{\phi}^2$$

\begin{wrapfigure}[12]{r}{0.35\textwidth}
	\centering	\includegraphics[height=4 cm , width=4 cm ]{f25.pdf}
	\begin{center}
		Fig. 3.22
	\end{center}
\end{wrapfigure}

As there are no external forces perpendicular to the $ZOP$ plane so
\begin{eqnarray}
	a^2\sin^2\theta\dot{\phi}=\mbox{constant}=\mbox{initial value}=Va\sin\alpha\nonumber\\
	\therefore~a\sin\theta\dot{\phi}=\mbox{horizontal velocity}=u=\frac{V\sin\alpha}{\sin\theta}\nonumber
\end{eqnarray}

The energy conservation equation gives 
$$v^2=2ag\cos\theta+c$$

Initially, $V^2=2ag\cos\alpha\implies c=0$
$$\therefore v^2=2ag\cos\theta$$

Also we have the equation of motion
$$m\frac{v^2}{\rho_0}\tan\chi=H,~~\rho_0=a\mbox{ in the present problem}$$

Let the tangent to the path at $P$ makes an angle $\psi$ with the meridian curve so
$$H=mg\sin\theta\sin\psi$$

Also $v\sin\psi=u=\sqrt{2ag\cos\alpha}\dfrac{\sin\alpha}{\sin\theta}$.
\begin{eqnarray}
	\therefore ~ m\frac{v^2}{a}\tan\chi&=&mg\sin\theta\frac{u}{v}\nonumber\\\therefore~\tan\chi&=&\frac{ag\sin\theta\sqrt{2ag\cos\alpha}}{2ag\cos\theta\sqrt{2ag\cos\theta}}=\frac{\sqrt{\sin^2\alpha\cos\alpha}}{\sqrt{4\cos^3\theta}}\nonumber\\\therefore~\frac{\sin\chi}{\sqrt{\sin^2\alpha\cos\alpha}}&=&\frac{\cos\chi}{\sqrt{4\cos^3\theta}}=\frac{1}{\sqrt{\sin^2\alpha\cos\alpha+4\cos^3\theta}}\nonumber\\\therefore~\cos\chi&=&\frac{1}{\sqrt{1+\frac{1}{4}\sin^2\alpha\cos\alpha\sec^3\theta}}\nonumber\\\therefore~\rho&=&\mbox{radius of curvature of the path at }P=\rho_0\cos\chi=a\cos\chi\nonumber\\&=&\frac{a}{\left(1+\frac{1}{4}\sin^2\alpha\cos\alpha\sec^3\theta\right)^\frac{1}{2}}\nonumber
\end{eqnarray}

{\bf 6. } A particle tied to a fixed point by an inextensible string is projected under gravity in any manner so that the string remains taut in the subsequent motion. Find the differential equation of the projection of the path on a horizontal plane and prove that the pedal equation of this curve is of the form
$$\frac{l^2}{p^2}=1+A\left(l^2-r^2\right)+B\left(l^2-r^2\right)^\frac{3}{2}$$ 

{\bf Solution: } Sine the string is taut throughout so the particle moves over a sphere. So the meridian section through the particle is a great circle:
\begin{equation}\label{eq3.6.1}
	z^2+r^2=l^2
\end{equation}

The differential equation of the projection of the path is
\begin{equation}
	\frac{\left(1+\{\phi'\}^2\right)}{r^4}\left(\frac{\mathrm{d}r}{\mathrm{d}\theta}\right)^2=\frac{c^2}{h^2}-\frac{2g}{h^2}z-\frac{1}{r^2}\nonumber
\end{equation}

Now, $\dot{\phi}=\dfrac{\mathrm{d}z}{\mathrm{d}r}=-\dfrac{r}{z}$ (using (\ref{eq3.6.1}))
\begin{eqnarray}
	\therefore ~\frac{l^2}{r^4z^2}\left(\frac{\mathrm{d}r}{\mathrm{d}\theta}\right)^2=\frac{c^2}{h^2}-\frac{2g}{h^2}z-\frac{1}{r^2}\nonumber\\\implies\frac{1}{r^2}\left(\frac{\mathrm{d}r}{\mathrm{d}\theta}\right)^2=\frac{c^2z^2r^2}{l^2h^2}-\frac{2gz^3r^2}{h^2l^2}-\frac{z^2}{l^2}\nonumber
\end{eqnarray}

As $p=r\sin\phi,~\tan\phi=r\dfrac{\mathrm{d}\theta}{\mathrm{d}r}$,
So,\begin{eqnarray}
	\frac{r^2}{p^2}&=&1+\frac{1}{r^2}\left(\frac{\mathrm{d}r}{\mathrm{d}\theta}\right)^2=\left(1-\frac{z^2}{l^2}\right)+\frac{c^2z^2r^2}{l^2h^2}-\frac{2gz^3r^2}{h^2l^2}\nonumber\\&=&\frac{r^2}{l^2}+\frac{c^2z^2r^2}{l^2h^2}-\frac{2gz^3r^2}{h^2l^2}\nonumber\\\implies\frac{l^2}{p^2}&=&1+A\left(l^2-r^2\right)+B\left(l^2-r^2\right)^\frac{3}{2}\nonumber
\end{eqnarray}

{\bf 7. } A heavy particle is projected horizontally with velocity $u$ along the inner surface of a smooth vertical circular cylinder of radius $a$. Show that the radius of curvature at any point of the path is $\dfrac{av^3}{u\sqrt{u^2v^2+a^2g^2}}$, $v$ being the velocity at the point.\\

{\bf Solution: } Let $\psi$ be the angle that the path of the particle makes with the vertical generator through $P$. Suppose $\rho_1$ and $\rho_2$ are the principal radius of curvature. Here $\rho_1=\infty$, $\rho_2=a$. Suppose $\rho_0$ be the radius of curvature of the normal section of the cylinder passing through that tangent to the curve at $P$. So we have,
$$\frac{1}{\rho_0}=\frac{\cos^2\psi}{\rho_1}+\frac{\sin^2\psi}{\rho_2}\implies\rho_0=a\mbox{cosec}^2\psi$$
\begin{wrapfigure}[12]{r}{0.35\textwidth}
	\centering	\includegraphics[height=4.5 cm , width=4 cm ]{f26.pdf}
	\begin{center}
		Fig. 3.23
	\end{center}
\end{wrapfigure}

Since the only force is the weight $mg$, so the horizontal component of velocity remains constant.
$$\therefore~v\sin\psi=u\implies\mbox{cosec}\psi=\frac{v}{u}$$

If $H$ be the component of the force along the tangent to the surface perpendicular to the tangent to the path then,
\begin{eqnarray}
	H&=&mg\sin\psi\nonumber\\\therefore~m\frac{v^2}{\rho_0}\tan\chi&=&mg\sin\psi\implies\tan\chi=\frac{ag}{uv}\nonumber\\\therefore~\frac{\sin\chi}{ag}&=&\frac{\cos\chi}{uv}=\frac{1}{\sqrt{a^2g^2+u^2v^2}}\nonumber\\\therefore~\rho&=&\mbox{radius of curvature of the path of the particle at }P\nonumber\\&=&\rho_0\cos\chi=a\mbox{cosec}^2\psi\cos\chi=\frac{av^3}{u\sqrt{u^2v^2+a^2g^2}}\nonumber
\end{eqnarray}

{\bf 8. } A heavy particle of mass $m$ moves on the smooth inner surface of a sphere of radius $a$ and its greatest and least depths below the centre are $\dfrac{a}{2}$ and $\dfrac{a}{4}$ respectively. Show that when the deep below the centre is $z$, the normal reaction of the sphere is $\dfrac{3mg}{a}\left(z+\frac{a}{2}\right)$. Also show that the time from maximum to minimum depth is $\sqrt{\dfrac{a}{g}}\int\limits_0^\frac{\pi}{2}\dfrac{\mathrm{d}\phi}{\sqrt{1-\frac{1}{8}\sin^2\phi}}$\\
\begin{wrapfigure}[12]{r}{0.35\textwidth}
	\centering	\includegraphics[height=5 cm , width=4 cm ]{f27.pdf}
	\begin{center}
		Fig. 3.24
	\end{center}
\end{wrapfigure}
{\bf Solution: } Let $v_1$ and $v_2$ be the horizontal velocities at the points $z=\dfrac{a}{2}$ and $z=\dfrac{a}{4}$. Then from the energy equation we have,
\begin{eqnarray}
	v^2&=&c+2gz\nonumber\\\therefore~v_1^2=c+2g\frac{a}{2}&,&v_2^2=c+2g\frac{a}{4}\nonumber\\\therefore~v_1^2-v_2^2&=&\frac{ag}{2}
\end{eqnarray} 

The equation of angular momentum gives
\begin{eqnarray}
	a^2\sin^2\theta\dot{\phi}=\mbox{constant}&=&v_1a\frac{\sqrt{3}}{2}=v_2a\frac{\sqrt{15}}{4}\nonumber\\\therefore~v_1=v_2\frac{\sqrt{5}}{2}\nonumber&&
\end{eqnarray} 

Thus, $v_2^2=2ag$, $v_1^2=\dfrac{5ag}{2}$ and $c=\dfrac{3}{2}ag$.
$$\therefore~h=	a^2\sin^2\theta\dot{\phi}=v_1a\frac{\sqrt{3}}{2}=\sqrt{\dfrac{5ag}{2}}a\frac{\sqrt{3}}{2}=\sqrt{\frac{15}{8}a^3g}$$ 

Now,
\begin{eqnarray}
	m\frac{v^2}{a}=R-mg\cos\theta&&\implies R=mg\left(\cos\theta+\frac{v^2}{ag}\right)\nonumber\\\therefore~R= mg\left(\frac{z}{a}+\frac{c}{ag}+\frac{2z}{a}\right)&=&mg\left(\frac{3z}{a}+\frac{3}{2}\right)=3\frac{mg}{a}\left(z+\frac{a}{2}\right)\nonumber
\end{eqnarray}

Also
\begin{eqnarray}
	v^2&=&a^2\dot{\theta}^2+a^2\sin^2\theta\dot{\phi}^2=2gz+c=2gz+\frac{3}{2}ag\nonumber\\\therefore&&a^2\dot{\theta}^2+a^2\sin^2\theta\cdot\frac{15}{8}\frac{a^3g}{a^4\sin^4\theta}=2gz+\frac{3}{2}ag\nonumber\\\therefore&&a^2\sin^2\theta\dot{\theta}^2+\frac{15}{8}ag=\left(1-\cos^2\theta\right)\left(2gz+\frac{3}{2}ag\right)\nonumber\\\therefore~\dot{z}^2&=&\frac{2g}{a^2}(a^2-z^2)\left(z+\frac{3a}{4}\right)-\frac{15}{8}ag\nonumber\\&=&\frac{2g}{a^2}\left[a^2z-z^3-\frac{3}{4}z^2a-\frac{3}{16}a^3\right]\nonumber
\end{eqnarray}

Since maximum and minimum depths are $\dfrac{a}{2}$ and $\dfrac{a}{4}$, so $\left(\dfrac{a}{2}-z\right)$ and $\left(z-\dfrac{a}{4}\right)$ are two factors. The remaining factor be $\left(z+\dfrac{3a}{2}\right)$. $$\dot{z}^2=\frac{2g}{a^2}\left[\left(\dfrac{a}{2}-z\right)\left(z-\dfrac{a}{4}\right)\left(z+\dfrac{3a}{2}\right)\right]$$

If $T$ the required time, then 
\begin{eqnarray}
	T&=&-\sqrt{\frac{a^2}{2g}}\int\limits_\frac{a}{2}^\frac{a}{4}\frac{\mathrm{d}z}{\sqrt{\left(\dfrac{a}{2}-z\right)\left(z-\dfrac{a}{4}\right)\left(z+\dfrac{3a}{2}\right)}}\nonumber\\
&&	\mbox{(Note that $\dot{z}<0$ in computing time from maximum to minimum depth)}\nonumber
\end{eqnarray}

Let us substitute
\begin{eqnarray}
z&=&\frac{a}{4}\cos^2\theta+\frac{a}{2}\sin^2\theta,~\mbox{i.e., }~\mathrm{d}z=\frac{a}{2}\sin\theta\cos\theta\mathrm{d}\theta\nonumber\\\therefore~T&=&\sqrt{\frac{a^2}{2g}}\int\limits_0^\frac{\pi}{2}\frac{\frac{a}{2}\sin\theta\cos\theta\mathrm{d}\theta}{\sqrt{\frac{a}{4}\cos^2\theta\cdot\frac{a}{4}\sin^2\theta\left(\frac{a}{4}\cos^2\theta+\frac{a}{2}\sin^2\theta+\frac{3a}{2}\right)}}\nonumber\\&=&\sqrt{\frac{a^2}{2g}}\int\limits_0^\frac{\pi}{2}\frac{\mathrm{d}\theta}{\sqrt{\frac{a}{16}\left(\cos^2\theta+2\sin^2\theta+6\right)}}\nonumber\\&=&\sqrt{\frac{a^2}{2g}}\int\limits_0^\frac{\pi}{2}\frac{\mathrm{d}\theta}{\sqrt{\frac{a}{16}\left(8-\cos^2\theta\right)}}=\sqrt{\dfrac{a}{g}}\int\limits_0^\frac{\pi}{2}\dfrac{\mathrm{d}\phi}{\sqrt{1-\frac{1}{8}\sin^2\phi}}\mbox{ (Substituting }\phi=\frac{\pi}{2}-\theta)\nonumber
\end{eqnarray}

{\bf 9. } A heavy particle moves on the surface of a paraboloid of revolution whose axis is vertical and vertex downwards being projected horizontally with velocity $\sqrt{2gh_2}$ at a height $h_1$ above the vertex. Show that the orbit touches alternatively the two horizontal circles at heights $h_1$, $h_2$ above the vertex and that the least value of the angle at which the orbit cuts the Meridian is $\tan^{-1}\dfrac{2\sqrt{h_1h_2}}{\left|h_1-h_2\right|}$\\

{\bf Solution: } In cylindrical co-ordinate system the equation of the Meridian section through the particle be $$r^2=2az$$ 

Also the equation of energy gives
\begin{eqnarray}
	v^2&=&c-2gz\nonumber\\\therefore~2gh_2&=&c-2gh_1\implies c=2h\left(h_1+h_2\right)\nonumber\\\therefore~v^2&=&2g\left(h_1+h_2-z\right)\nonumber
\end{eqnarray} 

Equation of angular momentum gives
\begin{eqnarray}
	r^2\dot{\theta}&=&h=\sqrt{2gh_2\cdot4ah_1}\nonumber\\\therefore v^2&=&\dot{r}^2+\dot{z}^2+r^2\dot{\theta}^2\nonumber\\&=&\frac{4a^2}{r^2}\dot{z}^2+\dot{z}^2+\frac{8agh_1h_2}{r^2}=2g\left(h_1+h_2-z\right)\nonumber\\\therefore\dot{z}^2\left(\frac{a}{z}+1\right)&=&2g\left(h_1+h_2-z\right)\frac{2gh_1h_2}{z}\nonumber\\\implies(a+z)\dot{z}^2&=&2g\left[\left(h_1+h_2\right)z-z^2-h_1h_2\right]=2g\left[\left(h_1-z\right)\left(z-h_2\right)\right]\nonumber
\end{eqnarray}
i.e., $\dot{z}$ vanishes  at $z=h_1$ and $z=h_2$.\\

Now, $r\dot{\theta}=\dfrac{8agh_1h_2}{4az}=\dfrac{2gh_1h_2}{z}=$ horizontal velocity.\\

If the path cuts the Meridian at an angle $\psi$ then,
$$r\dot{\theta}=v\sin\psi\implies\sin\psi=\sqrt{\frac{h_1h_2}{z\left(h_1+h_2-z\right)}}$$

So for least value of $\psi$, $z\left(h_1+h_2-z\right)$ should have a maximum. \\

This occurs when $z=\dfrac{h_1+h_2}{2}$, and then $\sin\psi=\dfrac{2\sqrt{h_1h_2}}{h_1+h_2}$.
$$\therefore~\cos\psi=\dfrac{\left|h_1-h_2\right|}{\left(h_1+h_2\right)} \mbox{~ i.e., ~}\tan\psi=\dfrac{2\sqrt{h_1h_2}}{\left|h_1-h_2\right|}$$

{\bf 10. } A particle is projected horizontally along the inner surface of a smooth cone whose access is vertical and vertex downwards. Find the pressure at any point in terms of the depth below the vertex and show that the particle leaves the cone at a depth $\left(\dfrac{V^2h^2}{g\tan^2\alpha}\right)^\frac{1}{3},$ where $h$ is the initial depth, $V$ is the initial velocity and $\alpha$ is the semi-vertical angle of the cone. Find also the radius of curvature at any point of the path.\\

{\bf Solution: } Using the cylindrical coordinates with $z$-axis vertically downward, let $P(r,\theta,z)$ be the position of the particle at any time $T$. If $v$ be velocity of the particle at $P$ then equation of energy gives $$v^2=c+2gz$$ 

\begin{wrapfigure}[13]{r}{0.35\textwidth}
	\centering	\includegraphics[height=5 cm , width=5 cm ]{f28.pdf}
	\begin{center}
		Fig. 3.25
	\end{center}
\end{wrapfigure}

From the initial condition
\begin{eqnarray}
	V^2=c+2gh\nonumber\\\therefore~v^2=V^2+2g(z-h)\nonumber
\end{eqnarray}

Equation of angular momentum gives
\begin{eqnarray}
	r^2\dot{\theta}&=&\mbox{constant}=Vh\tan\alpha\nonumber\\\therefore~r\dot{\theta}&=&\frac{Vh\tan\alpha}{r}=\frac{Vh\tan\alpha}{z\tan\alpha}=\frac{Vh}{z}=u \mbox{ (say)}\nonumber
\end{eqnarray}  

Resolving along the normal to the surface
$$m\frac{v^2}{\rho_0}=R+mg\sin\alpha$$ where $\dfrac{1}{\rho_0}=\dfrac{\cos^2\psi}{\rho_1}+\dfrac{\sin^2\psi}{\rho_2}$ with $\rho_1=\infty$, $\rho_2=PG$.\\

Here $\psi$ is the angle which the tangent to the path at $P$ makes with the generator of the cone through P.
\begin{eqnarray}
	\therefore~\frac{1}{\rho_0}=\frac{\sin^2\psi}{PG}&=&\frac{\sin^2\psi}{z\tan\alpha\sec\alpha}~~(\because PG\cos\alpha=r=z\tan\alpha)\nonumber\\\therefore~\frac{mv^2\sin^2\psi}{z\tan\alpha\sec\alpha}&=&R+mg\sin\alpha\nonumber
\end{eqnarray} 

Again, $v\sin\psi$= horizontal component of velocity= $u=V\dfrac{h}{z}$.
$$\therefore~R=\frac{mV^2h^2}{z^3\tan\alpha\sec\alpha}-mg\sin\alpha=\frac{mg}{\tan\alpha\sec\alpha}\left[\frac{V^2h^2}{z^3g}-\tan^2\alpha\right].$$ 

So R vanishes at a depth $z$ given by
\begin{equation}
	z=\left(\dfrac{V^2h^2}{g\tan^2\alpha}\right)^\frac{1}{3}
\end{equation}

Again $v^2=\dot{r}^2+r^2\dot{\theta}^2+\dot{z}^2=V^2+2g(z-h)$.
\begin{eqnarray}
	\implies \dot{z}^2\sec^2\alpha=-\frac{V^2h^2}{z^2}+V^2+2g(z-h)\nonumber\\=(z-h)\left[\frac{V^2}{z^2}(z+h)+2g\right]\nonumber
\end{eqnarray}

So for real value of $\dot{z}$ we should have $z>h$.\\

Thus the particle will leave the surface of the cone at a depth $z$, provided $z>h$ i.e., $\left(\dfrac{V^2h^2}{g\tan^2\alpha}\right)^\frac{1}{3}>h$ i.e., $\dfrac{V^2}{g\tan^2\alpha}>h$.\\

Now from the expression of $R$, we have if $R_0$ be the reaction of the surface at the initial position, then $R_0>0$, which implies $\dfrac{V^2}{gh}-\tan^2\alpha>0$ i.e., $\dfrac{V^2}{g\tan^2\alpha}>h$.\\

Hence the particle will leave the surface.\\

Also we have
\begin{eqnarray}
	m\frac{v^2}{\rho_0}\tan\chi&=&H=mg\cos\alpha\sin\psi\nonumber\\\mbox{i.e., }\frac{mv^2\sin^2\psi\cos\alpha}{z\tan\alpha}\tan\chi&=&mg\cos\alpha\sin\psi\nonumber\\\mbox{i.e., }\frac{u}{r}\tan\chi&=&\frac{g}{v}\nonumber\\\mbox{i.e., }\tan\chi&=&\frac{gr}{uv}\nonumber\\\therefore~\cos\chi&=&\frac{uv}{\sqrt{g^2r^2+u^2v^2}}\nonumber\\\therefore~\rho=\rho_0\cos\chi&=&\frac{r}{\frac{u^2}{v^2}\cos\alpha}\frac{uv}{\sqrt{g^2r^2+u^2v^2}}=\frac{r v^3\sec\alpha}{u\sqrt{g^2r^2+u^2v^2}}\nonumber
\end{eqnarray}

{\bf 11. } A heavy particle is projected with velocity $V$ from one end of the horizontal diameter of a smooth sphere of radius $a$ along the inner surface, the direction of projection making an angle $\beta$ with the equator. If the particle never leaves the surface, prove that $$3\sin^2\beta<2+\left(\frac{V^2}{3ag}\right)^2$$ 

{\bf Solution: } The equation of energy gives $v^2=c+2gz$. As $v=V$ at $z=0$, so $c=V^2$ and we have
\begin{equation}\label{eq3.11.1}
	v^2=V^2+2gz
\end{equation}

The conservation of angular momentum gives
\begin{equation}
	a^2\sin^2\theta\dot{\phi}=Va\cos\beta
\end{equation}

The equation of motion along the normal gives
$$m\frac{v^2}{a}=R-mg\cos\theta\mbox{ ~i.e., ~}R=mg\left[\frac{v^2}{ag}+\cos\theta\right]$$
 or using (\ref{eq3.11.1}) we have $$R=\frac{mg}{a}\left[\frac{V^2}{g}+3z\right]$$ 
 
 So $R=0$ when $z=-\dfrac{V^2}{3g}$.\\
 
 Now, \begin{eqnarray}
 	v^2=a^2\sin^2\theta\dot{\phi}^2+a^2\dot{\theta}^2=\frac{V^2\cos^2\beta}{\sin^2\theta}+a^2\dot{\theta}^2=V^2+2gz\nonumber\\\therefore~\dot{z}^2=\left(V^2+2gz\right)\frac{(a^2-z^2)}{a^2}-V^2\cos^2\beta,~~~~~~~~~~~z=a\cos\theta\nonumber
 \end{eqnarray}

So the particle will never leave the surface if
\begin{eqnarray}
	&&\dot{z}^2<0\mbox{ at }z=-\frac{V^2}{3g}\nonumber\\\mbox{i.e., }&&\left(V^2-\frac{2}{3}V^2\right)\left\{a^2-\left(\frac{V^2}{3g}\right)^2\right\}<a^2V^2\cos^2\beta\nonumber\\\mbox{i.e., }&&a^2-\left(\frac{V^2}{3g}\right)^2<3a^2\cos^2\beta=3a^2\left(1-\sin^2\beta\right)\nonumber\\\mbox{i.e., }&&1-\left(\frac{V^2}{3ag}\right)^2<3-3\sin^2\beta\nonumber\\\mbox{i.e., }&&3\sin^2\beta<2+\left(\frac{V^2}{3ag}\right)^2\nonumber
\end{eqnarray} 

{\bf 12. } A particle is projected horizontally under gravity with a velocity $V$ from a point on the inner surface of a smooth sphere at an angular distance $\alpha$ from the lowest point. Show that the $z$ co-ordinate of the highest point of the path of the particle on the surface is the smaller of the values of $z_1$ and $z_2$ given by the equation $z_1^2+\dfrac{V^2}{ag}(z_1+z_0)-a^2=0$ and $3z_2-2z_0+\dfrac{V^2}{g}=0$ where $z_0=a\cos\alpha$.\\

{\bf Solution: } The equation of energy gives
\begin{eqnarray}
	v^2&=&c+2gz\nonumber\\\therefore~V^2&=&c+2ag\cos\alpha\nonumber\\\therefore~v^2&=&V^2+2ag(\cos\theta-\cos\alpha)=V^2+2g(z-z_0)\nonumber
\end{eqnarray}

\begin{wrapfigure}[12]{r}{0.35\textwidth}
	\centering	\includegraphics[height=5 cm , width=4 cm ]{f29.pdf}
	\begin{center}
		Fig. 3.26
	\end{center}
\end{wrapfigure} 

The conservation of angular momentum gives
$$a^2\sin^2\theta\dot{\phi}=Va\sin\alpha$$

The equation of motion along the normal gives
\begin{eqnarray}
	m\frac{v^2}{a}&=&R-mg\cos\theta\nonumber\\\therefore~R&=&m\left\{\frac{V^2+2g(z-z_0)}{a}+\frac{gz}{a}\right\}=\frac{mg}{a}\left[\frac{V^2}{g}+3z-2z_0\right]\nonumber
\end{eqnarray}

So $R=0$ at a height $z=z_2$ given by
\begin{equation}
	\frac{V^2}{g}+3z_2-2z_0=0
\end{equation} 

Again,
 \begin{eqnarray}
	v^2&=&a^2\sin^2\theta\dot{\phi}^2+a^2\dot{\theta}^2=\frac{V^2\sin^2\alpha}{\sin^2\theta}+a^2\dot{\theta}^2=V^2+2g(z-z_0)\nonumber\\&=&V^2\frac{(z_0^2-z^2)}{a^2}+2g(z-z_0)\frac{(a^2-z^2)}{a^2}\nonumber\\\therefore~\dot{z}^2&=&\frac{2g}{z}(z-z_0)\left[a^2-z^2-\frac{V^2}{2g}(z+z_0)\right]\nonumber
\end{eqnarray}


Now, $\dot{z}=0$ when $z=z_0$ initially and when
\begin{equation}
	a^2-z_1^2-\frac{V^2}{2g}(z_1+z_0)=0
\end{equation}
so the particle will rise to the smaller of $z_1$ and $z_2$.\\

{\bf 13. } A particle is projected horizontally along the inner surface of a smooth hemisphere whose axis is vertical and whose vertex is downwards, the point of projection being at an angular distance $\beta$ from the lowest point. Show that the initial velocity so that the particle may just ascend to the rim of the hemisphere is $\sqrt{2ag\sec\beta}$.\\

{\bf Solution: } Let $V$ be the velocity of projection. The energy equation gives 
$$v^2=c+2gz=V^2+2g(z-a\cos\beta)$$.

The angular momentum conservation equation gives $$a^2\sin^2\theta\dot{\phi}=Va\sin\beta$$

So, \begin{eqnarray}
	v^2=a^2\sin^2\theta\dot{\phi}^2+a^2\dot{\theta}^2&=&\frac{V^2\sin^2\beta}{\sin^2\theta}+a^2\dot{\theta}^2=V^2+2g(z-a\cos\beta)\nonumber\\\therefore~\dot{z}^2+V^2\sin^2\beta&=&\frac{(a^2-z^2)}{a^2}\left[V^2+2g(z-a\cos\beta)\right], ~~~~~z=a\cos\theta\nonumber
\end{eqnarray}

As $\dot{z}=0$ when $z=0$, so $V^2\sin^2\beta=V^2-2ga\cos\beta$. $$V^2=2ag\sec\beta$$.\vspace*{-.7cm}
\begin{eqnarray}
	\therefore~R&=&m\left(\frac{v^2}{a}+g\frac{z}{a}\right)=m\left[\frac{V^2}{a}+\frac{2g}{a}(z-a\cos\beta)+\frac{gz}{a}\right]\nonumber\\&=&m\left[2g(\sec\beta-\cos\beta)+\frac{3gz}{a}\right]\nonumber\\&=&m\left[2g\sec\beta\sin^2\beta+\frac{3gz}{a}\right]>0,\mbox{ in the hemisphere.}\nonumber
\end{eqnarray}

So the particle will not leave the surface.\\

{\bf 14. } A particle moves on the interior of a smooth sphere of radius $a$ under a force producing an acceleration $\mu\omega^n$ along the perpendicular $\omega$ drawn to the fixed diameter. It is projected with velocity $V$ along the great circle to which this diameter is perpendicular and is slightly disturbed from its path. Show that the new path will cut the old one $m$ times in a revolution, where $$m^2=4\left[1-\frac{\mu a^{n+1}}{V^2}\right]$$

{\bf Solution: } We take the centre of the sphere as origin and the fixed diameter along the $z$-axis. Let $P(r,\theta,z)$ be the position of the particle in cylindrical coordinates. Suppose $v$ be the velocity of the particle at $P$ and $\psi$ is the potential of the force. The energy condition gives
\begin{equation}\label{eq3.14.1}
	\frac{1}{2}v^2+\psi=\mbox{constant}
\end{equation}

\begin{wrapfigure}[12]{r}{0.35\textwidth}
	\centering	\includegraphics[height=5 cm , width=4 cm ]{f30.pdf}
	\begin{center}
		Fig. 3.27
	\end{center}
\end{wrapfigure} 

Here $-\dfrac{\partial \psi}{\partial r}=-\mu r^n$ ($\because$ the force is in the negative direction)\\

$\mbox{i.e., }\psi=\dfrac{\mu r^{n+1}}{n+1}+\mbox{constant}$.\\

So from (\ref{eq3.14.1}) $\dfrac{1}{2}v^2+\dfrac{\mu r^{n+1}}{n+1}=\mbox{constant}=\dfrac{1}{2}V^2+\dfrac{\mu a^{n+1}}{n+1}$ $$\mbox{i.e.,} v^2=V^2+\frac{2\mu}{n+1}\left(a^{n+1}-r^{n+1}\right)$$

The equation of the sphere in cylindrical coordinate is
$$x^2+y^2+z^2=a^2\mbox{ i.e., }r^2+z^2=a^2$$ with, $x=r\cos\theta$, $y=r\sin\theta$, $0\leq\theta\leq2\pi$.\\

Now differentiating with respect to $t$ we have
\begin{eqnarray}
	r\dot{r}+z\dot{z}&=&0\mbox{ ~i.e.,~ }\dot{r}^2=\frac{z^2\dot{z}^2}{r^2}\nonumber\\
	\therefore~v^2=\dot{r}^2+r^2\dot{\theta}^2+\dot{z}^2&=&\dot{z}^2\left(1+\frac{z^2}{r^2}\right)+r^2\dot{\theta}^2=\frac{a^2}{r^2}\dot{z}^2+r^2\dot{\theta}^2\nonumber
\end{eqnarray} 

The conservation of angular momentum gives $r^2\dot{\theta}=Va$
\begin{eqnarray}
	\therefore~v^2&=&\frac{a^2}{r^2}\dot{z}^2+\frac{V^2a^2}{r^2}=V^2+\frac{2\mu}{n+1}\left(a^{n+1}-r^{n+1}\right)\nonumber\\\therefore~\dot{z}^2&=&-\frac{V^2z^2}{a^2}+\frac{2\mu}{a^2(n+1)}\left[a^{n+1}(a^2-z^2)-{(a^2-z^2)}^\frac{n+3}{2}\right]=f(z) \mbox{ (say)}\nonumber\\
	\mbox{i.e., }\ddot{z}&=&\frac{1}{2}f'(z)\nonumber
\end{eqnarray}

 To consider the disturbed motion of the particle at $z=0$ we put $z=0+\epsilon$.
 \begin{eqnarray}
 	\therefore~\ddot{z}=\ddot{\epsilon}&=&\frac{1}{2}f'(\epsilon)=\frac{1}{2}\left[f'(0)+\epsilon f''(0)+\cdots\right]\nonumber\\&\simeq&\frac{1}{2}\left[f'(0)+\epsilon f''(0)\right]
 \end{eqnarray}

Here it is assumed that $\epsilon$ is so small that $\epsilon^2$ and other higher powers of $\epsilon$ are neglected.\\

Now
\begin{eqnarray}
	f'(z)&=&-\frac{2V^2z}{a^2}+\frac{2\mu}{a^2(n+1)}\left[-2za^{n+1}+(n+3)z{\left(a^2-z^2\right)}^\frac{n+1}{2}\right]\nonumber\\\therefore~f''(z)&=&-\frac{2V^2}{a^2}+\frac{2\mu}{a^2(n+1)}\left[-2a^{n+1}+(n+3){\left(a^2-z^2\right)}^\frac{n+1}{2}-(n+3)(n+1)z^2{(a^2-z^2)}^\frac{n-1}{2}\right]\nonumber
\end{eqnarray}

So, $f'(0)=0$, $f''(0)=-2\dfrac{V^2}{a^2}+\dfrac{2\mu a^{n+1}}{a^2}$ \\

Thus $\ddot{\epsilon}=-\dfrac{1}{a^2}\left(V^2-\mu a^{n+1}\right)\epsilon=-K^2\epsilon$\\
where for stability $K^2=\dfrac{V^2-\mu a^{n+1}}{a^2}>0$.\\

Hence if $V^2>\mu a^{n+1}$, the motion is a simple harmonic motion, having period of oscillation $T=\dfrac{2\pi}{K}$.\\

Now, $r^2\dot{\theta}=Va$\\

 i.e., $\dot{\theta}=\dfrac{Va}{r^2}=\dfrac{Va}{a^2-z^2}\simeq\dfrac{Va}{a^2}=\dfrac{V}{a}$ (when $z$ very small).\\

So on integration, $$\theta=\frac{V}{a}t+c$$

Assuming $\theta=0$ at $t=0$ $\implies c=0$ and we have $\theta=\dfrac{V}{a}t$.\\

Let $\theta_1$ and $\theta_2$ be the values of $\theta$ where the new path of the particle cuts the old part at two consecutive points. If $t_1$ and $t_2$ be the corresponding times then
$$\theta_1-\theta_2=\frac{V}{a}(t_1-t_2)$$

Since the new path cuts the old path $m$ times, so we have 
\begin{eqnarray}
	\theta_1-\theta_2&=&\dfrac{2\pi}{m} \mbox{ ~and~ } t_1-t_2=\frac{T}{2}\nonumber\\\therefore~\frac{2\pi}{m}&=&\frac{V}{a}\frac{T}{2}=\frac{V}{a}\frac{\pi}{K}\nonumber\\\therefore~m&=&\dfrac{2aK}{V}\implies m^2=\frac{4a^2}{V^2}K^2=4\left(1-\frac{\mu a^{n+1}}{V^2}\right)\nonumber
\end{eqnarray}

{\bf 15. } A particle is moving under gravity in a horizontal circle on the inner surface of a smooth sphere. A slant disturbance is given to the particle keeping the angular momentum unaltered. Discuss the stability of the motion\\

{\bf Solution: } We take the centre of the sphere as origin and $z$-axis vertically download. Let $v$ the velocity of the particle at $P(a,\theta,\phi)$ (in spherical polar coordinate). Then the equation of energy gives $$v^2=c+2gz=c+2ag\cos\theta$$

\begin{wrapfigure}[11]{r}{0.35\textwidth}
	\centering	\includegraphics[height=4.5 cm , width=4 cm ]{f31.pdf}
	\begin{center}
		Fig. 3.28
	\end{center}
\end{wrapfigure}  

The equation of angular momentum gives
\begin{eqnarray}
	a^2\sin^2\theta\dot{\phi}&=&h\nonumber\\\therefore~v^2&=&a^2\dot{\theta}^2+a^2\sin^2\theta\dot{\phi}^2=a^2\dot{\theta}^2+\frac{h^2}{a^2\sin^2\theta}\nonumber\\\therefore~a^2\dot{\theta}^2&=&c-\frac{h^2}{a^2\sin^2\theta}+2ag\cos\theta\nonumber
\end{eqnarray}

Now differentiating both side with respect to $\theta$ gives
\begin{equation}\label{eq3.15.1}
	a^2\ddot{\theta}=\frac{h^2\cos\theta}{a^2\sin^3\theta}-ag\sin\theta
\end{equation}

If the particle describes a horizontal circle at $\theta=\alpha$ then $\dot{\theta}=0=\ddot{\theta}$ at $\theta=\alpha$. So from (\ref{eq3.15.1}),
\begin{equation}\label{eq3.15.2}
	0=\frac{h^2\cos\alpha}{a^2\sin^3\alpha}-ag\sin\alpha\mbox{~ i.e., ~}h^2\cos\alpha=a^3g\sin^4\alpha
\end{equation}

Let a slight disturbance is given to the particle so that its angular momentum remains unaltered. If $\theta$ differs form $\alpha$ by a small quantity $\xi$ where $\xi$ is small compared to $\alpha$, i.e., $\theta=\alpha+\xi$, then from (\ref{eq3.15.1})
\begin{eqnarray}
	a^2\ddot{\xi}&=&\frac{h^2\cos(\alpha+\xi)}{a^2\sin^3(\alpha+\xi)}-ag\sin(\alpha+\xi)\nonumber\\&=&\frac{h^2}{a^2}\frac{(\cos\alpha-\xi\sin\alpha)}{(\sin\alpha+\xi\cos\alpha)^3}-ag(\sin\alpha+\xi\cos\alpha)\nonumber\\&&\mbox{~~~~~~~ (since $\xi$ is small so $\cos\xi\approx1$ and $\sin\xi\approx\xi$)}\nonumber\\&=&\frac{h^2}{a^2\sin^3\alpha}(\cos\alpha-\xi\sin\alpha)(1+\xi\cot\alpha)^{-3}-ag(\sin\alpha+\xi\cos\alpha)\nonumber\\&=&\frac{a^3g\sin^4\alpha}{a^2\sin^3\alpha\cos\alpha}(\cos\alpha-\xi\sin\alpha)(1-3\xi\cot\alpha)-ag(\sin\alpha+\xi\cos\alpha)\nonumber\\&=&-\frac{ag}{\cos\alpha}\left(1+3\cos^2\alpha\right)\xi \mbox{~~~ (neglecting square and higher powers of $\xi$)}\nonumber
\end{eqnarray}

As from (\ref{eq3.15.2}), $\cos\alpha$ is positive, so
$$\ddot{\xi}=-\frac{g}{a\cos\alpha}\left(1+3\cos^2\alpha\right)\xi$$ 

Hence the disturbed motion of the particle is simple harmonic in nature and the path is a stable path. The period of oscillation is
$$T=\frac{2\pi}{\sqrt{\frac{g}{a\cos\alpha}\left(1+3\cos^2\alpha\right)}}=\frac{2\pi\sqrt{a\cos\alpha}}{\sqrt{g\left(1+3\cos^2\alpha\right)}}$$

Now, $a^2\sin^2\theta\dot{\phi}=h$ $\implies$ $\dot{\phi}=\dfrac{h}{a^2\sin^2\theta}\simeq\dfrac{h}{a^2\sin^2\alpha}$.
$$\therefore~\phi=\frac{h}{a^2\sin^2\alpha}t+c$$

Assuming $\phi=0$ at $t=0$, we get $c=0$. $$\therefore~\phi=\frac{h}{a^2\sin^2\alpha}t$$

If $\theta_1$, $\theta_2$ are the values of $\phi$ corresponding to two consecutive and minimum values of $\theta$ and $t_1$, $t_2$ are the values of $t$ at this two positions then 
\begin{eqnarray}
	\phi_1-\phi_2&=&\frac{h}{a^2\sin^2\alpha}\left(t_1-t_2\right)=\frac{h}{a^2\sin^2\alpha}\frac{T}{2}\nonumber\\&=&\frac{\pi}{\sqrt{1+3\cos^2\alpha}} \mbox{~~~ (Using (\ref{eq3.15.2}))}\nonumber
\end{eqnarray}

This may be called the apsidal angle of the path.\\

{\bf 15. } A particle tied at one end of a fine string of length $a$ whose other end is attached to a fixed point, is projected horizontally with velocity $V$ with the string inclined at an acute angle $\alpha$ with the downward vertical and is again moving horizontally when the string makes at acute angle $\beta$ with the upward vertical without the string going slack. Find $V$ and prove that (i) $\beta>\alpha$, (ii) $2\sin^2\alpha\geqslant\cos\beta(\cos\alpha-\cos\beta)$.\\

{\bf Solution: } From the equation of energy we have $$v^2=c+2gz=c+2ga\cos\theta$$ 

\begin{wrapfigure}[8]{r}{0.3\textwidth}
	\centering	\includegraphics[height=4.50 cm , width=4 cm ]{f32.pdf}
	\begin{center}
		Fig. 3.29
	\end{center}
\end{wrapfigure}  

As $v=V$ when $z=a\cos\alpha$, so $$v^2=V^2+2ga(\cos\theta-\cos\alpha)$$

Equation of angular momentum gives
\begin{eqnarray}
	a^2\sin^2\theta\dot{\phi}&=&\mbox{constant}=Va\sin\alpha\nonumber\\\therefore~v^2&=&a^2\dot{\theta^2}+a^2\sin^2\theta\dot{\phi}^2\nonumber\\&=&a^2\dot{\theta}^2+\frac{V^2\sin^2\alpha}{\sin^2\theta}=V^2+2ga(\cos\theta-\cos\alpha)\nonumber\\\therefore~a^2\dot{\theta}^2&=&V^2\frac{\left(\sin^2\theta-\sin^2\alpha\right)}{\sin^2\theta}+2ga(\cos\theta-\cos\alpha)\nonumber\\&=&V^2\frac{\left(\cos^2\alpha-\cos^2\theta\right)}{\sin^2\theta}-2ag(\cos\alpha-\cos\theta)\nonumber\\&=&(\cos\alpha-\cos\theta)\left[V^2\frac{(\cos\alpha+\cos\theta)}{\sin^2\theta}-2ag\right]\nonumber
\end{eqnarray} 

As $\dot{\theta}=0$ at $\theta=\alpha$ and $\theta=\pi-\beta$, so we have
$$V^2\frac{(\cos\alpha-\cos\beta)}{\sin^2\beta}-2ag=0$$

So we must have $\cos\alpha>\cos\beta$ i.e., $\alpha<\beta$ and
$$V^2=\frac{2ag\sin^2\beta}{(\cos\alpha-\cos\beta)}$$

Resolving along $OP$,
\begin{eqnarray}
	&&m\frac{v^2}{a}=T-mg\cos\theta\nonumber\\\implies&& m\left[\frac{V^2+2ag(\cos\theta-\cos\alpha)}{a}\right]=T-mg\cos\theta\nonumber\\\implies&& T=\frac{m}{a}\left[=\frac{2ag\sin^2\beta}{(\cos\alpha-\cos\beta)}+2ag(\cos\theta-\cos\alpha)\right]+mg\cos\theta\nonumber
\end{eqnarray}

As $T$ is positive throughout the motion and $T$ is positive at $\theta=\pi-\beta$, so
\begin{eqnarray}
	&&\frac{2\sin^2\beta}{(\cos\alpha-\cos\beta)}-2(\cos\alpha+\cos\beta)-\cos\beta\geqslant0\nonumber\\\mbox{i.e., }&&2\sin^2\alpha\geqslant\cos\beta(\cos\alpha-\cos\beta)\nonumber
\end{eqnarray} 

{\bf 16. } The surface of a smooth funnel is given by the equation $$z=c^2\left(x^2+y^2\right)^{-\frac{1}{2}}$$
 the positive axis of $z$ being vertically downwards. A particle is projected horizontally, along the inner surface, at the level $z=c$ with such a velocity that it is again moving horizontally at the level $z=2c$. Prove that the principal normal, at the highest point of the path, makes an angle $\tan^{-1}\left(\dfrac{1}{5}\right)$ with the horizontal.\\
 
 {\bf Solution: } The equation of energy gives $$v^2=V^2+2g(z-c)$$ 
 
 Equation of angular momentum gives $$r^2\dot{\theta}=Vc$$
 
 Equation of Meridian plane is $zr=c^2$ $$\mbox{i.e., }z\dot{r}+r\dot{z}=0\mbox{ ~i.e.,~ }\dot{r}=-\frac{r\dot{z}}{z}=-\frac{c^2\dot{z}}{z^2}$$
 	
 	Now,\begin{eqnarray}
 		v^2&=&\dot{r}^2+r^2\dot{\theta}^2+\dot{z}^2\nonumber\\&=&\frac{c^4}z^4\dot{z}^2+\dot{z}^2+\frac{V^2c^2}{r^2}=V^2+2g(z-c)\nonumber\\\therefore~\dot{z}^2\left(1+\frac{c^4}{z^4}\right)&=&V^2\frac{(c^2-z^2)}{c^2}+2g(z-c)\nonumber
 	\end{eqnarray}
 	
 	Hence $\dot{z}=0$ gives $$V^2\frac{(c^2-z^2)}{c^2}+2g(z-c)=0$$
 	
 	As $\dot{z}=0$ when $z=c$ and $z=2c$,so we have $$2g=\frac{3cV^2}{c^2}\mbox{~ i.e., ~}V^2=\frac{2}{3}gc$$
 	
 	Now, $rz=c^2$, so $\dfrac{\mathrm{d}r}{\mathrm{d}z}=-\dfrac{c^2}{z^2}=-1=\cot\theta$ (say) at $z=c$.\\
 	
 	Also, $$\tan\chi=\frac{gr_0}{v_0}=\frac{gc}{V}=\frac{gc}{\frac{2}{3}gc}=\frac{3}{2}$$
 	
 	Hence the required angle $=\theta+\chi$
 	\begin{eqnarray}
 	\therefore&&\tan(\theta+\chi)=\frac{\frac{3}{2}-1}{1+\frac{3}{2}}=\frac{1}{5}\nonumber\\\therefore&&\tan A=\tan\left(\chi-(90^o-\theta)\right)=-\cot(\theta+\chi)=-5\nonumber	
 	\end{eqnarray}
 
 {\bf 17. } A particle is moving under gravity in a horizontal circle in the inner surface of a smooth paraboloid of revolution with its axis vertical. A slight disturbance is given to the particle keeping the angular momentum about the axis unaltered. Discuss the  stability of motion of the particle.\\
 
 {\bf Solution: } We take origin at the vertex of the paraboloid, $z$-axis along the axis of the paraboloid and vertically upward. Let $P(r,\theta,z)$ be the position of the particle at time $t$ in cylindrical co-ordinates. Then the velocity $v$ of the particle is given by
 \begin{equation}\label{eq3.17.1}
 	v^2=\dot{r}^2+r^2\dot{\theta}^2+\dot{z}^2
 \end{equation}
 
 \begin{wrapfigure}[14]{r}{0.35\textwidth}
 	\centering	\includegraphics[height=5 cm , width=7 cm ]{f33.pdf}
 	\begin{center}
 		Fig. 3.30
 	\end{center}
 \end{wrapfigure}  

Suppose $z=f(r)$ be the equation of the meridian curve. From the equation of energy, we get
\begin{equation}\label{eq3.17.2}
	v^2=c-2gz
\end{equation}
and from the equation of angular momentum
\begin{equation}\label{eq3.17.3}
	r^2\dot{\theta}=h
\end{equation}

Using (\ref{eq3.17.2}) and (\ref{eq3.17.3}) in (\ref{eq3.17.1}) and also using the equation of the meridian plane we get
$$\dot{r}^2\left(1+\left\{f'(r)\right\}^2\right)+\frac{h^2}{r^2}=c-2gf(r)$$

Now, differentiating both side with respect to $r$, we have
\begin{eqnarray}
	2\ddot{r}\left(1+\left\{f'(r)\right\}^2\right)+2\dot{r}^2f'(r)f''(r)-2\frac{h^2}{r^3}=-2gf'(r)\nonumber\\\mbox{i.e., }\left(1+\left\{f'(r)\right\}^2\right)\ddot{r}+\dot{r}^2f'(r)f''(r)-\frac{h^2}{r^3}=-gf'(r)\label{eq3.17.4}
\end{eqnarray}

If the particle describes the circle $r=a$, then $\dot{r}=0=\ddot{r}$ at $r=a$, so from (\ref{eq3.17.4}) we have
\begin{equation}
	\frac{h^2}{a^3}=gf'(a)
\end{equation}

Let us now suppose that a slight disturbance is given to the particle such that the angular momentum about the axis remains unaltered. For this disturbed motion, $r=a+\xi$, where $\xi$ is small compared to $a$ so that we can neglect $\xi^2$ and higher power of $\xi$. We also assume that we can neglect $\dot{\xi}^2$, then from (\ref{eq3.17.4}) we get
\begin{eqnarray}
	&&\left(1+\left\{f'(a+\xi)\right\}^2\right)\ddot{\xi}+\dot{\xi}^2f'(a+\xi)f''(a+\xi)-\frac{h^2}{(a+\xi)^3}=-gf'(a+\xi)\nonumber\\\implies&&\ddot{\xi}\left(1+\{f'(a)\}^2+2\xi f'(a)f''(a)\right)-\frac{h^2}{a^3}\left(1-3\frac{\xi}{a}\right)=-g\left\{f'(a)+\xi f''(a)\right\}\nonumber\\&&\mbox{~~~~~~~~~~~~~~~ (neglecting square and higher powers of $\xi$ and its derivatives)}\nonumber\\\implies&&\ddot{\xi}\left(1+{\{f'(a)\}}^2\right)=-\frac{g}{a}\xi\left\{3f'(a)+af''(a)\right\}\nonumber\\\mbox{i.e., }&&\ddot{\xi}=-\frac{g}{a}\left\{\frac{3f'(a)+af''(a)}{1+{\{f'(a)\}}^2}\right\}\xi\nonumber
\end{eqnarray}

It follows that if, $3f'(a)+af''(a)>0$, the disturbed motion will be stable otherwise it is unstable.\\
	
	If the motion is stable, the period of complete oscillation is $\dfrac{2\pi}{K}$, where $K^2=\dfrac{3f'(a)+af''(a)}{1+{\{f'(a)\}}^2}$.\\
	
	Hrom (\ref{eq3.17.3}), $r^2\dot{\theta}=h$, i.e., $\dot{\theta}=\dfrac{h}{r^2}\simeq\dfrac{h}{a^2}$.\\
	
	Thus the change in azimuth form the maximum $r$ to the consecutive minimum is
	$$\frac{h}{a^2}\frac{T}{2}=\frac{\sqrt{ga^3f'(a)}}{a^2}\pi\sqrt{\frac{a}{g}\frac{{\left\{1+{\{f'(a)\}}^2\right\}}}{\{3f'(a)+af''(a)\}}}=\pi\sqrt{\frac{f'(a){\left\{1+{\{f'(a)\}}^2\right\}}}{\{3f'(a)+af''(a)\}}}$$
	
{\bf 18. } Discuss the motion of a heavy particle on the surface of a smooth right circular cone with axis vertical and vertex downward.\\

{\bf Solution: } The equation of motion along the line $OP$ is
$$\left(\ddot{r}-r\dot{\theta}^2\right)\sin\alpha+\ddot{z}\cos\alpha=-g\cos\alpha$$

Equation of angular momentum gives
$$r^2\dot{\theta}=h,~r=z\tan\alpha$$

So we have
\begin{eqnarray}
	\ddot{r}\left(sin\alpha+\frac{\cos^2\alpha}{\sin\alpha}\right)-\frac{h^2}{r^3}\sin\alpha=-g\cos\alpha\nonumber\\\mbox{i.e., }\ddot{r}-\frac{h^2}{r^3}\sin^2\alpha=-g\cos\alpha\sin\alpha\label{eq3.18.1}
\end{eqnarray}

 \begin{wrapfigure}[9]{r}{0.35\textwidth}
	\centering	\includegraphics[height=5 cm , width=5 cm ]{f34.pdf}
	\begin{center}
		Fig. 3.31
	\end{center}
\end{wrapfigure}  

Choosing $u=\dfrac{1}{r}$ we have
\begin{eqnarray}
	&&\dot{r}=-\frac{1}{u^2}\frac{\mathrm{d}u}{\mathrm{d}t}=-\frac{1}{u^2}\frac{\mathrm{d}u}{\mathrm{d}\theta}\dot{\theta}=-h\frac{\mathrm{d}u}{\mathrm{d}\theta}\nonumber\\
	&&\ddot{r}=-h\frac{\mathrm{d}^2u}{\mathrm{d}\theta^2}hu^2=-h^2u^2\frac{\mathrm{d}^2u}{\mathrm{d}\theta^2}\nonumber\\
	\therefore~&&-h^2u^2\frac{\mathrm{d}^2u}{\mathrm{d}\theta^2}-h^2u^3\sin^2\alpha=-g\cos\alpha\sin\alpha\nonumber\\
	\mbox{i.e., }&&\frac{\mathrm{d}^2u}{\mathrm{d}\theta'^2}
\end{eqnarray}

This is the differential equation of the path of the particle. Also from (\ref{eq3.18.1}) if $\dot{r}=0=\ddot{r}$ when $r=a$ then
\begin{eqnarray}
	&&-\frac{h^2}{a^3}\sin^2\alpha=-g\cos\alpha\sin\alpha\nonumber\\\mbox{i.e., }&&\frac{h^2}{a^3}=g\cot\alpha\nonumber
\end{eqnarray}

For small disturbance $r=a+\xi$, we get from (\ref{eq3.18.1})
\begin{eqnarray}
	&&\ddot{\xi}-\frac{h^2\sin^2\alpha}{a^3}\left(1+\frac{\xi}{a}\right)^{-3}=-g\cos\alpha\sin\alpha\nonumber\\\mbox{i.e., }&&\ddot{\xi}-g\cos\alpha\sin\alpha\left(1-3\frac{\xi}{a}\right)=-g\cos\alpha\sin\alpha\nonumber\\&&~~~~~~~~~\mbox{(neglecting square and higher powers of $\xi$)}\nonumber\\\mbox{i.e., }&&\ddot{\xi}+3\frac{g}{a}\sin\alpha\cos\alpha\xi=0\nonumber
\end{eqnarray}

So the disturbed motion will be stable with the period of complete oscillation $\dfrac{2\pi}{K}$, where $K^2=3\dfrac{g}{a}\sin\alpha\cos\alpha$.\\

To discuss the vertical motion, let $OP=r'$ and $\angle POQ=\mathrm{d}\theta'$. So we have
\begin{eqnarray}
	&&r=r'\sin\alpha,~PQ=r\mathrm{d}\theta=r\mathrm{d}\theta'\nonumber\\\mbox{i.e., }&&\mathrm{d}\theta=\frac{r'}{r}\mathrm{d}\theta'=\frac{1}{\sin\alpha}\mathrm{d}\theta'\nonumber\\\therefore~&&\frac{\mathrm{d}u}{\mathrm{d}\theta}=\sin\alpha\frac{\mathrm{d}u}{\mathrm{d}\theta'}=\frac{\mathrm{d}u}{\mathrm{d}\theta'}(u\sin\alpha)=\frac{\mathrm{d}u'}{\mathrm{d}\theta'}~~~\left(u'=\frac{1}{r'}\right)\nonumber\\\therefore~&&\frac{\mathrm{d}^2u}{\mathrm{d}\theta^2}=\frac{\mathrm{d}}{\mathrm{d}\theta}\left(\frac{\mathrm{d}u'}{\mathrm{d}\theta'}\right)=\sin\alpha\frac{\mathrm{d}^2u}{\mathrm{d}{\theta'}^2}\nonumber
\end{eqnarray}

So the path of the particle becomes
\begin{eqnarray}
	&&\sin\alpha\frac{\mathrm{d}^2u'}{\mathrm{d}{\theta'}^2}+u'\sin\alpha=\frac{g\sin^3\alpha\cos\alpha}{h^2{u'}^2}\nonumber\\\mbox{i.e., }&&\frac{\mathrm{d}^2u'}{\mathrm{d}{\theta'}^2}+u'=\frac{g\sin^2\alpha\cos\alpha}{h^2{u'}^2}\nonumber
\end{eqnarray}

It follows that if the cone be developed into a plane the trace of the path on the surface will be the same as if a particle moves in a plane under the action of a constant central force.\\

Further, from the energy equation,
$$\dot{r}^2+r^2\dot{\theta}^2+\dot{z}^2=c-2gz$$
with $r=z\tan\alpha$, $r^2\dot{\theta}=h$.\\

So we get $$\dot{z}^2\sec^2\alpha+\frac{h^2}{z^2}\cot^2\alpha+2gz=c$$

This gives the vertical motion of the particle.\\

{\bf 19. } A particle is moving under gravity in a horizontal circle in the inner surface of a smooth cone with axis vertical and vertex downwards. A slight disturbance is given to the particle, keeping the angular momentum about the axis unaltered. Discuss the stability of motion of the particle.\\

{\bf Solution: } We take the vertex of the cone $O$ as origin and $z$-axis along the axis of the cone and vertically upward. Using spherical polar co-ordinate $P(r,\alpha,\phi)$ the velocity of it at time $t$ is given by $$v^2=\dot{r}^2+r^2\sin^2\alpha~\dot{\phi}^2$$

 \begin{wrapfigure}[9]{r}{0.35\textwidth}
	\centering	\includegraphics[height=5 cm , width=5 cm ]{f35.pdf}
	\begin{center}
		Fig. 3.32
	\end{center}
\end{wrapfigure}  

The energy momentum gives 
\begin{equation}\label{eq3.19.1}
	v^2=c-2gz=c-2gr\cos\alpha
\end{equation}

The conservation of angular momentum gives
\begin{equation}\label{eq3.19.2}
	r^2\sin^2\alpha~\dot{\phi}=h
\end{equation}

Using (\ref{eq3.19.1}) and (\ref{eq3.19.2}) in the expression for velocity we get
\begin{eqnarray}
	\dot{r}^2+r^2\sin^2\alpha~\dot{\phi}^2=c-2gr\cos\alpha\nonumber\\\mbox{i.e.,~ }\dot{r}^2+\frac{h^2}{r^2\sin^2\alpha}=c-2gr\cos\alpha\nonumber
\end{eqnarray}

Now, differentiating both side with respect to $r$
\begin{equation}\label{eq3.19.3}
	\ddot{r}-\frac{h^2}{r^3\sin^2\alpha}=-g\cos\alpha
\end{equation}

If the particle describes the circle $r=a$, i.e., $\dot{r}=0=\ddot{r}$ when $r=a$, then from (\ref{eq3.19.3})
\begin{equation}
	\frac{h^2}{a^3}=g\sin^2\alpha\cos\alpha
\end{equation}

Let us now suppose that a slight disturbance is given to the particle such that the angular momentum about the axis remains unaltered. For this disturbed motion $r=a+\xi$, where $\xi$ is small compared to $a$ so that we can neglect $\xi^2$ and higher powers of $\xi$. Hence from (\ref{eq3.19.3})
\begin{eqnarray}
	&&	\ddot{\xi}-\frac{h^2}{(a+\xi)^3\sin^2\alpha}=-g\cos\alpha\nonumber\\\mbox{i.e., }&&	\ddot{\xi}-\frac{h^2}{a^3\sin^2\alpha}\left(1-3\frac{\xi}{a}\right)=-g\cos\alpha\nonumber\\
	\mbox{i.e., }&&\ddot{\xi}+3\frac{g}{a}\cos\alpha~\xi=0
\end{eqnarray}

As $K^2=3\dfrac{g}{a}\cos\alpha>0$ so the disturbed motion is stable and the period of oscillation is $\dfrac{2\pi}{K}$.\\

From (\ref{eq3.19.2}), $\dot{\phi}=\dfrac{h}{r^2\sin^2\alpha}\simeq\dfrac{h}{a^2\sin^2\alpha}$.\\

Tf $\phi_1$, $\phi_2$ are the values of $\phi$ corresponding to two consecutive maximum and minimum value of $r$ and $t_1$, $t_2$ are the times at this two position then
\begin{eqnarray}
	\phi_1-\phi_2&=&\dfrac{h}{a^2\sin^2\alpha}(t_1-t_2)=\dfrac{h}{a^2\sin^2\alpha}\frac{T}{2}\nonumber\\&=&\frac{\pi}{\sqrt{\frac{3g\cos\alpha}{a}}}\frac{\left(a^3g\sin^2\alpha\cos\alpha\right)^\frac{1}{2}}{a^2\sin^2\alpha}=\frac{\pi}{\sqrt{3}\sin\alpha},\nonumber
\end{eqnarray} 
represents the apsidal angle..\\

{\bf 20. } A heavy particle moves on the external surface of a smooth circular cone, which is fixed with its axis vertical and vertex upwards. The velocity of the particle is that which it would acquire by sliding from the vertex. Show that the path of the particle, when developed into a plane, has a polar equation of the form $$r^3=a^3\sec^2\left(\frac{3\theta}{2}\right)$$
referred to the vertex as origin.\\

{\bf Solution: } Let $P(r,\alpha,\phi)$ be the spherical polar co-ordinates of the particle on the cone at time $t$. Let $O$, $P$, $Q$ on the cone becomes $O'$, $P'$, $Q'$ on the developed plane. Then assuming arc $PQ=\mathrm{d}s$ and arc $P'Q'=\mathrm{d}s'$ we have $\mathrm{d}s=\mathrm{d}s'$. \\

Now, $r=OP=O'P'=r'$ (say)\\

So,\begin{eqnarray}
	&&\mathrm{d}s^2=\mathrm{d}r^2+r^2\sin^2\alpha\mathrm{d}\phi^2=(\mathrm{d}r')^2+r'^2(\mathrm{d}\theta')^2\nonumber\\\mbox{i.e., }&&\sin\alpha\mathrm{d}\phi=\mathrm{d}\theta'
\end{eqnarray}

Resolving the velocity perpendicular to the $ZOP$ plane we obtain
\begin{equation}
	r^2\sin^2\alpha\dot{\phi}=h~~\mbox{ (constant)}
\end{equation}

\begin{wrapfigure}[9]{r}{0.35\textwidth}
	\centering	\includegraphics[height=6 cm , width=6 cm ]{f36.pdf}
	\begin{center}
		Fig. 3.33
	\end{center}
\end{wrapfigure}  

The energy equation conservation equation gives
\begin{equation}
	v^2=c+2gz=c+2gr\cos\alpha\nonumber
\end{equation}

But $v=0$ at $z=0$, $\implies c=0$.
\begin{eqnarray} 
\therefore&&v^2=2gr\cos\alpha\nonumber\\\implies&&\dot{r}^2+r^2\sin^2\alpha\dot{\phi}^2=2gr\cos\alpha\nonumber\\\mbox{i.e.,}&&\dot{r}^2=2gr\cos\alpha-\frac{h^2}{r^2\sin^2\alpha}\nonumber\\&&~~~=\frac{h^2r}{\sin^2\alpha}\left[\frac{2g\cos\alpha\sin^2\alpha}{h^2}-\frac{1}{r^3}\right]\nonumber\\&&~~~=\frac{h^2r}{\sin^2\alpha}\left(\frac{1}{a^3}-\frac{1}{r^3}\right), ~~~~~a^3=\frac{h^2}{2g\cos\alpha\sin^2\alpha}\nonumber
\end{eqnarray}

As, $r^4\sin^4\alpha\dot{\phi}^2=h^2$, so
\begin{eqnarray}
	&&\frac{1}{r^5\sin^2\alpha}\left(\frac{\mathrm{d}r}{\mathrm{d}\phi}\right)^2=\frac{1}{a^3}-\frac{1}{r^3}=\frac{1}{a^3}\left(1-\frac{a^3}{r^3}\right)\nonumber\\\implies&&\frac{\frac{a^\frac{3}{2}}{r^\frac{5}{2}}\mathrm{d}r}{\sqrt{1-\frac{a^3}{r^3}}}=\pm\sin\alpha\mathrm{d}\phi\nonumber
\end{eqnarray}

Hence on the developed plane
\begin{eqnarray}
	\frac{\frac{a^\frac{3}{2}}{{r'}^\frac{5}{2}}\mathrm{d}r'}{\sqrt{1-\frac{a^3}{{r'}^3}}}&=&\pm\mathrm{d}\theta'\nonumber\\\implies\cos^{-1}\left(\frac{a}{r'}\right)^\frac{3}{2}&=&\pm\frac{3}{2}\left(\theta'+k\right)\nonumber
\end{eqnarray}

Assuming $r'=a$, $\theta=0$ $\implies$ $k=0$
\begin{equation}
	\therefore\left(\frac{a}{r'}\right)^3=\cos^2\left(\frac{3\theta'}{2}\right)\nonumber
\end{equation}

{\bf 21. } A particle moves on a smooth right circular cone under the action of a force from the vertex, the law of repulsion being  $$\mu\left(\frac{a\cos^2\alpha}{r^3}-\frac{1}{2r^2}\right)$$ where $\alpha$ is the semi-vertical angle of the cone. Prove that if it be projected from an apse at a distance $a$ with velocity $\sqrt{\dfrac{\mu}{a}}\sin\alpha$, the path will be a parabola.\\

{\bf Solution: } Let $P(r,\alpha,\phi)$ be the spherical polar co-ordinates of the particle at time $t$ and $v$ be the velocity at $P$.\\

\begin{wrapfigure}[9]{r}{0.35\textwidth}
	\centering	\includegraphics[height=5 cm , width=5 cm ]{f37.pdf}
	\begin{center}
		Fig. 3.34
	\end{center}
\end{wrapfigure} 

Now resolving the velocity perpendicular to $ZOP$-plane we get
\begin{eqnarray}
	&&r^2\sin^2\alpha\dot{\phi}=\mbox{constant}=\sqrt{\dfrac{\mu}{a}}\sin\alpha\cdot a\sin\alpha\nonumber\\\mbox{i.e., }&&r^2\dot{\phi}=\sqrt{\mu a}
\end{eqnarray}

If $\psi$ be the potential of the force, then
\begin{eqnarray}
	-\frac{\partial\psi}{\partial r}&=&\mu\left(\frac{a\cos^2\alpha}{r^3}-\frac{1}{2r^2}\right)\nonumber\\\mbox{i.e., }\psi&=&\frac{\mu}{2}\left(\frac{a\cos^2\alpha}{r^2}-\frac{1}{r}\right)+\mbox{ constant}\nonumber
\end{eqnarray} 

So the energy equation gives
\begin{eqnarray}
	&&\frac{1}{2}v^2+\psi=\mbox{constant}\nonumber\\\mbox{i.e.,}&&v^2+\mu\left(\frac{a\cos^2\alpha}{r^2}-\frac{1}{r}\right)-\mbox{constant}=\frac{\mu}{a}\sin^2\alpha+\left(\frac{\cos^2\alpha}{a}-\frac{1}{a}\right)=0\nonumber\\\mbox{i.e.,}&&\dot{r}^2+r^2\sin^2\alpha\dot{\phi}^2=\mu\left(\frac{1}{r}-\frac{a\cos^2\alpha}{r^2}\right)\nonumber\\\mbox{i.e.,}&&\dot{r}^2=\left(\frac{1}{r}-\frac{a\cos^2\alpha}{r^2}\right)-\frac{\mu a\sin^2\alpha}{r^2}=\mu\left(\frac{1}{r}-\frac{a}{r^2}\right)\nonumber
\end{eqnarray}

Also $r^4\dot{\phi}^2=\mu a$.
\begin{eqnarray}
	\therefore&&\frac{1}{r^4}\left(\frac{\mathrm{d}r}{\mathrm{d}\phi}\right)^2=\frac{1}{a}\left(\frac{1}{r}-\frac{a}{r^2}\right)=\frac{1}{ar^2}(r-a)\nonumber\\\mbox{i.e.,}&&\frac{a}{r^2(r-a)}\left(\frac{\mathrm{d}r}{\mathrm{d}\phi}\right)^2=1\nonumber\\\mbox{i.e.,}&&\frac{\frac{\sqrt{a}}{r^\frac{3}{2}}}{\sqrt{1-\frac{a}{r}}}\mathrm{d}r=\pm\mathrm{d}\phi\nonumber\\\mbox{i.e.,}&&2\cos^{-1}\sqrt{\frac{a}{r}}=\pm(\phi+k)\nonumber
\end{eqnarray}

Assuming $\phi=0$ at $r=a$ i.e., $k=0$.
\begin{eqnarray}\label{eq3.21.2}
	\therefore&&\frac{a}{r}=\cos^2\frac{\phi}{2}=\frac{1+\cos\phi}{2}\nonumber\\\mbox{i.e.,}&&2a=r+r\cos\phi 
\end{eqnarray} 

Now on the surface of the cone, the Cartesian co-ordinates $x$, $y$, $z$ are given by
$$x=r\sin\alpha\cos\phi,~y=r\sin\alpha\sin\phi,~z=r\cos\alpha$$

So equation (\ref{eq3.21.2}) can be written as
\begin{equation}\label{eq3.21.3}
	2a=\frac{z}{\cos\alpha}+\frac{x}{\sin\alpha}
\end{equation}
 which represents a plane. Hence the path of the particle on the cone is a plane curve. The direction cosine of the normal to the plane are proportional to $(1,0,\tan\alpha)$ and this normal is perpendicular to the generator
 \begin{equation}\label{eq3.21.4}
 	\frac{x}{\sin\alpha}=\frac{y}{0}=\frac{z}{-\cos\alpha}
 \end{equation} 

Hence the plane (\ref{eq3.21.3}) is parallel to the generator (\ref{eq3.21.4}). Since the path of the particle is the intersection of the plane (\ref{eq3.21.3}) with the cone, the path must be a parabolic path.\\

{\bf 22. } A particle moves along the smooth surface of a right circular cone under the action of a force parallel to the axis of the cone and proportional to the distance of the particle from the axis, its initial velocity being that which it would acquire in moving from the vertex. Prove that its path when the cone is developed into a plane is a rectangular hyperbola.\\

{\bf Solution: } If $\psi$ be the potential of the potential function then
\begin{eqnarray}
	&&-\frac{\partial\psi}{\partial z}=\lambda z\tan\alpha\nonumber\\\mbox{i.e.,}&&\psi=-\frac{\lambda z^2}{2}\tan\alpha\nonumber
\end{eqnarray}

\begin{wrapfigure}[12]{r}{0.35\textwidth}
	\centering	\includegraphics[height=5 cm , width=5 cm ]{f38.pdf}
	\begin{center}
		Fig. 3.35
	\end{center}
\end{wrapfigure} 

From the energy equation
\begin{eqnarray}
	&&\frac{1}{2}v^2+\psi=\mbox{constant}\nonumber\\\mbox{i.e.,}&&v^2-\lambda z^2\tan^2\alpha=c, \mbox{ a constant}\nonumber
\end{eqnarray} 

As $v=0$ at $z=0$ so $c=0$. \\

The equation of the angular momentum gives $r^2\sin^2\alpha\dot{\phi}=h$.
\begin{eqnarray}
	\therefore&&v^2=\dot{r}^2+r^2\sin^2\alpha\dot{\phi}^2=\lambda z^2\tan\alpha=\lambda r^2\sin\alpha\cos\alpha\nonumber\\\therefore&&\dot{r}^2+\frac{h^2}{r^2\sin^2\alpha}=\lambda r^2\sin\alpha\cos\alpha\nonumber\\\mbox{i.e.,}&&\dot{r}^2=\lambda r^2\sin\alpha\cos\alpha-\frac{h^2}{r^2\sin^2\alpha}\nonumber
\end{eqnarray} 

Also $r^4\sin^4\alpha\dot{\phi}^2=h^2$.
\begin{eqnarray}
	\therefore&&\frac{1}{r^6\sin^2\alpha}\left(\frac{\mathrm{d}r}{\mathrm{d}\phi}\right)^2=A-\frac{B}{r^4}\nonumber\\\mbox{i.e.,}&&\frac{\frac{1}{r^3}\mathrm{d}r}{\sqrt{A-\frac{B}{r^4}}}\mathrm{d}r=\sin\alpha\mathrm{d}\phi\nonumber
\end{eqnarray} 

So in developed plane
\begin{eqnarray}
	&&\frac{\frac{1}{{r'}^3}\mathrm{d}r'}{\sqrt{A-\frac{B}{{r'}^4}}}\mathrm{d}r'=\mathrm{d}\theta'\nonumber\\\mbox{i.e.,}&&\sin^{-1}\left(\frac{B}{\sqrt{A}{r'}^2}\right)=2k\theta'+c\nonumber\\\mbox{i.e.,}&&\frac{B}{\sqrt{A}{r'}^2}=\sin(2k\theta')\mbox{~~~~ (choosing $r'$, $\theta'$ in such a way that $c=0$)}\nonumber\\\mbox{i.e.,}&&x'y'=\mbox{constant, a rectangular hyperbola.}\nonumber
\end{eqnarray} 

 {\bf 23. } A heavy particle moves on a smooth right circular cone with axis vertical and vertex downwards, being projected with the velocity due to a fall from a height $z_0$ above the vertex to the point of projection. Prove that, if $z_1$, $z_2$ are the greatest and lowest heights attained then $$z_1^2+z_1z_2+z_2^2=z_0\left(z_1+z_2\right)$$
 
 {\bf Solution: } The energy equation gives
 $$v^2+2gz=c=2g(z_0-h)+2gh=2gz_0$$ 
 
 The equation of angular momentum can be written as
 $$r^2\sin^2\alpha\dot{\phi}=h$$
 
 Thus 
 \begin{eqnarray}
 	&&\dot{r}^2+r^2\sin^2\alpha\dot{\phi}^2=v^2=2g(z_0-z)\nonumber\\\mbox{i.e.,}&&\dot{z}^2\sec^2\alpha+\frac{h^2}{z^2\tan^2\alpha}=2g(z_0-z),~~~~~~~~~z=r\cos\alpha\nonumber
 \end{eqnarray}

Now, for maximum or minimum value, $\dot{z}=0$
\begin{eqnarray}\label{eq3.23.1}
	\mbox{i.e., }\frac{h^2}{z^2\tan^2\alpha}=2g(z_0-z)\nonumber\\\mbox{i.e., }\frac{h^2}{2g\tan^2\alpha}=z^2(z_0-z)
\end{eqnarray}

Let $z_1$, $z_2$ are two solutions of (\ref{eq3.23.1}). So we have 
\begin{eqnarray}
	&&\frac{h^2}{2g\tan^2\alpha}=z_1^2(z_0-z_1)=z_2^2(z_0-z_2)\nonumber\\\mbox{i.e.,}&&z_1^3-z_2^3=z_0\left(z_1^2-z_2^2\right)\nonumber\\\mbox{i.e.,}&&z_1^2+z_1z_2+z_2^2=z_0\left(z_1+z_2\right)\nonumber
\end{eqnarray}

As the vertex is downward and the particle moves on the surface so reaction can never be zero. \\
 
 {\bf 24. } Prove that if particles move on a right circular cone under no force, the projections of their paths on a plane perpendicular to the axis are similar curves of the type $r\sin n\theta=c$, whatever be their initial velocities.\\
 
 {\bf Solution: } In the cylindrical coordinates $(r,\theta,z)$ we have
 \begin{eqnarray}
 	&&v^2=c,~r^2\dot{\theta}=h, r=z\tan\alpha\nonumber\\\therefore&&\dot{r}^2+\dot{z}^2+r^2\dot{\theta}^2=c\nonumber\\\mbox{i.e.,}&&\dot{r}^2\mbox{cosec}^2\alpha+\frac{h^2}{r^2}=c\nonumber\\\mbox{i.e.,}&&\dot{r}^2=c\sin^2\alpha-\frac{h^2\sin^2\alpha}{r^2}=h^2\sin^2\alpha\left(\frac{1}{a^2}-\frac{1}{r^2}\right)\nonumber\\\mbox{i.e.,}&&\dot{r}=h\sin\alpha\left(\frac{1}{a^2}-\frac{1}{r^2}\right)^\frac{1}{2}\nonumber\\&&\dot{\theta}=\frac{h}{r^2}\nonumber\\\therefore&&\frac{\mathrm{d}r}{\mathrm{d}\theta}=r^2\sin\alpha\left(\frac{1}{a^2}-\frac{1}{r^2}\right)^\frac{1}{2}\nonumber\\\mbox{i.e.,}&&\frac{\frac{1}{r^2}\mathrm{d}r}{\sqrt{\frac{1}{a^2}-\frac{1}{r^2}}}=\sin\alpha\mathrm{d}\theta\nonumber\\\mbox{i.e.,}&&a=r\sin(n\theta),~~~n=\sin\alpha\nonumber\\&&~~~~~~~~~~\mbox{ and choosing $\theta$ in such a way that integration constant to be zero.}\nonumber
 \end{eqnarray} 

\section{Motion of a particle on a rough surface}

Let $\mu$ be the coefficient of friction and $R$ be the normal reaction of the surface, then equations of motion are
\begin{eqnarray}
	mv\frac{\mathrm{d}v}{\mathrm{d}s}&=&F-\mu R\label{eq3.61}\\m\frac{v^2}{\rho_0}&=&R+G\label{eq3.62}\\m\frac{v^2}{\rho_0}\tan\chi&=&H\label{eq3.63}
\end{eqnarray} 
where $F$, $G$, $H$ are the components of the external force acting on the particle along the path, along the normal to the surface, and along the tangent to the surface but perpendicular to the tangent to the path respectively. $\rho_0$ is the radius of curvature of the normal section of the surface through the tangent to the path and $v$ is the velocity of the particle at time $t$. Now eliminating $R$ between (\ref{eq3.61}) and (\ref{eq3.62}) we get
\begin{equation}
	mv\frac{\mathrm{d}v}{\mathrm{d}s}=F\left(\frac{v^2}{\rho_0}-G\right)\label{eq3.64}
\end{equation}

Thus equations (\ref{eq3.63}) and (\ref{eq3.64}) determine the motion of the particle on the surface. Further, knowing $v$ we can determine the normal reaction $R$ from equation (\ref{eq3.62}).\\

In the absence of external forces the equations of motion are
\begin{equation}
		mv\frac{\mathrm{d}v}{\mathrm{d}s}=-\mu R,~~m\frac{v^2}{\rho_0}=R,~~m\frac{v^2}{\rho_0}\tan\chi=0\nonumber
\end{equation}

So the last equation gives $\chi=0$ which shows that the path of the particle is a geodesic on the surface. The velocity of the particle is given by $$v\frac{\mathrm{d}v}{\mathrm{d}s}=-\mu\frac{v^2}{\rho_0}$$ 
and knowing $v$ we can determine $R$. \\

\subsection{Motion of a particle on a rough sphere under no external forces}

In the case of a sphere $\rho_0=a$, the radius of the sphere. The equations of motion are
\begin{equation}
	v\frac{\mathrm{d}v}{\mathrm{d}s}=-\frac{\mu}{m} R,~~\frac{v^2}{a}=\frac{R}{m},~~m\frac{v^2}{a}\tan\chi=0\nonumber
\end{equation}

So $\chi=0$ and the path is a geodesic i.e., a great circle.\\

 Also \begin{eqnarray}
 	&&v\frac{\mathrm{d}v}{\mathrm{d}s}=\frac{\mathrm{d}v}{\mathrm{d}t}=-\mu\frac{v^2}{a}\nonumber\\\mbox{i.e.,}&&\frac{1}{v}-\frac{1}{v_0}=\frac{\mu}{a}t\label{eq3.65}
 \end{eqnarray}
where we assume $v=v_0$ at $t=0$.
$$v=\frac{\mathrm{d}s}{\mathrm{d}t}=\frac{1}{\frac{1}{v_0}+\frac{\mu}{a}t}$$ 

So integrating once more
$$\frac{a}{\mu}\log\left(\frac{1}{v_0}+\frac{\mu}{a}t\right)=s+c$$

Choosing $s=0$ at $t=0$, we have
\begin{equation}\label{eq3.66}
	s=\frac{a}{\mu}\log\left(1+\frac{\mu v_0}{a}t\right)
\end{equation}

From equation (\ref{eq3.65}) we see that the velocity of the particle is decreasing with time and tends to zero as $t\to\infty$. Equation (\ref{eq3.66}) shows that $s$ is increasing with $t$ and tends to infinity as $t\to\infty$.\\

Thus the motion of the particle is along a great circle and the particle moves around the great circle indefinite number of times until its velocity vanishes.\\

\subsection{Motion of a particle on a right circular cylinder under no external forces}

We take $z$-axis along the axis of the cylinder and $P(r,\theta,z)$ be the position of the particle of the surface in cylindrical coordinates at time $t$. Since there is no external force acting on the particle the path of the particle on the surface will be a geodesic.\\

\begin{wrapfigure}[10]{r}{0.35\textwidth}
	\centering	\includegraphics[height=4 cm , width=4 cm ]{f39.pdf}
	\begin{center}
		Fig. 3.36
	\end{center}
\end{wrapfigure} 

Before, discussing the motion of the particle we shall discuss an important result for a geodesic on the surface of the evolution.\\

Let $x^2+y^2=\phi(z)$ be the equation of the surface of revolution. The direction ratio of the normal to the surface at $(x,y,z)$ are $(2x,2y,-\phi'(z))$. The direction ratio of the principal normal to the path are $(x'',y'',z'')$ $\left(x''=\dfrac{\mathrm{d}^x}{\mathrm{d}s^2}\right)$. Thus for a geodesic the principal normal coincides with the normal to the surface. Therefore,
\begin{eqnarray} 
	&&\frac{x''}{2x}=\frac{y''}{2y}=\frac{z''}{-\phi'(z)}\nonumber\\\mbox{i.e.,}&&xy''-yx''=0\nonumber\\\mbox{i.e.,}&&xy'-yx'=\mbox{constant}\nonumber
\end{eqnarray}

\begin{wrapfigure}[9]{l}{0.35\textwidth}
	\centering	\includegraphics[height=3 cm , width=6 cm ]{f40.pdf}
	\begin{center}
		Fig. 3.37
	\end{center}
\end{wrapfigure} 

Using $x=r\cos\theta$, $y=r\sin\theta$, we get $$r^2\frac{\mathrm{d}\theta}{\mathrm{d}s}=\mbox{constant}$$

From the figure, arc $PQ=r\mathrm{d}\theta=\sin\psi\mathrm{d}s$ where, $\psi$ is the angle at which the path cuts the meridian
\begin{eqnarray}
	\therefore&&r\frac{\mathrm{d}\theta}{\mathrm{d}s}=\sin\psi\nonumber\\\mbox{i.e.,}&&r\sin\psi=\mbox{constant}
\end{eqnarray}

In case of cylinder $r=a=$constant, so $\sin\psi=$constant. This means that the path of the particle cuts all the generators at a constant angle, Hence the path is a helix.\\

So the equation of motion are
\begin{eqnarray}
	mv\frac{\mathrm{d}v}{\mathrm{d}s}&=&-\mu R\label{eq3.67}\\m\frac{v^2}{\rho_0}&=&R\label{eq3.68}\\\mbox{and~~}m\frac{v^2}{\rho_0}\tan\chi&=&0\label{eq3.69}
\end{eqnarray} 

Also $\dfrac{1}{\rho_0}=\dfrac{\cos^2\psi}{\infty}+\dfrac{\sin^2\psi}{a}=\dfrac{\sin^2\psi}{a}$.\\

From (\ref{eq3.67}) and (\ref{eq3.68}) \begin{eqnarray}
	v\frac{\mathrm{d}v}{\mathrm{d}s}&=&-\mu\frac{v^2}{\rho_0}=-\mu\frac{v^2}{a}\sin\psi\nonumber\\\mbox{i.e., }\frac{\mathrm{d}v}{\mathrm{d}t}&=&-\mu\frac{v^2}{a}\sin^2\psi\nonumber
\end{eqnarray}

On integration $\dfrac{1}{v}-\dfrac{1}{v_0}=\dfrac{\mu}{a}\sin^2\psi~t$

assuming $v=v_0$ at $t=0$, $$\therefore~v=\frac{\mathrm{d}s}{\mathrm{d}t}=\frac{1}{\frac{1}{v_0}+\frac{\mu}{a}\sin^2\psi~t}$$

So integrating once more (assuming $s=0$ at $t=0$) we have
$$s=\frac{a}{\mu\sin^2\psi}\log\left[1+\frac{\mu v_0}{a}\sin^2\psi~t\right]$$

\subsection{Discuss the motion of a heavy particle on a smooth surface of revolution having vertical axis of revolution}

We choose $O$ as the origin and vertical direction $O$ through as the $z$- axis. $P$ is the position of the particle at any time $t$. Then in cylindrical coordinates the components of the velocity and acceleration of the particle are:\\

\begin{wrapfigure}[12]{r}{0.35\textwidth}
	\centering	\includegraphics[height=4 cm , width=7 cm ]{f41.pdf}
	\begin{center}
		Fig. 3.38
	\end{center}
\end{wrapfigure} 

 velocity: $\dot{r}$, $r\dot{\theta}$, and $\dot{z}$ along radial, cross radial and along the $z$-asis. The component of the acceleration along these directions are $\ddot{r}-r\dot{\theta}^2$, $\dfrac{1}{r}\dfrac{\mathrm{d}}{\mathrm{d}t}\left(r^2\dot{\theta}\right)$ and $\ddot{z}$ respectively.\\
 
 As gravity is the only force acting on the particle so the equation of energy gives
 \begin{equation}\label{eq3.71}
 	\frac{1}{2}m\left(\dot{r}^2+r^2\dot{\theta}^2+\dot{z}^2\right)=c-mgz
 \end{equation}
 
If $\phi$ be the angle which the normal $PN$ at $P$ makes with the z-axis then equation of motion along the tangent $PT$ to the meridian curve gives
\begin{equation}\label{eq3.72}
	m\left\{\left(\ddot{r}-r\dot{\theta}^2\right)\cos\phi+\ddot{z}\sin\phi\right\}=-mg\sin\phi
\end{equation}

Also the equation of motion along the normal direction $PN$ gives
\begin{equation}\label{eq3.73}
	m\left[\ddot{z}\cos\phi-\left(\ddot{r}-r\dot{\theta}^2\right)\sin\phi\right]=R-mg\cos\theta
\end{equation}
 
 As gravity is acting on the meridian plane so there is no force along the cross redial direction (i.e., perpendicular to the meridian plane) and we have
 \begin{eqnarray}
 	&&\frac{1}{r}\frac{\mathrm{d}}{\mathrm{d}t}\left(r^2\dot{\theta}\right)=0\nonumber\\\mbox{i.e.,}&&r^2\dot{\theta}=h, \mbox{ a constant}\label{eq3.74}
 \end{eqnarray} 
which can be interpreted as the moment of momentum along the axis of revolution is conserved. Let $z=f(r)$ be the equation of the surface of revolution, then from the figure
\begin{equation}\label{eq3.75}
	\tan\phi=\frac{\mathrm{d}z}{\mathrm{d}r}=f'(r)
\end{equation}

Now using equation (\ref{eq3.74}) to eliminate $\theta$ and equation (\ref{eq3.75}) to eliminate $\phi$ from equation (\ref{eq3.72}) we have
\begin{equation}\label{eq3.76}
	\left(1+\{f'(r)\}^2\right)\frac{\mathrm{d}^2r}{\mathrm{d}t^2}+f'(r)f''(r)\left(\frac{\mathrm{d}r}{\mathrm{d}t}\right)^2-\frac{h^2}{r^3}=-gf'(r)
\end{equation}

Further writing
\begin{eqnarray}\label{eq3.77}
	\dot{r}&=&\frac{\mathrm{d}r}{\mathrm{d}\theta}\dot{\theta}=\frac{h}{r^2}\frac{\mathrm{d}r}{\mathrm{d}\theta}\nonumber\\\mbox{and }\ddot{r}&=&-\frac{2h^2}{r^5}\left(\frac{\mathrm{d}r}{\mathrm{d}\theta}\right)^2+\frac{h^2}{r^4}\frac{\mathrm{d}^2r}{\mathrm{d}\theta^2}
\end{eqnarray}

Equation (\ref{eq3.76}) reduces to
\begin{equation}\label{eq3.78}
	\frac{\left(1+\{f'(r)\}^2\right)}{r}\frac{\mathrm{d}^2r}{\mathrm{d}\theta^2}+\left\{f'(r)f''(r)-\frac{2}{r}\left[1+\{f'(r)\}^2\right]\right\}\frac{1}{r}\left(\frac{\mathrm{d}r}{\mathrm{d}\theta}\right)^2=1-\frac{gr^3}{h^2}f'(r)
\end{equation}

A first integral of the above second order equation gives
\begin{equation}\label{eq3.79}
	\left(1+\{f'(r)\}^2\right)\left(\frac{1}{r}\frac{\mathrm{d}r}{\mathrm{d}\theta}\right)^2+\left\{1+\frac{2g}{h^2}r^2f(r)\right\}=\lambda
\end{equation}
$\lambda$, being the constant of integration.\\

Note that equation (\ref{eq3.79}) can also be obtained from the energy equation (\ref{eq3.71}) by suitable elimination of the other variables involved.\\

Now, if the path of the particle is supposed to be a circle of radius $a$ then from equation (\ref{eq3.78}) we have
\begin{equation}
	ga^3f'(a)=h^2
\end{equation}

This condition for circular path can also be obtained from the following:\\

The non-zero acceleration is only along $PC$ of magnitude $\dfrac{h^2}{a^3}$ and it is maintained by the weight of the particle and the normal reaction. So equation of motion along $PC$ and $z$ axis gives
\begin{eqnarray}
	\frac{h^2}{a^3}&=&\frac{R}{m}\sin\phi\mbox{~~ and~~ }0~=~R\cos\phi-mg\nonumber\\
	\mbox{i.e.,}&&\tan\phi=\frac{h^2}{a^3g}\nonumber\\\mbox{i.e.,}&&h^2=a^3gf'(a)\nonumber
\end{eqnarray}

\section{Problems}

{\bf 25. } A particle is moving under gravity in contact with the inside of the smooth surface of revolution: $z=f(r)$, the positive $z$-direction being vertically upward. Originally the particle is describing a horizontal circle of radius $a$, its angular momentum about $z$-axis being $h$. It is then slightly disturbed so that the angular momentum in the subsequent motion is $h+\delta h$. Prove that the particle oscillates with period $\dfrac{2\pi}{n}$ where $n^2=\dfrac{g}{a}\left\{\dfrac{3f'(a)+af''(a)}{1+{\{f'(a)\}}^2}\right\}$ about the circle $r+\delta a$ where $$\delta a=\frac{2~\delta h~af'(a)}{h\{3f'(a)+af''(a)\}}$$

{\bf Solution: } From equation (\ref{eq3.76}) of the previous article
\begin{equation}\label{eq3.25.1}
	\left(1+\{f'(r)\}^2\right)\frac{\mathrm{d}^2r}{\mathrm{d}t^2}+f'(r)f''(r)\left(\frac{\mathrm{d}r}{\mathrm{d}t}\right)^2-\frac{h^2}{r^3}=-gf'(r)
\end{equation}

As the particle describe the circle: $r=a$ so we have
\begin{equation}
	\frac{h^2}{a^3}=gf'(a)
\end{equation}

Now for infinitesimal variation we have $r=a+\xi$, where $\xi$ is small and $h$ changes to $h+\delta h$ so we get from (\ref{eq3.25.1})
\begin{equation}
	\ddot{\xi}\left(1+\{f'(a+\xi)\}^2\right)-\frac{(h^2+2h\delta h)}{(a+\xi)^3}=-gf'(a+\xi)
\end{equation}
 where we have neglected $\dot{\xi}^2$ and $\delta h^2$.\\
 
 Thus we obtain
 \begin{eqnarray}
 	\ddot{\xi}&=&\frac{h^2+2h\delta h-g(a^3+3a^2\xi)\left[f'(a)+\xi f''(a)\right]}{(a^3+3a^2\xi)\left[1+\{f'(a)\}^2+2\xi f'(a)f''(a)\right]}\nonumber\\
 	&=&\frac{\cancel{a^3gf'(a)}+2h\delta h-\cancel{ga^3f'(a)}-3a^2g\xi f'(a)-ga^3\xi f''(a)}{a^3\left(1+\{f'(a)\}^2\right)+3a^2\xi\left(1+\{f'(a)\}^2\right)+2a^3\xi f'(a)f''(a)}\nonumber\\
 	&=&\frac{2h\delta h-a^2g\xi\left[3f'(a)+af''(a)\right]}{a^3\left(1+\{f'(a)\}^2\right)\left[1+\xi\left\{\frac{3}{a}+\frac{2f'(a)f''(a)}{\left(1+\{f'(a)\}^2\right)}\right\}\right]}\nonumber\\
	&=&\frac{2h\delta h-a^2g\xi\left[3f'(a)+af''(a)\right]}{a^3\left(1+\{f'(a)\}^2\right)}\left[1-\xi\left\{\frac{3}{a}+\frac{2f'(a)f''(a)}{\left(1+\{f'(a)\}^2\right)}\right\}\right]\nonumber\\
	&=&-\frac{g}{a}\frac{\left\{3f'(a)+af''(a)\right\}}{\left(1+\{f'(a)\}^2\right)}\left[\xi-\frac{2h\delta h}{a^2g\left\{3f'(a)+af''(a)\right\}}\right]\nonumber
 \end{eqnarray}

But it is given that
\begin{eqnarray}
	\delta a&=&\frac{2~\delta h~af'(a)}{h\{3f'(a)+af''(a)\}}=\frac{2h~\delta h~af'(a)}{h^2\{3f'(a)+af''(a)\}}\nonumber\\&=&\frac{2h~\delta h~af'(a)}{a^3gf'(a)\{3f'(a)+af''(a)\}}=\frac{2h~\delta h}{a^2g\{3f'(a)+af''(a)\}}\nonumber\\\therefore~\ddot{\xi}&=&-\frac{g}{a}\frac{\left\{3f'(a)+af''(a)\right\}}{\left(1+\{f'(a)\}^2\right)}\left(\xi-\delta a\right)\nonumber
\end{eqnarray} 

Putting $\eta=\xi-\delta a=r-a-\delta a$, we have
$$\ddot{\eta}=-\frac{g}{a}\frac{\left\{3f'(a)+af''(a)\right\}}{\left(1+\{f'(a)\}^2\right)}\eta$$
which represents a simple harmonic motion for which the time period is given by
$$T=\frac{2\pi}{n},~~~~~~n^2=\frac{g}{a}\left\{\frac{3f'(a)+af''(a)}{1+{\{f'(a)\}}^2}\right\}$$

{\bf 26. } A particle is moving under gravity be in contact with the inside of a rough circular cylinder with vertical generators. At time $t$ the particle is moving with velocity $V$ in a direction that makes an acute angle $\phi$ with the downward vertical. Establish the equation
$$\frac{1}{V}\frac{\mathrm{d}V}{\mathrm{d}\phi}=\frac{\mu V^2}{ag}\sin\phi-\cot\phi$$ 
where $a$ is the radius of the cylinder, $\mu$ is the coefficient of friction between the particle and the cylinder. If the velocity of the particle was $U$ when it was moving horizontally, prove that at time $t$ the horizontal component $u$ of its velocity is given by the equation
$$\frac{1}{U^2}-\frac{1}{u^2}=\frac{2\mu}{ag}\log\tan\left(\frac{\phi}{2}\right)$$

{\bf Solution: } We take $z$-axis vertical downward and along the axis of the cylinder. Let $P(a,\theta,z)$ be the cylindrical co-ordinates of the particle $P$ at time $t$. Then equation of motion are
\begin{eqnarray}
		mV\frac{\mathrm{d}V}{\mathrm{d}s}&=&-\mu R+mg\cos\phi\label{eq3.26.1}\\m\frac{V^2}{\rho_0}&=&R\label{eq3.26.2}\\m\frac{V^2}{\rho_0}\tan\chi&=&mg\sin\phi\label{eq3.26.3}
\end{eqnarray}
 where $R$ is the normal reaction of the surface. Also
 \begin{equation}\label{eq3.26.4}
 	\frac{1}{\rho_0}=\frac{\sin^2\phi}{a}
 \end{equation} 
(Note that the component of acceleration are $\ddot{r}-r\dot{\theta}^2=-a\dot{\theta}^2$, $\dfrac{1}{r}\dfrac{\mathrm{d}}{\mathrm{d}t}\left(r^2\dot{\theta}\right)=a\ddot{\theta}$ and $\ddot{z}$)

\begin{wrapfigure}[10]{r}{0.35\textwidth}
	\centering	\includegraphics[height=4 cm , width=5 cm ]{f42.pdf}
	\begin{center}
		Fig. 3.39
	\end{center}
\end{wrapfigure}

Now\begin{eqnarray}
	\frac{V^2}{\rho_0}\tan\chi&=&\mbox{acceleration along }PT'\nonumber\\&=&\ddot{z}\sin\phi+a\ddot{\theta}\cos(\pi-\theta)\nonumber\\&=&\ddot{z}\sin\phi-a\ddot{\theta}\cos\theta\nonumber
\end{eqnarray}

But $\dot{z}=V\cos\phi$, $a\dot{\theta}=V\sin\phi$.
\begin{eqnarray}
	\therefore~	\frac{V^2}{\rho_0}\tan\chi&=&\sin\phi\frac{\mathrm{d}}{\mathrm{d}t}(V\cos\phi)-\cos\phi\frac{\mathrm{d}}{\mathrm{d}t}(V\sin\phi)\nonumber\\&=&-V\frac{\mathrm{d}\phi}{\mathrm{d}t}=g\sin\phi\mbox{~ (by equation (\ref{eq3.26.3}))}\nonumber\\
	\mbox{i.e., }\dot{\phi}&=&\dot{\phi}=-\frac{g}{V}\sin\phi
\end{eqnarray} 

From (\ref{eq3.26.1})
\begin{eqnarray}
	&&\frac{\mathrm{d}V}{\mathrm{d}t}=V\frac{\mathrm{d}V}{\mathrm{d}s}-\mu\frac{R}{m}+g\cos\phi\nonumber\\\implies&&\frac{\mathrm{d}V}{\mathrm{d}\phi}\dot{\phi}=-\mu\frac{V^2}{a}\sin^2\phi+g\cos\phi\mbox{~~(using (\ref{eq3.26.2}) and (\ref{eq3.26.4}))}\nonumber\\\implies&&-\frac{g}{V}\sin\phi\frac{\mathrm{d}V}{\mathrm{d}\phi}=-\mu\frac{V^2}{a}\sin^2\phi+g\cos\phi\nonumber\\\mbox{i.e.,}&&\frac{1}{V}\frac{\mathrm{d}V}{\mathrm{d}\phi}=\frac{\mu V^2}{ag}\sin\phi-\cot\phi\label{eq3.26.6}
\end{eqnarray}

\underline{\textbf{2nd Part : }} $u=V\sin\phi$\\

Now taking log and differentiating with respect to $\phi$ we get
\begin{eqnarray}
	\frac{1}{u}\frac{\mathrm{d}u}{\mathrm{d}\phi}&=&\frac{1}{V}\frac{\mathrm{d}V}{\mathrm{d}\phi}+\cot\phi=\frac{\mu V^2}{ag}\sin\phi \mbox{~ (by equation (\ref{eq3.26.6}))}\nonumber\\&=&\frac{\mu u^2}{ag}\mbox{ cosec}\phi\nonumber\\\mbox{i.e., }\frac{2}{u^3}\mathrm{d}u&=&\frac{2\mu}{ag}\mbox{ cosec}\phi~\mathrm{d}\phi\nonumber
\end{eqnarray} 

So integrating once we obtain
\begin{eqnarray}
	\int_U^u\frac{2}{u^3}\mathrm{d}u=\frac{2\mu}{ag}\int_{\frac{\pi}{2}}^{\phi}\mbox{cosec}\phi~\mathrm{d}\phi\nonumber\\\mbox{i.e.,~ }\frac{1}{U^2}-\frac{1}{u^2}=\frac{2\mu}{ag}\log\tan\left(\frac{\phi}{2}\right)\nonumber
\end{eqnarray}

{\bf 27. } A heavy particle moves on a rough vertical circular cylinder and is projected horizontally with a velocity $V$. Prove that at the point where the path cuts the generator at an angle $\phi$, the velocity $v$ is given by
$$\frac{ag}{v^2\sin^2\phi}=\frac{ag}{V^2}+2\mu\log(\cot\phi+\mbox{cosec }\phi)$$ 
and that the azimuthal angle $\theta$ and vertical decend $z$ are given by
$$ag\theta=-\int v^2~\mathrm{d}\phi\mbox{~ and~ }gz=-\int v^2\cot\phi~\mathrm{d}\phi$$ 
(the limits being $\dfrac{\pi}{2}$ to $\phi$)\\

{\bf Solution: } From the previous problem (see equation (\ref{eq3.26.6}))
\begin{equation}\label{eq3.27.1}
	\frac{1}{v}\frac{\mathrm{d}v}{\mathrm{d}\phi}=\frac{\mu v^2}{ag}\sin\phi-\cot\phi
\end{equation}

Now, \begin{eqnarray}
	\mathrm{d}\theta&=&\frac{\mathrm{d}\theta}{\mathrm{d}\phi}=\frac{\dot{\theta}}{\dot{\phi}}\mathrm{d}\phi\nonumber\\\mbox{i.e., }a\mathrm{d}\theta&=&a\frac{\dot{\theta}}{\dot{\phi}}\mathrm{d}\phi=\frac{v\sin\phi}{-\frac{g}{v}\sin\phi}\mathrm{d}\phi\nonumber\\\mbox{i.e., }ag\mathrm{d}\theta&=&-v^2\mathrm{d}\phi\nonumber
\end{eqnarray}

Integrating we get
\begin{eqnarray}
	ag\int_0^\theta\mathrm{d}\theta&=&-\int_\frac{\pi}{2}^\phi v^2~\mathrm{d}\phi\nonumber\\\mbox{i.e., ~~}ag\theta&=&-\int_\frac{\pi}{2}^\phi v^2~\mathrm{d}\phi\nonumber
\end{eqnarray}

Also, $\mathrm{d}z=\dfrac{\mathrm{d}z}{\mathrm{d}\phi}\mathrm{d}\phi=\dfrac{\dot{z}}{\dot{\phi}}\mathrm{d}\phi=\dfrac{V\cos\phi}{-\frac{g}{V}\sin\phi}\mathrm{d}\phi=-\dfrac{V^2}{g}\cot\phi$
\begin{eqnarray}
	\therefore&&g\int_0^z\mathrm{d}z=-\int_\frac{\pi}{2}V^2\cot\phi\mathrm{d}\phi\nonumber\\\mbox{i.e.,}&&gz=-\int_\frac{\pi}{2}V^2\cot\phi\mathrm{d}\phi\nonumber
\end{eqnarray}

Also from (\ref{eq3.27.1})
$$\frac{1}{v^3}\frac{\mathrm{d}v}{\mathrm{d}\phi}+\frac{1}{v^2}\cot\phi=\frac{\mu }{ag}\sin\phi$$

Put $\lambda=\dfrac{1}{v^2}$, $\dfrac{\mathrm{d}\lambda}{\mathrm{d}\phi}=-\dfrac{2}{v^3}\dfrac{\mathrm{d}v}{\mathrm{d}\phi}$
\begin{eqnarray}
	\therefore&&\frac{\mathrm{d}\lambda}{\mathrm{d}\phi}-2\lambda\cot\phi=-\frac{2\mu}{ag}\sin\phi\nonumber\\\implies&&\lambda e^{-\int2\cot\phi~\mathrm{d}\phi}=-\frac{2\mu}{ag}\int\sin\phi e^{-\int2\cot\phi~\mathrm{d}\phi}~\mathrm{d}\phi+c\nonumber\\\implies&&\lambda\mbox{ cosec}^2\phi=-\frac{2\mu}{ag}\int\mbox{ cosec}\phi~\mathrm{d}\phi+c\nonumber\\\implies&&\frac{1}{v^2}\mbox{ cosec}^2\phi=\frac{2\mu}{ag}\log(\mbox{cosec }\phi+\cot\phi)+c\nonumber
\end{eqnarray}

But $v=V$, $\phi=\dfrac{\pi}{2}$ $\implies$ $c=\dfrac{1}{v^2}$.
$$\therefore~\frac{ag}{v^2\sin^2\phi}=\frac{ag}{V^2}+2\mu\log(\cot\phi+\mbox{cosec }\phi)$$

\section{Discuss the motion of a particle on a rough right circular cone under no external force}

We take vertex of the cone as origin, $z$-axis along the axis of the cone and $(r,\theta,z)$ be the cylindrical coordinates of the position of the particle at $P$, on the surface of the cone at any time $t$.\\

\begin{wrapfigure}[14]{r}{0.35\textwidth}
	\centering	\includegraphics[height=5 cm , width=4.5 cm ]{f43.pdf}
	\begin{center}
		Fig. 3.40
	\end{center}
\end{wrapfigure}

Let the path of the particle at $P$ cuts the generator $OP$ of the cone at an angle $\psi$. As there are no external forces acting on the particle, so the path must be geodesic on the cone. Also we have
\begin{equation}
	r\sin\psi=\mbox{constant}=K(\mbox{say})\label{eq3.91}
\end{equation}

The equations of motion are
\begin{equation}
		mv\frac{\mathrm{d}v}{\mathrm{d}s}=-\mu\mbox{~ (along the path)} \label{eq3.92}
	\end{equation}and
\begin{equation} 
	m\frac{v^2}{\rho_0}=R\mbox{~ (along the normal to the surface)}\label{eq3.93}
\end{equation}
 where $R$ is the reaction of the surface, $\mu$ is the coefficient of friction, $v$ is the velocity of the particle at $P$ and $\rho_0$ is the radius of curvature of the normal section of the surface through tangent to the path at $P$.\\
 
Therefore,
\begin{equation}
	\frac{1}{\rho_0}=\frac{\cos^2\psi}{\infty}+\frac{\sin^2\psi}{PN}=\frac{\sin^2\psi}{r\sec\alpha}\label{eq3.94}
\end{equation}

Now eliminating $R$ between (\ref{eq3.92}) and (\ref{eq3.93}) and using $\rho_0$ from (\ref{eq3.94}) we get 
\begin{eqnarray}
	mv\frac{\mathrm{d}v}{\mathrm{d}s}&=&-\mu mv^2\frac{\sin^2\psi}{r\sec\alpha}\nonumber\\\mbox{i.e., }v\frac{\mathrm{d}v}{\mathrm{d}s}&=&-\mu v^2\cos\alpha\frac{K^2}{r^3}\label{eq3.95} ~~\mbox{using (\ref{eq3.92})}
\end{eqnarray}

\begin{wrapfigure}[10]{r}{0.35\textwidth}
	\centering	\includegraphics[height=4 cm , width=4.5 cm ]{f44.pdf}
	\begin{center}
		Fig. 3.41
	\end{center}
\end{wrapfigure}

Let $OP=r'$, $C_1P=r$, $PQ=\mathrm{d}r'$.\\

So $r=r'\sin\alpha$ ~i.e.,~ $\mathrm{d}r=\mathrm{d}r'\sin\alpha$.
\begin{eqnarray}
	\therefore&&\mathrm{d}s\cos\psi=\mathrm{d}r'=\mathrm{d}r\mbox{ cosec }\alpha\nonumber\\\therefore&&\mathrm{d}s=\frac{\mathrm{d}r\mbox{ cosec }\alpha}{\sqrt{1-\frac{K^2}{r^2}}}=\mbox{cosec }\alpha\frac{r\mathrm{d}r}{\sqrt{r^2-K^2}}\nonumber
\end{eqnarray}

on integration,
\begin{equation}\nonumber
	s=\sqrt{r^2-K^2}\mbox{cosec }\alpha+K'
\end{equation}

If we assume $s=0$, $r=K$ then $K'=0$
\begin{equation}
	\therefore~s^2\sin^2\alpha+K^2=r^2\nonumber
\end{equation}

Using this expression for $r$ in  (\ref{eq3.95}) we get
\begin{eqnarray}
	v\frac{\mathrm{d}v}{\mathrm{d}s}&=&-\mu v^2\cos\alpha\frac{K^2}{(s^2\sin^2\alpha+K^2)^\frac{3}{2}}\nonumber\\\mbox{i.e., }\frac{\mathrm{d}v}{v}&=&-\mu K^2 \cos\alpha\frac{\mathrm{d}s}{(s^2\sin^2\alpha+K^2)^\frac{3}{2}}\nonumber
\end{eqnarray} 

Now integrating
\begin{eqnarray}
	\int_{v_0}^v\frac{\mathrm{d}v}{v}&=&-\mu K^2 \cos\alpha\int_0^s\frac{\mathrm{d}s}{(s^2\sin^2\alpha+K^2)^\frac{3}{2}}\nonumber\\\mbox{i.e., }\ln\frac{v}{v_0}&=&\frac{\mu s}{\tan\alpha}\left(s^2+d^2\right)^{-\frac{1}{2}}\nonumber
\end{eqnarray}

{\bf 28. } A particle moves on a rough surface under no force but the reaction of the surface. The surface is one of revolution who is meridian curves are catenaries with the axis for the ditectrix and the particle is projected from the equator making an angle $\alpha$ with the meridian. Prove that the velocity of the particle when it crosses the meridian at an angle $\phi$ satisfies the equation
\begin{equation}
	\left(\frac{\mathrm{d}v}{\mathrm{d}\phi}\right)^2=\frac{2\mu^2v^2\cos^22\phi}{\cos2\phi-\cos2\alpha}, ~~~\mu\mbox{ being the coefficient of friction}\nonumber
\end{equation}

{\bf Solution: } The equation of the meridian curve be
\begin{equation}
	r=c\sec\psi
\end{equation}

\begin{wrapfigure}[10]{r}{0.35\textwidth}
	\centering	\includegraphics[height=4.5 cm , width=4.5 cm ]{f45.pdf}
	\begin{center}
		Fig. 3.42
	\end{center}
\end{wrapfigure}

Since the particle is under the action of no forces so the path will be a geodesic on the surface. If the path at $P$ cuts the meridian section at an angle $\phi$ then
\begin{eqnarray}
	&&r\sin\phi=\mbox{constant}=c\sin\alpha\nonumber\\\therefore&&\cos\psi=\frac{c}{r}=\frac{\sin\phi}{\sin\alpha}
\end{eqnarray}

The equation of motion along the path is
\begin{equation}
	mv\frac{\mathrm{d}v}{\mathrm{d}s}=-\mu R
\end{equation}
 and that along the axis ($z$-axis) is
 \begin{equation}
 	m\ddot{z}=-R\sin\psi-\mu R\cos\phi\cos\psi
 \end{equation}
 
Now, $\dot{z}=v\cos\phi\cos\psi$
 
so $\ddot{z}=\dot{v}\cos\phi\cos\psi+v\dfrac{\mathrm{d}}{\mathrm{d}t}(\cos\phi\cos\psi)$

Therefore 
\begin{eqnarray}
	m\dot{v}\cos\phi\cos\psi&=&m\ddot{z}-mv\dfrac{\mathrm{d}}{\mathrm{d}t}(\cos\phi\cos\psi)\nonumber\\&=&-R\left[\sin\psi+\mu\cos\phi\cos\psi\right]-mv\dfrac{\mathrm{d}}{\mathrm{d}t}(\cos\phi\cos\psi)\nonumber\\&=&\frac{m\dot{v}}{\mu}\left(\sin\psi+\mu\cos\phi\cos\psi\right)-mv\dfrac{\mathrm{d}}{\mathrm{d}t}\left(\cos\phi\frac{\sin\phi}{\sin\alpha}\right)\nonumber\\\mbox{i.e.,~ }\dot{v}\sin\psi&=&\mu v\dfrac{\mathrm{d}}{\mathrm{d}t}\left(\cos\phi\frac{\sin\phi}{\sin\alpha}\right)\nonumber\\\mbox{i.e.,~ }\frac{\mathrm{d}v}{\mathrm{d}\phi}\dot{\phi}\sin\psi&=&\frac{\mu v}{\sin\alpha}\cos2\phi\dot{\phi}\nonumber\\\therefore~\left(\frac{\mathrm{d}v}{\mathrm{d}\phi}\right)^2&=&\frac{\mu^2v^2\cos^22\phi}{\sin^2\alpha(1-\cos^2\psi)}=\frac{\mu^2v^2\cos^22\phi}{\sin^2\alpha-\sin^2\phi}=\frac{2\mu^2v^2\cos^22\phi}{\cos2\phi-\cos2\alpha}\nonumber
\end{eqnarray}

{\bf 29. } A particle moves on a helical wire whose axis is vertical. Prove that the velocity $v$ after describing an arc $s$ is given by the equation
$$v^2=ag\sec\alpha\sinh\phi~, ~~\frac{\mathrm{d}s}{\mathrm{d}\phi}=\frac{a}{2}\frac{\sec^2\alpha\cosh\phi}{(\tan\alpha-\mu\cosh\phi)}$$ where, $a$ is the radius of the cylinder on which the helix lies, $\alpha$ is the inclination of the helix to the horizon and $\mu$ is the coefficient of friction.\\

{\bf Solution: }  Resolving the forces acting on the particle along the tangent, principal normal and binormal to the path, we have the equations of motion
\begin{eqnarray}
	mv\frac{\mathrm{d}v}{\mathrm{d}s}&=&mg\sin\alpha-\mu R\nonumber\\m\frac{v^2}{\rho}=R_N&,&~~0=R_B-mg\cos\alpha\nonumber
\end{eqnarray} where $R$ is the reaction of the wire on the particle and $R_N$, $R_B$ the component of $R$ along the principal normal and the binormal respectively. As the path is helix so the principal normal and normal to the surface at $P$ coincide.\\

\begin{wrapfigure}[10]{r}{0.35\textwidth}\vspace{-.8cm}
	\centering	\includegraphics[height=4.7 cm , width=4.5 cm ]{f46.pdf}
	\begin{center}
		Fig. 3.43
	\end{center}
\end{wrapfigure}

 Now, $\rho=\rho_0\cos\psi=\rho_0$, the radius of curvature of the normal section of the cylinder through $P$ so we have
 \begin{eqnarray}
 	\frac{1}{\rho}&=&\frac{1}{\rho_0}=\frac{\cos^2\left(\frac{\pi}{2}-\alpha\right)}{\infty}+\frac{\sin^2\left(\frac{\pi}{2}-\alpha\right)}{a}=\frac{\cos^2\alpha}{a}\nonumber\\\therefore R_N&=&m\frac{v^2}{a}\cos\alpha, R_B=mg\cos\alpha\nonumber\\\therefore R^2&=&R_N^2+R_B^2=m^2g^2\cos^2\alpha\left(1+\frac{v^4\cos^2\alpha}{a^2g^2}\right)\nonumber
 \end{eqnarray}
 
 As $v^2=ag\sec\alpha\sinh\phi$, so $R=mg\cos\alpha\cosh\phi$.\\
 
 Hence $v\dfrac{\mathrm{d}v}{\mathrm{d}s}=\dfrac{ag}{2}\sec\alpha\cosh\phi\dfrac{\mathrm{d}\phi}{\mathrm{d}s}$.\\
 
 Thus from the equation of motion along the path we get
 \begin{eqnarray}
 	v\frac{\mathrm{d}v}{\mathrm{d}s}&=&g\sin\alpha-\mu g\cos\alpha\cosh\phi\nonumber\\\mbox{i.e., }\frac{\mathrm{d}s}{\mathrm{d}\phi}&=&\frac{a}{2}\frac{\sec\alpha\cosh\phi}{\sin\alpha-\mu\cos\alpha\cosh\phi}=\frac{a}{2}\frac{\sec^2\alpha\cosh\phi}{(\tan\alpha-\mu\cosh\phi)}\nonumber
 \end{eqnarray}
 
\section{Motion of a particle on a smooth revolving surface}

\begin{wrapfigure}[11]{r}{0.35\textwidth}
	\centering	\includegraphics[height=4.5 cm , width=4.5 cm ]{f47.pdf}
	\begin{center}
		Fig. 3.44
	\end{center}
\end{wrapfigure}

Let $ox$, $oy$, $oz$ be the coordinate axes fixed relative to the surface and the surface is moving about an axis through $O$ with an angular velocity $\omega$. Let $\phi(x,y,z)=0$ be the equation of the surface and $\vec{r}=x\vec{i}+y\vec{j}+z\vec{k}$ be the position of the particle at time $t$, where $\left(\vec{i},\vec{j},\vec{k}\right)$ are the unit vectors along the co-ordinate axes. Let $m$ be the mass of the particle, then the equation of motion is $$m\frac{\mathrm{d}^2\vec{r}}{\mathrm{d}t^2}=\vec{F}+\vec{R}$$
where $\vec{F}$ is the external force and $\vec{R}$ is the reaction of the surface surface. Suppose $\vec{\omega}=\left(\omega_x,\omega_y,\omega_z\right)$ be the angular velocity of rotation then
$$\frac{\mathrm{d}\vec{r}}{\mathrm{d}t}=\frac{\partial\vec{r}}{\partial t}+\vec{\omega}\times\vec{r}$$

So,
\begin{eqnarray}
	\dfrac{\mathrm{d}^2\vec{r}}{\mathrm{d}t^2}&=&\frac{\mathrm{d}}{\mathrm{d}t}\left(\frac{\mathrm{d}\vec{r}}{\mathrm{d}t}\right)=\frac{\partial}{\partial t}\left(\frac{\mathrm{d}\vec{r}}{\mathrm{d}t}\right)+\vec{\omega}\times\frac{\mathrm{d}\vec{r}}{\mathrm{d}t}\nonumber\\&=&\frac{\partial}{\partial t}\left(\frac{\partial\vec{r}}{\partial t}+\vec{\omega}\times\vec{r}\right)+\vec{\omega}\times\left(\frac{\partial\vec{r}}{\partial t}+\vec{\omega}\times\vec{r}\right)\nonumber\\&=&\frac{\partial^2\vec{r}}{\partial t^2}+\dot{\vec{\omega}}\times\vec{r}+2\vec{\omega}\times\frac{\partial\vec{r}}{\partial t}+\vec{\omega}\times\left(\vec{\omega}\times\vec{r}\right)\nonumber\\&=&\left(\ddot{x},\ddot{y},\ddot{z}\right)+\begin{vmatrix}
		\vec{i}&\vec{j}&\vec{k}\\\dot{\omega}_x&\dot{\omega}_y&\dot{\omega}_z\\x&y&z
	\end{vmatrix}+2\begin{vmatrix}
	\vec{i}&\vec{j}&\vec{k}\\\omega_x&\omega_y&\omega_z\\\dot{x}&\dot{y}&\dot{z}
\end{vmatrix}+\begin{vmatrix}
\vec{i}&\vec{j}&\vec{k}\\\omega_x&\omega_y&\omega_z\\z\omega_y-y\omega_z&x\omega_z-z\omega_x&y\omega_x-x\omega_y
\end{vmatrix}\nonumber\\&=&\left(f_x,f_y,f_z\right)\nonumber
\end{eqnarray}

Hence the equations of motion are
$$mf_x=F_x+R_x~,~~mf_y=F_y+R_y~,~~mf_z=F_z+R_z$$

 \section{Motion of a particle on a smooth revolving plane}

Let $OI$ be the axis of rotation and $OZ$ be the normal to the plane at $O$ making an angle $\alpha$ with $OI$. Let the plane $IOZ$ cut the given plane along $OX$. We take $X$-axis along $OX$ and $Z$-axis along $OZ$, the axis $Y$ is taken perpendicular to $OX$ and $OZ$. Let the plane rotate about $OI$ with uniform angular velocity $\omega$ so that $\dot{\omega}=0$. Let $P(x,y,0)$ be the position of the particle at time $t$. So we write

\begin{wrapfigure}[14]{r}{0.35\textwidth}\vspace{-.5cm}
	\centering	\includegraphics[height=4.5 cm , width=4.5 cm ]{f48.pdf}
	\begin{center}
		Fig. 3.45
	\end{center}
\end{wrapfigure}

$$\vec{r}=x\vec{i}+y\vec{j}+0\vec{k}$$
be the position of the particle and 
$$\vec{\omega}=\left(\omega_x,0,\omega_z\right)~,~~\omega_x=-\omega\sin\alpha~,~~\omega_z=\omega\cos\alpha$$
be the angular velocity of rotation.\\

Now, \begin{eqnarray}
	\dfrac{\mathrm{d}^2\vec{r}}{\mathrm{d}t^2}&=&\frac{\partial^2\vec{r}}{\partial t^2}+\dot{\vec{\omega}}\times\vec{r}+2\vec{\omega}\times\frac{\partial\vec{r}}{\partial t}+\vec{\omega}\times\left(\vec{\omega}\times\vec{r}\right)\nonumber\\&=&\left(\ddot{x},\ddot{y},0\right)+0+2\begin{vmatrix}
		\vec{i}&\vec{j}&\vec{k}\\\omega_x&\omega_y&\omega_z\\\dot{x}&\dot{y}&\dot{z}
	\end{vmatrix}+\begin{vmatrix}
		\vec{i}&\vec{j}&\vec{k}\\\omega_x&\omega_y&\omega_z\\-y\omega_z&x\omega_z&y\omega_x-x\omega_y
	\end{vmatrix}\nonumber\\&=&\left(f_x,f_y,f_z\right)\nonumber
\end{eqnarray}

So the equations of motion are $$mf_x=F_x+R_x~,~~mf_y=F_y+R_y~,~~mf_z=F_z+R_z$$
 
{\bf 30. } A particle slides on a smooth tube which is made to rotate with uniform angular velocity $\omega$ about a vertical axis. If the particle starts from relative rest from the point where the shortest distance between the axis and the tube meets the tube. Show that in time $t$ the particle has moved through a distance $\dfrac{2g}{\omega^2}\cot\alpha\mbox{ cosec }\alpha\sinh^2\left(\dfrac{1}{2}\omega\sin\alpha t\right)$ where $\alpha$ is the inclination of the tube to the vertical.\\

{\bf Solution: } Let the line of shortest distance between the axis of rotation and the tube be taken as axis of $y$. $z$-axis is taken parallel to the tube through $o$. $z$-axis is along $oz$ making an angle $\dfrac{\pi}{2}-\alpha$ with the axis of rotation. So the components of angular velocity are
$$\omega_x=-\omega\cos\alpha~,~~\omega_y=0~,~~\omega_z=\omega\sin\alpha$$ 

Let $OO'=a$, so the position of the particle at time $t$ is give by $(x,a,0)$. Then equation of motion about $x$-axis is
$$mf_x=mg\cos\alpha~,~~\mbox{i.e., ~}f_x=g\cos\alpha$$

Now
\begin{eqnarray}
	(f_x,f_y,f_z)=\frac{\mathrm{d}^2\vec{r}}{\mathrm{d}t^2}&=&\left(\ddot{x}\vec{i}+\ddot{y}\vec{j}+z\vec{k}\right)+2\begin{vmatrix}
		\vec{i}&\vec{j}&\vec{k}\\\omega_x&\omega_y&\omega_z\\\dot{x}&\dot{y}&\dot{z}
	\end{vmatrix}+\begin{vmatrix}
		\vec{i}&\vec{j}&\vec{k}\\\omega_x&\omega_y&\omega_z\\-y\omega_z&x\omega_z&y\omega_x-x\omega_y
	\end{vmatrix}\nonumber\\\therefore~f_x&=&\ddot{x}-\omega\sin\alpha(x\omega\sin\alpha)=\ddot{x}-\omega^2\sin^2\alpha x\nonumber\\\therefore&&\ddot{x}-\omega^2\sin^2\alpha x=g\cos\alpha\nonumber
\end{eqnarray}

\begin{wrapfigure}[10]{r}{0.35\textwidth}
	\centering	\includegraphics[height=4.5 cm , width=4.5 cm ]{f49.pdf}
	\begin{center}
		Fig. 3.46
	\end{center}
\end{wrapfigure}


The general solution is
\begin{eqnarray}
	x&=&A\cosh(\omega\sin\alpha t)+B\sinh(\omega\sin\alpha t)-\frac{g}{\omega^2}\mbox{ cosec }\alpha\cot\alpha\nonumber\\\therefore~\dot{x}&=&\omega\sin\alpha\left[A\sinh(\omega\sin\alpha t)+B\cosh(\omega\sin\alpha t)\right]\nonumber
\end{eqnarray}

Initial conditions: $x=0$, $\dot{x}=0$ at $t=0$.
\begin{eqnarray}
	\therefore0&=&A-\frac{g}{\omega^2}\mbox{ cosec }\alpha\cot\alpha\implies A=\frac{g}{\omega^2}\mbox{ cosec }\alpha\cot\alpha\nonumber\\0&=&B\nonumber\\
	\therefore~x&=&\frac{g}{\omega^2}\mbox{ cosec }\alpha\cot\alpha\left[\cosh(\omega\sin\alpha t)-1\right]\nonumber\\&=&\dfrac{2g}{\omega^2}\cot\alpha\mbox{ cosec }\alpha\sinh^2\left(\dfrac{1}{2}\omega\sin\alpha t\right)\nonumber
\end{eqnarray}

{\bf 31. } A particle is revolving on a smooth plane about a centre of force, with a force $\mu$ times the distance from it. When the body arrives at an apse the plane begins to revolve with an angular velocity $\dfrac{\sqrt{3\mu}}{2}$ about the apsidal line. Show that the subsequent orbit describe on the plane will be a portion of a parabola and that when the particle leaves the plane its velocity will be $\sqrt{3}$ times velocity at the vertex of the parabola.\\

{\bf Solution: } Let $O$ be the centre of force and $A$ be the apse. $OA=a$ be the apsidal distance. We choose the apse line $OA$ as the $x$-axis and $OY$ in the plane but perpendicular to $x$-axis is considered as $y$-axis. $oz$ is normal to the plane and is taken as $z$-axis. Let $P(x,y,0)$ be the position of the particle at any time $t$. Thus
\begin{eqnarray}
	\vec{F}&=&(-m\mu x,-m\mu y,0)\mbox{ be the external force,}\nonumber\\
		\vec{R}&=&(0,0,R)\mbox{ be the normal reation,}\nonumber\\
			\vec{r}&=&\overrightarrow{OP}=(x,y,0)\mbox{ be the position vector of $P$,}\nonumber\\
				\vec{\omega}&=&(\omega,0,0)\mbox{ be the angular velocity of rotation.}\nonumber
\end{eqnarray}

\begin{wrapfigure}[9]{r}{0.35\textwidth}
	\centering	\includegraphics[height=4.5 cm , width=4.5 cm ]{f50.pdf}
	\begin{center}
		Fig. 3.47
	\end{center}
\end{wrapfigure}

Thus the equation of motion is
\begin{equation}
	m\frac{\mathrm{d}^2\vec{r}}{\mathrm{d}t^2}=\vec{F}+\vec{R}\nonumber
\end{equation}

Now,\begin{eqnarray}
	\frac{\mathrm{d}\vec{r}}{\mathrm{d}t}&=&\frac{\partial\vec{r}}{\partial t}+\vec{\omega}\times\vec{r}=(\dot{x},\dot{y},0)+\begin{vmatrix}
		\vec{i}&\vec{j}&\vec{k}\\\omega&0&0\\x&y&0
	\end{vmatrix}\nonumber\\&=&(\dot{x},\dot{y},\omega y)\nonumber\\\therefore~	\frac{\mathrm{d}^2\vec{r}}{\mathrm{d}t^2}&=&(\ddot{x},\ddot{y},\omega\dot{y})+\begin{vmatrix}
	\vec{i}&\vec{j}&\vec{k}\\\omega&0&0\\\dot{x}&\dot{y}&\omega y
\end{vmatrix}\nonumber\\&=&(\ddot{x},\ddot{y}-\omega^2y,2\omega\dot{y})\nonumber
\end{eqnarray}

Hence the component wise equations of motion are
\begin{eqnarray}
	m\ddot{x}&=&-m\mu x\label{eq3.31.1}\\m(\ddot{y}-\omega^2y)&=&-m\mu y\label{eq3.31.2}\\\mbox{and~~ ~}2m\omega\dot{y}&=&R\label{eq3.31.3}
\end{eqnarray}

Solution of (\ref{eq3.31.1}) gives: 
\begin{equation} x=A\cos(\sqrt{\mu}t)+B\sin(\sqrt{\mu}t)\label{eq3.31.4}
\end{equation}

Solution of (\ref{eq3.31.2}) gives: 
\begin{equation} y=A'\cos\left(\frac{\sqrt{\mu}}{2}t\right)+B'\sin\left(\frac{\sqrt{\mu}}{2}t\right)\label{eq3.31.5}
\end{equation}

The initial conditions are:
$$x=a,~y=0,~\dot{x}=0,~\dot{y}=V\mbox{ (say) at } t=0$$

This gives from (\ref{eq3.31.4}) and (\ref{eq3.31.5})
\begin{eqnarray}
	A=a,~B=0,~A'=0,~B'=\frac{2V}{\sqrt{\mu}}\nonumber\\\therefore~x=a\cos\sqrt{\mu}t~,y=\frac{2V}{\sqrt{\mu}}\sin\left(\frac{\sqrt{\mu}}{2}t\right)\nonumber
\end{eqnarray} 

So eliminating $t$ we obtain
$$x-a=-\dfrac{\mu ay^2}{2V^2}$$
which describes a parabola with $(a,0)$ as vertex. \\

Now, when when the particle leaves the surface then $R=0$, so from (\ref{eq3.31.4}) $\dot{y}=0$ i.e., $\cos\left(\dfrac{\sqrt{\mu}}{2}t\right)=0$ i.e., $t=\dfrac{\pi}{\sqrt{\mu}}$.\\

Thus when the particle leaves the surface, then $\dot{x}=0=\dot{y}$ and the  velocities are $(\dot{x},\dot{y},\omega y)=\left(0,0,\dfrac{\sqrt{3\mu}}{2}\dfrac{2V}{\sqrt{\mu}}\right)=\sqrt{3}V=\sqrt{3}$ times the velocity at the vertex.

\section{Motion of a particle in a plane wire or a fine tube rotating about a point in its plane}

Let $C$ be the plane curve and let the plane rotate about an axis perpendicular to the plane with angular velocity $\omega$. This axis of rotation is chosen as $z$-axis and axes $x$ and $y$ are taken on the plane fixed relative to the curve. Let at time $t$ the $x$-axis $ox$ makes an angle $\psi$ with a fixed line $ox'$ in the plane. By condition, $\dot{\psi}=\omega$ and suppose $\vec{r}=\vec{r}(s)$ be the equation of the curve with $\vec{r}=\overrightarrow{OP}$. Thus the equation of motion of the particle is
\begin{equation}
	m\frac{\mathrm{d}^2\vec{r}}{\mathrm{d}t^2}=\vec{F}+\vec{R}\label{eq3.105}
\end{equation}
 where $\vec{F}$ is the external force and $\vec{R}$ is the reaction of the wire.\\
 
 \begin{wrapfigure}[9]{r}{0.35\textwidth}
 	\centering	\includegraphics[height=4.5 cm , width=4.5 cm ]{f51.pdf}
 	\begin{center}
 		Fig. 3.48
 	\end{center}
 \end{wrapfigure}
 
 Now,\begin{eqnarray}
 	\frac{\mathrm{d}\vec{r}}{\mathrm{d}t}&=&\frac{\partial\vec{r}}{\partial t}+\vec{\omega}\times\vec{r}\nonumber\\
	\dfrac{\mathrm{d}^2\vec{r}}{\mathrm{d}t^2}&=&\frac{\partial^2\vec{r}}{\partial t^2}+\dot{\vec{\omega}}\times\vec{r}+2\vec{\omega}\times\frac{\partial\vec{r}}{\partial t}+\vec{\omega}\times\left(\vec{\omega}\times\vec{r}\right)\nonumber\\&=&\frac{\partial^2\vec{r}}{\partial t^2}+\dot{\omega}r\vec{e}+2\dot{s}\omega\vec{N}+(\vec{\omega}\cdot\vec{r})\vec{\omega}-\omega^2\vec{r}\nonumber \\ 	
	\label{eq3.106}
 \end{eqnarray} 
where $\vec{e}$ is the unit vector along the cross radial direction, $\vec{N}$ is the unit vector along the normal to the curve and $\vec{T}$ is the unit vector along the tangent to the curve. Using (\ref{eq3.106}) in (\ref{eq3.105}) we have
\begin{eqnarray}
	m\frac{\partial^2\vec{r}}{\partial t^2}&=&\vec{F}+\vec{R}-m\dot{\omega}r\vec{e}-2m\dot{s}\omega\vec{N}+m\omega^2\vec{r}\nonumber\\&&~~~~(\vec{\omega}\cdot\vec{r}=0 \mbox{ as $\vec{\omega}$ and $\vec{r}$ are orthogonal})\nonumber
\end{eqnarray} 

This gives the motion of the particle relative to the curve. Thus the motion of the particle in a revolving curve is the same as if the curve were at rest and the following forces are added to the acting forces namely (i) a radial force across $m\omega^2r$, (ii) a  cross-radial force $-m\dot{\omega}r$ and (iii) a normal force $-2m\dot{s}\omega$. Hence we write
\begin{eqnarray}
	m\frac{\partial^2\vec{r}}{\partial t^2}&=&mv\frac{\mathrm{d}v}{\mathrm{d}s}\vec{T}+m\frac{v^2}{\rho}\vec{N}\nonumber\\\mbox{and }\vec{F}&=&F_T\vec{T}+F_N\vec{N}\nonumber
\end{eqnarray}
 where $\rho$is the radius of curvature of the curve at $P$.\\
 
 Thus\begin{eqnarray}
 	mv\frac{\mathrm{d}v}{\mathrm{d}s}\vec{T}+m\frac{v^2}{\rho}\vec{N}&=&F_T\vec{T}+F_N\vec{N}+R\vec{N}-m\dot{\omega}r\sin\phi\vec{T}-m\dot{\omega}r\cos\phi\vec{N}\nonumber\\&&-2m\dot{s}\omega\vec{N}+m\omega^2r\cos\phi\vec{T}-m\omega^2r\sin\phi\vec{N}\nonumber
 \end{eqnarray} 
or in component along the tangential and normal we have
\begin{eqnarray}
		mv\frac{\mathrm{d}v}{\mathrm{d}s}&=&F_T-m\dot{\omega}r\sin\phi+m\omega^2r\cos\phi\nonumber\\	m\frac{v^2}{\rho}&=&F_N+R-m\dot{\omega}r\cos\phi-2m\dot{s}\omega-m\omega^2r\sin\phi\nonumber
\end{eqnarray}

For uniform angular velocity we have $\dot{\omega}=0$ and hence
\begin{equation}
	mv\frac{\mathrm{d}v}{\mathrm{d}s}=F_T+m\omega^2r\cos\phi=F_T+m\omega^2r\frac{\mathrm{d}r}{\mathrm{d}s}\nonumber
\end{equation} 

Further if  the external forces are conservative i.e., $\vec{F}=-\mbox{grad } V$ then $F_T=-\dfrac{\partial V}{\partial s}$. So the above tangential equation of motion becomes
\begin{equation}
		mv\frac{\mathrm{d}v}{\mathrm{d}s}=-\dfrac{\partial V}{\partial s}+m\omega^2r\frac{\mathrm{d}r}{\mathrm{d}s}\nonumber
\end{equation} 

on integration
\begin{eqnarray}
	&&\frac{1}{2}mv^2=-V+\frac{1}{2}m\omega^2r^2+c\nonumber\\\mbox{i.e., }&&\frac{1}{2}mv^2+V'=c~,~~~V'=V-\frac{1}{2}m\omega^2r^2\nonumber
\end{eqnarray}

Here $V'$ is known as the modified potential energy and the above equation is known as the modified equation of the energy.\\

{\bf 32. } A bead that moves on a circular wire and initially at rest at a point $A$. The wire is made to rotate uniformly in its own plane with angular velocity $\omega$ about the other end of the diameter through $A$. Show that the pressure between the bead and the wire vanishes at a time $\dfrac{1}{\omega}\log_e\left(\dfrac{3+\sqrt{5}}{2}\right)$ after the start.\\

{\bf Solution: } When the curve is reduced to rest the equation of motion along the tangent and normal to the curve at $P$ are
\begin{eqnarray}
	mv\frac{\mathrm{d}v}{\mathrm{d}s}&=&m\omega^2r\frac{\mathrm{d}r}{\mathrm{d}s}\label{eq3.32.1}\\m\frac{v^2}{a}&=&R-2m\omega v-m\omega^2r\cos\theta\label{eq3.32.2}
\end{eqnarray}

 \begin{wrapfigure}[9]{r}{0.35\textwidth}
	\centering	\includegraphics[height=4.5 cm , width=4.5 cm ]{f52.pdf}
	\begin{center}
		Fig. 3.49
	\end{center}
\end{wrapfigure}

Integrating (\ref{eq3.32.1}) gives $v^2=\omega^2 r^2+c$.\\

Initially, the particle was at rest at $A$, when the wire is made to rotate about $O$ with angular velocity $\omega$, the velocity of the bead  relative to the wire is $-2a\omega$. Thus $v=-2a\omega$, $r=2a$ $\implies$ $c=0$. \\

$\therefore v=\pm\omega r$, but $v=-2a\omega$ initially, so $v=-\omega r$ at any time $t$.\\

From equation (\ref{eq3.32.2}), the reaction vanishes when
\begin{eqnarray}
	\frac{v^2}{a}&=&-2\omega v-\omega^2r\cos\theta\nonumber\\\mbox{i.e., }\frac{\omega^2r^2}{a}&=&2\omega^2r-\omega^2r\frac{r}{2a} ~~~(\because r=2a\cos\theta, \mbox{ is the equation of the circle})\nonumber\\\mbox{i.e., }r&=&\frac{4a}{3}\nonumber
\end{eqnarray}

Further,\begin{eqnarray}
	v&=&-\omega r\nonumber\\\mbox{i.e., }\frac{\mathrm{d}s}{\mathrm{d}r}\cdot\frac{\mathrm{d}r}{\mathrm{d}t}&=&-\omega r\nonumber\\\mbox{i.e., }\frac{1}{-\sin\theta}\cdot\frac{\mathrm{d}r}{\mathrm{d}t}&=&-\omega r~~~\left(\frac{\mathrm{d}r}{\mathrm{d}s}=\cos\phi=\cos\left(\frac{\pi}{2}+\theta\right)=-\sin\theta\right.\nonumber\\&&~~~~~~ \left.\mbox{Note that as $\theta$ is negative so $\sin\theta$ is negative.}\right)\nonumber\\\therefore~\frac{\mathrm{d}r}{\mathrm{d}t}&=&\omega r\sqrt{\frac{4a^2-r^2}{4a^2}}\nonumber\\\mbox{i.e. }\frac{\omega}{2a}T&=&\int_{2a}^{\frac{4a}{3}}\frac{\mathrm{d}r}{r\sqrt{4a^2-r^2}}=\int_{r=2a}^{r=\frac{4a}{3}}\frac{-\frac{1}{u^2}\mathrm{d}u}{\frac{1}{u^2}\sqrt{4a^2u^2-1}}~~~~~\left(\mbox{Putting } r=\frac{1}{u}\right)\nonumber\\&=&-\frac{1}{2a}\int_{r=2a}^{r=\frac{4a}{3}}\frac{\mathrm{d}u}{\sqrt{u^2-\frac{1}{4a^2}}}\nonumber=-\frac{1}{2a}\left.\log\left[u+\sqrt{u^2-\frac{1}{4a^2}}\right]\right|_{r=2a}^{r=\frac{4a}{3}}\nonumber\\&=&-\frac{1}{2a}\left.\log\left[\frac{1}{r}+\sqrt{\frac{1}{r^2}-\frac{1}{4a^2}}\right]\right|_{r=2a}^{r=\frac{4a}{3}}\nonumber\\\therefore~T&=&\dfrac{1}{\omega}\log_e\left(\dfrac{3+\sqrt{5}}{2}\right)\nonumber
\end{eqnarray}

\section{Motion of a particle on a plane wire rotating about an axis in its plane}

 \begin{wrapfigure}[9]{r}{0.35\textwidth}
	\centering	\includegraphics[height=4.5 cm , width=4.5 cm ]{f53.pdf}
	\begin{center}
		Fig. 3.50
	\end{center}
\end{wrapfigure}

Let the plane of the curve be the plane $zox$ and the pane is rotating with angular velocity $\omega$ about $oz$. let the equation of the curve be $z=f(x)$. Initially, let the plane of the curve coincides with a fixed plane $zox'$ in space. If at time $t$, $\phi$ be the angle between these two planes then $\omega=\dot{\phi}$. If $\vec{i}$, $\vec{j}$, $\vec{k}$ be the unit vectors along $ox$, $oy$ and $oz$ respectively and $P\equiv(x,0,z)$, then $\vec{r}=\overrightarrow{OP}=x\vec{i}+z\vec{k}$. The equation of motion of the particle is
\begin{equation}
	m\frac{\mathrm{d}^2\vec{r}}{\mathrm{d}t^2}=\vec{F}+\vec{R}\nonumber
\end{equation}
where $\vec{F}$ is the external force and $\vec{R}$ is the reaction on the wire.\\

Now,\begin{eqnarray}
		\frac{\mathrm{d}\vec{r}}{\mathrm{d}t}&=&\frac{\partial\vec{r}}{\partial t}+\vec{\omega}\times\vec{r}=(\dot{x},0,\dot{z})+\begin{vmatrix}
		\vec{i}&\vec{j}&\vec{k}\\0&0&\dot{\phi}\\x&0&z
	\end{vmatrix}\nonumber\\&=&(\dot{x},x\dot{\phi},\dot{z})\nonumber\\\therefore~	\frac{\mathrm{d}^2\vec{r}}{\mathrm{d}t^2}&=&(\ddot{x},x\ddot{\phi}+\dot{x}\dot{\phi},\ddot{z})+\begin{vmatrix}
		\vec{i}&\vec{j}&\vec{k}\\0&0&\dot{\phi}\\\dot{x}&x\dot{\phi}&\dot{z}
	\end{vmatrix}\nonumber\\&=&(\ddot{x}-x\dot{\phi}^2)\vec{i}+(2\dot{x}\dot{\phi}+x\ddot{\phi})\vec{j}+\ddot{z}\vec{k}\nonumber\\&=&\frac{\partial^2r}{\partial t^2}-x\omega^2\vec{i}+\frac{1}{x}\frac{\mathrm{d}}{\mathrm{d}t}\left(x^2\omega\right)\vec{j}\nonumber
\end{eqnarray}

Hence from the equation of motion
\begin{equation}
	m\frac{\partial^2r}{\partial t^2}=\vec{F}+\vec{R}+mx\omega^2\vec{i}-\frac{m}{x}\frac{\mathrm{d}}{\mathrm{d}t}\left(x^2\omega\right)\vec{j}\nonumber
\end{equation}

Thus the motion of the particle relative to the curve is the same as if the curve were at rest and the following forces are acting on the particle.\\

(i) a force $m\omega^2x$ is acting perpendicular to the axis of rotation and away from it.\\

(ii) a force $\dfrac{m}{x}\dfrac{\mathrm{d}}{\mathrm{d}t}\left(x^2\omega\right)$ perpendicular to the plane of the curve is acting in a sense opposite to the sense of rotation of the plane.\\

Now resolving along the tangent, principal normal and binormal to the curve at $P$ we get
\begin{eqnarray}
	mv\frac{\mathrm{d}v}{\mathrm{d}s}&=&F_T+m\omega^2x\frac{\mathrm{d}x}{\mathrm{d}s}\label{eq3.109}\\m\frac{v^2}{\rho}&=&F_N+R_N-m\omega^2x\frac{\mathrm{d}z}{\mathrm{d}s}\\0&=&F_B+R_B-\dfrac{m}{x}\dfrac{\mathrm{d}}{\mathrm{d}t}\left(x^2\omega\right)
\end{eqnarray}

Further, for conservative external forces, $F_T=-\dfrac{\partial V}{\partial s}$ and we have from (\ref{eq3.109})
\begin{equation}
	mv\frac{\mathrm{d}v}{\mathrm{d}s}=-\frac{\partial V}{\partial s}+m\omega^2x\frac{\mathrm{d}x}{\mathrm{d}s}\nonumber
\end{equation}
which on integration gives (assuming $\omega$ to be constant)
\begin{eqnarray}
	\frac{1}{2}mv^2+V-\frac{1}{2}m\omega^2x^2&=&\mbox{constant}\nonumber\\\mbox{i.e., }	\frac{1}{2}mv^2+V'&=&\mbox{constant}\longrightarrow\mbox{modified energy equation}\nonumber
\end{eqnarray}
with $V'=V-\dfrac{1}{2}m\omega^2x^2$ as the effective potential.\\

{\bf 33. } A smooth circular wire of radius $a$ rotates with uniform angular velocity $\omega$ around the tangent at the end of the diameter $AB$ of the wire. A small ring of unit mass is free to slide on the wire from $A$, $r$ being the distance from $A$. If $\omega^2=7\mu$ and if the ring is initially at $B$ with a velocity $4a\sqrt{2\mu}$ relative to the wire, show that after a time $t$ its angular distance from $O$ is $2\tan^{-1}\left(2\sqrt{2}\sinh(\sqrt{\mu}t)\right)$.\\

{\bf Solution: } When the curve is reduced  to rest the equation of motion along the tangent to the wire at $P$ is \begin{equation}
	v\frac{\mathrm{d}v}{\mathrm{d}s}=\mu r\frac{\mathrm{d}r}{\mathrm{d}s}+\omega^2x\frac{\mathrm{d}x}{\mathrm{d}s},\nonumber
\end{equation}
which on integration gives
\begin{equation}
	v^2=\mu r^2+\omega^2 x^2+c\nonumber
\end{equation}

\begin{wrapfigure}[9]{r}{0.35\textwidth}\vspace{-2cm}
	\hfill	\includegraphics[height=4.5 cm , width=4.5 cm ]{f54.pdf}
	\begin{center}
		Fig. 3.51
	\end{center}
\end{wrapfigure}

Initially, $v=4a\sqrt{2\mu}$, $r=2a$, $x=2a$, so $c=0$.
\begin{eqnarray}
	\therefore~v^2&=&\mu r^2+\omega^2 x^2=\mu\cdot4a^2\cos^2\theta+7\mu\cdot4a^2\cos^4\theta\nonumber\\\therefore~\frac{\mathrm{d}s}{\mathrm{d}\theta}\frac{\mathrm{d}\theta}{\mathrm{d}t}&=&\sqrt{\mu}2a\cos\theta\sqrt{1+7\cos^2\theta} ~~~~(\because \theta\mbox{ increases with time})\nonumber\\\mbox{i.e., }\frac{\mathrm{d}\theta}{\mathrm{d}t}&=&\sqrt{\mu}\cos\theta\sqrt{1+7\cos^2\theta}~~\left(\because \frac{\mathrm{d}s}{\mathrm{d}(2\theta)}=a\mbox{ i.e., }\frac{\mathrm{d}s}{\mathrm{d}\theta}=2a\right)\nonumber
\end{eqnarray}

Now integrating
\begin{eqnarray}
	\mu t&=&\int_0^\theta\frac{\sec^2\theta\mathrm{d}\theta}{\sqrt{\tan^2\theta-8}}=\sinh^{-1}\left(\frac{\tan\theta}{2\sqrt{2}}\right)\nonumber\\\therefore~\tan\theta&=&2\sqrt{2}\sinh(\sqrt{\mu}t)\nonumber\\\theta&=&\tan^{-1}\left(2\sqrt{2}\sinh(\sqrt{\mu}t)\right)\nonumber
\end{eqnarray}

So the angular distance $=2\theta=2\tan^{-1}\left(2\sqrt{2}\sinh(\sqrt{\mu}t)\right)$\\

{\bf 34. } A  smooth tube in the form of a parabola of latus rectum $4a$, rotates with constant angular velocity $\omega$ about its axis which is vertical. A particle is at rest at a height $x_0$ above the vertex. Show that if the angular velocity is reduced to $\omega'$, the particle will oscillates from the level $x_0$ in one limb to the same level in the other limb in time 
$$\frac{2}{\sqrt{\omega^2-{(\omega')}^2}}\int_0^\frac{\pi}{2}\sqrt{1+\frac{x_0}{a}\sin^2\theta}\mathrm{d}\theta$$

{\bf Solution: } Let the equation of the parabola is $$y^2=4ax$$

When the curve is reduced to rest the equation of motion along the tangent at $P$ is
\begin{equation}
	v\frac{\mathrm{d}v}{\mathrm{d}s}=\omega^2y\frac{\mathrm{d}y}{\mathrm{d}s}-g\frac{\mathrm{d}x}{\mathrm{d}s}\nonumber
\end{equation}
i.e., on integration, $$v^2=\omega^2 y^2-2gx+c$$

\begin{wrapfigure}[10]{r}{0.35\textwidth}
	\hfill	\includegraphics[height=4.5 cm , width=4.5 cm ]{f55.pdf}
	\begin{center}
		Fig. 3.52
	\end{center}
\end{wrapfigure}

But $v^2=\dot{x}^2+\dot{y}^2=\dot{x}^2\left\{1+\left(\dfrac{\mathrm{d}y}{\mathrm{d}x}\right)^2\right\}=\dot{x}^2\left\{1+\dfrac{a}{x}\right\}$.
\begin{equation}
	\therefore~\dot{x}^2\left(1+\frac{a}{x}\right)=4a\omega^2x-2gx+c\nonumber
\end{equation}

Now, differentiating both sides with respect to $x$, we have
\begin{equation}
	2\ddot{x}\left(1+\frac{a}{x}\right)-a\frac{\dot{x}^2}{x^2}=4a\omega^2-2g\nonumber
\end{equation}

For relative equilibrium at $x=x_0$, $\dot{x}=0=\ddot{x}$ at $x=x_0$.
$$\therefore~4a\omega^2=2g$$

When the angular velocity is $\omega'$, the equation of motion of the particle is
\begin{eqnarray}
	&&v\frac{\mathrm{d}v}{\mathrm{d}s}=\omega'^2y\frac{\mathrm{d}y}{\mathrm{d}s}-g\frac{\mathrm{d}x}{\mathrm{d}s}\nonumber\\\therefore&&v^2=\dot{x}^2+\dot{y}^2=\dot{x}^2\left(1+\frac{a}{x}\right)=\omega'^2 y^2-2gx+c'\nonumber\\\mbox{i.e.,}&&\dot{x}^2\left(1+\frac{a}{x}\right)-4ax\omega'^2+2gx=c'=0-4ax_0\omega'^2+2gx_0\nonumber\\\mbox{i.e.,}&&\dot{x}^2\left(1+\frac{a}{x}\right)=\left(2g-4a{\omega'}^2\right)(x_0-x)=4a\left(\omega^2-{\omega'}^2\right)(x_0-x)\nonumber
\end{eqnarray}

As $\omega>\omega'$ and $x>0$ so the motion of the particle will be confined to the region $x\leqslant x_0$ i.e., the particle will move from a level $x_0$ in one limb to the same level to the other. If $T$ be the corresponding time then $T$ can be evaluated as follows:
\begin{eqnarray}
	\dot{x}&=&-\sqrt{4a\left(\omega^2-{\omega'}^2\right)x\frac{(x_0-x)}{(a+x)}}~~~(\because x \mbox{ decreases with time})\nonumber\\
	\therefore-\int_{x_0}^0\sqrt{\frac{a+x}{x(x_0-x)}}\mathrm{d}x&=&\sqrt{4a\left(\omega^2-{\omega'}^2\right)}\int_0^\frac{T}{2}\mathrm{d}t\nonumber\\
	\mbox{i.e.,~ }\sqrt{a\left(\omega^2-{\omega'}^2\right)}T&=&\int_0^\frac{\pi}{2}\frac{\sqrt{a+x_0\sin^2\theta}}{x_0\sin\theta\cos\theta}2x_0\sin\theta\cos\theta\mathrm{d}\theta~~~(\mbox{Putting	 }x=x_0\sin^2\theta)\nonumber\\
	&=&2\int_0^\frac{\pi}{2}\sqrt{a+x_0\sin^2\theta}\mathrm{d}\theta\nonumber\\
	\therefore~T&=&\frac{2}{\sqrt{\left(\omega^2-{\omega'}^2\right)}}\int_0^\frac{\pi}{2}\sqrt{1+\frac{x_0}{a}\sin^2\theta}\mathrm{d}\theta\nonumber
\end{eqnarray}

{\bf 35. } A smooth wire in the form of the curve $y=b\cos\left(\frac{x}{a}\right)$, which lies between $x=\pm a\pi$, $a$ and $b$ being positive numbers, rotates with constant angular velocity $\omega$ about the axis of $y$ which is vertically downward. Prove that, if $a^2\omega^2<bg$, there are three positions of relative equilibrium for a bead which can slide on the wire and that two of them are unstable.\\

{\bf Solution: } The equation of motion is
\begin{equation}
	mv\frac{\mathrm{d}v}{\mathrm{d}s}=m\omega^2x\frac{\mathrm{d}x}{\mathrm{d}s}+mg\frac{\mathrm{d}y}{\mathrm{d}s}\nonumber
\end{equation}
so on integration 
\begin{eqnarray}
	&&v^2=\omega^2x^2+2gy+c\nonumber\\\mbox{i.e., }&&\dot{x}^2\left\{1+\left(\dfrac{\mathrm{d}y}{\mathrm{d}x}\right)^2\right\}=\omega^2x^2+2gb\cos\left(\frac{x}{a}\right)+c\nonumber\\\mbox{i.e., }&&\dot{x}^2\left\{1+\frac{b^2}{a^2}\sin^2\left(\frac{x}{a}\right)\right\}=\omega^2x^2+2gb\cos\left(\frac{x}{a}\right)+c\nonumber
\end{eqnarray}

\begin{wrapfigure}[12]{r}{0.35\textwidth}
	\hfill	\includegraphics[height=4.8 cm , width=4.5 cm ]{f56.pdf}
	\begin{center}
		Fig. 3.53
	\end{center}
\end{wrapfigure}

Now differentiating with respect to $x$ we have
\begin{equation}
	\ddot{x}\left\{1+\frac{b^2}{a^2}\sin^2\left(\frac{x}{a}\right)\right\}+\dot{x}^2\frac{b^2}{a^3}\sin\left(\frac{x}{a}\right)\sin\left(\frac{x}{a}\right)=\omega^2x-g\frac{b}{a}\sin\left(\frac{x}{a}\right)\nonumber
\end{equation}

So for relative equilibrium, $\dot{x}=0=\ddot{x}$ i.e., the position of relative equilibrium is given by
\begin{equation}\label{eq3.35.1}
	\omega^2x=\frac{gb}{a}\sin\left(\frac{x}{a}\right)
\end{equation}

Clearly, $x=0$ is a solution of equation (\ref{eq3.35.1}). Also $x=\pm x_0$ will be a solution of equation (\ref{eq3.35.1})if the line $y=\omega^2x$ intersects the curve $y=\dfrac{gb}{a}\sin\left(\dfrac{x}{a}\right)$ at $x_0$ $(0<x_0<a\pi)$. For $0<x<a\pi$, the gradient of the curve $y=\dfrac{gb}{a}\sin\left(\dfrac{x}{a}\right)$ is $\dfrac{gb}{a^2}\cos\left(\dfrac{x}{a}\right)$, which is monotonically decreasing and the gradient of the straight line is $\omega^2$. The straight line will intersect the curve at $\pm x_)$ if the gradient of the curve at $x=0$ is greater than the gradient of the straight line i.e., $\dfrac{gb}{a^2}>\omega^2$.\\

Thus if $a^2\omega^2<gb$ there are three positions of relative equilibrium. Also we have at $\pm x_0$ the gradient of the curve is less than that of the straight line i.e., $\dfrac{gb}{a^2}\cos\left(\dfrac{x_0}{a}\right)<\omega^2$ i.e, $a^2\omega^2-gb\cos\left(\dfrac{x_0}{a}\right)>0$.\\

Stability Criteria:\\

Case-I: $x=0$ is the equilibrium position. Let $x=0+\epsilon$ where $\epsilon$ is very small, be a displaced position. So $\epsilon^2$, $\dot{\epsilon}^2$, $\cdots$ are neglected. Now,
\begin{eqnarray}
	&&\ddot{x}\left\{1+\frac{b^2}{a^2}\sin^2\left(\frac{x}{a}\right)\right\}+\dot{x}^2\frac{b^2}{a^3}\sin\left(\frac{x}{a}\right)\sin\left(\frac{x}{a}\right)=\omega^2x-g\frac{b}{a}\sin\left(\frac{x}{a}\right)\nonumber\\\mbox{i,e., }&&\ddot{\epsilon}\left\{1+\frac{b^2}{a^2}\sin^2\left(\frac{\epsilon}{a}\right)\right\}=\omega^2\epsilon-\frac{gb}{a}\sin\left(\frac{\epsilon}{a}\right)\nonumber\\\mbox{i.e., }&&\ddot{\epsilon}\left(1+\frac{b^2}{a^2}\cdot\frac{\epsilon^2}{a^2}\right)=-\left(\frac{gb}{a^2}-\omega^2\right)\epsilon\nonumber\\\mbox{i.e., }&&\ddot{\epsilon}=-\left(\frac{gb}{a^2}-\omega^2\right)\epsilon\nonumber
\end{eqnarray}

As $\dfrac{gb}{a^2}-\omega^2>0$ so the motion is a simple harmonic motion. Hence the equilibrium position is stable.\\

Case-II: $x=\pm x_0$ are the equilibrium position.\\

Let $x=\pm x_0+\epsilon$.
\begin{eqnarray}
	\therefore&&\ddot{\epsilon}\left\{1+\frac{b^2}{a^2}\sin^2\left(\frac{\pm x_0+\epsilon}{a}\right)\right\}=\omega^2(\pm x_0+\epsilon)-\frac{gb}{a}\sin\left(\frac{\pm x_0+\epsilon}{a}\right)\nonumber\\\mbox{i.e.,}&&\ddot{\epsilon}\left[1+\frac{b^2}{a^2}\left\{\pm\sin\frac{x_0}{a}+\frac{\epsilon}{a}\cos\frac{x_0}{a}\right\}^2\right]=\cancel{\pm\omega^2 x_0}+\omega^2\epsilon-\frac{gb}{a}\left[\cancel{\pm\sin\frac{x_0}{a}}+\frac{\epsilon}{a}\cos\frac{x_0}{a}\right]\nonumber\\&&~~~~~~~~~~~~~~~~~~~~~~~~~~~~~~~~~~~~~~~~~~~~~~~~~~~\left(\mbox{As }\omega^2x_0=\frac{gb}{a}\sin\frac{x_0}{a}\right)\nonumber\\\mbox{i.e,}&&\ddot{\epsilon}\left[1+\frac{\omega^4x_0^2}{g^2}\pm2\frac{\omega^2x_0b\epsilon}{a^2g}\cos\frac{x_0}{a}\right]=\left(\omega^2-\frac{gb}{a^2}\cos\frac{x_0}{a}\right)\epsilon\nonumber\\\therefore&&\ddot{\epsilon}=\frac{\left(\omega^2-\frac{gb}{a^2}\cos\frac{x_0}{a}\right)}{\left(1+\frac{\omega^4x_0^2}{g^2}\right)}\epsilon\left[1\pm\frac{2\frac{\omega^2x_0b\epsilon}{a^2g}\cos\frac{x_0}{a}}{1\pm\frac{\omega^4x_0^2}{g^2}}\right]^{-1}\nonumber\\&&~~~=\frac{\left(\omega^2-\frac{gb}{a^2}\cos\frac{x_0}{a}\right)}{\left(1+\frac{\omega^4x_0^2}{g^2}\right)}\epsilon ~~~~~~~~~~~\left(a^2\omega^2>gb\cos\left(\dfrac{x_0}{a}\right)\right)\nonumber
\end{eqnarray}

Hence the motion is not simple harmonic motion. So $x=\pm x_0$ are unstable position of equilibrium.\\

\section{Motion relative to Earth}

Let $SN$ be the axis drawn from South to North, $O$ be any point on or near the surface of the Earth. $B$ is the foot of the perpendicular from $O$ on $SN$. $\vec{I}$ is the unit vector along $\overrightarrow{BO}$. The triad of unit vectors $\vec{i}$, $\vec{j}$, $\vec{k}$ fixed relative to earth are defined as:\\

\begin{wrapfigure}[12]{r}{0.35\textwidth}
	\hfill	\includegraphics[height=5 cm , width=4.9 cm ]{f57.pdf}
	\begin{center}
		Fig. 3.54
	\end{center}
\end{wrapfigure}

$\vec{i}$ is horizontal and points South\\

$\vec{j}$ is horizontal and points East\\

$\vec{k}$ is vertical and points upwards.\\

Let $\overrightarrow{BO}=\vec{r_0}$ and $\vec{r}$ be the position vector of a particle relative to $O$. Then the position vector of the particle relative to $B$ is $$\vec{r}_B=\vec{r}_0+\vec{r}$$

The absolute acceleration of the particle is
$$\ddot{\vec{r}}_B=\ddot{\vec{r}}_0+\ddot{\vec{r}}=\omega^2\vec{r}_0+\ddot{\vec{r}}$$

If $m$ be the mass of the particle then as the force of gravity is proportional to $m$, let it be $m\vec{P}$ and let $\vec{F}$ be the resultant of all other forces. Then the equation of motion of the particle is
\begin{eqnarray}
	m\ddot{\vec{r}}_B&=&m\vec{P}+\vec{F}\nonumber\\\mbox{i.e.,~ }m\ddot{\vec{r}}&=&m\vec{P}+\vec{F}+m\omega^2\vec{r}_0=\vec{F}+m\vec{g}\label{eq3.113}
\end{eqnarray} where 
\begin{equation}
	m\vec{g}=m\vec{P}+m\omega^2\vec{r}_0
\end{equation}
is termed as the apparent force of gravity.

Thus in considering the motion of a particle relative to $O$, we can assume the point $O$ at rest provided we replace the force of gravity by the apparent force of gravity.\\

Suppose we consider a plumb line in the equilibrium with its bob at $O$. Then $\vec{F}$ will be not the tension in the string along the unit vector $\vec{k}$. Since at $O$, $\vec{r}=0$, so from (\ref{eq3.113}) $\vec{F}+m\vec{g}_0=\vec{0}$, where $m\vec{g}_0=m\vec{P}_0+m\omega^2\vec{r}_0$ and $m\vec{P}_0$ is the force of gravity at $O$. Since $m\vec{g}_0=-\vec{F}$, $m\vec{g}_0$ is acting vertically downward at $O$ and the vertically downward direction at $O$ will be along $OC'$. The angles $OC'N$ and $OCN$ are called geographical and geocentric co-latitude of the point $O$ respectively.

\subsection{Acceleration of a particle}

Here $C$ is not the centre. Let $\lambda$ be the geographical latitude of the place at $O$, a point on near the surface of the earth. Let $\omega$ be the angular velocity of the Earth about its axis. We choose the axes as follows:\\

\begin{wrapfigure}[12]{r}{0.35\textwidth}\vspace{-.3cm}
	\hfill	\includegraphics[height=4.5 cm , width=4.5 cm ]{f58.pdf}
	\begin{center}
		Fig. 3.55
	\end{center}
\end{wrapfigure}

$z$-axis is vertical and points upwards\\

$x$-axis is horizontal and points south\\

$y$-axis is horizontal and points east.\\

Then components of $\omega$ about $x$, $y$, $z$ axes are 
$$\omega_x=\omega\cos(\pi-\lambda)=-\omega\cos\lambda~,~~\omega_y=0~,~~\omega_z=\omega\sin\lambda.$$

Now referred to $ox$, $oy$, $oz$ as axes let $(x,y,z)$ be the co-ordinates of the particle at time $t$. Suppose $(u_x,u_y,u_z)$ and $(f_x,f_y,f_z)$ are the components of velocity and acceleration of the particle. Then $$(u,v,w)=(\dot{x},\dot{y},\dot{z})+\begin{vmatrix}
	\vec{i}&\vec{j}&\vec{k}\\-\omega\cos\lambda&0&\omega\sin\lambda\\x&y&z
\end{vmatrix}$$
 where $\vec{i}$, $\vec{j}$, $\vec{k}$ are the unit vectors along the three axes $ox$, $oy$, $oz$ respectively.\begin{eqnarray}
 	\therefore~u_x=\dot{x}-\omega y\sin\lambda~,~~u_y=\dot{y}+\omega x\sin\lambda+\omega z\cos\lambda~,~~u_z=\dot{z}-\omega y\cos\lambda\nonumber
 \end{eqnarray}

Similarly\begin{eqnarray}
	&&(f_x,f_y,f_z)=(\dot{u_x},\dot{u_y},\dot{u_z})+\begin{vmatrix}
		\vec{i}&\vec{j}&\vec{k}\\-\omega\cos\lambda&0&\omega\sin\lambda\\u&v&w
	\end{vmatrix}\nonumber\\\mbox{i.e.,}&&f_x=\ddot{x}-2\omega\dot{y}\sin\lambda-\omega^2x\sin^2\lambda-\omega^2z\sin\lambda\cos\lambda\nonumber\\&&f_y=\ddot{y}+2\omega \dot{x}\sin\lambda+2\omega\dot{z}\cos\lambda-\omega^2y\nonumber\\&&f_z=\ddot{z}-2\omega\dot{y}\cos\lambda-\omega^2x\sin\lambda\cos\lambda-\omega^2z\cos^2\lambda\nonumber
\end{eqnarray}

\section{Motion of a freely falling body}

Suppose a particle of mass $m$ falls freely from a height $h$ above $O$. Let $m\vec{g}$ be the constant apparent force of gravity. Then the equation of motion of the particle are \begin{eqnarray}
	&&mf_x=0~,~~mf_y=0~,~~mf_z=-mg\nonumber\\\mbox{i.e., }&&\ddot{x}-2\omega\dot{y}\sin\lambda-\omega^2x\sin^2\lambda-\omega^2z\sin\lambda\cos\lambda=0\nonumber\\&&\ddot{y}+2\omega \dot{x}\sin\lambda+2\omega\dot{z}\cos\lambda-\omega^2y=0\nonumber\\&&\ddot{z}-2\omega\dot{y}\cos\lambda-\omega^2x\sin\lambda\cos\lambda-\omega^2z\cos^2\lambda=-g\label{eq3.115}
\end{eqnarray}

We get first approximation of the solution by noting that $\dot{x}$, $\dot{y}$ are small compared to $\dot{z}$ and neglecting $\omega^2$ so that the set of equations (\ref{eq3.115}) become
\begin{equation}\label{eq3.116}
	\ddot{x}=0~,~~\ddot{y}=-2\omega\dot{z}\cos\lambda~,~~\ddot{z}=-g
\end{equation}

The initial conditions at $t=0$ are $x=0=y$, $z=h$, $\dot{x}=0=\dot{y}=\dot{z}$, so the solution of equation (\ref{eq3.116}) gives
\begin{equation}\label{eq3.117}
	x=0~,~~y=\frac{1}{3}g\omega t^3\cos\lambda~,~~z=h-\frac{1}{2}gt^2
\end{equation}

Thus in the first approximation the time of falling from the height $h$ is $t=\sqrt{\dfrac{2h}{g}}$ and then $y=\dfrac{1}{3}g\omega\cos\lambda\sqrt{\dfrac{8h^3}{g^3}}$.\\

This shows that the particle will have a easterly deviation as it reaches the surface of the earth.\\

Now to find the second approximation of the solution of equation (\ref{eq3.115}) we substitute the solution (\ref{eq3.117}) in (\ref{eq3.115}) and neglect $\omega^3$. Hence we obtain
\begin{eqnarray}
	&&\ddot{x}-2\omega^2 gt^2\cos\lambda\sin\lambda-\omega^2\sin\lambda\cos\lambda\left(h-\frac{1}{2}gt^2\right)=0\nonumber\\\mbox{i.e.}&&\ddot{x}-\frac{3}{2}\omega^2 gt^2\cos\lambda\sin\lambda-h\omega^2\sin\lambda\cos\lambda=0,\nonumber\\&&\ddot{y}-2\omega gt\cos\lambda=0,\nonumber\\\mbox{and}&&\ddot{z}-2\omega^2gt^2\cos^2\lambda-\omega^2\cos^2\lambda\left(h-\frac{1}{2}gt^2\right)=-g\nonumber\\\mbox{i.e.,}&&\ddot{z}-\frac{3}{2}\omega^2gt^2\cos^2\lambda-h\omega^2\cos^2\lambda=-g\nonumber
\end{eqnarray}

So on integration using the initial conditions
\begin{eqnarray}
	x&=&\frac{1}{8}\omega^2 gt^4\cos\lambda\sin\lambda+\frac{1}{2}h\omega^2t^2\sin\lambda\cos\lambda\nonumber\\y&=&\frac{1}{3}g\omega t^3\cos\lambda\nonumber\\z&=&h-\frac{1}{2}gt^2+\frac{1}{8}\omega^2gt^4\cos^2\lambda+\frac{1}{2}h\omega^2t^2\cos^2\lambda\nonumber
\end{eqnarray}

When the particle reaches the ground then $z=0$ and the last equation gives the corresponding time. If $t_0$be the time of flight in the 1st approximation then $t_0^2=\dfrac{2h}{g}$. To get the second approximation of $t$, let $t^2=t_0^2+\epsilon$, where $\epsilon$ is small compared to $t_0^2$ and we neglect $\epsilon^2$. Thus we get
\begin{eqnarray}
	0&=&h-\frac{1}{2}g(t_0^2+\epsilon)+\frac{1}{8}\omega^2g(t_0^2+\epsilon)^2\cos^2\lambda+\frac{1}{2}h\omega^2(t_0^2+\epsilon)\cos^2\lambda\nonumber\\\mbox{i.e.,~ }0&=&-\frac{1}{2}g\epsilon+\frac{1}{8}\omega^2g\left(\frac{4h^2}{g^2}+\frac{2h}{g}\epsilon\right)\cos^2\lambda+\frac{1}{2}h\omega^2\left(\frac{2h}{g}+\epsilon\right)\cos^2\lambda\nonumber\\\mbox{i.e.,~ }\epsilon&=&\frac{3\omega^2h^2\cos^2\lambda}{g^2}\nonumber
\end{eqnarray}

Hence the time of flight $T^2=\dfrac{2h}{g}+\dfrac{3\omega^2h^2\cos^2\lambda}{g^2}$.\\

Then $x$ and $y$ are given by
\begin{eqnarray}
	y&=&\frac{1}{3}g\omega\cos\lambda\sqrt{\frac{8h^3}{g^3}}\left\{1+\frac{3}{2}\frac{3\omega^2h\cos^2\lambda}{g}\right\}^\frac{3}{2}\nonumber\\&=&\frac{1}{3}g\omega\cos\lambda\sqrt{\frac{8h^3}{g^3}}\nonumber\\x&=&\frac{1}{8}\omega^2gt^4\cos\lambda\sin\lambda+\frac{1}{2}h\omega^2t^2\sin\lambda\cos\lambda\nonumber\\&=&\frac{1}{8}\omega^2g\cos\lambda\sin\lambda\left\{\dfrac{2h}{g}+\dfrac{3\omega^2h^2\cos^2\lambda}{g^2}\right\}^2+\frac{1}{2}h\omega^2\sin\lambda\cos\lambda\left\{\dfrac{2h}{g}+\dfrac{3\omega^2h^2\cos^2\lambda}{g^2}\right\}\nonumber\\&=&\frac{1}{2g}h^2\omega^2\cos\lambda\sin\lambda+\frac{1}{g}h^2\omega^2\sin\lambda\cos\lambda\nonumber\\&=&\frac{3}{2g}h^2\omega^2\sin\lambda\cos\lambda\nonumber
\end{eqnarray}

\section{Motion of a particle projected vertically upwards}

Let a particle be projected vertically upward from the point $O$ on or near the surface of the earth, with initial velocity $\omega_0$ along the $z$-axis. The equations of motion, when $\omega^2$ is neglected with the natural assumption that $\dot{x}$, $\dot{y}$ are small compare to $\dot{z}$ are
$$\ddot{x}=0~,~~\ddot{y}+2\omega\dot{z}\cos\lambda=0~,~~\ddot{z}=-g$$

The initial conditions are $\dot{x}=0=\dot{y}$, $\dot{z}=\omega_0$, $x=0=y=z$ at $t=0$. So the solution of the above equations of motion are
$$x=0~,~~z=\omega_0t-\frac{1}{2}gt^2~,~~y=-g\omega \cos\lambda\left(\omega_0t^2-\frac{1}{3}gt^3\right)$$

So when the particle again reaches the level of projection then the time of flight is $T=\dfrac{2\omega_0}{g}$ and then $$y=-\frac{4\omega_0^2}{g^2}\omega\cos\lambda\left(\omega_0-\frac{2}{3}\omega_0\right)=-\frac{4}{3}\frac{\omega_0^3}{g^2}\omega\cos\lambda$$

Hence after reaching the level of projection the particle is deviated towards west by the amount $\dfrac{4}{3}\frac{\omega_0^3}{g^2}\omega\cos\lambda$. Further, from the solution we note that if $t>\dfrac{3\omega_0}{g}$ then $y>0$. Therefore, if the particle allows to fall below the level of projection and the particle does not reach the surface of earth before time $\dfrac{3\omega_0}{g}$, then there will be a deviation towards east.\\

\subsection{Motion of a projectile}\label{projectile}

Let the projectile be projected from $O$ with a velocity whose components along the axes are $u_0$, $v_0$, $w_0$. The equations of motion of the projectile (when $\omega^2$ is neglected) are
\begin{eqnarray}
	&&\ddot{x}-2\omega\dot{y}\sin\lambda=0\nonumber\\&&\ddot{y}+2\omega \dot{x}\sin\lambda+2\omega\dot{z}\cos\lambda=0\nonumber\\&&\ddot{z}-2\omega\dot{y}\cos\lambda=-g\nonumber
	\end{eqnarray}

\begin{wrapfigure}[10]{r}{0.3\textwidth}
	\hfill	\includegraphics[height=4.5 cm , width=4.5 cm ]{f59.pdf}
	\begin{center}
		Fig. 3.56
	\end{center}
\end{wrapfigure}

The initial conditions are $x=y=0=z$, $\dot{x}=u_0$, $\dot{y}=v_0$, $\dot{z}=w_0$ at $t=0$.\\

Thus neglecting $\omega^2$ and higher powers and using the above initial conditions the solution to the above equations of motion for the projectile are
\begin{eqnarray}
	x&=&u_0t+(\omega v_0\sin\lambda)t^2\nonumber\\y&=&v_0t+\frac{1}{3}g\omega t^3\cos\lambda-\omega t^2(u_0\sin\lambda+w_0\cos\lambda)\nonumber\\z&=&w_0t-\frac{1}{2}gt^2+\omega v_0 t^2\nonumber
\end{eqnarray}

So when the particle reaches the level of projection again then $z=0$ at $t=T$ and is given by
\begin{eqnarray}
	T&=&\frac{w_0}{\frac{g}{2}-\omega v_0\cos\lambda}=\frac{2w_0}{g}\left(1-\frac{2v_0\omega}{g}\cos\lambda\right)^{-1}\nonumber\\&\simeq&\frac{2w_0}{g}\left(1+\frac{2v_0\omega}{g}\cos\lambda\right)~,~~~\mbox{ neglecting $\omega^2$ and higher powers}\nonumber
\end{eqnarray}

If there is no rotation then the corresponding time of flight is $T_0=\dfrac{2w_0}{g}$. This shows that the time of flight is increased due to the rotation by an amount $T_0\times\dfrac{2v_0\omega}{g}\cos\lambda$.\\

\begin{wrapfigure}[10]{r}{0.3\textwidth}
	\hfill	\includegraphics[height=4.5 cm , width=4.5 cm ]{f60.pdf}
	\begin{center}
		Fig. 3.57
	\end{center}
\end{wrapfigure}

Suppose the vertical plane of projection of the projectile be $\theta$ east of south. Let $OX'$ be the horizontal line along this direction. $OY'$ is perpendicular to $OX'$ and horizontal, i.e., $OY'$ is horizontal and $\theta$ north of east. Then referred to $OX'$, $OY'$ and $OZ$ as axes let the co-ordinate of the projectile at time $t$ be $(x',y',z')$ where
\begin{eqnarray}
	x'&=&x\cos\theta+y\sin\theta\nonumber\\y'&=&-x\sin\theta+y\cos\theta\nonumber
\end{eqnarray} 

If $v$ be the horizontal component of initial velocity of projection then $u_0=v\cos\theta$, $v_0=v\sin\theta$
$$\mbox{i.e., ~}U=u_0\cos\theta+v_0\sin\theta \mbox{~ and ~}u_0\sin\theta=v_0\cos\theta$$

Thus\begin{eqnarray}
	x'&=&x\cos\theta+y\sin\theta\nonumber\\&=&(u_0t+\omega v_0\sin\lambda t^2)\cos\theta+\left[v_0t+\frac{1}{3}g\omega t^3\cos\lambda-\omega t^2(u_0\sin\lambda+w_0\cos\lambda)\right]\sin\theta\nonumber\\&=&Ut+\left(\frac{1}{3}g\omega\cos\lambda\sin\theta\right)t^3-(\omega w_0\cos\lambda\sin\theta)t^2\nonumber
\end{eqnarray}

Similarly\begin{eqnarray}
	y'&=&-x\sin\theta+y\cos\theta\nonumber\\&=&\omega t^2\left[\frac{1}{3}gt\cos\lambda\cos\theta-(U\sin\lambda+w_0\cos\lambda\cos\theta)\right]\nonumber
\end{eqnarray}

Thus the particle will have a deviation of amount $$\omega t^2(U\sin\lambda+w_0\cos\lambda\cos\theta)-\frac{1}{3}\omega gt^3\cos\lambda\cos\theta$$
to the right of vertical plane of projection when observed from $O$. So the amount of deviation when the particle reaches the ground i.e., when $t=\dfrac{2w_0}{g}\left(1+\dfrac{2v_0\omega}{g}\cos\lambda\right)=T$ will be
\begin{eqnarray}
	&=&\omega(U\sin\lambda+w_0\cos\lambda\cos\theta)\frac{4w_0^2}{g^2}-\frac{1}{3}\omega g\cos\lambda\cos\theta\frac{8w_0^3}{g^3}\nonumber\mbox{~~(neglecting $\omega^2$ and higher powers)}\\&=&\frac{4\omega w_0^2}{g^2}\left(U\sin\lambda+\frac{1}{3}w_0\cos\lambda\cos\theta\right)\nonumber
\end{eqnarray}

Similarly,
\begin{eqnarray}
	x'{\bigg|}_{t=T}=\frac{2w_0}{g}\left[U(1+\dfrac{2v_0\omega}{g}\cos\lambda)-\frac{2w_0^2}{3g}\omega \cos\lambda\sin\theta\right]\nonumber\mbox{~~(neglecting $\omega^2$ and higher powers)}
\end{eqnarray}

So the horizontal range \begin{eqnarray}
	R&=&\sqrt{(x')^2+(y')^2}=x'\sqrt{1+\left(\frac{y'}{x'}\right)^2}\nonumber\\	&\simeq&x'{\bigg|}_{t=T}\mbox{~~ (neglecting $\omega^2$ and higher powers)}\nonumber\\\therefore~R&=&\frac{2w_0U}{g}+\frac{4\omega w_0\cos\lambda}{g^2}\left[v_0U-\frac{1}{3}w_0^2\sin\theta\right]\nonumber
\end{eqnarray}

If there is no rotation then the range $R_0=\dfrac{2w_0U}{g}$,

 so increase in length = $\dfrac{4\omega w_0\cos\lambda}{g^2}\left[v_0U-\dfrac{1}{3}w_0^2\sin\theta\right]$.\\

\subsection{Horizontal pressure on railway lines}

Let $M$ be the mass of a train ($OA$) moving on a straight railway track in a direction $\theta$ east of south as shown in the figure. Let $R$ and $S$ denote the horizontal pressure of the rails (measured in the sense of $\theta$ north of east) and the vertical reaction.\\

\begin{wrapfigure}[12]{r}{0.3\textwidth}
	\hfill	\includegraphics[height=4.5 cm , width=4.5 cm ]{f61.pdf}
	\begin{center}
		Fig. 3.58
	\end{center}
\end{wrapfigure}

Then the equations of motion are
\begin{eqnarray}
	&&M(\ddot{x}-2\omega\dot{y}\sin\lambda-\omega^2x\sin^2\lambda)=-R\sin\theta\nonumber\\&&M(\ddot{y}+2\omega \dot{x}\sin\lambda-\omega^2y)=R\cos\theta\nonumber\\\mbox{and}&&M(-2\omega\dot{y}\cos\lambda-\omega^2x\sin\lambda\cos\lambda)=S-mg\nonumber
\end{eqnarray}

If $V$ be the uniform velocity of the train relative to the rail then we have
$$\ddot{x}=0=\ddot{y} \mbox{ ~and~ } \dot{x}=V\cos\theta~,~~\dot{y}=V\sin\theta$$

So at $x=0$, $y=0$ we have
\begin{eqnarray}
	R\sin\theta&=&2M\omega V\sin\theta\sin\lambda\nonumber\\R\cos\theta&=&2M\omega V\cos\theta\sin\lambda\nonumber\\\mbox{i.e.,~ }R&=&2M\omega v\sin\lambda\nonumber
\end{eqnarray}
gives pressure on the rails to the right in the direction of motion.\\

 Also $S=Mg-2M\omega V\sin\theta\cos\lambda$ gives the vertical reaction of the rail.\\

{\bf 36. } A particle is projected in latitude $\lambda$ with velocity $V$ at an elevation $\gamma$ in a direction $\theta$ east of south, Prove that due to earth's rotation with angular velocity $\omega$ the time of flight is increased by $2VT\left(\dfrac{\omega}{g}\right)\cos\lambda\cos\gamma\sin\theta$, and that the particle falls at a distance $\dfrac{4\omega\sin^2\gamma}{g^2}V^3\left(\cos\gamma\sin\lambda+\dfrac{1}{3}\sin\gamma\cos\lambda\cos\theta\right)$ to the right of the vertical plane of projection and that the range is increased by $4\omega\sin\gamma\cos\lambda\cos\theta\left(\cos^2\gamma-\dfrac{1}{3}\sin^2\gamma\right)\dfrac{V^3}{g^2}$ (approximately).\\

{\bf Solution: } In \ref{projectile}, we shall have to choose $U=V\cos\gamma$, $w_0=V\sin\gamma$, $u_0=U\cos\theta=V\cos\theta\cos\gamma$, $v_0=U\sin\theta=V\cos\gamma\sin\theta$.\\

When Earth's rotation is considered then time of flight $$T'=\frac{2w_0}{g}+\frac{4w_0v_0\omega}{g^2}\cos\lambda$$

$T=\dfrac{2w_0}{g}$ = time of flight when there is no rotation of earth.\\

Thus increase in time of flight = $\dfrac{4w_0v_0\omega}{g^2}\cos\lambda=\dfrac{2TV\omega}{g}\cos\gamma\cos\lambda\sin\theta$.\\

The amount of deviation to the right of the vertical plane of projection
\begin{eqnarray}
	&=&\frac{4\omega w_0^2}{g^2}\left(U\sin\lambda+\frac{1}{3}w_0\cos\lambda\cos\theta\right)\nonumber\\&=&\frac{4\omega V^2\sin^2\gamma}{g^2}\left(V\cos\gamma\sin\lambda+\frac{1}{3}V\sin\gamma\cos\lambda\cos\theta\right)\nonumber\\&=&\frac{4\omega V^3\sin^2\gamma}{g^2}\left(\cos\gamma\sin\lambda+\frac{1}{3}\sin\gamma\cos\lambda\cos\theta\right)\nonumber
\end{eqnarray}

Increase in the range
\begin{eqnarray}
	&=&\frac{4\omega w_0\cos\lambda}{g^2}\left[v_0U-\frac{1}{3}w_0^2\sin\theta\right]\nonumber\\&=&\frac{4\omega V\sin\gamma\cos\lambda}{g^2}\left[V^2\cos^2\gamma\sin\theta-\frac{1}{3}V^2\sin^2\gamma\sin\theta\right]\nonumber\\&=&\frac{4\omega V^3}{g^2}\sin\gamma\cos\lambda\sin\theta\left(\cos^2\gamma-\frac{1}{3}\sin^2\gamma\right)\nonumber
\end{eqnarray}

{\bf 37. } A shot is fired from a point on the earth's surface, the angle between the meridian and the plane of projection being $\theta$, measured from north to east. Show that, when the shot again reaches the earth's surface, its deviation from the plane of projection, due to the rotation of the earth, is to be the right or to the left according as $$3\tan\lambda-\tan\gamma\cos\theta$$
is positive or negative, where $\lambda$ is the latitude and $\gamma$ is the angle of projection. The square of the angular velocity of the earth is neglected, and the earth's surface within range is taken to be plane.\\

{\bf Solution: } In the previous problem, $\theta\rightarrow\pi-\theta$

So deviation = $\dfrac{4\omega w_0^2}{g^2}\left(U\sin\lambda+\dfrac{1}{3}w_0\cos\lambda\cos\theta\right)$.

Now $U=V\cos\gamma$, $w_0=V\sin\gamma$,
\begin{eqnarray}
	\therefore~\mbox{deviation}&=&\frac{4\omega V^2\sin^2\gamma}{g^2}\left(V\cos\gamma\sin\lambda+\frac{1}{3}V\sin\gamma\cos\lambda\cos(\pi-\theta)\right)\nonumber\\&=&\frac{4\omega V^3\sin^2\gamma}{g^2}\left(\cos\gamma\sin\lambda-\frac{1}{3}\sin\gamma\cos\lambda\cos\theta\right)\nonumber
\end{eqnarray}

Hence deviation will be to the right if
\begin{eqnarray}
	&&\cos\gamma\sin\lambda-\frac{1}{3}\sin\gamma\cos\lambda\cos\theta>0\nonumber\\\mbox{i.e.,}&&3\tan\lambda-\tan\gamma\cos\theta>0\nonumber
\end{eqnarray}
otherwise the deviation is to the left.\\

\section{Foucault's Pendulum}

\begin{wrapfigure}[12]{r}{0.3\textwidth}
	\hfill	\includegraphics[height=4.2 cm , width=4.5 cm ]{f62.pdf}
	\begin{center}
		Fig. 3.59
	\end{center}
\end{wrapfigure}


Let a pendulum be set up at earth's north pole. If started properly it may vibrate as a simple pendulum in a vertical plane. Since the earth under the pendulum turns around the axis with angular velocity $\omega$, the vertical plane of vibration appears to an observer at the surface of the earth to turn around the vertical with angular velocity $-\omega$. Foucault was the first to point out that a pendulum could be used to demonstrate earth's rotation. It is not necessary that the pendulum should be placed at the pole but the rotation of the vertical plane of oscillation of the pendulum may be observed from any place except on the equator.\\

\begin{wrapfigure}[12]{l}{0.3\textwidth}
\centering	\includegraphics[height=4.2 cm , width=4.5 cm ]{f63.pdf}
	\begin{center}
		Fig. 3.60
	\end{center}
\end{wrapfigure}

Let the origin be taken at $O$. a point on or near the surface of earth. The axes $X$, $Y$ and $Z$ are taken along the south, east and vertically upwards respectively. Let $\lambda$ be the geographical latitude of the place $O$. Let the pendulum consists of a particle of mass $m$ attached by light string of length $a$ from a point $(0,0,a)$ so that in equilibrium position the particle is at $O$. We shall discuss small oscillation of the pendulum about this position considering rotation of earth. Let $AP=a$, so the direction cosine of $PA$ are $\left(-\dfrac{x}{a},-\dfrac{y}{a},\dfrac{a-z}{a}\right)$. The focus acting on the particle are (i) apparent force of gravity acting vertically downward, (ii) tension $T$ in the string. Now components of $T$ along the axes are $$-T\dfrac{x}{a}~,~-T\dfrac{y}{a}~,~T\dfrac{a-z}{a}$$



We shall now consider two separate approximations namely (i) smallness of $\omega$ and (ii) smallness of oscillation. Due to first approximation we shall neglect $\omega^2$ and higher powers while second approximation tells us that $x$, $y$ and their derivatives are small so that $z$ and its derivatives are small quantities of second order and hence will be neglected.\\

The equation of motion of the particle are
\begin{eqnarray}
	&&m(\ddot{x}-2\omega\dot{y}\sin\lambda)=-T\frac{x}{a}\nonumber\\&&m(\ddot{y}+2\omega \dot{x}\sin\lambda)=-T\frac{y}{a}\nonumber\\\mbox{and}&&m(-2\omega\dot{y}\cos\lambda)=-mg+T\frac{(a-z)}{a}\simeq T-mg\nonumber\mbox{ ~~(neglecting $z$)}
\end{eqnarray}

So $T=mg-2m\omega\dot{y}\cos\lambda$\\

Using this expression for $T$ into the equations of motion along $X$ and $Y$ directions give
\begin{eqnarray}
	&&\ddot{x}-2\omega\dot{y}\sin\lambda=-g\frac{x}{a}\label{eq3.118}\\&&\ddot{y}+2\omega \dot{x}\sin\lambda=-g\frac{y}{a}\label{eq3.119}
\end{eqnarray}

Now, (\ref{eq3.118}) + $i\times$ (\ref{eq3.119})  gives $$\ddot{\xi}+2i\omega\sin\lambda\dot{\xi}+p^2\xi=0$$
with $\xi=x+iy$, $p^2=\dfrac{g}{a}$.\\

The solution takes the form (neglecting $\omega^2$)
$$\xi=Ae^{-i\omega t\sin\lambda+ipt}+Be^{-i\omega t\sin\lambda-ipt}$$
where the complex constants $A$ and $B$ depend on the initial conditions. Also the above solution can be written as $$(x+iy)e^{i\omega t\sin\lambda}=Ae^{ipt}+Be^{-ipt}$$

\begin{wrapfigure}[12]{r}{0.3\textwidth}
	\hfill	\includegraphics[height=5.2 cm , width=4 cm ]{f64.pdf}
	\begin{center}
		Fig. 3.61
	\end{center}
\end{wrapfigure}

Let us now choose $OX'$, $OY'$ in the plane $XOY$ such that the axis $OX'$ makes an angle $-\omega t\sin\lambda$ with $OX$ at time $t$. Thus the set of axes $(OX,OY')$  is rotating about $OZ$ with angular velocity $-\omega\sin\lambda$. If $(x',y',z')$ be the position of the particle at time $t$ referred to $OX'$, $OY'$, $OZ$ axes then 
\begin{eqnarray}
	x&=&x'\cos(\omega t\sin\lambda)+y'\sin(\omega t\sin\lambda)\nonumber\\y&=&-x'\sin(\omega t\sin\lambda)+y'\cos(\omega t\sin\lambda)\nonumber
\end{eqnarray}

Thus \begin{eqnarray}
	(x+iy)&=&(x'+iy')\cos(\omega t\sin\lambda)-i(x'+iy')\sin(\omega t\sin\lambda)\nonumber\\&=&(x'+iy')e^{-i\omega t\sin\lambda}\nonumber\\\mbox{i.e., }(x'+iy')&=&(x+iy)e^{i\omega t\sin\lambda}=Ae^{ipt}+Be^{-ipt}\nonumber
\end{eqnarray}

Suppose the particle was initially oscillating in the plane $ZOX$, i.e., the initial position of the plane $ZOX'$. We also assume that the particle was initially pulled to a distance $x_0$ and then let go, so that initially $x'=x_0$, $y'=0$, $\dot{x}'=0$, $\dot{y}'=0$ at $t=0$.\\

Hence $$(x'+iy')=(A_1+iA_2)e^{ipt}+(B_1+iB_2)e^{-ipt}$$
 where $A_i$'s and $B_i$'s $(i=1,2)$ are real constants.\\
 
  Now using the initial conditions
 \begin{eqnarray}
 	x_0=(A_1+B_1)+i(A_2+B_2)\nonumber\\0=(A_1-B_1)+i(A_2-B_2)\nonumber\\\mbox{i.e.,~ }A_1=\frac{x_0}{2}=B_1~,~~A_2=0=B_2\nonumber
 \end{eqnarray} 

So, \begin{eqnarray}
	(x'+iy')&=&\frac{x_0}{2}\left(e^{ipt}+e^{-ipt}\right)=x_0\cos pt\nonumber\\\mbox{i.e.,}&&x'=x_0\cos pt~, ~~y'=0\nonumber
\end{eqnarray}

This shows that the motion of the particle in the particle plane $ZOX'$ is simple harmonic and the vertical plane is rotating about the vertical at $O$ with angular velocity $\omega\sin\lambda$ from east to west through south.\\

Further, as the component of the angular velocity of Earth about the vertical through $O$ is $\omega\sin\lambda$ from west to east through south so that the vertical plane of oscillation of the pendulum turns round the vertical through $O$ with an angular velocity equal and opposite to that of Earth. Equivalently, the vertical plane of oscillation is fixed in space and earth is rotating wih an angular velocity $\omega\sin\lambda$ about the vertical through $O$.\\

However, in general, $$(x'+iy')=(A_1+iA_2)e^{ipt}+(B_1+iB_2)e^{-ipt}$$

Now, separating real and imaginary parts we can show that the projection of the path of the particle on the $X'Y'$ plane is an ellipse and this ellipse is rotating about the vertical with angular velocity $\omega\sin\lambda$.\\

\section {Nonlinear oscillations} 

Many important physical problems have evolution equation of the following general form:
\begin{equation}\label{eq3.120}
	\ddot{x}+\mu f(x,\dot{x})+\omega^2x=0
\end{equation}
 where $\mu$ is a small parameter. This type of equation is known as quasi-linear differential equation. There is no general solution of such differential equation -- only approximate solutions can be obtained for such type of differential equation. In the context of nonlinear oscillation we shall discuss the solution of (\ref{eq3.120}) using the method of Krylov and Bogoliubov.\\
 
Note that equation (\ref{eq3.120}) reduces to simple harmonic oscillation: $\ddot{x}+\omega^2x=0$ when $\mu=0$. So for $\mu=0$ the solution can be taken as
\begin{equation}\label{eq3.121}
	x=a\sin(\omega t+\phi)
\end{equation}
 where $a$ and $\phi$ are arbitrary constants.\\
 
 We now assume that the solution of equation (\ref{eq3.120}) is of the form (\ref{eq3.121}) with $a=a(t)$ and $\phi=\phi(t)$ i.e.,
 \begin{equation}\label{eq3.122}
 	x(t)=a(t)\sin\{\omega t+\phi(t)\}
 \end{equation}
 as well as the first order derivatives has the form
 \begin{equation}\label{eq3.123}
 	\dot{x}(t)=a\omega\cos(\omega t+\phi)
 \end{equation}

Now differentiating (\ref{eq3.122}) with respect to $t$ gives
\begin{equation}
	\dot{x}(t)=\dot{a}(t)\sin\{\omega t+\phi(t)\}+a(t)\omega\cos\{\omega t+\phi(t)\}+a(t)\dot{\phi}(t)\cos\{\omega t+\phi(t)\}\nonumber
\end{equation}

Now comparing with equation (\ref{eq3.123}) we get
\begin{equation}\label{eq3.124}
	\dot{a}\sin\{\omega t+\phi\}+a\dot{\phi}\cos\{\omega t+\phi\}=0
\end{equation}

Also differentiating (\ref{eq3.123}) gives
\begin{equation}
		\ddot{x}=\dot{a}\omega\cos\{\omega t+\phi\}-a\omega^2\sin\{\omega t+\phi\}-a(t)\omega\dot{\phi}\sin\{\omega t+\phi\}\nonumber
\end{equation}

Substituting $\theta=\omega t+\phi$ we obtain
\begin{eqnarray}\label{eq3.125}
	x=a\sin\theta~,~~\dot{x}=a\omega\cos\theta~,~~\ddot{x}=\dot{a}\omega\cos\theta-a\omega^2\sin\theta-a\omega\dot{\phi}\sin\theta
\end{eqnarray}

Using (\ref{eq3.125}) in (\ref{eq3.120}) gives
\begin{equation}\label{eq3.126}
	\dot{a}\omega\cos\theta-a\omega\dot{\phi}\sin\theta=-\mu f(a\sin\theta,a\omega\cos\theta)
\end{equation}

Now solving for $\dot{a}$ and $\dot{\phi}$ from equations (\ref{eq3.124}) and (\ref{eq3.126}) we have
\begin{eqnarray}
	\dot{a}&=&-\frac{\mu}{\omega} f(a\sin\theta,a\omega\cos\theta)\cos\theta\label{eq3.127}\\\dot{\phi}&=&\frac{\mu}{a\omega} f(a\sin\theta,a\omega\cos\theta)\sin\theta\label{eq3.128}
\end{eqnarray}
which are nothing but the time variation of amplitude (i.e., $a$) and phase (i.e., $\phi$) with time. Note that till now we have not made any approximation so equations (\ref{eq3.127}) and \ref{eq3.128} are exact in nature. As $\mu$ is small so $\dot{a}$ and $\dot{\phi}$ are small i.e., $a$ and $\phi$ are slowly varying function of $t$. Also in time $\dfrac{2\pi}{\omega}$, $\theta=\omega t+\phi$ will increase by an amount $2\pi$ approximately without appreciable change in $a$ and $\phi$.\\ 

According to Krylov and Bogoliubov, the right hand side of equations (\ref{eq3.127}) and (\ref{eq3.128}) can be replaced by their average values over the range of $\theta$ from $)$ to $2\pi$ as their first approximation, considering the amplitude $a$ as constant. Hence equations (\ref{eq3.127}) and (\ref{eq3.128}) become (in the first approximation)
\begin{eqnarray}
	\dot{a}&=&-\frac{\mu}{2\pi\omega}\int_0^{2\pi} f(a\sin\theta,a\omega\cos\theta)\cos\theta\mathrm{d}\theta\label{eq3.129}\\\dot{\phi}&=&\frac{\mu}{2\pi a\omega}\int_0^{2\pi} f(a\sin\theta,a\omega\cos\theta)\sin\theta\mathrm{d}\theta\label{eq3.130}
\end{eqnarray}

Thus equations (\ref{eq3.129}) and (\ref{eq3.130}) represent the general manner in which the amplitude $a$ and phase $\phi$ vary for various choices of $f(x,\dot{x})$.\\

As a particular case, if $f(x,\dot{x})=x^n$ then form (\ref{eq3.129}) it follows that $\dot{a}=0$ and $\dot{\phi}\neq0$ from equation (\ref{eq3.130}). Hence amplitude is constant But phase  varies. But reverse is the situation if $f(x,\dot{x})=\dot{x}^n$ i.e., $\phi$ is constant while the amplitude varies.\\

{\bf Example 1:} $$\ddot{x}+\mu x^3+\omega^2x=0\mbox{ , ~~$\mu$ is small}$$

We shall determine approximate solution of this equation using the method of Krylov and Bogoliubov. The approximate solution is given by
$$x=a(t)\sin\theta(t)~,~~\theta(t)=\omega t+\phi(t)$$ 
where the amplitude $a$ and phase $\phi$ change with time as
\begin{eqnarray}
	\dfrac{\mathrm{d}a}{\mathrm{d}t}&=&-\frac{\mu}{2\pi\omega}\int_0^{2\pi} f(a\sin\theta,a\omega\cos\theta)\cos\theta\mathrm{d}\theta\nonumber\\\dfrac{\mathrm{d}\phi}{\mathrm{d}t}&=&\frac{\mu}{2\pi a\omega}\int_0^{2\pi} f(a\sin\theta,a\omega\cos\theta)\sin\theta\mathrm{d}\theta\nonumber
\end{eqnarray}

In the present problem, $f(x,\dot{x})=x^3=a^3\sin^3\theta$, hence
\begin{eqnarray}
	\dfrac{\mathrm{d}a}{\mathrm{d}t}&=&-\frac{\mu a^3}{2\pi\omega}\int_0^{2\pi} \sin^3\theta\cos\theta\mathrm{d}\theta=0\nonumber\\\dfrac{\mathrm{d}\phi}{\mathrm{d}t}&=&\frac{\mu a^2}{2\pi \omega}\int_0^{2\pi} \sin^4\theta\mathrm{d}\theta=\frac{\mu a^2}{2\pi \omega}\cdot4\cdot\frac{3}{4}\cdot\frac{1}{2}\cdot\frac{\pi}{2}\nonumber\\&=&\frac{3\mu a^2}{8\omega}\nonumber\\\mbox{i.e., ~}\phi&=&\frac{3\mu a^2}{8\omega}t+\phi_0\mbox{~~(assuming $a$ remains constant and $\phi=\phi_0$ at $t=0$)}\nonumber\\\therefore~x&=&a\sin(\omega t+\phi)=a\sin\left\{\omega t+\frac{3\mu a^2}{8\omega}t+\phi_0\right\}\nonumber\\&=&a\sin\left\{\left(\omega +\frac{3\mu a^2}{8\omega}\right)t+\phi_0\right\}\nonumber
\end{eqnarray}

So the period of oscillation is $$\frac{2\pi}{\omega\left(1+\frac{3\mu a^2}{8\omega^2}\right)}=\frac{2\pi}{\omega}\left(1-\frac{3\mu a^2}{8\omega^2}\right)$$ 

Thus if $\mu>0$, the period decreases with the increase of amplitude $a$.\\ 

{\bf Example 2:} $$\ddot{x}+\mu (1-x^2)\dot{x}+x=0$$

Here $\ddot{x}+x=0$ has solution $$x=a\sin\theta~,~~\theta=t+\phi$$  

So the time variation of $a$ and $\phi$ are given by
\begin{eqnarray}
	\dfrac{\mathrm{d}a}{\mathrm{d}t}&=&-\frac{\mu}{2\pi}\int_0^{2\pi} (a^2\sin^2\theta-1)a\cos^2\theta\mathrm{d}\theta\nonumber\\&=&-\frac{\mu a^3}{2\pi}\int_0^{2\pi} \sin^2\theta\cos^2\theta\mathrm{d}\theta+\frac{\mu a}{2\pi}\int_0^{2\pi} \cos^2\theta\mathrm{d}\theta\nonumber\\&=&-\frac{\mu a^3}{2\pi}\cdot4\frac{\Gamma\left(\frac{3}{2}\right)~\Gamma\left(\frac{1}{2}\right)}{2\Gamma(3)}+\frac{\mu a}{2\pi}\frac{2\pi}{2}=\frac{\mu a}{2}\left(1-\frac{a^2}{4}\right)\nonumber\\\dfrac{\mathrm{d}\phi}{\mathrm{d}t}&=&\frac{\mu }{2\pi a}\int_0^{2\pi} (a^2\sin^2\theta-1)a\sin\theta\cos\theta\mathrm{d}\theta\nonumber\\&=&\frac{\mu a^2}{2\pi}\int_0^{2\pi} \sin^3\theta\cos\theta\mathrm{d}\theta-\frac{\mu}{2\pi}\int_0^{2\pi} \sin\theta\cos\theta\mathrm{d}\theta=0\nonumber\\\therefore~\phi=\phi_0~, \mbox{ a constant.}\nonumber\\\mbox{Also }\dfrac{\mathrm{d}a}{\mathrm{d}t}&=&\frac{\mu a}{2}\left(1-\frac{a^2}{4}\right)\nonumber\\\implies\int\frac{\mathrm{d}a}{a\left(1-\frac{a^2}{4}\right)}&=&\int\frac{\mu}{2}\mathrm{d}t\nonumber\\\mbox{i.e., ~}\frac{2a}{\sqrt{4-a^2}}&=&e^{\left(\frac{\mu}{2}t+c\right)}\nonumber\\\mbox{i.e.,~ } a^2&=&\frac{4e^{\left(\frac{\mu}{2}t+c\right)}}{4+e^{\left(\frac{\mu}{2}t+c\right)}}\nonumber\rightarrow\mbox{ this gives the explicit time dependence of }a.
\end{eqnarray}

Note that $a\rightarrow 2$ as $t\rightarrow\infty$. Also $\dfrac{\mathrm{d}a}{\mathrm{d}t}=0$ for $a=2$. Hence the oscillatory motion is steady i.e., a steady state is reached after infinite time. Thus the solution is $$x(t)=a(t)\sin(t+\phi_0)$$

{\bf Example 3:} $$\ddot{x}+\epsilon \dot{x}|\dot{x}|+x=0$$

For $\epsilon=0$ the solution is $$x=a\sin\theta~,~~\dot{x}=a\cos\theta~,~~\theta=t+\phi$$ 

Here $$f(x,\dot{x})=\dot{x}|\dot{x}|=\left\{\begin{array}{cl}
	\dot{x}^2&,~\dot{x}>0\\-\dot{x}^2&,~\dot{x}<0
\end{array}\right.$$

\begin{eqnarray}
	\dfrac{\mathrm{d}a}{\mathrm{d}t}&=&-\frac{\mu}{2\pi}\int_0^{2\pi}f(x,\dot{x})\cos\theta\mathrm{d}\theta\nonumber\\&=&-\frac{\epsilon}{2\pi}\left[\int_0^{\frac{\pi}{2}}a^2\cos^2\theta\cos\theta\mathrm{d}\theta-\int_{\frac{\pi}{2}}^{\frac{3\pi}{2}}a^2\cos^2\theta\cos\theta\mathrm{d}\theta+\int_{\frac{3\pi}{2}}^{2\pi}a^2\cos^2\theta\cos\theta\mathrm{d}\theta\right]\nonumber\\&=&-\frac{\epsilon a^2}{2\pi}\left[\frac{2}{3}+\frac{4}{3}+\frac{2}{3}\right]=-\frac{4\epsilon a^2}{3\pi}\nonumber\\\mbox{i.e., ~}\frac{1}{a}&=&\frac{4\epsilon}{3\pi}\frac{1}{t}+\frac{1}{a_0}\nonumber
\end{eqnarray} 

As before $\dfrac{\mathrm{d}\phi}{\mathrm{d}t}=0$. Hence $$x=a\sin(t+\phi_0)~,~~a=\left(\frac{4\epsilon}{3\pi}\frac{1}{t}+\frac{1}{a_0}\right)^{-1}.$$

{\bf Example 4:} $$\ddot{x}-2k\dot{x}+c\dot{x}^3+\omega^2x=0~,~~k,c \mbox{ are small positive constants}$$

So solution will be of the form $$x=a(t)\sin\theta~,~~\dot{x}=a\omega\cos\theta~,~~\theta=\omega t+\phi(t)$$
\begin{eqnarray}
	\dfrac{\mathrm{d}a}{\mathrm{d}t}&=&-\frac{c}{2\pi\omega}\int_0^{2\pi}\left(a^3\omega^3\cos^3\theta-\frac{2k}{c}a\omega\cos\theta\right)\cos\theta\mathrm{d}\theta\nonumber\\&=&-\frac{ca^3\omega^2}{2\pi}\frac{3\pi}{4}+\frac{ka}{\pi}\pi=-\frac{3a^3c\omega^2}{8}+ka\nonumber\\\dfrac{\mathrm{d}\phi}{\mathrm{d}t}=0\nonumber\\\therefore~\int\frac{\mathrm{d}a}{a\left(k-\frac{3}{8}a^2c\omega^2\right)}&=&\int\mathrm{d}t\nonumber\\\mbox{i.e., ~}\frac{k}{a^2}-\frac{3}{8}c\omega^2&=&a_0e^{-2t}\nonumber\\a&=&\frac{1}{\sqrt{k}}\left(a_0e^{-2t}+\frac{3}{8}c\omega^2\right)^{-\frac{1}{2}}\nonumber
\end{eqnarray}

Hence $a\rightarrow \dfrac{2\sqrt{2}}{\sqrt{3kc}}\dfrac{1}{\omega}$ as $t\rightarrow\infty$.\\

Hence there will be a steady motion after infinite time.\\

\section{Problem of three bodies: (three body problem)} 

We consider the motion of three bodies having masses $m_1$, $m_2$ and $m_3$. Suppose they are in motion under their mutual gravitational attraction. Let $(x_1,y_1,z_1)$, $(x_2,y_2,z_2)$ and $(x_3,y_3,z_3)$ be the coordinates of the bodies (i.e., coordinates of the centre of gravity of the bodies) at any time. The distances between the masses i.e., $(m_1,m_2)$, $(m_1,m_3)$ and $(m_2,m_3)$ are chosen as $r_{12}$, $r_{13}$ and $r_{23}$ respectively. So we have
$$r_{ij}^2=(x_i-x_j)^2+(y_i-y_j)^2+(z_i-z_j)^2~, ~~i,j=1,2,3$$

The potential energy of the masses $m_2$ and $m_3$ at the position of $m_1$ is 
$$V_1=G\frac{m_2}{r_{12}}+G\frac{m_3}{r_{13}}$$
 where $G$ is the gravitational constant. So the equations of motion of mass $m_1$ are 
 $$\frac{\mathrm{d}^2x_1}{\mathrm{d}t^2}=\frac{\partial V_1}{\partial x_1}~,~~\frac{\mathrm{d}^2y_1}{\mathrm{d}t^2}=\frac{\partial V_1}{\partial y_1}~,~~\frac{\mathrm{d}^2z_1}{\mathrm{d}t^2}=\frac{\partial V_1}{\partial z_1}$$
 
Note that if we define  $V=G\left(\dfrac{m_1m_2}{r_{12}}+\dfrac{m_2m_3}{r_{23}}+\dfrac{m_3m_1}{r_{31}}\right)$
  then in the above equations of motion we can replace $V_1$ by $V$ and similarly we can write down the equations of motion for the masses $m_2$ and $m_3$. In particular, the equations of motion of these three masses can be written in compact form as 
  $$\frac{\mathrm{d}^2x_i}{\mathrm{d}t^2}=\frac{\partial V}{\partial x_i}~,~~\frac{\mathrm{d}^2y_i}{\mathrm{d}t^2}=\frac{\partial V_1}{\partial y_i}~,~~\frac{\mathrm{d}^2z_i}{\mathrm{d}t^2}=\frac{\partial V_1}{\partial z_i}~,~~i=1,2,3$$
 
Since there are no external forces acting on the system except their mutual gravitational attraction, so the centre of mass (c.m) of the system moves uniformly and we write 
$$\sum\limits_{i=1}^3m_i\frac{\mathrm{d}x_i}{\mathrm{d}t}=a_x~,~~\sum\limits_{i=1}^3m_i\frac{\mathrm{d}y_i}{\mathrm{d}t}=a_y~,~~\sum\limits_{i=1}^3m_i\frac{\mathrm{d}z_i}{\mathrm{d}t}=a_z$$
 where $(a_x,a_y,a_z)$ are constants.\\
 
Now integrating the above equations of motion of the centre of mass we obtain 
$$\sum\limits_{i=1}^3m_ix_i=a_xt+b_x~,~~\sum\limits_{i=1}^3m_iy_i=a_yt+b_y~,~~\sum\limits_{i=1}^3m_iz_i=a_zt+b_z$$
 
 These are termed as integrals of the centre of mass of the system. Since the moments of the forces acting on the particles about the co-ordinate axes vanish therefore the angular momentum of the system about the coordinate axes remains constant. Thus we have 
 $$\sum\limits_{i=1}^3m_i\left(y_i\frac{\partial z_i}{\partial t}-z_i\frac{\partial y_i}{\partial t}\right)=c,$$ 
 and similar two equations. These are the integrals of the angular momentum of the system of bodies. The kinetic energy of the system is $$T=\frac{1}{2}\sum\limits_{i=1}^3m_i\left\{\left(\frac{\mathrm{d}x_i}{\mathrm{d}t}\right)^2+\left(\frac{\mathrm{d}y_i}{\mathrm{d}t}\right)^2+\left(\frac{\mathrm{d}z_i}{\mathrm{d}t}\right)^2\right\}$$
 
There do not exist any other integrals of the general three body problem.\\
 
 In terms of Polar coordinates the position of the centre of gravity of the bodies can be written as $(r_1,\theta_1)$, $(r_2,\theta_2)$ and $(r_3,\theta_3)$. Then the kinetic energy is given by $$T=\frac{1}{2}m_1\left(\dot{r}_1^2+r_1^2\dot{\theta}_1^2\right)+\frac{1}{2}m_2\left(\dot{r}_2^2+r_2^2\left(\dot{\theta}_1^2+\dot{\theta}_2^2\right)\right)+\frac{1}{2}m_3\left(\dot{r}_3^2+r_3^2\left(\dot{\theta}_1^2+\dot{\theta}_3^2\right)\right)$$ and $$V=\frac{m_2m_3}{\sqrt{r_2^2+r_3^2-2r_2r_3\cos(\theta_2-\theta_3)}}+\frac{m_3m_1}{\sqrt{r_3^2+r_1^2-2r_3r_1\cos\theta_3}}+\frac{m_1m_2}{\sqrt{r_1^2+r_2^2-2r_1r_2\cos\theta_2}}$$
 
 Note that here $\theta_1$ is a cyclic co-ordinate i.e., $$\frac{\partial L}{\partial\theta_1}=0$$
 
Hence the three bodies move in a plane like a single particle with 2 degrees of freedom.


