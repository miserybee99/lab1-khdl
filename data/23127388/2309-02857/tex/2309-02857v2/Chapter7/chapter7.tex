%\pagenumbering{arabic}



\chapter{Small oscillation}

\section{Oscillation}


Let us consider a holonomic dynamical system with $n$ degrees of freedom in which the coordinates are independent of time. Let $q_1,q_2, ...q_n$ be the be the generalised coordinates of the system. Let the external forces be conservative and derived from a potential function $V$.  The position of equilibrium of the system is given by

\begin{equation}\label{eqn1}
\frac{\partial V}{\partial q_1} =0=\frac{\partial V}{\partial q_2} ... =	\frac{\partial V}{\partial q_n}~.
\end{equation}


Solving these equations we get the position of the equilibrium of the system. In the position of the equilibrium of the system, we can without any loss of generality consider $q_1,q_2, ...q_n$ to be zero. The equation (\ref{eqn1}) shows that  V is stationary in the position of the equilibrium. If further V is  minimum, then the equilibrium is stable. This means that if we give a small disturbance to the system, the coordinates $q_1,q_2, ...q_n$ and the velocity $\dot{q}_1,\dot{q}_2,..\dot{q}_n$  remain small. Since the  connection of the system is a quadratic expression in $\dot{q}_1,\dot{q}_2,..\dot{q}_n$  and  therefore,

\begin{equation}
T=\frac{1}{2}\sum_{i}\sum_{j} a_{i,j}\dot{q_i}\dot{q_j}\label{n2}
\end{equation}
where $a_{ij}=a_{ji}$,~~$i,j=1,2,...,n$.

Since the subsequent motion,  $q_1,q_2, ...q_n$ and as well as $\dot{q}_1,\dot{q}_2,..\dot{q}_n$ remain small, if we want to retain the quantities upto 2nd order in the expression of $T$, we can neglect the 1st order quantities in the coeff $a_{i,j}~ i.e.,$ we can put $q_1=q_2=...=q_n=0$ in the expression for $a_{ij}$. So we consider $a_{ij}$ as constants. Since $V$ is indeterminate to the extent of an additive constant we can choose the constant in such a manner that $V$ vanishes in the position of equilibrium. Expanding $V$ in Taylor series about the point $q_1=q_2=...=q_n=0$ and taking account of the relation (1) , we can put $V$ in the form 
\begin{eqnarray}
V=\frac{1}{2}\sum_{i}\sum_{j} c_{i,j} q_i q_j,\label{n3}
\end{eqnarray} 
 in which $c_{ij}=c_{ji},~ i,j=1,2,3,...n.$


In this expression for $V$ we have neglected the 3rd and higher order terms.

Now the Lagrange's equation of motion of the system are
\begin{equation}\label{Leq}
\frac{d}{dt}\left(\frac{\partial T}{\partial{\dot{q}_r}}\right)-\frac{\partial T}{\partial{q}_r} +\frac{\partial V}{\partial{{q}_r}}=0~ ,~~~r=1,2,3...n.\nonumber
\end{equation}

Substituting the value of $T$ and $V$ given in equations (\ref{n2}) and (\ref{n3}) respectively, we get
\begin{equation}
a_{r1}\ddot{q_1}+a_{r2}\ddot{q_2}+...+a_{rn}\ddot{q_n}+c_{r1}q_1 +c_{r2}q_2 +...+c_{rn}q_n=0 ~, ~~r=1,2,...n.\label{n4}
\end{equation}


This is a set of linear homogeneous equations in $q_r$'s. To solve it we choose $q_r=A_re^{\lambda t}$ as a trial solution ($A_r$ and $\lambda$ are constants). Therefore substituting in (\ref{n4}) one gets,


\begin{equation}
(a_{r1}\lambda^2+c_{r1})A_1 +(a_{r2}\lambda^2+c_{r2})A_2+ ... +(a_{rn}\lambda^2+c_{rn})A_n=0,~ r=1,2,...n.
\end{equation}



or in explicit form  

\begin{eqnarray}
(a_{11}\lambda^2+c_{11})A_1 +(a_{12}\lambda^2+c_{12})A_2+ ... +(a_{1n}\lambda^2+c_{1n})A_n&=&0,\nonumber\\ \nonumber
(a_{21}\lambda^2+c_{21})A_1 +(a_{22}\lambda^2+c_{22})A_2+ ... +(a_{2n}\lambda^2+c_{2n})A_n&=&0, \\ \nonumber
-----------------------  \\ \nonumber
-----------------------  \\ \nonumber
(a_{n1}\lambda^2+c_{n1})A_1 +(a_{n2}\lambda^2+c_{n2})A_2+ ... +(a_{nn}\lambda^2+c_{nn})A_n&=&0.
\end{eqnarray}	

Now eliminating $A_1$, $A_2$, ... $A_n$, we have

\begin{equation}
\Delta(\lambda)\equiv
\begin{bmatrix} 
(a_{11}\lambda^2+c_{11}) & (a_{12}\lambda^2+c_{12})&\cdots & (a_{1n}\lambda^2+c_{1n}) \\
(a_{21}\lambda^2+c_{21}) & (a_{22}\lambda^2+c_{22})&\cdots & (a_{2n}\lambda^2+c_{2n}) \\
\vdots & & \ddots & \vdots \\ 
(a_{n1}\lambda^2+c_{n1}) & (a_{n2}\lambda^2+c_{n2})&\cdots & (a_{nn}\lambda^2+c_{nn})\\
\end{bmatrix}
=0\nonumber
\end{equation}
This equation is known as frequency equation and $\Delta(\lambda)$ is called a Lagrange's determinant or harmonic determinant. It can be proved that the roots of the Lagrange's determinantal equation in $\lambda^2$ are all real. Further, if $V$ is essentially $+ve$ the roots of the Lagrange's determinantal equation in $\lambda^2$ are $-ve$, $i.e.,$ we write $\lambda^2=-\sigma^2$ where $\sigma$ is real and $+ve$.

If $\alpha_1$, $\alpha_2$, ... $\alpha_n$, be the minors of any one  row of $\Delta(\lambda)$, then we have
%
\begin{equation}
\frac{A_1}{\alpha_1}=\frac{A_2}{\alpha_2}=...=\frac{A_n}{\alpha_n}=H ~(say)\nonumber
\end{equation}
Since $\lambda$ appears as $\lambda^2$ in the determinant $\Delta(\lambda)$, therefore $\alpha_1$, $\alpha_2$, ... $\alpha_n$ have the same values when we put $\lambda=\pm i\sigma$. Then corresponding to a root $\lambda^2=- \sigma^2$, we have the solution
%
\begin{eqnarray}
q_r&=&\alpha_r(H e^{i\sigma t} + K e^{-i\sigma t}) \nonumber\\ \nonumber
&=& c \alpha_r cos(\sigma t +\sigma)~,~r=1,2,...,n. ( \because \alpha_r ~\text{have the same values for} \lambda=\pm \sigma)
\end{eqnarray}
%
The constants $c$ and $\epsilon$ are same for all values of $r$. This solution is said to constitute a normal mode of oscillation or a fundamental mode of oscillation of the system corresponding to $n$ values of $\lambda^2$ satisfying Lagrange's determinantal equation. For the $n$ normal modes of oscillation
%
\begin{equation}
q_r=c\alpha_r cos(\sigma t+\epsilon)+c^1\alpha_r cos(\sigma t+\epsilon^1)~,\nonumber
\end{equation}
to calculate  $\alpha_r$, one substitutes $\lambda^2=-\sigma^2$.


The constants $c,c^1 ...$ and $\epsilon, \epsilon^1 ...$etc. are to be determined from initial conditions. Since $c,c^1$ are small, therefore in the subsequent motion $q_1,q_2...q_n$ and $\dot{q_1},\dot{q_2}...\dot{q_n}$ remains small.

\vspace{.5cm}

\underline{ \textbf{Note:}} We can from the consideration of mechanics prove that the roots of the determinantal equation in $\lambda^2$ are real. On the assumption that the position of the equilibrium is characterized $q_1=0=q_2=...=q_n$ and this is a stable equilibrium if $V$ is minimum there. If possible, let $\lambda=\pm (\tau+i \delta)$ corresponding to a root of the determinantal equation. Since the co-efficient of this equation are all real, therefore $\lambda=\pm (\tau-i \delta)$ could also give the root of the equation. As a particular solution of the equation of motion we can then take
%
\begin{equation}
q_r=A_r(e^{\tau t}+e^{-\tau t})cos(\delta t)+B_r(e^{\tau t}-e^{-\tau t})sin(\delta t)~, r=1,2,...n.\nonumber
\end{equation}
% 
The motion subsequent to the disturbance begins with small values of $q_r$'s and $\dot{q_r}$'s. But this co-ordinates and velocities increase indefinitely with time. This contradicts the assumption that the position of the equilibrium is stable. Thus the necessary condition for stability of equilibrium is that $\tau=0$, so that $\lambda^2=-\delta^2$. Thus the roots of the determinantal equation are not only real but $-ve$ also.

\vspace{.5cm}

$\bullet$ \textbf{Problem:} One point of a uniform circular hoop of mass $M$ and radius $a$ is fixed and hoop is free to move in a vertical plane through the fixed point. A bead of mass $m$ slides on the hoop and there is no friction. Prove that the periods of small oscillation about the position of the stable equilibrium are $2\pi \sqrt{\dfrac{2a}{g}}$ and $2\pi \sqrt{\dfrac{M}{m+M}\frac{a}{g}}$.

\vspace{.25cm}

\textbf{Solution:} The co-ordinates of the system are $\theta$ and $\phi$, the angles which $OC$ and $CP$ make with the downward vertical.


\begin{figure}[h!]
	\centering
	\includegraphics[scale=0.33]{chapter4-1st.pdf}
\end{figure}

\begin{eqnarray}
\therefore ~T&=&\frac{1}{2}M(2 a^2)\dot{\theta}^2 + \frac{1}{2}m[a^2 \dot{\theta}^2+a^2 \dot{\phi}^2+2a^2 \dot{\theta}\dot{\phi}cos(\theta-\phi)]  \nonumber \\ 
V&=&-M g a cos\theta - mg(a cos\theta +a cos\phi)+V_0\nonumber
\end{eqnarray}
%
( $V_0,$ a constant, the initial P.E.)

Since $\theta$ and $\phi$ are very small, so $\theta \approx 1-\theta^2/2$, $cos\phi \approx 1-\phi^2/2$, $cos(\theta-\phi)\approx 1.$

\begin{eqnarray}
T&=&\frac{1}{2}M(2 a^2)\dot{\theta}^2 + \frac{1}{2}m[a^2 \dot{\theta}^2+a^2 \dot{\phi}^2+2a^2 \dot{\theta}\dot{\phi}] \nonumber \\
V&=&-(M+m)ga(1-\theta^2/2)-mga(1-\phi^2/2)+V_0\nonumber
\end{eqnarray}
%
The Lagrange's equations are 
%
\begin{eqnarray}
\frac{d}{dt}\left( \frac{\partial T}{\partial \dot{\theta}}\right) - \frac{\partial T}{\partial \theta} +\frac{\partial V}{\partial \theta}=0 \nonumber \\
\frac{d}{dt}\left( \frac{\partial T}{\partial \dot{\phi}}\right) - \frac{\partial T}{\partial \phi} +\frac{\partial V}{\partial \phi}=0~.\nonumber
\end{eqnarray}
%
$\therefore$

\begin{eqnarray}
\frac{d}{dt}[Ma^22\dot{\theta}+\frac{1}{2}m(2a^2\dot{\theta}+2a^2\dot{\phi})]+(m+M)ga\theta&=&0 \nonumber \\
or, ~ 2Ma^2 \ddot{\theta}+ma^2(\ddot{\theta}+\ddot{\phi}) +(m+M)ag\theta&=&0 \nonumber \\
or,~ \ddot{\theta}(2M+m)a +ma\ddot{\phi}+(m+M)g\theta&=&0 \nonumber \\
\text{Similarly,}~~~~~~~~~~~~~~~~~~~~~~~~~~~~\nonumber \\ ~~ (m\ddot{\theta} +M\ddot{\phi})a +mg\phi&=&0\nonumber
\end{eqnarray}

If $\theta =A_1e^{\lambda t},~\phi=A_2e^{\lambda t}$. Then the above two equations become
\begin{eqnarray}
\left[a(2M+m)\lambda^2 +(M+m)g\right]A_1 +A_2(ma \lambda^2)&=&0	\nonumber \\
(am\lambda^2)A_1+(am\lambda^2+mg)A_2&=&0\nonumber
\end{eqnarray}
%
Now eliminating $A_1$ and $A_2$,


\begin{eqnarray}
\begin{bmatrix} 
a(2M+m)\lambda^2 +(m+_M)g & ma\lambda^2  \\
ma\lambda^2 & ma\lambda^2+mg \\
\end{bmatrix} &=&0 \nonumber \\
a^2(2Mm +m^2)\lambda^4 + (Mm +m^2)g^2 +~~ \nonumber \\  ~ag\lambda^2(Mm+m^2+2Mm+m^2) -m^2 a^2 \lambda^4 &=&0 \nonumber \\
2Ma^2\lambda^4 +ag(3M+2m)\lambda^2 +g^2(m+M)&=&0 \nonumber \\
\therefore \lambda^2 =\frac{-ag(3M+2m)\pm\sqrt{a^2g^2(3M+2m)^2-8a^2g^2M(m+M)}}{4Ma^2}&& \nonumber \\
=\frac{-ag(3M+2m)\pm ag(M+2m)}{4Ma^2}=-\frac{g}{2a},~-\frac{g(M+m)}{aM}~~~~&&\nonumber
\end{eqnarray}
%
$\because \lambda=-\sigma^2 \Rightarrow \sigma=\sqrt{\dfrac{g}{2a}}, \text{or}~ \sqrt{\dfrac{g(M+m)}{aM}}$ $\therefore \theta=A cos\left (\sqrt{\dfrac{g}{2a}}t +\epsilon_1 \right) $,$\phi=Bcos\left (\sqrt{\dfrac{g(M+m)}{aM}}t+\epsilon_2 \right ). $
%
The time periods are $2\pi\sqrt{\dfrac{2a}{g}},~~2\pi \sqrt{\dfrac{aM}{g(M+m)}}.$

\vspace{.5cm}

$\bullet$ \textbf{Problem:} A smooth circular wire of mass $8m$ and radius $a$ swings in a vertical plane being suspended by an inextensible string of length a attached to one point of it. A particle of mass $m$ can slide on the wire. Prove that, the period of small oscillation about the stable equilibrium are $2\pi \sqrt{\dfrac{8a}{3g}},2\pi \sqrt{\dfrac{a}{3g}},\text{and}~ 2\pi \sqrt{\dfrac{8a}{9g}}$. Also show that the angle between the radius to the bead and the diameter through the pt of suspension is a normal coordinate. 

\vspace{.25cm}

\textbf{Solution:} The K.E of the circular wire $E_k=\dfrac{1}{2}Mv^2+\dfrac{1}{2}M K^2\dot{\theta}^2$, where $v$ is the velocity of the centre, $K$ is the radius of gyration of the wire about a line through $C\perp$ to the plane of motion.
$\therefore v^2=a^2\dot{\psi}^2 +a^2\dot{\theta}^2+2a^2\dot{\theta}\dot{\psi} cos(\theta-\psi)$.
Since $\theta$ and $\psi$ are very small, so $cos(\theta-\psi)=1$.
%
\begin{eqnarray}
v^2&=&a^2(\dot{\theta}+\dot{\psi})^2, k^2=a^2 \nonumber \\
\therefore, T&=&\frac{1}{2}M a^2 (\dot{\theta}+\dot{\psi})^2 +\frac{1}{2}M a^2 \dot{\theta}^2 +\frac{1}{2}m a^2 (\dot{\theta}+\dot{\phi}+\dot{\psi})^2 \nonumber \\
V&=&-9mga(2-\frac{\theta^2}{2}-\frac{\psi^2}{2})-mga (1-\phi^2/2)\nonumber
\end{eqnarray}



\begin{figure}[h!]
	\centering
	\includegraphics[scale=0.33]{chapter4-2nd.pdf}
\end{figure}


%
So the L's equations are
%
\begin{eqnarray}
\frac{d}{dt}\left(  \frac {\partial T}{\partial  \dot{\theta}} \right)- \frac {\partial T}{\partial \theta}+\frac {\partial V}{\partial \theta}=0 & \Rightarrow &  17 a \ddot{\theta}+a \ddot{\phi} +9 a\ddot{\psi}+9 g\theta=0 \label{n6}  \\
\frac{d}{dt}\left(  \frac {\partial T}{\partial  \dot{\phi}} \right)- \frac {\partial T}{\partial \phi}+\frac {\partial V}{\partial \phi}=0 & \Rightarrow&  a \ddot{\theta}+a \ddot{\phi} + a\ddot{\psi}+ g\phi=0 \label{n7} \\
\frac{d}{dt}\left(  \frac {\partial T}{\partial  \dot{\psi}} \right)- \frac {\partial T}{\partial \psi}+\frac {\partial V}{\partial \psi}=0 & \Rightarrow&  9 a \ddot{\theta}+a \ddot{\phi} + 9 a\ddot{\psi}+ 9g\psi=0 \label{n8}
\end{eqnarray}
%
Substitute,  $\theta=A_1 e^{\lambda t},\phi=A_2e^{\lambda t}, \psi=A_3e^{\lambda t}$. So the above equations (\ref{n6})-(\ref{n8}) reduce to 
%
\begin{eqnarray}\label{4eqn17}
(17 a \lambda^2 +9g)A_1 + (a\lambda^2)A_2+ (9a\lambda^2)A_3=0\nonumber \\
(a\lambda^2)A_1+(a\lambda^2+g)A_2+(a\lambda^2)A_3=0 \nonumber \\
(9 a\lambda^2)A_1+(a\lambda^2)A_2+(9a\lambda^2+9g)A_3=0~.
\end{eqnarray}
Eliminating, $A_1,A_2,A_3$ we have
%
\begin{eqnarray}
\begin{bmatrix} 
(17 a \lambda^2 +9g) & a\lambda^2&9a\lambda^2\\
a\lambda^2& a\lambda^2+g & a\lambda^2\\
9a\lambda^2 & a\lambda^2)&9a\lambda^2+9g\\
\end{bmatrix}
&=&0 \nonumber \\
i.e., (8a\lambda^2+3g) (8a\lambda^2+9g) (a\lambda^2+3g)& = &0\nonumber
\end{eqnarray}
$\therefore  \lambda^2=-\dfrac{3g}{8a},-\dfrac{9g}{8a},-\dfrac{3g}{a}$

So the solutions are\\
$\theta=\alpha cos\left(\sqrt{\dfrac{3g}{8a}}t+\epsilon \right),\phi=\beta cos\left(\sqrt{\dfrac{9g}{8a}}t+\epsilon \right), \psi=\gamma cos\left(\sqrt{\dfrac{3g}{a}}t+\epsilon \right)$
Hence, periods of small oscillations are\\
$2\pi \sqrt{\dfrac{8a}{3g}},~2\pi \sqrt{\dfrac{8a}{9g}},~2\pi \sqrt{\dfrac{a}{3g}}$.\\
From equation (\ref{4eqn17}) putting $\lambda^2=-\dfrac{3g}{8a}$, we have 

\begin{equation}
-	\frac{3g}{8} A_1 +\frac{5g}{8}A_2-\frac{3g}{8}A_3=0\nonumber
\end{equation}

$i.e.,$ $3A_1-5A_2+3A_3=0\Rightarrow A_1:A_2:A_3=5:6:5$

putting $\lambda^2=-\dfrac{9g}{8a}$ in  equation (\ref{4eqn17})\\

$	-\dfrac{9g}{8}A_1-\dfrac{g}{8}A_2-\dfrac{9g}{8}A_3=0  $ \\
$9A_1+A_2+9A3=0\Rightarrow A_1:A_2:A_3=1:-18:1$

For $\lambda^2=-\dfrac{3g}{a}$ from  equation (\ref{4eqn17})\\
$-3gA_1-2gA_2-3gA_3=0$

$i.e,~~3A_1+2A_2+3A_3=0\Rightarrow A_1:A_2:A_3=1:-3:1$

$\therefore~\psi=5c_1 cos\left(\sqrt{\dfrac{3g}{a}}t+\epsilon \right)+c_2 cos\left(\sqrt{\dfrac{3g}{8a}}t+\epsilon \right)+c_3 cos\left(\sqrt{\dfrac{9g}{8a}}t+\epsilon \right)$ 	 \\
$~ ~~\theta=6c_1 cos\left(\sqrt{\dfrac{3g}{a}}t+\epsilon \right)-18c_2 cos\left(\sqrt{\dfrac{3g}{8a}}t+\epsilon \right)-3c_3 cos\left(\sqrt{\dfrac{9g}{8a}}t+\epsilon \right)$ 	 	\\
$~~~ \phi=5c_1 cos\left(\sqrt{\dfrac{3g}{a}}t+\epsilon \right)+c_2 cos\left(\sqrt{\dfrac{3g}{8a}}t+\epsilon \right)+c_3 cos\left(\sqrt{\dfrac{9g}{8a}}t+\epsilon \right)$ 	\\

\vspace{.5cm}

\section{Normal Co-ordinates and Small Oscillation}

\vspace{.25cm}

If the K.E. and P.E. of a vibrating system are in the form \\
\begin{eqnarray}
2T&=&a_{11}\dot{q_1}^2+a_{22}\dot{q_2}^2+...+a_{nn}\dot{q_n}^2+2a_{12}\dot{q_1}\dot{q_2}+...\nonumber\\ \nonumber
V&=&V_0+\frac{1}{2}(b_{11}q_1^2+b_{22}q_2^2+...+b_{nn}q_n^2+2b_{12}q_1q_2+...)
\end{eqnarray}

then it is always possible to find a linear transformation of the co-ordinates such that the K.E. and P.E. when expressed in terms of new co-ordinates have the form

\begin{eqnarray}
2T&=&\dot{Q_1}^2+\dot{Q_2}^2+...+\dot{Q_n}^2\nonumber \\ \nonumber
V&=&V_0+\frac{1}{2}(\delta_{1}^2Q_1^2+\delta_{2}^2Q_2^2+...+\delta_{n}^2Q_n^2)
\end{eqnarray}
where $\delta_1,\delta_2,...\delta_n$ are constants. These new cordnitaes are called normal co-ordinate or principle co-orrodinates.

When we have two homogeneous quadratic functions of any no. of variables all of which are $+ve$ for all values of the variables , then by a linear transformation,we may eliminate the product terms from both the expressions and the same time it is possible to make the coefficients of the square terms to be unity in any of the expressions.

Let the linear transformation of the coordinates be 
\begin{eqnarray}
q_1=b_{11}Q_1+b_{12}Q_2+....+b_{1n}Q_n\nonumber \\
q_2=b_{21}Q_1+b_{22}Q_1+....+b_{2n}Q_n \nonumber \\
--------------\nonumber \\
q_n=b_{n1}Q_1+b_{n2}Q_1+....+b_{nn}Q_n \label{n10}
\end{eqnarray}
%
where $b's$ are constants. So, obviously, the velocity will transform by the same set of relations. Hence by the above result it is possible to have a real transformation so that K.E. and P.E are transform to 


\begin{eqnarray}
2T&=&\dot{Q_1}^2+\dot{Q_2}^2+...+\dot{Q_n}^2\nonumber \\ \nonumber
2V&=&\lambda_{1}^2Q_1^2+\lambda_{2}^2Q_2^2+...+\lambda_{n}^2Q_n^2
\end{eqnarray}
$\lambda's$ are constant.


Thus the L's eq. of motion i.e.,

\begin{eqnarray}
\frac{d}{dt}\left( \frac{\partial T}{\partial \dot{Q_r}}\right) - \frac{\partial T}{\partial Q_r} =-\frac{\partial V}{\partial Q_r},~~~r =1,2,....n \nonumber 
\end{eqnarray}

gives ,
\\
$\ddot{Q_k}+\lambda {Q_k}=0,~~ k=1,2,...n$,\\
Assuming , $Q_k=A_ke^{\lambda t}$ i.e.,  $\ddot{Q}_k=\lambda^2Q_k$, Lagranges determinantal eq. becomes

\begin{eqnarray}
\begin{bmatrix} 
( \lambda^2 +\lambda_1) & 0&\dots&0\\
0& ( \lambda^2 +\lambda_2)& \dots&0\\
0 & 0&\dots&( \lambda^2 +\lambda_n)\\
\end{bmatrix}
=0\nonumber
\end{eqnarray}
It is shown in the theory of linear transformation of the variable that the roots of the determinantal equation are not affected by the transformation i.e., the periods are same for all systems of coordinates.

\vspace{.5cm}

\section{Effect of a constrain on the small oscillation of the system about equilibrium configuration }

\vspace{.25cm}

Let us consider the small oscillation of a holonomic dynamical system with n d.f. and let $q_1,q_2,...q_n$ be the normal coordinates of the system. Let us suppose that a constrain is introduced in the system and the constrain does no work. Then the equation of constrain is $f(q_1,q_2,...q_n)=0$. Since this relation is to be satisfied in the position of the equilibrium in which the coordinates vanish, it can be explained by the  Taylors theorem: 


\begin{equation}
f_0(q_1,q_2,...q_n)+\left[ \left( \frac{\partial f}{\partial q_1}\right)_0  q_1 + \left( \frac{\partial f}{\partial q_2}\right)_0  q_2+... \right]+...=0\nonumber
\end{equation}

Neglecting the higher powers of $q_r$, we have
%
\begin{eqnarray}\label{4eqn26}
\left( \frac{\partial f}{\partial q_1}\right)_0  q_1 + \left( \frac{\partial f}{\partial q_2}\right)_0  q_2+...+ \left( \frac{\partial f}{\partial q_n}\right)_0  q_n=0 \nonumber \\
i.e., ~ A_1q_1+A_2q_2+...+A_nq_n=0
\end{eqnarray} 
Where $A_r=\left( \dfrac{\partial f}{\partial q_r}\right)_0,~~r=1,2,...n. $ \\
The K.E. and P.E. of the system during this small oscillation can be written in the form

$T=\dfrac{1}{2}\sum \dot{q}_r^2,~~~V=V_0+\dfrac{1}{2}\sum \delta_r^2 q_r^2$\\
subjected to the above constraint (\ref{4eqn26}).\\
The Lagrange's equation for a connected holonomic system is given by
\begin{equation}
\frac{d}{dt}\left( \frac{\partial T}{\partial \dot{q_r}}\right) - \frac{\partial T}{\partial q_r} =-\frac{\partial V}{\partial q_r}+\lambda \frac{\partial F}{\partial q_r},~~~r =1,2,....n  ~,\nonumber
\end{equation}
%
where $\lambda$ is a function of time and \\
$F(q_1,q_2,...q_n)=A_1q_1+A_2q_2+...+A_nq_n$, So, we have the equation of motion \\

\begin{equation}\label{4eqn28}
\ddot{q_r}=-\delta_r^2+\lambda A_r
\end{equation} 
%
Let us assume the solution 
\begin{eqnarray}
q_r=a_re^{i\delta t}\label{n13}\\
 \mbox{and}~~ ~ \lambda =\alpha e^{i\delta t}\label{n14}
\end{eqnarray}
where $\alpha$ is a constant. Substituting (\ref{n13}) and (\ref{n14}) in (\ref{4eqn28}),  we get

\begin{eqnarray}
(-\delta^2+\delta_r^2)q_r=\alpha A_r e^{i\delta t} \nonumber \\
or, ~(-\delta^2+\delta_r^2)a_r e^{i\delta t}=\alpha A_r e^{i\delta t} \nonumber \\
\therefore~ a_r=\frac{\alpha A_r}{(\delta^2-\delta_r^2)} ~~, r=1,2,..,n.
\end{eqnarray}

Therefore, from (\ref{4eqn26}) and (\ref{n13})
\begin{eqnarray}
A_1a_1e^{i\delta t}+A_2a_2e^{i\delta t}+...	+A_na_ne^{i\delta t}=0 \nonumber \\
i.e.,~ A_1a_1+A_2a_2+...+A_na_n=0 (\because e^{i\delta t} \neq 0) \nonumber \\
or, ~ \frac{A_1^2}{(\delta^2-\delta_1^2)}+\frac{A_2^2}{(\delta^2-\delta_2^2)}+...+\frac{A_n^2}{(\delta^2-\delta_n^2)}=0\nonumber \\
X(\delta^2) \equiv  A_1^2(\delta^2-\delta_2^2)...(\delta^2-\delta_n^2)+A_2^2(\delta^2-\delta_1^2)(\delta^2-\delta_3^2)...(\delta^2-\delta_n^2)
\nonumber \\
+ A_n^2(\delta^2-\delta_1^2)(\delta^2-\delta_2^2)...(\delta^2-\delta_{n-1}^2)\nonumber
\end{eqnarray}


This is the frequency equation of the constrain oscillation and the freq. eq. is of the degree $(n-1)$.

Let, $\delta_1>\delta_2>\delta_3>....>\delta_n$, then $ X(\delta_1^2)>0, X(\delta_2^2)<0, X(\delta_3^2)>0...   $ and so on. So, there is atleast one root of $ X(\delta^2)=0$ between $(\delta_1^2, \delta_2^2), (\delta_2^2, \delta_3^2), ...(\delta_{(n-1)}^2,\delta_n^2).$  Thus the root of the freq. eq. of the constrain motion are separated by the roots of the equation of the original motion and conversely. Consequently, the periods of the oscillation of the constrained motion are separated by the periods of the oscillation of the original motion and conversely.

In small oscillation about  a position of stable equilibrium let $q_1,q_2,...q_n$ be the normal coordinates of the holonomic dynamical system. Then we can write the K.E. and P.E. as
\begin{eqnarray}
T&=&\frac{1}{2}\left[\dot{q_1}^2+...+\dot{q_n}^2\right] \nonumber \\
V&=&V_0+\frac{1}{2}\left[\sigma_1^2q_1^2+...+\sigma_n^2q_n^2\right] \nonumber
\end{eqnarray}
Let us introduce $n-1$ constrains in the system which do not work and consistent with the position equilibrium. The eqs. of the constrains can therefore be represented by $(n-1)$ homogeneous linear eqs. in $q_1,q_2,...q_n$.

\begin{eqnarray}
A_{11}q_1+A_{12}q_2+...+A_{1n}q_n=0\nonumber \\
A_{21}q_1+A_{22}q_2+...+A_{2n}q_n=0 \nonumber \\
-------------\nonumber \\
------------- \nonumber \\
A_{(n-1)1}q_1+A_{(n-1)2}q_2+...+A_{(n-1)n}q_n=0\nonumber
\end{eqnarray}
Solving these we get 
$\dfrac{q_1}{B_1}=\dfrac{q_2}{B_2}=...=\dfrac{q_n}{B_n}=q~(say)$ i.e., $q_1=B_1q,q_2=B_2q,...q_n=B_nq.$
Here $B_1,...B_n$ are functions of $A_{11},A_{12},...,A_{(n-1)n}$ and are minors of the elements of a row of a determinant from the above coefficients. The constrain system have therefore only 1 d.f.  and it can be expressed by the single coordinate $q$. Thus in the constrain system

$T=\dfrac{1}{2}(\sum B_r^2)\dot{q}^2~,V=V_0+\dfrac{q^2}{2}(\sum B_r^2\sigma_r^2)$. 
Then the equation of motion of constrain system is 
\begin{eqnarray}
\frac{d}{dt}\left( \frac{\partial T}{\partial \dot{q}}\right) - \frac{\partial T}{\partial q} =-\frac{\partial V}{\partial q}\nonumber \\
\mbox{or,}~~\left(\sum B_r^2\right)\ddot{q}+q\sum B_r^2\sigma_r^2=0\nonumber \\
\mbox{or,}~~\ddot{q}+\frac{\sum B_r^2\sigma_r^2}{\sum B_r^2}q=0\nonumber
\end{eqnarray}

Now , if $\dfrac{2\pi}{\sigma}$ be the period of the  constrain motion then we get from the  above expression $\sigma^2=\dfrac{\sum B_r^2\sigma_r^2}{\sum B_r^2}$, which shows that $\sigma^2$ is the wt.a.m. of $\sigma_r^2$ with wt. $B_r^2, ~~ r=1,2,...,n$ and hence it lies between the greatest and the least of the nos $\sigma_1^2, \sigma_2^2,...\sigma_n^2$. Thus the period of the constrain motion $\dfrac{2\pi}{\sigma}$ lies between the greatest and the least of the periods of the oscillation $\dfrac{2\pi}{\sigma_r}~~(r=1,2,...,n)$ of the original motion.

If the K.E. and P.E. of a vibrating system are given by $T=\dfrac{1}{2}\sum\sum a_{rs}\dot{q_r}\dot{q_s}, V=\dfrac{1}{2}\sum\sum b_{rs} q_r q_s $, where $a_{rs}=a_{sr}$ and $b_{rs}=b_{sr}$, then it is always possible  to obtain a linear transformation of $q's$ to the new system of $Q's$ such that  

$T=\dfrac{1}{2}\sum \dot{Q_k}^2, V=\dfrac{1}{2}\sum \sigma_k^2 Q_k^2  $ \\
For the normal mode of oscillation of the system \\
\begin{equation}\label{4eqn35}
q_r=\sum_s \alpha_{rs}c_s cos(\sigma_s t+\epsilon_s)~, 
\end{equation}

where $\alpha_{rs}$ are the minors of any of the rows of the harmonic determinant. Let

\begin{eqnarray}\label{4eqn36}
q_r'= c_{s}cos(\sigma_s t+\epsilon_s) \nonumber \\
\therefore q_r=\sum_k \alpha_{rk}q_{k}' ~~, ~~ q_s=\sum_{l=1} ^n\alpha_{sl}q_{l}' 
\end{eqnarray}

substituting the values of $q_r$ and $q_s$ in the expression of K.E. in the form

\begin{eqnarray}
2T=\sum_{	
	r=1}^{n}\sum_{s=1}^{n} a_{rs}\sum_{k=1}^{n} \alpha_{rk}\dot{q_k}'\sum_{l=1}^{n} \alpha_{sl}\dot{q_l}' \nonumber \\
=\sum_{k=1}^{n}\sum_{l=1}^{n}\left[\sum_{	r=1}^{n}\sum_{s=1}^{n} a_{rs} \alpha_{rk} \alpha_{sl}\right]\dot{q_k}'\dot{q_l}' \nonumber
\end{eqnarray}
Introducing the notation
\begin{equation}\label{4eqn37}
2T(\alpha_k,\alpha_l)=\sum_{	r=1}^{n}\sum_{s=1}^{n}a_{rs} \alpha_{rk} \alpha_{sl}
\end{equation}

\begin{equation}\label{4eqn38}
2V(\alpha_k,\alpha_l)=\sum_{	r=1}^{n}\sum_{s=1}^{n}b_{rs} \alpha_{rk} \alpha_{sl}
\end{equation}

We can  write 
\begin{equation}\label{4eqn39}
2T=\sum_k\sum_l 2T(\alpha_k,\alpha_l)\dot{q_k}'\dot{q_l}' 
\end{equation}
\begin{equation}\label{4eqn40}
2V=\sum_k\sum_l 2V(\alpha_k,\alpha_l)\dot{q_k}'\dot{q_l}' 
\end{equation}

Now we shall prove  
\begin{eqnarray} \label{4eqn41}
T(\alpha_k,\alpha_l)=0 ~~\text{when}  ~~k\neq l  
\end{eqnarray}

\begin{eqnarray} \label{4eqn42}
\text{and}~~  \sigma_k^2 T(\alpha_k,\alpha_l)=V(\alpha_k,\alpha_l)
\end{eqnarray}
From  equation (\ref{4eqn39}) and  equation (\ref{4eqn41})

\begin{eqnarray}\label{4eqn43}
2T=\sum_k 2T(\alpha_k,\alpha_l)\dot{q_k}'^2 =\sum \dot{Q}^2_k \nonumber \\
\text{where},~~\dot{Q}^2_k= 2T(\alpha_k,\alpha_l)\dot{q_k}'^2 ~~i.e.,~~ \dot{Q}_k= \sqrt{2T(\alpha_k,\alpha_l)}\dot{q_k}'
\end{eqnarray}

From equation (\ref{4eqn40}) and equation  (\ref{4eqn42}), when $l=k$,

\begin{eqnarray}
2V=\sum_k \sigma_k^2 2T(\alpha_k,\alpha_k) q_k^2=\sum_k \sigma_k^2Q_k^2 \nonumber \\
\text{where},~~Q_k=\sqrt{2T(\alpha_k,\alpha_k)},~~q_k'=c_k\sqrt{2T(\alpha_k,\alpha_k)} cos(\sigma_k t+ \epsilon _k)
\end{eqnarray}

\textbf{{Proof of (\ref{4eqn41}) and (\ref{4eqn42}):}} \\

\vspace{.25cm}

From $L's$ equation of motion \\
$\sum_s \left(a_{rs}\frac{d^2}{dt^2}+b_{rs}\right)q_s=0~,~,r=1,2,...,n.$\\
put $q_s=\sum_{	k}\alpha_{sk}q'_{k}$, then we get \\

$\sum_s \left[   - a_{rs}\sigma^2_k +b_{rs}   \right]\alpha_{sk}q'_{k}=0,$   where   $q'_k=c_k cos(\sigma_k t+ \epsilon _k) \neq 0$

\begin{equation}\label{4eqn45}
\therefore\sum_s \left[   - a_{rs}\sigma^2_k +b_{rs}   \right]\alpha_{sk}=0
\end{equation}

Now multiplying  (\ref{4eqn45}) by $\alpha_{rl}$ for the $l^{th}$ mode and adding for $r$ from $r=1~to ~n$ and we get

\begin{eqnarray}
\sum_r  \alpha_{rl} \sum_s\left[   - a_{rs}\sigma^2_k +b_{rs}   \right]\alpha_{sk}=0,\nonumber \\
or, ~  -\left[\sum_r \sum_s   a_{rs}  \alpha_{rl} \alpha_{sk}\right]\sigma_k^2 +\sum_r\sum_s b_{rs}\alpha_{rl} \alpha_{sk}=0 \nonumber \\
\mbox{or,}~~-2T(\alpha_k,\alpha_l) \sigma_k^2+V(\alpha_l,\alpha_k)=0 ~\text{(by (\ref{4eqn37}) and(\ref{4eqn38}) )}
\end{eqnarray}

From symmetry,
$-2T(\alpha_l,\alpha_k) \sigma_k^2+V(\alpha_k,\alpha_l)=0$\\
Interchanging $k$ and $l$ we get similarly, \\ 
$-2T(\alpha_k,\alpha_l) \sigma_l^2+2V(\alpha_k,\alpha_l)=0$

 Subtracting these two, $(\sigma_k^2-\sigma_l^2)2T(\alpha_k,\alpha_l) =0$, But $(\sigma_k^2\neq \sigma_l^2)$,
$\therefore 2T(\alpha_k,\alpha_l) =0,~for~k\neq l$\\
If $k=l$, $-2T(\alpha_k,\alpha_k) \sigma_k^2+2V(\alpha_k,\alpha_k)=0$\\
$\implies V(\alpha_k,\alpha_k)=T(\alpha_k,\alpha_k) \sigma_k^2$\\

\vspace{.5cm}

$\bullet$ {\textbf{Problem}} A light string OAB is tied to a fixed pt. at $O$ and carries a mass $2m$ at A and  a mass $m$ at B. The lengths OA and AB are $ l/2$  and $ 3l/4$   respectively. The string is free to move in a vertical plane and the system oscillates about the position of equilibrium. The inclination of OA and AB to the vertical are denoted by $\theta $ and $\phi$ respectively. Find the normal coordinates. The system is held with a string straight and inclined  a small angle $\alpha$ to the vertical and is let go from rest from this position at the instant $t=0,$ show that at any subsequent time $\theta =\dfrac{\alpha}{3}(2 cos ~nt +cos ~2nt), \phi=\dfrac{\alpha}{3}(4 cos ~nt - cos ~2nt)$ where $n=\sqrt{\dfrac{g}{l}}$ .\\

\vspace{.25cm}

{\textbf{Solution}} 

$T=\dfrac{1}{2}.2m \left(\dfrac{1}{2}l\dot{\theta}\right)^2+\dfrac{1}{2}m\left[ \dfrac{1}{2}l\dot{\theta}+\dfrac{3}{4}l\dot{\phi}\right]^2 $ \\
$V=V_0-3 mg \dfrac{l}{2}(1-\theta^2/2)-mg \dfrac{3l}{4}(1-\phi^2/2)$\\



\begin{figure}[h!]
	\centering
	\includegraphics[scale=0.33]{chapter4-3rd.pdf}
\end{figure}

So The $L's$ equations are

\begin{eqnarray}
\frac{d}{dt}\left( \frac{\partial T}{\partial \dot{\theta}}\right) - \frac{\partial T}{\partial \theta} +\frac{\partial V}{\partial \theta} =0\nonumber \\
\frac{d}{dt}\left( \frac{\partial T}{\partial \dot{\phi}}\right) - \frac{\partial T}{\partial \phi} +\frac{\partial V}{\partial \phi} =0 \nonumber \\
\therefore \frac{d}{dt}\left[  2m   (\frac{1}{2}l)^2 \dot{\theta} +m(\frac{1}{2}l \dot{\theta}+\frac{3}{4}l \dot{\phi}) \frac{l}{2}\right]+\frac{3}{2}mg l \theta=0 \nonumber \\
or,~\frac{1}{2}l \ddot{\theta}+\frac{1}{4}l \ddot{\theta} +\frac{3}{8}l \ddot{\phi}+\frac{3}{2}g \theta=0\nonumber \\
or,~	\frac{3}{4}l \ddot{\theta} +\frac{3}{8}l \ddot{\phi}+\frac{3}{2}g \theta=0 \nonumber \\
\Rightarrow	\frac{1}{2}l \ddot{\theta} +\frac{1}{4}l \ddot{\phi}+g \theta=0
\end{eqnarray}
%
\begin{eqnarray}
\frac{d}{dt}\left[    m\frac{3}{4}l\left(\frac{1}{2}l\dot{\theta}+\frac{3}{4}l\dot{\phi}  \right)  \right] +mg \frac{3}{4}l  \phi=0 \nonumber \\
0r,~\frac{l\ddot{\theta}}{2} +\frac{3}{4} l\ddot{\phi} +g\phi=0
\end{eqnarray}

Let, $\theta=A_1e^{\lambda t}$, $\phi=A_2e^{\lambda t}$, as a solutions,
\begin{eqnarray}
\frac{1}{2}l\lambda^2A_1+ \frac{1}{4}l\lambda^2A_2 +gA_1=0\nonumber \\
or,~\left(\frac{1}{2}l\lambda^2 +g\right)A_1 +\frac{1}{4}l\lambda^2A_2=0  \label{4eq53} \\
\text{similarly},~~  \frac{1}{2} l \lambda^2 A_1 + \left( \frac{3}{4}l\lambda^2 +g \right)A_2=0
\end{eqnarray}
Eliminating $A_1 $ and $ A_2$
\begin{eqnarray}
\begin{bmatrix} 
\left(\frac{1}{2}l\lambda^2 +g\right) & \frac{1}{4}l\lambda^2 \\
\frac{1}{2} l \lambda^2  &  \left( \frac{3}{4}l\lambda^2 +g \right) \\
\end{bmatrix} =0 \nonumber \\
or,~ \frac{3}{8}l^2 \lambda^4 +\frac{3}{4}gl \lambda^2  +\frac{1}{2}lg \lambda^2 +g^2 - \frac{1}{8}l^2\lambda^4=0 \nonumber \\
or,~ \frac{1}{4}l^2\lambda^4+\frac{5}{4}gl \lambda^2+g^2=0 \nonumber \\
or,~\lambda^4+5 n^2\lambda^2+4 n^4=0 \nonumber \\
\Rightarrow  \lambda^2=-4n^2,-n^2  \nonumber \\
i.e., ~\sigma^2=4n^2,n^2 
\end{eqnarray}

So, the periods are $\dfrac{2\pi}{2n}, \dfrac{2\pi}{n}.$

 Equation (\ref{4eq53}) on simplification gives \\
$(2 \lambda^2+4n^2) A+\lambda^2 B=0$ \\
Putting,  $\lambda^2=-n^2$, $2A=B$  $\Rightarrow \frac{A}{1}=\frac{B}{2}$ \\
Putting,  $\lambda^2=-4n^2$, $A+B=0$  $\Rightarrow \frac{A}{1}=\frac{B}{-1}$ \\

$\therefore~ \theta=ccos(nt+\epsilon)+c'cos(2nt+\epsilon ')$\\
$~~~~~~~\phi=2ccos(nt+\epsilon)-c'cos(2nt+\epsilon ')$\\
Initially, when $t=0,$ $\theta=\phi=\alpha$  and  $\dot{\theta}=\dot{\phi}=0$

$\therefore~ \alpha=ccos\epsilon +c'cos\epsilon '$\\
$ \alpha=2c ~cos\epsilon - c'cos\epsilon '$ \\
$\Rightarrow 2 \alpha=3 c ~cos \epsilon$ \\
$\dot{\theta}=0 ~at ~t=0 \Rightarrow 0=nc~ sin \epsilon +2nc'~sin\epsilon'$ \\
$\dot{\phi}=0~\mbox{at}~t=0\implies 0=2nc\sin\epsilon-2nc'\sin\epsilon'$\\
$\therefore  ~sin \epsilon =0=sin \epsilon' ~i.e. ~\epsilon=\epsilon'=0$ \\
$\therefore~ c=\dfrac{2\alpha}{3}, ~\therefore~ c'=\alpha-\dfrac{2\alpha}{3}=\dfrac{\alpha}{3}$ \\
$\theta=\dfrac{2}{3}\alpha cos nt + \dfrac{1}{3}\alpha cos 2nt=\dfrac{1}{3}\alpha \left(  2 cos nt +cos 2nt \right) $\\
$\phi=\dfrac{1}{3}\alpha \left(   cos nt - cos 2nt \right)$\\
If we take $\xi $ and $\eta$ the normal co-ordinates then the eqs. of motion will be of the form \\
$\ddot{\xi}=-n^2 \xi, \ddot{\eta}=-4n^2\eta$  \\
The periods of these two oscillations will correspond to $\lambda^2=-n^2$ and ${\lambda}^2=-4n^2$. Even when expressed in terms of the normal co-ordinates. Thus the expressions for the normal co-ordinates will contain terms of the type $cos (nt+\epsilon)$  and $cos (2nt+\epsilon')$ constraint with the given system. Adding we get $\theta+\phi=3c~ cos(nt+\epsilon)=\xi$ and $2\theta-\phi=3c'cos(2nt+\epsilon')=\eta$

\vspace{.5cm}

$\bullet$ {\textbf{Problem:}} 
A circular  arc of  radius $a$ is fixed in a vertical plane and a uniform circular disc of mass $M$ and radius $a/4$ is placed inside so as to roll on the arc. When the disc is in the position of equilibrium, a particle of mass $M/3$ is fixed to it in the vertical diameter through the center at a distance $a/6$ from the centre. Show that the time of  small oscillation about the position of equilibrium is  $\dfrac{\pi}{6} \sqrt{\dfrac{83 a}{g}}$.

\vspace{.25cm}

{\textbf{Solution:}}  For rolling $\dfrac{a}{4}(\theta+\psi)=a\psi . ~i.e. ~ \psi=\theta/3$\\
\begin{eqnarray}
V&=&V_0 -Mg\frac{3}{4} a cos \psi - \frac{Mg}{3}\left(   \frac{3}{4}a cos \psi +\frac{a}{6} cos \theta \right) \nonumber \\
&=&V_0-Mga cos \psi -\frac{Mga }{18} cos \theta \nonumber \\
&=&V_0-Mga (1-\frac{\psi^2}{2})-\frac{Mga }{18}(1-\frac{\theta^2}{2}) \nonumber  \\
&=&V_0-Mga (1-\frac{\psi^2}{2})-\frac{Mga }{18}(1-\frac{9 \psi^2}{2}) \nonumber \\
&=&V_0-\frac{19 Mga }{18} +\frac{3Mga }{4} \psi^2 \nonumber
\end{eqnarray}

\vspace{0.7 cm}


\begin{figure}[h!]
	\centering
	\includegraphics[scale=0.33]{chapter4-4th.pdf}
\end{figure}

\vspace{0.5 cm}

K.E. of the plate =$\dfrac{1}{2 }M(\dfrac{3a}{4})^2\dot{\psi}^2 +\dfrac{1}{2}M\dfrac{a^2}{3^2}\dot{\theta}^2\\
~~~~~~~~~~~~~~~~~~~~=\dfrac{1}{2}M\dfrac{27}{32}a^2 \dot{\psi}^2$\\
Velocity of the particle $=\dfrac{3}{4}a \dot{\psi}$ along $||$ to the tangent to the arc upward $+\dfrac{1}{6}a\dot{\theta}$  along $||$ to the tangent to the plate downward.\\
$=\dfrac{3}{4}a \dot{\psi}-\dfrac{1}{6}a3\dot{\psi}=\dfrac{a}{4}\dot{\psi}$ (Neglecting the small angle between the two tangents.)\\
$\therefore $ K.E. of the particle $=\dfrac{1}{2}\dfrac{M}{3}\dfrac{a^2 \dot{\psi}^2}{16}$\\
So, the total K.E. $=\dfrac{1}{2}Ma^2\dot{\phi}\left(\dfrac{27}{32}+\dfrac{1}{48}\right)=\dfrac{1}{2}Ma^2\dot{\phi}\left(\dfrac{83}{96}\right)$\\
Thus the L's eq. of motion \\
$	\dfrac{d}{dt}\left( \dfrac{\partial T}{\partial \dot{\psi}}\right) - \dfrac{\partial T}{\partial \psi} +\dfrac{\partial V}{\partial \psi} =0 \Rightarrow Ma^2\dfrac{83}{96}\ddot{\psi}+\dfrac{3}{2}Mga\psi=0$ \\
$\therefore $ Time period $=\dfrac{2 \pi}{\sqrt{144 g/a}}=\dfrac{2\pi }{12}\sqrt{\dfrac{83a}{g}}=\dfrac{\pi}{6}\sqrt{\dfrac{83a}{g}}$

\vspace{.5cm}

$\bullet$ {\textbf{Problem:}}  Two heavy uniform rods AB and AC, each of mass $m$ and length $2a$ are hinged at A and placed symmetrically over a smooth cylinder of radius c whose axis is horizontal. If they are slightly and symmetrically displaced from the position of the equilibrium, show that the time of small oscillation is $2\pi \sqrt{\frac{a sin \alpha}{3g}\frac{(1+3sin^2\alpha)}{(1+2sin^2\alpha)}}$, where $a cos^3\alpha=c sin \alpha$.

{\textbf{Solution:}} AB,AC are two rods hinged at A and they are placed symmetrically over a circular cylinder whose axis is horizontal. Let $\theta$ be the angle which the rods make with the vertical through $O$. We take the centre of the cylinder as origin and x-axis horizontal in the plane of the rods and y-axis vertically downwards.

Let G be the C.G of the rod, so that  $AG=a$. If $(x,y)$ be the co-ordinates of G, then $x=a sin\theta,y=a cos\theta -c ~cosec\theta$. As the potential energy of the system is \\
$V=-2mgy+A$  (A is a constant)
so for the equilibrium position $\frac{dV}{d \theta}=0 \Rightarrow ~\frac{dy}{d \theta}=0  \Rightarrow ~-a sin \theta +c cosec \theta cot \theta=0, \therefore c cos \theta=a sin^3 \theta$.\\

\begin{figure}[h!]
	\centering
	\includegraphics[scale=0.33]{chapter4-5th.pdf}
\end{figure}

If $\theta=\alpha$ be the position of the equilibrium, then 
\begin{equation}
c~cos \alpha= a sin^3 \alpha\label{n33}
\end{equation}
Let us put, $\theta=\alpha+\xi$, where $\xi$ is very small. Now expanding $y$ near about the equilibrium position 

\begin{eqnarray}
y&=&y|_{\theta=\alpha}+\frac{\partial y}{\partial \theta}_{ | \theta=\alpha}\xi+{\frac{\partial  ^2 y}{\partial \theta ^2}}_{ | \theta=\alpha} \frac{\xi^2}{2!}+...\nonumber \\
\text{As} ~\frac{d^2 y}{d \theta ^2}&=&-a cos \theta -c cosec \theta~ cot^2 \theta -c cosec^3 \theta \nonumber \\
\therefore   \frac{d^2 y}{d \theta ^2}_{ | \theta=\alpha}&=&-a cos \alpha-c \frac{cos^2 \alpha}{sin^3 \alpha}- \frac{c}{sin^3 \alpha} \nonumber \\
&=&\left[ -a cos \alpha   sin^3 \alpha -c(1+cos^2\alpha) \right]/sin^3 \alpha\nonumber \\
&=&-\frac{a}{cos\alpha}(1+2cos^2\alpha) \nonumber \\
\therefore V&=& 2mg \frac{a}{cos\alpha}(1+2cos^2\alpha) \frac{\xi^2}{2} +A~\text{(Neglecting higher power of $\xi$)} \nonumber
\end{eqnarray}

As , $x=a sin \theta \Rightarrow \dot{x}=a cos \theta \dot{\theta}=acos(\alpha+\xi)\dot{\xi}=acos\alpha \dot{\xi}$ (neglecting 2nd order small quantities) and $\dot{y}=\frac{d y}{d \theta}\dot{\theta}=0$.

$\therefore$ K.E. of the system $T=\frac{1}{2}2m \left[   a^2 cos^2\alpha \dot{\xi}^2 +\frac{a^2}{3}\dot{\xi}^2  \right]$\\

The L's equation is  

\begin{eqnarray}
\frac{d}{dt}\left( \frac{\partial T}{\partial \dot{\xi}}\right) - \frac{\partial T}{\partial \xi} +\frac{\partial V}{\partial \xi} =0 \nonumber \\
\Rightarrow 2a^2 m(cos^2\alpha \ddot{\xi}+\frac{1}{3}\ddot{\xi}) +2mg a \frac{(1+2cos^2 \alpha)}{cos \alpha} \xi=0 \nonumber \\
\Rightarrow \ddot{\xi} =-\frac{g(1+2 cos^2 \alpha)}{a cos \alpha (\frac{1}{3}+cos^2 \alpha)}\xi \nonumber \\
T=2\pi \sqrt{\frac{a cos \alpha}{3g }\frac{(1+3 cos^2\alpha)}{(1+2cos^2 \alpha)}}\nonumber
\end{eqnarray}
%
$\bullet$ {\textbf{Problem IV:}} (small oscillation)
A rhombus formed of four equal rods freely joint is placed over a fixed smooth sphere in a vertical plane so that only the upper pair is in  contact with the sphere. Show that the time of symmetrical oscillation in the vertical plane is in contact with the sphere. Show that the time of symmetrical oscillation in the vertical plane is 
$2\pi\sqrt{\frac{2a cos \alpha}{3g (1+2cos^2\alpha)}}$ where $2 a$ is the length of each rod and $\alpha$ is the angle it makes with the vertical in the position of the equilibrium.

$\bullet$ {\textbf{Solution:}}  We choose the center of the sphere $O$ as origin and $x$ and $y$ axes as shown in the figure. The depth of C.G.  G below O is
\begin{eqnarray}
y=2a cos \theta -c cosec \theta \nonumber \\
\therefore \frac{dy}{d\theta}=-2a sin \theta +c cosec \theta cot \theta \nonumber \\
\therefore \frac{dy}{d\theta}=0~ at ~\theta=\alpha  ~\Rightarrow \frac{2a}{cos \alpha}=\frac{c}{sin^3 \alpha} \label{n34} \\
\therefore \theta=\alpha+\xi  ~~\text{(where $\xi$ is very small)}\nonumber \\
\text{K.E.}=\frac{1}{2}4m \frac{4a^2}{3}\dot{\xi}^2 \nonumber 
\end{eqnarray}


\begin{figure}[h!]
	\centering
	\includegraphics[scale=0.4]{chapter4-6th.pdf}
\end{figure}

\begin{eqnarray}
V&=&-4mg y \nonumber \\
&=& -4mg \left[    y|_{\theta=\alpha}  +{\frac{dy}{d \theta}}|_{\theta=\alpha} \xi  +{\frac{d^2y}{d \theta^2}}|_{\theta=\alpha} \frac{\xi ^2}{2}+...\right]\label{n35}  \\
&=&-4mg \left[   (2a cos \alpha -\frac{c}{sin \alpha} ) +0+(-2a cos \theta-c cosec \theta cot^2 \theta -c cosec^3 \theta)_{|\theta=\alpha} \frac{\xi ^2}{2}  \right]  \nonumber \\
&=& -4mg \left[   (2a cos \alpha -\frac{c}{sin \alpha} ) - \frac{2a}{cos \alpha}(1+2 cos^2 \alpha)\frac{\xi ^2}{2} \right] \nonumber
\end{eqnarray}

So the Lagranges equation becomes \\

$\frac{d}{dt}\left( \frac{\partial T}{\partial \dot{\xi}}\right) - \frac{\partial T}{\partial \xi} +\frac{\partial V}{\partial \xi} =0$ \\
$\mbox{or}, ~ 4 \frac{4m}{3}a^2 \ddot{\xi}+4mg \frac{2a}{cos \alpha}(1+2cos^2\alpha)\xi=0$ \\
$\therefore T=2\pi\sqrt{\frac{2a cos \alpha}{3g (1+2cos^2\alpha)}}$

\vspace{.5cm}

$\bullet$ {\textbf{Problem V:}}  Two uniform rods of same mass and of same length $2a$ are freely jointed at a common extremity and rest upon two smooth pegs which are in the same horizontal plane so that  each rod is inclined at same angle $\alpha$ to the vertical. Show that the time of small oscillation when the join moves in a vertical  st. line through the centre of the line joining the pegs is $2\pi\sqrt{\frac{a}{9g}(\frac{1+3 cos^2 \alpha}{cos \alpha})}$.

\vspace{.25cm}

{\textbf{Solution:}}  Let $2c $ be the horizontal distance between the pegs. Take the mid-pt. of the line joining the pegs as the fixed origin and x-axis is along horizontal direction and y-axis is vertically downwards.

\begin{eqnarray}
\therefore y&=&a cos \theta -c cot \theta   \nonumber  \\
\frac{d y}{d \theta}&=&-a sin \theta +c cosec^2\theta \nonumber \\
\frac{d^2 y}{d \theta^2}&=&-a cos \theta -2c cosec^2\theta cot \theta\nonumber
\end{eqnarray}

At equilibrium position $\theta=\alpha,~  \frac{d y}{d \theta}=0$\\
$\therefore ~ a sin^3\alpha=c$\\

\begin{figure}[h!]
	\centering
	\includegraphics[scale=0.4]{chapter4-7th.pdf}
\end{figure}

Let us put $\theta=\alpha+\xi$.

\begin{eqnarray}
\therefore~ y&=&\left[    y|_{\theta=\alpha}  +\frac{dy}{d \theta}_{|\theta=\alpha} \xi  +\frac{d^2y}{d \theta^2}_{|\theta=\alpha} \frac{\xi ^2}{2} +... \right] \nonumber \\
&=&A+\left[-a cos \alpha -2 c~cosec^2 cot \alpha \right]\frac{\xi ^2}{2} \nonumber \\
&=&A-\frac{3}{2}\xi^2 a~cos \alpha\nonumber
\end{eqnarray}


$\therefore V=V_0 +3 mg a ~cos \alpha ~\xi^2$,\\
 $x=a sin \theta,~~ \dfrac{dx}{dt}=a cos \theta \dot{\theta}=a cos \alpha ~\dot{\xi}$ \\
$\therefore T=\frac{1}{2}2 m \left[a^2 cos^2 \alpha \dot{\xi}^2  + \frac{1}{3} a^2  \dot{\xi}^2 \right]$  \\
$=\frac{ma^2}{3}(1+3 cos^2 \alpha)\dot{\xi}^2$


So, by L's eqs. we have 

\begin{eqnarray}
\frac{d}{dt}\left( \frac{\partial T}{\partial \dot{\xi}}\right) - \frac{\partial T}{\partial \xi} +\frac{\partial V}{\partial \xi} =0  \nonumber \\
or, ~~\frac{2ma^2}{3}(1+3 cos^2 \alpha)\ddot{\xi}+3mg ~2a~cos \alpha ~\xi=0  \nonumber \\
\therefore\ddot{\xi}=-\frac{9g cos \alpha}{a(1+3 cos^2 \alpha)}\xi    \nonumber \\
\therefore~ ~T=2\pi \sqrt{\frac{a(1+3 cos^2\alpha)}{9g ~cos \alpha}} \nonumber
\end{eqnarray}



$\bullet$ {\textbf{Problem VI:}} 
A uniform beam rests with one end on a smooth horizontal plane and the other end is supported by a string of length $l$ which is attached to a fixed pt. show that the time of small oscillation in the vertical plane is $2\pi\sqrt{\frac{2l}{g}}$.


{\textbf{Solution:}}
From the figure
\begin{eqnarray}
x&=&l sin \theta +a cos(\phi_0+\phi) \label{n36}\\
y&=&a sin(\phi_0+\phi)\label{n37} \\
h&=&CO=l cos \theta +2a sin(\phi_0+\phi) \label{n38}
\end{eqnarray} 


\begin{figure}[h!]
	\centering
	\includegraphics[scale=0.4]{chapter4-8th.pdf}
\end{figure}

From equation (\ref{n38}), $O=dh=-lsin \theta d\theta + 2 a cos(\phi+\phi_0) d\phi$ \\
$\therefore 2 a ~cos\phi_0 d\phi=l \theta~ d\theta$ (for small $\theta $ and $\phi$)\\
which shows that $d\phi$ is a small quantity of 2nd order. 
\begin{eqnarray}
\dot{x}&=&l cos \theta \dot{\theta} - a sin(\phi_0+\phi) \dot{\phi} \approx l \dot{\theta} ~(\text{up to 1st order} )\nonumber \\
\dot{y}&=&a cos(\phi+\phi_0)  \dot{\phi} \approx 0~ (\text{up to 1st order})\nonumber \\
\therefore \text{P.E.} ~~ V&=&mgy=\frac{mg}{2}(h-l cos \theta) =\frac{mg}{2}\left(h-l(1-\frac{\theta^2}{2})\right) \nonumber \\
&=& V_0 +\frac{mgl}{4} \theta^2  \nonumber \\
\text{K.E.} ~~(T)&=&\frac{1}{2}m(\dot{x}^2+\dot{y}^2) + \frac{1}{2}m \frac{a^2}{3}\dot{\phi}^2 \nonumber \\
&=&\frac{1}{2}ml^2 \dot{\theta}^2 ~~(\text{up to 2nd  order})\nonumber
\end{eqnarray}

So by L's equation\\

$\frac{d}{dt}\left( \frac{\partial T}{\partial \dot{\theta}}\right) - \frac{\partial T}{\partial \theta} +\frac{\partial V}{\partial \theta} =0$\\
$\Rightarrow ml^2 \ddot{\theta} +\frac{mgl}{2}\theta=0~\Rightarrow T=2\pi\sqrt{\frac{2l}{g}}$


$\bullet$ {\textbf{Problem VII:}} 
Two equal uniform rods AB and BC each of length $l$ smoothly joined together at B are suspended from A and oscillates in a vertical plane through A. Show that the periods of normal oscillation are  $\frac{2\pi}{n},n^2=(3  \pm \frac{6}{\sqrt{7}})g/l$.

{\textbf{Solution:}}

\begin{figure}[h!]
	\centering
	\includegraphics[scale=0.4]{chapter4-9th.pdf}
\end{figure}

\begin{eqnarray}
x_1 &=& \frac{l}{2}cos \theta =  \frac{l}{2}(1-\theta^2/2)\nonumber \\
y_1 &=&  \frac{l}{2} sin\theta= \frac{l}{2} \theta   \nonumber \\
x_2 &=& l cos \theta + \frac{l}{2} cos \phi  \nonumber \\
&=&l(1-\theta^2/2) +\frac{l}{2}(1-\phi^2/2) \nonumber \\
y_2 &=& l sin \theta + \frac{l}{2} sin \phi=l\theta+\frac{l\phi}{2}  \nonumber \\
T&=&\frac{1}{2}m\left[\dot{x_1}^2+\dot{y_1}^2+\dot{x_2}^2+\dot{y_2}^2\right] +\frac{1}{2}m \left(\frac{l^2}{12} \dot{\theta}^2+\frac{l^2}{12} \dot{\phi}^2\right) \nonumber \\
&=&  \frac{m}{2}\left[\frac{l^2}{4} sin^2 \theta \dot{\theta}^2+\frac{l^2}{4} cos^2 \theta \dot{\theta}^2 +(lsin \theta \dot{\theta }+\frac{1}{2} sin \phi \dot{\phi})^2+\left(l\cos\theta\dot{\theta}+\frac{l}{2}\cos\phi\dot{\phi}\right)^2\right]+\frac{ml^2}{24}\left(\dot{\theta}^2+\dot{\phi}^2\right) \nonumber \\
&=& \frac{m}{2}\left[\frac{l^2}{4}  \dot{\theta}^2+{l^2} \dot{\theta}^2 +\frac{l^2}{4} \dot{\phi}^2+ l^2 \dot{\theta} \dot{\phi}
\right] + \frac{m l^2}{24}( \dot{\theta}^2 +  \dot{\phi}^2)\nonumber \\ 
&=& \frac{ml^2}{2}\left(\frac{16 \dot{\theta}^2 +4 \dot{\phi}^2 +12 \dot{\theta} \dot{\phi}}{12}\right)\nonumber \\
&=&\frac{1}{6}ml^2 (4\dot{\theta}^2+\dot{\phi}^2 + 3 \dot{\theta} \dot{\phi})\nonumber
\end{eqnarray} 

\begin{eqnarray}
V &=&-mg \frac{l}{2}\left(1-\frac{\dot{\theta}^2}{2}\right)-mg \left[ l (1-\frac{ \theta^2}{2}) +\frac{l}{2}\left(1-\frac{\phi^2}{2}\right)   \right] \nonumber \\
&=& V_0 + mgl \left( \frac{3 \theta^2}{4} +\frac{\phi^2}{4}  \right)\nonumber
\end{eqnarray}
So, the L's equations are 
\begin{eqnarray}\nonumber
\frac{d}{dt}\left( \frac{\partial T}{\partial \dot{\theta}}\right) - \frac{\partial T}{\partial \theta} +\frac{\partial V}{\partial \theta} =0  \end{eqnarray}

\begin{eqnarray}\label{4eqn65}
\frac{d}{dt}\left( \frac{\partial T}{\partial \dot{\phi}}\right) - \frac{\partial T}{\partial \phi} +\frac{\partial V}{\partial \phi} =0  
\end{eqnarray}

From Eqn. (\ref{4eqn65}), 
\begin{eqnarray}
\frac{1}{6}m l^2(8\ddot{\theta}+3 \ddot{\phi}) +\frac{3}{2} mgl \theta=0 \nonumber \\
\Rightarrow~~ 8\ddot{\theta}+3 \ddot{\phi} +\frac{9 g}{l} \theta =0
\end{eqnarray}


\begin{eqnarray}
\frac{1}{6}m l^2(2 \ddot{\phi} + 3\ddot{\theta}) +\frac{1}{2} mgl \phi =0 \nonumber \\
\Rightarrow~~ 2 \ddot{\phi} + 3\ddot{\theta}+ +\frac{3 g}{l} \phi =0 
\end{eqnarray}
Let $\theta=\theta_0e^{i\lambda t}$, $\phi=\phi_0 e^{i\lambda t}$

$\therefore (-8 \lambda^2 +\frac{9 g}{l} )\theta_0 - 3 \phi_0\lambda^2=0$ \\
$ \Rightarrow -3 \lambda^2 \theta_0 +(-2 \lambda^2 + \frac{3 g}{l})\phi_0=0$\\
Eliminating, $\theta_0$ and $\phi_0$, we have

\begin{eqnarray}
\begin{bmatrix} 
-8 \lambda^2 +\frac{9 g}{l} & - 3 \lambda^2 \\
- 3 \lambda^2  & - 2 \lambda^2 +\frac{3 g}{l}  \\
\end{bmatrix} &=&0 \nonumber \\
or, ~ 7\lambda^4 -42 \frac{42 g}{l}\lambda^2 +27 \frac{g^2}{l^2}&=&0 \nonumber \\
or, \lambda^2 = \frac{42 g/l \pm \sqrt{42^2 g^2/l^2 -4.7.27 g^2/l^2}}{2.7} \nonumber \\
=\frac{42 \frac{g}{l} \pm \sqrt{1008}\frac{g}{l}}{14}\nonumber
\end{eqnarray}
Hence the time periods $ =(3\pm \frac{6}{\sqrt{7}})g/l$.

\vspace{.5cm}

\section{Eulerian angles $\theta,~\phi$ and $\psi$: }

Consider a material system composed of $N$ particles. Let the material system rotates about a point O. The point O is fixed with respect to the material system as well as in the space. Suppose $ox,~oy,~oy$ be a rectangular coordinate system through O. These axes are fixed with respect to the material system. Let $\overrightarrow{i},~\overrightarrow{j},~\overrightarrow{k}$ be the unit vectors along the coordinate axes. Then the position of the rotating axes as well as the components of angular velocity with respect to the rotating axes can be expressed by the three angles $\theta,~\phi$ and $\psi$ and their derivatives. These angles are known as Eulerian angles. To define these angles we proceed as follows :

Since the end point of any unit vector having initial point O lies on the surface of a sphere of radius unity with centre at O. We restrict our discussion on the surface of this unit sphere (having center at O). Let $\overrightarrow{I},~\overrightarrow{J},~\overrightarrow{K}$ be the initial position of $\overrightarrow{i},~\overrightarrow{j}$ and $\overrightarrow{k}$ respectively. Let $\theta$ be the angle between $\overrightarrow{K}$ and $\overrightarrow{k}$. Let the grate circle through the terminal points of $\overrightarrow{K}$ and $\overrightarrow{k}$ meets the grate circle through the terminal points of $\overrightarrow{I}$ and $\overrightarrow{J}$ at the points A and B as shown in the figure. We denote the unit vector $\overrightarrow{OA}$ by $\overrightarrow{I}'$. Let $\phi$ be the angle between $\overrightarrow{I}$ and $\overrightarrow{I}'$. Let us consider the rotation of $\overrightarrow{I}$ and $\overrightarrow{J}$ through an angle $\phi$ in an anticlockwise sense in the plane of the grate circle through the terminal points of $\overrightarrow{I}$ and $\overrightarrow{J}$. Therefore, $\overrightarrow{I}'$ is the new position of $\overrightarrow{I}$. Therefore, $\overrightarrow{I}'$ is the new position of $\overrightarrow{I}$. Let $\overrightarrow{J}'$ be the new position of $\overrightarrow{J}$ and consequently, we get
\begin{equation}
	(\overrightarrow{I},~\overrightarrow{J},~\overrightarrow{K})\xrightarrow{\mbox{due to rotation}~\phi\overrightarrow{K}}(\overrightarrow{I}',~\overrightarrow{J}',~\overrightarrow{K})\nonumber
\end{equation} 
Next consider the rotation of $\overrightarrow{I}'$ and $\overrightarrow{K}$ through on angle $\theta$ in an anticlockwise sense in the plane of the grate circle through the terminal points of $\overrightarrow{I}'$ and $\overrightarrow{K}$. Therefore $\overrightarrow{I}''$ is the new position of $\overrightarrow{I}'$ and $\overrightarrow{k}$ be the new position of $\overrightarrow{K}$  and consequently we get  
\begin{equation}
	(\overrightarrow{I}',~\overrightarrow{J}',~\overrightarrow{K})\xrightarrow{\mbox{due to rotation}~\theta\overrightarrow{J}'}(\overrightarrow{I}'',~\overrightarrow{J}',~\overrightarrow{k})\nonumber
\end{equation} 
Finally, consider the rotation of $\overrightarrow{I}''$ and $\overrightarrow{J}'$ through on angle $\psi$ in an anticlockwise sense in the plane of the grate circle through the terminal points of $\overrightarrow{I}''$ and $\overrightarrow{J}'$. Therefore $\overrightarrow{i}$ is the new position of $\overrightarrow{I}'',~\overrightarrow{j}$ is the new position of $\overrightarrow{J}'$ and consequently we get
\begin{equation}
	(\overrightarrow{I}'',~\overrightarrow{J}',~\overrightarrow{k})\xrightarrow{\mbox{due to rotation}~\psi\overrightarrow{k}}(\overrightarrow{i},~\overrightarrow{j},~\overrightarrow{k})\nonumber
\end{equation} 
therefore, if $\overrightarrow{n}$ denotes the total rotation of $\overrightarrow{i},~\overrightarrow{j},~\overrightarrow{k}$ from the initial position $(\overrightarrow{I},~\overrightarrow{J},~\overrightarrow{K})$ then 
\begin{equation}
	\overrightarrow{n}=\phi\overrightarrow{K}+\theta\overrightarrow{J}'+\psi\overrightarrow{K}\nonumber
\end{equation}
So for infinitesimal rotation $\delta\overrightarrow{n}$ during the time interval $(t,t+\delta{t})$ we have 
\begin{equation}
	\delta\overrightarrow{n}=\delta\phi\overrightarrow{K}+\delta\theta\overrightarrow{J}'+\delta\psi\overrightarrow{K}\nonumber
\end{equation}
If $\overrightarrow{\omega}$ be the angular velocity of the material system about O then 
\begin{equation}
	\overrightarrow{\omega}=\lim_{\delta{t}\rightarrow{0}}\frac{\delta\overrightarrow{n}}{\delta{t}}=\dot{\phi}\overrightarrow{K}+\dot{\theta}\overrightarrow{J}'+\dot{\psi}\overrightarrow{K}\nonumber
\end{equation}
For the rotation : $\phi\overrightarrow{K}$ we have 
\begin{eqnarray}
	\overrightarrow{I}'=\overrightarrow{I}\cos\phi+\overrightarrow{J}\sin\phi,~\overrightarrow{J}'&=&\overrightarrow{I}\cos\left(\frac{\pi}{2}+\phi\right)+\overrightarrow{J}\sin\left(\frac{\pi}{2}+\phi\right)\nonumber\\
	&=&-\overrightarrow{I}\sin\phi+\overrightarrow{J}\cos\phi\nonumber
\end{eqnarray}
For the rotation : $\theta\overrightarrow{J}'$ we have 
\begin{eqnarray}
	\overrightarrow{k}=\overrightarrow{K}\cos\theta+\overrightarrow{I}'\sin\theta,~\overrightarrow{I}''&=&\overrightarrow{K}\cos\left(\frac{\pi}{2}+\theta\right)+\overrightarrow{I}'\sin\left(\frac{\pi}{2}+\theta\right)\nonumber\\
	&=&-\overrightarrow{K}\sin\theta+\overrightarrow{I}'\cos\theta\nonumber
\end{eqnarray}
For the rotation : $\psi\overrightarrow{k}$ we have 
\begin{eqnarray}
	\overrightarrow{i}=\overrightarrow{I}''\cos\psi+\overrightarrow{J}'\sin\psi,~\overrightarrow{j}&=&\overrightarrow{I}''\cos\left(\frac{\pi}{2}+\psi\right)+\overrightarrow{J}'\sin\left(\frac{\pi}{2}+\psi\right)\nonumber\\
	&=&-\overrightarrow{I}''\sin\psi+\overrightarrow{J}'\cos\psi\nonumber
\end{eqnarray}
Also we can write the above three rotations in the matrix form as 
\begin{eqnarray}
	\begin{pmatrix}
		\overrightarrow{I}'\\ \overrightarrow{J}'
	\end{pmatrix}
	&=&
	\begin{pmatrix}
		\cos\phi & \sin\phi\\ -\sin\phi & \cos\phi
	\end{pmatrix}
	\begin{pmatrix}
		\overrightarrow{I}\\ \overrightarrow{J}
	\end{pmatrix}\nonumber\\
	\begin{pmatrix}
		\overrightarrow{k}\\ \overrightarrow{I}''
	\end{pmatrix}
	&=&
	\begin{pmatrix}
		\cos\theta & \sin\theta\\ -\sin\theta & \cos\theta
	\end{pmatrix}
	\begin{pmatrix}
		\overrightarrow{K}\\ \overrightarrow{I}'
	\end{pmatrix}\nonumber\\
	\begin{pmatrix}
		\overrightarrow{i}\\ \overrightarrow{j}
	\end{pmatrix}
	&=&
	\begin{pmatrix}
		\cos\psi & \sin\psi\\ -\sin\psi & \cos\psi
	\end{pmatrix}
	\begin{pmatrix}
		\overrightarrow{I}''\\ \overrightarrow{J}'
	\end{pmatrix}\nonumber
\end{eqnarray}
As $\begin{pmatrix}
	\cos\alpha & \sin\alpha\\ -\sin\alpha & \cos\alpha
\end{pmatrix}\begin{pmatrix}
	\cos\alpha & -\sin\alpha\\ \sin\alpha & \cos\alpha
\end{pmatrix}=\begin{pmatrix}
	\cos\alpha & -\sin\alpha\\ \sin\alpha & \cos\alpha
\end{pmatrix}\begin{pmatrix}
	\cos\alpha & \sin\alpha\\ -\sin\alpha & \cos\alpha
\end{pmatrix}=\begin{pmatrix}
	1 & 0\\ 0 & 1
\end{pmatrix}$

i.e., $\begin{pmatrix}
	\cos\alpha & -\sin\alpha\\ \sin\alpha & \cos\alpha
\end{pmatrix}$ is the inverse of  $\begin{pmatrix}
	\cos\alpha & \sin\alpha\\ -\sin\alpha & \cos\alpha
\end{pmatrix}$. So we can write 
\begin{eqnarray}
	\begin{pmatrix}
		\cos\phi & -\sin\phi\\ \sin\phi & \cos\phi
	\end{pmatrix}
	\begin{pmatrix}
		\overrightarrow{I}'\\ \overrightarrow{J}'
	\end{pmatrix}
	=
	\begin{pmatrix}
		\overrightarrow{I}\\ \overrightarrow{J}
	\end{pmatrix}
	\implies
	\begin{matrix}
		\overrightarrow{I}=\overrightarrow{I}'\cos\phi-\overrightarrow{J}'\sin\phi\nonumber\\\overrightarrow{J}=\overrightarrow{I}'\sin\phi+\overrightarrow{J}'\cos\phi
	\end{matrix}\nonumber
\end{eqnarray}
\begin{eqnarray}
	\begin{pmatrix}
		\cos\theta & -\sin\theta\\ \sin\theta & \cos\theta
	\end{pmatrix}
	\begin{pmatrix}
		\overrightarrow{k}\\ \overrightarrow{I}''
	\end{pmatrix}
	=
	\begin{pmatrix}
		\overrightarrow{K}\\ \overrightarrow{I}'
	\end{pmatrix}
	\implies
	\begin{matrix}
		\overrightarrow{K}=\overrightarrow{k}\cos\theta-\overrightarrow{I}''\sin\theta\nonumber\\\overrightarrow{I}'=\overrightarrow{k}\sin\theta+\overrightarrow{I}''\cos\theta
	\end{matrix}\nonumber
\end{eqnarray}
\begin{eqnarray}
	\begin{pmatrix}
		\cos\psi & -\sin\psi\\ \sin\psi & \cos\psi
	\end{pmatrix}
	\begin{pmatrix}
		\overrightarrow{i}\\ \overrightarrow{j}
	\end{pmatrix}
	=
	\begin{pmatrix}
		\overrightarrow{I}''\\ \overrightarrow{J}'
	\end{pmatrix}
	\implies
	\begin{matrix}
		\overrightarrow{I}''=\overrightarrow{i}\cos\psi-\overrightarrow{j}\sin\psi\nonumber\\\overrightarrow{J}'=\overrightarrow{i}\sin\psi+\overrightarrow{j}\cos\psi
	\end{matrix}\nonumber
\end{eqnarray}
\begin{eqnarray}
	\therefore\overrightarrow{\omega}&=&\dot{\phi}\overrightarrow{K}+\dot{\theta}\overrightarrow{J}'+\dot{\psi}\overrightarrow{k}\nonumber\\
	&=&\dot{\phi}\left\{\overrightarrow{k}\cos\theta-(\overrightarrow{i}\cos\psi-\overrightarrow{j}\sin\psi)\sin\theta\right\}+\dot{\theta}\left\{\sin\psi\overrightarrow{i}+\cos\psi\overrightarrow{j}\right\}+\dot{\psi}\overrightarrow{k}\nonumber\\
	&=&\overrightarrow{i}\left\{\dot{\theta}\sin\psi-\dot{\phi}\cos\psi\sin\theta\right\}+\overrightarrow{j}\left\{\dot{\theta}\cos\psi+\dot{\phi}\sin\psi\sin\theta\right\}+\overrightarrow{k}\left\{\dot{\psi}+\dot{\phi}\cos\theta\right\}\nonumber
\end{eqnarray}
$\therefore\omega_1=\dot{\theta}\sin\psi-\dot{\phi}\cos\psi\sin\theta,~\omega_2=\dot{\theta}\cos\psi+\dot{\phi}\sin\psi\sin\theta,~\mbox{and}~\omega_3=\dot{\psi}+\dot{\phi}\cos\theta$

\vspace{.5cm}

\section{Motion of a symmetrical Top: }

A symmetrical top is a rigid body which is a solid of revolution about its axis of symmetry. Let the top is rotating about a point O on its axis of symmetry $oz$. We take $oz$ as the $z$-axis fixed with respect top. Choose $ox,~oy$ such that $ox,~oy$ and $oz$ forms a rectangular cartesian coordinate system at O such that they are fixed with respect to the top and assume them as the principle axes of inertia of the rigid body at O. Let, $A,~B,~C$ be the moment of inertia of the top with respect to $ox,~oy,~oz$ respectively. As the rigid body (Top) is symmetry about $oz$ so we have $A=B$. Let $\overrightarrow{i},~\overrightarrow{j}$ and $\overrightarrow{k}$ be the unit vectors along the three axes. Then the position of the rotating axes $ox,~oy,~oz$ at any time $t$ can be expressed with respect to the three Eulerian angles $\theta,~\phi$ and $\psi$. Also the components of angular velocity with respect to the rotating axes can be expressed with respect to the Eulerian angles and their derivatives $~\omega_1=\dot{\theta}\sin\psi-\dot{\phi}\cos\psi\sin\theta,~\omega_2=\dot{\theta}\cos\psi+\dot{\phi}\sin\psi\sin\theta,~\omega_3=\dot{\psi}+\dot{\phi}\cos\theta$.

As the rigid body is rotating about the point O with an an angular velocity $\omega$, its kinetic energy $T$ is given by 
\begin{eqnarray}
	T&=&\frac{1}{2}\left[A\omega_1^2+B\omega_2^2+C\omega_3^2\right]\nonumber\\
	&=&\frac{1}{2}\left[A\left(\dot{\theta}\sin\psi-\dot{\phi}\cos\psi\sin\theta\right)^2+A\left(\dot{\theta}\cos\psi+\dot{\phi}\sin\psi\sin\theta\right)^2+C\left(\dot{\psi}+\dot{\phi}\cos\theta\right)^2\right]\nonumber\\
	&=&\frac{1}{2}A\left(\dot{\theta}^2+\dot{\phi}^2\sin^2\theta\right)+\frac{1}{2}C\left(\dot{\psi}+\dot{\phi}\cos\theta\right)^2\nonumber
\end{eqnarray}
Let, $G$ be the centroid of the top and $OG=h$, then the potential energy of the top referred to the horizontal through O is given by $V=Mgh\cos\theta$, $M$-the mass of the top.

So the Lagrangian of the present system is 
\begin{equation}
	L=T-V=\frac{1}{2}A\left(\dot{\theta}^2+\dot{\phi}^2\sin^2\theta\right)+\frac{1}{2}C\left(\dot{\psi}+\dot{\phi}\cos\theta\right)^2-Mgh\cos\theta\nonumber
\end{equation}
Now, taking $\theta,~\phi$ and $\psi$ as the generalised coordinates, the Lagrange equation of motion can be written as 
\begin{equation}
	\frac{d}{dt}\left(\frac{\partial{L}}{\partial\dot{\theta}}\right)-\frac{\partial{L}}{\partial{\theta}}=0,~\frac{d}{dt}\left(\frac{\partial{L}}{\partial\dot{\phi}}\right)-\frac{\partial{L}}{\partial{\phi}}=0,~\frac{d}{dt}\left(\frac{\partial{L}}{\partial\dot{\phi}}\right)-\frac{\partial{L}}{\partial{\phi}}=0\nonumber
\end{equation}
\begin{eqnarray}
	\implies~A\ddot{\theta}-A\dot{\phi}^2\sin\theta\cos\theta+{C}\dot{\phi}\left(\dot{\psi}+\dot{\phi}\cos\theta\right)\sin\theta-Mgh\sin\theta&=&0\label{g1}
\end{eqnarray}
\begin{eqnarray}
	\mbox{and,~}A\frac{d}{dt}\left[\dot{\phi}\sin^2\theta\right]+C\frac{d}{dt}\left[\left(\dot{\psi}+\dot{\phi}\cos\theta\right)\cos\theta\right]&=&0\nonumber\\
	\implies~\frac{d}{dt}\left[A\dot{\phi}\sin^2\theta+C\left(\dot{\psi}+\dot{\phi}\cos\theta\right)\cos\theta\right]&=&0\label{g2}\\
	\mbox{and,~}\frac{d}{dt}\left[C\left(\dot{\psi}+\dot{\phi}\cos\theta\right)\right]&=&0\nonumber\\
	\mbox{i.e.,~}\left(\dot{\psi}+\dot{\phi}\cos\theta\right)=\mbox{Constant}&=&n(\mbox{say})\label{g3}
\end{eqnarray}
from (\ref{g2}) and (\ref{g3}) 
\begin{equation}
	A\dot{\phi}\sin^2\theta+Cn\cos\theta=\mbox{Constant}=D(\mbox{say})\label{g4}
\end{equation}
Using (\ref{g3}) and (\ref{g4}) in (\ref{g1}) we get
\begin{eqnarray}
	A\ddot{\theta}-A\dot{\phi}^2\sin\theta\cos\theta+{C}\dot{\phi}n\sin\theta-Mgh\sin\theta&=&0 \nonumber \\
	\label{g5} 
\end{eqnarray}
\begin{eqnarray}
	&~&A\ddot{\theta}-A\left[\frac{D-Cn\cos\theta}{A\sin^2\theta}\right]^2\sin\theta\cos\theta+{C}n\left[\frac{D-Cn\cos\theta}{A\sin^2\theta}\right]\sin\theta-Mgh\sin\theta=0\nonumber\\
	&~&\implies~A\ddot{\theta}+f(\theta)=0 \nonumber \\
	\label{g6}
\end{eqnarray}
where
\begin{eqnarray}
	f(\theta)&=&-\frac{1}{A}\frac{(D-Cn\cos\theta)^2}{\sin^3\theta}\cos\theta+\frac{Cn}{A}\frac{(D-Cn\cos\theta)}{\sin\theta}-Mgh\sin\theta\nonumber\\
	\mbox{i.e.,}A\sin^3\theta{f(\theta)}&=&-\cos\theta(D-Cn\cos\theta)^2+Cn\sin^2\theta(D-Cn\cos\theta)-AMgh\sin^4\theta\nonumber\\
	\label{g7}
\end{eqnarray}
Now, integrating (\ref{g6}) with respect to $\theta$ we have 
\begin{equation}
	\frac{1}{2}A\dot{\theta}^2+\Psi(\theta)=0\label{g8}
\end{equation}
Analyzing the real roots of $\Psi(\theta)=0$ and using the properties of Elliptic function one can easily find $\phi$ and $\psi$. Now we shall discuss the stability of the top when it executes steady motion at $\theta=\alpha=$constant and let $\dot{\phi}=\Omega$, a constant when $\theta=\alpha$. Form (\ref{g5}) 
\begin{equation}
	f(\alpha)=-\sin\alpha\left[A\Omega^2\cos\alpha-Cn\Omega+Mgh\right]=0\label{g9}
\end{equation}
Assuming $\theta\neq0$ (i.e.,$\alpha\neq0$) we have,
\begin{equation}
	A\Omega^2\cos\alpha-Cn\Omega+Mgh=0\label{g10}
\end{equation}
It has a pair of real and distinct roots $\Omega_1,~\Omega_2$ of $\Omega$ provided $C^2n^2>4AMgh\cos\alpha$.

Suppose that the above inequality holds good. As $f(\alpha)=0$ so if we put $\theta=\alpha+\epsilon$ where $\epsilon$ is small then $f(\theta)\approx\epsilon{f'(\alpha)}$ to the first order and hence (\ref{g6}) becomes
\begin{equation}
	A\ddot{\epsilon}+\epsilon{f'(\alpha)}=0\label{g11}
\end{equation}
Now, if we can show $f'(\alpha)>0$ then this equation will reduce to the form $\ddot{\epsilon}+\omega^2\epsilon=0$ where $\omega^2=\frac{f'(\alpha)}{A}$ showing stability about $\theta=\alpha$ for small perturbations. Now, differentiation (\ref{g7}) with respect to the $\theta$ and putting $\theta=\alpha$ and using ${f(\alpha)}=0$ we get 
\begin{eqnarray}
	A{f'(\alpha)}\sin^3\alpha&=&\sin\alpha(D-Cn\cos\alpha)^2-2Cn\cos\alpha\sin\alpha(D-Cn\cos\alpha)\nonumber \\ 
	&+&2Cn\cos\alpha\sin\alpha(D-Cn\cos\alpha)
	+C^2n^2\sin^3\alpha-4AMgh\sin^3\alpha\cos\alpha  \nonumber\\
	&=&\sin\alpha(D-Cn\cos\alpha)^2+C^2n^2\sin^3\alpha-4AMgh\sin^3\alpha\cos\alpha\nonumber
\end{eqnarray}
putting $\theta=\alpha$ in equation (\ref{g4}) we get
\begin{equation}
	D-Cn\cos\alpha=A\Omega\sin^2\alpha\nonumber
\end{equation}
So equation (\ref{g5}) $\implies~Cn=A\Omega\cos\alpha+\frac{Mgh}{\Omega}$. Hence,
\begin{eqnarray}
	A{f'(\alpha)}&=&A^2\Omega^2\sin^2\alpha+\left(A\Omega\cos\alpha+\frac{Mgh}{\Omega}\right)^2-4AMgh\cos\alpha\nonumber\\
	&=&A^2\Omega^2-2AMgh\cos\alpha+\left(\frac{Mgh}{\Omega}\right)^2\nonumber\\
	&=&\left(A\Omega-\frac{Mgh}{\Omega}\right)^2+2\frac{Mgh}{\Omega}(1-\cos\alpha)>0~\mbox{for}~\alpha\neq0\nonumber
\end{eqnarray} 
Hence for $\alpha\neq0$ the motion is S.H.M. type showing that the position $\theta=\alpha$ is one of stable configuration. 

$(\mbox{Alternatively,}$
\begin{eqnarray}
	f'(\alpha)&=&\frac{A^2\Omega^2\sin^5\alpha+C^2n^2\sin^3\alpha-4AMgh\sin^3\alpha\cos\alpha}{\sin^3\alpha}\nonumber\\
	&=&A^2\Omega^2\sin^2\alpha+\left(\frac{C^2n^2}{A}-4Mgh\cos\alpha\right)\nonumber\\
	&>&0\nonumber
\end{eqnarray}
Since $C^2n^2>4AMgh\cos\alpha$ for real roots of $\Omega$ in equation (\ref{g10}).$)$

\iffalse
\section{Appendix-I}

$\star$ \textbf{Calculus of Variation}

${\bullet}\textbf{{Fundamental lemma } :}$

If $f(x)$ is continuous in $[a,b]$ and  if $\int_{a}^{b}f(x)\phi(x)dx=0$ for all values of $\phi(x)$ which are continuous in $[a,b]$ and which vanishes at $x=a$ and $x=b$ then $f(x)\equiv0$.

\textbf{Proof } :  If possible let $f(x)$ be non-zero at a point in $[a,b]$. Without loss of generality we can take $f(x)$ to be positive at this point. Since $f(x)$ is continuous at the point and is positive there so we can find an interval $(x_0,~x_1)\subset[a,b]$ in which $f(x)$ is positive. Let $a<x_0<x_1<b$ and let us define $\phi(x)$ as follows :
\begin{eqnarray}
	\phi(x)&=&0,~a\le{x}\le{x_0}\nonumber\\
	&=&(x-x_0)^p(x_1-x)^p,~{x_0}\le{x}\le{x_1}\nonumber\\
	&=&0,~\le{x_1}\le{x}\le{b}\nonumber
\end{eqnarray}
Therefore, $\phi(x)$ is a function satisfying all the conditions of the theorem, where $p$ is a positive integer. Therefore by the condition of the theorem 
\begin{equation}
	\int_{x_0}^{x_1}f(x)(x-x_0)^p(x_1-x)^pdx=0\nonumber
\end{equation}
But this can not be true because the integrand is essentially positive in $[x_0,~x_1]$ except at the end points and the integrand is continuous. Hence our assumption that $f(x)\neq0$ is not correct and hence the lemma is true.


$\bullet${\textbf{Variation of the derivative}} :

Let us consider a curve $y=y(x)$ passing through the points $(a,b)$ and $(c,d)$. Let $y_1=y_1(x)$ be any arbitrary curve in the neighbourhood of the curve $y(x)$ and passing through the points $(a,b)$ and $(c,d)$. Let us denote the difference between the function $y_1(x)$ and $y(x)$ for the same value of $x$ as $\delta{y}$ i.e., $\delta{y}=y_1(x)-y(x)=\delta{y(x)}$, a function of $x$. This function $\delta{y}$ is called the variation of the function $y(x)$ for a given $x$.

Again '$dy$' refer to the differential change in the ordinate $y$ as we pass from a point on a curve $y=y(x)$ to a neighbouring point on the same curve, while $y_1(x)=y(x)+\delta{y(x)}$. In fact, $\delta{y(x)}$ is the infinitesimal virtual displacement suffered by the point $(x,y)$ on $y(x)$ for given $x$. Thus the points $\{x,~y(x)\}$ on the curve $y=y(x)$ shifts to the point $\{x,~y_1(x)\}$, on the neighbourhood curve, subject to the variation $\delta{y(x)}$. Similarly, variation $\delta{y'(x)}$ can define as 
\begin{equation}
	\delta{y'(x)}=y'_1(x)-y'(x)=\frac{d}{dx}(y_1-y)=\frac{d}{dx}(\delta{y})\nonumber
\end{equation}
Thus $\delta$ of $y'(x)=\frac{d}{dx}$ of $\delta{y}$ i.e., the variation of the derivative is the derivative of the variation.










$\bullet$ {\textbf{Variation of the integrals}} :

Let $I=\int_{t_0}^{t_1}f(q_1,...,q_n;\dot{q_1},...,\dot{q_n};t)dt$. Suppose by keeping $t$ fixed we may vary $q$'s such that they becomes $q_1+\delta{q_1},q_2+\delta{q_2},...,q_n+\delta{q_n}$, where $\delta{q_i=\epsilon\eta_i(t)},~i=1,2,...,n$. Here $\eta_i$'s  are continuous function of '$t$' possessing continuous derivative of 1st order and $\epsilon$ is an infinitesimal quantity. Further, it is assumed that the function $f$ together with its partial derivatives of 1st two orders are continuous. Now define 
\begin{equation}
	I_1=\int_{t_0}^{t_1}f(q_1+\epsilon{\eta_1},q_2+\epsilon{\eta_2},...,q_n+\epsilon{\eta_n};\dot{q_1}+\epsilon\dot{\eta},...,\dot{q_n}+\epsilon\dot{\eta_n};t)dt\nonumber
\end{equation}
so 
\begin{equation}
	I_1-I=\int_{t_0}^{t_1}\left[\sum_{i=1}^{n}\left\{\frac{\partial{f}}{\partial{q_i}}\epsilon\eta_i+\frac{\partial{f}}{\partial{\dot{q_i}}}\epsilon\dot{\eta_i}\right\}+R\right]dt\nonumber
\end{equation}
{(by Taylor's expansion)}

where $R$ is an infinitesimal in comparison to 1st term. Thus we call 
\begin{equation}
	I_1-I=\int_{t_0}^{t_1}\sum_{i=1}^{n}\left\{\frac{\partial{f}}{\partial{q_i}}\epsilon\eta_i+\frac{\partial{f}}{\partial{\dot{q_i}}}\epsilon\dot{\eta_i}\right\}dt\nonumber
\end{equation}
as the 1st variation of $I$ or simply the variation $I$ and we denote it by $\delta{I}$. Also one can denote 
\begin{equation}
	\sum_{i=1}\left(\frac{\partial{f}}{\partial{q_i}}\epsilon\eta_i+\frac{\partial{f}}{\partial{\dot{q_i}}}\epsilon\dot{\eta_i}\right)dt\nonumber
\end{equation}
by $\delta{f}$ i.e., 
\begin{equation}
	\delta{I}=\delta\int_{t_0}^{t_1}fdt=\int_{t_0}^{t_1}\delta{f}dt\nonumber
\end{equation}
i.e., variation of the integral =integral of the variation of the integrand.

\underline{Note } : As $\delta{q_i=\epsilon\eta_i(t)}$ so $\delta\dot{q_i}=\epsilon\dot{\eta_i}(t)=\frac{d}{dt}(\epsilon\eta_i)=\frac{d}{dt}(\delta{q_i})$.






$\bullet$ {\textbf{Variation of a function with		
		variation of path with respect to time}} :
\begin{figure}
	\centering
	\includegraphics[width=0.6\textwidth]{Gopal_Da_12.pdf}\\
	\label{fig1}
\end{figure}

Let us consider two points $A$ and $B$ in the configuration space and let us join these two points by two neighbouring curves $C$ and $C_1$. Let $q(t)$ denote the actual motion of a certain system along $C$ for the time $t_0\le{t}\le{t_1}$. Further, let $q_1(t)$denote the arbitrary varying motion of the same system along $C_1$ for $t'_0\le{t}\le{t'_1}$. Let us consider the difference 
\begin{eqnarray}
	q_1(t+\Delta)-q_1(t)&\simeq&{q_1(t)}+\Delta{t}\dot{q_1}-q(t)\nonumber\\
	&=&\delta{q}+\Delta{t}\frac{d}{dt}(q+\delta{q})\nonumber\\
	&=&\delta{q}+\Delta{t}\dot{q_1}~\mbox{upto 1st order}\nonumber
\end{eqnarray} 
We denote this difference by $\Delta{q}$ and call it the 1st variation of $q(t)$ with variation of time. 
\begin{subequations}
	\begin{equation}
		\therefore\Delta{q}=\delta{q}+\Delta{t}\dot{q_1}\label{g13}
	\end{equation}
\end{subequations}
Now differentiating equation (\ref{g13}) with respect to '$t$' we get 
\begin{subequations}
	\begin{eqnarray}
		\frac{d}{dt}(\Delta{q})&=&\frac{d}{dt}(\delta{q})+\Delta{t}\frac{d}{dt}(\dot{q})+\dot{q}\frac{d}{dt}(\Delta{t})\nonumber\\
		&=&\delta{\dot{q}}+\Delta{t}\frac{d}{dt}(\dot{q})+\dot{q}\frac{d}{dt}(\Delta{t})\nonumber\\
		&=&\Delta{\dot{q}}+\dot{q}\frac{d}{dt}(\Delta{t})\nonumber
	\end{eqnarray}
	\begin{equation}
		\mbox{i.e.,}~\frac{d}{dt}(\Delta{q})=\Delta{\dot{q}}+\dot{q}\frac{d}{dt}(\Delta{t})\label{g14}
	\end{equation}
\end{subequations}
Hence $\frac{d}{dt}(\Delta{q})\neq\Delta{\dot{q}}$.

Analogously, the variation of a function on $f(q,\dot{q},t)$ is defined as 
\begin{subequations}
	\begin{eqnarray}
		\therefore\Delta{f}&=&\delta{f}+\Delta{t}\dot{f}\nonumber\\
		&=&\frac{\partial{f}}{\partial{q}}\delta{q}+\frac{\partial{f}}{\partial{\dot{q}}}\delta{\dot{q}}+\Delta{t}\dot{f}\nonumber\\
		&=&\frac{\partial{f}}{\partial{q}}\delta{q}+\frac{\partial{f}}{\partial{\dot{q}}}\delta{\dot{q}}+\Delta{t}\left(\frac{\partial{f}}{\partial{q}}\dot{q}+\frac{\partial{f}}{\partial{\dot{q}}}{\ddot{q}}+\frac{\partial{f}}{\partial{t}}\right)\nonumber\\
		&=&\frac{\partial{f}}{\partial{q}}(\delta{q}+\dot{q}\Delta{t})+\frac{\partial{f}}{\partial{\dot{q}}}(\delta{\dot{q}}+\ddot{q}\Delta{t})+\Delta{t}\frac{\partial{f}}{\partial{t}}\nonumber\\
		&=&\frac{\partial{f}}{\partial{q}}\Delta{q}+\frac{\partial{f}}{\partial{\dot{q}}}\Delta{\dot{q}}+\Delta{t}\frac{\partial{f}}{\partial{t}}\nonumber
	\end{eqnarray}
	\begin{equation}
		\mbox{i.e.,}~\Delta{f}=\frac{\partial{f}}{\partial{q}}\Delta{q}+\frac{\partial{f}}{\partial{\dot{q}}}\Delta{\dot{q}}+\Delta{t}\frac{\partial{f}}{\partial{t}}\label{g15}
	\end{equation}
\end{subequations}

$\bullet$ {\textbf{Variation of integral with variation of time}} :

Let us consider the integral $I=\int_{t_0}^{t_1}fdt$ for the actual motion $C$ and $I=\int_{t_0+\Delta{t_0}}^{t_1+\Delta{t_1}}f_1dt$, for the varied path of motion $C_1$ with $f_1=f+\delta{f}$. Now, 
\begin{subequations}
	\begin{equation}
		I_1=\int_{t_0}^{t_1}f_1dt+\int_{t_1}^{t_1+\Delta{t_1}}f_1dt-\int_{t_0}^{t_0+\Delta{t_0}}f_1dt\nonumber
	\end{equation}
	\begin{eqnarray}
		\therefore{I_1-I}&=&\int_{t_0}^{t_1}(f_1-f)dt+(f_1)_{t_1}\Delta{t_1}-(f_1)_{t_0}\Delta{t_0}\nonumber\\
		&&\left(\mbox{By mean value theorem }\int_{a}^{b}f(x)\phi(x)dx=f(\xi)\int_{a}^{b}\phi(x)dx\right)\nonumber\\
		&=&\int_{t_0}^{t_1}\delta{f}dt+f_1\Delta{t_1}|_{t_0}^{t_1}~~(\mbox{assuming continuity of the function} f_1(t))\nonumber\\
		&=&\int_{t_0}^{t_1}\left[\delta{f}+\frac{d}{dt}(f\Delta{t})\right]dt~~(\mbox{upto 1st order})\nonumber
	\end{eqnarray}
\end{subequations}
Thus 
\begin{subequations}
	\begin{equation}
		\Delta{I}=\int_{t_0}^{t_1}\left[\delta{f}+\frac{d}{dt}(f\Delta{t})\right]dt\label{g16}
	\end{equation}
\end{subequations}













































\fi





































































































































































\section{Principle of Linear Momentum  : Conservation of Linear Momentum: } 

If a physical system is under the action of a system of externally applied forces and subject to bilateral constraints only, the rate of change of linear momentum of the system is equal to the vector sum of all the externally applied forces acting on the system.

Suppose we have a physical system imposed of $N$ particles. Let the system is under the action of a system of externally applied forces and subject to bilateral constraints only. Let $P_i$ be the position of the $i$-th particle having mass $m_i$ and position vector $\overrightarrow{r_i}$ with respect to a fixed point $O~i.e.~\overrightarrow{OP}=\overrightarrow{r_i}$. Suppose $\overrightarrow{F_i}$ be the resultant applied forces acting on the $i$-th particle. According to D'Alembert principle


\begin{equation}
\sum_{i=1}^{N}\left\{\overrightarrow{F_i}+(-m_i\ddot{\overrightarrow{r_i}})\right\}=0~~(\mbox{basis principle of statics})\nonumber
\end{equation}
\begin{eqnarray}
\mbox{i.e.,}~\sum{m_i\ddot{\overrightarrow{r_i}}}&=&\sum_{i}\overrightarrow{F_i}\nonumber\\
\mbox{i.e.,}~\frac{d}{dt}\left(\sum{m_i\dot{\overrightarrow{r_i}}}\right)&=&\sum_{i}\overrightarrow{F_i}\nonumber
\end{eqnarray}
Thus the rate of change of linear momentum of a material system in a given direction is equal to the vector sum of all the external forces applied to the system in that direction.

In particular, if $\sum\overrightarrow{F_i}=0$ then $\frac{d}{dt}\left(\sum{m_i\dot{\overrightarrow{r_i}}}\right)=0$ 

i.e., $\sum{m_i\dot{\overrightarrow{r_i}}}=$ Constant.

So the total linear momentum of a material system is conserved. This is known as conservation of linear momentum. 



\section{Principle of Angular Momentum  : Conservation of Angular Momentum: }	

If a physical system is under the action of a system of externally applied forces and subjected to bilateral constrains only, then the rate of change of angular momentum (i.e. moment of momentum) of the system in a given direction is equal to the vector sum of the moments of all the externally applied forces acting on the system in that direction.

Form the basic principle of statics if a system of external forces are in equilibrium then the sum of the moments of all forces about any point in space is equal to zero.

Thus taking moment about O
\begin{eqnarray}
\sum\overrightarrow{r_i}\times\left\{\overrightarrow{F_i}+(-m_i\ddot{\overrightarrow{r_i}})\right\}&=&0\nonumber\\
\mbox{i.e.,~}~\sum\overrightarrow{r_i}\times{m_i\ddot{\overrightarrow{r_i}}}&=&\sum_{i}\overrightarrow{r_i}\times{\overrightarrow{F_i}}\nonumber\\
\mbox{i.e.,~}~\frac{d}{dt}\left(\overrightarrow{r_i}\times{m_i\dot{\overrightarrow{r_i}}}\right)&=&\sum\overrightarrow{r_i}\times{\overrightarrow{F_i}}\nonumber\\
\mbox{i.e.,~}~\frac{d\overrightarrow{H}}{dt}&=&\sum\overrightarrow{r_i}\times{\overrightarrow{F_i}}=\overrightarrow{L}\nonumber
\end{eqnarray}
where $\overrightarrow{H}=\overrightarrow{r_i}\times{m_i\dot{\overrightarrow{r_i}}}$ is the angular momentum of the material system in a given direction and $\overrightarrow{L}$ is the vector sum of the moments of all externally applied forces acting on the system in that direction.

Now, if $\overrightarrow{L}=0$ then $\overrightarrow{L}=$Constant i.e., total angular momentum in that direction of the system is conserved. This is known as conservation of angular momentum.

\section{Euler's Dynamical Equation:}

Suppose a physical system rotate about O which is fixed with respect to the system as well as the space. Consider a rectangular coordinate system $ox,~oy,~oz$ through O. These axes are fixed with respect to the system. Let $\overrightarrow{i},~\overrightarrow{j}$ and $\overrightarrow{k}$ be the unit vector along these coordinate axes. Then $\overrightarrow{r}=x\overrightarrow{i}+y\overrightarrow{j}+z\overrightarrow{z}$ with $(x,y,z)$ be the coordinate of the point $P$ with respect to the coordinate system $ox,~oy,~oz$ which is fixed in the system . Now,
\begin{eqnarray}
\dot{\overrightarrow{r}}=\frac{d\overrightarrow{r}}{dt}&=&\frac{\partial{\overrightarrow{r}}}{\partial{t}}+\overrightarrow{\omega}\times{\overrightarrow{r}}\nonumber\\
&=&\frac{\partial}{\partial{t}}(x\overrightarrow{i}+y\overrightarrow{j}+z\overrightarrow{z})+\overrightarrow{\omega}\times{\overrightarrow{r}}\nonumber\\
&=&\dot{x}\overrightarrow{i}+\dot{y}\overrightarrow{j}+\dot{z}\overrightarrow{z}+\overrightarrow{\omega}\times{\overrightarrow{r}}\nonumber\\
&=&\overrightarrow{\omega}\times{\overrightarrow{r}}\nonumber
\end{eqnarray}

($\because{\dot{x}=\dot{y}=\dot{z}=0}$ as the axes are fixed in the rigid body)

Suppose $\overrightarrow{\omega}=\omega_1\overrightarrow{i}+\omega_2\overrightarrow{j}+\omega_3\overrightarrow{k}$ be the angular velocity of rotation through O. Thus
\begin{eqnarray}
\overrightarrow{H}=\sum\overrightarrow{r}\times({m_i\dot{\overrightarrow{r}}})&=&\sum{m}\overrightarrow{r}\times(\overrightarrow{\omega}\times{\overrightarrow{r}})\nonumber\\
&=&\sum m\left[(\overrightarrow{r}.\overrightarrow{r})\overrightarrow{\omega}-(\overrightarrow{r}.\overrightarrow{\omega})\overrightarrow{r}\right]\nonumber\\
&=&\sum{m}\Big{[}(x^2+y^2+z^2)(\omega_1\overrightarrow{i}+\omega_2\overrightarrow{j}+\omega_3\overrightarrow{k})\nonumber \\
&& -(x\omega_1+y\omega_2+z\omega_3)(x\overrightarrow{i}+y\overrightarrow{j}+z\overrightarrow{k})\Big{]}\nonumber
\end{eqnarray}

\begin{equation}
H=\sum\Big{[}\omega_1\{(y^2+z^2)\overrightarrow{i}-xy\overrightarrow{j}-xz\overrightarrow{k}\}+\omega_2\{(x^2+z^2)\overrightarrow{j}-xy\overrightarrow{i}-yz\overrightarrow{k}\} \nonumber \\
 +\omega_3\{(y^2+x^2)\overrightarrow{k}-xz\overrightarrow{i}-yz\overrightarrow{j}\}\Big{]}\nonumber
\end{equation}
As $\omega_1,~\omega_2,~\omega_3$ are same for all the particles of the material system so
\begin{eqnarray}
\overrightarrow{H}&=&\omega_1\left\{\left[\sum{m}(y^2+z^2)\right]\overrightarrow{i}-\left[\sum{mxy}\right]\overrightarrow{j}-\left[\sum{mxz}\right]\overrightarrow{k}\right\}\nonumber\\
&+&\omega_2\left\{\left[\sum{m}(x^2+z^2)\right]\overrightarrow{j}-\left[\sum{mxy}\right]\overrightarrow{i}-\left[\sum{myz}\right]\overrightarrow{k}\right\}\nonumber\\
&+&\omega_1\left\{\left[\sum{m}(y^2+x^2)\right]\overrightarrow{k}-\left[\sum{mxz}\right]\overrightarrow{i}-\left[\sum{mzy}\right]\overrightarrow{j}\right\}\nonumber
\end{eqnarray}
Now, if $A,~B,~C$ are the moments of inertia and $D,~E,~F$ are the product of inertia of the physical system with respect to the coordinate system $ox,~oy,~oz$ through O, then $A=\sum{m}(y^2+z^2),~B=\sum{m}(x^2+z^2),~C=\sum{m}(x^2+y^2),~D=\sum{mzy},~E=\sum{mzx},~F=\sum{mxy}$. Thus $\overrightarrow{H}=\omega_1\left(A\overrightarrow{i}-F\overrightarrow{j}-E\overrightarrow{k}\right)+\omega_2\left(B\overrightarrow{j}-F\overrightarrow{i}-D\overrightarrow{k}\right)+\omega_3\left(C\overrightarrow{k}-E\overrightarrow{i}-D\overrightarrow{j}\right)$.

Further, if the coordinate axes are the principal axes of inertia then $D=E=F=0$ and we get $\overrightarrow{H}=\omega_1A\overrightarrow{i}+\omega_2B\overrightarrow{j}+\omega_3C\overrightarrow{k}$

$\therefore\frac{dH}{dt}=\frac{\partial{\overrightarrow{H}}}{\partial{t}}+\overrightarrow{\omega}\times\overrightarrow{H}=A\dot{\omega}_1\overrightarrow{i}+B\dot{\omega}_2\overrightarrow{j}+C\dot{\omega}_3\overrightarrow{k}+\begin{vmatrix}
\overrightarrow{i} & \overrightarrow{j} & \overrightarrow{k}\\
\omega_1 & \omega_2 & \omega_3\\
H_1 & H_2 & H_3\\
\end{vmatrix}$

Also if $\overrightarrow{L}=L_1\overrightarrow{i}+L_2\overrightarrow{j}+L_3\overrightarrow{k}$, then
\begin{eqnarray}
L_1=A\dot{\omega}_1-(B-C)\omega_2\omega_3\nonumber\\
L_2=B\dot{\omega}_2-(C-A)\omega_3\omega_1\nonumber\\
L_3=C\dot{\omega}_3-(A-B)\omega_1\omega_2\nonumber
\end{eqnarray}
These are known as Euler's dynamical equations. Kinetic energy of a physical system with respect to a rotating coordinate system which is fixed with respect to the system can be derived as follows:
\begin{eqnarray}
T=\frac{1}{2}\sum{m}\left|\frac{d\overrightarrow{r}}{dt}\right|^2=\frac{1}{2}\sum{m}\frac{d\overrightarrow{r}}{dt}.\frac{d\overrightarrow{r}}{dt}\nonumber\\
\dot{\overrightarrow{r}}=\frac{d\overrightarrow{r}}{dt}=\frac{\partial{\overrightarrow{r}}}{\partial{t}}+\overrightarrow{\omega}\times{\overrightarrow{r}}=\overrightarrow{\omega}\times{\overrightarrow{r}}\nonumber
\end{eqnarray}
(as the axes are fixed in the system)
\begin{eqnarray}
\therefore{T}&=&\frac{1}{2}\sum{m}(\overrightarrow{\omega}\times\overrightarrow{r}).\frac{d\overrightarrow{r}}{dt}\nonumber\\
&=&\frac{1}{2}\sum{m}\overrightarrow{\omega}.\left(\overrightarrow{r}\times\frac{d\overrightarrow{r}}{dt}\right)\nonumber\\
&=&\frac{1}{2}\sum{m}\overrightarrow{\omega}.\left\{\overrightarrow{r}\times(\overrightarrow{\omega}\times\overrightarrow{r})\right\}\nonumber\\
&=&\frac{1}{2}\sum{m}\overrightarrow{\omega}.\left[(\overrightarrow{r}.\overrightarrow{r})\overrightarrow{\omega}-(\overrightarrow{r}.\overrightarrow{\omega})\overrightarrow{r}\right]\nonumber\\
&=&\frac{1}{2}\sum{m}\left[(\overrightarrow{r}.\overrightarrow{r})(\overrightarrow{\omega}.\overrightarrow{\omega})-(\overrightarrow{r}.\overrightarrow{\omega})(\overrightarrow{\omega}.\overrightarrow{r})\right]\nonumber\\
&=&\frac{1}{2}\sum{m}\left[(x^2+y^2+z^2)(\omega_1^2+\omega_2^2+\omega_3^2)-(x\omega_1+y\omega_2+z\omega_3)^2\right]\nonumber\\
&=&\frac{1}{2}\sum{m}\left[\omega_1^2(y^2+z^2)+\omega_2^2(x^2+z^2)+\omega_3^2(y^2+x^2)-2\omega_1\omega_2xy-2\omega_2\omega_3yz-2\omega_3\omega_1xz\right]\nonumber
\end{eqnarray}
As $\omega_1,~\omega_2$ and $\omega_3$ are same for all points so 
\begin{eqnarray}
{T}&=&\frac{1}{2}\Big{[}\omega_1^2\sum{m}(y^2+z^2)+\omega_2^2\sum{m}(x^2+z^2)+\omega_3^2\sum{m}(y^2+x^2)-2\omega_1\omega_2\sum{m}xy \nonumber \\
&-& 2\omega_2\omega_3\sum{m}zy\Big{]}
-\omega_1\omega_3\sum{m}xz\nonumber\\
&=&\frac{1}{2}\left[A\omega_1^2+B\omega_2^2+C\omega_3^2-2\omega_1\omega_2F-2\omega_2\omega_3D-2\omega_1\omega_3E\right]\nonumber
\end{eqnarray}
Further, if the axes are principal axes then
\begin{equation}
{T}=\frac{1}{2}\left[A\omega_1^2+B\omega_2^2+C\omega_3^2\right]\nonumber
\end{equation}



