%\pagenumbering{arabic}



\chapter{Appendix}



\section{Appendix-I: A simple note on vector calculus}


~~~\textbf{Result I:} If $\vec{a}(t)$ is a vector of constant magnitude i.e. $|\vec{a}(t)|=$ constant, then $\dfrac{\mathrm{d}\vec{a}}{\mathrm{d}t}$ is orthogonal to $\vec{a}$.

\textbf{Proof:} As $|\vec{a}(t)|=$ constant,

So, $\vec{a}\cdot\vec{a}=|\vec{a}|^2=$ constant

$\therefore ~2\vec{a}\cdot\dfrac{\mathrm{d}\vec{a}}{\mathrm{d}t}=0$

Hence for non-zero vector $\vec{a}$,  $\dfrac{\mathrm{d}\vec{a}}{\mathrm{d}t}$ is orthogonal to $\vec{a}$.\\

\textbf{Remark:} A natural question arises for a unit vector :

If $\vec{a}$ is an unit vector then $\dfrac{\mathrm{d}\vec{a}}{\mathrm{d}t}$ is orthogonal to $\vec{a}$. But does $\dfrac{\mathrm{d}\vec{a}}{\mathrm{d}t}$ is also an unit vector?\\

\textbf{Result II:} If $\vec{a}$ is an unit vector then $\dfrac{\mathrm{d}\vec{a}}{\mathrm{d}\theta}$ is an unit vector.

\begin{wrapfigure}[8]{r}{0.48\textwidth}
	\includegraphics[height=4.5 cm , width=7 cm ]{f1.pdf}
	\begin{center}
		Fig. 1.1
	\end{center}\vspace{-\intextsep}
\end{wrapfigure}

\textbf{Proof:}  Let $\overrightarrow{OP}=\vec{a}$ be the position of the unit vector in time $t$ and $\overrightarrow{OQ}$ be the position of the unit vector in time $t+\Delta t$. As $\vec{a}$ is a unit vector, so, $OP=OQ=1$.

Let $\angle POQ=\Delta\theta$.  Draw $ON\perp PQ$.

Let $\overrightarrow{OQ}=\vec{a}+\overrightarrow{\Delta a}$,  so that $\overrightarrow{PQ}=\overrightarrow{\Delta a}$ and $\overrightarrow{NQ}=\dfrac{1}{2}\overrightarrow{\Delta a}$.

Now, in $\triangle ONQ$, ~$\dfrac{NQ}{OQ}=\sin\left(\dfrac{1}{2}\Delta\theta\right)$

$\therefore~\sin\left(\dfrac{1}{2}\Delta\theta\right)=NQ=|\overrightarrow{NQ}|=\dfrac{1}{2}|\overrightarrow{\Delta a}|$

$\therefore~\dfrac{\sin\left(\dfrac{1}{2}\Delta\theta\right)}{\dfrac{1}{2}\Delta\theta}=\dfrac{\dfrac{1}{2}|\overrightarrow{\Delta a}|}{\dfrac{1}{2}\Delta\theta}$

Now proceed to the limit as $\Delta\theta\rightarrow0$, we have $\big|\dfrac{\mathrm{d}\vec{a}}{\mathrm{d}\theta}=1\big|$.

Hence $\dfrac{\mathrm{d}\vec{a}}{\mathrm{d}\theta}$ is perpendicular to $\vec{a}$ and is of unit magnitude.


\section{Appendix-II: Some Results on Calculus of variation}

$\star$ \textbf{Calculus of Variation}

${\bullet}\textbf{{Fundamental lemma } :}$

If $f(x)$ is continuous in $[a,b]$ and  if $\int_{a}^{b}f(x)\phi(x)dx=0$ for all values of $\phi(x)$ which are continuous in $[a,b]$ and which vanishes at $x=a$ and $x=b$ then $f(x)\equiv0$.

\textbf{Proof } :  If possible let $f(x)$ be non-zero at a point in $[a,b]$. Without loss of generality we can take $f(x)$ to be positive at this point. Since $f(x)$ is continuous at the point and is positive there so we can find an interval $(x_0,~x_1)\subset[a,b]$ in which $f(x)$ is positive. Let $a<x_0<x_1<b$ and let us define $\phi(x)$ as follows :
\begin{eqnarray}
	\phi(x)&=&0,~a\le{x}\le{x_0}\nonumber\\
	&=&(x-x_0)^p(x_1-x)^p,~{x_0}\le{x}\le{x_1}\nonumber\\
	&=&0,~\le{x_1}\le{x}\le{b}\nonumber
\end{eqnarray}
Therefore, $\phi(x)$ is a function satisfying all the conditions of the theorem, where $p$ is a positive integer. Therefore by the condition of the theorem 
\begin{equation}
	\int_{x_0}^{x_1}f(x)(x-x_0)^p(x_1-x)^pdx=0\nonumber
\end{equation}
But this can not be true because the integrand is essentially positive in $[x_0,~x_1]$ except at the end points and the integrand is continuous. Hence our assumption that $f(x)\neq0$ is not correct and hence the lemma is true.


$\bullet${\textbf{Variation of the derivative}} :

Let us consider a curve $y=y(x)$ passing through the points $(a,b)$ and $(c,d)$. Let $y_1=y_1(x)$ be any arbitrary curve in the neighbourhood of the curve $y(x)$ and passing through the points $(a,b)$ and $(c,d)$. Let us denote the difference between the function $y_1(x)$ and $y(x)$ for the same value of $x$ as $\delta{y}$ i.e., $\delta{y}=y_1(x)-y(x)=\delta{y(x)}$, a function of $x$. This function $\delta{y}$ is called the variation of the function $y(x)$ for a given $x$.

Again '$dy$' refer to the differential change in the ordinate $y$ as we pass from a point on a curve $y=y(x)$ to a neighbouring point on the same curve, while $y_1(x)=y(x)+\delta{y(x)}$. In fact, $\delta{y(x)}$ is the infinitesimal virtual displacement suffered by the point $(x,y)$ on $y(x)$ for given $x$. Thus the points $\{x,~y(x)\}$ on the curve $y=y(x)$ shifts to the point $\{x,~y_1(x)\}$, on the neighbourhood curve, subject to the variation $\delta{y(x)}$. Similarly, variation $\delta{y'(x)}$ can define as 
\begin{equation}
	\delta{y'(x)}=y'_1(x)-y'(x)=\frac{d}{dx}(y_1-y)=\frac{d}{dx}(\delta{y})\nonumber
\end{equation}
Thus $\delta$ of $y'(x)=\frac{d}{dx}$ of $\delta{y}$ i.e., the variation of the derivative is the derivative of the variation.










$\bullet$ {\textbf{Variation of the integrals}} :

Let $I=\int_{t_0}^{t_1}f(q_1,...,q_n;\dot{q_1},...,\dot{q_n};t)dt$. Suppose by keeping $t$ fixed we may vary $q$'s such that they becomes $q_1+\delta{q_1},q_2+\delta{q_2},...,q_n+\delta{q_n}$, where $\delta{q_i=\epsilon\eta_i(t)},~i=1,2,...,n$. Here $\eta_i$'s  are continuous function of '$t$' possessing continuous derivative of 1st order and $\epsilon$ is an infinitesimal quantity. Further, it is assumed that the function $f$ together with its partial derivatives of 1st two orders are continuous. Now define 
\begin{equation}
	I_1=\int_{t_0}^{t_1}f(q_1+\epsilon{\eta_1},q_2+\epsilon{\eta_2},...,q_n+\epsilon{\eta_n};\dot{q_1}+\epsilon\dot{\eta},...,\dot{q_n}+\epsilon\dot{\eta_n};t)dt\nonumber
\end{equation}
so 
\begin{equation}
	I_1-I=\int_{t_0}^{t_1}\left[\sum_{i=1}^{n}\left\{\frac{\partial{f}}{\partial{q_i}}\epsilon\eta_i+\frac{\partial{f}}{\partial{\dot{q_i}}}\epsilon\dot{\eta_i}\right\}+R\right]dt\nonumber
\end{equation}
{(by Taylor's expansion)}

where $R$ is an infinitesimal in comparison to 1st term. Thus we call 
\begin{equation}
	I_1-I=\int_{t_0}^{t_1}\sum_{i=1}^{n}\left\{\frac{\partial{f}}{\partial{q_i}}\epsilon\eta_i+\frac{\partial{f}}{\partial{\dot{q_i}}}\epsilon\dot{\eta_i}\right\}dt\nonumber
\end{equation}
as the 1st variation of $I$ or simply the variation $I$ and we denote it by $\delta{I}$. Also one can denote 
\begin{equation}
	\sum_{i=1}\left(\frac{\partial{f}}{\partial{q_i}}\epsilon\eta_i+\frac{\partial{f}}{\partial{\dot{q_i}}}\epsilon\dot{\eta_i}\right)dt\nonumber
\end{equation}
by $\delta{f}$ i.e., 
\begin{equation}
	\delta{I}=\delta\int_{t_0}^{t_1}fdt=\int_{t_0}^{t_1}\delta{f}dt\nonumber
\end{equation}
i.e., variation of the integral =integral of the variation of the integrand.

\underline{Note } : As $\delta{q_i=\epsilon\eta_i(t)}$ so $\delta\dot{q_i}=\epsilon\dot{\eta_i}(t)=\frac{d}{dt}(\epsilon\eta_i)=\frac{d}{dt}(\delta{q_i})$.






$\bullet$ {\textbf{Variation of a function with		
		variation of path with respect to time}} :
\begin{figure}
	\centering
	\includegraphics[width=0.4\textwidth]{Gopal_Da_12.pdf}\\
	\label{fig1}
\end{figure}

Let us consider two points $A$ and $B$ in the configuration space and let us join these two points by two neighbouring curves $C$ and $C_1$. Let $q(t)$ denote the actual motion of a certain system along $C$ for the time $t_0\le{t}\le{t_1}$. Further, let $q_1(t)$denote the arbitrary varying motion of the same system along $C_1$ for $t'_0\le{t}\le{t'_1}$. Let us consider the difference 
\begin{eqnarray}
	q_1(t+\Delta)-q_1(t)&\simeq&{q_1(t)}+\Delta{t}\dot{q_1}-q(t)\nonumber\\
	&=&\delta{q}+\Delta{t}\frac{d}{dt}(q+\delta{q})\nonumber\\
	&=&\delta{q}+\Delta{t}\dot{q_1}~\mbox{upto 1st order}\nonumber
\end{eqnarray} 
We denote this difference by $\Delta{q}$ and call it the 1st variation of $q(t)$ with variation of time. 
\begin{subequations}
	\begin{equation}
		\therefore\Delta{q}=\delta{q}+\Delta{t}\dot{q_1}\label{g13}
	\end{equation}
\end{subequations}
Now differentiating equation (\ref{g13}) with respect to '$t$' we get 
\begin{subequations}
	\begin{eqnarray}
		\frac{d}{dt}(\Delta{q})&=&\frac{d}{dt}(\delta{q})+\Delta{t}\frac{d}{dt}(\dot{q})+\dot{q}\frac{d}{dt}(\Delta{t})\nonumber\\
		&=&\delta{\dot{q}}+\Delta{t}\frac{d}{dt}(\dot{q})+\dot{q}\frac{d}{dt}(\Delta{t})\nonumber\\
		&=&\Delta{\dot{q}}+\dot{q}\frac{d}{dt}(\Delta{t})\nonumber
	\end{eqnarray}
	\begin{equation}
		\mbox{i.e.,}~\frac{d}{dt}(\Delta{q})=\Delta{\dot{q}}+\dot{q}\frac{d}{dt}(\Delta{t})\label{g14}
	\end{equation}
\end{subequations}
Hence $\frac{d}{dt}(\Delta{q})\neq\Delta{\dot{q}}$.

Analogously, the variation of a function on $f(q,\dot{q},t)$ is defined as 
\begin{subequations}
	\begin{eqnarray}
		\therefore\Delta{f}&=&\delta{f}+\Delta{t}\dot{f}\nonumber\\
		&=&\frac{\partial{f}}{\partial{q}}\delta{q}+\frac{\partial{f}}{\partial{\dot{q}}}\delta{\dot{q}}+\Delta{t}\dot{f}\nonumber\\
		&=&\frac{\partial{f}}{\partial{q}}\delta{q}+\frac{\partial{f}}{\partial{\dot{q}}}\delta{\dot{q}}+\Delta{t}\left(\frac{\partial{f}}{\partial{q}}\dot{q}+\frac{\partial{f}}{\partial{\dot{q}}}{\ddot{q}}+\frac{\partial{f}}{\partial{t}}\right)\nonumber\\
		&=&\frac{\partial{f}}{\partial{q}}(\delta{q}+\dot{q}\Delta{t})+\frac{\partial{f}}{\partial{\dot{q}}}(\delta{\dot{q}}+\ddot{q}\Delta{t})+\Delta{t}\frac{\partial{f}}{\partial{t}}\nonumber\\
		&=&\frac{\partial{f}}{\partial{q}}\Delta{q}+\frac{\partial{f}}{\partial{\dot{q}}}\Delta{\dot{q}}+\Delta{t}\frac{\partial{f}}{\partial{t}}\nonumber
	\end{eqnarray}
	\begin{equation}
		\mbox{i.e.,}~\Delta{f}=\frac{\partial{f}}{\partial{q}}\Delta{q}+\frac{\partial{f}}{\partial{\dot{q}}}\Delta{\dot{q}}+\Delta{t}\frac{\partial{f}}{\partial{t}}\label{g15}
	\end{equation}
\end{subequations}

$\bullet$ {\textbf{Variation of integral with variation of time}} :

Let us consider the integral $I=\int_{t_0}^{t_1}fdt$ for the actual motion $C$ and $I=\int_{t_0+\Delta{t_0}}^{t_1+\Delta{t_1}}f_1dt$, for the varied path of motion $C_1$ with $f_1=f+\delta{f}$. Now, 
\begin{subequations}
	\begin{equation}
		I_1=\int_{t_0}^{t_1}f_1dt+\int_{t_1}^{t_1+\Delta{t_1}}f_1dt-\int_{t_0}^{t_0+\Delta{t_0}}f_1dt\nonumber
	\end{equation}
	\begin{eqnarray}
		\therefore{I_1-I}&=&\int_{t_0}^{t_1}(f_1-f)dt+(f_1)_{t_1}\Delta{t_1}-(f_1)_{t_0}\Delta{t_0}\nonumber\\
		&&\left(\mbox{By mean value theorem }\int_{a}^{b}f(x)\phi(x)dx=f(\xi)\int_{a}^{b}\phi(x)dx\right)\nonumber\\
		&=&\int_{t_0}^{t_1}\delta{f}dt+f_1\Delta{t_1}|_{t_0}^{t_1}~~(\mbox{assuming continuity of the function} f_1(t))\nonumber\\
		&=&\int_{t_0}^{t_1}\left[\delta{f}+\frac{d}{dt}(f\Delta{t})\right]dt~~(\mbox{upto 1st order})\nonumber
	\end{eqnarray}
\end{subequations}
Thus 
\begin{subequations}
	\begin{equation}
		\Delta{I}=\int_{t_0}^{t_1}\left[\delta{f}+\frac{d}{dt}(f\Delta{t})\right]dt\label{g16}
	\end{equation}
\end{subequations}






 
 \section{Appendix III: Uniqueness and other properties of Lagrangian function}
 
For a system of $n$ second order differential equations of the form $F_i(t,x_j,\dot{x}_j,\ddot{x}_j)=0,~ i,j=1,~2,~3~...~n$, a natural question arises whether these differential equations may be the Euler-Lagrange equations corresponding to a Lagrangian $L=L(t,x_j,\dot{x}_j)$. From the point of view of dynamics, a physical system with $n$ d.f can be described by $n$ number of second order differential equations.

In the literature, one has Helmholtz conditions which are necessary and sufficient conditions for the above 2nd order diff. eqs to be the Euler-Lagrange eqs. corresponding to a Lagrangian. The conditions are

$(i)~\frac{\partial F_i}{\partial \ddot{x}_j}=\frac{\partial F_j}{\partial \ddot{x}_i}$, $(ii)~ \frac{\partial F_i}{\partial {x}_j}-\frac{\partial F_j}{\partial {x}_i}=\frac{1}{2}\frac{d}{dt}\left(\frac{\partial F_i}{\partial \dot{x}_j}-\frac{\partial F_j}{\partial \dot{x}_i}\right)$ and $(iii)~ \frac{\partial F_i}{\partial \dot{x}_j}+\frac{\partial F_j}{\partial \dot{x}_i}=2\frac{d}{dt}\left(\frac{\partial F_j}{\partial \ddot{x}_i}\right)$, $\forall ~i,j=1,~2,~3~...~ n$.

For a single variable the conditions $(i)$ and $(ii)$ are identically satisfied and relation $(iii)$ simplifies to $\frac{\partial F}{\partial \dot{x}_i}=\frac{d}{dt}\left( \frac{dF}{d\ddot{x}}\right)$. Note that Helmholtz conditions do not yield the Lagrangian of the system, only thing one can say is that the differential equations may be the E-L equations for some Lagrangian. A natural question that immediately arises is that whether Helmholtz conditions are unique for Euler-Lagrange equations.

  We shall illustrate it by two examples: \\
  1. The Euler-Lagrange equation is   $\ddot{q}f(\dot{q},q) + g(\dot{q},q)=0$.
  From H's condition\\
  $~~~~~~~~~~~~~\ddot{q}\frac{\partial f}{\partial{\dot{q}}} + \frac{\partial g}{\partial{\dot{q}}}=\frac{d}{dt}{\{f(q,\dot{q})} \}$\\
  $~~~~~~~~\Rightarrow \ddot{q}\frac{\partial f}{\partial{\dot{q}}} + \frac{\partial g}{\partial{\dot{q}}}=\frac{\partial f}{\partial q} \dot{q}+ \frac{\partial f}{\partial \dot{q}} \ddot{q}$\\
  $~~~~~~~~\Rightarrow  \frac{\partial g}{\partial{\dot{q}}}=\frac{\partial f}{\partial q} \dot{q}$
    
$\hspace{-0.7 cm}$ 2. The E-L equation is  $\ddot{q}f(q)+g(\dot{q},q)=0$. From the H's condition: $\frac{\partial g}{\partial \dot{q}}=\frac{d (f(q))}{dt}=\dot{q}\frac{\partial f}{\partial q}$\\
$i.e., \frac{\partial g}{\partial{\dot{q}}}=\frac{\partial f}{\partial q} \dot{q}$.

Thus we see that though the E-L equations are distinct in the above two examples still the H's condition is identical for the above diff. equations to be Euler-Lagrange equations. Hence Helmholtz conditions are not unique to identify the Euler-Lagrange equations.

We shall now try to determine the possible Lagrangian for examples 1 and 2.

  \textbf{Example 1:} suppose $L=\frac{\dot{q}^2}{2}h(q) +\chi(\dot{q},q)$ \\
$\frac{\partial L}{\partial{\dot{q}}}=\dot{q}h(q)+ \frac{\partial \chi}{\partial \dot{q}}~,~\frac{\partial L}{\partial q}=\frac{\dot{q}^2}{2}h'(q)+\frac{\partial \chi}{\partial q}$

$\therefore$ The E-L equation is \\
$\frac{d}{dt}\left(\frac{\partial L}{\partial \dot{q}}\right)-\frac{\partial L}{\partial q}=0 ~~\Rightarrow \ddot{q}h(q)+\dot{q}^2h'(q) +\ddot{q} \frac{\partial^2\chi}{\partial \dot{q}^2}+\dot{q}\frac{\partial^2\chi}{\partial q \partial \dot{q} }- \frac{\dot{q}^2}{2}h'(q)-\frac{\partial \chi}{\partial q}=0$ \\
$\Rightarrow \ddot{q} \{ h(q)+\frac{\partial^2 \chi }{\partial {\ddot{q}^2}}\} +\frac{\dot{q}^2}{2}h'(q)+\dot{q}\frac{\partial^2 \chi}{\partial q \partial \dot{q}} -\frac{\partial \chi}{\partial q}=0$

Now comparing with E-L equation\\
$f(q,\dot{q})=h(q)+\frac{\partial^2 \chi}{\partial \dot{q}^2}~~....~(a)$ \\ $g(q,\dot{q})=\frac{\dot{q}^2}{2}h'(q)+\dot{q}\frac{\partial^2 \chi}{\partial q \partial \dot{q}}-\frac{\partial \chi}{\partial q}~~....~(b)$

\vspace{1 cm}
 \hspace{-0.6 cm } From (b),\\     $\frac{\partial g}{\partial \dot{q}}=\dot{q}h'(q)+\frac{\partial^2 \chi}{\partial q\partial \dot{q}}+\dot{q}\frac{\partial^3 \chi}{\partial q\partial \dot{q}^2}-\frac{\partial^2 \chi}{\partial q\partial \dot{q}}$\\
$~\hspace{0.5 cm}=\dot{q}h'(q)+\dot{q}\frac{\partial^3 \chi}{\partial q\partial \dot{q}^2}$\\

 \hspace{-0.6 cm } Also    $~~~\dot{q}\frac{\partial f}{\partial q}=\dot{q}h'(q)+\dot{q}\frac{\partial^3 \chi}{\partial q\partial \dot{q}^2}$\\
 $\bullet$ Thus, $\frac{\partial g}{\partial \dot{q}}=\dot{q} \frac{\partial f}{\partial q}~~\rightarrow$ The Helmholtz condition is identically satisfied.
 
\hspace{-0.65 cm} $\bullet$ One may note that eqs. (a) and (b) can not determine $h$ and $\chi$.

\hspace{-0.65 cm} $\bullet$ If one assumes that $\frac{\partial \chi}{\partial q}$ is a hom. fn. of $\dot{q}$ of degree $n$, then\\
$\dot{q}\frac{\partial}{\partial \dot{q}}\left( \frac{\partial \chi}{\partial q}\right)=n\frac{\partial \chi}{\partial q}$\\
 \hspace{-0.6 cm } From (b), $g=\frac{\dot{q}^2}{2}h'(q)+(n-1)\frac{\partial \chi}{\partial q}$\\
  \hspace{-0.6 cm } So for $n=1, ~~g=\frac{\dot{q}^2}{2}h'(q),~~~~h(q)=\int\frac{2g}{\dot{q}^2} dq$. Then $\chi$ can be determined from (a).
  
  \textbf{Example 2:} Suppose $L=\frac{\dot{q}^2}{2}f(q)+h(q)$\\
  Then, $\frac{\partial L}{\partial \dot{q}}=\dot{q}f(q)$ , $\frac{\partial L}{\partial q}=\frac{\dot{q}^2}{2}f'(q)+h'(q)$\\
  So, the Euler-Lagrange eq. is \\
  
  $\frac{d}{dt}\left(\frac{\partial L}{\partial \dot{q}}\right)-\frac{\partial L}{\partial q}=0 \Rightarrow \ddot{q}f(q)+\dot q^2f'(q)-\frac{1}{2}\dot{q}^2f'(q)-h'(q)=0$\\
 $~~~\hspace{3.2 cm} \Rightarrow~\ddot{q}f(q)+\frac{1}{2}\dot{q}^2f'(q)-h'(q)=0 $ \\
 Using H's condition $\Rightarrow \ddot{q}f(q)+\frac{1}{2} \dot{q}\frac{\partial g}{\partial q}-h'(q)=0$\\
 Comparing with the E-L eq. in the example, $g(q, \dot{q})=\frac{1}{2}\dot{q}\frac{\partial g}{\partial \dot{q}}-h'(q)$\\
 $\Rightarrow h(q)=\bigintss\left[g(q, \dot{q})- \frac{1}{2}\dot{q}\frac{\partial g}{\partial \dot{q}}  \right]dq=-\left(\frac{n-2}{2}\right)\bigintsss g(q, \dot{q}) dq,~$   if $g$ is a hom. fn. of $\dot{q}$ of degree $n$.  
  In particular, if $n=2$, then $h(q)=0$.
  
  We shall now discuss some typical form of the Lagrangian and the corresponding E-L eqs.

  \textbf{Example 3:}  $L(x,\dot{x})=\frac{1}{2}\dot{x}^2f(x)+g(x)$, then the E-L eq. is: $f(x)\ddot{x}+\frac{1}{2}\dot{x}^2f'(x)-g'(x)=0$.
  
  \textbf{Example 4:} $L(x,\dot{x})=\frac{1}{2}\dot{x}^2f(x)+\dot{x}h(x)+g(x)$, here the E-L eq. is: $f(x)\ddot{x}+\frac{1}{2}\dot{x}^2f'(x)-g'(x)=0$ \\
  Note: for both the Lagrangians in Ex.-3 and Ex.-4 the E-L eq. is identical. So, Lagrangian for a physical system is note unique.
  
  \textbf{Example 5:} $L=\dot{x}^nf(x)+g(x)$\\
  Here, $\frac{\partial L}{\partial \dot{x}}=n\dot{x}^{n-1}f(x),~\frac{\partial L}{\partial x}=\dot{x}^{n}f'(x)+g'(x)$\\
  So, the E-L eq. is $\frac{d }{dt}(n\dot{x}^{n-1}f(x))-\dot{x}^{n}f'(x)-g'(x)=0$\\
  $\Rightarrow n(n-1)\dot{x}^{n-2}\ddot{x}f(x)+n\dot{x}^nf'(x)-\dot{x}^nf'(x)-g'(x)=0$
  
  
    \textbf{Example 6:} $L=\dot{x}^nf(x)+\dot{x}^mh(x)+g(x)$\\
    $\frac{\partial L}{\partial \dot{x}}=n\dot{x}^{n-1}f(x)+m\dot{x}^{m-1}h(x), \frac{\partial L}{\partial x}=\dot{x}^{n}f'(x)+\dot{x}^{m}h'(x)+g'(x) $\\
    The Euler-Lagrange eq. is:\\ $n(n-1)\dot{x}^{n-2}\ddot{x}f(x)+m(m-1)\dot{x}^{m-2}h(x)\ddot{x}+n\dot{x}^{n}f'(x)+m\dot{x}^{m}h'(x)-\dot{x}^{n}f'(x)-\dot{x}^{m}h'(x)-g'(x)=0 $\\
    $\Rightarrow \{n(n-1)\dot{x}^{n-2} f(x)+m(m-1)\dot{x}^{m-2} h(x)\}\ddot{x}+(n-1)\dot{x}^nf'(x)+(m-1)\dot{x}^mh(x)-g'(x)=0$
    
    \textbf{Note:} (a) A second order differential equation is a Euler-Lagrange equation corresponding to a Lagrangian if it contains second order differential in linear form.\\
    (b) If the first order derivative  is of degree $n$ in the diff. eq. then the corresponding Lagrangian has $\dot{q}$ of at most degree $(n+1)$.\\
    Conversely, if the Lagrangian has $\dot{q}$ of degree $n$, then the corresponding E-L equation contains $\dot{q}$ at most of degree $n$ \\
    (c) The Lagrangian $L=L(q^{\alpha},\dot{q}^{\alpha},t)$ and the E-L eqs. are: $\frac{d}{dt}\left(\frac{\partial L}{\partial \dot{q}_{\alpha}}\right)-\frac{\partial L}{\partial q_{\alpha}}=0$,\\
    Then, $dL=\frac{\partial L}{\partial q^{\alpha}}dq^{\alpha}+\frac{\partial L}{\partial \dot{q}^{\alpha}}d\dot{q}^{\alpha}=\frac{d}{dt}(\frac{\partial L}{\partial \dot{q}^{\alpha}})dq^{\alpha}+\frac{\partial L}{\partial \dot{q}^{\alpha}} \frac{d}{dt}(dq^{\alpha})=\frac{d}{dt}(\frac{\partial L}{\partial \dot{q}^{\alpha}}dq^{\alpha})$\\
    Now from the E-L equation:\\
    $\lambda^{\alpha}\frac{d}{dt}(\frac{\partial L}{\partial \dot{q}^{\alpha}})-\lambda^{\alpha}\frac{\partial L}{\partial q^{\alpha}}=0$\\
    $\Rightarrow \frac{d}{dt}(\lambda^{\alpha}\frac{\partial L}{\partial \dot{q}^{\alpha}})-\lambda^{\alpha}\frac{\partial L}{\partial q^{\alpha}}=\left(\frac{d \lambda^{\alpha}}{dt~}\right)\frac{\partial L}{\partial \dot{q}^{\alpha}}-\lambda^{\alpha}\frac{\partial L}{\partial q^{\alpha}}+\frac{d}{dt}(\lambda^{\alpha}\frac{\partial L}{\partial \dot{q}^{\alpha}})=0$\\
    $\Rightarrow \lambda^{\alpha}\frac{\partial L}{\partial q^{\alpha} } +\left(\frac{d \lambda^{\alpha}}{dt~}\right)\frac{\partial L}{\partial \dot{q}^{\alpha}}=\frac{d}{dt}(\lambda^{\alpha}\frac{\partial L}{\partial \dot{q}^{\alpha}})$\\
    $\Rightarrow \vec{X}L=\frac{d}{dt}(\lambda^{\alpha}\frac{\partial L}{\partial \dot{q}^{\alpha}})$, where $\vec{X}=\lambda^{\alpha}\frac{\partial}{\partial q^{\alpha}}+ \left(\frac{d \lambda^{\alpha}}{dt~}\right)\frac{\partial }{\partial \dot{q}^{\alpha}}$. 
    
    If $Q=\lambda^{\alpha}\frac{\partial L}{\partial \dot{q}^{\alpha}}$ is conserved i.e., $\frac{dQ}{dt}=0$, then, $\vec{X}$ is a Noether symmetry vector of the physical system and Q is termed as Noether charge.


{\Large{\textbf{{\underline{Non-Uniqueness of the Lagrangian :}}}}}
 
The Lagrangian of a given system is not unique. A Lagrangian $L$ can be multiplied by a non-zero constant $'a'$ and shift by an arbitrary constant $'b'$ and the new Lagrangian $L'=aL+b$ will describe the same motion as $L$. If one restricts the trajectories $\overrightarrow{q}$ over given time interval $[t_{in},t_{fin}]$ and fixed end points $P_{in}=q(t_{st})$ and $P_{fin}=q(t_{fin}),$ then two Lagrangians describing the same system can differ by total time derivative of a function $f(\overrightarrow{q},t)$ i.e.,
\begin{equation}
	L'(\overrightarrow{q},\dot{\overrightarrow{q}},t)=L+\frac{d}{dt}f(\overrightarrow{q},t)\nonumber
\end{equation}
As 
\begin{equation}
	\frac{df}{dt}=\frac{\partial{f}}{\partial{t}}+\sum\frac{\partial{f}}{\partial{q_i}}\dot{q_i}~~,\nonumber
\end{equation}
so
\begin{eqnarray}
	S'[\overrightarrow{q}]&=&\int_{t_{in}}^{t_{fin}}L'(\overrightarrow{q},\dot{\overrightarrow{q}},t)dt=\int_{t_{in}}^{t_{fin}}L(\overrightarrow{q},\dot{\overrightarrow{q}},t)dt+\int_{t_{in}}^{t_{fin}}\frac{df}{dt}dt\nonumber\\
	&=&S[\overrightarrow{q}]+f(P_{fin},t_{fin})-f(P_{in},t_{in})\nonumber
\end{eqnarray}
As the 2nd and 3rd terms are independent of $q$ so both the actions $S$ and $S'$  have same equations of motion i.e., both the Lagrangian have the same equations of motion.

{\Large{\textbf{\underline{Invariance under point transformation :}}}}

Let us consider a point transformation $q_{\alpha}=q_{\alpha}(s_{\beta})$ in the configuration space. $\dot{q_{\alpha}}=\frac{\partial{q_{\alpha}}}{\partial{s_{\beta}}}\dot{s_{\beta}}$.
Then 
\begin{eqnarray}
	\frac{\partial{L}}{\partial{s_{\beta}}}&=&\frac{\partial{L}}{\partial{q_{\alpha}}}\frac{\partial{q_{\alpha}}}{\partial{s_{\beta}}},~~~\frac{\partial{L}}{\partial{\dot{s_{\beta}}}}=\frac{\partial{L}}{\partial{\dot{q_{\alpha}}}}\frac{\partial{\dot{q_{\alpha}}}}{\partial{\dot{s_{\beta}}}}=\frac{\partial{L}}{\partial{\dot{q_{\alpha}}}}\frac{\partial{q_{\alpha}}}{\partial{s_{\beta}}}\nonumber\\
	&\implies&\frac{d}{dt}\left(\frac{\partial{L}}{\partial{\dot{s_{\beta}}}}\right)-\frac{\partial{L}}{\partial{{s_{\beta}}}}=0\nonumber\\
	&\implies&\frac{d}{dt}\left(\frac{\partial{L}}{\partial{\dot{q_{\alpha}}}}\frac{\partial{q_{\alpha}}}{\partial{s_{\beta}}}\right)-\frac{\partial{L}}{\partial{{q_{\alpha}}}}\frac{\partial{q_{\alpha}}}{\partial{s_{\beta}}}=0\nonumber\\
	&\implies&\frac{\partial{q_{\alpha}}}{\partial{s_{\beta}}}\frac{d}{dt}\left(\frac{\partial{L}}{\partial{\dot{q_{\alpha}}}}\right)-\frac{\partial{q_{\alpha}}}{\partial{s_{\beta}}}\frac{\partial{L}}{\partial{{q_{\alpha}}}}=0\nonumber\\
	&\implies&\frac{d}{dt}\left(\frac{\partial{L}}{\partial{\dot{q_{\alpha}}}}\right)-\frac{\partial{L}}{\partial{{q_{\alpha}}}}=0\nonumber\\
	&\implies&\mbox{Lagrangian's equations of motion is invariant under point transformation}\nonumber\\ &&\mbox{in configuration space.}\nonumber
\end{eqnarray}

{\Large{\textbf{{A comparison between Newtonian Mechanics and Lagrangian mechanics :}}}}

In Newtonian mechanics, the time varying constraint forces which keep the particle in the constrained motion are solved by Newton's law of motion. In particular, If the size and shape of a massive object is negligible then it can be treated as a point particle. Thus for a system  of $N$ such point particles having masses $m_1,~m_2,...,m_N$ and position vectors $\overrightarrow{r_1},~\overrightarrow{r_1},\overrightarrow{r_2},...,\overrightarrow{r_N}$ (with $\overrightarrow{r_k}=(x_k,y_k,z_k)$) then the Newtonian equations of motion i.e., Newton's second law are given by
\begin{equation}
	\Sigma{m_i}\frac{d^2\overrightarrow{r_i}}{dt^2}=\Sigma\overrightarrow{F_i}
\end{equation}

These are $3N$ number of second order differential equations. In Lagrangian formulation, instead of forces the energies of the system is used. The primary object is the Lagrangian function which characterized the state of a physical system or the dynamics of the system. The Lagrangian function has the dimension of energy and in mechanics it is just the difference between the K.E.(energy of motion) and the potential energy (energy of position) i.e.,$L=T-V$.

In Lagrangian mechanics, a convenient set of independent variables (known as generalised coordinates) are chosen to characterised the possible motion of the system (i.e., particles). In Lagrangian formulation, the constraint forces do not appear into the system of equations and the number of equations will be less as the influence of the constraints on the particles is not taken into account. The equations of motion in Lagrangian formulation are obtained from the action principle which states that the action functional of the system derivative from $L$ must remain at a stationary point (namely a maxima/a minimum/a saddle) throughout the time evolution of the system.

As the K.E. is the energy of the system's motion, so it is a function only of the velocities $\overrightarrow{v_k}$ but not of the positions $\overrightarrow{r_k}$ and time $t$ i.e.,$T=T(\overrightarrow{v_1},\overrightarrow{v_2},...,\overrightarrow{v_N}).$ For example for a system of point particles $T=\frac{1}{2}\sum_{k=1}^{N}m_kv_k^2.$ The P.E. of a physical system measures the amount of energy to one particle due to all the others and others external influences. However, for conservative system the P.E. is a function vectors of the particles only i.e.,$V=V(\overrightarrow{r_1},\overrightarrow{r_2},...,\overrightarrow{r_N})$ while for non-conservative forces (which can be derived from an appropriate potential (e.g. electromagnetic potential)) the P.E. is a function of both position and velocity i.e.,$V=V(\overrightarrow{r_1},\overrightarrow{r_2},...,\overrightarrow{r_N},\overrightarrow{V_1},\overrightarrow{V_2},...,\overrightarrow{V_N})$. Further, due to presence of external forces the above potential function also depends on time i.e., $V=V(\overrightarrow{r_k},\overrightarrow{V_k},t),k=1,2,...,N$.

\vspace{0.5 cm}
{\Large{\textbf{{\underline{Comments :}}}}}


\begin{itemize}
	\item For dissipative forces another function must be introduced along side $L$.
	
	\item The above form of $L$ does not hold in relativistic lagragian mechanics and it must be replaced by a function consistent with special or general relativity.
	
	\item For holonomic constraint system, the constrained equations determine the allowed path the particle can move along but where they are or how fast they go at every instant of time. Lagrangian mechanics is applicable to systems whose constraints are holonmic in nature.
	
	\item Lagrange formulation is not applicable to system having non-holonomic constraints which depend on the particle velocities, accelerations or higher derivatives of position.
	
	\item If both $T$ and $V$ have explicit time dependence then $L$ will have explicit time dependence. On the other hand, if both of them do not have explicit time dependence then,  $L$  will also have no explicit time dependence but has implicit dependence on time through the generalized co-ordinates.
	
	\item The Lagrangian equation of motion :
	\begin{equation}
		\frac{d}{dt}\left(\frac{\partial{L}}{\partial{\dot{q_j}}}\right)-\frac{\partial{L}}{\partial{{q_j}}}\nonumber
	\end{equation} 
	are second order differential equations in the generalized co-ordinates. These equations do not include constraint forces at all, only non-constraint forces are taken into account in Lagrangian formulation.
	
\end{itemize}



{\Large{\textbf{{Lagrange's equation of motion in curved space-time :}}
{\Large{\textbf{{{A path from Newtonian to Lagrangian mechanics }}}}}}}

For a point particle Newton's law of motion is $\overrightarrow{F}=m\overrightarrow{a}$ or in component form : $F^{\mu}=ma^{\mu}$. In Euclidean geometry, cartesian co-ordinate system is commonly used and Newton's law is very suitable in cartesian co-ordinate system. For any curvilinear co-ordinate system or in curved space time one can extend the idea as follows : in curved space time the acceleration vector is defined as 

\begin{eqnarray}
	a^{\mu}&=&\frac{\delta{v^{\mu}}}{d\tau}=(\nabla_{\alpha}v^{\mu})\frac{d{x^{\alpha}}}{d\tau}=\left(\frac{\partial{v^{\mu}}}{\partial{x^{\alpha}}}+\Gamma^{\mu}_{\alpha\beta} v^{\beta}\right)\frac{d{x^{\alpha}}}{d\tau}\nonumber\\
	&=&\frac{dv^{\mu}}{d\tau}+\Gamma^{\mu}_{\alpha\beta}v^{\alpha}v^{\beta}=\frac{d^2{x^{\alpha}}}{d\tau^2}+\Gamma^{\mu}_{\alpha\beta}\frac{d{x^{\alpha}}}{d\tau}\frac{d{x^{\beta}}}{d\tau}\nonumber
\end{eqnarray}


So the Newton's laws of motion becomes 
\begin{eqnarray}
	F^{\mu}=m\left(\frac{d^2x^{\mu}}{d\tau^2}+\Gamma^{\mu}_{\alpha\beta}\frac{dx^{\alpha}}{d\tau}\frac{dx^{\beta}}{d\tau}\right)\nonumber
\end{eqnarray}
Here $\Gamma^{\mu}_{\alpha\beta}$ is the christoffel symbol of second kind in curved space-time.

Now we shall show that the above laws of motion can be derived from the point-like Lagrangian  

\begin{equation}
	L=T-V=\frac{1}{2}mg_{bc}\frac{dx^b}{d\tau}\frac{dx^c}{d\tau}-V(x)\nonumber
\end{equation}
Now, 
\begin{equation}
	\frac{\partial L}{\partial\dot{x}^l}=\frac{\partial{T}}{\partial\dot{x}^l}=mg_{lc}\dot{x}^c\nonumber
\end{equation}
\begin{eqnarray}
	\frac{d}{dt}\left(\frac{\partial L}{\partial\dot{x}^l}\right)&=&m\frac{\partial{g_{lc}}}{\partial{x^k}}\dot{x^k}\dot{x^c}+mg_{lc}\ddot{x}^c\nonumber\\
	&=&\frac{m}{2}\left\{\frac{\partial{g_{lc}}}{\partial{x^k}}\dot{x^k}\dot{x^c}+\frac{\partial{g_{lk}}}{\partial{x^c}}\dot{x^c}\dot{x^k}\right\}+mg_{lc}\ddot{x}^c\nonumber\\
	\frac{\partial L}{\partial{x}^l}&=&\frac{m}{2}\frac{\partial{g_{lc}}}{\partial{x^l}}\dot{x^l}\dot{x^c}\nonumber
\end{eqnarray}
From Euler-Lagrange equations
\begin{eqnarray}
	g^{al}\left\{\frac{d}{dt}\left(\frac{\partial{T}}{\partial{x^l}}\right)-\frac{\partial{T}}{\partial{x^l}}\right\}=g^{al}\left(-\frac{\partial{V}}{\partial{x^l}}\right)&&\nonumber\\
	\implies{g^{al}}\left\{mg_{lc}\ddot{x}+\frac{m}{2}\left[\frac{\partial{g_{lc}}}{\partial{x^k}}\dot{x^k}\dot{x^c}+\frac{\partial{g_{lk}}}{\partial{x^c}}\dot{x^c}\dot{x^k}\right]\right\}-\frac{\partial{g_{bc}}}{\partial{x^l}}\dot{x}^b\dot{x}^c&=&g^{al}F_{l}\nonumber\\
	\implies{m}\delta_{c}^{a}\ddot{x}^c+mg^{al}\Gamma_{kcl}\dot{x}^k\dot{x}^c&=&F^a\nonumber\\
	\implies{m}(\ddot{x}^a+\Gamma^{a}_{kc}\dot{x}^k\dot{x}^c)&=&F^a\nonumber
\end{eqnarray}
Thus in curved space-time the general form of Lagrangian as 
\begin{equation}
	L=\frac{m}{2}g_{\alpha\beta}\frac{dx^{\alpha}}{d\tau}\frac{dx^{\beta}}{d\tau}-V(x)\nonumber
\end{equation}
and the generalized form of the Newton's law of motion is 
\begin{equation}
	m\left(\frac{d^2x^{\alpha}}{d\tau^2}+\Gamma_{\beta\delta}^{\alpha}\frac{dx^{\alpha}}{d\tau}\frac{dx^{\beta}}{d\tau}\right)=F^{\alpha}\nonumber
\end{equation}
It is to be noted that the above equation of motion is nothing but the Euler-Lagrange equation corresponding to the above Lagrangian.



















































































































































































































\iffalse




\section{Principle of Linear Momentum  : Conservation of Linear Momentum: } 

If a physical system is under the action of a system of externally applied forces and subject to bilateral constraints only, the rate of change of linear momentum of the system is equal to the vector sum of all the externally applied forces acting on the system.

Suppose we have a physical system imposed of $N$ particles. Let the system is under the action of a system of externally applied forces and subject to bilateral constraints only. Let $P_i$ be the position of the $i$-th particle having mass $m_i$ and position vector $\overrightarrow{r_i}$ with respect to a fixed point $O~i.e.~\overrightarrow{OP}=\overrightarrow{r_i}$. Suppose $\overrightarrow{F_i}$ be the resultant applied forces acting on the $i$-th particle. According to D'Alembert principle


\begin{equation}
\sum_{i=1}^{N}\left\{\overrightarrow{F_i}+(-m_i\ddot{\overrightarrow{r_i}})\right\}=0~~(\mbox{basis principle of statics})\nonumber
\end{equation}
\begin{eqnarray}
\mbox{i.e.,}~\sum{m_i\ddot{\overrightarrow{r_i}}}&=&\sum_{i}\overrightarrow{F_i}\nonumber\\
\mbox{i.e.,}~\frac{d}{dt}\left(\sum{m_i\dot{\overrightarrow{r_i}}}\right)&=&\sum_{i}\overrightarrow{F_i}\nonumber
\end{eqnarray}
Thus the rate of change of linear momentum of a material system in a given direction is equal to the vector sum of all the external forces applied to the system in that direction.

In particular, if $\sum\overrightarrow{F_i}=0$ then $\frac{d}{dt}\left(\sum{m_i\dot{\overrightarrow{r_i}}}\right)=0$ 

i.e., $\sum{m_i\dot{\overrightarrow{r_i}}}=$ Constant.

So the total linear momentum of a material system is conserved. This is known as conservation of linear momentum. 



\section{Principle of Angular Momentum  : Conservation of Angular Momentum: }	

If a physical system is under the action of a system of externally applied forces and subjected to bilateral constrains only, then the rate of change of angular momentum (i.e. moment of momentum) of the system in a given direction is equal to the vector sum of the moments of all the externally applied forces acting on the system in that direction.

Form the basic principle of statics if a system of external forces are in equilibrium then the sum of the moments of all forces about any point in space is equal to zero.

Thus taking moment about O
\begin{eqnarray}
\sum\overrightarrow{r_i}\times\left\{\overrightarrow{F_i}+(-m_i\ddot{\overrightarrow{r_i}})\right\}&=&0\nonumber\\
\mbox{i.e.,~}~\sum\overrightarrow{r_i}\times{m_i\ddot{\overrightarrow{r_i}}}&=&\sum_{i}\overrightarrow{r_i}\times{\overrightarrow{F_i}}\nonumber\\
\mbox{i.e.,~}~\frac{d}{dt}\left(\overrightarrow{r_i}\times{m_i\dot{\overrightarrow{r_i}}}\right)&=&\sum\overrightarrow{r_i}\times{\overrightarrow{F_i}}\nonumber\\
\mbox{i.e.,~}~\frac{d\overrightarrow{H}}{dt}&=&\sum\overrightarrow{r_i}\times{\overrightarrow{F_i}}=\overrightarrow{L}\nonumber
\end{eqnarray}
where $\overrightarrow{H}=\overrightarrow{r_i}\times{m_i\dot{\overrightarrow{r_i}}}$ is the angular momentum of the material system in a given direction and $\overrightarrow{L}$ is the vector sum of the moments of all externally applied forces acting on the system in that direction.

Now, if $\overrightarrow{L}=0$ then $\overrightarrow{L}=$Constant i.e., total angular momentum in that direction of the system is conserved. This is known as conservation of angular momentum.

\section{Euler's Dynamical Equation:}

Suppose a physical system rotate about O which is fixed with respect to the system as well as the space. Consider a rectangular coordinate system $ox,~oy,~oz$ through O. These axes are fixed with respect to the system. Let $\overrightarrow{i},~\overrightarrow{j}$ and $\overrightarrow{k}$ be the unit vector along these coordinate axes. Then $\overrightarrow{r}=x\overrightarrow{i}+y\overrightarrow{j}+z\overrightarrow{z}$ with $(x,y,z)$ be the coordinate of the point $P$ with respect to the coordinate system $ox.~oy,~ox$ which is fixed in the system . Now,
\begin{eqnarray}
\dot{\overrightarrow{r}}=\frac{d\overrightarrow{r}}{dt}&=&\frac{\partial{\overrightarrow{r}}}{\partial{t}}+\overrightarrow{\omega}\times{\overrightarrow{r}}\nonumber\\
&=&\frac{\partial}{\partial{t}}(x\overrightarrow{i}+y\overrightarrow{j}+z\overrightarrow{z})+\overrightarrow{\omega}\times{\overrightarrow{r}}\nonumber\\
&=&\dot{x}\overrightarrow{i}+\dot{y}\overrightarrow{j}+\dot{z}\overrightarrow{z}+\overrightarrow{\omega}\times{\overrightarrow{r}}\nonumber\\
&=&\overrightarrow{\omega}\times{\overrightarrow{r}}\nonumber
\end{eqnarray}

($\because{\dot{x}=\dot{y}=\dot{z}=0}$ as the axes are fixed in the rigid body)

Suppose $\overrightarrow{\omega}=\omega_1\overrightarrow{i}+\omega_2\overrightarrow{j}+\omega_3\overrightarrow{k}$ be the angular velocity of rotation through O. Thus
\begin{eqnarray}
\overrightarrow{H}=\sum\overrightarrow{r}\times({m_i\dot{\overrightarrow{r}}})&=&\sum{m}\overrightarrow{r}\times(\overrightarrow{\omega}\times{\overrightarrow{r}})\nonumber\\
&=&\sum m\left[(\overrightarrow{r}.\overrightarrow{r})\overrightarrow{\omega}-(\overrightarrow{r}.\overrightarrow{\omega})\overrightarrow{r}\right]\nonumber\\
&=&\sum{m}\Big{[}(x^2+y^2+z^2)(\omega_1\overrightarrow{i}+\omega_2\overrightarrow{j}+\omega_3\overrightarrow{k})\nonumber \\
&& -(x\omega_1+y\omega_2+z\omega_3)(x\overrightarrow{i}+y\overrightarrow{j}+z\overrightarrow{k})\Big{]}\nonumber
\end{eqnarray}

\begin{equation}
H=\sum\Big{[}\omega_1\{(y^2+z^2)\overrightarrow{i}-xy\overrightarrow{j}-xz\overrightarrow{k}\}+\omega_2\{(x^2+z^2)\overrightarrow{j}-xy\overrightarrow{i}-yz\overrightarrow{k}\} \nonumber \\
 +\omega_3\{(y^2+x^2)\overrightarrow{k}-xz\overrightarrow{i}-yz\overrightarrow{j}\}\Big{]}\nonumber
\end{equation}
As $\omega_1,~\omega_2,~\omega_3$ are same for all the particles of the material system so
\begin{eqnarray}
\overrightarrow{H}&=&\omega_1\left\{\left[\sum{m}(y^2+z^2)\right]\overrightarrow{i}-\left[\sum{mxy}\right]\overrightarrow{j}-\left[\sum{mxz}\right]\overrightarrow{k}\right\}\nonumber\\
&+&\omega_2\left\{\left[\sum{m}(x^2+z^2)\right]\overrightarrow{j}-\left[\sum{mxy}\right]\overrightarrow{i}-\left[\sum{myz}\right]\overrightarrow{k}\right\}\nonumber\\
&+&\omega_1\left\{\left[\sum{m}(y^2+x^2)\right]\overrightarrow{k}-\left[\sum{mxz}\right]\overrightarrow{i}-\left[\sum{mzy}\right]\overrightarrow{j}\right\}\nonumber
\end{eqnarray}
Now, if $A,~B,~C$ are the moments of inertia and $D,~E,~F$ are the product of inertia of the physical system with respect to the coordinate system $ox,~oy,~oz$ through O, then $A=\sum{m}(y^2+z^2),~B=\sum{m}(x^2+z^2),~C=\sum{m}(x^2+y^2),~D=\sum{mzy},~E=\sum{mzx},~F=\sum{mxy}$. Thus $\overrightarrow{H}=\omega_1\left(A\overrightarrow{i}-F\overrightarrow{j}-E\overrightarrow{k}\right)+\omega_2\left(B\overrightarrow{j}-F\overrightarrow{i}-D\overrightarrow{k}\right)+\omega_3\left(C\overrightarrow{k}-E\overrightarrow{i}-D\overrightarrow{j}\right)$.

Further, if the coordinate axes are the principal axes of inertia then $D=E=F=0$ and we get $\overrightarrow{H}=\omega_1A\overrightarrow{i}+\omega_2B\overrightarrow{j}+\omega_3C\overrightarrow{k}$

$\therefore\frac{dH}{dt}=\frac{\partial{\overrightarrow{H}}}{\partial{t}}+\overrightarrow{\omega}\times\overrightarrow{H}=A\dot{\omega}_1\overrightarrow{i}+B\dot{\omega}_2\overrightarrow{j}+C\dot{\omega}_3\overrightarrow{k}+\begin{vmatrix}
\overrightarrow{i} & \overrightarrow{j} & \overrightarrow{k}\\
\omega_1 & \omega_2 & \omega_3\\
H_1 & H_2 & H_3\\
\end{vmatrix}$

Also if $\overrightarrow{L}=L_1\overrightarrow{i}+L_2\overrightarrow{j}+L_3\overrightarrow{k}$, then
\begin{eqnarray}
L_1=A\dot{\omega}_1-(B-C)\omega_2\omega_3\nonumber\\
L_2=B\dot{\omega}_2-(C-A)\omega_3\omega_1\nonumber\\
L_3=C\dot{\omega}_3-(A-B)\omega_1\omega_2\nonumber
\end{eqnarray}
These are known as Euler's dynamical equations. Kinetic energy of a physical system with respect to a rotating coordinate system which is fixed with respect to the system can be derived as follows:
\begin{eqnarray}
T=\frac{1}{2}\sum{m}\left|\frac{d\overrightarrow{r}}{dt}\right|^2=\frac{1}{2}\sum{m}\frac{d\overrightarrow{r}}{dt}.\frac{d\overrightarrow{r}}{dt}\nonumber\\
\dot{\overrightarrow{r}}=\frac{d\overrightarrow{r}}{dt}=\frac{\partial{\overrightarrow{r}}}{\partial{t}}+\overrightarrow{\omega}\times{\overrightarrow{r}}=\overrightarrow{\omega}\times{\overrightarrow{r}}\nonumber
\end{eqnarray}
(as the axes are fixed in the system)
\begin{eqnarray}
\therefore{T}&=&\frac{1}{2}\sum{m}(\overrightarrow{\omega}\times\overrightarrow{r}).\frac{d\overrightarrow{r}}{dt}\nonumber\\
&=&\frac{1}{2}\sum{m}\overrightarrow{\omega}.\left(\overrightarrow{r}\times\frac{d\overrightarrow{r}}{dt}\right)\nonumber\\
&=&\frac{1}{2}\sum{m}\overrightarrow{\omega}.\left\{\overrightarrow{r}\times(\overrightarrow{\omega}\times\overrightarrow{r})\right\}\nonumber\\
&=&\frac{1}{2}\sum{m}\overrightarrow{\omega}.\left[(\overrightarrow{r}.\overrightarrow{r})\overrightarrow{\omega}-(\overrightarrow{r}.\overrightarrow{\omega})\overrightarrow{r}\right]\nonumber\\
&=&\frac{1}{2}\sum{m}\left[(\overrightarrow{r}.\overrightarrow{r})(\overrightarrow{\omega}.\overrightarrow{\omega})-(\overrightarrow{r}.\overrightarrow{\omega})(\overrightarrow{\omega}.\overrightarrow{r})\right]\nonumber\\
&=&\frac{1}{2}\sum{m}\left[(x^2+y^2+z^2)(\omega_1^2+\omega_2^2+\omega_3^2)-(x\omega_1+y\omega_2+z\omega_3)^2\right]\nonumber\\
&=&\frac{1}{2}\sum{m}\left[\omega_1^2(y^2+z^2)+\omega_2^2(x^2+z^2)+\omega_3^2(y^2+x^2)-2\omega_1\omega_2xy-2\omega_2\omega_3yz-2\omega_3\omega_1xz\right]\nonumber
\end{eqnarray}
As $\omega_1,~\omega_2$ and $\omega_3$ are same for all points so 
\begin{eqnarray}
{T}&=&\frac{1}{2}\Big{[}\omega_1^2\sum{m}(y^2+z^2)+\omega_2^2\sum{m}(x^2+z^2)+\omega_3^2\sum{m}(y^2+x^2)-2\omega_1\omega_2\sum{m}xy \nonumber \\
&-& 2\omega_2\omega_3\sum{m}zy\Big{]}
-\omega_1\omega_3\sum{m}xz\nonumber\\
&=&\frac{1}{2}\left[A\omega_1^2+B\omega_2^2+C\omega_3^2-2\omega_1\omega_2F-2\omega_2\omega_3D-2\omega_1\omega_3E\right]\nonumber
\end{eqnarray}
Further, if the axes are principal axes then
\begin{equation}
{T}=\frac{1}{2}\left[A\omega_1^2+B\omega_2^2+C\omega_3^2\right]\nonumber
\end{equation}


\fi



 \section{Appendix IV: Symplectic structures on Manifolds}
 Let, $M^{2n}$ be an even dimensional differentiable manifold. A symplectic structure on $M^{2n}$ is a closed non degenerate differential 2-form $\omega^{2}$ such that $d\omega^{2}=0$ and $\forall \xi\neq0$, there exists $\eta:\omega^{2}(\xi,\eta)\neq 0, \xi,\eta\in TM_{x}$. The pair $(M^{2n}, \omega^{2})$ is called a symplectic manifold.
 \begin{itemize}
 	\item A symplectic structure on a manifold is a closed non-degenerate differential 2-form.
 	\item On a symplectic manifold (on Riemannian manifold), there is a natural isomorphism between vector fields and 1-forms. 
 	\item A vector field on a symplectic manifold corresponding to the differential of a function is called a Hamiltonian vector field.
 	\item A vector field on a manifold determines a phase flow, a one-parameter group of diffeomorphisms.
 	\item The phase flow of a Hamiltonian vector field (hvf) on a symplectic manifold preserves the symplectic structure of phase space.
 	\item The vector fields on a manifold form a Lie algebra. The hvf on a symplectic manifold also form a Lie algebra. The operation in this algebra is called the Poisson bracket.
 \end{itemize}
\textbf{Note:} In $R^{2n}$ with coordinates $(q^{i},p^{i})$, $\omega^{2}=\sum_{i}dp_{i}\wedge dq_{i}$. In $R^{2}$ the pair $(R^{2},\omega^{2})$ is the pair: (the plane, area).\\ \\
\textbf{Cotangent Bundle}
Let $V$ be an $n$-dimensional differentiable manifold. A $1$-form on the tangent space to $V$ at a point $x$ is called a cotangent vector to $V$ at $x$. The set of all cotangent vectors to $V$ at $x$ form an $n$-D vector space, dual to the tangent space $TV_{x}$. This vector space (VS) of cotangent vectors is denoted by $T^{*}V_{x}$ and call it the cotangent space to V at $x$.

	The union of the cotangent spaces to the manifold at all of its points is called the cotangent bundle of V and is denoted by $T^{*}V$. It has a natural structure of a differentiable manifold of dimension $2n$.
	
	An element of $T^{*}V$ is a $1$-form on the tangent space to $V$ at some point of $V$. If $q$ is a choice of $n$ local coordinates for points in $V$, then such a form is given by its $n$ components $p$. Together, the $2n$ numbers $(p,q)$ form a collection of local coordinates for points in $T^{*}V$. There is a natural projection $f:T^{*}V\rightarrow V$ (i.e, sending every one-form on $T^{*}V$ to the point $x$). This projection $f$ is differentiable and surjective. The pre-image of a point $x\in V$ under $f$ is the cotangent space $T^{*}V_{x}$.\\
	\textbf{Note:} In Lagrangian mechanics if $V$ be the configuration manifold and $L$ be the Lagrangian function then the Lagrangian generalized velocity $\dot{q}$ is a tangent vector to $V$ and the generalized momentum $p=\dfrac{\partial L}{\partial \dot{q}}$ is a cotangent vector. Then the $``p,q"$ phase space of the Lagrangian system is the cotangent bundle of the configuration manifold. The phase space has a natural symplectic manifold structure. \\
	\textbf{Note:} A Riemannian structure on a manifold establishes an isomorphism between the spaces of tangent vectors and $1$-forms. A symplectic structure establishes a similar isomorphism.\\
	\textbf{Note} To each vectors $\xi$, tangent to a symplectic manifold $(M^{2n}, \omega^{2})$ at the point $x$, one can associate a $1$-form $\omega_{\xi}^{1}$ on $TM_{x}$ by the relation: $\omega_{\xi}^{1}(\eta)=\omega^{2}(\eta,\xi),$ for all $\eta \in TM_{x}$. This isomorphism $T^{*}M_{x}\rightarrow TM_{x}$ is denoted by $I$. Suppose, $H$ be a function on a symplectic manifold $M^{2n}$. Then $dH$ is a differential $1$-form on $M$. At every point there is a tangent vector to $M$ associated to it. We denote this vector field as $IdH$ on $M$. This vector field $IdH$ is called a Hamiltonian vector field and $H$ is called the Hamiltonian function.\\
	Example- If $M^{2n}=R^{2n}=\{(p,q)\}$ then the velocity vector field of Hamilton's canonical equations:\\
	$\dot{x}=IdH(x)\implies \dot{p}=-\dfrac{\partial H}{\partial q}$ and $\dot{q}=\dfrac{\partial H}{\partial p}$.\\
	Definition: A Lie algebra is a vector space $L$, together with a bi-linear skew-symmetric operation $L\times L\rightarrow L$ which satisfies the Jacobi identity.\\
	Vector field and differential operators: Let $M$ be a smooth manifold and $\vec{A}$ is a smooth vector field on $M$; at every point $x\in M$ we are given a tangent vector $\vec{A}(x)\in TM_{x}$. Then with such vector field we have:\\
	One parameter group of diffeomorphisms or flow: Let $A^{t}: M\rightarrow M$ for which $\vec{A}$ is the velocity vector field i.e, $\dfrac{d}{dt}|_{t=0}A^{t}x=\vec{A}(x)$. Then $A^{t}$ is called a one parameter group of diffeomorphisms or flow $M\rightarrow M$. Let $L_{A}$ denotes a first order differential operator denoting  differentiation along the direction of the field $\vec{A}$ then for any function $\phi:M\rightarrow R$, the derivative of $\phi$ in the direction of $\vec{A}$ is a new function $L_{A}\phi$ whose value at a point $x$ is $(L_{A}\phi)(x)=\dfrac{d}{dt}|_{t=0} \phi(A^{t}x)$.\\
	The Poisson bracket of vector fields: Let $\vec{A}$ and $\vec{B}$ are two given vector fields on a manifold $M$. The corresponding flows $\vec{A^{t}}$ and $\vec{B^{t}}$ do not commute in general i.e, $\vec{A^{t}}B^{s}\neq B^{s}\vec{A^{t}}$. The Poisson bracket or commutator of two vector fields $\vec{A}$ and $\vec{B}$ on a manifold $M$ is the vector field $\vec{C}$ for which $L_{c}=L_{B}L_{A}-L_{A}L_{B}$ i.e, $\vec{C}=[\vec{A},\vec{B}]$.\\
	The Poisson/ commutator bracket makes the vector space of vector fields on a manifold $M$ into a Lie algebra as they satisfy the Jacobi identity.\\
	\textbf{Note:} To measure the degree of non-commutativity of the two flows $A^{t}$ and $B^{s}$ we consider the points $A^{t}B^{s}x$ and $B^{s}A^{t}x$. To estimate the difference between these points we compare the value of some smooth function $\phi$ on the manifold $M$ at those two points. The difference $\Delta (t;s;x)=\phi(A^{t}B^{s}x)-\phi(B^{s}A^{t}x)$ is a differentiable function which is zero for $s=0$ and for $t=0$. Note that $\Delta(t;s;x)|_{t=0}=0$, $\Delta(t;s;x)|_{s=0}=0$. So the Taylor series in $s$ and $t$ at $(0,0)$ contains $st$ and the other terms of second order vanish. This principal bi-linear term of $\Delta$ at $(0,0)$ gives the commutator $[L_{A},L_{B}](x)|_{(0,0)}$. \\
	\textbf{Note:} The Hamiltonian vector fields on a symplectic manifold form a sub algebra of the Lie algebra of al fields. The Hamiltonian functions also form a Lie algebra. The operation in this algebra is called the Poisson bracket of functions.
	\section{Appendix V:  Application of differential geometry to General Mechanics}
	The space of all generalized co-ordinates is termed as configuration space. The fiber bundle with the configuration space as the base manifold and fibers along the velocity field is called a tangent bundle or tangent space. Similarly, the fiber bundle with base manifold as the configuration space and fibers along the momenta co-vectors is termed as co-tangent bundle or cotangent space. If there are $n$ independent generalized co-ordinates then the configuration manifold is of dimension $n$ while both tangent and co-tangent spaces are of dimension $2n$.
	
	Let $\mathcal{L}=\mathcal{L}(q,\dot{q})$ is defined over tangent bundle while $\mathcal{H}=\mathcal{H}(q,p)$ is defined over co-tangent space or phase space. In phase space one can define a simplectic 2-form as
	\begin{equation}
		 \underline{\omega}=\underbar{d}q\wedge \underbar{d}p=\underline{d}q\bigotimes\underline{p}-\underline{d}p\bigotimes\underline{d}q\nonumber
		\end{equation}
\textbf{Result}: In phase-space, there exists a vector field $\vec{V}$ in the phase-space for which $\mathcal{L}_{\vec{V}}\underline{\omega}=0$. This vector field identifies the dynamical path.\\
	\textbf{Proof}: As, $\mathcal{L}_{\vec{V}}~\underline{\omega}=0$ i.e, $\underline{d}~[\underline{\omega}(\vec{V})]+(\underline{d}~\underline{w})~(\vec{V})=0$. From the very definition $\underline{d}~\underline{\omega}=0$ and hence $\mathcal{L}_{\vec{V}}~\underline{\omega}=0\implies \underline{d}~[\underline{\omega}(\vec{V})]=0$. Now as $\vec{V}=\dfrac{d}{dt}=\dot{q}\dfrac{\partial}{\partial q}+\dot{p}\dfrac{\partial}{\partial p}$, so $\underline{\omega}(\vec{V})=\dot{q}\underline{d}p-\dot{p}\underline{d}q=\dfrac{\partial H}{\partial p}\underline{d}p+\dfrac{\partial H}{\partial q}\underline{d}q\implies \underline{d}[\underline{\omega}(\vec{V})]=\dfrac{\partial^{2}H}{\partial q \partial p}\underline{d}q\wedge\underline{d}p+\dfrac{\partial^{2}H}{\partial p\partial q}\underline{d}p\wedge \underline{d}q=0$. On the other hand, if $\underline{d}[\underline{\omega}(\vec{V})]=0$, then $\underline{\omega}(\vec{V})=\underline {d}H$ for some function $H=H(q,p)$ in the phase space. Thus, $\dot{q}\underline{d}p-\dot{p}\underline{d}q=\dfrac{\partial H}{\partial q}\underline{q}+\dfrac{\partial H}{\partial p}\underline{p}$ i.e, $\dot{q}=\dfrac{\partial H}{\partial p}$ and $\dot{p}=-\dfrac{\partial H}{\partial q}$, the Hamilton's equation of motion. Thus a vector field $\vec{V}$ in phase-space (i.e, co-tangent space) which satisfies $\mathcal{L}_{\vec{V}}\underline{\omega}=0$ is called a Hamiltonian vector field. Now, a natural question arises: can we have a Hamiltonian vector field corresponding to a function $f=f(q,p)$ in the phase-space. 	The answer is as follows:\\
	If $\vec{X_{f}}$ be the Hamiltonian vector field for the function $f$ then it is defined as $\underline{\omega}(\vec{X_{f}})=\underline{d}f$. Suppose, $\vec{X_{f}}=a\dfrac{\partial}{\partial q}+b\dfrac{\partial}{\partial p}$, then the above relation gives $a~\underline{d}p-b~\underline{d}q=\dfrac{\partial f}{\partial p}\underline{d}p+\dfrac{\partial f}{\partial q}\underline{d}q$. Thus, comparing the coefficients of $\underline{d}p$ and $\underline{d}q$ we have, $a=\dfrac{\partial f}{\partial p}$ and $b=-\dfrac{\partial f}{\partial q}$. Hence, the Hamiltonian vector field corresponding to the function $f$ is $\vec{X_{f}}=\dfrac{\partial f}{\partial p}\dfrac{\partial}{\partial q}-\dfrac{\partial f}{\partial q}\dfrac{\partial}{\partial p}$. 
		Note: If $\vec{v}$ be a Hamiltonian vector field, then $\vec{v}$ is tangent to the solution curves in phase-space and $\mathcal{L}_{\vec{v}}H=0$. Also, we can say that the system is conservative in nature.\\ \\
	\textbf{Canonical Transformation:} A transformation from  $(q,p)\rightarrow(Q,P)$ in phase space is said to be canonical if it leaves the Hamiltonian's equation of motion to be invariant. In differential geometric notion, a canonical transformation is defined as that transformation in phase space which leaves the $2$-form $\underline{\omega}$ to be invariant. Thus $\underline{\omega}=\underline{d}q\wedge\underline{d}p=\underline{d}Q\wedge\underline{d}P$ with $Q=Q(q,p)$ and $P=P(q,p)$. So, $\underline{d}Q=\dfrac{\partial Q}{\partial q}\underline{d}q+\dfrac{\partial Q}{\partial p}\underline{d}p$ and $\underline{d}P=\dfrac{\partial P}{\partial q}\underline{d}q+\dfrac{\partial P}{\partial p}\underline{d}p$. So, $\underline{d}Q\wedge\underline{d}P=\left(\dfrac{\partial Q}{\partial q}\dfrac{\partial P}{\partial p}-\dfrac{\partial Q}{\partial p}\dfrac{\partial P}{\partial q}\right)\underline{d}q\wedge \underline{d}p$. Hence, $\dfrac{\partial Q}{\partial q}\dfrac{\partial P}{\partial p}-\dfrac{\partial Q}{\partial p}\dfrac{\partial P}{\partial q}=1$, is the necessary and sufficient condition for canonical transformation. A trivial choice is $Q=p$ and $P=-q$. Now suppose, $p=p(q,Q)$ and $P=P(q,Q)$ then $\underline{d}p=\dfrac{\partial p}{\partial q}\underline{d}q+\dfrac{\partial P}{\partial Q}\underline{d}Q$ and $\underline{d}P=\dfrac{\partial P}{\partial q}\underline{d}q+\dfrac{\partial P}{\partial Q}\underline{d}Q$. So, $\underline{d}q\wedge \underline{d}p=\dfrac{\partial p}{\partial Q}\underline{d}q\wedge \underline{d}Q$ and $\underline{d}Q\wedge \underline{d}P=-\dfrac{\partial P}{\partial q}\underline{d}q\wedge\underline{d}Q$. Thus, for Canonical Transformation $\dfrac{\partial F_{1}}{\partial Q}=-P$ and $\dfrac{\partial F_{1}}{\partial q}=p$ then the above condition for Canonical Transformation is automatically satisfied. Hence $F_{1}$ is called a generating function for Canonical Transformation. There are three other types of generating functions namely $F_{2}(q,P)$, $F_{3}(p,Q)$ and $F_{4}(p,P)$. Thus, four types of generating functions are possible for Canonical Transformation.\\ \\
	\textbf{Poisson Bracket:} Let $f=f(q,p)$ and $g=g(q,p)$ are functions in phase-space. Suppose, $\vec{X_{f}}$ and $\vec{X_{g}}$ be the corresponding Hamiltonian vector fields. Then Poisson bracket (PB) between two functions is defined as 
	\begin{eqnarray}
		\{f,g\}=\underline{\omega}(\vec{X_{f}},\vec{X_{g}})=<\underline{d}f,\vec{X_{g}}>
	=<\dfrac{\partial f}{\partial q}\underline{d}q+\dfrac{\partial f}{\partial p}\underline{d}p,~\dfrac{\partial g}{\partial p}\dfrac{\partial}{\partial q}-\dfrac{\partial g}{\partial q}\dfrac{\partial}{\partial p}>\nonumber\\
		=\dfrac{\partial f}{\partial q}\dfrac{\partial g}{\partial p}-\dfrac{\partial f}{\partial p}\dfrac{\partial g}{\partial q}\nonumber
	\end{eqnarray}
Note: Poisson Bracket is independent of the choice of co-ordinates in phase-space, it depends only on $\underline{\omega}$.\\
Properties:
\begin{enumerate}
	\item The Poisson Bracket satisfies the Jacobi identity:\\
	$\{f,\{g,h\}\}+\{g,\{h,f\}\}+\{h,\{f,g\}\}=0$, for any $C^{2}$ functions $f,~g$ and $h$. For proof see page 175.
	\item $\vec{X_{f}}(g)=-\dfrac{\partial f}{\partial q}\dfrac{\partial g}{\partial p}+\dfrac{\partial f}{\partial p}\dfrac{\partial g}{\partial q}=\{g,f\}$. If $f=H$, then $\vec{X_{H}}(f)=\dfrac{df}{dt}$.
	\item $[\vec{X_{f}},\vec{X_{g}}]=-\vec{X}_{\{f,g\}}$.\\
	Proof: As $\{f,h\}=-\vec{X_{f}}(h)$, so from Jacobi's identity \\
	$\{f,\{g,h\}\}+\{g,\{h,f\}\}=\{f,-\vec{X_{g}}(h)\}+\{g,\vec{X_{f}}(h)\}=\vec{X_{f}}\vec{X_{g}}(h)-\vec{X_{g}}\vec{X_{f}}(h)=[\vec{X_{f}},\vec{X_{g}}](h)$. So, $\{h,\{f,g\}\}=\vec{X}_{\{f,g\}}(h)$. Thus, $\vec{X}_{\{f,g\}}(h)=-[\vec{X_{f}},\vec{X_{g}}](h)$, for arbitrary $h$ i.e, $\vec{X}_{\{f,g\}}=-[\vec{X_{f}},\vec{X_{g}}]$.
\end{enumerate}
\textbf{Note}: The above result shows that if $\vec{X_{f}}$ and $\vec{X_{g}}$ are Hamiltonian vector fields then $\vec{X}_{\{f,g\}}$ is also a Hamiltonian vector field. So, the set of all Hamiltonian vector fields form a Lie algebra.
	\section{Appendix VI: Rigid body motion}
	\textbf{d'Alembert's principle: Virtual displacement and virtual work}
	In a mechanical system, virtual work arises as an application to the principle of least action. As an application of a force on a particle, the work done changes with the change in displacement. Now, among all possible displacements, the one which minimize the action is called the virtual displacement and the corresponding work on the particle is termed as virtual work.
	
	An important concept in mechanics is the principle of least constraint (by Gauss). It is essentially a variational formulation of classical mechanics. It states that the acceleration of a constrained physical system should be similar as possible to that of the corresponding unconstrained system.
	
	The principle of virtual work basically corresponds to equilibrium configuration (i.e, static situation) and it states that the total virtual work done by the applied forces in a physical system is zero. Due to Newton's second law of motion, in equilibrium state the algebraic sum of the applied forces is equal and opposite to the algebraic sum of the reactions/ constraint forces of the system and hence the algebraic sum of the virtual works done by the constraint forces should be zero. 
	
	d'Alembert's principle is a generalization of the principle of virtual work from static to dynamical system. By introducing the notion of forces of inertia to the applied forces in a physical system result in dynamic equilibrium.
	
	d'Alembert's principle also known as Lagrange-d'Alembert principle is a statement of the fundamental classical laws of motion. It generalizes the principle of virtual work from static to dynamical systems by introducing forces of inertia which when added to the applied forces in a system result in dynamic equilibrium.
	
\textbf{Note:} This principle can be applied to kinetic constraints which depend on velocities. However, this principle is not applicable for irreversible displacements.

Mathematically, Newton's second law of motion can be written as 
\begin{equation}
	\sum \vec{F}^{I}-\sum \dot{\vec{p_{i}}}=0
\end{equation}
where $F_{i}^{I}$ denotes the total forces acting on the i-th particle, $\vec{p_{i}}$ is the momentum of the $i$-th particle and $\dot{\vec{p_{i}}}$ is termed as forces of inertial or the effective force on the $i$-th particle. Thus, Newton's second law of motion states that the algebraic sum of the total forces on the particles and the algebraic sum of the reverse effective forces (or the reverse forces of inertia) makes the system in a dynamic equilibrium. Consequently, the mathematical statement of the d'Alembert's principle can be written as
\begin{equation}
	\sum (F^{I}_{i}-\dot{\vec{p_{i}}})~\delta\vec{r_{i}}=0
\end{equation}
Here $\delta\vec{r_{i}}$ is the virtual displacement of the i-th particle, consistent with the constraints. As by definition $\vec{p_{i}}=m_{i}\vec{v_{i}}$ so the above statement can be written as
\begin{equation}
	\sum \left(F_{i}^{I}-m_{i}\dot{\vec{v_{i}}}-\dot{m_{i}}\vec{v_{i}}\right)\delta\vec{r_{i}}=0\label{eq7.9}
\end{equation} where $m_{i}$ is the mass of the i-th partcile, $\vec{v_{i}}$ is the velocity vector of the i-th particle and $\vec{a_{i}}=\dot{\vec{v_{i}}}$ measures the acceleration of the i-th particle, Now, splitting the total forces as the algebraic sum of the applied forces and the constraint forces i.e,
\begin{equation}
	\vec{F_{i}^{I}}=\vec{F_{i}}+\vec{c_{i}}
\end{equation}
then equation (\ref{eq7.9}) has the explicit form as
\begin{equation}
	\sum \vec{F_{i}}\delta\vec{r_{i}}+\sum \vec{c_{i}}\vec{\delta r_{i}}-\sum m_{i}(\vec{a_{i}}\delta \vec{r_{i}})=0
\end{equation}
where masses of the system of particle are assumed to be constants. Further, if the arbitrary displacements are restricted to orthogonal to the corresponding constraint forces i.e, $\sum \vec{c_{i}}\delta\vec{r_{i}}=0$ then the above equation simplifies to
\begin{equation}
	\sum(\vec{F_{i}}-m_{i}\vec{a_{i}})\delta \vec{r_{i}}=0\label{eq7.99}
\end{equation}
The above special displacements are termed as constraint consistent displacements and the relation (\ref{eq7.99}) is termed as d' Alembert's principle. Also, it is known as principle of virtual work for applied forces. \\
\textbf{Rigid body motion:}
Now due to rotation if $\vec{\omega}$ be the angular velocity then the linear velocity can be written as 
\begin{equation}
	\vec{v}=\vec{\omega}\times \vec{r}
\end{equation}
So for a particle of mass $m$, the angular momentum is given by
\begin{equation}
	\vec{L}=\vec{r}\times m\vec{v}=m\left(\vec{r}\times (\vec{\omega}\times \vec{r})\right)
\end{equation}
Hence for a rotating rigid body about an axis through a point O of the rigid body, the total angular momentum is given by
\begin{equation}
	\vec{M}=\sum_{	i}m_{i}\left(\vec{r_{i}}\times (\vec{\Omega}\times \vec{r_{i}})\right)
\end{equation}
with $\vec{\Omega}$, the angular velocity about the axis through O. The above angular momentum equation can be written as an operator equation of the form 
\begin{equation}
	A\vec{\Omega}=\vec{M}
\end{equation}
i.e, there exists a linear operator $A$ which operating on $\vec{\Omega}$ gives the angular momentum vector. Now for any two vectors $\vec{X}$ and $\vec{Y}$ one has the inner product
\begin{eqnarray}
	(A\vec{X},\vec{Y})=
	=\sum_{	i}m_{i}[(\vec{r_{i}}\times (\vec{X}\times \vec{r_{i}})).\vec{Y}]\nonumber\\
	=-\sum_{i}m_{i}[((\vec{X}\times \vec{r_{i}})\times \vec{r_{i}}).\vec{Y}]\nonumber\\
	=-\sum_{i}m_{i}(\vec{X}\times \vec{r_{i}}).(\vec{r_{i}}\times \vec{Y})\nonumber\\
	=\sum_{	i}m_{i}(\vec{X}\times \vec{r_{i}}).(\vec{Y}\times \vec{r_{i}})
\end{eqnarray}
The last line shows that the above inner product is symmetric in $\vec{X}$ and $\vec{Y}$ and hence $A$ is a symmetric linear operator. Now choosing, $\vec{X}=\vec{Y}=\vec{\Omega}$,
\begin{eqnarray}
	\left(A\vec{\Omega}, \vec{\Omega}\right)=\sum_{	i}m_{i}(\vec{\Omega}\times \vec{r_{i}}).(\vec{\Omega}\times \vec{r_{i}})\nonumber\\
	=\sum_{	i}m_{i}(\vec{\Omega}\times \vec{r_{i}})^{2}=\sum_{	i}m_{i}v_{i}^{2}
\end{eqnarray}
Thus, the K.E of the rigid body can be written as
\begin{equation}
	T=\dfrac{1}{2}\sum m_{i}v_{i}^{2}=\dfrac{1}{2}\sum m_{i}(\vec{\Omega}\times \vec{r_{i}})^{2}=\dfrac{1}{2}	\left(A\vec{\Omega}, \vec{\Omega}\right)=\dfrac{1}{2}\left(\vec{M}, \vec{\Omega}\right)
\end{equation}
The above symmetric operator $A$ is called the \underline{inertia operation} of the rigid body.
Now due to symmetric nature of the linear operator $A$ there are three mutually orthogonal characteristic directions $(\vec{e_{1}},\vec{e_{2}},\vec{e_{3}})$ along which the eigen values are $I_{i}~(i=1,2,3)$ i.e, 
\begin{eqnarray}
	A\vec{e_{i}}=I_{i}\vec{e_{i}}, ~i=1,2,3\nonumber\\
	A\vec{\Omega}.\vec{e_{i}}=I_{i}\vec{\Omega}.\vec{e_{i}}\nonumber\\
	\vec{M}\vec{e_{i}}=M_{i}(\hat{\Omega}.\vec{e_{i}}),~M_{i}=I_{i}|\vec{\Omega}|
\end{eqnarray}
So, the K.E $T$ has the expression 
\begin{equation}
	T=\dfrac{1}{2}(I_{1}\Omega_1^{2}+I_{2}\Omega_2^{2}+I_{3}\Omega_{3}^{2})
\end{equation}
\textbf{Note:}
\begin{enumerate}
	\item The eigen directions $\vec{e_{i}}$ are the principal axes of the rigid body about O.
	\item The eigen values $I_{i}$ of the inertial operator $A$ are the moments of inertia of the rigid body w.r.t the principal axes $\vec{e_{i}}$.
\end{enumerate}
In general, suppose the rigid body rotates about an axis $\vec{e}$, the unit vector along the axis of rotation. Then, $\vec{\Omega}=\Omega_{e}\vec{e}$ and $\vec{v_{i}}=\vec{\Omega}\times \vec{r_{ie}}=\Omega_{e}.(\vec{e}\times \vec{r_{ie}})=\Omega_{e}|\vec{r_{ie}}|\tilde{e_{l}}$ where $\tilde{e_{l}}$ is the unit vector perpendicular to $\vec{e}$ and $\vec{r_{ie}}$. Now, the Moment of inertia of the rigid body about $\vec{e}$ axis is $I_{e}=\sum m_{i}|\vec{r_{ie}}|^{2}$. Hence the total angular momentum of the rigid body is 
\begin{eqnarray}
	\vec{M}=\sum_{	i}m_{i}\left(\vec{r_{ie}}\times (\vec{\Omega}\times \vec{r_{ie}})\right)=\sum_{	i}m_{i}\left(\vec{r_{ie}}\times (\Omega_{e}\vec{e}\times \vec{r_{ie}})\right)\nonumber\\
	=\sum_{	i}m_{i}\left(|\vec{r_{i}}|^{2}\Omega_{e}.\vec{e}\right)\nonumber\\
	=\left(\sum m_{i}|\vec{r_{ie}}|^{2}\right)\Omega_{e}\vec{e}=I_{e}.\Omega_{e}\vec{e}
\end{eqnarray}
Suppose, $\vec{\Omega_{e}}=\dfrac{\vec{e}}{\sqrt{I_{e}}}$ then $\vec{M}=\sqrt{I_{e}}\vec{e}$ and $T=\dfrac{1}{2}I_{e}\Omega_{e}^{2}=\dfrac{1}{2}$. But, $T=\dfrac{1}{2}(A\vec{\Omega},\vec{\omega})=\dfrac{1}{2}\implies(A\vec{\Omega},\vec{\Omega})=1$ i.e, $I_{1}\Omega_1^{2}+I_{2}\Omega_2^{2}+I_{3}\Omega_{3}^{2}=1$, an ellipsoid. This ellipsoid consists of those angular velocity vector $\vec{\Omega}$ whose K.E  is $\dfrac{1}{2}$. In particular, this ellipsoid: $\{\vec{\Omega}: (A\vec{\Omega},\vec{\Omega})=1\}$ is called the inertia  ellipsoid of the rigid body about O. Here, the principal axes of the inertia ellipsoid are directed along the principal axes of the rigid body about O and their lengths are inversely proportional to $\pm\sqrt{I_{i}}$. 

\textbf{Note:} Suppose a rigid body is stretched out along a direction, then MI w.r.t this axis is small and consequently the inertia ellipsoid is also stretched out along this axis. So the inertia may resemble the shape of the body.

\textbf{Euler's dynamical equations:}
Let, $\vec{M}=A\vec{\Omega}$ be the angular momentum vector of the rigid body. Suppose, the motion of the rigid body is considered around a stationary point O and $\vec{M}$ be the angular momentum of the body relative to O in the body. Then,
\begin{equation}
	\dfrac{d\vec{M}}{dt}=\dfrac{\partial \vec{M}}{\partial t}+\vec{\Omega}\times \vec{M}=0
\end{equation}
i.e, $\dfrac{\partial \vec{M}}{\partial t}=\vec{M}\times \vec{\Omega}$. This is known as Euler's equations. Suppose, $\vec{M}=M_{1}\vec{e_{1}}+M_{2}\vec{e_{2}}+M_{3}\vec{e_{3}}$ and $\vec{\Omega}=\Omega_1\vec{e_{1}}+\Omega_2\vec{e_{2}}+\Omega_{3}\vec{e_{3}}$ be the decomposition of $\vec{M}$ and $\vec{\Omega}$ about the principal axes at O. The components of these two vectors are related as $M_{i}=I_{i}\Omega_{i}$, $i=1,2,3$ where $I_{i}$ is the moment of inertia about the principal axis $\vec{e_{i}}$. Now, in component form the Euler's equations are
\begin{eqnarray}
	\dfrac{\partial M_{1}}{\partial t}=M_{2}\Omega_{3}-M_{3}\Omega_2=M_{2}\dfrac{M_{3}}{I_{3}}-M_{3}\dfrac{M_{2}}{I_{2}}=\dfrac{(I_{2}-I_{3})}{I_{2}I_{3}}M_{2}M_{3}\label{eq7.24}
\end{eqnarray}
$\dfrac{\partial M_{1}}{\partial t}=a_{1}M_{2}M_{3}$, $a_{1}=\dfrac{(I_{2}-I_{3})}{I_{2}I_{3}}$. The above Euler's equation can also be written as 
\begin{equation}
	I_{1}\dfrac{\partial \Omega_1}{\partial t}=(I_{2}-I_{3})\Omega_2\Omega_{3}
\end{equation}
From equation (\ref{eq7.24}), 
\begin{equation}
	M_{1}\dfrac{\partial M_{1}}{\partial t}+M_{2}\dfrac{\partial M_{2}}{\partial t}+M_{3}\dfrac{\partial M_{3}}{\partial t}=0
\end{equation}
i.e, $M_{1}^{2}+M_{2}^{2}+M_{3}^{2}=M^{2}$, a constant. Thus, total angular momentum is conserved. Similarly, 
\begin{eqnarray}
	\dfrac{M_{1}}{I_{1}}\dfrac{\partial M_{1}}{\partial t}+\dfrac{M_{2}}{\partial t}+\dfrac{M_{3}}{I_{3}}\dfrac{\partial M_{3}}{\partial t}=0\nonumber\\
	\implies \dfrac{M_{1}^{2}}{I_{1}}+\dfrac{M_{2}^{2}}{I_{2}}+\dfrac{M_{3}^{2}}{I_{3}}=2E
\end{eqnarray}
This equation is nothing but the conservation of energy. Note that the above equations are nothing but the first integrals of the Euler's equations.  From the above conservation of angular momentum and conservation of energy it is clear that the angular momentum vector $\vec{M}$ lies in the intersection of an ellipsoid and a sphere. To study, the structure of the curves of intersection, at first fix the ellipsoid for a given $E>0$ and change the radius $M$ of the sphere. Suppose, $I_{1}>I_{2}>I_{3}$. Then semi-axes of the ellipsoid will be $\sqrt{2EI_{1}}>\sqrt{2EI_{2}}>\sqrt{2EI_{3}}$. Now, the following cases will arise:
\begin{enumerate}
	\item $M<\sqrt{2EI_{3}}$ or $M>\sqrt{2EI_{1}}$. In this case, the sphere and the ellipsoid do not intersect. So, no rigid body motion is possible corresponding to such values of $M$ and $E$.
	\item $M=\sqrt{2EI_{3}}/\sqrt{2EI_{1}}$. Here the sphere and the ellipsoid intersect at two points at the end of the smallest/ largest semi axes.
	\item $\sqrt{2EI_{3}}<M<\sqrt{2EI_{2}}$. There are two curves around the ends of the smallest semi axes along which the motion is possible.
	\item $M=\sqrt{2EI_{2}}$. The intersection consists of two circles with centre at the two ends of the semi-axes of the middle principal axes.
	\item $\sqrt{2EI_{2}}<M<\sqrt{2EI_{1}}$. This case is same as 3.
	\item $M=\sqrt{2EI_{1}}$. This cone is same as case 2. 
\end{enumerate}
\textbf{Observations:}
\begin{enumerate}
	\item Each of the six end points of the semi-axes of the ellipsoid is a separate trajectory of the Euler equations. These trajectories correspond to stationary position of $\vec{M}$. So, they correspond to fixed values of the angular velocity vector directed along one of the principal axes $\vec{e_{i}}$. So, along these motions $\vec{\Omega}$ is collinear with $\vec{M}$. The body simply rotates with fixed angular velocity around the principal axis of inertia $\vec{e_{i}}$, which is stationary in space. (The motion of a rigid body under which its angular velocity remains constant is called a stationary rotation.)
	\item A rigid body fixed at a point O admits a stationary rotation around any of the three principal axes $\vec{e_{1}},~\vec{e_{2}}$ and $\vec{e_{3}}$.
	\item If, $I_{1}>I_{2}>I_{3}$, then the r.h.s of the Euler equation do not become zero anywhere else. So, there are no other stationary rotations.
	\item The stationary solutions $\vec{M_{1}}=M_{1}\vec{e_{1}},~\vec{M_{3}}=M_{3}\vec{e_{3}}$ of the Euler equations corresponding to the largest and smallest principal axes are stable, while the solution with $\vec{M_{2}}=M_{2}\vec{e_{2}}$ is unstable.
	\item The motion of the angular momentum and angular velocity vectors in a rigid body ($\vec{M}$ and $\vec{\Omega}$) will be periodic if $M\neq \sqrt{2EI_{i}},~i=1,2,3.$
\end{enumerate}