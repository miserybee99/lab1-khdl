%\pagenumbering{arabic}









\chapter{Action principle $\&$ consequences}

\section{Hamilton's Principle Function}
Let, $$S=\int_{t_0}^{t_1}(T-V)dt.$$
Suppose a given problem is solved. `$S$' may be regarded as a function of $t_0$, $t_1$, $a_1$, $a_2$,...,$a_n$ and $\dot{a_1}$, $\dot{a_2}$,...,$\dot{a_n}$, where $a$'s and $\dot{a}$'s are the initial values of $q$'s and $\dot{q}$'s at $t=t_0$. The co-ordinates $q_1$, $q_2$,..., $q_n$ at time $t$ may be regarded as a function of the same quantities so that we may consider that we have eliminated $\dot{a_1}$, $\dot{a_2}$,...,$\dot{a_n}$ and regard `$S$' as a function of $t_0$, $t_1$, $a_1$, $a_2$,...,$a_n$, $q_1$, $q_2$,..., $q_n$. `$S$' thus expressed is called Hamilton's principle function.




\section{Hamilton-Jacobi partial differential equation}
Let us consider $$S=\int_{t_0}^{t}(T-V)dt=\int_{t_0}^{t}Ldt$$
The value of the Lagrangian function $L$ on the varied path is $L+\delta L$.

\begin{figure}[h!]
	\centering
	\includegraphics[scale=0.4]{photo1.pdf}
\end{figure}

Now, 
\begin{eqnarray}
\Delta S=\Delta\int_{t_0}^{t}Ldt&=&\int_{t_0}^{t+\Delta t}(L+\delta L)dt-\int_{t_0}^{t}Ldt\nonumber\\
&=&\int_{t_0}^{t}(L+\delta L)dt+\int_{t}^{t+\Delta t}(L+\delta L)dt-\int_{t_0}^{t}Ldt\nonumber\\
&=&\int_{t}^{t+\Delta t}(L+\delta L)dt+\int_{t_0}^{t}\delta Ldt\nonumber\\
&=&L.\Delta t+\int_{t_0}^{t}\sum\left(\frac{\partial L}{\partial q_i}\delta q_i+\frac{\partial L}{\partial\dot{q_i}}\delta\dot{q_i}\right)dt\nonumber\\
&~&~~~~~~~~~~~~~~~~~~~~~~~~~~~~~~~~~~~~~~~~~\mbox{(by mean value theorem)}\nonumber\\
&=&L.\Delta t+\int_{t_0}^{t}\sum\left\{\frac{\partial L}{\partial q_i}-\frac{d}{dt}\left(\frac{\partial L}{\partial\dot{q_i}}\right)\right\}\delta q_idt+\sum\frac{\partial L}{\partial\dot{q_i}}\delta q_i\bigg|^t_{t_0}\nonumber\\
&=&L.\Delta t+\sum\left(\frac{\partial L}{\partial\dot{q_i}}\delta q_i\right)\bigg|^t\nonumber\\
&~&~~~~~(\mbox{using Lagrange's equation of motion and}~~\delta q_i=0~~\mbox{at}~~t=t_0)\nonumber\\
&=&L.\Delta t+\sum\frac{\partial L}{\partial\dot{q_i}}\left(\Delta q_i-\dot{q_i}\Delta t\right)\nonumber\\
&~&~~~~~~~~~~~~~~~~~~~~~~~~~~~~~~~~~~~~~~~~~(\because \Delta q=\delta q+\dot{q}\Delta t)\nonumber\\
&=&\left(L-\sum\frac{\partial L}{\partial\dot{q_i}}\dot{q_i}\right)\Delta t+\sum\frac{\partial L}{\partial\dot{q_i}}\Delta q_i\nonumber
\end{eqnarray}

\begin{equation}
\therefore \Delta S=-H\Delta t+\sum\frac{\partial L}{\partial\dot{q_i}}\Delta q_i\nonumber
\end{equation}

\begin{equation}
\therefore \frac{\partial S}{\partial t}=-H,~~~~~~\frac{\partial S}{\partial q_i}=\frac{\partial L}{\partial\dot{q_i}}=p_i,~~~~~~i=1,2,...n.\nonumber
\end{equation}

\begin{eqnarray}
\frac{\partial S}{\partial t}&=&-H(t,q_1,q_2,...,q_n,p_1,p_2,...,p_n)\nonumber\\
&=&-H\left(t,q_1,q_2,...,q_n,\frac{\partial S}{\partial q_1},\frac{\partial S}{\partial q_2},...,\frac{\partial S}{\partial q_n}\right)\nonumber
\end{eqnarray}
or, $$\frac{\partial S}{\partial t}+H\left(t,q_1,q_2,...,q_n,\frac{\partial S}{\partial q_1},\frac{\partial S}{\partial q_2},...,\frac{\partial S}{\partial q_n}\right)=0$$
This is called the Hamilton-Jacobi partial differential equation.


\vspace{.5cm}


$\bullet$ \textbf{Problem I}: A particle oscillates in a straight line about a centre of force which varies as the distance. Show that the Hamilton's principle function is $$S=\dfrac{\sqrt{\mu}}{2}\left[\dfrac{\left(x_0^2+x^2\right)\cos\left\{\sqrt{\mu}(t-t_0)\right\}-2xx_0}{\sin\left\{\sqrt{\mu}(t-t_0)\right\}}\right]$$.

\vspace{.25cm}

\textbf{Solution:} The equation of motion is $\ddot{x}=-n^2x$.

$\therefore x=A\cos nt+B\sin nt$.

Suppose at $t=t_0$, $x=x_0$, $\dot{x}=\dot{x_0}$.

$\therefore x_0=A\cos nt_0+B\sin nt_0$.

and $\dfrac{\dot{x_0}}{n}=-A\sin nt_0+B\cos nt_0$.

$\therefore A=\left(x_0\cos nt_0-\dfrac{\dot{x_0}}{n}\sin nt_0\right)$, $B=\left(x_0\sin nt_0+\dfrac{\dot{x_0}}{n}\cos nt_0\right)$.

\begin{equation}
\therefore x=x_0\cos n\tau+\dfrac{\dot{x_0}}{n}\sin n\tau,~~~~~~~~~~\tau=t-t_0.\label{eq3..1}
\end{equation}
If $V$ be the potential, then $~~-\dfrac{\partial V}{\partial x}=-n^2x$.

i.e, $V=\dfrac{n^2x^2}{2}$ and $T=\dfrac{1}{2}\dot{x}^2$.

\begin{eqnarray}
V=\frac{1}{2}n^2\left[x_0^2\cos^2n\tau+\frac{\dot{x_0}^2}{n^2}\sin^2n\tau+\frac{2x_0\dot{x_0}}{n}\sin n\tau\cos n\tau\right]\nonumber.\\
T=\frac{1}{2}n^2\left[x_0^2\sin^2n\tau+\frac{\dot{x_0}^2}{n^2}\cos^2n\tau-\frac{2x_0\dot{x_0}}{n}\sin n\tau\cos n\tau\right]\nonumber.
\end{eqnarray} 
\begin{eqnarray}
\therefore S&=&\int_{0}^{t}(T-V)dt=\int_{0}^{\tau}(T-V)d\tau\nonumber\\
&=&\frac{n^2}{2}\int_{0}^{\tau}\left[\frac{\dot{x_0}^2}{n^2}\cos 2n\tau-x_0^2\cos 2n\tau-\frac{2x_0\dot{x_0}}{n}\sin 2n\tau\right]d\tau\nonumber\\
&=&\frac{n^2}{2}\left[\left(\frac{\dot{x_0}^2}{n^2}-x_0^2\right)\left\{\frac{\sin 2n\tau}{2n}\right\}_0^\tau+\frac{2x_0\dot{x_0}}{n}\left\{\frac{\cos 2n\tau}{2n}\right\}_0^\tau\right]\nonumber\\
&=&\frac{n}{4}\left(\frac{\dot{x_0}^2}{n^2}-x_0^2\right)\sin 2n\tau+\frac{x_0\dot{x_0}}{2}(\cos 2n\tau-1)\nonumber
\end{eqnarray}

From (\ref{eq3..1}), $$\frac{\dot{x_0}}{n}=\frac{x-x_0\cos n\tau}{\sin n\tau}$$

\begin{eqnarray}
\therefore S=\frac{n}{4}\left[\frac{x^2+x_0^2\cos^2n\tau-2xx_0\cos n\tau-x_0^2\sin^2n\tau}{\sin^2n\tau}\right]\sin 2n\tau\nonumber\\
~~~~~~~~~~~~~~~~~~~~~~+\frac{x_0n}{2}\frac{(x-x_0\cos n\tau)(\cos 2n\tau-1)}{\sin n\tau}\nonumber\\
=\frac{1}{2}\sqrt{\mu}\left[\frac{(x^2+x_0^2)\cos(\sqrt{\mu}\tau)-2xx_0}{\sin(\sqrt{\mu}\tau)}\right],~~~~~~~~~~~~~~~~~~~~~~\nonumber\\
~~~~~~~~~~~~~~~~~~~~~~~~~~~~~~~~~~~~~~~~~~~~~\tau=t-t_0, ~~n=\sqrt{\mu}\nonumber.
\end{eqnarray}

\vspace{.5cm}

$\bullet$ \textbf{Problem II}: Prove that in case of a particle of unit mass moving in a plane $xy$ under a central acceleration $n^2r$, the value of the Hamilton's principal function is $S=\dfrac{n}{2\sin n\tau}\left\{\left(x_0^2+y_0^2+x^2+y^2\right)\cos n\tau-2(xx_0+yy_0)\right\}$ and $\tau=t-t_0$.



\vspace{.5cm}






\section{Virial Theorem}
Let us consider the motion of a system subject to a force $\overrightarrow{F_i}$ in the position $\overrightarrow{r_i}$.Then the kinetic energy of the system is
$$T=\frac{1}{2}\sum m\dot{\overrightarrow{r}}^2.$$
\begin{eqnarray}
\therefore 2T&=&\sum m\dot{\overrightarrow{r}}^2=\sum m\dot{\overrightarrow{r}}\dot{\overrightarrow{r}}=\frac{d}{dt}\left(\sum m\dot{\overrightarrow{r}}.\overrightarrow{r}\right)-\sum\overrightarrow{r}\frac{d}{dt}\left(m\dot{\overrightarrow{r}}\right)\nonumber\\
&=&\frac{d}{dt}\left(\sum m\dot{\overrightarrow{r}}.\overrightarrow{r}\right)-\sum\overrightarrow{r}.\overrightarrow{F},~~~~\overrightarrow{F}=\frac{d}{dt}\left(m\dot{\overrightarrow{r}}\right)=m\ddot{\overrightarrow{r}}.\nonumber
\end{eqnarray} 
or, 
\begin{equation}
2T+\sum\overrightarrow{r}.\overrightarrow{F}=\frac{d}{dt}\left(\sum m\dot{\overrightarrow{r}}.\overrightarrow{r}\right).\label{eq3..2}
\end{equation}
$\big(\sum\overrightarrow{r}.\overrightarrow{F}$ is called the virial of the system$\big).$

Let us now average the equation (\ref{eq3..2}) over an arbitrary interval of time $(0,\tau)$, then 
\begin{eqnarray}
\frac{1}{\tau}\int_{0}^{\tau}\left(2T+\sum\overrightarrow{r}.\overrightarrow{F}\right)dt=\frac{1}{\tau}\left[\sum m\dot{\overrightarrow{r}}.\overrightarrow{r}\right]_0^\tau~~~~~~~~~~~~~~~~~~~~~~~~~~~~~~~~\nonumber\\
~~~~~~~~~~~~~~~~~~~~~~~~~~~~~=\frac{1}{\tau}\left[\left(\sum m\dot{\overrightarrow{r}}.\overrightarrow{r}\right)\bigg|_\tau-\left(\sum m\dot{\overrightarrow{r}}.\overrightarrow{r}\right)\bigg|_{\tau=0}\right]\label{eq3..3}
\end{eqnarray} 
Let us now assume that the system executed a finite motion in a finite region of space, then the numerator of the right hand side of equation (\ref{eq3..3}) is finite for all values of $\tau$, since it is bounded for all values of $\tau$. Consequently, taking the limit as $\tau\rightarrow\infty$ we get from (\ref{eq3..3})
\begin{eqnarray}
\lim_{\tau\rightarrow\infty}\frac{1}{\tau}\int_{0}^{\tau}\left(2T+\sum\overrightarrow{r}.\overrightarrow{F}\right)dt=0\nonumber\\
\mbox{i.e,}~~2\overline{T}+\overline{\left(\sum\overrightarrow{r}.\overrightarrow{F}\right)}=0\label{eq3..4}
\end{eqnarray}
where $\overline{T}=\lim_{\tau\rightarrow\infty}\frac{1}{\tau}\int_{0}^{\tau}Tdt$ and $\overline{\left(\sum\overrightarrow{r}.\overrightarrow{F}\right)}=\lim_{\tau\rightarrow\infty}\frac{1}{\tau}\int_{0}^{\tau}\left(\overrightarrow{r}.\overrightarrow{F}\right)dt$.

Equation (\ref{eq3..4}) states that for finite motion of a system in a finite region of space twice the time average value of the kinetic energy is equal to $(-1)\times$ time average value of the virial of the system. This is known as \underline{virial theorem}.

If the field be conservative, then $\overrightarrow{F}=-\mbox{grad}V=-\dfrac{\partial V}{\partial\overrightarrow{r}}$; so equation (\ref{eq3..4}) becomes 
$$2\overline{T}=\sum\overrightarrow{r}.\frac{\partial V}{\partial\overrightarrow{r}}$$
In particular, if $V$ is homogeneous in the co-ordinates of degree $n$ then 
\begin{eqnarray}
\sum\overrightarrow{r}.\frac{\partial V}{\partial\overrightarrow{r}}=nV\nonumber\\
\therefore 2\overline{T}=n\overline{V}\label{eq3..5}
\end{eqnarray}


We obtain from the equation of energy

$$\overline{T}+\overline{V}=\mbox{constant}=E~~(\mbox{say})$$
\begin{eqnarray}
\therefore\overline{V}&=&E-\overline{T}\nonumber\\
\mbox{or},~~~2\overline{T}&=&n\left(E-\overline{T}\right)\nonumber\\
\mbox{or},~~~\overline{T}&=&\frac{n}{n+2}E~~~\mbox{and}~~~\overline{V}=\frac{2}{n+2}E\nonumber
\end{eqnarray}















\section{Vibration of String and Membrence:}
$\star$ \textbf{Transverse Vibration of a Stretched String:}

We take $x-$ axis along $AB$ the equilibrium position of the string and the end points at $x=0$ and $x=l$. We consider the plane of vibration as $xy-$ plane. The transverse displacement is $y$. We assume it to be so small that the tangent to the string at any point in the displaced position makes a small inclination with the $x-$ axis. Now we apply Hamilton's principle to deduce the equation of motion. Let $\rho$ be the density of the material of the string. The kinetic energy is 
$$T=\frac{1}{2}\int\rho\dot{y}^2dx$$
The integration being taken along the length of the string. The potential energy of the string is given by 
$$V=\int T_1(ds-dx),$$
taken over the length of the string, $T_1$ being the constant tension. Now 
$$ds^2=dx^2+dy^2~~\mbox{i.e,}~~\frac{ds}{dx}=\sqrt{1+\left(\frac{dy}{dx}\right)^2}\approx 1+\frac{1}{2}\left(\frac{dy}{dx}\right)^2$$
so that the potential energy is 
$$V\approx\frac{1}{2}\int T_1\left(\frac{dy}{dx}\right)^2dx$$
Now by Hamilton's principle, we have 
\begin{eqnarray}
\int_{t_1}^{t_2}\delta(T-V)dt=0~~~~~~~~~~~~~~~~~~~~~~~~\nonumber\\
\mbox{or,}~~\int_{t_1}^{t_2}dt\int\left[\rho\dot{y}\frac{d}{dt}\delta y-T_1\frac{dy}{dx}\frac{d}{dx}\delta y\right]dx=0\nonumber\\
\mbox{or,}~~\rho\dot{y}\delta ydx\bigg|_{t_1}^{t_2}-\int_{t_1}^{t_2}dt\int\left[\rho\ddot{y}\delta y+T_1\frac{dy}{dx}\delta y\right]dx=0\nonumber
\end{eqnarray}
~~~~~~~~~~~~~~~~~~~~~~~~~~~~~~~~~~~~~~~~~~~~~(Integrating by parts with respect to $t$ alone)

As $\delta y=0$ at $t=t_1$ and $t=t_2$, so again integrating by parts with respect to `$x$' we have
$$\int T_1\frac{dy}{dx}\delta ydt\bigg|_0^l+\int_{t_1}^{t_2}dt\int\left[\rho\ddot{y}-T_1\frac{d^2y}{dx^2}\right]\delta ydx=0$$
As $\delta y=0$ at $x=0$ and $x=l$ so
$$\int_{t_1}^{t_2}dt\int\left[\rho\ddot{y}-T_1\frac{\partial^2y}{\partial x^2}\right]\delta ydx=0$$
In order to satisfy this we have in local form
\begin{eqnarray}
\rho\ddot{y}-T_1\frac{\partial^2y}{\partial x^2}=0~~~~~~~~~~~~~~~\nonumber\\
\mbox{or,}~~\frac{\partial^2y}{\partial t^2}=c^2\frac{\partial^2y}{\partial x^2},~~~c^2=\frac{T_1}{\rho}\nonumber
\end{eqnarray}
This is the differential equation of vibrating string and is known as wave equation.

To find a solution of the differential equation we put, $\xi_1=x-ct$, $\xi_2=x+ct$.
\begin{eqnarray}
\therefore\frac{\partial}{\partial x}=\frac{\partial}{\partial\xi_1}+\frac{\partial}{\partial\xi_2},~~~\frac{\partial}{\partial t}=-c\frac{\partial}{\partial\xi_1}+c\frac{\partial}{\partial\xi_2}\nonumber\\
\therefore\left(-c\frac{\partial}{\partial\xi_1}+c\frac{\partial}{\partial\xi_2}\right)^2y=c^2\left(\frac{\partial}{\partial\xi_1}+\frac{\partial}{\partial\xi_2}\right)^2y\nonumber\\
\mbox{i.e,}~~\frac{\partial^2y}{\partial\xi_1\partial\xi_2}=0~~~~~~~~~~~~~~~~~~~~~~~~~~~~~~~~\nonumber\\
\mbox{or,}~~\frac{\partial y}{\partial\xi_2}=F(\xi_2)\nonumber~~~~~~~~~~~~~~~~~~~~~~~~~~~~~~~\\
\mbox{or,}~~y=f_1(\xi_1)+f_2(\xi_2)=f_1(x-ct)+f_2(x+ct)\nonumber
\end{eqnarray}

\begin{figure}[h!]
	\centering
	\includegraphics[ height=0.6 \textheight, width=0.6\textheight]{photo2.pdf}
\end{figure}


Let us assume, $y=f(x)\cos(nt+\epsilon)$ as a solution of the equation. Then from the wave equation
\begin{eqnarray}
f''(x)+\frac{n^2}{c^2}f(x)=0~~~~~~~~~~~~~~~~~~~~~\nonumber\\
\mbox{i.e,}~~f(x)=A\cos\left(\frac{nx}{c}\right)+B\sin\left(\frac{nx}{c}\right)\nonumber
\end{eqnarray}
Hence the solution is 
$$y=\left[A\cos\left(\frac{nx}{c}\right)+B\sin\left(\frac{nx}{c}\right)\right]\cos(nt+\epsilon)$$
As $y=0$ when $x=0$ and $x=l$ so we have 
$$A=0~~~\mbox{and}~~~0=B\sin\left(\frac{nl}{c}\right)$$
(Note that $B=0$ gives the trivial solution $y=0$)

\begin{eqnarray}
\therefore\sin\left(\frac{nl}{c}\right)=0=\sin s\pi~~~~~~(s~\mbox{is an integer including zero})\nonumber\\
\therefore n=\frac{s\pi c}{l}~~~~~~~~~~~~~~~~~~~~~~~~~~~~~~~~~~~~~~~~~~~~~~~~~~~~~~~~~~~~\nonumber\\
\therefore y=B_s\sin\left(\frac{s\pi x}{l}\right)\cos\left(\frac{s\pi c}{l}t+\epsilon\right)~~~~~~~~~~~~~~~~~~~~~~~~~~~~\nonumber
\end{eqnarray}
Hence the general solution is 
\begin{equation}
y=\sum_{s=1}^{\infty}B_s\sin\left(\frac{s\pi x}{l}\right)\cos\left(\frac{s\pi c}{l}t+\epsilon_s\right)\nonumber
\end{equation}
The period of the $s$th mode is $\dfrac{2\pi}{n}=\dfrac{2l}{sc}=\dfrac{2l}{s}\sqrt{\dfrac{\rho}{T_1}}$~~~~~~$(s=1,~2,...)$.

The frequency of the $s$-th mode is $\dfrac{s}{2l}\sqrt{\dfrac{T_1}{\rho}}$.

The frequency for greatest mode (i.e, $s=1$) is $\dfrac{1}{2l}\sqrt{\dfrac{T}{\rho}}$.

To determine the constants $B_s$ and $\epsilon_s$ we write the general solution as 
\begin{equation}
y=\sum_{s=1}^{\infty}\sin\left(\frac{s\pi x}{l}\right)\left[C_s\cos\left(\frac{s\pi ct}{l}\right)+D_s\sin\left(\frac{s\pi ct}{l}\right)\right]\nonumber
\end{equation}
where $C_s$ and $D_s$ are arbitrary constants, which will be determined from initial conditions. The vibration given by
\begin{equation}
y=\sin\left(\frac{s\pi x}{l}\right)\left[C_s\cos\left(\frac{s\pi ct}{l}\right)+D_s\sin\left(\frac{s\pi ct}{l}\right)\right]\nonumber
\end{equation}
is called the normal mode of vibration of the string. The general vibration of the string is obtained by superposition of normal modes. For a fixed `$s$', $y$ vanishes, when 
$$x=\frac{l}{s},~~\frac{2l}{s},~~\frac{3l}{s},...,\frac{(s-1)l}{s}$$
besides the points $x=0$ and $x=l$.

These points are called the nodes. For $s=2$, $x=\dfrac{l}{2}$ is the only node i.e, the middle point of the string is the only node. Now we shall determine the constants $C_s$ and $D_s$ as follows:

Let initially i.e, at $t=0$, $y=y_0(x)$ and $\dfrac{\partial y}{\partial t}=\dot{y_0}(x)$.

So when $t=0$, we have 
$$y_0(x)=\sum_{s=1}^{\infty}C_s\sin\left(\frac{s\pi x}{l}\right)$$
By Fourier's method, multiplying both side by $\sin\left(\dfrac{s\pi x}{l}\right)$ and integrating with respect to $x$ between the limits $x=0$ and $x=l$, we have
\begin{eqnarray}
\int_{0}^{l}y_0(x)\sin\frac{s\pi x}{l}dx=\int_{0}^{l}\sin^2\frac{s\pi x}{l}dx=\frac{C_s}{2}l\nonumber\\
\therefore C_s=\frac{2}{l}\int_{0}^{l}y_0(x)\sin\left(\frac{s\pi x}{l}\right)dx~~~~~~~~~~~~~~~~\nonumber
\end{eqnarray}
Similarly differentiating the expression for $y$ with respect to `$t$' and putting $t=0$, we have 
$$\dot{y_0}(x)=\sum_{s=1}^{\infty}\left(\frac{s\pi c}{l}\right)D_s\sin\left(\frac{s\pi x}{l}\right)$$
Now proceeding as before, we have
$$D_s=\frac{2}{s\pi c}\int_{0}^{l}\dot{y_0}(x)\sin\left(\frac{s\pi x}{l}\right)dx$$











$\star$ \textbf{Plucked String:}


\begin{figure}[h!]
	\centering
	\includegraphics[scale=0.5]{photo3.pdf}
\end{figure}

Let $x$-axis is along the string and the end points are at $x=0$, $x=l$. The equation of vibration is 
$$\frac{\partial^2y}{\partial t^2}=c^2\frac{\partial^2y}{\partial x^2},~~~\mbox{where}~~c^2=\frac{T_1}{\rho}$$
The motion starts from rest by displacing the point $x=b$ of the string through a distance `$r$' transversly and then letting it go. When $t=0$, $y=f(x)$ and $\dot{y}=0$, $f(x)$ defines the form of the string initially. Therefore $y=0$ when $x=0$ and $x=l$ and initially,
\begin{eqnarray}
f(x)&=&\frac{rx}{b},~~~0\leq x\leq b,~~~t=0\nonumber\\
&=&\frac{r(l-x)}{(l-b)},~~~b\leq x\leq l,~~~t=0\nonumber
\end{eqnarray}
As before, the solution of the wave equation is of the form 
$$y=\sum_{s=1}^{\infty}\sin\left(\frac{s\pi x}{l}\right)\left[C_s\cos\left(\frac{s\pi ct}{l}\right)+D_s\sin\left(\frac{s\pi ct}{l}\right)\right]$$
Differentiating term-by-term, we have
\begin{eqnarray}
0=\dot{y}\big|_{t=0}=\sum_{s=1}^{\infty}D_s\sin\left(\frac{s\pi x}{l}\right)\left(\frac{s\pi c}{l}\right)\nonumber\\
\therefore D_s=0~~~~~~~~~~~~~~~~~~~~~~~~~~~~~\nonumber
\end{eqnarray}
Also at $t=0$, $y=f(x)$ gives
$$f(x)=\sum_{s=1}^{\infty}C_s\sin\left(\frac{s\pi x}{l}\right)$$
To determine $C_s$ we multiply both side by $\sin\left(\dfrac{s\pi x}{l}\right)$ and integrating between $x=0$ to $x=l$ we have 
\begin{eqnarray}
C_s&=&\frac{2}{l}\int_{0}^{l}f(x)\sin\left(\frac{s\pi x}{l}\right)dx\nonumber\\
&=&\frac{2}{l}\left[\int_{0}^{b}\frac{rx}{b}\sin\left(\frac{s\pi x}{l}\right)dx+\int_{b}^{l}\frac{r(l-x)}{l-b}\sin\left(\frac{s\pi x}{l}\right)dx\right]\nonumber\\
&=&\frac{2rl^2}{b(l-b)s^2\pi^2}\sin\frac{s\pi b}{l}\nonumber
\end{eqnarray}
Hence the general solution is 
\begin{eqnarray}
y&=&\sum_{s=1}^{\infty}C_s\sin\frac{s\pi x}{l}\cos\frac{s\pi c}{l}t\nonumber\\
&=&\frac{2rl^2}{b(l-b)\pi^2}\sum_{s=1}^{\infty}\frac{1}{s^2}\sin\left(\frac{s\pi b}{l}\right)\sin\left(\frac{s\pi x}{l}\right)\cos\left(\frac{s\pi c}{l}t\right)\nonumber
\end{eqnarray}

\vspace{1cm}

$\bullet$ \textbf{Problem:} If a slightly elastic string is stressed between two fixed points and the motion is started by drawing aside through a distance `$b$', a point on the string distance $\dfrac{1}{5}$ of the length `$l$' of the string from one end, the displacement at any instant will be given by the equation 
$$y=\frac{25b}{2\pi^2}\sum_{n=1}^{\infty}\left[\frac{1}{n^2}\sin\left(\frac{n\pi}{5}\right)\sin\left(\frac{n\pi x}{l}\right)\cos\left(\frac{n\pi ct}{l}\right)\right]$$











$\star$ \textbf{Forced Vibration of a String:}

Suppose $x$-axis is along the length of the string with $x=0$ and $x=l$ are the two end points. Let the displacement of the point $x=b$ at any time is represented by $y=r\cos(pt+\epsilon)$. The equation of motion of the string is 
$$\frac{\partial^2y}{\partial t^2}=c^2\frac{\partial^2y}{\partial x^2}$$
We assume the solution of the form 
$$y=f(x)\cos(pt+\epsilon),$$
then from the above differential equation $f(x)$ has the solution of the form 
$$f(x)=A\cos\frac{p}{c}x+B\sin\frac{p}{c}x$$
Hence we write the solution in the form 
\begin{eqnarray}
y&=&\left(A\cos\frac{px}{c}+B\sin\frac{px}{c}\right)\cos(pt+\epsilon),~~~0\leq x\leq b\nonumber\\
&=&\left(C\cos\frac{px}{c}+D\sin\frac{px}{c}\right)\cos(pt+\epsilon),~~~b\leq x\leq l\nonumber
\end{eqnarray}
To find the constants we have $y=0$, at $x=0$ and $x=l$ and $y=r\cos(pt+\epsilon)$ at $x=b$, therefore,
\begin{eqnarray}
0=A,~~~~~0=C\cos\left(\frac{pl}{c}\right)+D\sin\left(\frac{pl}{c}\right)\nonumber\\
r\cos(pt+\epsilon)=B\sin\frac{pb}{c}\cos(pt+\epsilon)~~~~~~~\nonumber\\
\mbox{and}~~~r\cos(pt+\epsilon)=\left(C\cos\frac{pb}{c}+D\sin\frac{pb}{c}\right)\cos(pt+\epsilon)\nonumber\\
\therefore B=r~cosec\left(\frac{pb}{c}\right)~~~~~~~~~~~~~~~~~~~~~~~\nonumber
\end{eqnarray}
\begin{eqnarray}
\mbox{Also}~~~r&=&C\cos\frac{pb}{c}+D\sin\frac{pb}{c}=\frac{c}{\sin\left(\frac{pl}{c}\right)}\left[\cos\frac{pb}{c}\sin\frac{pl}{c}-\cos\frac{pl}{c}\sin\frac{pb}{c}\right]\nonumber\\
&=&\frac{c}{\sin\frac{pl}{c}}\sin\left[\frac{p}{c}(l-b)\right]\nonumber
\end{eqnarray}
\begin{eqnarray}
\therefore C=\frac{r\sin\left(\frac{pl}{c}\right)}{\sin\left[\frac{p}{c}(l-b)\right]},~~~~~D=\frac{-r\cos\left(\frac{pl}{c}\right)}{\sin\left[\frac{p}{c}(l-b)\right]}\nonumber
\end{eqnarray}
Thus
\begin{eqnarray}
y&=&\frac{r\sin\left(\frac{px}{c}\right)}{\sin\left(\frac{pb}{c}\right)}\cos(pt+\epsilon),~~~0\leq x\leq b\nonumber\\
&=&\frac{r\sin\left[\frac{p}{c}(l-x)\right]}{\sin\left[\frac{p}{c}(l-b)\right]}\cos(pt+\epsilon),~~~b\leq x\leq l\nonumber
\end{eqnarray}
If there be a force $F\cos(pt+\epsilon)$ at $x=b$, then 
\begin{equation}
F\cos(pt+\epsilon)=T_1\left(\frac{dy}{dx}\right)_{x=b^-}-T_1\left(\frac{dy}{dx}\right)_{x=b^+}\nonumber
\end{equation}
\begin{eqnarray}
\mbox{But,}~~~\frac{dy}{dx}&=&\frac{r\frac{p}{c}\cos\left(\frac{px}{c}\right)}{\sin\left(\frac{pb}{c}\right)}\cos(pt+\epsilon),~~~\mbox{for}~~0\leq x\leq b\nonumber\\
\mbox{and}~~~\frac{dy}{dx}&=&-\frac{r\frac{p}{c}\cos\left[\frac{p}{c}(l-x)\right]}{\sin\left[\frac{p}{c}(l-b)\right]}\cos(pt+\epsilon),~~~\mbox{for}~~b\leq x\leq l\nonumber
\end{eqnarray}
and so
\begin{eqnarray}
F&=&T_1\left(r\frac{p}{c}\right)\left[\cot\left(\frac{pb}{c}\right)-\cot\left(\frac{p(l-b)}{c}\right)\right]\nonumber\\
&=&T_1\frac{rp}{c}\frac{\sin\left(\frac{pl}{c}\right)}{\sin\left(\frac{pb}{c}\right)\sin\left[\frac{p(l-b)}{c}\right]}\nonumber
\end{eqnarray}
So $r$ can be obtained from above and $y$ can be uniquely determined.

\vspace{1cm}

$\bullet$ \textbf{Problem I:} A string of length $\left(l+l'\right)$ is stretch with tension between two fixed points. The length `$l$' has mass $m$ per unit length, the length `$l'$' has mass $m'$ per unit length. Prove that the possible period $T$ of transverse vibration are given by the equation
$$\frac{\tan\frac{2\pi l}{T}\sqrt{\frac{m}{P}}}{\tan\frac{2\pi l'}{T}\sqrt{\frac{m'}{P}}}+\sqrt{\frac{m}{m'}}=0$$

\vspace{.5cm}

\textbf{Solution:} 

\begin{figure}[h!]
	\centering
	\includegraphics[scale=0.5]{photo4.pdf}
\end{figure}

Let
\begin{eqnarray}
y_1&=&\left(A\cos\frac{px}{c_1}+B\sin\frac{px}{c_1}\right)\cos(pt+\epsilon),~~~\mbox{for}~0\leq x\leq l\nonumber\\
y_2&=&\left\{C\cos\frac{p\left(l+l'-x\right)}{c_2}+D\sin\frac{p\left(l+l'-x\right)}{c_2}\right\}\cos(pt+\epsilon),~~~\mbox{for}~l\leq x\leq l+l'\nonumber
\end{eqnarray}
\begin{eqnarray}
\mbox{Now,}~~~y_1&=&0,~~\mbox{at}~x=0\implies A=0~~~~~~~~~~~~~~~~~~~~~~~~~~\nonumber\\
y_2&=&0,~~\mbox{at}~x=l+l'\implies C=0\nonumber
\end{eqnarray}
\begin{eqnarray}
y_1=P~~\mbox{at}~x=l,~~~y_2=P~~\mbox{at}~x=l.~~~~~~~~~~\nonumber\\
\mbox{i.e,}~~y_1=y_2~~\mbox{at}~x=l~~\mbox{and}~~\frac{dy_1}{dx}=\frac{dy_2}{dx}~~\mbox{at}~x=l\nonumber
\end{eqnarray}
\begin{equation}
\therefore B\sin\frac{pl}{c_1}=D\sin\frac{pl'}{c_2}\label{eq3..6}
\end{equation}
\begin{eqnarray}
\mbox{Also}~~~\frac{dy_1}{dx}&=&B\frac{p}{c_1}\cos\frac{pl}{c_1}~~\mbox{at}~x=l\nonumber\\
\mbox{and}~~~\frac{dy_2}{dx}&=&-D\frac{p}{c_2}\cos\frac{pl'}{c_2}~~\mbox{at}~x=l\nonumber
\end{eqnarray}
\begin{equation}
\therefore \frac{B}{c_1}\cos\frac{pl}{c_1}=-\frac{D}{c_2}\cos\frac{pl'}{c_2}\label{eq3..7}
\end{equation}
\begin{equation}
(\ref{eq3..6})\div(\ref{eq3..7})\implies c_1\tan\frac{pl}{c_1}=-c_2\tan\frac{pl'}{c_2}\label{eq3..8}
\end{equation}
$$\mbox{But,}~~~t=\frac{2\pi}{p},~~c_1^2=\frac{P}{m},~~c_2^2=\frac{P}{m'}$$
Hence from (\ref{eq3..8}),
$$\frac{\tan\frac{2\pi l}{T}\sqrt{\frac{m}{P}}}{\tan\frac{2\pi l'}{T}\sqrt{\frac{m'}{P}}}+\sqrt{\frac{m}{m'}}=0$$

\vspace{1cm}

$\bullet$ \textbf{Problem II:} Three strings $AB$, $BC$, $CD$ are stretched in a straight line $AD$ and freely joined at $B$ and $C$ and vibrate transversely. Show that the frequency equation is given by 
$$\frac{\tan n_1l_1}{n_1}+\frac{\tan n_2l_2}{n_2}+\frac{\tan n_3l_3}{n_3}=n_2^2\frac{\tan n_1l_1}{n_1}.\frac{\tan n_2l_2}{n_2}.\frac{\tan n_3l_3}{n_3}$$
where, $n_r=\dfrac{p}{c_r}$, $c_r=\sqrt{\dfrac{T_1}{\rho_r}}$, $T_1$ is the tension of the string, $\rho_r$ is the line density and $l_1$, $l_2$ and $l_3$ are the lengths of $AB$, $BC$ and $CD$ respectively.

\vspace{.5cm}

\textbf{Solution:}


\begin{figure}[h!]
	\centering
	\includegraphics[scale=0.7]{photo5.pdf}
\end{figure}

Let the solution be 
\begin{eqnarray}
y_1&=&\left(A\cos\frac{px}{c_1}+B\sin\frac{px}{c_1}\right)\cos(pt+\epsilon)~~~\mbox{for}~AB\nonumber\\
y_2&=&\left(C\cos\left\{\frac{p(l_1+l_2-x)}{c_2}\right\}+D\sin\left\{\frac{p(l_1+l_2-x)}{c_2}\right\}\right)\cos(pt+\epsilon)~~~\mbox{for}~BC\nonumber\\
\mbox{and}~~~y_3&=&\left[E\cos\left\{\frac{p(l_1+l_2+l_3-x)}{c_3}\right\}+F\sin\left\{\frac{p(l_1+l_2+l_3-x)}{c_3}\right\}\right]\cos(pt+\epsilon)~~~\mbox{for}~CD\nonumber
\end{eqnarray}
The boundary conditions are
\begin{eqnarray}
(i)~~y_1=0,~~x=0\implies A=0~~~~~~~~~~~~~\nonumber\\
(ii)~~y_3=0,~~x=l_1+l_2+l_3\implies E=0\nonumber\\
(iii)~~y_1=y_2~~\mbox{when}~x=l_1~~~~~~~~~~~~~~~~~~~\nonumber
\end{eqnarray}
\begin{equation}
\therefore B\sin\frac{pl_1}{c_1}=\left[C\cos\frac{pl_2}{c_2}+D\sin\frac{pl_2}{c_2}\right]\label{eq3..9}
\end{equation}
\begin{eqnarray}
(iv)~~y_2=y_3,~~\mbox{when}~x=l_1+l_2~~~~~~~~~~~~~\nonumber\\
\therefore C=F\sin\frac{pl_3}{c_3}~~~~~~~~~~~~~~~~~~~~~~~\label{eq3..10}
\end{eqnarray}
\begin{eqnarray}
(v)~~\frac{dy_1}{dx}=\frac{dy_2}{dx}~~\mbox{at}~x=l_1~~~~~~~~~~~~~~~~~~~~~~~~~\nonumber\\
\therefore \frac{Bp}{c_1}\cos\frac{pl_1}{c_1}=\frac{p}{c_2}\left[C\sin\frac{pl_2}{c_2}-D\cos\frac{pl_2}{c_2}\right]\label{eq3..11}
\end{eqnarray}
\begin{eqnarray}
(vi)~~\frac{dy_2}{dx}=\frac{dy_3}{dx},~\mbox{at}~x=l_1+l_2~~~~~~~~~~~~~~~~~~~~~\nonumber\\
\therefore -\frac{P}{c_2}D=-\frac{FP}{c_3}\cos\frac{pl_3}{c_3}\implies D=\frac{Fc_2}{c_3}\cos\frac{pl_3}{c_3}\label{eq3..12}
\end{eqnarray}
(\ref{eq3..9}) $\div$ (\ref{eq3..11}) using (\ref{eq3..10}) and (\ref{eq3..12}),
\begin{eqnarray}
c_1\tan\frac{pl_1}{c_1}&=&c_2\frac{\cos\frac{pl_2}{c_2}\sin\frac{pl_3}{c_3}+\frac{c_2}{c_3}\cos\frac{pl_3}{c_3}\sin\frac{pl_2}{c_2}}{\sin\frac{pl_2}{c_2}\sin\frac{pl_3}{c_3}-\frac{c_2}{c_3}\cos\frac{pl_2}{c_2}\cos\frac{pl_3}{c_3}}\nonumber\\
&=&c_2\frac{\tan\frac{pl_3}{c_3}+\frac{c_2}{c_3}\tan\frac{pl_2}{c_2}}{-\frac{c_2}{c_3}+\tan\frac{pl_2}{c_2}\tan\frac{pl_3}{c_3}}\nonumber
\end{eqnarray}
\begin{eqnarray}
\mbox{or,}~~-\frac{c_1c_2}{c_3}\tan\frac{pl_1}{c_1}+c_1\tan\frac{pl_1}{c_1}\tan\frac{pl_2}{c_2}\tan\frac{pl_3}{c_3}=c_2\tan\frac{pl_3}{c_3}+\frac{c_2^2}{c_3}\tan\frac{pl_2}{c_2}\nonumber\\
\mbox{or,}~~c_2\tan\frac{pl_3}{c_3}+\frac{c_2^2}{c_3}\tan\frac{pl_2}{c_2}+\frac{c_1c_2}{c_3}\tan\frac{pl_1}{c_1}=c_1\tan\frac{pl_1}{c_1}\tan\frac{pl_2}{c_2}\tan\frac{pl_3}{c_3}\nonumber\\
\mbox{or,}~~c_1\tan\frac{pl_1}{c_1}+c_2\tan\frac{pl_2}{c_2}+c_3\tan\frac{pl_3}{c_3}=\frac{c_1c_3}{c_2}\tan\frac{pl_1}{c_1}\tan\frac{pl_2}{c_2}\tan\frac{pl_3}{c_3}\nonumber\\
\mbox{or,}~~\frac{\tan n_1l_1}{n_1}+\frac{\tan n_2l_2}{n_2}+\frac{\tan n_3l_3}{n_3}=n_2^2\frac{\tan n_1l_1}{n_1}.\frac{\tan n_2l_2}{n_2}.\frac{\tan n_3l_3}{n_3}\nonumber
\end{eqnarray}

\vspace{1cm}

$\bullet$ \textbf{Problem III:} A uniform string whose length is $2l$ and mass $2lm$ is stretched at tension $T$ between two fixed points. The middle point of the string being displaced a small distance $b$ $\perp$ to the string and then released. Show that the subsequent motion of the string referred to the string is given by the equation
$$y=\frac{8b}{\pi^2}\sum_{r=0}^{\infty}\frac{1}{(2r+1)^2}\cos\left[\frac{(2r+1)}{2l}\pi x\right]\cos\left[\frac{(2r+1)}{2l}\pi ct\right]$$
where $c$ is given by the equation $mc^2=T$.

\vspace{.5cm}

\textbf{Solution:}

\begin{figure}[h!]
	\centering
	\includegraphics[scale=0.5]{photo6.pdf}
\end{figure}


Let the solution 
$$y=\left(A\cos\frac{nx}{c}+B\sin\frac{nx}{c}\right)\cos(nt+\epsilon)$$
Since the string is symmetrical about $y$-axis i.e, $y$ is same for $+x$, $-x$.

$\therefore B=0$.

$\therefore$ $y=0$ when $x=\pm l$~~$\implies$~~$\cos\dfrac{nl}{c}=0=\cos(2r+1)\frac{\pi}{2}$

$\therefore\dfrac{nl}{c}=(2r+1)\dfrac{\pi}{2}$, $r$ is any integer even zero.
\begin{eqnarray}
\therefore y&=&A\cos\frac{nx}{c}\cos(nt+\epsilon)\nonumber\\
&=&\sum_{r=0}^{\infty}\cos\frac{(2r+1)\pi x}{2l}\left[C_r\cos\left\{\frac{(2r+1)}{2l}\pi ct\right\}+D_r\sin\left\{\frac{(2r+1)}{2l}\pi ct\right\}\right]\nonumber
\end{eqnarray}
Now at $t=0$, $\dfrac{dy}{dt}=0$ $\implies$ $D_r=0$.

Also at $t=0$, 
\begin{eqnarray}
y=f(x)&=&\frac{b}{l}(l-x),~~~0\leq x\leq l\nonumber\\
&=&\frac{b}{l}(l+x),~~~-l\leq x\leq 0\nonumber
\end{eqnarray}
\begin{eqnarray}
\therefore y&=&\sum_{r=0}^{\infty}C_r\cos\left[\frac{(2r+1)}{2l}\pi x\right]\cos\left[\frac{(2r+1)}{2l}\pi ct\right]\label{eq3..13}\\
\therefore f(x)&=&\sum_{r=0}^{\infty}C_r\cos\left[\frac{(2r+1)}{2l}\pi x\right]\nonumber
\end{eqnarray}
Now multiplying both sides by $\cos\frac{(2r+1)}{2l}\pi x$ and integrating over $x$ from $-l$ to $+l$ we have
\begin{eqnarray}
C_r&=&\frac{1}{l}\int_{-l}^{l}f(x).\cos\left[\frac{(2r+1)}{2l}\pi x\right]dx\nonumber\\
&=&\frac{b}{l}\left\{\int_{0}^{l}(l-x)\cos\left[\frac{(2r+1)}{2l}\pi x\right]dx+\int_{-l}^{0}(l+x)\cos\left[\frac{(2r+1)}{2l}\pi x\right]dx\right\}\nonumber\\
&=&\frac{2b}{l}\int_{0}^{l}(l-x)\cos\left[\frac{(2r+1)}{2l}\pi x\right]dx\nonumber\\
&=&\frac{2b}{l}\left\{\left[(l-x)\frac{2l}{(2r+1)\pi}\sin\frac{(2r+1)\pi x}{2l}\right]_0^l+\int_{0}^{l}\frac{2l}{(2r+1)\pi}\sin\left[\frac{(2r+1)\pi x}{2l}\right]dx\right\}\nonumber\\
&=&\frac{2b}{l}\frac{2l}{(2r+1)\pi}\frac{2l}{(2r+1)\pi}\left[-\cos\frac{(2r+1)\pi x}{2l}\right]_0^l\nonumber\\
&=&\frac{8b}{(2r+1)^2\pi^2}\nonumber
\end{eqnarray}
\begin{equation}
\therefore C_r=\frac{8b}{(2r+1)^2\pi^2}\nonumber
\end{equation}
\begin{equation}
\therefore y=\frac{8b}{\pi^2}\sum_{r=0}^{\infty}\frac{1}{(2r+1)^2}\cos\frac{(2r+1)\pi x}{2l}\cos\frac{(2r+1)\pi ct}{2l}\nonumber
\end{equation}
When $\rho$ is not constant then mass of the element $\delta x$ is $\rho\delta x$. For transverse vibration the acceleration of $\rho\delta x$ along $OX$ is zero and $\ddot{y}$ along $OY$. So the equation of motion will be 
\begin{equation}
\rho\delta x.0=-T\cos\psi+\left[T\cos\psi+\frac{\partial}{\partial y}(T\cos\psi).\delta y\right]\nonumber
\end{equation}
\begin{equation}
\therefore T\cos\psi=\mbox{constant}~~~\mbox{i.e,}~~T=\mbox{constant upto 1st order.}\nonumber
\end{equation}
\begin{eqnarray}
\rho\delta x\frac{\partial^2 y}{\partial t^2}=-T\sin\psi+\left[T\sin\psi+\frac{\partial}{\partial x}(T\sin\psi)\delta x\right]=T\frac{\partial}{\partial x}(\tan\psi)\delta x\nonumber\\
~~~~~~~~~~~~~~~~~~~~~~~~~(\because~~\mbox{for small}~\psi,~~\tan\psi=\sin\psi)\nonumber
\end{eqnarray}
\begin{equation}
\therefore\rho\frac{\partial^2y}{\partial t^2}=T\frac{\partial}{\partial x}\left(\frac{\partial y}{\partial x}\right)\nonumber
\end{equation}
\begin{equation}
\mbox{Hence,}~~~\frac{\partial^2y}{\partial t^2}=\frac{T}{\rho}\frac{\partial^2y}{\partial x^2}=c^2\frac{\partial^2y}{\partial x^2}\nonumber
\end{equation}
Note that here $c^2=\dfrac{T}{\rho}$ is not a constant.

\vspace{1cm}

$\bullet$ \textbf{Problem I:} If the density of a stressed string be $\dfrac{m}{x^2}$ where $x$ is the distance measured from a point in the line of propagation, the ends of the string being $x=l_1$, $x=l_2$. Show that the frequency equation is $$\dfrac{4p^2}{c^2}=1+\left\{\dfrac{2n\pi}{\log\left(\frac{l_2}{l_1}\right)}\right\}^2$$
where $c^2=\dfrac{T}{m}$, $T$ is the tension of the string and the vibration being transverse.

\vspace{.5cm}

\textbf{Solution:} The differential equation of vibrating string is
$$\frac{\partial^2y}{\partial t^2}=\frac{T^2}{\rho}\frac{\partial^2y}{\partial x^2}=c^2x^2\frac{\partial^2y}{\partial x^2}$$
Let us write the solution as
$$y=f(x)\cos(pt+\epsilon)$$
then the differential equation for $f(x)$ is
$$x^2\frac{d^2f}{dx^2}+\frac{p^2}{c^2}=0$$
putting $x=e^u$, the differential equation becomes
$$\frac{d^2f}{du^2}-\frac{df}{du}+\frac{p^2}{c^2}=0,$$
which has the solution 
$$f(x)=c_1e^{\frac{u}{2}}\cos\left(\sqrt{\mu}u+\epsilon\right)=c_1\sqrt{x}\cos\left(\sqrt{\mu}\log x+\epsilon\right),~~~~~\mu=\frac{p^2}{c^2}-\frac{1}{4}$$
\begin{eqnarray}
\mbox{As}~~y=0~~\mbox{at}~~x=l_1,~~\mbox{so}~~c_1\sqrt{l_1}\cos\left(\sqrt{\mu}\log l_1+\epsilon\right)=0\nonumber\\
\mbox{Also}~~y=0~~\mbox{at}~~x=l_2,~~\mbox{so}~~c_1\sqrt{l_2}\cos\left(\sqrt{\mu}\log l_2+\epsilon\right)=0\nonumber
\end{eqnarray}
\begin{eqnarray}
\mbox{Hence}~~\sqrt{\mu}\log l_1+\epsilon=(2s+1)\frac{\pi}{2}\nonumber\\
\sqrt{\mu}\log l_2+\epsilon=(2s'+1)\frac{\pi}{2}\nonumber
\end{eqnarray}
\begin{eqnarray}
\therefore \sqrt{\mu}\log\frac{l_1}{l_2}=2(s'-s)\frac{\pi}{2}=n\pi~~~~~~~~~~~~~~~~~~~~~~~\nonumber\\
\mbox{or},~~\left(\frac{p^2}{c^2}-\frac{1}{4}\right)^{\frac{1}{2}}=n\pi\log\frac{l_2}{l_1}\implies\frac{4p^2}{c^2}=1+\left\{\frac{2n\pi}{\log\frac{l_1}{l_2}}\right\}^2\nonumber\\
~~~~~~~~~~~~~~~~~~~~~~~~~~~~~~~~~~~~~~~~~~~~~~~~~~~~~~~~~~~~~~~~~~~~~~[\mbox{Proved}]\nonumber
\end{eqnarray}

\vspace{1cm}

$\bullet$ \textbf{Problem II:} If a string of length `$l$' and tension $T_0$ is stretched between two points and is not uniform but of line density $\dfrac{\rho_0}{(1+kx)^2}$ where $x$ is the distance from one end then show that the transverse vibration is of period $\dfrac{2\pi}{n}$ where $\sqrt{4n^2-k^2c^2}\log(1+kl)=2ick\pi~~~(4n^2>k^2c^2),~~~c^2=\dfrac{T_0}{\rho_0}$ and `$i$' is a +ve integer. Examine the case $i=0$.

\vspace{.5cm}

\textbf{Solution:} The differential equation of the vibrating string 
$$\frac{\partial^2y}{\partial t^2}=\frac{T^2}{\rho}\frac{\partial^2y}{\partial x^2}=c^2(1+kx)^2\frac{\partial^2y}{\partial x^2}=c^2k^2z^2\frac{\partial^2y}{\partial z^2},~~~1+kx=z$$
$$\mbox{or},~~~z^2\frac{\partial^2y}{\partial z^2}-\frac{1}{c^2k^2}\frac{\partial^2y}{\partial t^2}=0$$
Let the solution be $y=f(z)\cos(pt+\epsilon)$
$$\therefore z^2\frac{d^2f}{dz^2}+\frac{1}{c^2k^2}p^2=0$$
The solution is $f(z)=c_1\sqrt{z}\cos(\sqrt{\mu}\log z+\epsilon),~~~\mu=\dfrac{p^2}{c^2k^2}-\dfrac{1}{4}$
$$\therefore y=c_1\sqrt{1+kx}\cos\left\{\sqrt{\mu}\log(1+kx)+\epsilon\right\}$$
$$\mbox{As}~~y=0,~~x=0\implies\epsilon=(2s+1)\frac{\pi}{2}$$
$$\mbox{Also},~~y=0,~~x=l\implies\sqrt{\mu}\log(1+kl)+\epsilon=(2s'+1)\frac{\pi}{2}$$
$$\therefore\sqrt{\mu}\log(1+kl)=i\pi,~~~i~~\mbox{is a +ve integer.}$$
$$\therefore\sqrt{4p^2-k^2c^2}\log(1+kl)=2ick\pi$$
When $i=0$, then $4p^2=k^2c^2$ i.e, $p=\dfrac{kc}{2}$.

\vspace{1cm}

$\bullet$ \textbf{Problem III:} A transverse force $Y\sin pt$ acts at the point of junction of two strings of different mass per unit length which are joined and stretched between two points  at a distance `$l$' apart. The length of the strings being `$b$' and `$l-b$'. Prove that if $c_1$ and $c_2$ be the velocity of transverse wave in the two strings the displacement of the point of junction of the string at time `$t$' is $\dfrac{Y\sin pt}{\left[\dfrac{pT}{c_1}\cot\dfrac{pb}{c_1}+\dfrac{pT}{c_2}\cot\dfrac{p(l-b)}{c_2}\right]}$ where $T$ is the tension of the string. 

\vspace{.5cm}

\textbf{Solution:} 


\begin{figure}[h!]
	\centering
	\includegraphics[scale=0.7]{photo7.pdf}
\end{figure}

For the part $OP$ the solution is 
$$y_1=\left(A\cos\dfrac{px}{c_1}+B\sin\dfrac{px}{c_2}\right)\sin pt,~~~0\leq x\leq b$$
For the part $PA$ of the string the solution is 
$$y_2=\left[C\cos\frac{p(l-x)}{c_2}+D\sin\frac{p(l-x)}{c_2}\right]\sin pt,~~~b\leq x\leq l.$$
The boundary conditions: $y_1=0$ at $x=0$ gives $A=0$.

$y_2=0$ at $x=l$ gives $c=0.$
$$\therefore y_1=B\sin\frac{px}{c_1}\sin pt~~~\mbox{and}~~~y_2=D\sin\left\{\frac{p(l-x)}{c_2}\right\}\sin pt$$
The condition: $y_1=y_2$ at $x=b$ $\implies$
\begin{equation}
B\sin\frac{pb}{c_1}=D\sin\frac{p(l-b)}{c_2}\label{eq3..14}
\end{equation}
Now, 
$$Y\sin pt=T\left(\frac{dy_1}{dx}\right)_{x=b}-T\left(\frac{dy_2}{dx}\right)_{x=b}$$
\begin{eqnarray}
\therefore Y&=&pT\left[\frac{B}{c_1}\cos\frac{pb}{c_1}+\frac{D}{c_2}\cos\frac{p(l-b)}{c_2}\right]\nonumber\\
\mbox{or},~~~\frac{Y}{pT}&=&D\sin\frac{p(l-b)}{c_2}\left[\frac{1}{c_1}\cot\frac{pb}{c_1}+\frac{1}{c_2}\cot\left\{\frac{p(l-b)}{c_2}\right\}\right]\nonumber\\
\therefore D&=&\frac{Y}{\sin\frac{p(l-b)}{c_2}}\frac{1}{\left[\frac{pT}{c_1}\cot\frac{pb}{c_1}+\frac{pT}{c_2}\cot\left\{\frac{p(l-b)}{c_2}\right\}\right]}\nonumber\\
\therefore y_2&=&D\sin\frac{p}{c_2}(l-x)\sin pt\nonumber
\end{eqnarray}
$$\mbox{Hence},~~~\left(y_2\right)_b=\frac{Y\sin pt}{\left[\frac{pT}{c_1}\cot\frac{pb}{c_1}+\frac{pT}{c_2}\cot\left\{\frac{p(l-b)}{c_2}\right\}\right]}$$

\vspace{1cm}

$\bullet$ \textbf{Problem IV:} The ends of a stretched uniform string of length `$l$' are attached to two small rings without mass which can slide on two parallel rods at right angle to the string. The middle point of the string is acted on by the transverse force $F\sin pt$. Prove that the force vibration at a distance $\xi$ from either end is given by 
$$y=-\frac{cF}{2pT}\frac{\cos\frac{p\xi}{c}\sin pt}{\sin\left(\frac{pl}{2c}\right)}$$

\vspace{.5cm}

\textbf{Solution:}


\begin{figure}[h!]
	\centering
	\includegraphics[scale=0.7]{roshni2.pdf}
\end{figure}

Let
\begin{eqnarray}
y_1&=&\left[A\cos\frac{px}{c}+B\sin\frac{px}{c}\right]\sin pt,~~~0\leq x\leq\frac{l}{2}\nonumber\\
y_2&=&\left[C\cos\frac{p(l-x)}{c}+D\sin\frac{p(l-x)}{c}\right]\sin pt,~~~\frac{l}{2}\leq x\leq l\nonumber
\end{eqnarray}
Since the rings are massless, we can imagine the string upto the rings so that $\dfrac{dy_1}{dx}=0$ at $x=0$, $\dfrac{dy_2}{dx}=0$ at $x=l$.
$$\therefore B=D=0$$.
$$y_1=y_2~~\mbox{at}~~x=\frac{l}{2}\implies A=C$$.
\begin{eqnarray}
\mbox{Also},~~~F\sin pt&=&\left[T\left(\frac{dy_1}{dx}\right)_{x=\frac{l}{2}}-T\left(\frac{dy_2}{dx}\right)_{x=\frac{l}{2}}\right]\nonumber\\
&=&-2T\frac{p}{c}A\sin\left(\frac{pl}{2c}\right)\sin pt\nonumber
\end{eqnarray}
$$\therefore A=-\frac{Fc}{2Tp}\frac{1}{\sin\left(\frac{pl}{2c}\right)}$$
$$\therefore y_1=-\frac{Fc}{2pT}\frac{\cos\left(\frac{px}{c}\right)\sin pt}{\sin\left(\frac{pl}{2c}\right)}$$
$$\therefore\mbox{at}~~x=\xi,~~\mbox{displacement}=-\frac{Fc}{2pT}\frac{\cos\left(\frac{p\xi}{c}\right)\sin pt}{\sin\frac{pl}{2c}}$$

\vspace{1cm}

$\bullet$ \textbf{Problem V:} If a stretched string held at its middle point be drawn aside at a point of quadric section and released from rest. Prove that in the ensuing vibration the energy in the harmonic of order `$r$' is proportional to $r^{-2}\sin^2\left(\dfrac{r\pi}{4}\right)\sin^4\left(\dfrac{r\pi}{8}\right)$.

\vspace{.5cm}

\textbf{Solution:}


\begin{figure}[h!]
	\centering
	\includegraphics[scale=0.4]{photo9.pdf}
\end{figure}

Let the solution be 
$$y=\left(A\cos\frac{px}{c}+B\sin\frac{px}{c}\right)\cos(pt+\epsilon)$$
\begin{eqnarray}
\mbox{As},~~~y&=&0,~~x=0\implies A=0\nonumber\\
y&=&0,~~x=l\implies\sin\frac{pl}{c}=0~~\mbox{i.e,}~~p=\frac{S\pi c}{l},~~~~~S~\mbox{is an integer including zero.}\nonumber
\end{eqnarray}
\begin{eqnarray}
\therefore y&=&B\sin\frac{S\pi x}{l}\cos\left(\frac{S\pi ct}{l}+\epsilon\right)\nonumber\\
&=&\sin\left(\frac{S\pi x}{l}\right)\left[C_s\cos\left(\frac{S\pi ct}{l}\right)+D_s\sin\left(\frac{S\pi ct}{l}\right)\right]\nonumber
\end{eqnarray}
$$\dot{y}=0~~\mbox{at}~~t=0~~\mbox{gives}~~D_s=0$$
\begin{equation}
\therefore y=C_s\sin\left(\frac{S\pi x}{l}\right)\cos\left(\frac{S\pi ct}{l}\right)\label{eq3..15}
\end{equation}
Initially at $t=0$, $y=y_0$, given by
\begin{eqnarray}
y_0&=&\frac{4\gamma x}{l},~~~0\leq x\leq\frac{l}{4}\nonumber\\
&=&\frac{\gamma\left(\frac{l}{2}-x\right)}{\frac{l}{4}},~~~\frac{l}{4}\leq x\leq\frac{l}{2}\nonumber\\
&=&0,~~~\frac{l}{2}\leq x\leq l\label{eq3..16}
\end{eqnarray}
From (\ref{eq3..15}), 
$$y_0=C_s\sin\left(\frac{S\pi x}{l}\right)$$
$$\mbox{or},~~~C_s\int\sin^2\left(\frac{S\pi x}{l}\right)dx=\int y_0\sin\left(\frac{S\pi x}{l}\right)dx$$
\begin{eqnarray}
C_s\frac{l}{2}&=&\int_{0}^{\frac{l}{4}}\frac{4\gamma x}{l}\sin\left(\frac{S\pi x}{l}\right)+\int_{\frac{l}{4}}^{\frac{l}{2}}\frac{4\gamma}{l}\left(\frac{l}{2}-x\right)\sin\left(\frac{S\pi x}{l}\right)dx\nonumber\\
&=&\frac{4\gamma}{l}\bigg\{-\left(\frac{l}{S\pi}\right)\left[x\cos\left(\frac{S\pi x}{l}\right)\right]_0^{\frac{l}{4}}+\int_{0}^{\frac{l}{4}}\left(\frac{l}{S\pi}\right)\cos\left(\frac{S\pi x}{l}\right)dx\nonumber\\
&~&+\left[-\left(\frac{l}{2}-x\right)\frac{l}{S\pi}\cos\left(\frac{S\pi x}{l}\right)\right]_{\frac{l}{4}}^{\frac{l}{2}}-\int_{\frac{l}{4}}^{\frac{l}{2}}\frac{l}{S\pi}\cos\left(\frac{S\pi x}{l}\right)dx\bigg\}\nonumber\\
&=&\frac{4\gamma}{l}\bigg\{\cancel{-\frac{l}{S\pi}\frac{l}{4}\cos\frac{S\pi}{4}}+\frac{l^2}{S^2\pi^2}\sin\left(\frac{S\pi}{4}\right)+\cancel{\frac{l}{S\pi}\frac{l}{4}\cos\frac{S\pi}{4}}\nonumber\\
&~&~~~~~~~~~~~~~~~~~~~~~~~~~~~~~~~~~~-\frac{l^2}{S^2\pi^2}\sin\frac{S\pi}{2}+\frac{l^2}{S^2\pi^2}\sin\frac{S\pi}{4}\bigg\}\nonumber\\
&=&\frac{4\gamma}{l}\left(\frac{l^2}{S^2\pi^2}\right)\left[2\sin\frac{S\pi}{4}-\sin\frac{S\pi}{2}\right]\nonumber\\
&=&\frac{4\gamma l}{S^2\pi^2}\left[2\sin\frac{S\pi}{4}-\sin\frac{S\pi}{2}\right]\nonumber
\end{eqnarray}
\begin{eqnarray}
\therefore	C_s&=&\frac{8\gamma}{S^2\pi^2}.2\sin\frac{S\pi}{4}.2\sin^2\frac{S\pi}{8}\nonumber\\
&=&\frac{32\gamma}{S^2\pi^2}\sin\left(\frac{S\pi}{4}\right)\sin^2\left(\frac{S\pi}{8}\right)\nonumber
\end{eqnarray}
The kinetic energy is $=\frac{1}{2}\rho\int_{0}^{l}\left(\frac{\partial y}{\partial t}\right)^2dx$, potential energy is $=\frac{1}{2}\rho c^2\int_{0}^{l}\left(\frac{\partial y}{\partial x}\right)^2dx$.
\begin{eqnarray}
\mbox{As},~~~y&=&C_s\sin\left(\frac{S\pi x}{l}\right)\sin\left(\frac{S\pi ct}{l}\right)\nonumber\\
\therefore\frac{\partial y}{\partial x}&=&C_s.\frac{S\pi}{l}.\cos\left(\frac{S\pi x}{l}\right)\sin\left(\frac{S\pi ct}{l}\right)=\lambda_1\cos\left(\frac{S\pi x}{l}\right)\sin\left(\frac{S\pi ct}{l}\right)\nonumber\\
\frac{\partial y}{\partial t}&=&C_s.\frac{S\pi c}{l}.\sin\left(\frac{S\pi x}{l}\right).\cos\left(\frac{S\pi ct}{l}\right)=\lambda_2\sin\left(\frac{S\pi x}{l}\right)\cos\left(\frac{S\pi ct}{l}\right)\nonumber\\
&~&~~~~~~~~~~~~~~~~~~~~~~~~~~~~~~~~~~~~~~~~~~~~~~~~~~~~~~~~~~~~~~~~~~~~~~~~~(\lambda_2=c\lambda_1)\nonumber
\end{eqnarray}
\begin{eqnarray}
\therefore\mbox{Kinetic energy (K.E.)}&=&\frac{1}{2}\rho\lambda_2^2\int_{0}^{l}\sin^2\left(\frac{S\pi x}{l}\right)\cos^2\left(\frac{S\pi ct}{l}\right)dx\nonumber\\
&=&\frac{1}{2}\rho\lambda_2^2\cos^2\left(\frac{S\pi ct}{l}\right).\frac{1}{2}\int_{0}^{l}\left[1+\cos\left(\frac{2S\pi x}{l}\right)\right]dx\nonumber\\
&=&\frac{1}{4}\rho\lambda_2^2\cos^2\left(\frac{S\pi ct}{l}\right).l\nonumber\\
&=&\frac{l}{4}\rho\lambda_2^2\cos^2\left(\frac{S\pi ct}{l}\right)\nonumber
\end{eqnarray}
\begin{eqnarray}
\mbox{Potential energy (P.E.)}&=&\frac{1}{4}\rho c^2\lambda_1^2\sin^2\left(\frac{S\pi ct}{l}\right)\int_{0}^{l}\left(1-\cos\left(\frac{S\pi x}{l}\right)\right)dx\nonumber\\
&=&\frac{l}{4}\rho c^2\lambda_1^2\sin^2\left(\frac{S\pi ct}{l}\right)\nonumber\\
&=&\frac{l}{4}\rho\lambda_2^2\sin^2\left(\frac{S\pi ct}{l}\right)\nonumber
\end{eqnarray}
$\therefore\mbox{K.E.+P.E.}=\dfrac{l}{4}\rho\lambda_2^2$.

\vspace{1cm}

$\bullet$ \textbf{Problem VI:} A uniformly stretched string of which the extremities are fixed starts from rest in the form $y=A\sin\left(\dfrac{m\pi x}{l}\right)$, where $m$ is an integer, $l$ the distance between two fixed ends. Prove that if the resistance of air be taken into account and be assumed to be $2K$ times the momentum per unit length then the displacement after any time $t$ is given by
$$y=Ae^{-Kt}\left(\cos m't+\frac{K}{m'}\sin m't\right)\sin\left(\frac{m\pi x}{l}\right)$$
where $m'^2=\dfrac{m^2\pi^2c^2}{l^2}-K^2$, with $c$ the velocity of the wave of transverse vibration of the string.

\vspace{.5cm}

\textbf{Solution:} In the resisting medium the equation of motion is 
$$\frac{\partial^2y}{\partial t^2}=c^2\frac{\partial^2y}{\partial x^2}-2K\frac{\partial y}{\partial t}$$
\begin{equation}
\mbox{We assume},~~~y=f(t).\sin\left(\frac{m\pi x}{l}\right)\nonumber
\end{equation}
So the differential equation becomes 
$$f''(t)\sin\left(\frac{m\pi x}{l}\right)=-c^2.\left(\frac{m^2\pi^2}{l^2}\right)\sin\left(\frac{m\pi x}{l}\right).f(t)-2Kf'(t)\sin\left(\frac{m\pi x}{l}\right)$$
$$\mbox{or},~~~f''(t)+2Kf'(t)+\frac{c^2m^2\pi^2}{l^2}f(t)=0$$
If $z=f(t)$ then 
$$\frac{d^2z}{dt^2}+2K\frac{dz}{dt}+\frac{c^2m^2\pi^2}{l^2}z=0$$
the auxiliary equation is
$$\alpha^2+2K\alpha+\frac{c^2m^2\pi^2}{l^2}=0$$
$$\mbox{or},~~~\alpha=\frac{-2K\pm\sqrt{4K^2-4\frac{c^2m^2\pi^2}{l^2}}}{2}=-K\pm im'$$
\begin{eqnarray}
\therefore z&=&f(t)=e^{-Kt}\left[c_1\cos m't+c_2\sin m't\right]\nonumber\\
\therefore y&=&e^{-Kt}\left[c_1\cos m't+c_2\sin m't\right]\sin\left(\frac{m\pi x}{l}\right)\nonumber
\end{eqnarray}
$$\mbox{But at}~~t=0,~~y=A\sin\frac{m\pi x}{l}\implies c_1=A$$
$\dot{y}=0$ at $t=0$
\begin{eqnarray}
\therefore\frac{dy}{dt}&=&-Ke^{-Kt}\left[c_1\cos m't+c_2\sin m't\right]\sin\left(\frac{m\pi x}{l}\right)\nonumber\\
&~&~~~~~~~~+e^{-Kt}\left[-c_1m'\sin m't+c_2m'\cos m't\right]\sin\left(\frac{m\pi x}{l}\right)\nonumber
\end{eqnarray}
$$\mbox{So at}~~t=0,~~\dot{y}=0\implies-Kc_1+c_2m'=0~~\mbox{i.e,}~~c_2=\frac{AK}{m'}$$
$$\therefore y=e^{-Kt}A\left(\cos m't+\frac{K}{m'}\sin m't\right)\sin\left(\frac{m\pi x}{l}\right)$$

\vspace{1cm}

$\bullet$ \textbf{Problem VII:} Two uniform strings are attached together and stretched in a straight line between two fixed points with tension $T$ and carry a particle of mass $M$ attached at the point of junction. The line densities are $\rho$ and $\rho'$ and lengths are $l$ and $l'$. Show that $T=c^2\rho=c'^2\rho'$. The period $\dfrac{2\pi}{n}$ of transverse vibration are given by 
$$M.n=c\rho\cot\left(\frac{nl}{c}\right)+c'\rho'\cot\left(\frac{nl'}{c'}\right)$$

\vspace{.5cm}

\textbf{Solution:} 

\begin{figure}[h!]
	\centering
	\includegraphics[scale=0.6]{photo10.pdf}
\end{figure}

Let
\begin{eqnarray}
y_1&=&\left(A\cos\frac{px}{c_1}+B\sin\frac{px}{c_1}\right)\sin pt,~~~~0<x<l\nonumber\\
y_2&=&\left[C\cos\frac{p(l+l'-x)}{c_2}+D\sin\frac{p(l+l'-x)}{c_2}\right]\sin pt,~~~~l<x<l+l'\nonumber
\end{eqnarray}
be the solutions at the two parts of the strings.
\begin{eqnarray}
\mbox{Now},~~~y_1&=&0,~~x=l\implies A=0\nonumber\\
y_2&=&0,~~x=l+l'\implies C=0\nonumber\\
y_1&=&y_2~~\mbox{at}~~x=l\implies B\sin\frac{pl}{c_1}=D\sin\frac{pl'}{c_2}\nonumber
\end{eqnarray}
Now equation of motion of $M$ is 
$$M\left(\frac{d^2y}{dt^2}\right)_{x=l}=T\left[\left(\frac{dy_2}{dx}\right)_{x=l}-\left(\frac{dy_1}{dx}\right)_{x=l}\right]$$
$$\mbox{or},~~Mp\sin\frac{pl}{c_1}p^2\sin pt=T\left[\frac{Dp}{c_2}\cos\frac{pl'}{c_2}+\frac{Bp}{c_1}\cos\frac{pl}{c_1}\right]\sin pt$$
\begin{eqnarray}
\mbox{or},~~M.p&=&T\left[\frac{1}{c_1}\cot\frac{pl}{c_1}+\frac{1}{c_2}\cot\frac{pl'}{c_2}\right]\nonumber\\
&=&c_1\rho_1\cot\frac{pl}{c_1}+c_2\rho_2\cot\frac{pl'}{c_2}\nonumber
\end{eqnarray}




\section{Longitudinal Vibration of a Stretched Elastic String:}
We take the $x$-axis along the equilibrium position of the stretched string. Let the end points be then given by $x=0$ and $x=l$ so that `$l$' is the stretched equilibrium length. Let $\rho$ be the line density in the stretched equilibrium position. Suppose $l_0$ be the natural length of the string and $\rho_0$ is the line density when unstretched. Since the mass of the string remains unaltered we have $l_0\rho_0=l\rho$.


\begin{figure}[h!]
	\centering
	\includegraphics[scale=0.7]{photo11.pdf}
\end{figure}



Let us consider an element of length $\delta x$ of the string in the stretched equilibrium position, the end points being given by $x$ and $x+\delta x$. Now, in course of vibration let the end point $x$ be shifted to the point $x+\xi$, ($\xi$ is very small) and the point $x+\delta x$ will be shifted to the position $x+\delta x+\xi+\dfrac{\partial \xi}{\partial x}\delta x$, so that the length of the element becomes $\delta x+\dfrac{\partial\xi}{\partial x}\delta x$. If $\delta x_0$ be the unstretched length of the element, whose length is $\delta x$ in the stretched equilibrium position then the tension of the string at any time is given by 
$$T=E\frac{\delta x+\frac{\partial\xi}{\partial x}\delta x-\delta x_0}{\delta x_0},$$
where $E$ is the Young's modulus of the material of the string.
$$\mbox{Now},~~T=E\frac{\delta x-\delta x_0}{\delta x_0}+E\frac{\partial\xi}{\partial x}\frac{\delta x}{\delta x_0}=T_0+E\frac{\partial\xi}{\partial x}\frac{\delta x}{\delta x_0}$$
where $T_0$ is the tension of the string in the stretched equilibrium position and has the same value for every point of the string.

As $\dfrac{\delta x}{\delta x_0}=\dfrac{l}{l_0}$, so putting $E\dfrac{l}{l_0}=E'$, we get $T=T_0+E'\dfrac{\partial\xi}{\partial x}$, $T_0$ is constant.

Let us write down the equation of motion of the element $\delta x$ of the string
$$\rho\delta x\ddot{\xi}=-T+T+\frac{\partial T}{\partial x}\delta x=\frac{\partial}{\partial x}\left[T_0+E'\frac{\partial\xi}{\partial x}\right]\delta x$$
$$\mbox{or},~~~\frac{\partial^2\xi}{\partial t^2}=\frac{E'}{\rho}\frac{\partial^2\xi}{\partial x^2}=a^2\frac{\partial^2\xi}{\partial x^2},~~~\mbox{with}~~a^2=\frac{E'}{\rho}.$$
Here `$a$' is the velocity of propagation of the disturbance refer to the stretched equilibrium position. Also we have 
$$a^2=\frac{E'}{\rho}=E\frac{l}{l_0}.\frac{1}{\frac{\rho_0l_0}{l}}=\left(\frac{E}{\rho_0}\right)\left(\frac{l}{l_0}\right)^2$$
The boundary conditions are $\xi=0$ when $x=0$ and $\xi=0$ when $x=l$.

Let $\xi=f(x)\cos(pt+\epsilon)$ be the solution, then 
$$f(x)=A\cos\frac{px}{a}+B\sin\frac{px}{a},~~~A~\mbox{and}~B~\mbox{are constants}.$$
$$\therefore\xi=\left(A\cos\frac{px}{a}+B\sin\frac{px}{a}\right)\cos(pt+\epsilon)$$
$$\xi=0,~~x=0\implies A=0.$$
$$\xi=0,~~x=l\implies\sin\frac{pl}{a}=0~~\mbox{i.e,}~~\frac{pl}{a}=s\pi\implies p=\frac{s\pi a}{l}$$
$$\therefore p=\frac{s\pi}{l}\sqrt{\frac{E}{\rho_0}}\frac{l}{l_0}=\frac{s\pi}{l_0}\sqrt{\frac{E}{\rho_0}}$$



\section{Transverse Vibration of a Stretched Membrence}
We take the plane of the membrence in the equilibrium position as the plane of $xy$. Assuming that the membrence is subjected to stretching. Let the tension be $T_1$ and $z$ be the normal displacement of the point of the membrence which occupy the position $(x,y)$ when in equilibrium. Let us construct a prism on the element of area $dxdy$ on the $xy$ plane and this prism intersects the surface of the displaced membrence in a figure which is approximately rectangular and lengths of the sides of this rectangle are 
$$\sqrt{1+\left(\frac{\partial z}{\partial x}\right)^2}dx,~~~\sqrt{1+\left(\frac{\partial z}{\partial y}\right)^2}dy$$ 
\begin{eqnarray}
\mbox{i.e,}~~\left[1+\frac{1}{2}\left(\frac{\partial z}{\partial x}\right)^2\right]dx~&~&\mbox{and}~~~\left[1+\frac{1}{2}\left(\frac{\partial z}{\partial y}\right)^2\right]dy\nonumber\\
&~&\left(\mbox{assuming }~\frac{\partial z}{\partial x}~\mbox{and}~\frac{\partial z}{\partial y}~\mbox{to be very small}\right)\nonumber
\end{eqnarray}
Therefore, the area of the portion is
$$\left[1+\frac{1}{2}\left\{\left(\frac{\partial z}{\partial x}\right)^2+\left(\frac{\partial z}{\partial y}\right)^2\right\}\right]dxdy,$$ 
neglecting the quantities of higher order than the 2nd. Therefore, in the displaced position the increment of the area of the portion of the membrence which occupy the position $dxdy$ in the equilibrium position is  $$\frac{1}{2}\left\{\left(\frac{\partial z}{\partial x}\right)^2+\left(\frac{\partial z}{\partial y}\right)^2\right\}dxdy.$$
Therefore, potential energy of the membrence which is the work done by the tension $T_1$ is 
$$V=\frac{1}{2}T_1\int\int\left\{\left(\frac{\partial z}{\partial x}\right)^2+\left(\frac{\partial z}{\partial y}\right)^2\right\}dxdy$$ 
The integral is taken over the area of the membrence in the equilibrium position. The kinetic energy of the membrence at any time is 
$$T=\frac{1}{2}\int\int\rho\left(\frac{\partial z}{\partial t}\right)^2dxdy,~~~(\rho~\mbox{is the uniform density of the membrence.})$$
from Hamilton's principle we have
$$\int_{t_1}^{t_2}dt.\delta(T-V)=0$$
\begin{eqnarray}
\mbox{or,}~~\int_{t_1}^{t_2}dt\int\int\left[\rho\frac{\partial z}{\partial t}\frac{\partial}{\partial t}\delta z-T_1\frac{\partial z}{\partial x}.\frac{\partial}{\partial x}(\delta z)-T_1\frac{\partial z}{\partial y}.\frac{\partial}{\partial y}(\delta y)\right]dxdy=0\nonumber\\
\mbox{or,}~~\int\int\left[\rho\frac{\partial z}{\partial t}\delta zdxdy\right]_{t_1}^{t_2}+\int_{t_1}^{t_2}dt\int\int\bigg[-\rho\frac{\partial^2 z}{\partial t^2}\delta z-T_1\frac{\partial}{\partial x}\left(\frac{\partial z}{\partial x}.\delta z\right)\nonumber\\
-T_1\frac{\partial}{\partial y}\left(\frac{\partial z}{\partial y}\delta z\right)+T_1\left(\frac{\partial^2z}{\partial x^2}+\frac{\partial^2z}{\partial y^2}\right)\delta z\bigg]dxdy=0\nonumber
\end{eqnarray}
the first term on the left hand side $=0$ since $\delta z=0$ when $t=t_1$ and $t=t_2$.
$$\mbox{Now,}~~\int\int\left[\frac{\partial}{\partial x}\left(\frac{\partial z}{\partial x}\delta z\right)+\frac{\partial}{\partial y}\left(\frac{\partial z}{\partial y}\delta z\right)\right]dxdy,$$
can be converted into surface integral integral taken over the boundary of the membrence as 
$$-\int\left(l\frac{\partial z}{\partial x}+m\frac{\partial z}{\partial y}\right)\delta zds$$
Here $(l,m,0)$ being the direction cosine of the inward drawn normal. As $\delta z=0$ on the boundary of the membrence (which is fixed) so the above surface integral vanishes. Thus we have 
$$\int_{t_1}^{t_2}dt\left[-\rho\frac{\partial^2z}{\partial t^2}+T_1\left(\frac{\partial^2z}{\partial x^2}+\frac{\partial^2 z}{\partial y^2}\right)\right]dxdy=0$$
Hence the equation of vibration of the membrence is
$$c^2\left(\frac{\partial^2z}{\partial x^2}+\frac{\partial^2z}{\partial y^2}\right)=\frac{\partial^2z}{\partial t^2},~~~~c^2=\frac{T_1}{\rho}$$






\section{Vibration of a rectangular membrence}

\begin{figure}[h!]
	\centering
	\includegraphics[scale=0.4]{photo12.pdf}
\end{figure}


We consider the equation of the boundary as 
$$x=0,~~x=2a,~~y=0,~~y=2b.$$
We take the equilibrium plane of the membrence as $xoy$ plane. So the equation of transverse vibration of the membrence is given by 
$$\frac{\partial^2z}{\partial t^2}=c^2\left(\frac{\partial^2z}{\partial x^2}+\frac{\partial^2z}{\partial y^2}\right),~~~c^2=\frac{T_1}{\rho}$$ 
Boundary conditions are $z=0$ when $x=0$, $x=2a$ and $y=0$ and $y=2b$.

Let us assume separable form of the solution of the wave equation as
$$z=F_1(x).F_2(y).F_3(t)$$
and we have
$$\frac{1}{F_3}\frac{d^2F_3}{dt^2}=c^2\left[\frac{1}{F_1}\frac{d^2F_1}{dx^2}+\frac{1}{F_2}\frac{d^2F_2}{dy^2}\right]$$
Due to vibrating nature of the membrence let 
$$F_3(t)=\cos(pt+\epsilon),$$
so that the above equation becomes
$$-p^2=c^2\left[\frac{1}{F_1}\frac{d^2F_1}{dx^2}+\frac{1}{F_2}\frac{d^2F_2}{dy^2}\right]$$
$$\mbox{or,}~~\frac{1}{F_1}\frac{d^2F_1}{dx^2}=-\frac{p^2}{c^2}-\frac{1}{F_2}\frac{d^2F_2}{dy^2}=\mbox{constant}=-m^2~(\mbox{say})$$
Considering these two ordinary differential equation of the form 
$$\frac{d^2F_i}{dx_i^2}+m^2F_i=0,$$\vspace{-0.6 cm}\\
where $i=1,~2$ and $x_1=x, ~x_2=y$.
Then we have
\begin{eqnarray}
F_1&=&A\cos mx+B\sin mx\nonumber\\
F_2&=&A'\cos ny+B'\sin ny\nonumber
\end{eqnarray} 
Thus the complete solution is 
$$z=(A\cos mx+B\sin mx)(A'\cos ny+B'\sin ny)\cos(pt+\epsilon)$$
The boundary conditions give: 
$$(i)~~z=0~\mbox{at}~x=0\implies A=0~~~~~~~~~~~~~~~~~~~~~~~~~~~~~~~~~~~~~~~~~~~~~~~~~~~~~~~~~~$$
\begin{eqnarray}
(ii)~~z=0~\mbox{at}~x=2a\implies&~&B\sin 2am=0\nonumber\\
\implies&~&\sin 2am=0~~(\because B\neq 0~\mbox{for nontrivial solution})\nonumber\\
&~&\therefore 2am=s\pi~~\mbox{i.e,}~~m=\frac{s\pi}{2a},~~s~\mbox{is an integer}\nonumber
\end{eqnarray}
\begin{eqnarray}
(iii)~~z=0~\mbox{at}~y=0\implies A'=0~~~~~~~~~~~~~~~~~~~~~~~~~~~~~~~~~~~~~~~~~~~~~~~~~~~~~~~\nonumber
\end{eqnarray}
\begin{eqnarray}
(iv)~~z=0~\mbox{at}~y=2b\implies&~& 2bn=s'\pi~~~(s'~\mbox{is an integer})~~~~~~~~~~~~~~~~~~~~~~~~\nonumber\\
&~&\mbox{i.e,}~~n=\frac{s'\pi}{2b}\nonumber
\end{eqnarray}
Therefore the complete solution with the given boundary condition is
$$z=c'\sin\frac{s\pi x}{2a}\sin\frac{s'\pi y}{2b}\cos(pt+\epsilon)$$
Substituting this value of $z$ in the equation of vibration we have 
\begin{eqnarray}
p^2=\frac{c^2\pi^2}{4}\left(\frac{s^2}{a^2}+\frac{s'^2}{b^2}\right),~~~\mbox{with}~s,~s'~\mbox{as integers}.\nonumber
\end{eqnarray}
This gives the frequency of vibration and $\frac{2\pi}{p}$ gives the period of vibration.


\section{Circular Membrence}
The plane of the membrance is chosen as $xy$-plane. The circular boundary is characterised by $x^2+y^2=a^2$. Now the equation of transverse vibration of the membrence is
$$\frac{\partial^2z}{\partial t^2}=c^2\left[\frac{\partial^2z}{\partial x^2}+\frac{\partial^2z}{\partial y^2}\right].$$  
Let $(r,\theta)$ be the polar co-ordinates of a point in the plane of the membrence with the centre of the circle as pole. Then the boundary is characterised by $r=a$. In polar co-ordinate the equation of transverse vibration of the membrence becomes 
$$\frac{\partial^2z}{\partial t^2}=c^2\left[\frac{\partial^2z}{\partial r^2}+\frac{1}{r}\frac{\partial z}{\partial r}+\frac{1}{r^2}\frac{\partial^2z}{\partial\theta^2}\right]$$
where $x=r\cos\theta$, $y=r\sin\theta$ and $c^2=\dfrac{T_1}{\rho}$. Here $T_1$ is the uniform tension and $\rho$ is the uniform surface density. 

The boundary condition is $z=0$ when $r=a$. Now assuming separable form of the solution as
$$z=F(r).f(\theta)\cos(pt+\epsilon)$$
We have from the equation of vibration 
$$-r^2\left[\frac{p^2}{c^2}+\frac{1}{F}\left\{\frac{d^2F}{dr^2}+\frac{1}{r}\frac{dF}{dr}\right\}\right]=\frac{1}{f}\frac{d^2f}{d\theta^2}=\mbox{Constant}=-m^2~(\mbox{say})$$
$$\therefore \frac{d^2f}{d\theta^2}+m^2f=0~~~\mbox{i.e,}~~f=A\cos(m\theta+\epsilon')$$
$$\mbox{and}~~~\frac{d^2F}{dr^2}+\frac{1}{r}\frac{dF}{dr}+\left(\frac{p^2}{c^2}-\frac{m^2}{r^2}\right)F=0,~~~~~~~~$$
which is Bessel equation of order $m$. The integral of this equation regular at the origin is 
$$F(r)=c''J_m\left(\frac{pr}{c}\right)$$
$$\therefore z=c'J_m\left(\frac{pr}{c}\right)\cos(m\theta+\epsilon')\cos(pt+\epsilon)$$
Since the boundary is fixed at $r=a$ so $z=0$ at $r=a$. This gives $J_m\left(\dfrac{pa}{c}\right)=0$, which gives $p$ and hence the frequency of vibration.

\vspace{1cm}

$\bullet$ \textbf{Problem:} A circular membrence of radius $a$ is fixed at the edge and is subjected to uniform tension $T_1$. Prove that in a normal displacement $z$, symmetrical about the centre, the potential energy is $V=T_1\int_{0}^{a}\left(\dfrac{\partial z}{\partial r}\right)^2\pi rdr$, assuming $z=\zeta\left(1-\frac{r^2}{a^2}\right)$. Obtain the approximate period of vibration.

\vspace{.5cm}

\textbf{Solution:} The expression for potential energy is
$$V=\frac{1}{2}T_1\int\int\left[\left(\frac{\partial z}{\partial x}\right)^2+\left(\frac{\partial z}{\partial y}\right)^2\right]dxdy$$
$$\mbox{Now,}~~\frac{\partial z}{\partial x}=\frac{\partial z}{\partial r}.\frac{\partial r}{\partial x}+\frac{\partial z}{\partial\theta}.\frac{\partial\theta}{\partial x}=\frac{\partial z}{\partial r}.\frac{\partial r}{\partial x}~~\left(\because\frac{\partial z}{\partial\theta}=0~\mbox{as}~z~\mbox{is symmetrical about the centre}\right)$$
$$\mbox{As}~~r^2=x^2+y^2~~~~~~~~~~~~~~~~~~~~~~~~~~~~~~~~~~~~~$$
$$\therefore r\frac{\partial r}{\partial x}=x~~~~\therefore\frac{\partial r}{\partial x}=\frac{x}{r}~~~~~~~~~~~~~~~~~~~$$
$$\mbox{Similarly,}~~~\frac{\partial r}{\partial y}=\frac{y}{r}~~~~~~~~~~~~~~~~~~~~~~~~~~~~~~~~~~~~~~~~~~~~~~~$$
$$\therefore\left(\frac{\partial z}{\partial x}\right)^2+\left(\frac{\partial z}{\partial y}\right)^2=\frac{(x^2+y^2)}{r^2}\left(\frac{\partial z}{\partial r}\right)^2=\left(\frac{\partial z}{\partial r}\right)^2$$
\begin{eqnarray}
\therefore V&=&\frac{1}{2}T_1\int_{\theta=0}^{2\pi}\int_{r=0}^{a}\left(\frac{\partial z}{\partial r}\right)^2dr.rd\theta\nonumber\\
&=&T_1\int_{0}^{a}\left(\frac{\partial z}{\partial r}\right)^2\pi rdr\nonumber\\
&=&T_1\int_{0}^{a}\frac{4\zeta^2r^2}{a^4}\pi rdr\nonumber\\
&=&\frac{4\pi\zeta^2}{a^4}T_1\int_{0}^{a}r^3dr\nonumber\\
&=&\pi T_1\zeta^2\nonumber
\end{eqnarray}
\begin{eqnarray}
\mbox{Kinetic Energy}~(T)&=&\frac{1}{2}\rho\int_{0}^{2\pi}\int_{0}^{a}\left(\frac{\partial z}{\partial t}\right)^2rd\theta dr\nonumber\\
&=&\rho\pi\int_{0}^{a}\left(\frac{\partial z}{\partial t}\right)^2rdr\nonumber\\
&=&\rho\pi\int_{0}^{a}\dot{\zeta}^2(t)\left(1-\frac{r^2}{a^2}\right)^2rdr\nonumber\\
&=&\rho\pi\dot{\zeta}^2(t)\int_{0}^{a}\left(1-\frac{2r^2}{a^2}+\frac{r^4}{a^4}\right)rdr\nonumber\\
&=&\rho\pi\dot{\zeta}^2(t)\left[\frac{a^2}{2}-\frac{a^2}{2}+\frac{a^2}{6}\right]\nonumber\\
&=&\frac{\pi}{6}\rho\dot{\zeta}^2(t)a^2\nonumber
\end{eqnarray}
From Lagrange's equation of motion
$$\frac{d}{dt}\left(\frac{\partial T}{\partial\dot{\zeta}}\right)-\frac{\partial T}{\partial\zeta}=-\frac{\partial V}{\partial\zeta}$$
$$\mbox{or,}~~\rho\pi a^2\frac{\ddot{\zeta}}{3}-0=-2\pi T_1\zeta\implies\ddot{\zeta}=-\frac{6T_1\zeta}{\rho a^2},~~\mbox{a simple harmonic motion (S.H.M.)}$$
$$\therefore\mbox{Time period}=\frac{2\pi}{\sqrt{\frac{6\pi}{\rho a^2}}}=\frac{2\pi\sqrt{\rho a^2}}{\sqrt{6\pi}}$$

\vspace{1cm}

$\bullet$ \textbf{Problem:} If $z=\zeta\left(1-\dfrac{r^2}{a^2}\right)\left(1+\dfrac{\beta r^2}{a^2}\right)$, then show that the period will be $2\pi\sqrt{\dfrac{\rho a^2}{20T_1}\dfrac{(\beta^2+5\beta+10)}{(\beta^2+2\beta+3)}}$.

\vspace{.5cm}

\textbf{Solution:} 
$$z=\zeta\left(1-\frac{r^2}{a^2}\right)\left(1+\frac{\beta r^2}{a^2}\right)$$
$$\frac{\partial z}{\partial r}=\zeta\left[-\frac{2r}{a^2}\left(1+\frac{\beta r^2}{a^2}\right)+\left(1-\frac{r^2}{a^2}\right)\frac{2\beta r}{a^2}\right]$$
$$\frac{\partial z}{\partial t}=\dot{\zeta}\left(1-\frac{r^2}{a^2}\right)\left(1+\frac{\beta r^2}{a^2}\right)~~~~~~~~~~~~~~~~~~$$
\begin{eqnarray}
\therefore\mbox{K.E.}=T&=&\frac{1}{2}\rho\int_{0}^{2\pi}\int_{0}^{a}\dot{\zeta}^2\left(1-\frac{r^2}{a^2}\right)^2\left(1+\frac{\beta r^2}{a^2}\right)^2rdrd\theta\nonumber\\
&=&\pi\rho\dot{\zeta}^2\int_{0}^{a}\left(1-\frac{2r^2}{a^2}+\frac{r^4}{a^4}\right)\left(1+\frac{\beta^2r^4}{a^4}+\frac{2\beta r^2}{a^2}\right)rdr\nonumber\\
&=&\frac{\pi\rho\dot{\zeta}^2a^2}{60}\left(\beta^2+5\beta+10\right)\nonumber
\end{eqnarray}
\begin{eqnarray}
V&=&T_1\pi\zeta^2\int_{0}^{a}\left[\frac{2r}{a^2}\left(\beta-\frac{\beta r^2}{a^2}-1-\frac{\beta r^2}{a^2}\right)\right]^2rdr\nonumber\\
&=&\frac{4T_1\pi\zeta^2}{12}\left(\beta^2+2\beta+3\right)\nonumber
\end{eqnarray}
Now, from Lagrange's equation of motion
$$\frac{d}{dt}\left(\frac{\partial T}{\partial\dot{\zeta}}\right)-\frac{\partial T}{\partial\zeta}=-\frac{\partial V}{\partial\zeta}$$
$$\mbox{or,}~~\frac{2\pi\rho\ddot{\zeta}a^2}{60}\left(\beta^2+5\beta+10\right)=-\frac{8T_1\pi\rho}{12}\left(\beta^2+2\beta+3\right)$$
$$\therefore\ddot{\zeta}=-\frac{20\pi}{\rho a^2}\frac{\left(\beta^2+2\beta+3\right)}{\left(\beta^2+5\beta+10\right)}\zeta\implies T=2\pi\sqrt{\frac{\rho a^2}{20T_1}\frac{(\beta^2+5\beta+10)}{(\beta^2+2\beta+3)}}$$
