%!TEX root = ../main.tex

\section{Introduction}
Smart contracts are an essential part of the ecosystem in many modern blockchain platforms.
%
Smart contracts allow developers to implement decentralized applications (DApps) that encode business logic on the blockchain, thereby facilitating a number of use cases.
For instance, smart contracts enable the creation of non-fungible token (NFT) marketplaces.
Artists use NFT marketplaces to auction their creations.
Furthermore, established companies and sport franchises, like Nike~\cite{nike-nft}, Budweiser~\cite{budweiser-nft}, Lacoste~\cite{lacoste-nft}, and the NBA~\cite{nba-nft} use these marketplaces to sell NFT collections to fans and investors all over the world. 
 
The Solana blockchain~\cite{solana} has become a key platform in the DApps and NFT space, because of its high performance and low transaction fees.
In comparison to the more established smart contract platform Ethereum~\cite{ethereum}, Solana can execute 100--1000 times more transactions per second~\cite{Rouhani2017-bg,Lee_undated-wi} while charging a fraction of a USD cent as a fee~\cite{solana,Lee_undated-wi}.
%
As a result, the number of all transactions in the Solana network significantly exceeds the number of all transactions made in Ethereum by a factor of 85\footnote{On April 12, 2023, Ethereum processed \num{1933} million (\url{https://etherscan.io}) transactions, while Solana already reached \num{164839} million transactions (\url{https://solscan.io}).}.

From a smart contract perspective, the Solana platform achieves a high transaction rate because its execution layer decouples program logic from state, i.e., smart contracts cannot store any dynamic state.
This enables Solana to execute transactions operating on different data in parallel. 
However, this also introduces new attack patterns that are specific to Solana.
In fact, attacks against Solana smart contracts already caused multi-million Dollar losses~\cite{White2022-mango,White2022-solend}---the popular Wormhole attack induced losses of up to 320 million USD~\cite{Goodin2022-wormhole}.

Smart contract security research ranges across different disciplines: from formal verification~\cite{Schneidewind2020-mj} and static analysis~\cite{Torres2018-je, Mossberg2019-xp} to dynamic analysis~\cite{Ding2021-qg}, with a high focus on the Ethereum platform.
However, Solana suffers from different vulnerabilities than Ethereum, and the aforementioned techniques are not applicable to Solana due to its unique features:
In comparison to Ethereum, the Solana blockchain is stateless and smart contracts have no direct association with the state.
The stateless nature of Solana's execution environment requires stricter handling of user input.
However, vulnerabilities often come from developers not checking security-critical properties in smart contracts, like missing transaction \emph{signer checks}.
Unlike Ethereum smart contracts, which implicitly trust their state if it is not compromised.
Moreover, the stateless approach of Solana and the impact on its security model is largely unexplored.

Research on Solana security and tooling is limited: At the time of writing, VRust~\cite{Cui2022-nm} is the only existing static analysis approach that covers Solana smart contracts.
VRust incorporates detection patterns for common vulnerabilities in Solana smart contracts and was able to detect 12 vulnerabilities in popular open-source smart contracts~\cite{Cui2022-nm}.
However, VRust suffers from several limitations:
\begin{inparaenum}[1)]
  \item it strictly requires source code to conduct analyses,
  \item it suffers from a high number of false alarms, and
  \item it does not provide an analyst with enough data to (re-)construct exploit transactions.
\end{inparaenum}
%
%
In contrast, fuzzing is a technique that does not suffer from any of these limitations~\cite{Rawat2017-lj,Aschermann2019-ha,aflpp,libafl}.
%
The fuzzing input given to the analysis target can also usually be crafted into exploit transactions.
%
Smart contract fuzzing is a valuable technique that has been extensively researched with promising results.~\cite{sfuzz, echidna, rodler2023efcf}.

\paragraph{Contributions}
In this work, we propose a solution to detect bugs in Solana called \tool: the \emph{first} binary-only coverage-guided fuzzer for Solana smart contracts.
We developed a set of bug detection oracles to facilitate the detection of Solana-specific smart contract bugs, namely 
\begin{inparaenum}[1)]
  \item missing signer checks, that is, the smart contract performing critical operations without checking for signatures,
  \item missing owner checks, which allow a smart contract to use untrusted data,
  \item arbitrary cross program invocation, i.e., a smart contract calls \emph{any} other smart contract,
  \item missing key checks, which, similar to the missing owner check, enables a smart contract to use spoofed accounts as system variables, and
  \item integer bugs.
\end{inparaenum}
In addition, we also design a generic oracle for \tool to detect vulnerabilities based on lamport gains.
Lamports are the smallest denomination of Solana's native currency SOL, and 1 SOL corresponds to \num{1000000000} (one billion) lamports.
We use this oracle to detect arbitrary leaking funds and lamport-theft.
%

Our extensive evaluation of \tool consists of several experiments.
%
In our first experiment, we test \tool with a set of vulnerable smart contracts provided by the community~\cite{Neodyme2021-vw}.
We demonstrate that our approach is capable of quickly detecting Solana-specific vulnerabilities.
Compared to VRust, \tool does not report any false alarm for the dataset provided in~\cite{Neodyme2021-vw}. In addition, \tool is also able to precisely trace back the vulnerability classes.

Next, we perform a large-scale bug-finding evaluation on all Solana smart contracts present on the mainnet on March 27, 2023, which amounts to a total of \noc smart contracts.
At the time of writing, this is the largest analyzed dataset of Solana smart contracts.
\tool reports 92 bugs in these smart contracts.
We analyzed 16 reports in-depth and confirmed that only 2 were false alarms, thus demonstrating the high accuracy of \tool in detecting bugs.
\tool is the only analysis tool available that is able to analyze these contracts on the mainnet. 
%

Third, our performance evaluation on 16 smart contracts from well-known bug bounty programs demonstrates that \tool is able to analyze complex smart contracts.
Here, \tool's generated transactions are able to consistently find new code paths in the programs. 
In this experiment, \tool reports a true bug and only has a single false alarm, i.e., a single wrongly reported vulnerability.
We summarize our contributions as follows:

\begin{itemize}
  \item We present the \emph{first} fuzzing architecture for Solana smart contracts.
  We conceptualize \tool (\Cref{sec:design}) around the original Solana runtime to \tool faithfully model runtime specifics, such as smart contract interaction.
  Moreover, this design guarantees reproducibility and validity of transactions that \tool generates:
  Every transaction that generates a vulnerability report can be replayed, e.g., on a test network.
  \item We design and implement new bug oracles to detect Solana-specific vulnerabilities (\Cref{sub:sol-vulnerabilities}).
  Due to our design choices, \tool detects impactful bugs in smart contracts regardless of source code availability.
  \item Our extensive evaluation on \noc smart contracts shows that \tool's bug oracles find bugs with a high precision and recall.
  This is the largest evaluation of the security landscape on the Solana mainnet.
  \item \tool detects the infamous Wormhole bug.
\end{itemize}