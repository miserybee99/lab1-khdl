% todo setup
\newcommand{\orange}[1]{\textcolor{orange}{#1}}
\newcommand{\green}[1]{\textcolor{green}{#1}}
\newcommand{\blue}[1]{\textcolor{blue}{#1}}
\newcommand{\gray}[1]{\textcolor{gray}{#1}}
\newcommand{\red}[1]{\textcolor{red}{#1}}

\newcommand{\todoinl}[1]{\todo[size=\small,inline]{TODO: #1}}
\newcommand{\note}[1]{\todo[size=\small,inline,color=green]{Note: #1}}

\newcommand{\says}[3]{\todo[size=\small,color=#2,inline,author=#1]{#3}}

\newcommand{\sam}[1]{\says{Oussama}{green}{#1}}
\newcommand{\jens}[1]{\says{Jens}{cyan}{#1}}
\newcommand{\sven}[1]{\says{Sven}{orange}{#1}}
\newcommand{\pascal}[1]{\says{Pascal}{gray}{#1}}
\newcommand{\lucas}[1]{\says{Lucas}{red}{#1}}
\newcommand{\ghassan}[1]{\says{Ghassan}{yellow}{#1}}

% hlfix command
\makeatletter%
\if@todonotes@disabled%
	\newcommand{\hlfix}[2][]{#2}%
	\newcommand{\hltodo}[2]{#1}%
\else%
	\newcommand{\hlfix}[2][fixme]{\texthl{#2}\todo{#1}}%
	\newcommand{\hltodo}[2]{\texthl{#1}\todo{#2}}%
\fi
\makeatother


\newcommand\tool{\textsc{FuzzDelSol}\xspace}
\newcommand\ledgersnap{ledger snapshot\xspace}
\newcommand\blockemu{blockchain emulator\xspace}

% TODO: NUMBER OF CONTRACTS ANALYZED
\newcommand\noc{\num{6049}\xspace}

\newcommand\crashes{\num{91}\xspace}
\newcommand\tpcrashes{\num{15}\xspace}
\newcommand\fpcrashes{\num{1}\xspace}

% circled numbers

\newcommand\circled[1]{%
    \tikz[baseline=(char.base)]{
        \node[shape=circle,draw,inner sep=.5pt] (char) {#1};
}}

\newcommand{\tno}{{\color{red}{$\times$}}\xspace}
\newcommand{\tyes}{{\color{teal}{$\checkmark$}}\xspace}
\newcommand{\tpossible}{{\color{cyan}{$\checkmark^{\ast}$}}\xspace}
\newcommand{\tdou}{{\color{cyan}{$\checkmark$ (2)}}\xspace}

% counter, environment and refs for challenges
% use with \begin{Challenge} \label[challenge]{<name>} \end{Challenge} and \Cref{<name>}
\newcounter{challenges}
\setcounter{challenges}{0}

\newenvironment{Challenge}{%
  \refstepcounter{challenges}%
}{}

\crefname{challenge}{challenge}{challenges}

\newcounter{cfinding}
% \newcommand{\finding}[1]{%
% 	\stepcounter{cfinding}%
% 	\medskip\noindent%
% 	\fbox{\parbox{\dimexpr\linewidth-.6em}{\textbf{Finding~\thecfinding: \normalfont\emph{#1}}}}%
% 	\par\medskip\noindent\ignorespaces%
% }

\crefname{cfinding}{finding}{findings}
\Crefname{cfinding}{Finding}{Findings}

\newcommand{\finding}[1]{%
	\refstepcounter{cfinding}%
	\paragraph{Finding~\thecfinding: #1}%
}

% Command to place a TikZ anchor at the current position
\newcommand{\settikzmark}[1]{%
  \tikz[overlay,remember picture,baseline] \coordinate (#1) at (0,0) {};}

\newcommand{\greenplus}{{\color{green} +}}

\newcommand{\highlight}[2]{%
  \draw[YellowGreen,line width=6pt,opacity=0.2]%
    ([yshift=1.5pt]#1) -- ([yshift=1.5pt]#2);%
}

\newcommand{\tikzlinehighlight}[3]{%
  %\node [fill=#1,inner ysep=4pt,fit=(#2) (#3),opacity=0.2] {};%
  \draw[#1,line width=9pt,opacity=0.2]%
  ([yshift=1.5pt]#2) -- ([yshift=1.5pt]#3);%
}
\newcommand{\tikzmultilinehighlight}[4]{%
  \node [fill=#1,inner ysep=4pt,fit= (#2) (#3) (#4),opacity=0.2] {};%
}
  %\node [above=1pt of (#2)] (bound) {};%

\newcommand{\diffhighlightadd}[2]{%
  \tikzlinehighlight{YellowGreen}{#1}{#2}
}
\newcommand{\diffhighlightaddmulti}[3]{%
  \tikzmultilinehighlight{YellowGreen}{#1}{#2}{#3}
}