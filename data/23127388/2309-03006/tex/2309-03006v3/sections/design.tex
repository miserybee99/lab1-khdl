\section{Overview of \NoCaseChange{\tool}}
\label{sec:design}

In this section, we introduce the design of \tool and its main components.
Furthermore, we describe how our design tackles the challenges mentioned in \Cref{sec:challenges}.

\paragraph{Intended Use of \tool}
FuzzDelSol aims to explore the prevalence of vulnerabilities in Solana programs. 
Although similar bytecode-based security studies have been conducted for other blockchain platforms, such as Ethereum~\cite{ethbmc,Schneidewind2020-mj}, or other application domains like Android~\cite{enck2011study}, there does not yet exist any comprehensive study about the security of Solana programs. 
Hence, for the first time, we aim to raise awareness for Solana program security with \tool. 
For 98\% of Solana programs, no source code is available (see \Cref{sec:solana-security-analysis}).
Hence, we argue that source code-based analysis techniques like VRust~\cite{Cui2022-nm} are not applicable to analyze the vast majority of Solana programs. 
%
Moreover, \tool can be used to find vulnerabilities with the intention of forming a better understanding of the vulnerability types.
This is necessary to develop appropriate countermeasures for Solana-specific vulnerabilities.
%
Further, Solana program developers may use \tool to vet closed-source third-party programs interacting with their own programs.
The same applies to users of Solana programs: \tool helps in ensuring that closed-source programs a user wants to invest in are secure, before investing funds. 

\paragraph{Overview}
Our high-level architecture is shown in \Cref{fig:architecture}. 
The main idea of \tool is to \circled{1} create a valid blockchain snapshot using the \emph{blockchain emulator}; comprising the Solana program to analyze, an \emph{attacker} account and additional accounts, e.g., user and non-executable data accounts, or executable programs.
The next component is the \emph{transaction generator} \circled{2}, which receives random and mutated bytes from a fuzzer to generate valid transactions. 
\tool executes these transactions in an instrumented Solana runtime \circled{3} called \emph{RunDelSol}.
In particular, we extended the original Solana runtime with patches to detect Solana-specific vulnerabilities (cf.~\Cref{sub:sol-vulnerabilities}) induced by the generated transactions.
Finally, the \emph{transaction evaluator} \circled{4} analyzes the aftermath of the transactions, and extracts valuable insights for following fuzzing iterations.
However, if the transactions signal an erroneous \ledgersnap, \tool generates a vulnerability report with information to reproduce this \ledgersnap. 

\begin{figure*}
\centering
\includegraphics[width=.8\linewidth, keepaspectratio]{figures/architecture.pdf}
\caption{\tool Design}
\label{fig:architecture}
\end{figure*}

\paragraph{Blockchain Emulator (\Cref{sec:blockchain-init})}
A challenge for fuzzing Solana programs is that program execution is largely dependent on the \ledgersnap. 
Thus, we developed a component called \emph{\blockemu}~\circled{1} to prepare the snapshot of the ledger available for analysis.
The modeled \ledgersnap contains the program being fuzzed as well as additional accounts that are relevant to the execution context.
Moreover, this component provides all public keys that can be passed to the program as input.
Operating on a valid \ledgersnap allows a program, for example, to manage and modify lamports and account data at runtime across multiple transactions. 
\tool uses the \blockemu to model a \ledgersnap for programs to operate on during program execution, thereby addressing~\Cref{c1}.
In addition, the \blockemu---along with RunDelSol---also enables \tool to address \Cref{c3}, since the public keys provided by the \blockemu include those of the sysvar accounts used by the program to retrieve cluster information. 
Furthermore, the \blockemu incorporates an account generator that creates attacker-controlled accounts containing malicious data to trigger Solana-specific vulnerabilities.
Therefore, the \blockemu supports addressing \Cref{c4}.
Finally, the \blockemu uses PDA seed structures obtained from the transaction evaluator to derive valid PDAs, which assists in tackling \Cref{c5}.

\paragraph{Transaction Generator (\Cref{sec:transaction-gen})}
For each fuzzing iteration, \tool \emph{mimics} real Solana transactions to find reproducible bugs.
However, Solana transactions contain structured data.
Thus, \tool incorporates a \emph{transaction generator}~\circled{2}, which \emph{transforms} the randomly generated bytes from a fuzzer into valid Solana transactions.
As the transaction generator produces valid and reproducible transactions, \tool covers \Cref{c2}.

\paragraph{RunDelSol (\Cref{sec:rundelsol})}
Effective fuzzing requires coverage feedback~\cite{afl, aflpp} to guide the generation of test inputs. 
Source code is commonly used to instrument a program for achieving accurate results in this feedback mechanism.
However, given the absence of source code for the large majority of Solana programs, we cannot rely on the source code.
Therefore, instead of instrumenting programs, \tool implements a specialized Solana runtime environment~\circled{3}, called \emph{RunDelSol}.
This environment---besides instantiating and executing Solana programs on the previously generated \ledgersnap---includes instrumentation to measure coverage.
Furthermore, it allows programs to invoke functions of passed sysvar accounts to retrieve \emph{cluster information} (addressing \Cref{c3} along with the \blockemu).   

Moreover, RunDelSol features a \emph{taint tracking engine} to trace the data-flow during the execution of a program.
This enables us to implement Solana program vulnerability detectors or \emph{bug oracles}, as well as Solana-specific runtime information extractors which extract PDA seed structures.
Each oracle aims to detect a potential vulnerability without source code information and uses the taint tracking engine differently, and therefore tackling \Cref{c4}. 
Moreover, the Solana-specific runtime information extractors of RunDelSol (along with the transaction evaluator, see below) allow \tool to address  \Cref{c5} by extracting PDA seed structures.

Moreover, RunDelSol can invoke multiple programs using CPI and run them in separate eBPF VMs for on-chain programs.
In the case of native programs, RunDelSol executes them in the Solana runtime.
Otherwise, for invoking on-chain programs, RunDelSol traces their data-flow using the taint tracking engine independently to the callee program.
This allows the oracle to detect vulnerabilities across multiple CPI invocations.
Enabling programs to call other programs using CPI and analyzing their interaction allows \tool to solve \Cref{c6}.

\paragraph{Transaction Evaluator (\Cref{sec:transaction-eval})}
Transactions can impact the state of the blockchain.
%
For correct adjustment of the \ledgersnap and preparation of the next fuzzing iteration, the \emph{transaction evaluator} \circled{4} extracts relevant information from RunDelSol after the execution of transactions, including PDA seed structures, eBPF VM signals, and feedback information for the fuzzer, e.g., coverage.
Next, the transaction evaluator forwards the information to the fuzzer and the \blockemu and decides whether it should re-generate the \ledgersnap for subsequent fuzzing iterations, taking into account the newly received semantic program information. 
As a result, the transaction evaluator, together with RunDelSol, allows addressing \Cref{c5}. 
