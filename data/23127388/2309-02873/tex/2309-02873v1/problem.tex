% !TEX root = main.tex

\section{Problem Setting} \label{section:problem}
In this section, we formulate our problem and demonstrate how the concepts in Sec. \ref{section:background} are related to it.
To this end, consider system \eqref{eq:dyn} with outputs $y_n$ (cf Eq.\eqref{eq:obs}).
Our goal is to make accurate predictions for $y_n$. 
Here, we consider the situation, where access to an inaccurate physics-based approximation $(\hat s_n)_{n=0}^N \in \mathbb{R}^{d_s}$ of the outputs is provided by a black-box simulator.
Thus, access to the simulator's latent states or the dynamics is not available. 
We propose to leverage the concepts in Sec. \ref{section:background} and learn how to reconstruct latent states from the simulator that are meaningful for the prediction task, thus maximizing the simulator's influence on the predictions. 
Since the simulator can typically not capture all dynamics, we introduce an additional residuum $r$. 

We model these concepts by splitting the latent state $x$ (cf Eq.~\eqref{eq:dyn}) into $u \in \mathbb{R}^{d_u}$ and $v \in \mathbb{R}^{d_v}$, where $d_x = d_u+d_v$.
With $u$, we denote the latent states that can be reconstructed from the simulator and refer to them as being \textbf{observable via the simulator (OVS)}.
The non-OVS states $v$ on the other hand cannot be reconstructed from the simulator.
We propose to formulate a common dynamics model for simulator and data by extending system \eqref{eq:partially_obs} via
\begin{equation}\label{eq:partially_obs}
\begin{aligned}
u_{n+1}&=f_u(u_n) \\
v_{n+1}&=f_v(u_n, v_n) \\ 
y_n & = g(u_n)+r(u_n,v_n) \\
\hat{s}_n & = h(u_n),
\end{aligned}
\end{equation}
with $f_u:\mathbb{R}^{d_u} \to \mathbb{R}^{d_u}$ and $f_v:\mathbb{R}^{d_u} \times \mathbb{R}^{d_v} \to \mathbb{R}^{d_v}$.
Here, $f_u$ determines the propagation of the OVS latent states $u$, while $f_v$ determines the propagation of the non-OVS latent states $v$. 
The observation model $g$ maps the OVS states $u$ to the OVS part of the predictions, while $r$ maps the OVS states $u$ and non-OVS states $v$ to the non-OVS residuum. 
The simulator is reconstructed via the observation model $h$.  
Our goal is to learn $f_u,f_v,g,h$ and $r$ from measurement data $\hat y$ and simulator outputs $\hat{s}$.

The key insight is that we force $u$ to be OVS by addressing $f_u$ via a trainable observer that receives $\hat s$ as an input.
By design, we obtain simulator-informed latent states $u$ and corresponding outputs $g(u)$.
In general, we can represent every system via Eq. \eqref{eq:partially_obs}.
By setting $d_v=0$ and removing the non-OVS residuum, we obtain the fully OVS case.  
By setting $d_u=0$, our model is reduced to the pure learning-based case. 
We choose an additive structure for the observation model in order to obtain control over the corresponding non-OVS residuum $r$.
However, different structures are also possible. 

 

