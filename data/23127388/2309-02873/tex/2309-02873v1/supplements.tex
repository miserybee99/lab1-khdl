\documentclass{article}

% if you need to pass options to natbib, use, e.g.:
%\PassOptionsToPackage{round}{natbib}
% before loading neurips_2020
\PassOptionsToPackage{numbers}{natbib}

% ready for submission
% \usepackage{neurips_2020}

% to compile a preprint version, e.g., for submission to arXiv, add add the
% [preprint] option:
%     \usepackage[preprint]{neurips_2020}

% to compile a camera-ready version, add the [final] option, e.g.:
%     \usepackage[final]{neurips_2020}

% to avoid loading the natbib package, add option nonatbib:
\usepackage[preprint]{neurips_2023}
\usepackage{paralist}
\usepackage[utf8]{inputenc} % allow utf-8 input
\usepackage[T1]{fontenc}    % use 8-bit T1 fonts
%\usepackage{xr}
\usepackage{hyperref}       % hyperlinks
\usepackage{url}            % simple URL typesetting
\usepackage{booktabs}       % professional-quality tables
\usepackage{amsfonts}       % blackboard math symbols
\usepackage{nicefrac}       % compact symbols for 1/2, etc.
\usepackage{microtype}      % microtypography

\usepackage[english]{babel}

\usepackage{csquotes}
\usepackage{amsthm}
\usepackage{amsmath}
\usepackage{amssymb}
\newtheorem{theorem}{Theorem}
\newtheorem{proposition}{Proposition}
\newtheorem{lemma}{Lemma}
\usepackage{float}
\usepackage{placeins}
%\usepackage{floatx}
\usepackage{graphicx}
\usepackage{subcaption} % subfigure
\usepackage[dvipsnames]{xcolor}
\usepackage{tikz,tkz-euclide,pgfplots}
\pgfplotsset{compat=1.15}



%\externaldocument[sup-]{supplement} 

\setcounter{topnumber}{3}

% If you use BibTeX in apalike style, activate the following line:
%\bibliographystyle{apalike}
\bibliographystyle{apalike}

\renewcommand{\refname}{References}
\renewcommand{\bibsection}{\section*{\refname}}

% To have backward compatibility with the original draft
\newcommand*{\parencite}{\citep}


\newcommand{\comment}[1]{\textcolor{RubineRed}{#1}}
%\newcommand{\draft}[1]{\textcolor{Gray}{#1}}

% \renewcommand{\michael}[1]{}
% \renewcommand{\katharina}[1]{}

\newcommand{\ie}{i\/.\/e\/.,\/~}
\newcommand{\eg}{e\/.\/g\/.,\/~}
\newcommand{\cf}{cf\/.\/~}
\newcommand{\abvFig}{Fig\/.\/\,}
\newcommand{\abvThm}{Thm\/.\/\,}
\newcommand{\abvSec}{Sec\/.\/\,}
\newcommand{\abvDef}{Def\/.\/\,}
% \renewcommand{\friedrich}[1]{}
% \renewcommand{\sebastian}[1]{}
% \renewcommand{\draft}[1]{}
\newcommand\norm[1]{\left\lVert#1\right\rVert}
\title{Supplementary Material: Learning Hybrid Dynamics Models with Simulator-Informed Latent States}

% The \author macro works with any number of authors. There are two commands
% used to separate the names and addresses of multiple authors: \And and \AND.
%
% Using \And between authors leaves it to LaTeX to determine where to break the
% lines. Using \AND forces a line break at that point. So, if LaTeX puts 3 of 4
% authors names on the first line, and the last on the second line, try using
% \AND instead of \And before the third author name.

%\author{Anonymous Authors}

\begin{document}

\maketitle
 
 

\section{Mathematical background} \label{section:mathematics}
In this section, we will provide the necessary mathematical background and equations.
First, we will provide the background on KKL observers \citep{cdc-2019}.
Then, we will demonstrate in which cases the hybrid GRU could behave like an observer as described in the method and experimental section. 
We will also demonstrate, in which cases the properties are not fulfilled and the GRU is not able to act as an observer. 

\subsection{KKL observer} \label{section:background}
In this section, we add the necessary background on observer design.
In particular, we define backward distinguishability mathematically and list the full set of assumptions.
Analoguous to the background section, we consider a system with dynamics $f_u:\mathbb{R}^{d_u} \rightarrow \mathbb{R}^{d_u}$,
states $u$ and the observation model $h:\mathbb{R}^{d_u} \rightarrow \mathbb{R}^{d_s}$ with measurements $\hat s$.

\paragraph{Backward distinguishability: }
Consider a system with invertable dynamics $f_u$ and observation function $h$.  
Let $\mathcal{O}$ be an open bounded set containing $\mathcal{U}$.
Then, the system is denoted backward $\mathcal{O}$-distinguishable on $\mathcal{U}$ if
for any trajectories $u^a$ and $u^b$ with $(u^a_0, u^b_0) \in \mathcal{U}\times\mathcal{U}$ and $u^a_0 \neq u^b_0$,
there exists $T > 0$ such that $h(f_u^{-T}(u^a_0)) \neq h(f_u^{-T}(u^b_0))$ and
$(f_u^{-n}(u^b_0), f^{-n}(u^b_0)) \in \mathcal{O}\times\mathcal{O}$ for $n=0, \dots, T$.
A system is called backward distinguishable if there is such an $\mathcal O$.


\paragraph{Assumption 1: }
The dynamics function $f_u$ is invertible and $f_u^{-1}$ and $h$ are of class $C^1$ and globally Lipschitz.  
\paragraph{Assumption 2: }
The system with dynamics $f_u$ and observation function $h$ is backward-distinguishable.

Based on the assumptions the following theorem holds (cf. \citep{cdc-2019}).
\begin{theorem}[KKL-observer]\label{Thm1}
	Suppose assumptions 1 and 2 hold.
	Define $d_z = d_y(d_u + 1)$. 
	Let $c=\sup\{|(f^{-1})^{\prime}(u)| \mid u \in \mathcal{U}\}$ and $\mathcal{D}$ be the open disc in $\mathbb C$ of
	radius $\min\{1,1/c\}$. Then, there exists a set $S$ of zero measure in $\mathbb{C}^{d_z}$ such that for any
	diagonalizable matrix $D \in \mathbb R^{d_z \times d_z}$ with eigenvalues $(\lambda_1,\dots,\lambda_{d_z})$ in
	$\mathcal{D}^{d_z}\setminus S$ and any $F \in \mathbb{R}^{d_z \times d_s}$ such that $(D,F)$ is a controllable pair and 
	$\max(|\lambda_i|)<1,$
	there exists a continuous injective mapping $T:\mathbb R^{d_u} \rightarrow \mathbb{R}^{d_z}$ that satisfies the
	following equation on $\mathcal U$
	\begin{equation}
	\begin{aligned}
	T(f_u(u))=DT(u)+Fh(u),
	\end{aligned}
	\end{equation}
	and its continuous pseudo-inverse $T^{\star}:\mathbb C^{d_z} \rightarrow \mathbb{R}^{d_u}$ such that any trajectory of $z_{n+1}=Dz_n+F \hat s_n$ satisfies
	\begin{equation} \label{eq:observation}
	\lim_{n \rightarrow \infty} |u_n-T^{\star}(z_n)|=0.
	\end{equation}
	Thus, $T^{\star}(z)$ is an observer for $u$. 
\end{theorem}

 
\subsection{GRU architecture}
In this section, we will present the central results for GRUs that serve as a backbone and a baseline for our architecture. 
We will closely follow the notations and results in \citet{BONASSI2021105049}.
We will first recap the GRU dynamics and the warmup phase. 
The transition function $x_{n+1}=f(x_n, \tilde y_n, i_n)$ of a GRU is given as 
\begin{equation}\label{eq:GRUequation} 
\begin{aligned}
x_{n+1} &=z_n \circ x_n+(1-z_n) \circ \phi(\tilde W_r i_n+ W_r \tilde y_n+U_r h_n \circ x_n+b_r) \\
z_n &=\sigma(\tilde W_z i_n+W_z \tilde y_n+U_zx_n+b_z) \\
h_n& = \sigma(\tilde W_f i_n+W_f \tilde y_n+U_fx_n+b_f), 
\end{aligned}
\end{equation}
where $x_n \in \mathbb{R}^{d_x}$ is the state vector and $i_n \in \mathbb{R}^{d_s}$ is the control input.
In our case, the control input is given by the simulator $\hat s$. 
For the reconstructed observations $y \in \mathbb{R}^{d_y}$ it holds that 
\begin{equation}\label{eq:obs}
y_n=g(x_n)=U_ox_n
\end{equation}
with trainable matrix $U_0$. 
In the following, we specify the input $\tilde y_n \in \mathbb{R}^{d_y}$.  

\paragraph{GRU warmup phase}
A GRU ~\eqref{eq:GRUequation} is trained by feeding the measurements $\hat y_n \in \mathbb{R}^{d_y}$ as input on a small warmup phase of length $R$. 
After the warmup phase, the reconstructed observations are provided as inputs. 
This yields
\begin{equation}\label{eq:warmup}
\tilde y_n = 
\begin{cases}
\hat y_n & \textrm{ for } n \leq R, \\
y_n=U_o x_n & \textrm{ for } n > R. 
\end{cases}
\end{equation}


We will first demonstrate, under which conditions the system we consider could act as an observer.
Further, we will demonstrate conditions, under which it can not act as an observer. 
For both cases, we will derive some special cases. 
We will show, how a GRU could reproduce a partially OVS system and how it can reproduce a fully OVS system.
We will further demonstrate, how the GRU could ignore the simulator. 

In the following, we will focus on transitions after the warmup phase, thus $y_n = U_o x_n$.
Analogous to \citet{BONASSI2021105049} consider the following assumption.

\paragraph{Assumption 1: }
The initial state of the GRU network \eqref{eq:GRUequation} belongs to an arbitrary large but bounded set 
$\check{\mathcal X} \supseteq \mathcal X$, defined as 
\begin{equation}
\check{\mathcal X} = \{x \in \mathbb R^{d_n}: \Vert x \Vert_{\infty} \leq \check \lambda\},
\end{equation} 
with $\lambda \geq 1$. 

The central property of an observer is the independence of the initial conditions.
For the GRU this is the case, if the system forms a contraction.
Thus, we will show under which conditions the GRU is a contraction.  
The following results are based on \citet{BONASSI2021105049} adapted to our setting. 
\begin{lemma}[GRU properties]\label{lem:GRUproperties}
	Consider the GRU \eqref{eq:GRUequation}, two different initial values $x_a$ and $x_b$ after the warmup phase and identical control inputs $i$. 
	If 
	\begin{equation}
	\begin{aligned}
    \check \sigma_z & + (1-\check \sigma_z)\left(\Vert U_r \Vert_{\infty} (\frac{1}{4} \check \lambda \Vert U_f \Vert_{\infty}+\check \sigma_f)+\Vert W_r \Vert_{\infty}\Vert U_o \Vert_{\infty} 
    +\frac{1}{4} \check \lambda \Vert U_r \Vert_{\infty}\Vert W_f \Vert_{\infty}\Vert U_o \Vert_{\infty}\right)\\
    & +\frac{1}{4}(\check \lambda + \check \phi_r)(\Vert U_z \Vert_{\infty}+\Vert W_z \Vert_{\infty}\Vert U_o \Vert_{\infty})<1
  \end{aligned}
  \end{equation}
	with 
	\begin{equation}
	\begin{aligned}
	\check{\sigma}_z &=\sigma(\Vert W_z \ \check \lambda U_z \ b_z \Vert_{\infty}) \\
	\check{\sigma}_f & = \sigma(\Vert W_f \ \check \lambda U_f \ b_f \Vert_{\infty}) \\
	\check{\phi}_r &= \phi(\Vert W_r \ \check \lambda U_r \ b_r) \Vert_{\infty}),
	\end{aligned}
	\end{equation}
	then it holds that 
	\begin{equation}
	\Vert f(x_a, y_a, i)-f(x_b, y_b, i) \Vert \leq  C \Vert x_a-x_b \Vert,
	\end{equation}	
	where $C \in (0,1).$ 
\end{lemma} 
\begin{proof}
	Consider
	\begin{equation}
	\Delta x^+ = f(x_a, i, y_a)-f(x_b, i, y_b)
	\end{equation}
	and 
	\begin{equation}
	\Delta x = x_a-x_b.
	\end{equation}
	The proof of Theorem 2 in \citet{BONASSI2021105049} yields for the jth component of $\Delta x^+$
	\begin{equation}
	\begin{aligned}
	|\Delta x_j^+| & \leq \alpha_{\Delta_x} \Vert \Delta x \Vert_{\infty} + \alpha_{\Delta_u} \Vert U_o \Delta_x \Vert_{\infty} \\
	& \leq \alpha_{\Delta_x} \Vert \Delta x \Vert_{\infty} + \alpha_{\Delta_u}  \Vert U_o \Vert_{\infty} \Vert \Delta x \Vert_{\infty}
	\end{aligned}
	\end{equation}
	with 
	\begin{equation}
	\begin{aligned}
	   \alpha_{\Delta_x} & = z_{aj}+\frac{1}{4}(\check \lambda + \check \phi_r)\Vert U_z \Vert_{\infty}+(1-z_{aj})
	   \Vert U_r \Vert_{\infty}(\frac{1}{4} \check \lambda \Vert U_f \Vert_{\infty}+\check \sigma_f), \\
	   \alpha_{\Delta_u} & = \frac{1}{4}(\check \lambda + \check \phi_r) \Vert W_z \Vert_{\infty}+(1-z_{aj})(\Vert W_r \Vert_{\infty}+\frac{1}{4} \check \lambda \Vert U_r \Vert_{\infty}\Vert W_f \Vert_{\infty}).
	\end{aligned}
	\end{equation}
Here $z_{aj}$ repectively $z_{bj}$ denote the latent state corresponding to $x_{aj}$ and $x_{bj}$.
It holds that 
\begin{equation}
\begin{aligned}
&\alpha_{\Delta_x}+\Vert U_o \Vert \alpha_{\Delta_u} \\
= & z_{aj}\left(1-\Vert U_r \Vert_{\infty}(\frac{1}{4}\check \lambda \Vert U_f \Vert_{\infty}+\check \sigma_f)-\Vert U_o \Vert_{\infty}(\Vert W_r \Vert_{\infty}+\frac{1}{4} \check \lambda \Vert U_r \Vert_{\infty} \Vert W_f \Vert_{\infty}) \right) \\
& +\frac{1}{4}(\check \lambda + \check \phi_r)\Vert U_z \Vert_{\infty}+\Vert U_r \Vert_{\infty}(\frac{1}{4} \check \lambda \Vert U_f \Vert_{\infty}+\check \sigma_f) \\
&+\Vert U_o \Vert_{\infty} \frac{1}{4}(\check \lambda + \check \phi_r) \Vert W_z \Vert_{\infty} + \Vert U_o \Vert_{\infty} (\Vert W_r \Vert_{\infty}+\frac{1}{4} \check \lambda \Vert U_r \Vert_{\infty}\Vert W_f \Vert_{\infty}).
\end{aligned}
\end{equation}
\end{proof}

Since $z_{aj} \in [1-\check \sigma_z, \check \sigma_z]$ this yields
\begin{equation}
\begin{aligned}
&\alpha_{\Delta_x}+\Vert U_o \Vert \alpha_{\Delta_u} \\
\leq &\check \sigma_z+\Vert U_r \Vert_{\infty}(\frac{1}{4} \check \lambda \Vert U_f \Vert_{\infty}+\check \sigma_f)(1-\check \sigma_z) +\Vert U_o \Vert_{\infty}(\Vert W_r \Vert_{\infty}+\frac{1}{4} \check \lambda \Vert U_r \Vert_{\infty} \Vert W_f \Vert_{\infty})(1-\check \sigma_z) \\
&+\Vert U_o \Vert_{\infty} \frac{1}{4}(\check \lambda + \check \phi_r) \Vert W_z \Vert_{\infty} \\
& \leq \check \sigma_z+(1-\sigma_z)\left(\Vert U_r \Vert_{\infty}(\frac{1}{4} \check \lambda \Vert U_f \Vert_{\infty}+\check \sigma_f)+\Vert U_o \Vert_{\infty}(\Vert W_r \Vert_{\infty}+\frac{1}{4} \check \lambda \Vert U_r \Vert_{\infty} \Vert W_f \Vert_{\infty}) \right) \\
&+\frac{1}{4}(\check \lambda + \check \phi_r)(\Vert U_z \Vert_{\infty}+\Vert W_z \Vert_{\infty} \Vert U_o \Vert_{\infty}).
\end{aligned}
\end{equation}

\subsection{GRU as an observer}
Lemma \ref{lem:GRUproperties} demonstrates that the GRU is a contraction under some conditions.
From there, it is easy to see that the GRU forgets its initial values. 
Note that we consider the rollouts after the warmup phase, where the GRU receives output feedback. 
This is done since we are interested in the long-term behavior of the system. 
During the warmup phase, the GRU receives the measumements as an input.
In this case, the setting reduces to the setting presented in \citet{BONASSI2021105049}.
This leads to weaker requirements for the system to be a contraction. 

\begin{theorem}[Observer GRU] \label{observer_GRU}
Consider a GRU \eqref{eq:GRUequation} and assume that the requirements in Lemma \ref{lem:GRUproperties} hold. 
Then, the GRU forgets is initial conditions.
Further, consider two rollouts $x$ and $\tilde x$ with initial values $x_a$ and $x_b$. 
Then it holds that 
\begin{equation}
\Vert \tilde x_n -x_n \Vert \rightarrow 0, n \rightarrow \infty. 
\end{equation}
\end{theorem}
	
\begin{proof}
Holds due to the contraction property.  
\end{proof}	

Theorem \ref{observer_GRU} shows that under certain conditions, the GRU dynamics forgets its initial conditions.
Intuitively, if the dynamics fit the data, it will behave as an observer via the simulator.
However, the GRU is not restricted to this properties.  
Further, there is no guarantee that the dynamics and observations can indeed be represented with GRU dynamics fulfilling the required properties.
In contrast, the existence is guaranteed for the KKL observer under mild assumptions. 
By design, the KKL also forgets its initial conditions. 
This can be seen by considering the transformation $T$ with Lipschitz constant $L$. It holds that 
\begin{equation}
\Vert T(Dz_n+B\hat s_n)-T(D\tilde z_n+B\hat s_n) \Vert \leq L \Vert D^n (z_0-\tilde z_0) \Vert \leq L |\lambda|^n\Vert z_0-\tilde z_0\Vert,
\end{equation}
where $\lambda$ denotes the maximum eigenvalue of $D$.

In the following, we will derive specific GRU architectures for special cases.
Theorem \ref{observer_GRU} provides some requirements, under which the GRU reconstructs the full latent state via the simulator. 
However, this is usually not the case that we consider in practice.
As derived in the main paper, we typically consider a partially OVS system with states $u$ that can be reconstructed via the simulator and states $v$ that can not. 
Therefore, we will demonstrate conditions under which a GRU can represent such a partially OVS system.
Intuitively, the GRU matrices are chosen, such that one part of the states is only influenced by the simulator and itself. 
After that we will derive the criteria, under which the GRU could act as an observer for the first part of the states.  
 


\begin{lemma}[Partially OVS Representation] \label{lem:GRUSplit}
Consider the split of $x$ in $u$ and $v$ with $u \in \mathbb{R}^{d_u}, v \in \mathbb{R}^{d_x}$ and $d_x = d_u+d_v$. 
Further, consider GRU dynamics $f$ with matrices 
\begin{equation}
\begin{aligned}
W_r&=(0, W_r^v), \textrm{ with } W_r^v \in \mathbb{R}^{d_v \times d_y}, U_r = \begin{pmatrix}
U_r^u & 0 \\
U_r^{v,l} &  U_r^{v,r}
\end{pmatrix}, \\
W_z&=(0, W_z^v), \textrm{ with } W_z^v \in \mathbb{R}^{d_v \times d_y}, U_z= \begin{pmatrix}
U_z^u & 0 \\
U_z^{v,l} &  U_z^{v,r}
\end{pmatrix}, \\
W_f&=(0, W_f^v), \textrm{ with } W_f^v \in \mathbb{R}^{d_v \times d_y}, U_f= \begin{pmatrix}
U_f^u & 0 \\
U_f^{v,l} &  U_f^{v,r}.
\end{pmatrix}
\end{aligned}
\end{equation} 	
Then, the GRU dynamics $f$ with observation function $U_o$ can be split into $f_u$, $f_v$  
and observation functions $g$ and $r$, where 
\begin{equation}\label{eq:partially_obs}
\begin{aligned}
u_{n+1}&=f_u(u_n, i_n) \\
v_{n+1}&=f_v(u_n, v_n, \tilde y_n, i_n) \\ 
y_n & = g(u_n)+r(v_n). \\
\end{aligned}
\end{equation}
\end{lemma}
 \begin{proof}
 The matrices are constructed, such that $u$ is only influenced by $u$ and $i$ due to the choice of $W_r, W_z, W_u, U_r, U_z$ and $U_f$. The activation functions and Hadamard product are applied component-wise.  
 Note, that $f_u$ can be represented as GRU again and is thus a sub-GRU. 
 \end{proof}
\begin{lemma}[Partially OVS GRU]\label{lem:PartiallyOVSGRU}
	Consider the GRU \eqref{eq:GRUequation}, two rollouts $x$ and $\tilde x$ with initial values $x_a$ and $x_b$ and identical control inputs $i$. 
	If the partially OVS conditions from Lemma \ref{lem:GRUSplit} hold and further 
	\begin{equation}
	\Vert U_r^u \Vert_{\infty}\left(\frac{1}{4} \check \lambda \Vert U_f^u \Vert_{\infty} + \check \sigma_f\right)<1-\frac{1}{4} \frac{\check \lambda + \check \phi_r}{1-\check \sigma_z}
	\Vert U_z^u \Vert_{\infty}, \end{equation}
	with 
	\begin{equation}
	\begin{aligned}
	\check{\sigma}_z &=\sigma(\Vert \tilde W_z^u \ \check \lambda U_z^u \ b_z^u \Vert_{\infty}) \\
	\check{\sigma}_f & = \sigma(\Vert \tilde W_f^u \ \check \lambda U_f^u \ b_f^u \Vert_{\infty}) \\
	\check{\phi}_r &= \phi(\Vert \tilde W_r^u \ \check \lambda U_r^u \ b_r^u) \Vert_{\infty}),
	\end{aligned}
	\end{equation}
	where the superscript $u$ denotes the submatrices that affect $u$ 
	then it holds that 
	\begin{equation}
\Vert u_n-\tilde u_n \Vert \rightarrow 0
	\end{equation}	
\end{lemma} 
\begin{proof}
Since the matrices are constructed such that $u$ forms an independent system that is not influenced by $y$ and the other part of the latent states $v$, the proof of Theorem 2 in \citet{BONASSI2021105049} can be applied. In particular consider $\Delta u^+ = f_u(u_a,i_0)-f_u(u_b,i_0)$ and $\Delta u = u_a-u_b$. 
Then it holds that
\begin{equation}
\Vert \Delta u^+ \Vert_{\infty} \leq (1-\delta)\Vert \Delta u \Vert_{\infty},
\end{equation}
with $\delta \in (0,1)$. 
Thus, the system is a contraction and it holds that 
\begin{equation}
\Vert u_n-\tilde u_n \Vert \rightarrow 0.
\end{equation}
\end{proof}

The results show that the GRU architectur can represent a split in $u$ and $v$, where the transitions of $u$ form a sub-GRU.
Thus, under some requirements, the sub-GRU is a contraction for $u$ and thus, $u$ forgets its initial conditions. 
This means that if we can reproduce the original OVS system with this architecture, then the sub-GRU is able to learn a system, where $u$ can be reconstructed via the simulator. 
We expect that it in this cases it can also balance the small model mismatch, continuously correcting the latent states $u$ similar to the KKL-RNN.
However, as before, there is no guarantee that a required system exists for all data.
Further, the required properties are not enforced in the GRU architecture by design. 
Also, the desired split is not enforced in cotrast to our KKL-RNN architecture. 
We will show later that this could lead to a model that ignores the simulator. 
For completeness, we will first show how to obtain the fully OVS case. 
 
\paragraph{Fully OVS case: }
The fully OVS case is obtained by ignoring the data input, thus $W_r=0$, $W_z=0$, and $W_f=0$.
In this case, the system reduced to a system with control input $i$ and without output feedback. 
Thus, if the conditions from \citet{BONASSI2021105049} hold, the GRU is a contraction and can behave as an observer. 
Again, this requires that the original system can be represented with such a GRU architecture.
This includes that the original system is OVS. 

\paragraph{Ignoring the simulator: }
In any case, it is also possible to construct a dynamical system that is able to predict the data $y$ without considering the simulator inputs $\hat s$. 
Intuitively, this holds since the data are observable via the data itself.
Thus, latent dynamics can be constructed explaining the data solely via the data.
This can be easily seen by considering $\tilde W_r=0$, $\tilde W_z=0$, and $\tilde W_f=0$, thus ignoring the simulator input.
By ignoring the simulator it is visible easily that the acthitecture can not inform the latent states via the simulator.
However, this is also possible in other scenarios if the system violates the observer properties and allows errors to accumulate. 

\subsection{Summary and interpreting the results}
In the experiments, we have seen that the hybrid GRU acts similarly to our KKL-RNN in some cases.
Indeed, under certain conditions the GRU dynamics are a contraction and thus, the GRU forgets its initial condition.
The GRU architecture further allows to split the internal latent states into an OVS and non-OVS part, where the OVS part is solely addressed by the simulator and not by output feedback. 
Thus, if the dynamics additionally match the data, the GRU could act similar as an observer. 
In this cases, the setting allows to inform the OVS latent states via the simulator as before and we expect it to balance small modeling mismatchtes, thus correcting and stabilizing the predictions similar to our KKL-RNN. 
However, there is no formal proof that such a GRU exists for all system. 
Further, the necessary properties for a contraction are not fulfilled by design. 
We further showed that it is also possible to ignore the simulator inputs. 
In the experiments we observed, that the KKL-RNN was especially beneficial in case the simulator is only partially informative, e.g. in the partially OVS case. 
It can be interpreted that it is easier for the GRU to act as an observer if the simulator data are easy to process and no split in OVS and non-OVS has to be learned.
To interpret the results further, consider a fully OVS system. 
The hybrid GRU constantly receives the simulator input.
In contrast, the data are only provided during a warmup phase and are further noisy.
Thus, it might be easier to inform the latent states via the simulator than via the data.
The partially OVS system on the other hand requires learning the correct split, making the learning-task harder again.  
Additionally, if the system can not easily detect the states that are OVS via the simulator, this could also cause problems.
Later we will demonstrate experimentally that also GRUs specifically trained on the partially OVS systems are not able to learn the correct split. 
  

\newpage
\newpage
\section{Additional results and experiments} \label{section:add}
In this section, we will provide additional results.
We will first complete the results from the experiments section by adding plots, RMSE over time and runtimes. 
To further analyze the method, we will add an ablation study based on the partially OVS system presented in the problem formulation. 
Further, we will demonstrate how our method can be extended to invertible neural networks, allowing to directly address the dynamics $f_u$. 

\subsection{System v): Pure learning-based scneario}
We present an additional example for the pure learning-based scenario and refer to it as system (v).
We extend the experiment in \citep{ensinger2023combining} Exp. ii) and use identical data from the double-torsion pendulum \citep{lisowski} with a varied control input.
As in their setting, we learn a simulator substitute by training a GRU on the low-pass filtered and downsampled signal. 
We refer to their method as Filter. 
However in contrast to their setting, we inform the high-pass via the low-pass components. 
This is done by feeding them as an input to our KKL-RNN and the hybrid GRU.
The results demonstrate that we can further improve the concept in \citet{ensinger2023combining}.
While the GRU shows accumulating errors, Residual model and Filter both benefit from the stable long-term behavior of the low-pass component.
However, since low-pass and high-pass component are not linked for those models, the high-pass component still suffers from accumulating errors and deteriorated behavior, which is prevented by our KKL-RNN and the hybrid GRU.
This finding is more clearly visible in the rollout plots \ref{subfig:4_baseline} than in the RMSE \ref{t:lb} since it mainly refers to the high-frequency components. 
The signal however is mainly dominated by the low frequency behavior that Residual Model, Filter, hybrid GRU and hybrid KKL-RNN share.

\begin{table*}[h!]
	\centering 	
	\caption{Total RMSEs for systems iv)-v) (mean (std)) over 5 independent runs.}
	\label{t:lb}
	\begin{tabular}{rccccc}
		\noalign{\smallskip} \hline    \noalign{\smallskip}
		task & GRU & Residual Model & Filter &Hybrid GRU & Hybrid KKL-RNN (ours)  \\
		\hline
		v) & 1.18 (0.29) & 0.64 (0.07) & 0.33 (0.06) & \textbf{0.29} (0.07) & \textbf{0.29} (0.09)  \\
		\noalign{\smallskip} \hline \noalign{\smallskip}
	\end{tabular}\quad
\end{table*} 

\subsection{GRU ablation study}
As a recap, our proposed KKL-RNN architecture consists of
\begin{itemize}
	\item Partially observable system containing OVS and non-OVS latent states; 
	\item Specific loss function penalizing the non-OVS components;
	\item KKL-observer. 
\end{itemize}
The goal of this study is to analyze how different parts of the architecture affect the results.
In particular, we try to find out how the split in OVS and non-OVS affects the results and how the KKL observer affects the results in contrast to an architecture built from different GRUs.  
To this end, we consider three GRU architectures to further analyze the system based on the partially OVS system presented in Section 2
with $f_u:\mathbb{R}^{d_u} \times \mathbb{R}^{d_s} \rightarrow \mathbb{R}^{d_u}$ and $f_v:\mathbb{R}^{d_u} \times \mathbb{R}^{d_v} \rightarrow \mathbb{R}^{d_v}$.
Next, we consider single components of the partially OVS system or the whole system.
We model all components with GRUs. 
In the first scenario, we extend the hybrid GRU with an observation model $h$ that aims to reproduce the simulator. 
The transitions remain the same as in the hybrid GRU setting. 
With this experiment we analyze if it is sufficient to simultanously model the simulator. 
This forces the model to learn latent states that are able to reproduce the simulator. 
For the remaining architectures, we aim to reproduce the dynamics in the partially OVS systems.
To this end, we will train separate GRUs on $f_u$ and $f_v$. 
As before, $g$, $r$ and $h$ will be modeled with linear layers. 
We vary the input the GRUs receive. 
To make the architecture comparable to the hybrid KKL-RNN, we will apply the same losses to Scenario 2 and Scenario 3.
Providing the observations as an input to both GRUs in Scenario 2), we analyze whether it is sufficient to consider the proposed split and the corresponding loss function.
In the third scenario, we aim to reproduce our KKL-RNN architecture as closely as possible.
This is done by providing only the simulator as an input to the first model and only the data as an input to the second model. 
With this experiment, we study, whether the KKL observer could also be replaced with an additional GRU receiving identical input data. 
In summary, we consider the following architectures. 
\begin{itemize}
	\item Scenario 1: We extend the hybrid GRU with an additional observation function $h$ modeling the simulator. 
	\item Scenario 2: $f_u$ and $f_v$ are modeled with standard GRUs receiving output feedback. I.e., both receive the data and later output feedback as a control input. 
	\item Scenario 3: $f_u$ receives only the simulator as control input, $f_v$ receives the simulator and output feedback as control input. This mimics the setting of our KKL-RNN.  
\end{itemize}


\subsubsection{Results} 
The results are presented in Table \ref{t:ablation}. 
They indicate that Scenario 2 is not enough to model the whole system, probably because the data do not contain enough information to model also the simulator for all systems. 
This is for example demonstrated in Fig. \ref{subfig:S2mass}.
In contrast, this information is provided in Scenario 1 and 3 by providing the simulator as control input to both models. 
It can be seen that Scenario 1 has a similar accuracy as the hybrid GRU and suffers from similar problems. 
Especially, it also shows the deteriorated behavior in system i) (cf. Fig. \ref{subfig:S1mass}) and can not reproduce the oscillations for system ii) 
(cf. Fig \ref{subfig:S1DT}) and iv) (cf. Fig. \ref{subfig:S1VDP}). 
In contrast to Scenario 1, Scenario 3 is informed about the split in OVS and non-OVS part in its architecture similar to the KKL-RNN.
However, still Scenario 3 has similar problems with learning the correct oscillations. 
Further, it is also visible that it does not learn the correct split in OVS and non-OVS (cf. Fig. \ref{subfig:S3mass}, Fig. \ref{subfig:S2DT} and Fig. \ref{subfig:VDP}). 
The results indicate that the KKL-observer is a central component of our method and the split into the partially OVS system not enough.  
At the same time, the results show that the highest accuracy is in general achieved with Scenario 3, indicating that the proposed architecture is already useful in itself. 
This includes the split in OVS and non-OVS, the loss and an observer-inspired setup.  
The proposed architecture could thus also be applied in combination with standard recurrent networks. 

\begin{table*}[h!]
	\centering 
	\caption{Results for Scenario 1)-3) on systems i)-v) (mean (std)) over 5 indep. runs.}
	\label{t:ablation}
	\begin{tabular}{rccc}
		\noalign{\smallskip} \hline \noalign{\smallskip}
		System & Scenario 1 & Scenario 2  & Scenario 3   \\
		\hline
		i) &  \textbf{0.25} (0.12) & 0.97 (0.14) & \textbf{0.25} (0.07) \\
		ii)& 0.54 (0.29)  & 1.28 (0.03) & \textbf{0.41} (0.02) \\
		iii) & \textbf{0.24} (0.02) & 0.47 (0.12) & 0.27 (0.01) \\
		iv) & 0.1 (0.08) & 0.63 (0.43) & \textbf{0.06} (0.04)  \\
		v) & \textbf{0.28} (0.09) & 1.35 (0.24) & \textbf{0.28} (0.08) \\
		\noalign{\smallskip} \hline \noalign{\smallskip}
	\end{tabular}\quad
\end{table*}
\begin{figure*}[ht!]
	\begin{subfigure}{0.49\textwidth}
		\centering
		\includegraphics[width= \textwidth]{Plots/Scenario3_mass.pdf}
		\caption{Scenario 1 on system i)} \label{subfig:S1mass}
	\end{subfigure}	
	\begin{subfigure}{0.49\textwidth}
		\centering
		\includegraphics[width= \textwidth]{Plots/Scenario2_mass.pdf}
		\caption{Scenario 2 on system i)} \label{subfig:S2mass}
	\end{subfigure}	
	\begin{subfigure}{0.49\textwidth}
		\centering
		\includegraphics[width= \textwidth]{Plots/Scenario1_mass.pdf}
		\caption{Scenario 3 on system iii)} \label{subfig:S3mass}
	\end{subfigure}	
	\caption{Plots for system i). Shown are the results for Scenario 1 in Fig. \ref{subfig:S1mass}, Scenario 2 in Fig. \ref{subfig:S2mass} and Scenario 3 in Fig. \ref{subfig:S3mass}. All models have problems to reproduce the oscillations and Scenario 1 even shows upswinging oscillations. Further, the models are not able to learn the correct split in OVS and non-OVS. For Scenario 3 it is clearly visible that the residuum takes over most of the prediction task.}
	\label{fig:mass}
\end{figure*}

\begin{figure*}[ht!]
	\begin{subfigure}{0.49\textwidth}
		\centering
		\includegraphics[width= \textwidth]{Plots/Scenario_1_double_torsion.pdf}
		\caption{Scenario 1 on system ii)} \label{subfig:S1DT}
	\end{subfigure}	
	\begin{subfigure}{0.49\textwidth}
		\centering
		\includegraphics[width= \textwidth]{Plots/Scenario_3_double_torsion.pdf}
		\caption{Scenario 3 on system ii)} \label{subfig:S2DT}
	\end{subfigure}	
	\caption{Plots for system ii). Shown are the results for Scenario 1 in Fig. \ref{subfig:S1DT} and Scenario 3 in Fig. \ref{subfig:S2DT}.
		Both methods are not able to reproduce the oscillations correctly and Scenario 3 learns a wrong split in OVS and non-OVS.}
	\label{fig:mass}
\end{figure*}

\begin{figure*}[ht!]
	\begin{subfigure}{0.49\textwidth}
		\centering
		\includegraphics[width= \textwidth]{Plots/Scenario_1_VDP.pdf}
		\caption{Scenario 1 on system iv)} \label{subfig:S1VDP}
	\end{subfigure}	
	\begin{subfigure}{0.49\textwidth}
		\centering
		\includegraphics[width= \textwidth]{Plots/Scenario_2_VDP.pdf}
		\caption{Scenario 2 on system iv)} \label{subfig:S2VDP}
	\end{subfigure}	
	\begin{subfigure}{0.49\textwidth}
		\centering
		\includegraphics[width= \textwidth]{Plots/Scenario_3_VDP.pdf}
		\caption{Scenario 3 on system iv)} \label{subfig:VDP}
	\end{subfigure}	
	\caption{Plots for system i). Shown are the results for Scenario 1) in Fig. \ref{subfig:S1VDP}, Scenario 2) in Fig. \ref{subfig:S2VDP} and Scenario 3) in Fig. \ref{subfig:VDP}. Again, the models learn a wrong split and are not able to reproduce the oscillations correctly.}
	\label{fig:mass}
\end{figure*}



\subsection{Invertible neural network} 
As described in Section 4, we provide the option to train with invertible neural networks in case direct access to the dynamics $f_u$ is required, e.g. to analyze for fixed points, include symmetries etc.
In detail, we learn $T_{\theta}^{\star}$ as invertible network allowing to access $T_{\theta}$ by inversion. 
An invertible network is obtained by stacking coupling layers \citep{DBLP:conf/iclr/DinhSB17}. 
For $f_u$ it holds that 
\begin{equation}
f_u=T^{\star}_{\theta}(D_{\theta}T_{\theta}u_n+F\hat{s}_n). 
\end{equation}
This allows to obtain $u_{n+1}$ directly via $f_u(u_n)$. 
Later, in Sec. \ref{sec:KKL}, we provide details on the architecture. 
We train the architecture on system i) for 5 independent random seeds.
This provides similar results as the default scenario containing a standard MLP. 
Fig. \ref{subfig:mass} shows a plot of the rollouts demonstrating that also the qualitative results are similar to the default setting.
In particular, the split in OVS and non-OVS is accurate and the simulator can be reconstructed. 


\subsection{Additional plots} 
We add missing plots for all experiments in the experimental section. 
In Fig. \ref{fig:hybrid}, we report the accumulated RMSEs over time for systems i-iii).
In Fig. \ref{fig:baselines}, we report the accumulated RMSEs over time for systems iv) and v). 
We provide plots for the baselines on systems i) and iii) in Fig. \ref{fig:hybrid_KKL}, indicating that they are not able to reproduce the dynamics well.
Fig. \ref{subfig:3_KKLRNN} shows the results of our hybrid KKL-RNN for system ii). 
It indicates that it reproduces the dynamics well even if it does not perfectly reproduce the decaying behavior of the oscillations.
The split in OVS and non-OVS component however is reproduced perfectly.
Further, we provide missing plots for systems iv) and v) representing the pure learning-based scenario. 
In Fig. \ref{fig:lb} we demonstrate that the KKL-RNN is able to reproduce the dynamics of both systems and learns the correct split in OVS and non-OVS.
Fig. \ref{subfig:5_KKLRMM} further shows that the model can jointly learn the sine oscillations and VDP oscillator that is reconstructed from the sine oscillations. 
Fig. \ref{fig:lb_baselines} shows that the baselines struggle on these tasks. 
    

\begin{figure*}[h!]
	\begin{subfigure}{0.49\textwidth}
		\centering
		\includegraphics[width= \textwidth]{Plots/rollout_plot_mass.pdf}
		\caption{System i), RMSE over time} \label{subfig:RMSE_mass}
	\end{subfigure}	
	\begin{subfigure}{0.49\textwidth}
		\centering
		\includegraphics[width= \textwidth]{Plots/rollout_plot_double_torsion_distorted.pdf}
		\caption{System ii), RMSE over time} \label{subfig:RMSE_double_torsion}
	\end{subfigure}	
	\begin{subfigure}{0.49\textwidth}
	\centering
	\includegraphics[width= \textwidth]{Plots/rollout_plot_friction.pdf}
	\caption{System iii), RMSE over time} \label{subfig:RMSE_friction}
\end{subfigure}	
	\caption{Accumulated RMSEs over time for systems i)-iii). Fig. \ref{subfig:RMSE_mass} shows the results for system i), 
		Fig. \ref{subfig:RMSE_double_torsion} shows the results for system ii) and Fig. \ref{subfig:RMSE_friction} shows the results for system iii).}
	\label{fig:hybrid}
\end{figure*}

\begin{figure*}[h!]
	\begin{subfigure}{0.49\textwidth}
		\centering
		\includegraphics[width= \textwidth]{Plots/rollout_plot_VDP.pdf}
		\caption{System iv), RMSE over time} \label{subfig:RMSE_VDP}
	\end{subfigure}	
	\begin{subfigure}{0.49\textwidth}
		\centering
		\includegraphics[width= \textwidth]{Plots/rollout_train_sim.pdf}
		\caption{System v), RMSE over time} \label{subfig:RMSE_sim}
	\end{subfigure}	
	\caption{Accumulated RMSEs over time for systems iv) and v). Fig. \ref{subfig:RMSE_VDP} shows the results for system iv) 
  and Fig. \ref{subfig:RMSE_double_torsion} shows the results for system ii), \ref{subfig:RMSE_sim} shows the results for system v).}
	\label{fig:baselines}
\end{figure*}
%
\begin{figure*}[h!]
	\begin{subfigure}{0.49\textwidth}
		\centering
		\includegraphics[width= \textwidth]{Plots/exp1_baselines.pdf}
		\caption{System i), baselines} \label{subfig:1_baselines}
	\end{subfigure}	
	\begin{subfigure}{0.49\textwidth}
		\centering
		\includegraphics[width= \textwidth]{Plots/exp4_baselines.pdf}
		\caption{System iv), baselines} \label{subfig:4_baselines}
	\end{subfigure}	
	\caption{Baselines for system i) in Fig. \ref{subfig:1_baselines} and system iii) in Fig. \ref{subfig:4_baselines}. The baselines do not provide a good representation of the rollouts in both cases. Especially, they show deteriorated long-term behavior.}
	\label{fig:hybrid_KKL}
\end{figure*}

\begin{figure*}
	\begin{subfigure}{0.49\textwidth}
	\centering
	\includegraphics[width= \textwidth]{Plots/exp3_KKL_RNN.pdf}
	\caption{System i), baselines} \label{subfig:3_KKLRNN}
\end{subfigure}	
\begin{subfigure}{0.49\textwidth}
	\centering
	\includegraphics[width= \textwidth]{Plots/exp_1_hybrid_KKL_invertible.pdf}
	\caption{System i) with invertible neural network.} \label{subfig:mass}
\end{subfigure}
	\caption{Hybrid KKL-RNN for system 3) in Fig. \ref{subfig:3_KKLRNN}. The model learns the correct split in OVS and non-OVS. The oscillations are not perfectly reproduced but still a good representation. Fig. \ref{subfig:4_baselines} depicts system i) with the hybrid KKL-RNN and invertible transformation.}
\label{fig:hybrid_KKL_new}
\end{figure*}

\begin{figure*}[h!]
		\begin{subfigure}{0.49\textwidth}
			\centering
			\includegraphics[width= \textwidth]{Plots/exp6_KKLRNN.pdf}
			\caption{System iv), KKL-RNN} \label{subfig:6_baselines}
		\end{subfigure}	
		\begin{subfigure}{0.49\textwidth}
		\centering
		\includegraphics[width= \textwidth]{Plots/exp5_KKLRNN.pdf}
		\caption{System v), KKL-RNN} \label{subfig:5_KKLRMM}
	\end{subfigure}	
		\caption{KKL-RNN for system iv) in Fig. \ref{subfig:6_baselines} and system v) in Fig. \ref{subfig:5_KKLRMM}.
		In both cases, the KKL-RNN learns the correct split in OVS and non-OVS and reproduces the predictions accurately.}
		\label{fig:lb}	
\end{figure*}
%%
\begin{figure*}[h!]
	\begin{subfigure}{0.49\textwidth}
		\centering
		\includegraphics[width= \textwidth]{Plots/exp6_baselines.pdf}
		\caption{System iv), baselines} \label{subfig:6_baseline}
	\end{subfigure}	
	\begin{subfigure}{0.49\textwidth}
		\centering
		\includegraphics[width= \textwidth]{Plots/exp5_baselines.pdf}
		\caption{System v), baselines} \label{subfig:4_baseline}
	\end{subfigure}	
	\caption{Baselines for system iv) in Fig. \ref{subfig:6_baseline} and system v) in Fig. \ref{subfig:4_baseline}.
	The GRU suffers from the typical drift in system iv), while the residual model doesn't learn the oscillations correctly.
    For system v), the baselines show deteriorated behavior in the high-pass components.}
	\label{fig:lb_baselines}
\end{figure*}


\subsection{Runtimes}
All experiments are conducted on CPUs of an internal cluster. 
The runtimes for experiments i)-iii) are provided in Table \ref{t:hybrid}, the runtimes for experiments iv)-v) are provided in Table \ref{t:lb}.
\begin{table*}[h!]
	\centering 
	\caption{Mean of total runtimes in seconds for systems i)-iii) over 5 indep. runs.}
	\label{t:hybrid}
	\begin{tabular}{rccccc}
		\noalign{\smallskip} \hline \noalign{\smallskip}
		task & GRU & Residual Model  & Hybrid GRU & Filter & KKL-RNN (ours)  \\
		\hline
		i) & 283 & 276 & 264 & 262 & 535 \\
		ii)& 119 & 114 & 31  &  92 & 65  \\
		iii) & 3865   & 3967 & 3104 & 5310 & 8449  \\
		\noalign{\smallskip} \hline \noalign{\smallskip}
	\end{tabular}\quad
\end{table*}

\begin{table*}[ht]
	\centering 	
	\caption{Mean of total runtimes in seconds for systems iv)-v) over 5 indep. runs.}
	\label{t:lb}
	\begin{tabular}{rccccc}
		\noalign{\smallskip} \hline \noalign{\smallskip}
		task & GRU & Residual Model & Filter &Obs-GRU (ours) &KKL-RNN (ours)  \\
		\hline
		iv)  & 763  & 514 &- (-)&  556& 1078  \\
		v) & 1423  & 361  & 1436 &  118 & 207   \\
		\noalign{\smallskip} \hline \noalign{\smallskip}
	\end{tabular}\quad
\end{table*} 
\newpage




\FloatBarrier
\newpage
\section{Experimental details} \label{section:hyperparameters}
In this section, we will specify the configurations for each experiment.
This includes detailed descrpitions of the models as well as hyperparameters and training details. 

\subsection{Models} \label{section:mathematics}
In this section, we will add the necessary background on the models and data that we consider. 

\paragraph{Damped system: }
The data $x(t)$ are generated by adding a sine wave and a second sine oscillations that is exponentially damped over time. 
This yields the system
\begin{equation}
x(t)= \sin(0.1 t)+2\exp(-0.01 t)\sin(t).
\end{equation}
For the simulator, a simple sine wave with slight mismatch in the amplitude is chosen.
This yields
\begin{equation}
s(t)=0.7 \sin(0.1 t).
\end{equation}
The signals are obtained, by evaluating at $t=0, \dots 400$ with a discretization $\Delta t = 0.1$. Further the observation data are corrupted with white noise with variance $0.1$.  
\paragraph{Double torsion pendulum: }
For system ii), we consider the data from \citet{lisowski}.
We artificially add a transient component again $\Delta x$ via
\begin{equation}
\Delta x(t)=\exp(-0.5 t)\sin(50 t),
\end{equation}
evaluated at the interval $t = 0,1,2, \dots$. 
\paragraph{Van der Pol oscillator: }
For experiment iv), we consider a Van-der-Pol oscillator with additional control input. 
The differential equation for the Van-der-Pol oscillator with external force is given as
\begin{equation}
\begin{pmatrix}
\dot{x}(t) \\
\dot{y}(t) \\
\dot{u}(t) \\
\dot{v}(t)
\end{pmatrix}
=
\begin{pmatrix}
y \\
-x+a(1-x^2)y+bu \\
v \\
-\omega^2 u
\end{pmatrix},
\end{equation}
where $a,b$ and $\omega$ are system parameters. 
We choose $a=5, b=80, \omega=7.0$ and initial conditions 
\begin{equation}
\begin{pmatrix}
\dot{x}_0 \\
\dot{y}_0 \\
\dot{u}_0 \\
\dot{v}_0
\end{pmatrix}
=
\begin{pmatrix}
-2 \\
1 \\
0.31 \\
1
\end{pmatrix}.
\end{equation}
The system is evaluated at $t=5,\dots,100$ with step size $\Delta t=0.1$. 



\subsection{Architecture details}\label{sec:arch}
Here, we will add details on the chosen architectures. 
This includes details on the networks and training details such as hyperparameter choice. 
\subsubsection{KKL model}\label{sec:KKL}
We will first describe the details of the KKL architecture.
This includes the options for the nonlinear transformation, the controllable pair and modeling variants for the non-OVS residuum. 
\paragraph{Nonlinear transformation: }
As proposed in \citet{9683277}, we consider an MLP with three layers and ReLU activation functions in between for the default scenario.
The number of neurons for each experiment will be documented in Sec \ref{sec:hyperparameters}. 
In the default scenario, we jointly propagate $z$ and $u$ through time and do not explicitly model $f_u$ but directly the observer. 
Due to the observer property, $u_{n+1}$ converges to $f_u(u_n)$. 
One advantage of this choice is that the whole rollout $z_0,\dots,z_N$ can be precomputed and the transformation can be applied in a batched manner making it efficient. 

However, sometimes direct access to $f_u$ might be required, e.g. in order to analyze fixed points or include symmetries. 
In this case, we provide the option to train $T^{\star}_{\theta}$ with an invertible neural network. 
The network is then built by stacking affine coupling layers \citep{DBLP:conf/iclr/DinhSB17}.
In particular, we stack 2 coupling layers, where the inner neural networks are modeled via 2 linear layers with ReLU activation functions. 
In contrast to the default setting, the transformation $T_{\theta}$ and the inverse $T_{\theta}^{\star}$ have to be computed in each time step, making it computationally more expensive. 
Furthermore, the invertible NN approach is less flexible since the number of neurons per layer has to match the latent dimensionality.  
Thus, if no direct access to the dynamis $f_u$ is required, the default setting is favorable. 
\paragraph{Controllable pair: }
The matrix $D_{\theta}$ is chosen as trainable diagonal matrix with sigmoid activation functions on the entries.
This ensures that $\lambda \in (0,1)$ for the eigenvalues $\lambda$.
However, of course this restricts the eigenvalues to positive ones.
Still, it worked well for our experiments.
Other activation functions such as tanh etc. can be easily incorporated.
It would also be possible to train the matrix freely as proposed in \citet{9683277}.
However, the pair $(D,F)$ would not meet the required properties in Thm \ref{Thm1} by design anymore.  
\subsubsection{Remaining components}
Here, we will describe the setting for the remaining components, in particular the non-OVS residuum.
\paragraph{Non-OVS residuum: } 
As stated in the method section, we model the non-OVS part via
\begin{equation}
v_{n+1}=f_{\theta}^v(u_n,v_n,y_n),
\end{equation}
where $f_{\theta}^v$ is modeled as GRU architecture. 
For efficiency reasons, we omit to provide $u_n$ as an input. 
Note that this is possible without loss of generality. 
However, the implementation of the method additionally contains the option to provide the simulator as a control input to $f_{\theta}^v$. 
\paragraph{Exponentially damped non-OVS residuum: }
In order to obtain a non-OVS component that vanishes over time, we propose to exponentially damp the observations. 
This is achieved by constructing a non-linear observation model $g$.
The latent states evolve according to the GRU transitions in Eq.~\eqref{eq:GRUequation}.
However, the linear observation function $g(x)=U_o x$ (cf. Eq.~\eqref{eq:obs}) is replaced by a non-linear observation model.
In particular, we consider 
\begin{equation}
g_{\textrm{damp}}(x)=a \exp(-\textrm{softplus}(b)t_n)\tanh(U_0 x),
\end{equation}
with trainable parameters $a$, $b$, a trainable linear model $U_0$ and the time interval $t=t_0, \dots, t_n$. 


\subsection{GRU architecture}
We consider GRUs for our non-OVS residuum as well as for the baselines. 
We consider GRU dynamics $f$ as presented in Eq. \eqref{eq:GRUequation} with output feedback $y$ and control input $i$. 
The hidden GRU states are mapped to the observations via the observation model $g$ (cf. Eq. \eqref{eq:obs}).


\paragraph{Warmup: }
The hidden states of recurrent architectures such as GRUs are usually obtained with a short warmup phase of length $Rn$, where the architecture receives the observations as an input as described in Eq. \eqref{eq:warmup}.
The outputs are initialized with the observations on that horizon and therefore do not contribute to training. 
For the KKL observer, such a warmup phase is not required since it constantly receives the simulator outputs as an input.
However, the non-OVS residuum is initialized with standard GRU initialization via
\begin{equation}
y^v_{0:R}=\hat{y}_{0:R}-\left(g_{\theta}(u_n)\right)_{0:R}.
\end{equation}
This yields automatically that
\begin{equation}
y_{0:R}=\hat{y}_{0:R}. 
\end{equation}
Due to this architecture, the KKL observer contributes to training during the warmup phase but only via $s$, while the GRU does not contribute. 
However, this is balanzed by taking the MSE in the loss functions, which counterbalances different lengths of training trajectories. 
For the ablation study, all GRUs are initialized with exact data and start contributing to training after the warmup phase.  


\paragraph{Hybrid GRU: }
For the hybrid GRU, we provide the simulator $s$ as control input $i$ to the GRU dynamics \eqref{eq:GRUequation}.

\paragraph{Residual model: }
For the residual model, we consider the sum of GRU predictions $r$ and simulations $s$.
In particular, we consider the transitions 
\begin{equation}
\begin{aligned}
x_{n+1}&=f(x_n,\tilde r_n) \\
\end{aligned}
\end{equation}
with corresponding observation model 
\begin{equation}
\begin{aligned}
r_n &=U_ox_n \\
y_n &=r_n+s_n.
\end{aligned}
\end{equation}
The warmup is obtained by feeding the residuum of data for $R$ steps.
This yields
\begin{equation}\label{eq:warmup}
\begin{aligned}
\tilde r_n = 
\begin{cases}
\hat y_n-s_n & \textrm{ for } n \leq R, \\
r_n=U_o x_n & \textrm{ for } n > R. 
\end{cases}
\end{aligned}
\end{equation}
The hyperparameters are optimized by minimizing $\Vert \hat y_n-y_n \Vert$. 




\subsubsection{Filter design}\label{filter}
For the hybrid experiments, we reimplement a method similar to the hybrid model proposed in \citet{ensinger2023combining}.
It is implemented as follows:
\begin{itemize}
	\item Based on the properties of the simulator, the cutoff frequency is obtained, such that $L(s) \approx L(y)$.  
	\item The simulator signal is low-pass filtered with low-pass filter $L$, which yields $\tilde s = L(s)$.
	 This signal is stored.  
	\item During training and predictions the computation $\tilde s+H(y)$ is performed.
\end{itemize}
In contrast to their implementation, we apply forward-backward filtering provided by torch.
Further, the initialization slightly changes, since we filter the signals separately and add them.
Further, we precompute the low-pass filtered simulator signal and store it.  
The filter parameters are obtained with the help of \texttt{scipy.signal.iirfilter}. 
For the experiments here, we consider butterworth filters of order one. 
Thus, for a lowpass filter with cutoff frequency $\omega$, we call \texttt{iirfilter(N=1,Wn=$\omega$,rp=None,rs=None,btype="lowpass",analog=False,\\
ftype="butter",output="ba",fs=10)}.
Appropriate cutoff frequencies are obtained by analyzing the systems.
For training and predictions, we use the filters provided by \texttt{torchaudio}.
We use their implementation of forward-backward filtering \texttt{torchaudio.functional.filtfilt}, which first filters a signal and then filters it backwards.
 
 
\subsection{Hyperparameters}\label{sec:hyperparameters}
Here, we report the number of training steps, learning rates, latent dimensions, length of the warmup phase and additional parameters as cutoff frequency of the filters, damping factor etc. 
All models are trained with Adam optimizer. 
For all experiments, we train on subtrajectories of the full trajectory and consider batch sizes of length 50. 
To make a fair comparison, we choose an identical total amount of latent states for each experiment. 
\paragraph{i) Damped system: }
All models are trained with the learning rate $10^{-3}$. 
All models are trained for 300 steps. 
All models are trained on subtrajectories of length 100, the GRUs model receive 50 steps as a warmup phase. 
For our hybrid KKL-RNN, we consider a 32-dimensional latent space and an MLP with 100 neurons. 
The non-OVS residuum has a 32-dimensional latent space.
For the other models, we consider a GRU with 64 hidden states.
For the Filter, we consider the cutoff frequency 0.05.
%TO ADD: INITIAL DAMPING
\paragraph{ii): Double-torsion pendulum: }
All models are trained with the learning rate $10^{-3}$. 
The GRU is trained with an additional scheduler that multiplies the learning rate with 0.05 after 800 steps.  
All models are trained for 250 steps, the GRU is trained for 1000 steps. 
All models are trained on subtrajectories of length 50, the GRUs receive 20 steps for warmup. 
For our hybrid KKL-RNN, we consider a 64-dimensional latent space and an MLP with 100 neurons. 
The non-OVS residuum is regularized with weight 0.5. 
The non-OVS residuum has a 32-dimensional space. 
All other GRUs have 96-dimensional latent states. 
For the Filter, we consider the cutoff frequency 0.1.

\paragraph{iii) Drill-string system: }
All models are trained with the learning rate $10^{-3}$. 
All models are trained for 250 steps, the GRU is trained for 300 steps. 
All models are trained on subtrajectories of length 500, the GRUs receive 50 steps for the warmup phase.
For our hybrid KKL-RNN, we consider a 64-dimensional latent space and an MLP with 100 neurons. 
The non-OVS residuum is regularized with weight 0.5. 
The non-OVS residuum has a 32-dimensional space. 
For all other models, we consider GRUs with 96-dimensional hidden states. 
For the Filter, we consider the cutoff frequency 0.05.

\paragraph{iv) Van-der-Pol oscillator: } 
All models are trained with the learning rate $10^{-3}$.  
All models are trained for 500 steps, the GRU is trained for 800 steps. 
All models are trained on subtrajectories of length 200, the GRUs receive 20 steps for the warmup phase.
For our KKL-RNN, we consider a 32-dimensional latent space and an MLP with 100 neurons. 
The non-OVS part is regularized with weight 0.5. 
The non-observable GRU residuum has a 32-dimensional space. 
For all other models, we consider GRUs with 64-dimensional hidden states. 

\paragraph{v) Double torsion system: }
For the Filter baseline, we report the results provided in \citep{ensinger2023combining}.
For system v), we first pretrain the simulate substitute. 
To this, end the training signal is downsampled with a rate of 2 and low-pass filtered. 
On this signal, a GRU with 32 hidden dimensions is trained for 1000 training steps with a learning rate of $10^{-3}$. 
The trajectory is split into subtrajectories of length 125, 50 steps are used for warmup.
After training, the signal is upsampled with a rate of 2 again via the torch method \texttt{torch.nn.Upsample(scale\_factor = 2, mode="linear", align\_corners=False)}. 
The pretrained signal is then used as an input for the Residual model, the KKL-RNN and the Obs-GRU, while the GRU is trained in the standard setting on the whole input. 
All models are trained with a learning rate of $10^{-3}$.
KKL-RNN and Obs-GRU are trained on 150 steps, the Residual model is trained on 500 steps, the GRU is trained on 1151 steps. 
For all models, we consider subtrajectories of length 100 and for all GRUs, we consider 50 steps for the warmup phase. 
For our KKL-RNN, we consider a 32-dimensional latent space and an MLP with 100 neurons. 
The non-observable GRU residuum contains 32 hidden states. 
The residual model consists of a GRU with 64 hidden states. 
The Obs-GRU consists of a GRU with 64 latent states, the standard GRU has 96 latent states. 







%\newpage
\bibliography{bib}

%%%%%%%%%%%%%%%%%%%%%%%%%%%%%%%%%%%%%%%%%%%%%%%%%%%%%%%%%%%%
 
% Supplementary material has been moved to supplements, because website says so

\end{document}
