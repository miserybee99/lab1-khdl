   %File: formatting-instructions-latex-2024.tex
%release 2024.0
\documentclass[letterpaper]{article} % DO NOT CHANGE THIS
\usepackage{aaai24}  % DO NOT CHANGE THIS
\usepackage{times}  % DO NOT CHANGE THIS
\usepackage{helvet}  % DO NOT CHANGE THIS
\usepackage{courier}  % DO NOT CHANGE THIS
\usepackage[hyphens]{url}  % DO NOT CHANGE THIS
\usepackage{graphicx} % DO NOT CHANGE THIS
\usepackage{subcaption}

\usepackage{paralist}
\usepackage[utf8]{inputenc} % allow utf-8 input
\usepackage[T1]{fontenc}    % use 8-bit T1 fonts
%\usepackage{xr}
\usepackage{url}            % simple URL typesetting
\usepackage{booktabs}       % professional-quality tables
\usepackage{amsfonts}       % blackboard math symbols
\usepackage{nicefrac}       % compact symbols for 1/2, etc.
\usepackage{microtype}      % microtypography
\usepackage{paralist} 
\usepackage[english]{babel}

\usepackage{csquotes}
%\usepackage{float} 

\usepackage{amsmath}
\usepackage{amssymb}
%\usepackage{float}
\usepackage{placeins}
%\usepackage{floatx}
\usepackage{subcaption} % subfigure
\usepackage{mwe}
\usepackage{caption}
\usepackage{subcaption}
\usepackage[dvipsnames]{xcolor}
\usepackage{tikz,tkz-euclide,pgfplots}
\pgfplotsset{compat=1.15}
\newtheorem{theorem}{Theorem}

\newcommand{\ie}{i\/.\/e\/.,\/~}
\newcommand{\eg}{e\/.\/g\/.,\/~}
\newcommand{\cf}{cf\/.\/~}
\newcommand{\abvFig}{Fig\/.\/\,}
\newcommand{\abvThm}{Thm\/.\/\,}
\newcommand{\abvSec}{Sec\/.\/\,}
\newcommand{\abvDef}{Def\/.\/\,}
\newcommand*{\parencite}{\citep}
\urlstyle{rm} % DO NOT CHANGE THIS
\def\UrlFont{\rm}  % DO NOT CHANGE THIS
\usepackage{natbib}  % DO NOT CHANGE THIS AND DO NOT ADD ANY OPTIONS TO IT
\usepackage{caption} % DO NOT CHANGE THIS AND DO NOT ADD ANY OPTIONS TO IT
\frenchspacing  % DO NOT CHANGE THIS
\setlength{\pdfpagewidth}{8.5in} % DO NOT CHANGE THIS
\setlength{\pdfpageheight}{11in} % DO NOT CHANGE THIS
%
% These are recommended to typeset algorithms but not required. See the subsubsection on algorithms. Remove them if you don't have algorithms in your paper.
\usepackage{algorithm}
\usepackage{algorithmic}

%
% These are are recommended to typeset listings but not required. See the subsubsection on listing. Remove this block if you don't have listings in your paper.
\usepackage{newfloat}
\usepackage{listings}
\DeclareCaptionStyle{ruled}{labelfont=normalfont,labelsep=colon,strut=off} % DO NOT CHANGE THIS
\lstset{%
	basicstyle={\footnotesize\ttfamily},% footnotesize acceptable for monospace
	numbers=left,numberstyle=\footnotesize,xleftmargin=2em,% show line numbers, remove this entire line if you don't want the numbers.
	aboveskip=0pt,belowskip=0pt,%
	showstringspaces=false,tabsize=2,breaklines=true}
\floatstyle{ruled}
\newfloat{listing}{tb}{lst}{}
\floatname{listing}{Listing}
%
% Keep the \pdfinfo as shown here. There's no need
% for you to add the /Title and /Author tags.
\pdfinfo{
	/TemplateVersion (2024.1)
}

% DISALLOWED PACKAGES
% \usepackage{authblk} -- This package is specifically forbidden
% \usepackage{balance} -- This package is specifically forbidden
% \usepackage{color (if used in text)
% \usepackage{CJK} -- This package is specifically forbidden
% \usepackage{float} -- This package is specifically forbidden
% \usepackage{flushend} -- This package is specifically forbidden
% \usepackage{fontenc} -- This package is specifically forbidden
% \usepackage{fullpage} -- This package is specifically forbidden
% \usepackage{geometry} -- This package is specifically forbidden
% \usepackage{grffile} -- This package is specifically forbidden
% \usepackage{hyperref} -- This package is specifically forbidden
% \usepackage{navigator} -- This package is specifically forbidden
% (or any other package that embeds links such as navigator or hyperref)
% \indentfirst} -- This package is specifically forbidden
% \layout} -- This package is specifically forbidden
% \multicol} -- This package is specifically forbidden
% \nameref} -- This package is specifically forbidden
% \usepackage{savetrees} -- This package is specifically forbidden
% \usepackage{setspace} -- This package is specifically forbidden
% \usepackage{stfloats} -- This package is specifically forbidden
% \usepackage{tabu} -- This package is specifically forbidden
% \usepackage{titlesec} -- This package is specifically forbidden
% \usepackage{tocbibind} -- This package is specifically forbidden
% \usepackage{ulem} -- This package is specifically forbidden
% \usepackage{wrapfig} -- This package is specifically forbidden
% DISALLOWED COMMANDS
% \nocopyright -- Your paper will not be published if you use this command
% \addtolength -- This command may not be used
% \balance -- This command may not be used
% \baselinestretch -- Your paper will not be published if you use this command
% \clearpage -- No page breaks of any kind may be used for the final version of your paper
% \columnsep -- This command may not be used
% \newpage -- No page breaks of any kind may be used for the final version of your paper
% \pagebreak -- No page breaks of any kind may be used for the final version of your paperr
% \pagestyle -- This command may not be used
% \tiny -- This is not an acceptable font size.
% \vspace{- -- No negative value may be used in proximity of a caption, figure, table, section, subsection, subsubsection, or reference
% \vskip{- -- No negative value may be used to alter spacing above or below a caption, figure, table, section, subsection, subsubsection, or reference

\setcounter{secnumdepth}{1} %May be changed to 1 or 2 if section numbers are desired.

% The file aaai24.sty is the style file for AAAI Press
% proceedings, working notes, and technical reports.
%

% Title

% Your title must be in mixed case, not sentence case.
% That means all verbs (including short verbs like be, is, using,and go),
% nouns, adverbs, adjectives should be capitalized, including both words in hyphenated terms, while
% articles, conjunctions, and prepositions are lower case unless they
% directly follow a colon or long dash
\title{Learning Hybrid Dynamics Models with Simulator-Informed Latent States}
\author {
	% Authors
	Katharina Ensinger\textsuperscript{\rm{1,2}},
	Sebastian Ziesche\textsuperscript{\rm1},
	Sebastian Trimpe\textsuperscript{\rm2}
}
\affiliations {
	% Affiliations
	\textsuperscript{\rm 1} Bosch Center for Artificial Intelligence, Renningen, Germany \\
	\textsuperscript{\rm 2} Institute for Data Science in Mechanical Engineering, RWTH Aachen Univeristy\\
	katharina.ensinger@bosch.com
}
%Example, Single Author, ->> remove \iffalse,\fi and place them surrounding AAAI title to use it


% REMOVE THIS: bibentry
% This is only needed to show inline citations in the guidelines document. You should not need it and can safely delete it.
\usepackage{bibentry}
% END REMOVE bibentry

\begin{document}

\maketitle

\begin{abstract}
	Dynamics model learning deals with the task of inferring unknown dynamics from measurement data and predicting the future behavior of the system.
	A typical approach to address this problem is to train recurrent models. 
	However, predictions with these models are often not physically meaningful.
	Further, they suffer from deteriorated behavior over time due to accumulating errors. 
	Often, simulators building on first principles are available being physically meaningful by design.
	However, modeling simplifications typically cause inaccuracies in these models.
	Consequently, hybrid modeling is an emerging trend that aims to combine the best of both worlds.  
	In this paper, we propose a new approach to hybrid modeling, where we inform the latent states of a learned model via a black-box simulator.
	This allows to control the predictions via the simulator preventing them from accumulating errors.
	This is especially challenging since, in contrast to previous approaches, access to the simulator's latent states is not available. 
	We tackle the task by leveraging observers, a well-known concept from control theory, inferring unknown latent states from observations and dynamics over time.  
	In our learning-based setting, we jointly learn the dynamics and an observer that infers the latent states via the simulator.
	Thus, the simulator constantly corrects the latent states, compensating for modeling mismatch caused by learning. 
	To maintain flexibility, we train an RNN-based residuum for the latent states that cannot be informed by the simulator. 
\end{abstract}

\section{Introduction}

Text-Attributed Graphs (TAGs) are a type of graph that have textual data as node attributes. 
These types of graphs are prevalent in the real world, such as in citation networks \cite{hu2020open} where the node attribute is the paper's abstract. TAGs have diverse potential applications, including paper classification \cite{chien2021node} and user profiling\cite{kim2020multimodal}. 
However, studying TAGs presents a significant challenge: how to model the intricate interplay between graph structures and textual features. 
This issue has been extensively explored in several fields, including natural language processing, information extraction, and graph representation learning. 

% Text-Attributed Graphs (TAGs) are a type of graph that is widely present in the real world. 
% In practical applications, many node features can be composed of text. For example, in citation networks, the node feature is the abstract of a paper, and in social networks, the node feature is the user's profile. 
% TAGs have broad potential application values, such as paper classification and user identification. 
% Modeling TAGs involves techniques from multiple fields, including information extraction, natural language processing, and graph representation learning, making it a hot academic topic currently.

An idealized approach involves combining pre-trained language models (PLMs) \cite{he2020deberta,liu2019roberta} with graph neural networks and jointly training them \cite{zhao2022learning,mavromatis2023train}. Nevertheless, this method requires fine-tuning the PLMs, which demands substantial computational resources. Additionally, trained models are hard to be reused in other tasks because finetuning PLM may bring catastrophic forgetting\cite{chen2020recall}. 

Therefore, a more commonly used and efficient approach is the two-stage process \cite{yang2021bert,zhang2022stance,malhotra2020classification}: (1) utilizing pre-trained language models (PLMs) for unsupervised modeling of the nodes' textual features. 
(2) supervised learning using Graph Neural Networks (GNNs). 
Compared to joint training of PLMs and GNNs, this approach offers several advantages in practical applications. 
For example, it can be combined with numerous GNN frameworks or PLMs, and this approach does not require fine-tuning PLMs for every downstream task.
However, PLMs are unable to fully leverage the wealth of information contained in the graph structure, which represents significant information. 
To overcome these limitations, some works propose self-supervised fine-tuning PLMs using graph information to extract graph-aware node features \cite{chien2021node}. Such methods have achieved significant success across various benchmark datasets\cite{hu2020open}. 
% Unsupervised modeling of nodes' textual features by language models (LM) and subsequent supervised learning of the graph feature by Graph Neural Networks (GNNs) is a classical and effective approach for processing TAGs. 
% However, the generated node representation is untrainable in downstream tasks, a unsuitable representation may affect the performance of subsequent GNNs learning. 
% To address limitations, many works merged recently, which investigate how to better utilize pre-trained language models in TAGs modeling. 
% A method is joint PLMs with GNNs by knowledge distillation. 
% and self-supervised fine-tuning PLMs to adapt graph data.   
% First, PLMs are fine-tuned by self-supervised tasks related to graphs, enabling them to capture and comprehend graph information. Then, the fine-tuned PLM is used to generate node representations.
% This approach has achieved significant results in numerous public datasets.


% However, these SSL-based node feature extraction methods suffer from the few-shot challenge. are based on graphs with over 100,000 nodes. 
% This means that during the self-supervised training phase, there are enough samples, and downstream task training samples are also abundant. 
% For example, in Ogbn-arxiv, there are over 70,000 training samples (60\%). 
% However, this situation poses a significant gap from the real world. 
% Firstly, training labels are often expensive, and secondly, there exist many small graphs in the real world.  

However, both self-supervised methods and using language models directly to process TAG suffer from a fundamental drawback. They cannot incorporate downstream task information, which results in identical representations being generated for all downstream tasks. This is evidently counterintuitive as the required information may vary for different tasks. For example, height is useful information in predicting a user's weight but fails to accurately predict age. This issue can be resolved by utilizing task-specific prompts combined with language models \cite{petroni2019language} to extract downstream task-related representations. For example, suppose we have a paper's abstract $\{\mathbf{Abstract}\}$ in a citation network, and the task is to classify the subject of the paper. We can add some prompts to a node's sentence:
$
    \{This, is, a, paper, of, [\mathbf{mask}], subject, its, abstract, is,:, \mathbf{Abstract}\}
$. And then use the embedding corresponding to the [mask] token generated by PLMs as the node feature. Yet this approach fails to effectively integrate graph information. 

To better integrate task-specific information and knowledge within graph structure, this paper proposes a novel framework called G-Prompt. G-Prompt combines a graph adapter and task-specific prompts to extract node features. Specifically, G-Prompt contains a graph adapter that helps PLMs become aware of graph structures. This graph adapter is self-supervised and trained by fill-mask tasks on specific TAGs. G-Prompt then incorporates task-specific prompts to obtain interpretable node representations for downstream tasks.



% However, we observe the SSL-based methods are in the small-sample scenario and found that: \\
% 1. The representations generated by large-scale language models perform similarly to word2vec in small-sample situations. This is clearly counterintuitive, as numerous experiments have shown that pre-trained language models can learn rich knowledge from massive text. \\
% 2. The representation of entire BERT models finetuned on graph self-supervised tasks such as GIANT performs similarly to the frozen language model's representation through GAE pre-training in extremely small sample sizes. However, overall, it outperforms graph-free representations. \\
% 3. Since using PLM-generated representations did not yield good results, we experimented with RoBERTa-based representations with task prompts, which performed the best in small-sample scenarios.

% This implies that both Graph-aware and Task-aware representations are crucial for node representation. 
% However, current methods \textbf{can not effectively combine} the two because current unsupervised node feature generation methods do not consider downstream tasks. 
% Meanwhile, pre-trained models cannot be task-specifically transformed. 
% There is a significant gap between self-supervised tasks and BERT's own pre-training tasks. 
% Directly finetuning BERT would destroy the prior knowledge learned from massive text data.

% Furthermore, current methods generate node features that \textbf{lack interpretability}. 
% The features generated by current methods are continuous and lack interpretability. 
% It is challenging to explain why a particular representation works, and it is difficult to manually select a few features for downstream tasks.
% Meanwhile, the current state-of-the-art methods require finetuning of pre-trained language models (PLMs). However, with the increasing size of PLMs, the computational cost of finetuning has become prohibitively high, often requiring a substantial amount of data to achieve good performance. Thus, it is challenging to integrate these methods with even more powerful language models.

% Therefore, this paper aims to explore the possibility of generating task-aware and graph-aware representations with BERT without finetuning. For the former, a naive method is to use prompts, which are manually input task-related hints, along with text features to generate corresponding words using a language model. For example, for citation networks, we can add prompt information before the abstract: "This is a paper published on <mask> subject, its abstract is [content]." We then use the word distribution after decoding the <mask> as a node feature. However, incorporating graph information into the prompt is challenging. To address this issue, we propose a new framework called GPrompt. This framework combines graph adapters and prompts to extract node features. The graph adapter operates on the last linear transformation layer that predicts words in the LM, i.e., a learnable graph neural network is added to that layer. The goal of the GNN is to help the language model perceive neighbor information of nodes and better predict the masked word. The graph adapter is trained through the language model's native fill-mask task. After the adapter is trained, GPrompt incorporates task-related prompts based on the fill-mask framework of the language model, combined with the graph adapter, to generate task-related representations that are interpretable and perceive graph information.


% pithc on parameter-efficient tuning, cite lora/adaptor
% However, replacing the linear transformation with GNN imposes huge computational costs, and it is not feasible to aggregate neighbors once for each token of every word. To speed up the training process, we adopt DecoupleGNN and use geometric mean to aggregate information from each neighbor. The geometric mean is equivalent to training neighbor nodes with the target node's label in the cross-entropy loss function, so there is no need to globally aggregate neighbor information during GraphAdapter training. This strategy accelerates training effectively through global edge sampling.

We conduct extensive experiments on three real-world datasets in the domains of few-shot and zero-shot learning, in order to demonstrate the effectiveness of our proposed method. The results of our experiments show that G-Prompt achieves state-of-the-art performance in few-shot learning, with an average improvement of \textit{avg.} 4.1\% compared to the best baseline. Besides, our G-Prompt embeddings are also highly robust in zero-shot settings, outperforming PLMs by \textit{avg.} 2.7\%. Furthermore, we conduct an analysis of the representations generated by G-Prompt and found that they have high interpretability with respect to task performance.








\section{Background}
\subsection{Text-Attributed Graph}

Let $G = \{V,A\}$ denotes a text-attributed graph (TAG), where $V$ is the node set and $A$ is the adjacency matrix. Each node $i \in V$ is associated with a sentence $S_i = \{s_{i,0},s_{i,1},...,s_{i,|S_i|}\}$, which represents the textual feature of the node. In most cases, the first token in each sentence (i.e., $s_{i,0}$) is $[\mathbf{cls}]$, indicating the beginning of the sentence. This paper focuses on how to unsupervised extract high-quality node features on TAGs for various downstream tasks.

\subsection{Pretrained Language Models}

Before we introduce G-Prompt, we require some basic concepts of pre-trained language models.

\textbf{Framework of PLMs}. A PLM consists of a multi-layer transformer encoder that takes a sentence $S_i$ as input and outputs the hidden states of each token:
\begin{equation}
    \mathbf{PLM}(\{s_{i,0}, s_{i,1},...,s_{i,|S_i|}\}) = \{h_{i,0}, h_{i,1},...,h_{i,|S_i|}\},
\end{equation}
where $h_{i,k}$ is the dense hidden state of $s_{i,k}$.

\textbf{Pretraining of PLMs}. The fill-mask task is commonly used to pre-train PLMs \cite{devlin2018bert,liu2019roberta,he2020deberta}. Given a sentence $S_i$, the mask stage involves randomly selecting some tokens and replacing them with either $[\mathbf{mask}]$ or random tokens, resulting in a modified sentence $\hat{S}_i = \{s_{i,0}, s_{i,1},...,\hat{s}_{i,k},...,s_{i,|S_i|}\}$, where $\hat{s}_{i,k}$ represents the masked token. In the filling stage, $\hat{S}_i$ is passed through the transformer encoder, which outputs the hidden states of each token. We denote the hidden state of the masked token $\hat{s}_{i,k}$ as $\hat{h}_{i,k}$, which is used to predict the ID of the masked token:
\begin{equation}
    \hat{y}_{i,k} = f_{\rm{LM}}(\hat{h}_{i,k}),
\end{equation}
where $f_{LM}$ is a linear transformation with softmax fuction, $\hat{y}_{i,k} \in \mathbb{N}^{1\times T}$, and $T$ is the size of the vocabulary. The loss function of the fill-mask task is defined as $\mathcal{L} = \rm{CE}(\hat{y}_{i,k}, y_{i,k})$, where $y_{i,k}$ is the ID of the masked token, and $\rm{CE}(\cdot,\cdot)$ is the cross-entropy loss.

\textbf{Sentence Embedding}. The hidden state of the $[\mathbf{cls}]$ token ($h_{i,0}$) and the mean-pooling of all hidden states are commonly used as sentence embeddings \cite{reimers2019sentence, gao2021simcse}.

\textbf{Prompting on PLMs}. Sentence embedding and token embedding are simultaneously pre-trained in many PLMs. However, due to the gap between pretraining tasks and downstream tasks, sentence embedding always requires fine-tuning for specific tasks. To address this issue, some studies utilize prompts to extract sentence features \cite{jiang2022promptbert}. For example, suppose we have a paper's abstract $\{\mathbf{Abstract}\}$, and the task is to classify the subject of it. We can add some prompts to the sentence:
\begin{equation}
    \{This, is, a, paper, of, [\mathbf{mask}], subject, its, abstract, is,:, \mathbf{Abstract}\}
\end{equation} 
Then this sentence is encoded by PLMs, and we let $h_{i|p}$ denote the hidden state of the $[\mathbf{mask}]$ token in prompts. Extensive experiment shows that using prompts can shorten the gap between PLMs and downstream tasks and maximize the utilization of the knowledge PLMs learned during pretraining.

\subsection{Graph Neural Networks}

Graph Neural Networks (GNNs) have achieved remarkable success in modeling graph-structured data\cite{velivckovic2017graph,gasteiger2018predict}. The message-passing framework is a commonly used architecture of GNN. At a high level, GNNs take a set of node features $X^0$ and an adjacency matrix $A$ as input and iteratively capture neighbors' information via message-passing. More specifically, for a given node $i \in V$, each layer of message-passing can be expressed as:
\begin{equation}
    x_i^{k} = \mathbf{Pool}\{f_\theta(x^{k-1}_j) | j\in \mathcal{N}_i\}
\end{equation} 
where $\mathbf{Pool}\{\cdot\}$ is an aggregation function that combines the features of neighboring nodes, such as mean-pooling. And $\mathcal{N}_i$ denotes the set of neighbors of node $i$. 
%Different GNN architectures employ different aggregation methods; for instance, GraphSAGE utilizes mean-pool while GAT incorporates an attention mechanism.


% \subsection{Modeling TAGs}
% Most GNNs are designed to operate on continuous node features and cannot handle textual features directly. As a result, modeling TAGs requires combining LMs and GNNs. The most straightforward approach is to join the structure of GNNs and LMs and then end-to-end train them. However, most current LMs are based on Transformers with enormous trainable parameters, so end-to-end training requires significant computing resources.

% Recently, impressive results have been achieved by combining LM and GNNs using the soft connection, e.g., knowledge distillation, and expectation-maximization framework. However, this approach involves fine-tuning LM, which is also extremely computationally expensive. Furthermore, the finetuned model is task-specific, are hard to employ in other downstream tasks.

% A convenient framework commonly used in various applications involves using PLMs to unsupervised convert the textual features of nodes into continuous representations. Then, the extracted node representation and graph structure can be input into GNNs for end-to-end training. It's worth noting that the converted node feature is reusable for many downstream tasks.


% \hxw{problem}
% !TEX root = main.tex

\section{Problem Setting} \label{section:problem}
In this section, we formulate our problem and demonstrate how the concepts in Sec. \ref{section:background} are related to it.
To this end, consider system \eqref{eq:dyn} with outputs $y_n$ (cf Eq.\eqref{eq:obs}).
Our goal is to make accurate predictions for $y_n$. 
Here, we consider the situation, where access to an inaccurate physics-based approximation $(\hat s_n)_{n=0}^N \in \mathbb{R}^{d_s}$ of the outputs is provided by a black-box simulator.
Thus, access to the simulator's latent states or the dynamics is not available. 
We propose to leverage the concepts in Sec. \ref{section:background} and learn how to reconstruct latent states from the simulator that are meaningful for the prediction task, thus maximizing the simulator's influence on the predictions. 
Since the simulator can typically not capture all dynamics, we introduce an additional residuum $r$. 

We model these concepts by splitting the latent state $x$ (cf Eq.~\eqref{eq:dyn}) into $u \in \mathbb{R}^{d_u}$ and $v \in \mathbb{R}^{d_v}$, where $d_x = d_u+d_v$.
With $u$, we denote the latent states that can be reconstructed from the simulator and refer to them as being \textbf{observable via the simulator (OVS)}.
The non-OVS states $v$ on the other hand cannot be reconstructed from the simulator.
We propose to formulate a common dynamics model for simulator and data by extending system \eqref{eq:partially_obs} via
\begin{equation}\label{eq:partially_obs}
\begin{aligned}
u_{n+1}&=f_u(u_n) \\
v_{n+1}&=f_v(u_n, v_n) \\ 
y_n & = g(u_n)+r(u_n,v_n) \\
\hat{s}_n & = h(u_n),
\end{aligned}
\end{equation}
with $f_u:\mathbb{R}^{d_u} \to \mathbb{R}^{d_u}$ and $f_v:\mathbb{R}^{d_u} \times \mathbb{R}^{d_v} \to \mathbb{R}^{d_v}$.
Here, $f_u$ determines the propagation of the OVS latent states $u$, while $f_v$ determines the propagation of the non-OVS latent states $v$. 
The observation model $g$ maps the OVS states $u$ to the OVS part of the predictions, while $r$ maps the OVS states $u$ and non-OVS states $v$ to the non-OVS residuum. 
The simulator is reconstructed via the observation model $h$.  
Our goal is to learn $f_u,f_v,g,h$ and $r$ from measurement data $\hat y$ and simulator outputs $\hat{s}$.

The key insight is that we force $u$ to be OVS by addressing $f_u$ via a trainable observer that receives $\hat s$ as an input.
By design, we obtain simulator-informed latent states $u$ and corresponding outputs $g(u)$.
In general, we can represent every system via Eq. \eqref{eq:partially_obs}.
By setting $d_v=0$ and removing the non-OVS residuum, we obtain the fully OVS case.  
By setting $d_u=0$, our model is reduced to the pure learning-based case. 
We choose an additive structure for the observation model in order to obtain control over the corresponding non-OVS residuum $r$.
However, different structures are also possible. 

 

 
\section{Method: G-Prompt}
Utilizing the information of downstream tasks and graphs is crucial for generating high-quality node representations. 
The term ``high quality'' is inherently task-specific, as exemplified by the fact that height is a useful feature in predicting user weight but fails to accurately predict age. 
Besides,  the valuable topological information of TAGs can significantly enhance the understanding of textual features in TAGs. 
However, extracting node features using both task and graph information simultaneously is significantly challenging. 
Current PLMs used for handling textual features are graph-free, while current graph-based methods employed to extract node features are primarily task-free. Therefore, this paper proposes a novel self-supervised method, G-Prompt, capable of extracting task-specific and graph-aware node representations. 

\begin{figure*}[t]
	\centering
	\includegraphics[width=0.95\textwidth]{./picture/Model.pdf}
	\caption{Framework of G-Prompt}
	\label{fig:exp}
\end{figure*}

\subsection{Overview}

While previous works have frequently employed PLMs to process TAGs, these investigations have been constrained in extracting a broad node representation from the text-based characteristics and have not incorporated task-specific prior knowledge. 
Consequently, additional learning supervision via GNNs is needed to enable the effective adaptation of these node representations to downstream tasks. 
To address this limitation, the paper suggests incorporating prompts and PLMs into the process of extracting task-relevant node features from TAGs.
%\hkq{which should be highlighted in the next sentence }
Nevertheless, PLMs only utilize contextual information to generate the prompts-related output, which may be insufficient for handling TAGs.
Graph structures often contain essential information that can facilitate a better understanding of textual features.
For instance, in a citation network, a masked sentence such as \textit{``This paper focuses on [MASK] learning in AI domain''} could have multiple candidate tokens based solely on context.
However, if many papers related to graphs are cited, we can infer with greater confidence that the masked token is likely \textit{``graph''}. 
At present, PLMs operate solely based on context, and their structure is graph-free. 
Directly incorporating graph information into PLMs by prompts is not feasible because limited prompts cannot describe the entire topological structure adequately.

Therefore, the proposed G-Prompt leverages a self-supervised based graph adapter and prompts to make PLMs aware of the graph information and downstream task. Given a specific TAG, the pipeline of G-Prompt is as follows: 
(1) Training an adapter on the given TAG to make PLMs graph-aware. 
Specifically, we propose a graph adapter that operates on the prediction layer of PLMs to assist in capturing graph information, which is fine-tuned by the fill-mask task based on the textual data contained by the given TAG. 
(2) Employing task-specific prompts and fine-tuned graph adapters to generate task-aware and graph-aware node features.

\subsection{Fine-Tuning PLMs with the Graph Adapter}

% Currently, PLMs that have undergone large-scale text data pre-training have strong contextual understanding abilities and generalization abilities which form the basis for us to extract specific task information using prompts. 
Using adapters to enable PLMs to perceive graph information is a straightforward idea. 
However, unlike adapters used for downstream task fine-tuning \cite{hu2021lora,liu2022few}, the graph adapter is used to combine prompts in order to extract task-relevant node representations. 
This is an unsupervised process, which means that the graph adapter only receives self-supervised training on given TAGs. 
Consequently, the most challenging aspect of graph adapters is how to assist PLMs in perceiving graph information while also maintaining their contextual understanding capability. 
Additionally, the graph adapter is only trained on a given TAG, generalizing to prompt tokens can also be quite difficult.
Next, we introduce the graph adapter and discuss how it overcomes these challenges in detail.

% The focus of this paper is on promoting the PLM to extract node features of TAGs, which is essentially a fill-mask task. Therefore, this paper proposes Graph Adapter, which \textbf{targets maximally retaining LMs' contextual modeling ability while enabling them to incorporate graph information during the fill-mask process.}

\textbf{Context-friendly adapter placement.} 
The fill-mask task involves two modules of PLMs: a transformer-based module that models context information to obtain representations of masked tokens and a linear transformation that decodes the representation to output the probable IDs of the masked token.
To avoid compromising the contextual modeling ability of PLMs, the Graph Adapter only perform on the last layer of PLMs.
More specifically, the graph adapter is a GNN structure combing with the pre-trained final layer of the PLMs.
Given a specific masked token $\hat{s}_{i,k}$, The inputs of the Graph Adapter are the masked token $\hat{h}_{i,k}$, sentence representations of node $i$ and its neighbors and output is the prediction of the IDs' of the masked token. 
This process aligns with intuition — inferring a possible token based on context first and then determining the final token based on graph information. Formally,
\begin{equation}
    \hat{y}_{i,k} =  \textbf{GraphAdapter} \{f_{\rm{LM}}, \hat{h}_{i, k}, z_i, \{z_j \in \mathcal{N}_i\}, \Theta\},
\end{equation}
where the $z_i$ and $z_j$ denote the sentence embedding of node $i$ and $j$. Note, sentence embedding is task-free and only used to model nodes' influence on their neighbor.
In this paper, we utilize sentence embedding of nodes' textual features as their node feature. 
$\Theta$ is all trainable parameters of the Graph Adapter. 

\textbf{Prompting-friendly network structure}.
% The hidden state of the masked token contains contextual information extracted through the transformer in PLMs. 
% Therefore, directly manipulating it may also affect the contextual information it contains.
The parameters of the adapter are only trained on the fill-mask task based on the textual data contained by the target TAG. 
But the adapter will be used for combining prompts to generate task-related node features in various subsequent downstream tasks.
So the generalization ability of the adapter is crucial. 
On the one hand, the distribution of hidden states responding to masked tokens in prompts may be different from the hidden states used to train the adapter. 
On the other hand, the candidate tokens for task-specific prompts may not appear in the tokens of the TAG. 
Therefore, we carefully design the network structure of the graph adapter and utilize the pre-trained prediction layer of PLM to improve its generalization ability of it.

When it comes to the graph adapter's training stage, it's possible that the hidden states associated with certain prompts may not be present. This means that directly manipulating those hidden states could result in overfitting the tokens already present in the TAGs.
Therefore, the graph adapter models the influence of each modeled node on the distribution of surrounding neighbor tokens based on node feature, which remains unchanged when prompts are added. Considering that some tokens can be predicted well based solely on their context and that different neighbors may have different influences on the same node, the impact of a neighbor on a token is determined jointly by a gate mechanism and the token's context. Give a specific node $i$, it's neighbor $j$, and hidden states of a masked token $\hat{h}_{i,j}$,
\begin{equation}
    \tilde{h}_{i, k, j} = a_{ij}\hat{h}_{i,k} + (1-a_{ij})g(z_j,\Theta_g)
\end{equation}
where $a_{ij} = \mathrm{sigmoid}((z_iW_q)(z_jW_k)^T)$. Here, $g(\cdot)$ represents multi-layer perceptions (MLPs) with parameters $\Theta_g$ that model the influence of node $j$.
It is worth noting that when considering the entire graph, $g(z_j, \Theta_g)$ will be combined with many marked tokens of node $j$'s neighbors, which can help to prevent $g(z_j, \Theta_g)$ from being overfitted on a few tokens.

Subsequently, the graph adapter combines all neighbor influence to infer the final prediction result. Since the prediction layer of PLM (i.e., $f_{LM}(\cdot)$) is well-trained on massive tokens, it also contains an amount of knowledge. Therefore, the graph adapter reuses this layer to predict the final result. 
\begin{equation}
    \tilde{y}_{i,k} =  \mathbf{Pool}\{f_{\rm{LM}}(\tilde{h}_{i, k, j}) | j\in \mathcal{N}_i\},
\end{equation}
In this equation, the $\mathbf{Pool}(\cdot)$ used in this paper is mean-pooling. 
It is worth noting that the position of $f_{\rm{LM}}(\cdot)$ can be interchanged with pooling since it is just a linear transformation. All trainable parameters in the graph adapter are denoted by $\Theta = \{\Theta_g, W_q, W_k\}$.


\subsection{Model optimization of G-Prompt}

The graph adapter is optimized by the original fill-mask loss, $\mathcal{L}_{i,k} = \mathrm{CE} (\tilde{y}_{i,k}, y_{i,k})$, where $\hat{y}_{i,k}$ is the predicted probability of the $k$-th masked token for the node $i$ and $y_{i,k}$ is the true label. We aim to minimize $\mathcal{L}_{i,k}$ with respect to $\Theta$. 

However, in actual optimization, the prediction results of $\tilde{y}_{i,k,j} = f_{\rm{LM}}(\tilde{h}_{i, k, j})$ consist of many small values because of the large vocabulary size of the language model. 
Therefore, using mean-pooling presents a significant problem as it is insensitive to these small values. For example, during some stages of the optimization process, a node may have mostly $0.9$ predictions for the ground truth based on each edge, with only a few being $0.1$. 
Averaging them together would result in a very smooth loss, making it difficult to train the influence of neighbors with temporarily predicted values of 0.1. 
To address this issue, we use geometric mean instead of mean-pooling in the finetuning stage of the graph adapter, which is more sensitive to small values. 
It is easy to prove that the geometric mean of positive numbers is smaller than the arithmetic means, making it harder to smooth and helping the model converge faster. formally, in finetuning stage, the loss function is:
\begin{equation}
    \mathcal{L}_{i,k} = - y_{i,k} \odot \log\{(\prod_{j\in \mathcal{N}_i}{\tilde{y}_{i,k,j}})^{1/|\mathcal{N}_i|}\}
    = -\sum_{j\in \mathcal{N}_i}{ \frac{1}{|\mathcal{N}_i|}y_{i,k}\odot \log(\tilde{y}_{i,k,j})} 
\end{equation}
On the right-hand side of the equation, we are essentially minimizing $\tilde{y}_{i,k,j}$ through the cross-entropy loss $\mathcal{L}_{i,k,j}= \frac{1}{|\mathcal{N}_i|}\mathrm{CE}(\tilde{y}_{i,k,j},y_{i,k})$. It is worth noting that the graph adapter is only performed on the last layer of PLMs. As a result, we can sample a set of masked tokens and preserve their hidden states inferred by the PLMs before training. This implies that training of graph adapters can be achieved with very few computing resources.
\subsection{Prompt-based Node Representations}
After training the graph adapter, it can be combined with task-specific prompts to generate task-specific and graph-aware node representations. Similar to other prompt-based approaches, we simply add task-specific prompts directly into the textual feature. For example, we might use the prompt ``This is a [MASK] user, consider their profile: [textual feature].'' Formally, this process can be expressed as $\hat{h}_{i|p} = \mathbf{PLM}(\{[P_0],[P_1]...[MASK],S_i\})$.
Where, $\hat{h}_{i|p}$ represents the hidden state of the inserted [MASK], while $[P_0],[P_1]...$ refers to the task-specific prompts. The resulting hidden state is then fed into the graph encoder to decode the most probable token.
\begin{equation}
    \hat{y}_{i|p} = \mathbf{Pool}\{{f_{\rm{LM}}(a_{i,j}\hat{h}_{i|p}+(1-a_{i,j})g(z_j,\Theta_g))} | j\in \mathcal{N}_i\}
\end{equation}
$\hat{y}_{i|p}$ is a $|D|$-dimensional vector, where $|D|$ is the size of the PLM vocabulary. Therefore, directly using this prediction result for node representation is not conducive to downstream tasks and storage. Thus, we use the filtered results as node features, denoted by 
$
    x_{i|p} = \mathrm{Filter}(\hat{y}_{i|p})
$. 
Note, each dimension represents the probability of a token being inferred by PLMs with the graph adapter based on node textual features, neighbors' information, and task-respected prompts. Intuitively, tokens that are unrelated to downstream tasks are almost the same for all nodes. 
Therefore, for $Y_{p} \in \mathbb{N}^{|V|\times|D|}$, which denotes prediction results of all nodes. This paper sorts all columns of $Y_p$ in descending order of standard deviation and keeps the top $M$ columns as the node features. Note, we can also manually select task-relevant tokens based on prior knowledge of the task and use them as node features. 
% \subsection{Model optimization of G-Prompt}
% The optimization objective of G-Prompt is straightforward: (1) model the impact of each neighbor on a node's word distribution and (2) infer masked words based on the influence of all neighbors. 
% However, in actual optimization, the large vocabulary size of the language model leads to the prediction results of $\hat{y}_{i,j,k}$ that consist of many small values after softmax. 
% Therefore, using mean-pooling during optimization presents a significant problem as it is insensitive to these small values. For example, during some stages of the optimization process, a node may have mostly 0.9 predictions for the ground truth based on each edge, with only a few being 0.1. 
% Averaging them together would result in a very smooth loss, making it difficult to train the neighbors with temporarily predicted values of 0.1. 
% To address this issue, we use geometric mean instead of mean-pooling in the finetuning stage of the Graph Adapter, which is more sensitive to small values. 
% It is easy to prove that the geometric mean of positive numbers is smaller than the arithmetic means, making it harder to smooth and helping the model converge faster. formally, in finetuning stage, the pooling function is:
% \begin{equation}
%     \mathbf{Pool}^{tr}\{\hat{y}_{i,j,k} | j\in \mathcal{N}_i\} = (\prod_{j\in \mathcal{N}_i}{\hat{y}_{i,j,k}})^{1/|\mathcal{N}_i|}
% \end{equation}

% Considering that multiplication may easily exceed the precision of calculations, we have expanded the loss function based on geometric mean aggregation during optimization. The formula is as follows:
% \begin{equation}
%     \mathcal{L}_{i,k} = - y_{i,k} \odot \log\{(\prod_{j\in \mathcal{N}_i}{\hat{y}_{i,j,k}})^{1/|\mathcal{N}_i|}\}
%     = -\sum_{j\in \mathcal{N}_i}{ \frac{1}{\mathcal{N}_i}y_{i,k}\odot \log(\hat{y}_{i,j,k})}
% \end{equation}

% Therefore, the whole training pipeline is: 
% !TEX root = neurips_2021.tex
\section{Related Work} \label{section:rel}
HM is an emerging trend and a certain type of grey-box modeling. 
While general grey-box models often respect structural prior knowledge as energy-preservation, invariants etc. \citep{NEURIPS2019_26cd8eca, gaussprinciple2020geist, geist2021corl}, HM combines physics-based simulations with data-driven models.
Many works consider HM for dynamical systems or time-series.
Like us, \citet{Linial2020GenerativeOM} approach their HM task with RNNs.
In particular, they infer the initial states and the parameters of an ODE via an LSTM. 
However, in contrast to our setting, the system can be fully explained by the simulator with optimized parameters.
Another common approach in HM is extending a physics-based dynamics model, e.g. additively, with neural ODEs \citep{yin2021augmenting, qian2021integrating, 9337893} or modeling unknown parts of the dynamics with NNs \citep{SU1992327}.
\citet{rackauckas2021universal} present a unified view on these types of models, allowing to jointly learn physical and NN parameters. 
Recently, variational autoencoders are used to decode the latent states of observations and simulator from data and encode predictions from latent states and the physical model \citep{takeishi2021physicsintegrated}. 
Existing approaches are extended in \citep{wehenkel2023robust} via a data augmentation concept improving the behavior on unseen data. 
However, all these models assume direct access to the latent simulator states. 
In contrast, we consider black-box simulators that provide access only to output trajectories.

However, some approaches consider black-box simulators as well. 
One branch of HM with black-box simulators deals with optimizing parameters of the simulator \citep{DBLP:conf/iclr/RuizSC19, aushev2020likelihoodfree}.
However, this is not the setting that we consider since the simulator is fully informative once the parameters are adapted. 
A typical approach in our setting is to learn the errors or residua of simulator predictions and data \citep{Forssell97combiningsemi-physical, Suhartono_2017}.
Similar to our approach, \citet{ensinger2023combining} aim to control the long-term behavior of the predictions via the simulator.
To this end, they propose a complementary filtering approach. 
However, in contrast to their approach, our method is not restricted to simulators with correct low-frequency behavior. 
Furthermore, none of these works informs the latent states of the learning-based component. 
Another possibility is to provide the simulator as control input to an RNN \citep{SCHON202219, 10.1145/3447814}.
As discussed in detail in Sec. \ref{sec:method}, this can easily lead to an underestimation of the simulator. 

KKL observers have been combined with learning in different ways.  
In \citep{9304163}, nonlinear regression via NNs is performed in order to learn the nonlinear
transformation of a KKL observer. \citet{2204.00318} build up on the approach by optimizing the choice of
the controllable pair. \citet{9683485} propose a learning-based observer design by learning the nonlinear
transformation of a KKL observer with autoencoders. In contrast to our approach, all of these works consider
observer design for a dynamical system with \emph{known} dynamics.
However, some approaches consider KKL observers in the context of dynamics model learning.
\citet{buisson-fenet2023recognition} propose a KKL-based recognition model for neural ODEs. 
The initial latent state is obtained by running a KKL observer forward or backward. 
In contrast to our setting, the remaining rollout is not observer-based.
Furthermore, they do not consider HM. 
\citet{9683277} propose to construct an output predictor via a KKL observer.
The framework can be trained similar to standard recurrent architectures.
But in contrast to those architectures, mathematical guarantees for the output predictor can be obtained.
However, they do not consider HM. 
In contrast, we leverage the properties of KKL observers in order to control the behavior of the system by informing the latent states via the simulator. 



% SECTION Experiements %
\section{Experiments and Analysis} \label{Experiments}
\subsection{Experimental Design}
% This section outlines our methodology to propose a new classification model ViCGCN. Firstly, three benchmark datasets mentioned in Section \ref{Experiments/Datasets} are collected, and they be cleaned as described in the following. Consequently, the data after pre-processing is used to train our baselines and proposed model. With each model implemented, we fine-tune to find optimal hyper-parameters and improve their performance. Then, we evaluated the performance of models by Macro F1-score and Weighted F1-score deputed in Section \ref{Experiments/Metrics}. Section \ref{Experiments/Result} details the model evaluation results. To better understand our proposed model, we analyze and discuss the proposed model from various aspects: Impact of graph convolutional networks (see Section \ref{imapactGCN}) and impact of lambda (see Section \ref{impactlamda}). In addition, comparisons with previous studies were made to assess the achievement of our study correctly (see Section \ref{comparisonprestudies}). We also select the model that exhibits the most exceptional performance to conduct an error analysis on the inaccurate predictions detected within our proposed model (see Section \ref{erroranalysis}). At the end of the experiment, an ablation study was made to investigate the effectiveness and contribution of our proposed approach ViCGCN (see Section \ref{ablationstudy}). Figure \ref{fig::Experiments/Procedure/Overview} illustrates our methodology, including data preparation, fine-tuning baselines, proposed model, and performance analysis.

This section delineates our approach for introducing a novel classification model called ViCGCN. Initially, we gather three benchmark datasets, as mentioned in Section \ref{Experiments/Datasets}, and subject them to a cleaning process described subsequently. Subsequently, the pre-processed data is employed for training both our baseline models and the proposed model. We fine-tune each model to identify optimal hyperparameters and enhance their performance. The evaluation of model performance is conducted using Macro F1-score and Weighted F1-score, as discussed in Section \ref{Experiments/Metrics}. Detailed results of model evaluations are presented in Section \ref{Experiments/Result}.

To gain a deeper insight into our proposed model, we conduct a comprehensive analysis and discussion from various angles. This includes assessing the impact of graph convolutional networks (see Section \ref{imapactGCN}) and the influence of the lambda parameter (see Section \ref{impactlamda}). Additionally, we make comparisons with prior studies to accurately gauge the accomplishments of our research (see Section \ref{comparisonprestudies}). Furthermore, we select the model that exhibits the most outstanding performance to carry out an error analysis on the inaccuracies detected within our proposed model (see Section \ref{erroranalysis}). Towards the end of the experiment, we conduct an ablation study to investigate the effectiveness and contribution of our proposed approach, ViCGCN (see Section \ref{ablationstudy}). Figure \ref{fig::Experiments/Procedure/Overview} provides an overview of our methodology, encompassing data preparation, baseline fine-tuning, the proposed model, and performance analysis.

\begin{figure}[!hpbt]
    \centering
    \includegraphics[width=\textwidth]{Procedures.png}
    \caption{Overview of our experimental design.}
    \label{fig::Experiments/Procedure/Overview}
\end{figure}

\subsection{Baseline Models} \label{Experiments/Baseline}
Contextualized language models have been extensively used in various natural language processing tasks, including text classification. Additionally, since PhoBERT and viBERT are monolingual models specifically designed for the Vietnamese language, comparing their performance with a widely used and established model like mBERT is essential. Furthermore, as GCN has been shown to effectively capture the context and relationships between words in a text, integrating it with a contextualized language model could improve its performance in text classification tasks. Because of the following reasons, we compare our ViCGCN model with baseline models.
\subsubsection{Contextualized Language Models}
\begin{itemize}
    % \item \textbf{BERT\footnote{\url{https://github.com/google-research/bert}}}: BERT is a contextualized word representation model pre-trained using bidirectional transformers and based on a masked language model. BERT showed power in various NLP tasks. BERT and its variants are called the BERTology, two versions of BERT, base and large, respectively. Moreover, each version has two different versions: cased\footnote{\url{https://huggingface.co/bert-base-cased}} and uncased\footnote{\url{https://huggingface.co/bert-base-uncased}}, respectively. The only difference is that in \textit{BERT case uncased}, the text has been lowercase before the WordPiece tokenization step, while in the mBERT cased version, the text is the same as the input text.
    % \item \textbf{BERT (case uncased)}
    \item \textbf{Multilingual BERT (mBERT)\footnote{\url{https://github.com/google-research/bert}}}: mBERT, introduced by \citet{devlin-etal-2019-bert}, is a BERT-based model with specific characteristics. It consists of 12 layers, 768 hidden units, 12 attention heads, and a total of 110 million parameters. Remarkably, mBERT is designed to support 104 distinct languages, and it has been trained on and can be applied to text in these 104 languages using a combination of masked language modeling (MLM) and next sentence prediction objectives. This training corpus includes content from Wikipedia\footnote{\url{https://www.wikipedia.org/}}. \textit{mBERT} consists of two versions cased\footnote{\url{https://huggingface.co/bert-base-multilingual-cased}} and uncased\footnote{\url{https://huggingface.co/bert-base-multilingual-uncased}}.
    \item \textbf{RoBERTa\footnote{\url{https://huggingface.co/roberta-base}}}: \citet{DBLP:journals/corr/abs-1907-11692} proposed RoBERTa. They utilize a dynamic masking technique during the training process, instructing the model to predict intentionally hidden segments of text within unannotated language samples. RoBERTa, implemented using the PyTorch framework, makes critical adjustments to BERT's essential hyperparameters.
    \item \textbf{XLM-RoBERTa (XLM-R)\footnote{\url{https://github.com/facebookresearch/XLM}}}: \citet{XLMR}  proposed XLM-R a masked language model based on the transformer architecture. This model stands out as a multilingual powerhouse, having been pre-trained on text from a staggering 100 languages. What makes XLM-R particularly impressive is the extensive and careful curation of over 2.5TB of data from CommonCrawl. Among its notable contributions are the improvements made for low-resource languages through specialized training and vocabulary expansion. Moreover, XLM-R boasts a more expansive shared vocabulary and a substantial increase in its overall model capacity, incorporating a whopping 550 million parameters. XLM-R includes \textit{base}\footnote{\url{https://huggingface.co/xlm-roberta-base}} and \textit{large}\footnote{\url{https://huggingface.co/xlm-roberta-large}} version.
    \item \textbf{PhoBERT\footnote{\url{https://huggingface.co/vinai/phobert-base}}}: \citet{nguyen-tuan-nguyen-2020-phobert} introduced a set of large-scale monolingual language models specifically designed for the Vietnamese language. Among these models, PhoBERT stands out as the state-of-the-art contextualized language model for Vietnamese. PhoBERT's architecture is built upon the RoBERTa model, but it has been optimized for training on a substantial Vietnamese corpus to effectively handle Vietnamese text. PhoBERT comes in two versions: \textit{base} and the \textit{large} versions.
    \item \textbf{viBERT\footnote{\url{https://huggingface.co/FPTAI/vibert-base-cased}}}: \citet{viBERT} introduced viBERT, a pre-trained language model for Vietnamese based on the BERT architecture. The architecture of viBERT is similar to that of mBERT, and it has been pre-trained on a large corpus of 10GB of uncompressed Vietnamese text. However, unlike mBERT, viBERT excludes insufficient vocabulary due to the inclusion of languages other than Vietnamese in the mBERT vocabulary.
    \item \textbf{vELECTRA\footnote{\url{https://huggingface.co/FPTAI/velectra-base-discriminator-cased}}}: \citet{viBERT}  unveiled vELECTRA, a pre-trained language model tailored for Vietnamese that adheres to the ELECTRA framework. vELECTRA shares a parallel architectural structure with ELECTRA and has undergone pretraining on an extensive corpus comprising 60GB of uncompressed Vietnamese text.
\end{itemize}

\subsubsection{Other Graph Neural Networks}
Bert-GCN was introduced by \citet{BertGCN}, presenting a novel approach that harnesses the benefits of extensive pretraining alongside transductive learning for the purpose of text classification. Bert-GCN achieves this by constructing a diverse graph over the dataset, where documents are represented as nodes, all leveraging the embedding power of BERT. Consequently, this research undertakes the implementation of various BERT variations, such as multilingual and Vietnamese monolingual models, in conjunction with GCN-combined models to assess their effectiveness in text classification for Vietnamese tasks. Additionally, when compared to mBERT-GCN, RoBERTa-GCN, viBERT-GCN, and vELECTRA-GCN, our proposed ViCGCN model offers valuable insights into the impact of integrating both monolingual and multilingual Contextualized Language Models with GCN on three standardized benchmark datasets. 

\subsection{Benchmark Datasets} \label{Experiments/Datasets}
\subsubsection{Benchmark Datasets} \label{Experiments/Datasets/Data}
To verify the efficiency of our proposed approach to text classification on Vietnamese social media, we conducted our experiments on three widely used Vietnamese social media corpora, including Vietnamese Social Media Emotion Corpus (UIT-VSMEC) that was made available by Ho et al. \citet{DBLP:journals/corr/abs-1911-09339}, Vietnamese Students' Feedback Corpus (UIT-VSFC) built by \citet{VSFC}, and Vietnamese Constructive and Toxic Speech Detection (UIT-ViCTSD) introduced by \citet{DBLP:journals/corr/abs-2103-10069}.


\begin{itemize}
    \item \textbf{UIT-VSMEC \citet{DBLP:journals/corr/abs-1911-09339}}: UIT-VSMEC consists of 6,927 sentences that have been annotated with emotions to tackle the challenge of identifying emotions in Vietnamese social media comments. This dataset encompasses seven emotion categories: Enjoyment, Disgust, Sadness, Anger, Fear, Surprise, and Other.
    \item \textbf{UIT-VSFC \citet{VSFC}}: UIT-VSFC comprises 16,000 sentences that have been investigated for two distinct purposes: one related to sentiment analysis and the other related to topic classification. The sentiment analysis task involves categorizing sentences into three classes: Positive, Negative, and Neutral. Meanwhile, the topic classification task involves assigning sentences to one of four categories: Lecturer, Curriculum, Facility, or Others.
    \item \textbf{UIT-ViCTSD \cite{DBLP:journals/corr/abs-2103-10069}}: UIT-ViCTSD consists of 10,000 human-annotated comments on ten domains. Each comment is categorized into two tasks: constructiveness and toxicity in Vietnamese social media, which are binary classifications. Two categories are used to denote feedback: constructive and non-constructive. Similarly, comments can be labeled as toxic or non-toxic to identify harmful behavior.
\end{itemize}

\subsubsection{Pre-processing techniques}
A few efficient pre-processing techniques for Vietnamese text in general and Vietnamese social media text in particular were presented \cite{nguyen2020exploiting, PhoBERT-CNN}. However, we only follow some simple preprocessed techniques according to the quality of the three benchmark datasets mentioned in Section \ref{Experiments/Datasets/Data} and more essential to prove the outperform and efficiency of our model ViCGCN on Vietnamese social media raw text. Firstly, we removed stopwords defined in Vietnamese stopwords dict\footnote{\url{https://github.com/stopwords/vietnamese-stopwords}}. We, then, segment sentences into words by applying Word Segmenter of VnCoreNLP\footnote{\url{https://github.com/vncorenlp/VnCoreNLP}} for all of the models. Finally, the Regex\footnote{\url{https://docs.python.org/3/library/re.html}} library in Python is used to remove all punctuations in three benchmark datasets.

% remove stopwords, segmentation, remove punctuation 

The statistics of the pre-processed datasets are summarized in Table \ref{4/Dataset}.

% Experiments/Datasets/Table %
    
\begin{table}[!ht]
\centering
\caption{Statistics and descriptions of tasks of each dataset in this study.}
\resizebox{\linewidth}{!}{%
\begin{tabular}{lrrrlr} 
\hline
\textbf{Dataset}            & \multicolumn{1}{l}{\textbf{Train}} & \multicolumn{1}{l}{\textbf{Dev}} & \multicolumn{1}{l}{\textbf{Test}} & \multicolumn{1}{c}{\textbf{Task}}         & \multicolumn{1}{l}{\textbf{Classes}}  \\ 
\hline
\multicolumn{6}{c}{\textit{Binary text classification}}                                                                                                                                                                     \\ 
\hline
\multirow{2}{*}{UIT-ViCTSD} & 7,000                              & 2,000                            & 1,000                             & Constructive speech detection             & 2                                     \\
                            & 7,000                              & 2,000                            & 1,000                             & Toxic speech detection                    & 2                                     \\ 
\hline
\multicolumn{6}{c}{\textit{Multi-class text classification}}                                                                                                                                                                \\ 
\hline
\multirow{2}{*}{UIT-VSMEC}  & 5,548                              & 686                              & 693                               & Emotion recognition (with Other label)    & 7                                     \\
                            & 4,527                              & 583                              & 589                               & Emotion recognition (without Other label) & 6                                     \\ 
\cline{1-1}
\multirow{2}{*}{UIT-VSFC}   & 11,426                             & 1,583                            & 3,166                             & Sentiment-based classification            & 3                                     \\
                            & 11,426                             & 1,583                            & 3,166                             & Topic-based classification                & 4                                     \\
\hline
\end{tabular}}
\label{4/Dataset}
\end{table}


\subsection{Evaluation Metric} \label{Experiments/Metrics}
% This section describes the performance evaluation metrics employed in this study. The commonly used metric for classification tasks, particularly for the three datasets mentioned in this study, is the Average Macro F1-score (\%). However, owing to significantly imbalanced classes in the given datasets, the most suitable metric for this study is the average macro F1-score, which is the harmonic mean of Precision and Recall. Additionally, to facilitate comparisons with previous studies, we used the corresponding measure based on the metrics used in those studies, such as the average weighted F1-score (\%) for both UIT-VSMEC and UIT-VSFC.

This section outlines the performance evaluation criteria utilized in this research. In the realm of classification tasks, especially concerning the three datasets highlighted within this study, the conventional metric employed is the Average Macro F1-score (\%). However, given the significant class imbalances in the provided datasets, the most appropriate metric for this study is the Average Macro F1-score, which is derived as the harmonic mean of Precision and Recall. Furthermore, to facilitate comparisons with prior studies, we have also adopted relevant measures based on the metrics employed in those studies, such as the Average Weighted F1-score (\%) for both UIT-VSMEC and UIT-VSFC datasets.

 To compute the average macro F1-score, firstly, we calculate Precision and Recall by Equation (\ref{eq::Experiments/Metrics/Presision}) and Equation (\ref{eq::Experiments/Metrics/Recall}) respectively. Then, Equation (\ref{eq::Experiments/Metrics/F1-score}) is used to determine F1-score per class in the dataset. $tp$ are truly positive, $fp$ – false positive, $fn$ – false negative, and $tn$ – true negative counts, respectively.
\begin{equation}
    Precision = \frac{tp}{tp+fp} \label{eq::Experiments/Metrics/Presision}
\end{equation}
\begin{equation}
    Recall=\frac{tp}{tp+fn} \label{eq::Experiments/Metrics/Recall}
\end{equation}
\begin{equation}
    \textit{F1-score}=2\times\frac{Precision\times Recall}{Precision+Recall} \label{eq::Experiments/Metrics/F1-score}
\end{equation}

We compute the average macro F1-score (mF1) and weighted F1-score (wF1) after acquiring the F1 scores for all classes. Equation (\ref{eq::Experiments/Metrics/macro F1-score}) and Equation (\ref{eq::Experiments/Metrics/weighted F1-score}) present the macro F1-score and weighted F1-score, respectively, for multi-class classification for multi classes $C_{i}$, i $\in$ \{1, 2,... n\} (denoted for every class of the dataset). Where $\textit{F1-score}_{i}$ and $W_{i}$ are the \textit{F1-score} and weight of class \textit{i} of the dataset, respectively.

\begin{equation}
    \textit{mF1} = \frac{{\sum_{i=1}^{n} \textit{F1-score}_{i}}}{n} \label{eq::Experiments/Metrics/macro F1-score}
\end{equation}
\begin{equation}
    \textit{wF1} = \frac{\sum_{i=1}^{n} {\textit{F1-score}_{i} \times W_{i}}}{\sum_{i=1}^{n} W_{i}} \label{eq::Experiments/Metrics/weighted F1-score}
\end{equation}
\subsection{Experiment Configuration}
% In this study, we implemented many transfer learning models including $\text{BERT}_{base}$ \textit{cased}, $\text{BERT}_{base}$ \textit{uncased}, mBERT \textit{cased}, mBERT \textit{uncased}, $\text{RoBERTa}_{large}$, $\text{PhoBERT}_{base}$, $\text{PhoBERT}_{large}$. Besides, several combined models are conducted along with Text-GCN, Bert-GCN and mBERT-GCN. 
Section \ref{Experiments/Configuration/Baseline} and Section \ref{Experiments/Configuration/Proposed} provide our settings for both baselines and the proposed approach in detail.

\subsubsection{Basesline models' configuration} \label{Experiments/Configuration/Baseline}
We implemented many transfer learning models including mBERT both \textit{cased} and \textit{uncased}, $\text{RoBERTa}$, XLM-R, $\text{PhoBERT}_{base}$, $\text{PhoBERT}_{large}$, vELECTRA, and viBERT in this study. They run with their max sequence length of 256, batch size of 32, epoch of 10, and Adam optimizer \cite{https://doi.org/10.48550/arxiv.1412.6980} with a fixed learning rate of 2e-5.
 
\subsubsection{Our approach's configuration} \label{Experiments/Configuration/Proposed}
In our proposed approach, $\text{PhoBERT}_{base}$ is the output feature of the [CLS] token as the sentence node, followed by a feedforward layer to derive the final prediction. We use $\text{PhoBERT}_{base}$ pre-trained model from HuggingFace combined with a two-layer GCN to implement ViCGCN. We initialize Adam optimizer \cite{https://doi.org/10.48550/arxiv.1412.6980} with a fixed learning rate of 1e-3 and 1e-5 for the GCN and PhoBERT module, respectively. Moreover, PhoBERT runs with a 256 max sequence length.

\subsection{Experimental Results} \label{Experiments/Result}


% \begin{table}[!hpbt]
% \centering
% \caption{F1-score performances of models on the test sets of various Vietnamese social media textual datasets. Improvement (1) and Improvement (2) denoted the improvement over BERTology models and the improvement over BERTology integrated with GCN models, respectively}
% \label{tab::Experiments/Result}
% \resizebox{\linewidth}{!}{%
% \begin{tabular}{c|cc|cc|cc|cc|cc|cc} 
% \hline
% \textbf{Datasets}                & \multicolumn{4}{c|}{\textbf{UIT-VSMEC}}                                               & \multicolumn{4}{c|}{\textbf{UIT-ViCTSD}}                                                & \multicolumn{4}{c}{\textbf{UIT-VSFC}}                                                     \\ 
% \hline
% \textbf{Tasks}                   & \multicolumn{2}{c|}{\textbf{Seven labels}} & \multicolumn{2}{c|}{\textbf{Six labels}} & \multicolumn{2}{c|}{\textbf{Constructiveness}} & \multicolumn{2}{c|}{\textbf{Toxicity}} & \multicolumn{2}{c|}{\textbf{Sentiment-based}} & \multicolumn{2}{c}{\textbf{Topic-based}}  \\ 
% \hline
%                                  & \textbf{wF1}   & \textbf{mF1}              & \textbf{wF1}   & \textbf{mF1}            & \textbf{wF1}   & \textbf{mF1}                  & \textbf{wF1}   & \textbf{mF1}          & \textbf{wF1}   & \textbf{mF1}                 & \textbf{wF1}   & \textbf{mF1}             \\ 
% \hline
% BERT (cased)                     & 55.06          & 55.88                     & 66.34          & 65.31                   & 79.34          & 77.29                         & 87.45          & 64.49                 & 86.32          & 72.38                        & 86.17          & 73.62                    \\
% BERT (uncased)                   & 55.98          & 56.40                     & 58.55          & 56.65                   & 80.42          & 78.90                         & 85.21          & 63.28                 & 85.51          & 71.44                        & 85.98          & 72.56                    \\
% mBERT (\textit{cased)}           & 61.23          & 59.57                     & 66.72          & 66.72                   & 78.15          & 76.93                         & 86.71          & 64.36                 & 91.21          & 78.94                        & 88.06          & 77.63                    \\
% mBERT (\textit{uncased)}         & 58.31          & 57.03                     & 67.13          & 67.70                   & 78.11          & 76.63                         & 87.65          & 64.96                 & 89.52          & 76.60                        & 87.12          & 76.96                    \\
% RoBERTa                          & 56.51          & 56.22                     & 62.11          & 64.64                   & 78.92          & 77.71                         & 83.82          & 61.73                 & 90.11          & 76.98                        & 87.03          & 76.77                    \\
% XLM-R              & 69.55          & 68.12                     & 68.66          & 68.32                   & 81.97          & 80.02                         & 88.35          & 65.21                 & 91.02          & 76.95                        & 87.32          & 76.25                    \\
% PhoBERT \textit{base}            & 71.86          & 69.58                     & 75.19          & 74.32                   & 81.03          & 79.53                         & 88.83          & 65.61                 & 90.51          & 76.47                        & 87.84          & 76.98                    \\
% PhoBERT \textit{large}           & 72.87          & 70.22                     & 76.22          & 75.32                   & 83.21          & 80.22                         & 89.32          & 66.21                 & 91.81          & 77.81                        & 88.12          & 77.22                    \\
% viBERT                           & 67.55          & 65.32                     & 69.95          & 69.08                   & 82.27          & 80.12                         & 88.13          & 64.35                 & 89.77          & 76.25                        & 87.43          & 76.25                    \\ 
% \hline
% Bert-GCN (BERT \textit{cased)}   & 74.22          & 73.51                     & 78.25          & 77.02                   & 81.13          & 79.72                         & 88.10          & 64.52                 & 88.72          & 76.23                        & 88.15          & 76.25                    \\
% Bert-GCN (BERT \textit{uncased)} & 74.18          & 73.29                     & 80.54          & 78.31                   & 81.05          & 79.60                         & 88.67          & 64.96                 & 88.51          & 75.99                        & 87.75          & 76.06                    \\
% mBERT-GCN (mBERT cased)          & 75.12          & 74.83                     & 79.55          & 77.84                   & 82.15          & 80.33                         & 89.13          & 65.13                 & 91.89          & 79.84                        & 89.73          & 79.02                    \\
% mBERT-GCN (mBERT uncased)        & 74.56          & 73.98                     & 80.21          & 78.12                   & 82.88          & 81.12                         & 89.83          & 65.89                 & 91.72          & 79.64                        & 88.89          & 78.82                    \\
% RoBERTa-GCN                      & 74.82          & 74.22                     & 79.33          & 78.32                   & 83.47          & 82.77                         & 86.55          & 64.33                 & 91.12          & 79.32                        & 90.12          & 79.34                    \\
% viBERT-GCN                       & 78.25          & 78.37                     & 82.33          & 81.98                   & 86.12          & 85.02                         & 91.27          & 75.93                 & 93.27          & 87.52                        & 92.11          & 88.35                    \\ 
% \hline
% \textbf{ViCGCN}                  & \textbf{80.24} & \textbf{80.96}            & \textbf{84.91} & \textbf{83.27}          & \textbf{86.97} & \textbf{85.81}                & \textbf{91.95} & \textbf{76.29}        & \textbf{94.81} & \textbf{88.80}               & \textbf{93.91} & \textbf{89.61}           \\
% Improvement (1)                  & $\uparrow$7.37 & $\uparrow$10.74           & $\uparrow$8.69 & $\uparrow$7.95          & $\uparrow$3.76 & $\uparrow$5.59                & $\uparrow$2.63 & $\uparrow$9.98        & $\uparrow$3.00 & $\uparrow$9.86               & $\uparrow$5.69 & $\uparrow$11.98          \\
% Improvement (2)                  & $\uparrow$1.99 & $\uparrow$2.59            & $\uparrow$2.58 & $\uparrow$1.29          & $\uparrow$0.85 & $\uparrow$0.79                & $\uparrow$0.68 & $\uparrow$0.26        & $\uparrow$1.54 & $\uparrow$1.28               & $\uparrow$1.70 & $\uparrow$1.26           \\
% \hline
% \end{tabular}}
% \end{table}

\begin{table}[!ht]
\centering
\caption{F1-score performances of models on the test sets of various Vietnamese social media textual datasets. Improvement (1) and Improvement (2) denoted the improvement over BERTology models and the improvement over BERTology integrated with GCN models, respectively.}
\label{tab::Experiments/Result}
\resizebox{\linewidth}{!}{%
\begin{tabular}{c|cc|cc|cc|cc|cc|cc} 
\hline
\textbf{Datasets}        & \multicolumn{4}{c|}{\textbf{UIT-VSMEC}}                                               & \multicolumn{4}{c|}{\textbf{UIT-ViCTSD}}                                                & \multicolumn{4}{c}{\textbf{UIT-VSFC}}                                                     \\ 
\hline
\textbf{Tasks}           & \multicolumn{2}{c|}{\textbf{Seven labels}} & \multicolumn{2}{c|}{\textbf{Six labels}} & \multicolumn{2}{c|}{\textbf{Constructiveness}} & \multicolumn{2}{c|}{\textbf{Toxicity}} & \multicolumn{2}{c|}{\textbf{Sentiment-based}} & \multicolumn{2}{c}{\textbf{Topic-based}}  \\ 
\hline
                         & \textbf{wF1}   & \textbf{mF1}              & \textbf{wF1}   & \textbf{mF1}            & \textbf{wF1}   & \textbf{mF1}                  & \textbf{wF1}   & \textbf{mF1}          & \textbf{wF1}   & \textbf{mF1}                 & \textbf{wF1}   & \textbf{mF1}             \\ 
\hline
mBERT (\textit{cased)}   & 60.47          & 59.48                     & 65.02          & 62.65                   & 81.03          & 79.55                         & 88.32          & 65.63                 & 90.39          & 77.15                        & 87.32          & 77.93                    \\
mBERT (\textit{uncased)} & 60.17          & 59.18                     & 64.93          & 62.11                   & 80.89          & 79.47                         & 87.6           & 64.77                 & 89.95          & 77.8                         & 87.62          & 77.58                    \\
RoBERTa                  & 58.17          & 57.32                     & 63.32          & 59.97                   & 77.41          & 75.62                         & 85.85          & 59.71                 & 87.13          & 75.52                        & 86.77          & 75.30                    \\
XLM-R                    & 62.02          & 61.01                     & 68.19          & 63.70                   & 81.81          & 80.85                         & 89.92          & 73.09                 & 93.03          & 82.61                        & 89.67          & 79.25                    \\
PhoBERT \textit{base}    & 64.36          & 61.41                     & 69.02          & 64.12                   & 81.65          & 80.24                         & 89.58          & 72.12                 & 92.94          & 82.15                        & 88.29          & 78.54                    \\
PhoBERT \textit{large}   & 65.12          & 63.23                     & 71.13          & 65.12                   & 82.07          & 81.27                         & 90.12          & 73.32                 & 93.24          & 82.96                        & 88.72          & 79.12                    \\
vELECTRA                 & 63.58          & 61.38                     & 68.33          & 63.12                   & 82.41          & 80.82                         & 89.33          & 72.02                 & 91.89          & 82.01                        & 88.12          & 78.12                    \\
viBERT                   & 61.33          & 60.28                     & 68.48          & 62.09                   & 81.62          & 80.07                         & 89.14          & 71.87                 & 91.29          & 81.95                        & 88.22          & 78.35                    \\ 
\hline
mBERT-GCN (cased)        & 68.32          & 64.32                     & 69.32          & 66.18                   & 83.12          & 82.88                         & 90.32          & 69.42                 & 92.12          & 79.32                        & 88.32          & 79.42                    \\
mBERT-GCN (uncased)      & 67.98          & 64.11                     & 69.12          & 65.89                   & 82.32          & 82.01                         & 89.15          & 68.32                 & 91.01          & 79.02                        & 88.07          & 79.02                    \\
RoBERTa-GCN              & 66.17          & 62.12                     & 67.12          & 64.17                   & 81.33          & 80.96                         & 89.02          & 64.32                 & 90.12          & 78.42                        & 87.45          & 78.12                    \\
vELECTRA-GCN             & 69.42          & 65.44                     & 70.95          & 67.20                   & 84.62          & 84.62                         & 91.88          & 74.85                 & 93.56          & 83.12                        & 89.95          & 80.02                    \\
viBERT-GCN               & 69.32          & 65.12                     & 70.83          & 66.68                   & 84.32          & 83.12                         & 91.12          & 74.25                 & 93.12          & 82.47                        & 89.42          & 79.63                    \\ 
\hline
\textbf{ViCGCN (base)}   & \textbf{70.32} & \textbf{67.17}            & \textbf{71.02} & \textbf{67.48}          & \textbf{85.64} & \textbf{85.12}                & \textbf{92.22} & \textbf{75.32}        & \textbf{94.12} & \textbf{83.67}               & \textbf{90.12} & \textbf{80.11}           \\
\textbf{ViCGCN (large)}  & \textbf{71.33} & \textbf{67.82}            & \textbf{72.08} & \textbf{68.12}          & \textbf{86.12} & \textbf{85.88}                & \textbf{93.11} & \textbf{76.12}        & \textbf{94.83} & \textbf{84.23}               & \textbf{91.02} & \textbf{81.88}           \\ 
\hline
Improvement (1)          & $\uparrow$6.21 & $\uparrow$4.59            & $\uparrow$0.95 & $\uparrow$3.00          & $\uparrow$3.71 & $\uparrow$4.61                & $\uparrow$2.99 & $\uparrow$2.80        & $\uparrow$1.59 & $\uparrow$1.27               & $\uparrow$1.35 & $\uparrow$2.63           \\
Improvement (2)          & $\uparrow$1.91 & $\uparrow$2.38            & $\uparrow$1.13 & $\uparrow$0.92          & $\uparrow$1.50 & $\uparrow$1.26                & $\uparrow$1.23 & $\uparrow$1.27        & $\uparrow$1.27 & $\uparrow$1.11               & $\uparrow$1.07 & $\uparrow$1.86           \\
\hline
\end{tabular}}
\end{table}

To demonstrate the classification performance of our model ViCGCN, we compare it with other state-of-the-art and Integrated models as mentioned in Section \ref{Experiments/Baseline}. The F1-score results for both baseline and proposed models on the test sets of three Vietnamese social media text datasets are shown in Table \ref{tab::Experiments/Result} and we obtain the following observations.

Among BERTology models, RoBERTa and mBERT, including \textit{cased} and \textit{uncased}, have the most unfavorable performance of almost tasks of the three benchmark datasets. Moreover, the results show that monolingual models such as PhoBERT and viBERT perform better than other BERTology models. Additionally, through the execution of parallel computations for words, the problem of vanishing gradients is minimized, and PhoBERT archives the highest results in nearly all the tasks. However, in general, BERTology baseline models still find it hard to handle the complexity of social media: imbalanced and noisy data, which leads to poor performance compared to the integrated GCN model.
    
Our baseline integrated models can also benefit from graph structure by combining GCN as the final prediction module. Compared to BERTology baseline models, the performance boost from contextualized pre-trained language models with the GCN module is significant. Moreover, the multilingual and monolingual models integrated with GCN perform massively better than others. This explains the significance of incorporating both the Contextualized and GCN models into the integrated models can be attributed to their complementary nature in addressing the limitations of each other.

Compared to baseline models, our approach ViCGCN adopts large-scale, monolingual Vietnamese language model PhoBERT. Our integrated model ViCGCN obtains the ability to compute the new features of a node as the weighted average of itself and its second-order neighbors. In the context of imbalanced and noisy datasets, such as UIT-ViCTSD, the proposed ViCGCN model has demonstrated significant performance improvements compared to other baseline models, making it a promising approach for social media mining tasks. Moreover, Our proposed model demonstrated superior performance to the current state-of-the-art Vietnamese model, achieving improvements of 6.21\%, 4.61\%, and 2.63\% on three benchmark datasets. These results demonstrate the efficacy and validity of ViCGCN for Vietnamese text classification. As a result, our method achieves the best performance among all the tasks on three benchmark datasets in terms of UIT-VSMEC, UIT-ViCTSD, and UIT-VSFC, respectively.

\subsection{Analysis and Discussion}

\subsubsection{Impact of graph convolutional networks}
\label{imapactGCN}

Although we can implicitly infer the effectiveness of graph convolutional networks from Table \ref{tab::Experiments/Result}, we would like to discuss more the contribution of graph convolutional networks in contextualized language models. Table \ref{Result/Graph/VSMEC/table}, Table \ref{/Result/Graph/ViCTSD/table} and Table \ref{/Result/Graph/VSFC/table} display the comparisons between with and without GCN on three benchmark datasets as we can find that contextualized language model integrated with GCN outperformed all of the corresponding single models, respectively. As mentioned in Section \ref{Experiments/Result}, Contextualized Language Models have not performed well on three benchmark datasets. Integrating GCN with the BERTology model massively enhances the performance, which leads to improvements of up to 8.00\%, 7.99\%, 5.84\%, and 7.99\% of RoBERTa, viBERT, vELECTRA, and $\text{PhoBERT}_{base}$, respectively, on three benchmark datasets, UIT-VSMEC, UIT-ViCTSD, and UIT-VSFC, respectively. The average length of three datasets in UIT-VSMEC, UIT-ViCTSD, and UIT-VSFC is approximately 14. Additionally, the short sequence lengths can construct more dense graphs that provide richer contextual information, which may explain better performance by combining contextualized language models with GCN. This further demonstrates that Graph Convolutional Networks are essential in improving text classification performance.


% \begin{table}[!hp] 
% \centering
% \caption{Model performance on UIT-VSMEC.}
% \label{Result/Graph/VSMEC/table}
% %\centerline{
% \resizebox{\linewidth}{!}{
% \begin{tabular}{l|cc|cc} 
% \hline
% \textbf{Tasks}            & \multicolumn{2}{c|}{\textbf{Seven labels}}                                           & \multicolumn{2}{c}{\textbf{Six labels}}                                              \\ 
% \hline
%                           & \textbf{wF1}                             & \textbf{mF1}                              & \textbf{wF1}                             & \textbf{mF1}                              \\ 
% \hline
% BERT (cased)              & 55.06                                    & 55.88                                     & 66.34                                    & 65.31                                     \\
% Bert-GCN (BERT cased)     & 74.22 ($\uparrow$19.16)                  & 73.51 ($\uparrow$17.63)                   & 78.25 ($\uparrow$11.91)                  & 77.02 ($\uparrow$11.71)                   \\ 
% \hline
% BERT (uncased)            & 55.98                                    & 56.40                                     & 58.55                                    & 56.65                                     \\
% Bert-GCN (BERT uncased)   & 74.18 ($\uparrow$18.20)                  & 73.29 ($\uparrow$16.89)                   & 80.54 ($\uparrow$21.99)                  & 78.31 ($\uparrow$21.66)                   \\ 
% \hline
% mBERT (mBERT cased)       & 61.23                                    & 59.57                                     & 66.72                                    & 66.72                                     \\
% mBERT-GCN (mBERT cased)   & 75.12 ($\uparrow$13.89)                  & 74.83 ($\uparrow$15.26)                   & 79.55 ($\uparrow$12.83)                  & 77.84 ($\uparrow$11.12)                   \\ 
% \hline
% mBERT (uncased)           & 58.31                                    & 57.03                                     & 67.13                                    & 67.70                                     \\
% mBERT-GCN (mBERT uncased) & 74.56 ($\uparrow$16.25)                  & 73.98 ($\uparrow$16.95)                   & 80.21 ($\uparrow$13.08)                  & 78.12 ($\uparrow$10.42)                   \\ 
% \hline
% RoBERTa                   & 56.51                                    & 56.22                                     & 62.11                                    & 64.64                                     \\
% RoBERTa-GCN               & 74.82 ($\uparrow$18.31)                  & 74.22 ($\uparrow$18.00)                   & 79.33 ($\uparrow$17.22)                  & 78.32 ($\uparrow$13.68)                   \\ 
% \hline
% viBERT                    & 67.55                                    & 65.32                                     & 69.95                                    & 69.08                                     \\
% viBERT-GCN                & 78.25 ($\uparrow$10.70)                  & 78.37 ($\uparrow$13.05)                   & 82.33 ($\uparrow$12.38)                  & 81.98 ($\uparrow$12.90)                   \\ 
% \hline
% PhoBERT                   & 71.86                                    & 69.58                                     & 75.19                                    & 74.32                                     \\
% \textbf{ViCGCN (Ours)}    & \textbf{80.24 ($\uparrow$\textbf{8.38)}} & \textbf{80.96 ($\uparrow$\textbf{11.38)}} & \textbf{84.91 ($\uparrow$\textbf{9.72)}} & \textbf{83.27 ($\uparrow$\textbf{8.95)}}  \\
% \hline
% \end{tabular}}
% \end{table}

% \begin{table}[!hpbt] 
% \centering
% \caption{Model performance on UIT-ViCTSD.}
% \label{/Result/Graph/ViCTSD/table}
% \resizebox{\linewidth}{!}{%
% \begin{tabular}{l|cc|cc} 
% \hline
% \textbf{Tasks}            & \multicolumn{2}{c|}{\textbf{Constructiveness}}                    & \multicolumn{2}{c}{\textbf{Toxicity}}                               \\ 
% \hline
%                           & \textbf{wF1}                    & \textbf{mF1}                    & \textbf{wF1}                    & \textbf{mF1}                      \\ 
% \hline
% BERT (cased)              & 79.34                           & 77.29                           & 87.45                           & 64.49                             \\
% Bert-GCN (BERT cased)     & 81.13 ($\uparrow$1.79)          & 79.72 ($\uparrow$2.43)          & 88.10 ($\uparrow$0.65)          & 64.52 ($\uparrow$0.03)            \\ 
% \hline
% BERT (uncased)            & 80.42                           & 78.90                           & 85.21                           & 63.28                             \\
% Bert-GCN (BERT uncased)   & 81.05 ($\uparrow$0.63)          & 79.6 ($\uparrow$0.70)           & 88.67 ($\uparrow$3.46)          & 64.96 ($\uparrow$1.68)            \\ 
% \hline
% mBERT (mBERT cased)       & 78.15                           & 76.93                           & 86.71                           & 64.36                             \\
% mBERT-GCN (mBERT cased)   & 82.15 ($\uparrow$4.00)          & 80.33 ($\uparrow$3.40)          & 89.13 ($\uparrow$0.46)          & 65.13 ($\uparrow$0.77)            \\ 
% \hline
% mBERT (uncased)           & 78.11                           & 76.63                           & 87.65                           & 64.96                             \\
% mBERT-GCN (mBERT uncased) & 82.88 ($\uparrow$4.77)          & 81.12 ($\uparrow$4.49)          & 89.93 ($\uparrow$2.28)          & 65.89 ($\uparrow$0.93)            \\ 
% \hline
% RoBERTa                   & 78.92                           & 77.71                           & 83.82                           & 61.73                             \\
% RoBERTa-GCN               & 83.47 ($\uparrow$4.55)          & 82.77 ($\uparrow$5.06)          & 86.55 ($\uparrow$2.73)          & 64.33 ($\uparrow$2.60)            \\ 
% \hline
% viBERT                    & 82.27                           & 80.12                           & 88.13                           & 64.35                             \\
% viBERT-GCN                & 86.12 ($\uparrow$3.85)          & 85.02 ($\uparrow$4.90)          & 91.27 ($\uparrow$3.14)          & 75.93 ($\uparrow$11.58)           \\ 
% \hline
% PhoBERT                   & 81.03                           & 79.53                           & 88.83                           & 65.61                             \\
% \textbf{ViCGCN (Ours)}    & \textbf{86.97 ($\uparrow$5.94)} & \textbf{85.81 ($\uparrow$6.28)} & \textbf{91.95 ($\uparrow$3.12)} & \textbf{76.29 ($\uparrow$10.68)}  \\
% \hline
% \end{tabular}}
% \end{table}

% \begin{table}[!hpbt] 
% \centering
% \caption{Model performance on UIT-VSFC.}
% \label{/Result/Graph/VSFC/table}
% \resizebox{\linewidth}{!}{%
% \begin{tabular}{l|cc|cc} 
% \hline
% \textbf{Tasks}            & \multicolumn{2}{c|}{\textbf{Sentiment-based}}                      & \multicolumn{2}{c}{\textbf{Topic-based}}                                  \\ 
% \hline
%                           & \textbf{wF1}                    & \textbf{mF1}                     & \textbf{wF1}                    & \textbf{mF1}                            \\ 
% \hline
% BERT (cased)              & 86.32                           & 72.38                            & 86.17                           & 73.62                                   \\
% Bert-GCN (BERT cased)     & 88.72 ($\uparrow$2.40)          & 76.23 ($\uparrow$3.85)           & 88.15 ($\uparrow$1.98)          & 76.25 ($\uparrow$2.63)                  \\ 
% \hline
% BERT (uncased)            & 85.51                           & 71.44                            & 85.98                           & 72.56                                   \\
% Bert-GCN (BERT uncased)   & 88.51 ($\uparrow$3.00)          & 75.99 ($\uparrow$4.55)           & 87.75 ($\uparrow$1.77)          & 76.06 ($\uparrow$3.50)                  \\ 
% \hline
% mBERT (mBERT cased)       & 91.21                           & 78.94                            & 88.06                           & 77.63                                   \\
% mBERT-GCN (mBERT cased)   & 91.89 ($\uparrow$0.68)          & 79.84 ($\uparrow$0.90)           & 89.73 ($\uparrow$1.98)          & 79.02 ($\uparrow$1.39)                  \\ 
% \hline
% mBERT (uncased)           & 89.52                           & 76.60                            & 87.12                           & 76.96                                   \\
% mBERT-GCN (mBERT uncased) & 91.72 ($\uparrow$2.20)          & 79.64 ($\uparrow$3.04)           & 88.89 ($\uparrow$1.77)          & 78.82 ($\uparrow$1.86)                  \\ 
% \hline
% RoBERTa                   & 90.11                           & 76.98                            & 87.03                           & 76.77                                   \\
% RoBERTa-GCN               & 91.12 ($\uparrow$1.01)          & 79.32 ($\uparrow$2.34)           & 90.12 ($\uparrow$3.09)          & 79.34 ($\uparrow$2.57)                  \\ 
% \hline
% viBERT                    & 89.77                           & 76.25                            & 87.43                           & 76.55                                   \\
% viBERT-GCN                & 93.27 ($\uparrow$3.50)          & 87.52 ($\uparrow$11.27)          & 92.11 ($\uparrow$4.68)          & 88.35 ($\uparrow$11.80)  \\ 
% \hline
% PhoBERT                   & 90.51                           & 76.47                            & 87.84                           & 76.98                                   \\
% \textbf{ViCGCN (Ours)}    & \textbf{94.81 ($\uparrow$4.30)} & \textbf{88.80 ($\uparrow$12.33)} & \textbf{93.81 ($\uparrow$5.97)} & \textbf{89.61 ($\uparrow$12.63)}        \\
% \hline
% \end{tabular}}
% \end{table}

\begin{table}[!ht]
\centering
\caption{Model performance on UIT-VSMEC.}
\label{Result/Graph/VSMEC/table}
\resizebox{\linewidth}{!}{%
\begin{tabular}{l|cc|cc} 
\hline
\textbf{Tasks}          & \multicolumn{2}{c|}{\textbf{Seven labels}}                        & \multicolumn{2}{c}{\textbf{Six labels}}                            \\ 
\hline
                        & \textbf{wF1}                    & \textbf{mF1}                    & \textbf{wF1}                    & \textbf{mF1}                     \\ 
\hline
mBERT (cased)           & 60.47                           & 59.48                           & 65.02                           & 62.65                            \\
mBERT-GCN (cased)       & 68.32 ($\uparrow$7.85)          & 64.32 ($\uparrow$4.84)          & 69.32 ($\uparrow$4.30)          & 66.18 ($\uparrow$3.53)           \\ 
\hline
mBERT (uncased)         & 60.17                           & 59.18                           & 64.93                           & 62.11                            \\
mBERT-GCN (uncased)     & 67.98 ($\uparrow$7.81)          & 64.11 ($\uparrow$4.93)          & 69.12 ($\uparrow$4.90)          & 65.89 ($\uparrow$3.78)           \\ 
\hline
RoBERTa                 & 58.17                           & 57.32                           & 63.32                           & 59.97                            \\
RoBERTa-GCN             & 66.17 ($\uparrow$8.00)          & 62.12 ($\uparrow$4.80)          & 67.12 ($\uparrow$3.80)          & 64.17 ($\uparrow$4.20)           \\ 
\hline
viBERT                  & 61.33                           & 60.28                           & 68.48                           & 62.09                            \\
viBERT-GCN              & 69.32 ($\uparrow$7.99)          & 78.37 ($\uparrow$4.84)          & 82.33 ($\uparrow$2.35)          & 81.98 ($\uparrow$4.59)           \\ 
\hline
vELECTRA                & 63.58                           & 61.38                           & 68.33                           & 63.12                            \\
vELETRA-GCN             & 69.42 ($\uparrow$5.84)          & 65.44 ($\uparrow$4.06)          & 70.95 ($\uparrow$2.62)          & 67.20 ($\uparrow$4.08)           \\ 
\hline
PhoBERT (base)          & 64.36                           & 61.41                           & 69.02                           & 64.12                            \\
\textbf{ViCGCN (base)}  & \textbf{69.32 ($\uparrow$7.99)} & \textbf{65.12 ($\uparrow$4.84)} & \textbf{70.83 ($\uparrow$2.35)} & \textbf{66.68 ($\uparrow$4.59)}  \\ 
\hline
PhoBERT (large)         & 65.12                           & 71.13                           & 63.23                           & 65.12                            \\
\textbf{ViCGCN (large)} & \textbf{71.33 ($\uparrow$6.21)} & \textbf{72.08 ($\uparrow$0.95)} & \textbf{67.82 ($\uparrow$4.59)} & \textbf{68.12 ($\uparrow$3.00)}  \\
\hline
\end{tabular}}
\end{table}

\begin{table}[H]
\centering
\caption{Model performance on UIT-ViCTSD.}
\label{/Result/Graph/ViCTSD/table}
\resizebox{\linewidth}{!}{%
\begin{tabular}{l|cc|cc} 
\hline
\textbf{Tasks}            & \multicolumn{2}{c|}{\textbf{Constructiveness}}                    & \multicolumn{2}{c}{\textbf{Toxicity}}                              \\ 
\hline
                          & \textbf{wF1}                    & \textbf{mF1}                    & \textbf{wF1}                    & \textbf{mF1}                     \\ 
\hline
mBERT (mBERT cased)       & 81.03                           & 79.55                           & 88.32                           & 65.63                            \\
mBERT-GCN (mBERT cased)   & 83.12 ($\uparrow$2.09)          & 82.88 ($\uparrow$3.33)          & 90.32 ($\uparrow$2.00)          & 69.42 ($\uparrow$3.79)           \\ 
\hline
mBERT (uncased)           & 80.89                           & 79.47                           & 87.60                           & 64.77                            \\
mBERT-GCN (mBERT uncased) & 82.32 ($\uparrow$1.43)          & 82.01 ($\uparrow$2.54)          & 89.15 ($\uparrow$1.55)          & 68.32 ($\uparrow$3.55)           \\ 
\hline
RoBERTa                   & 77.41                           & 75.62                           & 85.85                           & 59.71                            \\
RoBERTa-GCN               & 81.33 ($\uparrow$3.92)          & 80.96 ($\uparrow$5.34)          & 89.02 ($\uparrow$3.17)          & 64.32 ($\uparrow$4.61)           \\ 
\hline
viBERT                    & 81.62                           & 80.07                           & 89.14                           & 71.87                            \\
viBERT-GCN                & 84.32 ($\uparrow$2.70)          & 83.12 ($\uparrow$3.05)          & 91.12 ($\uparrow$1.98)          & 74.25 ($\uparrow$2.38)           \\ 
\hline
vELECTRA                  & 82.41                           & 80.82                           & 89.33                           & 72.02                            \\
vELETRA-GCN               & 84.62 ($\uparrow$2.21)          & 84.62 ($\uparrow$3.80)          & 91.88 ($\uparrow$2.55)          & 74.85 ($\uparrow$2.83)           \\ 
\hline
PhoBERT (base)            & 81.65                           & 80.24                           & 89.58                           & 72.12                            \\
\textbf{ViCGCN (base)}    & \textbf{85.64 ($\uparrow$3.99)} & \textbf{85.12 ($\uparrow$4.88)} & \textbf{92.22 ($\uparrow$2.64)} & \textbf{75.32 ($\uparrow$3.20)}  \\ 
\hline
PhoBERT large             & 82.07                           & 90.12                           & 81.27                           & 73.32                            \\
\textbf{ViCGCN (large)}   & \textbf{86.12 ($\uparrow$4.05)} & \textbf{93.11 ($\uparrow$2.99)} & \textbf{85.88 ($\uparrow$4.61)} & \textbf{76.12 ($\uparrow$2.80)}  \\
\hline
\end{tabular}}
\end{table}

\begin{table}[!ht]
\centering
\caption{Model performance on UIT-VSFC.}
\label{/Result/Graph/VSFC/table}
\resizebox{\linewidth}{!}{%
\begin{tabular}{l|cc|cc} 
\hline
\textbf{Tasks}            & \multicolumn{2}{c|}{\textbf{Sentiment-based}}                                                                                          & \multicolumn{2}{c}{\textbf{Topic-based}}                                             \\ 
\hline
                          & \textbf{wF1}                                                                                & \textbf{mF1}                             & \textbf{wF1}                             & \textbf{mF1}                              \\ 
\hline
mBERT (cased)             & 90.39                                                                                       & 77.15                                    & 87.32                                    & 77.93                                     \\
mBERT-GCN (cased)         & 92.12 ($\uparrow$1.73)                                                                      & 79.32 ($\uparrow$2.17)                   & 88.32 ($\uparrow$1.00)                   & 79.42 ($\uparrow$1.49)                    \\
mBERT (uncased)           & 89.95                                                                                       & 77.80                                    & 87.62                                    & 77.58                                     \\
mBERT-GCN (mBERT uncased) & 91.01 ($\uparrow$1.06)                                                                      & 79.02 ($\uparrow$1.22)                   & 88.07 ($\uparrow$0.45)                   & 79.02 ($\uparrow$1.44)                    \\ 
\hline
RoBERTa                   & 87.13                                                                                       & 75.52                                    & 86.77                                    & 75.30                                     \\
RoBERTa-GCN               & 90.12 ($\uparrow$2.99)                                                                      & 78.42 ($\uparrow$2.90)                   & 87.45 ($\uparrow$0.68)                   & 78.12 ($\uparrow$2.82)                    \\ 
\hline
viBERT                    & 91.29                                                                                       & 81.95                                    & 88.22                                    & 78.35                                     \\
viBERT-GCN                & 93.12 ($\uparrow$1.83)                                                                      & 82.47 ($\uparrow$0.52)                   & 89.42 ($\uparrow$1.20)                   & 79.63 ($\uparrow$1.28)                    \\ 
\hline
vELECTRA                  & 91.89                                                                                       & 82.01                                    & 88.12                                    & 78.12                                     \\
vELETRA-GCN               & 93.56 ($\uparrow$1.67)                                                                      & 83.12 $(\uparrow$1.11)                   & 89.95 ($\uparrow$1.83)                   & 80.02 ($\uparrow$1.90)                    \\ 
\hline
PhoBERT (base)            & 92.94                                                                                       & 82.15                                    & 88.29                                    & 78.54                                     \\
\textbf{ViCGCN (base)}    & \begin{tabular}[c]{@{}c@{}}\textbf{94.12~}($\uparrow$\textbf{}\textbf{1.18)}\end{tabular} & \textbf{83.67~}($\uparrow$\textbf{1.52)} & \textbf{90.12~}($\uparrow$\textbf{1.83)} & \textbf{80.11~}($\uparrow$\textbf{1.57)}  \\ 
\hline
PhoBERT (large)           & 93.24                                                                                       & 88.72                                    & 82.96                                    & 79.12                                     \\
\textbf{ViCGCN (large)}   & \textbf{94.83~}($\uparrow$\textbf{1.59)}                                                    & \textbf{91.02~}($\uparrow$\textbf{2.3)}  & \textbf{84.23~}($\uparrow$\textbf{1.27)} & \textbf{81.88~}($\uparrow$\textbf{2.76)}  \\
\hline
\end{tabular}}
\end{table}

\subsubsection{Impact of lambda ($\lambda$)}
\label{impactlamda}

According to Equation \ref{equa::lambda}, the hyperparameter $\lambda$ controls the trade-off between two objectives, ViCGCN and PhoBERT, respectively. The optimal value of $\lambda$ may vary depending on the task. Therefore, extensive experiments on the dev set were conducted to determine the optimal value of $\lambda$. Figure \ref{fig::Experiments/Lamda/UIT-VSFC} shows the performances of ViCGCN on three benchmark datasets in terms of UIT-VSMEC, UIT-ViCTSD, and UIT-VSFC with different $\lambda$. On all three benchmark datasets, the F1-score is consistently higher with a more enormous $\lambda$ value. Moreover, taking only ViCGCN ($\lambda = 1$) as the final training objective consistently achieves a better performance than considering only PhoBERT ($\lambda = 0$). Setting $\lambda$ to a value from 0.6 to 0.8 is more desirable and can make the model reach its best when $\lambda = 0.6$ on all datasets. These observations indicate that the linear interpolation of the prediction from ViCGCN and the prediction from PhoBERT with higher ViCGCN weight can improve the Vietnamese social media text classification performance. On the other hand, the PhoBERT module is also indispensable.
% \begin{figure}[H]
%     \centering
%     \subfigure[Seven labels task]{\includegraphics[width=0.49\textwidth]{ldVSMEC_7.pdf}}
%     \subfigure[Six labels task]{\includegraphics[width=0.49\textwidth]{ldVSMEC_6.pdf}}
%     \caption{F1-score of ViCGCN when varying $\lambda$ on UIT-VSMEC dev set.}
%     \label{fig::Experiments/Lamda/VSMEC}
% \end{figure}
% \begin{figure}[H]
%     \centering
%     \subfigure[Constructiveness task]{\includegraphics[width=0.49\textwidth]{ldViCTSD_Constructiveness.pdf}}
%     \subfigure[Toxicity task]{\includegraphics[width=0.49\textwidth]{ldViCTSD_toxic.pdf}}
%     \caption{F1-score of ViCGCN when varying $\lambda$ on UIT-ViCTSD dev set.}
%     \label{fig::Experiments/Lamda/ViCTSD}
% \end{figure}
\begin{figure}[!hpt]
    \centering

    \subfigure[Seven labels task]{\includegraphics[width=0.49\textwidth]{ldVSMEC_7.pdf}}
    \subfigure[Six labels task]{\includegraphics[width=0.49\textwidth]{ldVSMEC_6.pdf}}

     \subfigure[Constructiveness task]{\includegraphics[width=0.49\textwidth]{ldViCTSD_Constructiveness.pdf}}
    \subfigure[Toxicity task]{\includegraphics[width=0.49\textwidth]{ldViCTSD_toxic.pdf}}
    
    \subfigure[Sentiment-based task]{\includegraphics[width=0.49\textwidth]{ldVSFC_sentiment.pdf}}
    \subfigure[Topic-based task]{\includegraphics[width=0.49\textwidth]{ldVSFC_topic.pdf}}
    % \caption{F1-score of ViCGCN when varying $\lambda$ on UIT-VSFC dev set.}
    
    \caption{F1-score of ViCGCN when varying $\lambda$ on the dev set.}
    
    \label{fig::Experiments/Lamda/UIT-VSFC}
\end{figure}

\subsubsection{Comparison with Previous Studies}
\label{comparisonprestudies}
% \subsubsection{UIT-VSMEC}
We conducted a number of surveys to evaluate how well our suggested technique performed in comparison to earlier studies. On the UIT-VSMEC, UIT-ViCTSD, and UIT-VSFC datasets, our method fared better than in any prior research. To provide for fair comparisons, similar evaluation metrics from earlier studies are employed. For all datasets used in this study, we use the average macro F1-score (\%) and average weighted F1-score (\%). 

Our integrated model ViCGCN outperformed the best results of each previous study on the VSMEC dataset by achieving 80.24\% weighted F1-score and 80.96\% macro F1-score on task Seven labels, which improves by 10.18\% and 13.93\% compared to the best previous study. Additionally, our model obtains 84.91\% weighted F1-score on the Six labels task as shown in Table \ref{tab::Experiments/Comparison/VSMEC}, increased by 13.92\% in comparison to the highest previous ones. Furthermore, Table \ref{tab::Experiments/Comparison/ViCTSD} deputed that ViCGCN achieves the best results, with a macro F1-score of 85.81\% for UIT-ViCTSD Constructiveness task, and 76.29\% macro F1-score for UIT-ViCSTD Toxicity task,  increased by 16.89\%. By obtaining 88.80\% macro F1-score and 94.81\% weighted F1-score, 89.61\% macro F1-score, and 93.81\% weighted F1-score on task Sentiment-based and Topic-based, respectively, our integrated model ViCGCN surpassed every previous study's top result on the UIT-VSFC dataset as describes in Table \ref{tab::Experiments/Comparison/VSFC}. In addition, our proposed approach reached new state-of-the-art performances on three Vietnamese benchmark social media datasets, UIT-VSMEC, UIT-ViCTSD, and UIT-VSFC, respectively. As a result, the proposed approach ViCGCN is significantly suitable and efficient for dealing with Vietnamese text in general and Vietnamese social media text classification tasks in particular.

\begin{table}[!hpt]
\centering
\caption{The comparison with previous studies on UIT-VSMEC.} \label{tab::Experiments/Comparison/VSMEC}
\resizebox{\linewidth}{!}{
\begin{tabular}{l|cc|cc} 
\hline
\textbf{Tasks}                                  & \multicolumn{2}{c|}{\textbf{Seven labels }} & \multicolumn{2}{c}{\textbf{Six labels }}  \\ 
\hline
                                                & \textbf{wF1}   & \textbf{mF1}               & \textbf{wF1}   & \textbf{mF1}             \\ 
\hline
CNN + Word2Vec                                  & 59.74          & -                          & 66.34          & -                        \\
MLR + TF-IDF Vectorizer + Key-clause extraction & 64.40          & -                          & -              & -                        \\
GRU + CNN + BiLSTM + LSTM                       & 65.79          & -                          & 70.99          & -                        \\
PhoBERT                                         & -              & 65.44                      & -              & -                        \\
XLM-R + VnEmolex                                & 70.06          & 67.03                      & -              & -                        \\ 
\hline
\textbf{ViCGCN (base)}                          & \textbf{70.32} & \textbf{67.17}             & \textbf{71.02} & \textbf{67.48}           \\
\textbf{ViCGCN (large)}                         & \textbf{71.33} & \textbf{67.82}             & \textbf{72.08} & \textbf{68.12}           \\
\hline
\end{tabular}}
\end{table}


% \subsubsection{ViCTSD}
\begin{table}[!ht] 
\centering
\caption{The comparison with previous studies on UIT-ViCTSD.}
\label{tab::Experiments/Comparison/ViCTSD}
\begin{tabular}{l|cc|cc} 
\hline
\textbf{Tasks}          & \multicolumn{2}{c|}{\textbf{Constructiveness }} & \multicolumn{2}{c}{\textbf{Toxicity }}  \\ 
\hline
                        & \textbf{wF1}   & \textbf{mF1}                   & \textbf{wF1}   & \textbf{mF1}           \\ 
\hline
PhoBERT                 & -              & 78.59                          & -              & 59.40                  \\
viBERT4news             & -              & 84.15                          & -              & -                      \\ 
\hline
\textbf{ViCGCN (base)}  & \textbf{85.64} & \textbf{85.12}                 & \textbf{92.22} & \textbf{75.32}         \\
\textbf{ViCGCN (large)} & \textbf{86.12} & \textbf{85.88}                 & \textbf{93.11} & \textbf{76.12}         \\
\hline
\end{tabular}
\end{table}

% \subsubsection{VSFC}
\begin{table}[!ht] 
\centering
\caption{The comparison with previous studies on UIT-VSFC.}
\label{tab::Experiments/Comparison/VSFC}
\resizebox{\linewidth}{!}{
\begin{tabular}{l|cc|cc} 
\hline
\textbf{Tasks}             & \multicolumn{2}{c|}{\textbf{Sentiment-based }} & \multicolumn{2}{c}{\textbf{Topic-based }}  \\ 
\hline
                           & \textbf{wF1}   & \textbf{mF1}                  & \textbf{wF1}   & \textbf{mF1}              \\ 
\hline
Maximum Entropy            & 87.64          & -                             & 84.03          & -                         \\
BiLSTM +Word2Vec~          & 92.03          & -                             & 89.62          & -                         \\
LD + SVM ()                & 92.20          & -                             & -              & -                         \\
BERT + CNN + BiLSTM + LSTM & 92.79          & -                             & 89.38          & -                         \\
BERT + CNN + BiLSTM        & 92.13          & -                             & 89.70          & -                         \\
XLM-R + VnEmoLex           & 93.97          & 83.40                         & -              & -                         \\ 
\hline
\textbf{ViCGCN (base)}     & \textbf{94.12} & \textbf{83.67}                & \textbf{90.12} & \textbf{80.11}            \\
\textbf{ViCGCN (large)}    & \textbf{94.83} & \textbf{84.23}                & \textbf{91.02} & \textbf{81.88}            \\
\hline
\end{tabular}}
\end{table}

% SECTION Errors and analysis %
\subsubsection{Errors Analysis}
\label{erroranalysis}

We utilize the error analysis of ViCGCN, our top-performing model, to analyze the errors observed in our proposed model. Figure\ref{fig::Experiments/CfMatrix/VSMEC}, Figure \ref{fig::Experiments/CfMatrix/ViCTSD} and Figure \ref{fig::Experiments/CfMatrix/VSFC}, respectively, show the confusion matrices for our best model's predictions on the test set for UIT-VSMEC, UIT-ViCTSD, and UIT-VSFC.

\begin{figure}[!ht]
    \centering
    \subfigure[Seven labels task]{\includegraphics[width=0.49\textwidth]{cfVSMEC_7.pdf}}
    \subfigure[Six labels task]{\includegraphics[width=0.49\textwidth]{cfVSMEC_6.pdf}}
    \caption{Error analysis of our proposed approach for UIT-VSMEC dataset.}
    \label{fig::Experiments/CfMatrix/VSMEC}
\end{figure}
\begin{figure}[!ht]
    \centering
    \subfigure[Constructiveness task]{\includegraphics[width=0.49\textwidth]{cfViCTSD_contructive.pdf}}
    \subfigure[Toxicity task]{\includegraphics[width=0.49\textwidth]{cfViCTSD_toxic.pdf}}
    \caption{Error analysis of our proposed approach for UIT-ViCTSD dataset.}
    \label{fig::Experiments/CfMatrix/ViCTSD}
\end{figure}
\begin{figure}[!ht]
    \centering
    \subfigure[Sentiment-based task]{\includegraphics[width=0.49\textwidth]{cfVSFC_sentiment.pdf}}
    \subfigure[Topic-based task]{\includegraphics[width=0.49\textwidth]{cfVSFC_topic.pdf}}
    \caption{Error analysis of our proposed approach for UIT-VSFC dataset.}
    \label{fig::Experiments/CfMatrix/VSFC}
\end{figure}

As described in Section \ref{Proposed model}, by incorporating contextualized language models such as BERT into GCN, ViCGCN can better capture the context and meaning of words and phrases, which can lead to more accurate identification of critical nodes. However, ViCGCN may not be able to explain why those nodes are essential or why specific nodes were not influential in the decision-making process. This can make it difficult for researchers to address specific issues in our proposed approach. Table \ref{fig:erroranalysissampleonViCTSD}, Table \ref{fig:erroranalysissampleonVSMEC}, and Table \ref{fig:erroranalysissampleonVSFC} contain a few illustrations of prediction errors. The results show that misclassifications were primarily due to the use of sarcasm, irony, and figurative language in social media comments. Furthermore, some misclassifications were due to the presence of multiple topics in a single comment, making it challenging to identify the primary intention. Additionally, ambiguity in identifying the labels of the datasets also leads to misclassifying of our proposed approach ViCGCN.

% \subsubsection{VSMEC dataset} \label{Errors/VSMEC}


\begin{table}[H]
\centering
\caption{Several examples of classification error on UIT-VSMEC dataset.}\label{fig:erroranalysissampleonVSMEC}
\resizebox{\linewidth}{!}{%
\begin{tblr}{
  row{1} = {c},
  cell{2}{2} = {c},
  cell{2}{3} = {c},
  cell{3}{2} = {c},
  cell{3}{3} = {c},
  hline{1-2,4} = {-}{},
}
\textbf{Comment}                                                       & \textbf{True Label} & \textbf{Predicted Label} \\
{mấy ai được như vậy\\(\textbf{English:} not many people can do that)} & other               & surprise               \\
{kinh khủng thật\\(\textbf{English:} it's terrible)}                   & fear                & sadness                
\end{tblr}}
\end{table}

% \subsubsection{ViCTSD dataset} \label{Errors/ViCTSD}

\begin{table}[!ht]
\centering
\caption{Several examples of classification error on UIT-ViCTSD dataset.}\label{fig:erroranalysissampleonViCTSD}
\resizebox{\linewidth}{!}{%
\begin{tblr}{
  row{1} = {c},
  cell{2}{2} = {c},
  cell{2}{3} = {c},
  cell{3}{2} = {c},
  cell{3}{3} = {c},
  hline{1-2,4} = {-}{},
}
\textbf{Comment}                                                                                                                                           & \textbf{True Label} & \textbf{Predicted Label} \\
{Người ăn không hết kẻ lần không ra\\(\textbf{English:} This man has much to eat but that \\
man finds no small piece.)}                                       & non\_constructive   & constructive           \\
{người trẻ còn sức khoẻ k lo làm ăn đi ăn trộm\\(\textbf{English:} Young people who are still healthy \\ don't worry about doing business but go to steal)} & non\_toxic          & toxic                  
\end{tblr}}
\end{table}


% % \subsubsection{UIT-VSFC dataset} \label{Errors/VSFC}

\begin{table}[!ht]
\centering
\caption{Several examples of classification error on UIT-VSFC dataset.}\label{fig:erroranalysissampleonVSFC}
\resizebox{\linewidth}{!}{%
% \centerline{
\begin{tblr}{
  row{1} = {c},
  cell{2}{2} = {c},
  cell{2}{3} = {c},
  cell{3}{2} = {c},
  cell{3}{3} = {c},
  hline{1-2,4} = {-}{},
}
\textbf{Comment}                                                                                                                                          & \textbf{True Label} & \textbf{Predicted Label} \\
{ví dụ phù hợp với nội dung kiến thức , hướng dẫn chi tiết\\(\textbf{English:}~Examples are consistent with content knowledge, \\ detailed instructions)} & neural          & positive           \\
{đảm bảo chất lượng tốt\\(\textbf{English:}~Good quality guarantee)}                                                                                               & others          & facility topic     
\end{tblr}}
% }
\end{table}
%%


\subsubsection{Ablation Study}
\label{ablationstudy}

\begin{table}[H]
\centering
\caption{Ablation test on our proposed approach. w/o GCN and w/o PhoBERT denoted the result of the ablation GCN and the result of the ablation PhoBERT, respectively}
\label{tab::Ablation}
\resizebox{\linewidth}{!}{%
% \centerline{
\begin{tabular}{l|cc|cc|cc|cc|cc|cc} 
\hline
\textbf{Datasets} & \multicolumn{4}{c|}{\textbf{VSMEC}}                                                                          & \multicolumn{4}{c|}{\textbf{ViCTSD}}                                                                         & \multicolumn{4}{c}{\textbf{VSFC}}                                                                \\ 
\hline
\textbf{Tasks}    & \multicolumn{2}{c|}{\textbf{Seven labels}}            & \multicolumn{2}{c|}{\textbf{Six labels}}             & \multicolumn{2}{c|}{\textbf{Constructiveness}}       & \multicolumn{2}{c|}{\textbf{Toxicity}}                & \multicolumn{2}{c|}{\textbf{Sentiment-based}}        & \multicolumn{2}{c}{\textbf{Topic-based}}  \\ 
\hline
                  & \multicolumn{1}{c|}{\textbf{wF1}} & \textbf{mF1}      & \multicolumn{1}{c|}{\textbf{wF1}} & \textbf{mF1}     & \multicolumn{1}{c|}{\textbf{wF1}} & \textbf{mF1}     & \multicolumn{1}{c|}{\textbf{wF1}} & \textbf{mF1}      & \multicolumn{1}{c|}{\textbf{wF1}} & \textbf{mF1}     & \textbf{wF1}     & \textbf{mF1}           \\ 
\hline
\multicolumn{13}{c}{\textbf{ViCGCN}}                                                                                                                                                                                                                                                                                                               \\ 
\hline
Performance       & \textbf{71.33}                    & \textbf{67.82}    & \textbf{72.08}                    & \textbf{68.12}   & \textbf{86.12}                    & \textbf{85.88}   & \textbf{93.11}                    & \textbf{76.12}    & \textbf{94.83}                    & \textbf{84.23}   & \textbf{91.02}   & \textbf{81.88}         \\ 
\hline
\multicolumn{13}{c}{\textbf{w/o GCN}}                                                                                                                                                                                                                                                                                                              \\ 
\hline
Performance       & 65.12                             & 63.23             & 71.13                             & 65.12            & 81.03                             & 79.53            & 90.12                             & 73.32             & 93.24                             & 82.96            & 88.72            & 79.12                  \\
Decrease          & $\downarrow$6.21                  & $\downarrow$4.59  & $\downarrow$0.95                  & $\downarrow$3.00 & $\downarrow$5.09                  & $\downarrow$6.35 & $\downarrow$2.99                  & $\downarrow$2.80  & $\downarrow$1.59                  & $\downarrow$1.27 & $\downarrow$2.30 & $\downarrow$2.76       \\ 
\hline
\multicolumn{13}{c}{\textbf{w/o PhoBERT}}                                                                                                                                                                                                                                                                                                          \\ 
\hline
Performance       & 52.32                             & 51.32             & 61.34                             & 58.42            & 79.63                             & 78.37            & 87.63                             & 64.32             & 88.32                             & 75.32            & 85.36            & 75.21                  \\
Decrease          & $\downarrow$19.01                 & $\downarrow$16.50 & $\downarrow$10.74                 & $\downarrow$9.70 & $\downarrow$6.49                  & $\downarrow$7.51 & $\downarrow$5.48                  & $\downarrow$11.80 & $\downarrow$6.51                  & $\downarrow$8.91 & $\downarrow$5.66 & $\downarrow$6.67       \\
\hline
\end{tabular}}
\end{table}

Our proposed method is considerably more effective than most current techniques for classifying text on social media. Ablation experiments were carried out on the proposed approach to prove the effectiveness of these two modules, PhoBERT and GCN. Table \ref{tab::Ablation} shows the ablation experiment results of the text classification module. For the model with GCN ablation, the experimental results are inferior to the model without ablation. While results of the \textit{w/o PhoBERT} model are not as good as those of the model with the contextualized pre-trained language model. The results of the ablation experiments demonstrate the effectiveness of the proposed importance of each module in general, as well as the combination of our proposed approach in particular. Our proposed approach, especially contextualized language models Integrated with graph neural networks, yield promising outcome for improving performance in further study. As a result, we conclude that all proposed modules are crucial in text classification on social media.
\section{Limitations and Open Questions}
\label{sec:limitations}
Though we have proposed two effective non-``detect-then-describe'' methods for 3D dense captioning, the captions do not have much diversity because of the limited text annotations, beam search, and self-critical sequence training with the CiDEr reward.
% 
We believe that multi-modal pre-training on 3D vision-language tasks with more training data and the utilization of \textbf{L}arge \textbf{L}anguage \textbf{M}odels(LLM) trained on large corpus would increase the diversity of the generated captions.
% 
Additionally, other reward functions designed for 3D dense captioning will increase the diversity among object descriptions in the same scene.
% 
We will leave these topics for future study.


\section{Conclusions}
\label{sec:conclusion}
%
\whatsnew{
In this work, we decouple the caption generation from caption generation, and propose a set of two transformer-based approaches, namely Vote2Cap-DETR and Vote2Cap-DETR++, for 3D dense captioning.
%
Comparing with the sophisticated and explicit relation modules in conventional ``detect-then-describe'' pipelines, our proposed methods efficiently capture the object-object and object-scene relation through the attention mechanism.
%
The preliminary model, Vote2Cap-DETR, decouples the decoding process to generate captions and box estimations in parallel.
% 
We also propose vote queries for fast convergence, and develop a novel lightweight query-driven caption head for informative caption generation.
% 
In the advanced model, Vote2Cap-DETR++, we further decouple the queries to capture task-specific features for object localization and description generation.
% 
Additionally, we introduce an iterative spatial refinement strategy for vote queries, and insert 3D spatial information for more accurate captions.
%
Extensive experiments on two widely used datasets validate that both the proposed methods surpass prior ``detect-then-describe'' pipelines by a large margin.
}
\section*{Acknowledgements}
The authors thank Barbara Rakitsch and Mona Buisson-Fenet for valuable discussions.
%\newpage
%\FloatBarrier
\bibliography{bib}

\end{document}

 
