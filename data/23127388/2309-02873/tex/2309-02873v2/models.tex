\subsection{Models} \label{section:mathematics}
In this section, we will add the necessary background on the models and data that we consider. 

\paragraph{Damped system: }
The data $x(t)$ are generated by adding a sine wave and a second sine oscillations that is exponentially damped over time. 
This yields the system
\begin{equation}
x(t)= \sin(0.1 t)+2\exp(-0.01 t)\sin(t).
\end{equation}
For the simulator, a simple sine wave with slight mismatch in the amplitude is chosen.
This yields
\begin{equation}
s(t)=0.7 \sin(0.1 t).
\end{equation}
The signals are obtained, by evaluating at $t=0, \dots 400$ with a discretization $\Delta t = 0.1$. Further the observation data are corrupted with white noise with variance $0.1$.  
\paragraph{Double torsion pendulum: }
For system ii), we consider the data from \citet{lisowski}.
We artificially add a transient component again $\Delta x$ via
\begin{equation}
\Delta x(t)=\exp(-0.5 t)\sin(50 t),
\end{equation}
evaluated at the interval $t = 0,1,2, \dots$. 
\paragraph{Van der Pol oscillator: }
For experiment iv), we consider a Van-der-Pol oscillator with additional control input. 
The differential equation for the Van-der-Pol oscillator with external force is given as
\begin{equation}
\begin{pmatrix}
\dot{x}(t) \\
\dot{y}(t) \\
\dot{u}(t) \\
\dot{v}(t)
\end{pmatrix}
=
\begin{pmatrix}
y \\
-x+a(1-x^2)y+bu \\
v \\
-\omega^2 u
\end{pmatrix},
\end{equation}
where $a,b$ and $\omega$ are system parameters. 
We choose $a=5, b=80, \omega=7.0$ and initial conditions 
\begin{equation}
\begin{pmatrix}
\dot{x}_0 \\
\dot{y}_0 \\
\dot{u}_0 \\
\dot{v}_0
\end{pmatrix}
=
\begin{pmatrix}
-2 \\
1 \\
0.31 \\
1
\end{pmatrix}.
\end{equation}
The system is evaluated at $t=5,\dots,100$ with step size $\Delta t=0.1$. 
