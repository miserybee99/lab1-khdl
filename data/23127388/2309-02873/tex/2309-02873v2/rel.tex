% !TEX root = neurips_2021.tex
\section{Related Work} \label{section:rel}
HM is an emerging trend and a certain type of grey-box modeling. 
While general grey-box models often respect structural prior knowledge as energy-preservation, invariants etc. \citep{NEURIPS2019_26cd8eca, gaussprinciple2020geist, geist2021corl}, HM combines physics-based simulations with data-driven models.
Many works consider HM for dynamical systems or time-series.
Like us, \citet{Linial2020GenerativeOM} approach their HM task with RNNs.
In particular, they infer the initial states and the parameters of an ODE via an LSTM. 
However, in contrast to our setting, the system can be fully explained by the simulator with optimized parameters.
Another common approach in HM is extending a physics-based dynamics model, e.g. additively, with neural ODEs \citep{yin2021augmenting, qian2021integrating, 9337893} or modeling unknown parts of the dynamics with NNs \citep{SU1992327}.
\citet{rackauckas2021universal} present a unified view on these types of models, allowing to jointly learn physical and NN parameters. 
Recently, variational autoencoders are used to decode the latent states of observations and simulator from data and encode predictions from latent states and the physical model \citep{takeishi2021physicsintegrated}. 
Existing approaches are extended in \citep{wehenkel2023robust} via a data augmentation concept improving the behavior on unseen data. 
However, all these models assume direct access to the latent simulator states. 
In contrast, we consider black-box simulators that provide access only to output trajectories.

However, some approaches consider black-box simulators as well. 
One branch of HM with black-box simulators deals with optimizing parameters of the simulator \citep{DBLP:conf/iclr/RuizSC19, aushev2020likelihoodfree}.
However, this is not the setting that we consider since the simulator is fully informative once the parameters are adapted. 
A typical approach in our setting is to learn the errors or residua of simulator predictions and data \citep{Forssell97combiningsemi-physical, Suhartono_2017}.
Similar to our approach, \citet{ensinger2023combining} aim to control the long-term behavior of the predictions via the simulator.
To this end, they propose a complementary filtering approach. 
However, in contrast to their approach, our method is not restricted to simulators with correct low-frequency behavior. 
Furthermore, none of these works informs the latent states of the learning-based component. 
Another possibility is to provide the simulator as control input to an RNN \citep{SCHON202219, 10.1145/3447814}.
As discussed in detail in Sec. \ref{sec:method}, this can easily lead to an underestimation of the simulator. 

KKL observers have been combined with learning in different ways.  
In \citep{9304163}, nonlinear regression via NNs is performed in order to learn the nonlinear
transformation of a KKL observer. \citet{2204.00318} build up on the approach by optimizing the choice of
the controllable pair. \citet{9683485} propose a learning-based observer design by learning the nonlinear
transformation of a KKL observer with autoencoders. In contrast to our approach, all of these works consider
observer design for a dynamical system with \emph{known} dynamics.
However, some approaches consider KKL observers in the context of dynamics model learning.
\citet{buisson-fenet2023recognition} propose a KKL-based recognition model for neural ODEs. 
The initial latent state is obtained by running a KKL observer forward or backward. 
In contrast to our setting, the remaining rollout is not observer-based.
Furthermore, they do not consider HM. 
\citet{9683277} propose to construct an output predictor via a KKL observer.
The framework can be trained similar to standard recurrent architectures.
But in contrast to those architectures, mathematical guarantees for the output predictor can be obtained.
However, they do not consider HM. 
In contrast, we leverage the properties of KKL observers in order to control the behavior of the system by informing the latent states via the simulator. 


