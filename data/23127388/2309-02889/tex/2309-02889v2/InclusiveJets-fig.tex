\begin{figure}[p]
    \pdfbookmark[section]{Figures}{fig}
    \vfill
    \begin{center}
        \includegraphics{Fig1.pdf}
    \end{center}
    \caption{Single-jet production via the QPM-like process (top row) and dijet production via the QCD-Compton process (bottom row). The left column depicts the Feynman graphs corresponding to each interaction with time running from left to right. The right column depicts the same graphs, arranged in such a way that the directions of the particle lines correspond to the direction of the particle momenta in the longitudinal and radial directions in the Breit frame of reference. The labels $e$, $e'$, $p$, $X$ and $V^*$ denote the incoming and scattered electron, the incoming proton, the proton remnant and the exchanged boson, respectively.}
    \label{fig:breit frame}
    \vfill
\end{figure}

\begin{figure}[p]
    \vfill
    \begin{center}
        \includegraphics{Fig2.pdf}
    \end{center}
    \caption{Detector-level comparison of data (dots) and the \textsc{Ariadne} (solid, green) and \textsc{Lepto} (dashed, blue) MC distributions after corrections for the $p_{\perp,\text{Breit}}$ distribution in different regions of $Q^2$. The data are shown after subtracting the background from photoproduction and low-$Q^2$ DIS events. The error bars represent the statistical uncertainties of the data. The MC models are scaled globally to match the normalisation of the data in the fiducial range as defined in Sections~\ref{sec:selection} and \ref{sec:corrections}.}
    \label{fig:control jet}
    \vfill
\end{figure}

\begin{figure}[p]
    \vfill
    \begin{center}
        \includegraphics{Fig3.pdf}
    \end{center}
    \caption{Contributions of the different sources of systematic uncertainty, added in quadrature. The unfolding uncertainty is shown separately, without being added.
    The entry `MC model' includes the uncertainty due to exchanging the MC model ($\delta_\text{model}$) and the uncertainty in the reweighting of the MC models ($\delta_\text{rew.}$).
    The entry `Electron uncertainties' represents the sum of the uncertainties associated with the electron-energy scale ($\delta_\text{EES}$), electron-energy calibration ($\delta_\text{EL}$) and electron-finding algorithm ($\delta_\text{EM}$). Uncertainties due to photoproduction ($\delta_\text{PHP}$), low-$Q^2$ DIS ($\delta_\text{Low-$Q^2$}$) and unmatched jets ($\delta_\text{fake}$) are shown as the entry `Background contribution'.
    The polarisation uncertainty ($\delta_\text{pol.}$), track-association uncertainty ($\delta_\text{TME}$) and the uncertainty of the track reconstruction ($\delta_\text{FLT}$) are combined into the entry `Other corrections'.
    }
    \label{fig:systematics}
    \vfill
\end{figure}

\begin{figure}[p]
    \vfill
    \begin{center}
        \includegraphics{Fig4.pdf}
    \end{center}
    \caption{The measured double-differential inclusive jet cross sections with $\unit[7]{\GeV} < p_{\perp,\text{Breit}} < \unit[50]{\GeV}$ and $-1 < \eta_\text{lab} < 2.5$, in the kinematic range $\unit[150]{\GeV^2} < Q^2 < \unit[15000]{\GeV^2}$ and $0.2 < y < 0.7$. Shown are the present measurement from ZEUS (full dots, black), the corresponding measurement from H1 (open dots, red) \protect\cite{h1highq2newjets} and the NNLO QCD predictions (blue boxes). The inner error bars of the measurements represent the unfolding uncertainty and the outer error bars the total uncertainty. For the ZEUS measurement, the shaded band shows the uncertainty associated with the jet-energy scale. The NNLO QCD calculation is computed at $\alpha_s(M_Z^2) = 0.1155$ using the HERAPDF2.0Jets NNLO PDF set and scales of $\mu_\text{r}^2 = \mu_\text{f}^2 = Q^2+p_{\perp,\text{Breit}}^2$. The predictions were corrected for hadronisation and for $Z$-boson exchange. Also shown is the ratio of those cross sections to the NNLO QCD predictions.
    }
    \label{fig:cross sections}
    \vfill
\end{figure}

\begin{figure}[p]
    \vfill
    \begin{center}
        \includegraphics{Fig5.pdf}
    \end{center}
    \caption{Correlation matrix of the unfolding uncertainty for the inclusive-jet cross-section measurement. By definition, the matrix is symmetric and all entries on the diagonal are $100\%$. Negative correlations due to the finite detector resolution arise mostly in adjacent bins at small $Q^2$ and small $p_{\perp,\text{Breit}}$. Adjacent bins that do not belong to this region and non-adjacent bins are not strongly correlated.
    }
    \label{fig:correlation1}
    \vfill
\end{figure}

\begin{figure}[p]
    \vfill
    \begin{center}
        \includegraphics{Fig6.pdf}
    \end{center}
    \caption{Correlation matrix between the unfolding uncertainty of the inclusive-jet measurement and the statistical uncertainty of the previous dijet measurement\protect\cite{zeusdijets}. Correlations are mostly positive, as they arise predominantly from jets originating from the same events. A structure of more strongly correlated bins is visible, which can be explained by the differing bin boundaries of the two measurements.
    }
    \label{fig:correlation2}
    \vfill
\end{figure}


\begin{figure}[p]
    \vfill
    \begin{center}
        \includegraphics{Fig7.pdf}
    \end{center}
    \caption{Summary of different determinations of $\alpha_s(M_Z^2)$ at NNLO or higher order, adapted from PDG\protect\cite{10.1093/ptep/ptac097}, see references therein. The red points are included in the PDG world average. The averages from each sub-field are shown as yellow bands and the world average as a blue band. A recent measurement from CMS\protect\cite{CMSjets} using jet cross sections and the latest determination from HERAPDF\protect\cite{HERAPDF20NNLO}, which are not yet included in the world average, are shown in green. The current determination, assuming half-correlated and half-uncorrelated scale uncertainties, is shown in black.}
    \label{fig:strong coupling}
    \vfill
\end{figure}

\begin{figure}[p]
    \vfill
    \begin{center}
        \includegraphics{Fig8.pdf}
    \end{center}
    \caption{Difference between $\chi^2$ and $\chi^2_\text{min}$ as a function of $\alpha_s(M_Z^2)$ for fits with fixed $\alpha_s(M_Z^2)$ at NNLO. The central value and the experimental/fit, model/parameterisation and scale uncertainties determined for the free $\alpha_s(M_Z^2)$-fit assuming fully correlated scale uncertainties are also shown, added in quadrature. For reference, the corresponding plot from the HERAPDF2.0Jets NNLO analysis is also shown\protect\cite{HERAPDF20NNLO}.}
    \label{fig:scan}
    \vfill
\end{figure}

\begin{figure}[p]
    \vfill
    \begin{center}
        \includegraphics{Fig9.pdf}
    \end{center}
    \caption{Double-differential inclusive jet cross-section predictions based on the NNLO fit (solid, green) compared to the data (dots). Additionally, the predictions are shown before including the current inclusive-jet dataset in the fit (dashed, blue).
    The uncertainties of the fit results are not shown. When including the current dataset, the experimental/fit uncertainty decreases slightly.
    The ratios of the cross sections as calculated before and after the fit to the data are also shown. Other details as given in Fig.~\ref{fig:cross sections}.}
    \label{fig:fit nnlo}
    \vfill
\end{figure}

\begin{figure}[p]
    \vfill
    \begin{center}
        \includegraphics{Fig10.pdf}
    \end{center}
    \caption{Value of the strong coupling $\alpha_s(\mu^2)$ as a function of the scale $\mu$. The data points indicate determinations from measurements that were performed close to the indicated scale. The uncertainties represent the full uncertainty of each determination. All depicted results were obtained at least at NNLO. They are based on data from $e^+e^-$\protect\cite{Abbiendi_2011,Schieck_2013,Dissertori_2008}, $ep$\protect\cite{Britzger_2019,H1alphas} and $pp$\protect\cite{CMSalphas} collisions, as well as from $\tau$ lepton decays\protect\cite{Pich_2014} and quarkonium states\protect\cite{Narison_2018}. The solid blue line shows the PDG world average\protect\cite{10.1093/ptep/ptac097}. Also shown are the $\alpha_s(M_Z^2)$ values corresponding to each data point.
    }
    \label{fig:running}
    \vfill
\end{figure}
