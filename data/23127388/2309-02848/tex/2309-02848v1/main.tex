\pdfoutput=1 
\documentclass{article}
\PassOptionsToPackage{numbers, compress}{natbib}


% if you need to pass options to natbib, use, e.g.:
%     \PassOptionsToPackage{numbers, compress}{natbib}
% before loading neurips_2023


% ready for submission
\usepackage[preprint]{neurips_2023}


% to compile a preprint version, e.g., for submission to arXiv, add add the
% [preprint] option:
%     \usepackage[preprint]{neurips_2023}


% to compile a camera-ready version, add the [final] option, e.g.:
%     \usepackage[final]{neurips_2023}


% to avoid loading the natbib package, add option nonatbib:
%    \usepackage[nonatbib]{neurips_2023}


\usepackage[utf8]{inputenc} % allow utf-8 input
\usepackage[T1]{fontenc}    % use 8-bit T1 fonts
\usepackage{hyperref}       % hyperlinks
\usepackage{url}            % simple URL typesetting
\usepackage{booktabs}       % professional-quality tables
\usepackage{amsfonts}       % blackboard math symbols
\usepackage{nicefrac}       % compact symbols for 1/2, etc.
\usepackage{microtype}      % microtypography
\usepackage{xcolor}         % colors
\usepackage{makecell}
\usepackage{footnote}
\usepackage{multirow}
\usepackage{tablefootnote}
\usepackage{enumitem} 

\usepackage{soul} 
\usepackage[misc]{ifsym}
\newcommand{\figref}[1]{Fig.~\ref{#1}}
\newcommand{\eqnref}[1]{Eq.~(\ref{#1})}
\newcommand{\defref}[1]{Definition.~\ref{#1}}
\newcommand{\secref}[1]{Sec.~\ref{#1}}
\newcommand{\tableref}[1]{Table~\ref{#1}} 
\newcommand{\algref}[1]{Algorithm~\ref{#1}}
\usepackage{graphicx}  %Required

\newcommand{\QZ}[1]{\textcolor{blue}{#1}}
\newcommand{\hxw}[1]{\textbf{\color{red}[** #1 ** --hxw]}}
\newcommand{\zzs}[1]{\textbf{\color{blue}[** #1 ** --zzs]}}
\newcommand{\hkq}[1]{\textbf{\color{green}[** #1 ** --hkq]}}

\newcommand{\rebuttal}[1]{\textbf{\color{red}[** #1 ** --rebuttal]}}

\makeatletter
\newif\if@restonecol
\makeatother
\let\algorithm\relax
\let\endalgorithm\relax
 
%引入伪代码模块需要的包,第三代
\usepackage[linesnumbered,ruled,vlined]{algorithm2e}%[ruled,vlined]{
 
%\usepackage[ruled]{algorithm2e} %带竖线
%\usepackage[ruled,vlined]{algorithm2e} %带竖线和折线
%\usepackage[linesnumbered,boxed]{algorithm2e} %方框格式
%\usepackage[lined,algonl,boxed]{algorithm2e} %可以显示EndIf等


\title{Prompt-based Node Feature Extractor for Few-shot Learning on Text-Attributed Graphs}


% The \author macro works with any number of authors. There are two commands
% used to separate the names and addresses of multiple authors: \And and \AND.
%
% Using \And between authors leaves it to LaTeX to determine where to break the
% lines. Using \AND forces a line break at that point. So, if LaTeX puts 3 of 4
% authors names on the first line, and the last on the second line, try using
% \AND instead of \And before the third author name.


\author{%
Xuanwen Huang$^{\dagger}$, Kaiqiao Han$^{\dagger}$, Dezheng Bao$^{\dagger}$, Quanjin Tao, Zhisheng Zhang$^{\dagger}$,\\
\textbf{Yang Yang$^{\dagger}$, Qi Zhu$^{\S}$}\\
$^{\dagger}$Zhejiang University\\$^{\S}$ University of Illinois Urbana-Champaign\\  
\texttt{\{xwhuang, kaiqiaohan, baodezheng, taoquanjin,\}@zju.edu.cn} \\
\texttt{\{zhangzhsh6, yangya\}@zju.edu.cn} \\
\texttt{qiz3@illinois.edu}\\
}

\usepackage{caption}
\begin{document}


\maketitle



% Step 0: filling-mask, update GNN adator, replace linear layer
% Step 1: task-specific prompt + Graph Adaptor(+PLM freeze) + GNN


% baseline, prompt as input on for some baselines
% baseline, compared your interpretable representations vs. CLS emebdding or mean pooling embeddings and [MASK] embedding

\begin{abstract}
Text-attributed Graphs (TAGs) are commonly found in the real world, such as social networks and citation networks, and consist of nodes represented by textual descriptions. 
Currently, mainstream machine learning methods on TAGs involve a two-stage modeling approach: (1) unsupervised node feature extraction with pre-trained language models (PLMs); and (2) supervised learning using Graph Neural Networks (GNNs). 
However, we observe that these representations, which have undergone large-scale pre-training, do not significantly improve performance with a limited amount of training samples. 
The main issue is that existing methods have not effectively integrated information from the graph and downstream tasks simultaneously. 
%to generate interpretable representations directly related to the graph and downstream tasks.
In this paper, we propose a novel framework called G-Prompt, which combines a graph adapter and task-specific prompts to extract node features. 
%The graph adapter operates on the last linear transformation layer of PLM, which predicts the ID of the masked token in the filling-mask task. 
First, G-Prompt introduces a learnable GNN layer (\emph{i.e.,} adaptor) at the end of PLMs, which is fine-tuned to better capture the masked tokens considering graph neighborhood information.
%can assist the language model in perceiving neighboring node information and better predicting masked tokens. The graph adapter is trained to utilize the fill-mask task native to the PLMs. 
After the adapter is trained, G-Prompt incorporates task-specific prompts to obtain \emph{interpretable} node representations for the downstream task.
%based on the language model's fill-mask framework, combined with 
%the graph adapter to generate interpretable task-related representations that perceive graph information. 
Our experiment results demonstrate that our proposed method outperforms current state-of-the-art (SOTA) methods on few-shot node classification. More importantly, in zero-shot settings, the G-Prompt embeddings can not only provide better task interpretability than vanilla PLMs
but also achieve comparable performance with fully-supervised baselines.


\end{abstract}

\section{Introduction}

Text-Attributed Graphs (TAGs) are a type of graph that have textual data as node attributes. 
These types of graphs are prevalent in the real world, such as in citation networks \cite{hu2020open} where the node attribute is the paper's abstract. TAGs have diverse potential applications, including paper classification \cite{chien2021node} and user profiling\cite{kim2020multimodal}. 
However, studying TAGs presents a significant challenge: how to model the intricate interplay between graph structures and textual features. 
This issue has been extensively explored in several fields, including natural language processing, information extraction, and graph representation learning. 

% Text-Attributed Graphs (TAGs) are a type of graph that is widely present in the real world. 
% In practical applications, many node features can be composed of text. For example, in citation networks, the node feature is the abstract of a paper, and in social networks, the node feature is the user's profile. 
% TAGs have broad potential application values, such as paper classification and user identification. 
% Modeling TAGs involves techniques from multiple fields, including information extraction, natural language processing, and graph representation learning, making it a hot academic topic currently.

An idealized approach involves combining pre-trained language models (PLMs) \cite{he2020deberta,liu2019roberta} with graph neural networks and jointly training them \cite{zhao2022learning,mavromatis2023train}. Nevertheless, this method requires fine-tuning the PLMs, which demands substantial computational resources. Additionally, trained models are hard to be reused in other tasks because finetuning PLM may bring catastrophic forgetting\cite{chen2020recall}. 

Therefore, a more commonly used and efficient approach is the two-stage process \cite{yang2021bert,zhang2022stance,malhotra2020classification}: (1) utilizing pre-trained language models (PLMs) for unsupervised modeling of the nodes' textual features. 
(2) supervised learning using Graph Neural Networks (GNNs). 
Compared to joint training of PLMs and GNNs, this approach offers several advantages in practical applications. 
For example, it can be combined with numerous GNN frameworks or PLMs, and this approach does not require fine-tuning PLMs for every downstream task.
However, PLMs are unable to fully leverage the wealth of information contained in the graph structure, which represents significant information. 
To overcome these limitations, some works propose self-supervised fine-tuning PLMs using graph information to extract graph-aware node features \cite{chien2021node}. Such methods have achieved significant success across various benchmark datasets\cite{hu2020open}. 
% Unsupervised modeling of nodes' textual features by language models (LM) and subsequent supervised learning of the graph feature by Graph Neural Networks (GNNs) is a classical and effective approach for processing TAGs. 
% However, the generated node representation is untrainable in downstream tasks, a unsuitable representation may affect the performance of subsequent GNNs learning. 
% To address limitations, many works merged recently, which investigate how to better utilize pre-trained language models in TAGs modeling. 
% A method is joint PLMs with GNNs by knowledge distillation. 
% and self-supervised fine-tuning PLMs to adapt graph data.   
% First, PLMs are fine-tuned by self-supervised tasks related to graphs, enabling them to capture and comprehend graph information. Then, the fine-tuned PLM is used to generate node representations.
% This approach has achieved significant results in numerous public datasets.


% However, these SSL-based node feature extraction methods suffer from the few-shot challenge. are based on graphs with over 100,000 nodes. 
% This means that during the self-supervised training phase, there are enough samples, and downstream task training samples are also abundant. 
% For example, in Ogbn-arxiv, there are over 70,000 training samples (60\%). 
% However, this situation poses a significant gap from the real world. 
% Firstly, training labels are often expensive, and secondly, there exist many small graphs in the real world.  

However, both self-supervised methods and using language models directly to process TAG suffer from a fundamental drawback. They cannot incorporate downstream task information, which results in identical representations being generated for all downstream tasks. This is evidently counterintuitive as the required information may vary for different tasks. For example, height is useful information in predicting a user's weight but fails to accurately predict age. This issue can be resolved by utilizing task-specific prompts combined with language models \cite{petroni2019language} to extract downstream task-related representations. For example, suppose we have a paper's abstract $\{\mathbf{Abstract}\}$ in a citation network, and the task is to classify the subject of the paper. We can add some prompts to a node's sentence:
$
    \{This, is, a, paper, of, [\mathbf{mask}], subject, its, abstract, is,:, \mathbf{Abstract}\}
$. And then use the embedding corresponding to the [mask] token generated by PLMs as the node feature. Yet this approach fails to effectively integrate graph information. 

To better integrate task-specific information and knowledge within graph structure, this paper proposes a novel framework called G-Prompt. G-Prompt combines a graph adapter and task-specific prompts to extract node features. Specifically, G-Prompt contains a graph adapter that helps PLMs become aware of graph structures. This graph adapter is self-supervised and trained by fill-mask tasks on specific TAGs. G-Prompt then incorporates task-specific prompts to obtain interpretable node representations for downstream tasks.



% However, we observe the SSL-based methods are in the small-sample scenario and found that: \\
% 1. The representations generated by large-scale language models perform similarly to word2vec in small-sample situations. This is clearly counterintuitive, as numerous experiments have shown that pre-trained language models can learn rich knowledge from massive text. \\
% 2. The representation of entire BERT models finetuned on graph self-supervised tasks such as GIANT performs similarly to the frozen language model's representation through GAE pre-training in extremely small sample sizes. However, overall, it outperforms graph-free representations. \\
% 3. Since using PLM-generated representations did not yield good results, we experimented with RoBERTa-based representations with task prompts, which performed the best in small-sample scenarios.

% This implies that both Graph-aware and Task-aware representations are crucial for node representation. 
% However, current methods \textbf{can not effectively combine} the two because current unsupervised node feature generation methods do not consider downstream tasks. 
% Meanwhile, pre-trained models cannot be task-specifically transformed. 
% There is a significant gap between self-supervised tasks and BERT's own pre-training tasks. 
% Directly finetuning BERT would destroy the prior knowledge learned from massive text data.

% Furthermore, current methods generate node features that \textbf{lack interpretability}. 
% The features generated by current methods are continuous and lack interpretability. 
% It is challenging to explain why a particular representation works, and it is difficult to manually select a few features for downstream tasks.
% Meanwhile, the current state-of-the-art methods require finetuning of pre-trained language models (PLMs). However, with the increasing size of PLMs, the computational cost of finetuning has become prohibitively high, often requiring a substantial amount of data to achieve good performance. Thus, it is challenging to integrate these methods with even more powerful language models.

% Therefore, this paper aims to explore the possibility of generating task-aware and graph-aware representations with BERT without finetuning. For the former, a naive method is to use prompts, which are manually input task-related hints, along with text features to generate corresponding words using a language model. For example, for citation networks, we can add prompt information before the abstract: "This is a paper published on <mask> subject, its abstract is [content]." We then use the word distribution after decoding the <mask> as a node feature. However, incorporating graph information into the prompt is challenging. To address this issue, we propose a new framework called GPrompt. This framework combines graph adapters and prompts to extract node features. The graph adapter operates on the last linear transformation layer that predicts words in the LM, i.e., a learnable graph neural network is added to that layer. The goal of the GNN is to help the language model perceive neighbor information of nodes and better predict the masked word. The graph adapter is trained through the language model's native fill-mask task. After the adapter is trained, GPrompt incorporates task-related prompts based on the fill-mask framework of the language model, combined with the graph adapter, to generate task-related representations that are interpretable and perceive graph information.


% pithc on parameter-efficient tuning, cite lora/adaptor
% However, replacing the linear transformation with GNN imposes huge computational costs, and it is not feasible to aggregate neighbors once for each token of every word. To speed up the training process, we adopt DecoupleGNN and use geometric mean to aggregate information from each neighbor. The geometric mean is equivalent to training neighbor nodes with the target node's label in the cross-entropy loss function, so there is no need to globally aggregate neighbor information during GraphAdapter training. This strategy accelerates training effectively through global edge sampling.

We conduct extensive experiments on three real-world datasets in the domains of few-shot and zero-shot learning, in order to demonstrate the effectiveness of our proposed method. The results of our experiments show that G-Prompt achieves state-of-the-art performance in few-shot learning, with an average improvement of \textit{avg.} 4.1\% compared to the best baseline. Besides, our G-Prompt embeddings are also highly robust in zero-shot settings, outperforming PLMs by \textit{avg.} 2.7\%. Furthermore, we conduct an analysis of the representations generated by G-Prompt and found that they have high interpretability with respect to task performance.








\section{Background}
\subsection{Text-Attributed Graph}

Let $G = \{V,A\}$ denotes a text-attributed graph (TAG), where $V$ is the node set and $A$ is the adjacency matrix. Each node $i \in V$ is associated with a sentence $S_i = \{s_{i,0},s_{i,1},...,s_{i,|S_i|}\}$, which represents the textual feature of the node. In most cases, the first token in each sentence (i.e., $s_{i,0}$) is $[\mathbf{cls}]$, indicating the beginning of the sentence. This paper focuses on how to unsupervised extract high-quality node features on TAGs for various downstream tasks.

\subsection{Pretrained Language Models}

Before we introduce G-Prompt, we require some basic concepts of pre-trained language models.

\textbf{Framework of PLMs}. A PLM consists of a multi-layer transformer encoder that takes a sentence $S_i$ as input and outputs the hidden states of each token:
\begin{equation}
    \mathbf{PLM}(\{s_{i,0}, s_{i,1},...,s_{i,|S_i|}\}) = \{h_{i,0}, h_{i,1},...,h_{i,|S_i|}\},
\end{equation}
where $h_{i,k}$ is the dense hidden state of $s_{i,k}$.

\textbf{Pretraining of PLMs}. The fill-mask task is commonly used to pre-train PLMs \cite{devlin2018bert,liu2019roberta,he2020deberta}. Given a sentence $S_i$, the mask stage involves randomly selecting some tokens and replacing them with either $[\mathbf{mask}]$ or random tokens, resulting in a modified sentence $\hat{S}_i = \{s_{i,0}, s_{i,1},...,\hat{s}_{i,k},...,s_{i,|S_i|}\}$, where $\hat{s}_{i,k}$ represents the masked token. In the filling stage, $\hat{S}_i$ is passed through the transformer encoder, which outputs the hidden states of each token. We denote the hidden state of the masked token $\hat{s}_{i,k}$ as $\hat{h}_{i,k}$, which is used to predict the ID of the masked token:
\begin{equation}
    \hat{y}_{i,k} = f_{\rm{LM}}(\hat{h}_{i,k}),
\end{equation}
where $f_{LM}$ is a linear transformation with softmax fuction, $\hat{y}_{i,k} \in \mathbb{N}^{1\times T}$, and $T$ is the size of the vocabulary. The loss function of the fill-mask task is defined as $\mathcal{L} = \rm{CE}(\hat{y}_{i,k}, y_{i,k})$, where $y_{i,k}$ is the ID of the masked token, and $\rm{CE}(\cdot,\cdot)$ is the cross-entropy loss.

\textbf{Sentence Embedding}. The hidden state of the $[\mathbf{cls}]$ token ($h_{i,0}$) and the mean-pooling of all hidden states are commonly used as sentence embeddings \cite{reimers2019sentence, gao2021simcse}.

\textbf{Prompting on PLMs}. Sentence embedding and token embedding are simultaneously pre-trained in many PLMs. However, due to the gap between pretraining tasks and downstream tasks, sentence embedding always requires fine-tuning for specific tasks. To address this issue, some studies utilize prompts to extract sentence features \cite{jiang2022promptbert}. For example, suppose we have a paper's abstract $\{\mathbf{Abstract}\}$, and the task is to classify the subject of it. We can add some prompts to the sentence:
\begin{equation}
    \{This, is, a, paper, of, [\mathbf{mask}], subject, its, abstract, is,:, \mathbf{Abstract}\}
\end{equation} 
Then this sentence is encoded by PLMs, and we let $h_{i|p}$ denote the hidden state of the $[\mathbf{mask}]$ token in prompts. Extensive experiment shows that using prompts can shorten the gap between PLMs and downstream tasks and maximize the utilization of the knowledge PLMs learned during pretraining.

\subsection{Graph Neural Networks}

Graph Neural Networks (GNNs) have achieved remarkable success in modeling graph-structured data\cite{velivckovic2017graph,gasteiger2018predict}. The message-passing framework is a commonly used architecture of GNN. At a high level, GNNs take a set of node features $X^0$ and an adjacency matrix $A$ as input and iteratively capture neighbors' information via message-passing. More specifically, for a given node $i \in V$, each layer of message-passing can be expressed as:
\begin{equation}
    x_i^{k} = \mathbf{Pool}\{f_\theta(x^{k-1}_j) | j\in \mathcal{N}_i\}
\end{equation} 
where $\mathbf{Pool}\{\cdot\}$ is an aggregation function that combines the features of neighboring nodes, such as mean-pooling. And $\mathcal{N}_i$ denotes the set of neighbors of node $i$. 
%Different GNN architectures employ different aggregation methods; for instance, GraphSAGE utilizes mean-pool while GAT incorporates an attention mechanism.


% \subsection{Modeling TAGs}
% Most GNNs are designed to operate on continuous node features and cannot handle textual features directly. As a result, modeling TAGs requires combining LMs and GNNs. The most straightforward approach is to join the structure of GNNs and LMs and then end-to-end train them. However, most current LMs are based on Transformers with enormous trainable parameters, so end-to-end training requires significant computing resources.

% Recently, impressive results have been achieved by combining LM and GNNs using the soft connection, e.g., knowledge distillation, and expectation-maximization framework. However, this approach involves fine-tuning LM, which is also extremely computationally expensive. Furthermore, the finetuned model is task-specific, are hard to employ in other downstream tasks.

% A convenient framework commonly used in various applications involves using PLMs to unsupervised convert the textual features of nodes into continuous representations. Then, the extracted node representation and graph structure can be input into GNNs for end-to-end training. It's worth noting that the converted node feature is reusable for many downstream tasks.


% \hxw{problem}
\section{Method: G-Prompt}
Utilizing the information of downstream tasks and graphs is crucial for generating high-quality node representations. 
The term ``high quality'' is inherently task-specific, as exemplified by the fact that height is a useful feature in predicting user weight but fails to accurately predict age. 
Besides,  the valuable topological information of TAGs can significantly enhance the understanding of textual features in TAGs. 
However, extracting node features using both task and graph information simultaneously is significantly challenging. 
Current PLMs used for handling textual features are graph-free, while current graph-based methods employed to extract node features are primarily task-free. Therefore, this paper proposes a novel self-supervised method, G-Prompt, capable of extracting task-specific and graph-aware node representations. 

\begin{figure*}[t]
	\centering
	\includegraphics[width=0.95\textwidth]{./picture/Model.pdf}
	\caption{Framework of G-Prompt}
	\label{fig:exp}
\end{figure*}

\subsection{Overview}

While previous works have frequently employed PLMs to process TAGs, these investigations have been constrained in extracting a broad node representation from the text-based characteristics and have not incorporated task-specific prior knowledge. 
Consequently, additional learning supervision via GNNs is needed to enable the effective adaptation of these node representations to downstream tasks. 
To address this limitation, the paper suggests incorporating prompts and PLMs into the process of extracting task-relevant node features from TAGs.
%\hkq{which should be highlighted in the next sentence }
Nevertheless, PLMs only utilize contextual information to generate the prompts-related output, which may be insufficient for handling TAGs.
Graph structures often contain essential information that can facilitate a better understanding of textual features.
For instance, in a citation network, a masked sentence such as \textit{``This paper focuses on [MASK] learning in AI domain''} could have multiple candidate tokens based solely on context.
However, if many papers related to graphs are cited, we can infer with greater confidence that the masked token is likely \textit{``graph''}. 
At present, PLMs operate solely based on context, and their structure is graph-free. 
Directly incorporating graph information into PLMs by prompts is not feasible because limited prompts cannot describe the entire topological structure adequately.

Therefore, the proposed G-Prompt leverages a self-supervised based graph adapter and prompts to make PLMs aware of the graph information and downstream task. Given a specific TAG, the pipeline of G-Prompt is as follows: 
(1) Training an adapter on the given TAG to make PLMs graph-aware. 
Specifically, we propose a graph adapter that operates on the prediction layer of PLMs to assist in capturing graph information, which is fine-tuned by the fill-mask task based on the textual data contained by the given TAG. 
(2) Employing task-specific prompts and fine-tuned graph adapters to generate task-aware and graph-aware node features.

\subsection{Fine-Tuning PLMs with the Graph Adapter}

% Currently, PLMs that have undergone large-scale text data pre-training have strong contextual understanding abilities and generalization abilities which form the basis for us to extract specific task information using prompts. 
Using adapters to enable PLMs to perceive graph information is a straightforward idea. 
However, unlike adapters used for downstream task fine-tuning \cite{hu2021lora,liu2022few}, the graph adapter is used to combine prompts in order to extract task-relevant node representations. 
This is an unsupervised process, which means that the graph adapter only receives self-supervised training on given TAGs. 
Consequently, the most challenging aspect of graph adapters is how to assist PLMs in perceiving graph information while also maintaining their contextual understanding capability. 
Additionally, the graph adapter is only trained on a given TAG, generalizing to prompt tokens can also be quite difficult.
Next, we introduce the graph adapter and discuss how it overcomes these challenges in detail.

% The focus of this paper is on promoting the PLM to extract node features of TAGs, which is essentially a fill-mask task. Therefore, this paper proposes Graph Adapter, which \textbf{targets maximally retaining LMs' contextual modeling ability while enabling them to incorporate graph information during the fill-mask process.}

\textbf{Context-friendly adapter placement.} 
The fill-mask task involves two modules of PLMs: a transformer-based module that models context information to obtain representations of masked tokens and a linear transformation that decodes the representation to output the probable IDs of the masked token.
To avoid compromising the contextual modeling ability of PLMs, the Graph Adapter only perform on the last layer of PLMs.
More specifically, the graph adapter is a GNN structure combing with the pre-trained final layer of the PLMs.
Given a specific masked token $\hat{s}_{i,k}$, The inputs of the Graph Adapter are the masked token $\hat{h}_{i,k}$, sentence representations of node $i$ and its neighbors and output is the prediction of the IDs' of the masked token. 
This process aligns with intuition — inferring a possible token based on context first and then determining the final token based on graph information. Formally,
\begin{equation}
    \hat{y}_{i,k} =  \textbf{GraphAdapter} \{f_{\rm{LM}}, \hat{h}_{i, k}, z_i, \{z_j \in \mathcal{N}_i\}, \Theta\},
\end{equation}
where the $z_i$ and $z_j$ denote the sentence embedding of node $i$ and $j$. Note, sentence embedding is task-free and only used to model nodes' influence on their neighbor.
In this paper, we utilize sentence embedding of nodes' textual features as their node feature. 
$\Theta$ is all trainable parameters of the Graph Adapter. 

\textbf{Prompting-friendly network structure}.
% The hidden state of the masked token contains contextual information extracted through the transformer in PLMs. 
% Therefore, directly manipulating it may also affect the contextual information it contains.
The parameters of the adapter are only trained on the fill-mask task based on the textual data contained by the target TAG. 
But the adapter will be used for combining prompts to generate task-related node features in various subsequent downstream tasks.
So the generalization ability of the adapter is crucial. 
On the one hand, the distribution of hidden states responding to masked tokens in prompts may be different from the hidden states used to train the adapter. 
On the other hand, the candidate tokens for task-specific prompts may not appear in the tokens of the TAG. 
Therefore, we carefully design the network structure of the graph adapter and utilize the pre-trained prediction layer of PLM to improve its generalization ability of it.

When it comes to the graph adapter's training stage, it's possible that the hidden states associated with certain prompts may not be present. This means that directly manipulating those hidden states could result in overfitting the tokens already present in the TAGs.
Therefore, the graph adapter models the influence of each modeled node on the distribution of surrounding neighbor tokens based on node feature, which remains unchanged when prompts are added. Considering that some tokens can be predicted well based solely on their context and that different neighbors may have different influences on the same node, the impact of a neighbor on a token is determined jointly by a gate mechanism and the token's context. Give a specific node $i$, it's neighbor $j$, and hidden states of a masked token $\hat{h}_{i,j}$,
\begin{equation}
    \tilde{h}_{i, k, j} = a_{ij}\hat{h}_{i,k} + (1-a_{ij})g(z_j,\Theta_g)
\end{equation}
where $a_{ij} = \mathrm{sigmoid}((z_iW_q)(z_jW_k)^T)$. Here, $g(\cdot)$ represents multi-layer perceptions (MLPs) with parameters $\Theta_g$ that model the influence of node $j$.
It is worth noting that when considering the entire graph, $g(z_j, \Theta_g)$ will be combined with many marked tokens of node $j$'s neighbors, which can help to prevent $g(z_j, \Theta_g)$ from being overfitted on a few tokens.

Subsequently, the graph adapter combines all neighbor influence to infer the final prediction result. Since the prediction layer of PLM (i.e., $f_{LM}(\cdot)$) is well-trained on massive tokens, it also contains an amount of knowledge. Therefore, the graph adapter reuses this layer to predict the final result. 
\begin{equation}
    \tilde{y}_{i,k} =  \mathbf{Pool}\{f_{\rm{LM}}(\tilde{h}_{i, k, j}) | j\in \mathcal{N}_i\},
\end{equation}
In this equation, the $\mathbf{Pool}(\cdot)$ used in this paper is mean-pooling. 
It is worth noting that the position of $f_{\rm{LM}}(\cdot)$ can be interchanged with pooling since it is just a linear transformation. All trainable parameters in the graph adapter are denoted by $\Theta = \{\Theta_g, W_q, W_k\}$.


\subsection{Model optimization of G-Prompt}

The graph adapter is optimized by the original fill-mask loss, $\mathcal{L}_{i,k} = \mathrm{CE} (\tilde{y}_{i,k}, y_{i,k})$, where $\hat{y}_{i,k}$ is the predicted probability of the $k$-th masked token for the node $i$ and $y_{i,k}$ is the true label. We aim to minimize $\mathcal{L}_{i,k}$ with respect to $\Theta$. 

However, in actual optimization, the prediction results of $\tilde{y}_{i,k,j} = f_{\rm{LM}}(\tilde{h}_{i, k, j})$ consist of many small values because of the large vocabulary size of the language model. 
Therefore, using mean-pooling presents a significant problem as it is insensitive to these small values. For example, during some stages of the optimization process, a node may have mostly $0.9$ predictions for the ground truth based on each edge, with only a few being $0.1$. 
Averaging them together would result in a very smooth loss, making it difficult to train the influence of neighbors with temporarily predicted values of 0.1. 
To address this issue, we use geometric mean instead of mean-pooling in the finetuning stage of the graph adapter, which is more sensitive to small values. 
It is easy to prove that the geometric mean of positive numbers is smaller than the arithmetic means, making it harder to smooth and helping the model converge faster. formally, in finetuning stage, the loss function is:
\begin{equation}
    \mathcal{L}_{i,k} = - y_{i,k} \odot \log\{(\prod_{j\in \mathcal{N}_i}{\tilde{y}_{i,k,j}})^{1/|\mathcal{N}_i|}\}
    = -\sum_{j\in \mathcal{N}_i}{ \frac{1}{|\mathcal{N}_i|}y_{i,k}\odot \log(\tilde{y}_{i,k,j})} 
\end{equation}
On the right-hand side of the equation, we are essentially minimizing $\tilde{y}_{i,k,j}$ through the cross-entropy loss $\mathcal{L}_{i,k,j}= \frac{1}{|\mathcal{N}_i|}\mathrm{CE}(\tilde{y}_{i,k,j},y_{i,k})$. It is worth noting that the graph adapter is only performed on the last layer of PLMs. As a result, we can sample a set of masked tokens and preserve their hidden states inferred by the PLMs before training. This implies that training of graph adapters can be achieved with very few computing resources.
\subsection{Prompt-based Node Representations}
After training the graph adapter, it can be combined with task-specific prompts to generate task-specific and graph-aware node representations. Similar to other prompt-based approaches, we simply add task-specific prompts directly into the textual feature. For example, we might use the prompt ``This is a [MASK] user, consider their profile: [textual feature].'' Formally, this process can be expressed as $\hat{h}_{i|p} = \mathbf{PLM}(\{[P_0],[P_1]...[MASK],S_i\})$.
Where, $\hat{h}_{i|p}$ represents the hidden state of the inserted [MASK], while $[P_0],[P_1]...$ refers to the task-specific prompts. The resulting hidden state is then fed into the graph encoder to decode the most probable token.
\begin{equation}
    \hat{y}_{i|p} = \mathbf{Pool}\{{f_{\rm{LM}}(a_{i,j}\hat{h}_{i|p}+(1-a_{i,j})g(z_j,\Theta_g))} | j\in \mathcal{N}_i\}
\end{equation}
$\hat{y}_{i|p}$ is a $|D|$-dimensional vector, where $|D|$ is the size of the PLM vocabulary. Therefore, directly using this prediction result for node representation is not conducive to downstream tasks and storage. Thus, we use the filtered results as node features, denoted by 
$
    x_{i|p} = \mathrm{Filter}(\hat{y}_{i|p})
$. 
Note, each dimension represents the probability of a token being inferred by PLMs with the graph adapter based on node textual features, neighbors' information, and task-respected prompts. Intuitively, tokens that are unrelated to downstream tasks are almost the same for all nodes. 
Therefore, for $Y_{p} \in \mathbb{N}^{|V|\times|D|}$, which denotes prediction results of all nodes. This paper sorts all columns of $Y_p$ in descending order of standard deviation and keeps the top $M$ columns as the node features. Note, we can also manually select task-relevant tokens based on prior knowledge of the task and use them as node features. 
% \subsection{Model optimization of G-Prompt}
% The optimization objective of G-Prompt is straightforward: (1) model the impact of each neighbor on a node's word distribution and (2) infer masked words based on the influence of all neighbors. 
% However, in actual optimization, the large vocabulary size of the language model leads to the prediction results of $\hat{y}_{i,j,k}$ that consist of many small values after softmax. 
% Therefore, using mean-pooling during optimization presents a significant problem as it is insensitive to these small values. For example, during some stages of the optimization process, a node may have mostly 0.9 predictions for the ground truth based on each edge, with only a few being 0.1. 
% Averaging them together would result in a very smooth loss, making it difficult to train the neighbors with temporarily predicted values of 0.1. 
% To address this issue, we use geometric mean instead of mean-pooling in the finetuning stage of the Graph Adapter, which is more sensitive to small values. 
% It is easy to prove that the geometric mean of positive numbers is smaller than the arithmetic means, making it harder to smooth and helping the model converge faster. formally, in finetuning stage, the pooling function is:
% \begin{equation}
%     \mathbf{Pool}^{tr}\{\hat{y}_{i,j,k} | j\in \mathcal{N}_i\} = (\prod_{j\in \mathcal{N}_i}{\hat{y}_{i,j,k}})^{1/|\mathcal{N}_i|}
% \end{equation}

% Considering that multiplication may easily exceed the precision of calculations, we have expanded the loss function based on geometric mean aggregation during optimization. The formula is as follows:
% \begin{equation}
%     \mathcal{L}_{i,k} = - y_{i,k} \odot \log\{(\prod_{j\in \mathcal{N}_i}{\hat{y}_{i,j,k}})^{1/|\mathcal{N}_i|}\}
%     = -\sum_{j\in \mathcal{N}_i}{ \frac{1}{\mathcal{N}_i}y_{i,k}\odot \log(\hat{y}_{i,j,k})}
% \end{equation}

% Therefore, the whole training pipeline is: 
\section{Experiment}
\subsection{Experiment setup}
\textbf{Dataset.} We conduct experiments on three public and real-world datasets, which are Ogbn-arxiv\cite{hu2020open} (shorted as Arxiv), Instagram\cite{kim2020multimodal}, and Reddit\footnote{\url{https://convokit.cornell.edu/documentation/subreddit.html}}, to evaluate the effectiveness of the proposed method G-Prompt. Specifically, Ogbn-arxiv is a citation network where edges represent citation relationships, nodes represent papers and the text attribute is the abstracts of papers. The task on this graph is to predict paper subjects. Instagram is a social network where edges represent following relationships, nodes represent users, and the prediction task is to classify commercial users and normal users in this network. The text attribute is the users' profile. Reddit is also a social network where each node denotes a user, the node features are the content of users' historically published subreddits, and edges denote whether two users have replied to each other. The prediction task is to classify whether a user is in the top 50\% popular (average score of all subreddits). Table 1 shows detailed statistics of these datasets. More details about Instagram and Reddit are provided in the Appendix.

\textbf{Evaluate: compare different representations generated by different methods. } We compare the proposed G-Prompt with PLM-based and Graph-based node feature-extracting methods. For the PLM-based methods, we consider three options: (1) direct use of sentence embedding as node features, and (2) use of the hidden states of masked tokens based on hard prompts as node features. (3) use of the prediction result of masked tokens based on prompts as node feature. For graph-based methods, we compare our proposed method with GAE and GIANT, which first conduct self-supervised learning on graphs to train PLMs or node feature encoders. To ensure a fair comparison, we add prompts into graph-based baselines. Except for GAINT and OGB features, the PLM we use in this paper is RoBERTa-Large\cite{liu2019roberta}. Note that all prompts used in baselines are the same as those in G-Prompt.

\textbf{Implementation details.} For G-Prompt, we first train three graph adapters of G-Prompt on Arxiv, Instagram, and Reddit with 50 epochs, 100 epochs, and 100 epochs respectively. All of them are optimized using AdamW\cite{loshchilov2017decoupled} with warm-up. For more details on the hyper-parameter settings, please refer to the Appendix. For each node, we replace 10\% tokens with [mask] and use these masked tokens to train the graph adapter. During the whole training stage, all task-related prompts are invisible. Then we use prompts, finetuned graph adapters, and PLMs to jointly extract node features. For graph-based methods, we train them on each dataset with searched hyper-parameters.

\begin{table}[t!]
  \caption{ Statistics of the  datasets}
    \label{Table:dataset}
  \centering
\resizebox{0.7\textwidth}{!}{
      \begin{tabular}{cccccc}
        \toprule
        
        \textbf{Dataset}&\textbf{\# Nodes} &\textbf{\# Eeges}&\textbf{Avg. Node Degree}&\textbf{Test Ratio (\%)}&\textbf{Metric} \\ 
         \hline
        \textbf{Arxiv}& 169,343& 1,166,243& 13.7&28&ACC\\
        \textbf{Instagram}& 11,339 &377,812& 66.6& 60&ROC\\
        \textbf{Reddit}&33,434&198,448&11.9& 33&ROC\\

        %\cmidrule(r){1-5}
        \bottomrule
      \end{tabular}
      }
\end{table}
\subsection{Few-shot learning}

To evaluate the performance of representations generated by different methods in few-shot learning, we compare the performance of different representations at different shot numbers based on the same GNN backbone. The GNN backbone used in the performance comparison on different shot numbers is GraphSAGE\cite{velivckovic2017graph}. In addition, we also compare the performance of different representations combined with three different neural network architectures (i.e., MLP, and RevGAT\cite{li2021training}) on downstream tasks with the same number of shots. For Arxiv, we use a publicly available partitioned test set, while for Instagram and Reddit, we randomly sample 60\% and 33\% of the data as the test sets, respectively. To consider the randomness of partitioning and training, each experimental result is based on five random partitions (the partitions are the same for different baselines), the experiment is repeated five times for each partition, and the variance of 5$\times$5 results is reported.

The experiment results on different shots-num are shown in Table 2. The experiment shows that: (1) \textbf{Graph Information can improve the performance of node representation}. In general, approaches that use sentence representations or those that involve self-supervised training with graph information tend to outperform non-trained representations. For example, GAE shows an average improvement of \textit{avg.} 6.2\% compared to RoBERTa's [cls], and GIANT shows \textit{avg.} 6.2\% improvement over cls representation. For graph-based self-supervised tasks, fine-tuning language models might be more suitable for larger datasets. GIANT outperforms GAE by \textit{avg.} 3.0\% on Arxiv, but lags behind by \textit{avg.} 1.4\% on Instagram and Reddit.
(2) 
\textbf{Downstream task-related prompts can improve performance for most methods}. For graph-free language models, prompt-based representations can improve performance by \textit{avg.} 5.7\%, and the overall performance of prediction values and hidden states corresponding to prompts is similar. For graph-based methods, prompts in GAE improve performance by \textit{avg.} 1.3\%, while prompts in GIANT lead to an average improvement of \textit{avg.} 1.2\%. However, we note that prompts are unstable for graph-based pre-trained models. GAE shows a decline in 4 experiments, while prompts only bring a slight improvement in GIANT (compared to language models).
(3) \textbf{Our method is capable of utilizing both graph information and downstream task prompts simultaneously}, achieving state-of-the-art performance. Compared to PLM representations without prompts, our method improves by \textit{avg.} 10.6\%. Compared to PLM-prompt, it improves by \textit{avg.} 4.6\%, and compared to GIANT, it improves by \textit{avg.} 4.1\%.

Besides, we also compared these methods under different GNN backbone. as Figure 2 shows, the node representation extracted by G-Prompt in different GNN-backbone also achieves the SOTA performance compared to other baseline methods. 

\begin{table}[t!]
\caption{The performance in different shots on three datasets. Each row corresponds to a specific method. Every column lists the performance of the models in specific the shot number per class of the dataset   (mean ± std\%, the best results are bolded and the runner-ups are underlined). Accuracy is used as evaluation metric for the task in Arxiv while AUC is used as evaluation metric for the other two datasets.}
\label{table:table}
\setlength{\tabcolsep}{0.5mm} 
\renewcommand{\arraystretch}{0.85}
\centering
\resizebox{1.0\textwidth}{!}{
\begin{tabular}{c|ccc|ccc|ccc}

\toprule
{\textbf{Dataset}} & \multicolumn{3}{c|}{\textbf{Arxiv}} & \multicolumn{3}{c|}{\textbf{Instagram}} & \multicolumn{3}{c}{\textbf{Reddit}} \\
%\cmidrule(r){1-5}
\# shots per class & 10 & 50 &100 & 10 & 50 &100 & 10 & 50 &100 \\
\midrule


\renewcommand{\arraystretch}{0.85}
\thead{ OGB-Feature} & \thead{0.4576  \tiny{(0.0324)}} & \thead{0.5495  \tiny{(0.0171)}}&\thead{ 0.5875  \tiny{(0.0146)}} & -&-&-&-&-&-\\

    
\thead{ PLM+GAE} & \thead{0.5016  \tiny{(0.0510)} }& \thead{ 0.5608  \tiny{(0.0101)}} & \thead{0.5810  \tiny{(0.0125)} } &\thead{ 0.5258  \tiny{(0.0635)}} & \thead{0.5818 \tiny{(0.0101)} }& \thead{0.5821 \tiny{(0.0058)}} & \thead{ 0.5653  \tiny{(0.0256)} }& \thead{0.6019 \tiny{(0.0174)} }& \thead{0.6154  \tiny{(0.0128)}}\\

    
\thead{ PLM+GAE+prompt} &\thead{0.5189  \tiny{(0.0333)} }& \thead{ 0.5801  \tiny{(0.0102)}} & \thead{0.6063  \tiny{(0.0109)} } &\thead{ 0.5418  \tiny{(0.0298)}} & \thead{0.5705  \tiny{(0.0233)} }& \thead{0.5867 \tiny{(0.0100)}} & \thead{ 0.5619  \tiny{(0.0425)} }& \thead{0.5968 \tiny{(0.0237)} }& \thead{0.6173  \tiny{(0.0160)}}\\
    
    \thead{ GIANT} & \thead{0.5050  \tiny{(0.0308)} }& \thead{ 0.5798  \tiny{(0.0119)}} & \thead{0.6081  \tiny{(0.0109)} } &\thead{ 0.5185  \tiny{(0.0323)}} & \thead{0.5601  \tiny{(0.0304)} }& \thead{0.5752 \tiny{(0.0251)}} & \thead{ 0.5618  \tiny{(0.0431)} }& \thead{0.5954 \tiny{(0.0131)} }& \thead{0.6130 
    \tiny{(0.0117)}}\\

     
    \thead{ GIANT + prompt} & \thead{0.5140  \tiny{(0.0320)} }& \thead{ 0.5809  \tiny{(0.0223)}} & \thead{0.6126  \tiny{(0.0159)} } &\thead{ 0.5239  \tiny{(0.0309)}} & \thead{0.5721  \tiny{(0.0361)} }& \thead{0.5949 \tiny{(0.0089)}} & \thead{ 0.5661  \tiny{(0.0459)} }& \thead{0.5968 \tiny{(0.0096)} }& \thead{0.6145  \tiny{(0.0105)}}\\
    \hline
    PLM-cls & \thead{0.4697  \tiny{(0.0577)} }& \thead{ 0.5414  \tiny{(0.0400)}} & \thead{0.5869  \tiny{(0.0300)} } &\thead{ 0.5165  \tiny{(0.0217)}} & \thead{0.5385  \tiny{(0.0344) }}& \thead{0.5690 \tiny{(0.0253)}} & \thead{ 0.4965  \tiny{(0.0373)} }& \thead{0.5236 \tiny{(0.0394)} }& \thead{0.5754  \tiny{(0.0348)}}\\

  
    \thead{ PLM-Prompt-dense} & \thead{0.5117  \tiny{(0.0398)} }& \thead{ 0.5631  \tiny{(0.0352)}} & \thead{0.5865  \tiny{(0.0296)} } &\thead{ 0.5458  \tiny{(0.0420)}} & \thead{0.5796  \tiny{(0.0276)} }& \thead{\underline{0.6055 \tiny{(0.0122)}}} & \thead{ 0.5363  \tiny{(0.0530)} }& \thead{0.5648 \tiny{(0.0385)} }& \thead{0.5998 \tiny{(0.0383)}}
                   \\

                   
    \thead{ PLM-Prompt-sparse} & \thead{0.5201  \tiny{(0.0284)} }& \thead{ 0.5784  \tiny{(0.0213)}} & \thead{0.6085  \tiny{(0.0203)} } &\thead{ 0.5363  \tiny{(0.0348)}} & \thead{0.5757  \tiny{(0.0225)} }& \thead{0.5910 \tiny{(0.0229)}} & \thead{ 0.5403  \tiny{(0.0424)} }& \thead{0.5761 \tiny{(0.0359)} }& \thead{0.6082  \tiny{(0.0192)}}\\
               \hline 
             
  
      % \thead{G-Prompt \\ } & \thead{0.5232 \\ \scriptsize{$\pm$0.0348} }& \thead{ \underline{0.5909} \\ \underline{\scriptsize{$\pm$0.0159}}} & \thead{\textbf{0.6240$^*$} \\ \textbf{\scriptsize{$\pm$0.0156}} } &\thead{ \underline{0.5519} \\ \underline{\scriptsize{$\pm$0.0355}}} & \thead{\underline{0.5787} \\ \underline{\scriptsize{$\pm$0.0333}} }& \thead{\underline{0.6084} \\\underline{\scriptsize{$\pm$0.0127}}} & \thead{ \textbf{0.5794$^*$} \\ \textbf{\scriptsize{$\pm$0.0495}} }& \thead{\underline{0.6149} \\\underline{\scriptsize{$\pm$0.0270} }}& \thead{\underline{0.6427} \\ \underline{\scriptsize{$\pm$0.0185}}}\\
      
      \thead{G-Prompt} & \thead{\underline{0.5248 \tiny{(0.0382)}} }& \thead{ \textbf{0.5927}  \textbf{\tiny{(0.0142)}}} & \thead{\underline{0.6167  \tiny{(0.0138)}}}  &\thead{ \textbf{0.5576}  \textbf{\tiny{(0.0330)}}} & \thead{\textbf{0.5917} \textbf{\tiny{(0.0242)}} }& \thead{\textbf{0.6090} \textbf{\tiny{(0.0135)}}} & \thead{ \textbf{0.5728} \textbf{\tiny{(0.0491)}} }& \thead{\textbf{0.6167} \textbf{\tiny{(0.0289)}} }& \thead{\textbf{0.6472} \textbf{\tiny{(0.0224)}}}\\

     \hline
     
 \thead{G-Prompt  w/o gate} & \thead{\textbf{0.5291}  \textbf{\tiny{(0.0315)}} }& \thead{ 0.5877 \tiny{(0.0192)}} & \thead{\textbf{0.6212} \textbf{\tiny{(0.0190)}} } &\thead{ \underline{0.5507 \tiny{(0.0336)}}} & \thead{0.5706  \tiny{(0.0262)} }& \thead{0.5942 \tiny{(0.0178)}} & \thead{ 0.5501  \tiny{(0.0604)} }& \thead{\underline{0.5926 \tiny{(0.0385)}} }& \thead{\underline{0.6361 \tiny{(0.0268)}}}\\

     

   
     
    \thead{G-Prompt w/o graph} & \thead{0.5226 \tiny{(0.0322)}} & \thead{\underline{0.5880 \tiny{(0.0168)}}} & \thead{0.6059 \tiny{(0.0101)}} \
     & \thead{0.5234 \tiny{(0.0236)}} & \thead{0.5657 \tiny{(0.0377)}} & \thead{0.5914 \tiny{(0.0199)}}     & \thead{\underline{0.5536  \tiny{(0.0438)}}} & \thead{0.5683  \tiny{(0.0390)}} & \thead{0.6054 \tiny{(0.0263)}} \\ 

     
    \thead{G-Prompt w/o SSL} & \thead{0.5210  \tiny{(0.0372)}} & \thead{0.5793  \tiny{(0.0168)}} & \thead{0.6092  \tiny{(0.0168)}} \
     & \thead{0.5378  \tiny{(0.0419)}} & \thead{\underline{0.5801 \tiny{(0.0269)}}} & \thead{0.6004 \tiny{(0.0193)}}     & \thead{0.5494  \tiny{(0.0502)}} & \thead{0.5885  \tiny{(0.0365)}} & \thead{0.6149  \tiny{(0.0263)}} \\ 
   
    \bottomrule
  \end{tabular}
  }
\end{table}

\begin{figure*}[ht]
	\centering
	\includegraphics[width=0.8\textwidth]{./picture/exp.pdf}
	\caption{Comparison with different GNN backbone on 50-shots setting. Three picture correspond to the performance on three different datasets respectively. }
	\label{fig:exp}
\end{figure*}



\subsection{In-depth analysis of G-Prompt}
To validate the rationality of G-Prompt, we conduct ablation study to compare the performance of G-Prompt and its variants. These variants include removing the gate mechanism in graph-adapter (denoted as ``w/o gate''), keeping only self-loops while removing the input graph (denoted as ``w/o graph''), and not training graph-adapter by self-supervised learning (denoted as ``w/o SSL''). The experiment results show that all variants perform worse than G-Prompt. Specifically, removing the Graph-Adapter training process leads to  \textit{avg.} 2.8\% decrease in performance, which demonstrates the effectiveness of training graph-adapter through the fill- mask task. After removing the graph input, the performance of G-Prompt decreases by \textit{avg.} 3.8\%, which further confirms that the improvement of G-Prompt stems from the graph adapter's ability to assist language models in comprehending graph structures compared to using language model prompts directly. Moreover, removing the gate mechanism results in a \textit{avg.} 1.8\% decrease in performance, indicating that the design of the graph-adapter structure is reasonable.




















% saved in 5:03

% \begin{table}[t!]
% \caption{This is the title}
% \label{table:table}
% \renewcommand{\arraystretch}{0.85}
% \centering
% \resizebox{0.9\textwidth}{!}{
% \begin{tabular}{c|ccc|ccc|ccc}
% \toprule
% {\textbf{Dataset}} & \multicolumn{3}{c|}{\textbf{Arxiv}} & \multicolumn{3}{c|}{\textbf{Ins}} & \multicolumn{3}{c}{\textbf{Reddit}} \\
% %\cmidrule(r){1-5}
% \# shots per class & 10 & 50 &100 & 10 & 50 &100 & 10 & 50 &100 \\
% \midrule


% \renewcommand{\arraystretch}{0.85}
% \thead{ OGB-Feature} & \thead{0.4576 \\ \tiny{$\pm$0.0324}} & \thead{0.5495 \\ \tiny{$\pm$0.0171}}&\thead{ 0.5875 \\ \tiny{$\pm$0.0146}} & -&-&-&-&-&-\\

    
% \thead{ RoBERTa*+GAE} & \thead{0.5016 \\ \tiny{$\pm$0.0510} }& \thead{ 0.5608 \\ \tiny{$\pm$0.0101}} & \thead{0.5810 \\ \tiny{$\pm$0.0125} } &\thead{ 0.5258 \\ \tiny{$\pm$0.0635}} & \thead{0.5818 \\ \tiny{$\pm$0.0101} }& \thead{0.5821 \\\tiny{$\pm$0.0058}} & \thead{ 0.5653 \\ \tiny{$\pm$0.0256} }& \thead{0.6019 \\\tiny{$\pm$0.0174} }& \thead{0.6154 \\ \tiny{$\pm$0.0128}}\\

    
% \thead{ RoBERTa*+GAE\\+prompt} &\thead{0.5189 \\ \tiny{$\pm$0.0333} }& \thead{ 0.5801 \\ \tiny{$\pm$0.0102}} & \thead{0.6063 \\ \tiny{$\pm$0.0109} } &\thead{ 0.5418 \\ \tiny{$\pm$0.0298}} & \thead{0.5705 \\ \tiny{$\pm$0.0233} }& \thead{0.5867 \\\tiny{$\pm$0.0100}} & \thead{ 0.5619 \\ \tiny{$\pm$0.0425} }& \thead{0.5968 \\\tiny{$\pm$0.0237} }& \thead{0.6173 \\ \tiny{$\pm$0.016}}\\
    
%     \thead{ GIANT} & \thead{0.5050 \\ \tiny{$\pm$0.0308} }& \thead{ 0.5798 \\ \tiny{$\pm$0.0119}} & \thead{0.6081 \\ \tiny{$\pm$0.0109} } &\thead{ 0.5185 \\ \tiny{$\pm$0.0323}} & \thead{0.5601 \\ \tiny{$\pm$0.0304} }& \thead{0.5752 \\\tiny{$\pm$0.0251}} & \thead{ 0.5618 \\ \tiny{$\pm$0.0431} }& \thead{0.5954 \\\tiny{$\pm$0.0131} }& \thead{0.6130 \\ \tiny{$\pm$0.0117}}\\

     
%     \thead{ GIANT \\+ prompt} & \thead{0.5140 \\ \tiny{$\pm$0.0320} }& \thead{ 0.5809 \\ \tiny{$\pm$0.0223}} & \thead{0.6126 \\ \tiny{$\pm$0.0159} } &\thead{ 0.5239 \\ \tiny{$\pm$0.0309}} & \thead{0.5721 \\ \tiny{$\pm$0.0361} }& \thead{0.5949 \\\tiny{$\pm$0.0089}} & \thead{ 0.5661 \\ \tiny{$\pm$0.0459} }& \thead{0.5968 \\\tiny{$\pm$0.0096} }& \thead{0.6145 \\ \tiny{$\pm$0.0105}}\\
%     \hline
%     RoBERTa-cls & \thead{0.4697 \\ \tiny{$\pm$0.0577} }& \thead{ 0.5414 \\ \tiny{$\pm$0.0400}} & \thead{0.5869 \\ \tiny{$\pm$0.0300} } &\thead{ 0.5165 \\ \tiny{$\pm$0.0217}} & \thead{0.5385 \\ \tiny{$\pm$0.0344} }& \thead{0.5690 \\\tiny{$\pm$0.0253}} & \thead{ 0.4965 \\ \tiny{$\pm$0.0373} }& \thead{0.5236 \\\tiny{$\pm$0.0394} }& \thead{0.5754 \\ \tiny{$\pm$0.0348}}\\

  
%     \thead{ RoBERTa-Prompt\\-dense} & \thead{0.5117 \\ \tiny{$\pm$0.0398} }& \thead{ 0.5631 \\ \tiny{$\pm$0.0352}} & \thead{0.5865 \\ \tiny{$\pm$0.0296} } &\thead{ 0.5458 \\ \tiny{$\pm$0.0420}} & \thead{0.5796 \\ \tiny{$\pm$0.0276} }& \thead{0.6055 \\\tiny{$\pm$0.0122}} & \thead{ 0.5363 \\ \tiny{$\pm$0.0530} }& \thead{0.5648 \\\tiny{$\pm$0.0385} }& \thead{0.5998 \\ \tiny{$\pm$0.0383}}
%                    \\

                   
%     \thead{ RoBERTa-Prompt\\-vocabulary} & \thead{0.5201 \\ \tiny{$\pm$0.0284} }& \thead{ 0.5784 \\ \tiny{$\pm$0.0213}} & \thead{0.6085 \\ \tiny{$\pm$0.0203} } &\thead{ 0.5363 \\ \tiny{$\pm$0.0348}} & \thead{0.5757 \\ \tiny{$\pm$0.0225} }& \thead{0.5910 \\\tiny{$\pm$0.0229}} & \thead{ 0.5403 \\ \tiny{$\pm$0.0424} }& \thead{0.5761 \\\tiny{$\pm$0.0359} }& \thead{0.6082 \\ \tiny{$\pm$0.0192}}\\
%                \hline 
             
  
%       % \thead{G-Prompt \\ } & \thead{0.5232 \\ \scriptsize{$\pm$0.0348} }& \thead{ \underline{0.5909} \\ \underline{\scriptsize{$\pm$0.0159}}} & \thead{\textbf{0.6240$^*$} \\ \textbf{\scriptsize{$\pm$0.0156}} } &\thead{ \underline{0.5519} \\ \underline{\scriptsize{$\pm$0.0355}}} & \thead{\underline{0.5787} \\ \underline{\scriptsize{$\pm$0.0333}} }& \thead{\underline{0.6084} \\\underline{\scriptsize{$\pm$0.0127}}} & \thead{ \textbf{0.5794$^*$} \\ \textbf{\scriptsize{$\pm$0.0495}} }& \thead{\underline{0.6149} \\\underline{\scriptsize{$\pm$0.0270} }}& \thead{\underline{0.6427} \\ \underline{\scriptsize{$\pm$0.0185}}}\\
      
%       \thead{G-Prompt} & \thead{\underline{0.5248} \\ \underline{\tiny{$\pm$0.0382}} }& \thead{ \textbf{0.5927$^*$} \\ \textbf{\tiny{$\pm$0.0142}}} & \thead{\underline{0.6167} \\ \tiny{\underline{$\pm$0.0138}}}  &\thead{ \textbf{0.5576$^*$} \\ \textbf{\tiny{$\pm$0.0330}}} & \thead{\textbf{0.5917$^*$} \\ \textbf{\tiny{$\pm$0.0242}} }& \thead{\textbf{0.6090$^*$} \\\textbf{\tiny{$\pm$0.0135}}} & \thead{ \textbf{0.5728}$^*$ \\ \textbf{\tiny{$\pm$0.0491}$^*$} }& \thead{\textbf{0.6167$^*$}\\\textbf{\tiny{$\pm$0.0289}} }& \thead{\textbf{0.6472$^*$} \\\textbf{ \tiny{$\pm$0.0224}}}\\

%      \hline
     
%  \thead{G-Prompt \\ w/o gate} & \thead{\textbf{0.5291$^*$} \\ \textbf{\tiny{$\pm$0.0315}} }& \thead{ 0.5877 \\ \tiny{$\pm$0.0192}} & \thead{\textbf{0.6212$^*$} \\ \textbf{\tiny{$\pm$0.0190}} } &\thead{ \underline{0.5507} \\ \underline{\tiny{$\pm$0.0336}}} & \thead{0.5706 \\ \tiny{$\pm$0.0262} }& \thead{0.5942 \\\tiny{$\pm$0.0178}} & \thead{ 0.5501 \\ \tiny{$\pm$0.0604} }& \thead{\underline{0.5926} \\\underline{\tiny{$\pm$0.0385}} }& \thead{\underline{0.6361} \\ \underline{\tiny{$\pm$0.0268}}}\\

     

   
     
%     \thead{G-Prompt \\ w/o graph} & \thead{0.5226 \\ \tiny{$\pm$0.0322}} & \thead{\underline{0.5880} \\ \underline{\tiny{$\pm$0.0168}}} & \thead{0.6059 \\ \tiny{$\pm$0.0101}} \
%      & \thead{0.5234 \\ \tiny{$\pm$0.0236}} & \thead{0.5657 \\ \tiny{$\pm$0.0377}} & \thead{0.5914 \\ \tiny{$\pm$0.0199}}     & \thead{\underline{0.5536} \\ \underline{\tiny{$\pm$0.0438}}} & \thead{0.5683 \\ \tiny{$\pm$0.0390}} & \thead{0.6054 \\ \tiny{$\pm$0.0263}} \\ 

    
%     \thead{G-Prompt \\ w/o SSL} & \thead{0.5210 \\ \tiny{$\pm$0.0372}} & \thead{0.5793 \\ \tiny{$\pm$0.0168}} & \thead{0.6092 \\ \tiny{$\pm$0.0168}} \
%      & \thead{0.5378 \\ \tiny{$\pm$0.0419}} & \thead{\underline{0.5801} \\ \underline{\tiny{$\pm$0.0269}}} & \thead{\underline{0.6004} \\ \underline{\tiny{$\pm$0.0193}}}     & \thead{0.5494 \\ \tiny{$\pm$0.0502}} & \thead{0.5885 \\ \tiny{$\pm$0.0365}} & \thead{0.6149 \\ \tiny{$\pm$0.0263}} \\ 
   
%     \bottomrule
%   \end{tabular}
%   }
% \end{table}


% \begin{figure*}[ht]
% 	\centering
% 	\includegraphics[width=0.7\textwidth]{./picture/exp.pdf}
% 	\caption{\hxw{Comparsion}\zzs{Comparison} with different GNN backbone on 50-shots setting}
% 	\label{fig:exp}
% \end{figure*}








% \begin{table}
%    % \fontsize{5}{12}\selectfont
%   \caption{This is the title}
%   \label{table:table}
%   \centering
%   \resizebox{\textwidth}{!}{
%     \begin{tabular}{c|ccc|ccc|ccc}
%       \toprule
%        {\textbf{Dataset}} & \multicolumn{3}{c|}{\textbf{Arxiv}}   & \multicolumn{3}{c|}{\textbf{Insdata}}      & \multicolumn{3}{c}{\textbf{Reddit}}              \\
%       %\cmidrule(r){1-5}
%                  # Shot per class & 10  & 50 &100    & 10  & 50 &100  & 10  & 50 &100   \\
%       \midrule
%       OGB-Feature & 0.4576  \scriptsize{$\pm$0.0.032} & 0.5495 \scriptsize{$\pm$0.0171}& 0.5875 \scriptsize{$\pm$0.0146} & -&-&-&-&-&-\\
%        RoBERTa*+GAE& 0.5016 \scriptsize{$\pm$0.0510} & 0.5608 \scriptsize{$\pm$0.0101} & 0.5810 \scriptsize{$\pm$0.0125} \
%       & 0.5037 \scriptsize{$\pm$0.0684} & 0.5553 \scriptsize{$\pm$0.0472} & 0.5765 \scriptsize{$\pm$0.0282} &\
%       0.5709 \scriptsize{$\pm$0.0282} & 0.6020 \scriptsize{$\pm$0.0141}& 0.6140 \scriptsize{$\pm$0.0130}\\
%       RoBERTa*+GAE\_prompt\\
%       GIANT & 0.5050 \scriptsize{$\pm$0.0308} & 0.5798\scriptsize{$\pm$0.0119} & 0.6081 \scriptsize{$\pm$0.0109} 
%         & 0.5224 \scriptsize{$\pm$0.0308} & 0.5645 \scriptsize{$\pm$0.0284} & 0.5868 \scriptsize{$\pm$0.0242} 
%         & 0.5639 \scriptsize{$\pm$0.0387} & 0.5948 \scriptsize{$\pm$0.0131} & 0.6114 \scriptsize{$\pm$0.0107} 
%        \\
%       GIANT\_Prompt\\ \hline
%       RoBERTa-cls & 0.4697 \scriptsize{$\pm$0.0577} & 0.5414 \scriptsize{$\pm$0.0400} & 0.5869 \scriptsize{$\pm$0.0300} 
%               & 0.5063 \scriptsize{$\pm$0.0163} & 0.5177 \scriptsize{$\pm$0.0279} & 0.5580 \scriptsize{$\pm$0.0375} 
%               & 0.5018 \scriptsize{$\pm$0.0410} & 0.5166 \scriptsize{$\pm$0.0361} & 0.5716 \scriptsize{$\pm$0.0255} 
%     \\
%       RoBERTa-prompt-emb & 0.5117 \scriptsize{$\pm$0.0398} & 0.5631 \scriptsize{$\pm$0.0352} & 0.5865 \scriptsize{$\pm$0.0296} \
%                      & 0.5180 \scriptsize{$\pm$0.0189} & 0.5622 \scriptsize{$\pm$0.0507} & 0.5831 \scriptsize{$\pm$0.0254} \
%                      & 0.5341 \scriptsize{$\pm$0.0493} & 0.5634 \scriptsize{$\pm$0.0398} & 0.6104 \scriptsize{$\pm$0.0274} \
%                      \\
%       RoBERTa-vob & \multirow{2}{*}{0.5201} \raisebox{-0.5ex}{\scriptsize{$\pm$0.0284}} 
%                    & \multirow{2}{*}{0.5784} \raisebox{-0.5ex}{\scriptsize{$\pm$0.0213}} 
%                    & \multirow{2}{*}{0.6085} \raisebox{-0.5ex}{\scriptsize{$\pm$0.0203}} 
%                    & \multirow{2}{*}{0.5290} \raisebox{-0.5ex}{\scriptsize{$\pm$0.0271}} 
%                    & \multirow{2}{*}{0.5618} \raisebox{-0.5ex}{\scriptsize{$\pm$0.0508}} 
%                    & \multirow{2}{*}{0.5825} \raisebox{-0.5ex}{\scriptsize{$\pm$0.0180}} 
%                    & \multirow{2}{*}{0.5287} \raisebox{-0.5ex}{\scriptsize{$\pm$0.0514}} 
%                    & \multirow{2}{*}{0.5613} \raisebox{-0.5ex}{\scriptsize{$\pm$0.0348}} 
%                    & \multirow{2}{*}{0.5935} \raisebox{-0.5ex}{\scriptsize{$\pm$0.0452}} \\
%       & & & & & & & & & \\ % 添加一个空行,使得第六行的内容下移
%       \hline
%       Ours & 0.5232 \scriptsize{$\pm$0.0348} & 0.5909 \scriptsize{$\pm$0.0159} & 0.6240 \scriptsize{$\pm$0.0156} \
%        & 0.5288 \scriptsize{$\pm$0.0335} & 0.5809 \scriptsize{$\pm$0.0487} & 0.6010 \scriptsize{$\pm$0.0123} \
%        & 0.5686 \scriptsize{$\pm$0.0620} & 0.6088 \scriptsize{$\pm$0.0438} & 0.6504 \scriptsize{$\pm$0.0162} \\
%       \bottomrule
%     \end{tabular}
%   }
% \end{table}


\subsection{Zero-shot node classification and interpretability}

The node features generated through GPrompt represent the probability of each possible token for nodes given task-related prompts, where each dimension corresponds to a specific token. 
This probability generation incorporates prior knowledge from PLMs, graph information, and nodes' context.
Two natural questions arise: \textbf{How much knowledge is contained within this word probability? Whether the node feature can help us interpret the downstream task?}
Therefore, we further conduct zero-shot node classification experiments  based on the generated node representation. Meanwhile, we conduct a case study on Instagram.


\textbf{Zero-shot node classification.} Firstly, we select different sets of candidate tokens for each node class. Then, for each class, we sum up the probability of each tokes in the correspondding set as the final prediction results. We employ the AUC-ROC as the evaluation metric to assess the performance of node classification.
For simplicity, ArXiv dataset only selects two categories, ``Artificial Intelligence'' and ``Linguistics and Language''. The other two datasets remain unchanged. 
We select completely random  (denoted as ``Rand.''), bag-of-words (denoted as ``BOW''), RoBERTa-base (denoted as ``LM-B''), and RoBERTa-large  (denoted as ``LM-L'') as baselines, using the same prompts as G-Prompt for PLMs. 
We provide experiment results of G-Prompt based on RoBERTa-base (Ours-B) and RoBERTa-large (Ours-L).

According to the results shown in Table 3. 
(1) The bag-of-words method has almost no predictive ability. 
(2) The PLM through prompts has predictive ability on different tasks (improvement compared to BOW by \textit{avg.} 13\%). 
But there is a performance difference between base and large even with the same prompt due to the sensitivity of language models to prompts \cite{lu2021fantastically}.
(3) Compared to a language model, G-Prompt shows significant performance improvement. 
Specifically, G-Prompt-base improved \textit{avg.} 2.7\% compared to the language model. 
However, it should be noted that the basic predictive ability of the language model and G-Prompt are correlated. 
Specifically, the correlation coefficient between the results of GPrompt-L and LM-L is 0.64, while the correlation coefficient with LM-B is  0.84. 
(4) Moreover, selecting more candidate words through prior knowledge can effectively help G-Prompt improve its zero-shot capability, with an average improvement of \textit{avg.} 4.8\% for the base and \textit{avg.} 5.3\% for the large. However, there is no significant improvement for language models and bag-of-words. Surprisingly, by adding a small number of candidate words, G-Prompt's zero-shot performance is already close to or even sometimes surpasses  supervised training with 100 shots. This result indicates that combining language models and graphs for zero-shot learning on TAG is feasible.

% In order to evaluate the ability of our model in a zero-shot setting, we choose some words highly related to the task based on prior knowledge and use the prediction of the words to get the results. 
% We generate the distribution of each word in the vocabulary of the language model with RoBERTa-large and RoBERTa-small separately. For each sample, the distribution of positive words and negative words is added distinctly and simply compared to their values to determine the label of the sample.
% For Axiv, two categories are selected to make the results compared more clear. 

% We compare our methods with some baselines. Rand. the method gives the results of the prediction randomly. By counting the frequency of word occurrences in the sample, the word bag method predicts the labels according to the frequency of given words. The feature generated by only RoBERTa-large or RoBERTa-small is used by the same way as our model.

% As shown in Table 3, our models have competitive results, showing the importance of text attributes in a TAG compared with other models, showing our method utilizes the text feature better and has sufficient interpretability. By comparing the results with the performance of supervised model on 100 shots, our models achieve comparable performance with supervised baselines, which demonstrate the advantages of our model in zero-shot tasks.

% To show the excellent interpretability of our model further, the top 10 tokens that are most related to the result of prediction and their scores are collected for the task to detect the commercial users in ins. As shown in Table 4, the result of our model is more interpretable than RoBERTa-large. For example,  \textit{premium} represents something given free or at a reduced price with the purchase of a product or service\cite{} and \textit{niche} represents a specific interest or topic that a user's content focuses on. Both of them are highly related to commercial users and assist us to understand the model more deeply.
\textbf{Interpretability.}
The task on Instagram is to determine whether a node is a commercial user. We use the probability corresponding to each token as the prediction value, calculate its corresponding ROC of prediction performance, and then display the top 7 tokens with the highest scores. For comparison, we also show the scores of tokens corresponding to RoBERTa-Large under the same prompt. Overall, the top 7 tokens given by our model have considerably higher ROC scores than RoBERTa-Large resulting in \textit{avg.} 7.0\% improvement. Additionally, our results are intuitive and can even help explain the task, for example, ``premium.'' Based on this result, we search and find that there are ``premium creator subscriptions'' on Instagram, which means ``Users can set their own prices and earn extra cash each month,''\footnote{\url{https://www.pcmag.com/news/instagram-introduces-premium-creator-subscriptions}} and this information is indeed related to commercial activity. Similarly, ``niche'' is also a word related to Instagram business behavior.

\renewcommand{\multirowsetup}{\centering}
\begin{table}[t!]
   % \fontsize{5}{12}\selectfont
  \caption{The performance of different models in zero-shot learning. For each dataset, two categories of words corresponding to the two labels are selected according to the piror knowledge noted as Pos.vocab and Neg.vocab. AUC is used as evaluation metric (mean ± std\%, the best results are bolded and the runner-ups are underlined).}
  \label{table:table_unsurperviesd}
  
  \centering
  \resizebox{0.8\textwidth}{!}{
      \begin{tabular}{c|c|c|cccc|cc|c}
            \toprule
             {\textbf{Dataset}} & 
             {\textbf{Pos. vocab}} &
             {\textbf{Neg. vocab}} &
             {\textbf{Rand.}} &
             {\textbf{BOW}} &
             {\textbf{LM-B}} &
             {\textbf{LM-L}} &
             {\textbf{Ours-B}} &
             {\textbf{Ours-L}} &
             {\textbf{100 shot.}}
             \\
            \midrule
            \midrule
            \multirow{4}{*}{\textbf{Arxiv}} & 
            \thead{\{\textit{intellectual}\}} & 
            \thead{ \{\textit{language}\}}&
           \thead{ 0.5021\\ \scriptsize{(0.0124)}} & 
           \thead{ 0.4994\\ \scriptsize{(0.0000)}} & 
           \thead{ 0.5955\\ \scriptsize{(0.0000)}} &
           \thead{ \underline{0.6747}\\ \underline{\scriptsize{(0.0000)}}} &
           \thead{ 0.5840\\ \scriptsize{(0.0000)}} &
           \thead{ \textbf{0.6765$^*$}\\ \scriptsize{\textbf{(0.0000)}}} &
           \multirow{4}{*}{\thead{0.9040 \\ \scriptsize{(0.0253)}}}\\
           
          \cline{2-9} & 
          \thead{\{\textit{intellectual}, \\ \textit{decision},\\\textit{logic}, ...\}}& 
           \thead{ \{\textit{language}, \\\textit{translation},\\ \textit{speech}, ...\}}&
           \thead{ 0.4988\\ \scriptsize{(0.0139)}} & 
           \thead{ 0.5474\\ \scriptsize{(0.0000)}} &
           \thead{ \underline{0.6284}\\ \underline{\scriptsize{(0.0000)}}}&
           \thead{ 0.6075\\ \scriptsize{(0.0000)}} &
           \thead{ 0.6006\\ \scriptsize{(0.0000)}} &
           \thead{ \textbf{0.7064$^*$}\\ \scriptsize{\textbf{(0.0000)}}} &
           \\
         \hline
         \hline
        
        
        
        
         
            \multirow{4}{*}{\textbf{Instagram}}& 
            \thead{\{\textit{commercial}\}} & 
            \thead{\{\textit{normal}\}}&
           \thead{ 0.5004 \\ \scriptsize{(0.0151)}} & 
           \thead{ 0.5001 \\ \scriptsize{(0.0007)}}&
           \thead{ \textbf{0.5509$^*$} \\ \scriptsize{\textbf{(0.0163)}}}&
           \thead{ 0.5365 \\ \scriptsize{(0.0054)}}&
           \thead{\underline{ 0.5403} \\ \scriptsize{\underline{(0.0078)}}}&
           \thead{ 0.5382 \\ \scriptsize{(0.0095)}}&
        
           \multirow{4}{*}{\thead{0.5690 \\ \scriptsize{(0.0253)}}}\\
           
             \cline{2-9}&
            \thead{\{\textit{commercial},\\ \textit{sponsored}, \\ \textit{brand}, ...\} }& 
            \thead{\{\textit{normal}, \\ \textit{personality},\\ \textit{private}, ...\}}& 
            \thead{ 0.5007 \\ \scriptsize{(0.0131)} } &
            \thead{ 0.5022 \\ \scriptsize{(0.0008)}} & 
            \thead{ 0.5586 \\ \scriptsize{(0.0117)}} & 
            \thead{ 0.5577 \\\scriptsize{(0.0068)}} & 
            \thead{ \textbf{0.5995$^*$} \\ \scriptsize{\textbf{(0.0074)}} }& 
            \thead{ \underline{0.5957} \\\scriptsize{\underline{(0.0081)}} }& \\
            \hline
            \hline
        

            \multirow{4}{*}{\textbf{Reddit}}& 
            \thead{\{\textit{pretty}\}} 
            & \thead{\{\textit{simple}\}}&
           \thead{ 0.5034 \\ \scriptsize{(0.0073)}} &
           \thead{ 0.5053 \\ \scriptsize{(0.0019)}} &
           \thead{ 0.5608 \\ \scriptsize{(0.0050)}} &
           \thead{ 0.5352 \\ \scriptsize{(0.0027)}} &
           \thead{ \underline{0.5630} \\ \underline{\scriptsize{(0.0082)}}} &
           \thead{ \textbf{0.5673$^*$} \\ \scriptsize{\textbf{(0.0070)}}} &
           
           \multirow{4}{*}{\thead{0.5754  \\ \scriptsize{(0.0348)}}}\\
           \cline{2-9} &
            \thead{\{\textit{pretty},\\\textit{hilarious},\\\textit{funny}, ...\}}& 
            \thead{\{\textit{simple},\\ \textit{anonymous},\\ \textit{standard}, ...\}} &
            \thead{0.4990 \\ \scriptsize{(0.0042)} } &
            \thead{ 0.5034 \\ \scriptsize{(0.0017)}} & 
            \thead{0.5604 \\ \scriptsize{(0.0081)} }& 
            \thead{0.5587 \\\scriptsize{(0.0052)}} & 
            \thead{ \underline{0.5674} \\ \scriptsize{\underline{(0.0058)} }}& 
            \thead{\textbf{0.5742$^*$} \\\scriptsize{\textbf{(0.0066)}} }& \\
            \bottomrule
        \end{tabular}}
\end{table}

\begin{table}[t!]
  \caption{Top 7 Tokens related to predicting commercial users on Instagram}
    \label{Table:overview}
  \centering
  \resizebox{0.4\textwidth}{!}{
      \begin{tabular}{cc|cc}
        \toprule
         \multicolumn{2}{c|}{\textbf{RoBERTa-large}}   & \multicolumn{2}{c}{\textbf{G-Prompt}}          \\
         \textbf{Top 7 tokens}&\textbf{ROC}&\textbf{Top 7 tokens}&\textbf{ROC}\\
         \hline
            \textit{{critical}}& 0.546& \textit{{special}}& 0.592\\
        \textit{{convenient}}& 0.542 &\textit{{convenient}}& 0.579\\
        \textit{{terrific}}&0.542&\textit{{premium}}& 0.579\\
        \textit{{banner}}& 0.542 &\textit{{unique}}&  0.577 \\
        \textit{{gateway}}& 0.539 &\textit{{great}}&  0.575 \\
        \textit{{compelling}}& 0.539 &\textit{{pioneer}}& 0.575\\
        \textit{{neat}}& 0.538 &\textit{{niche}}&  0.575 \\
        %\cmidrule(r){1-5}
        \bottomrule
      \end{tabular}}
\end{table}



\section{Related Works}

In this section, we list works on the same topic as ours. \cref{sec:preliminary} contains works on different topics that our explanation depends on, we omit their details for simplicity here.

In our point of view, the research of activation sparsity in MLP modules starts from the discovery of the relation between MLP and knowledge gained during training. \citet{mlp_as_database} first rewrite MLPs in Transformers into an unnormalized attention mechanism where queries are inputs to the MLP block while keys and values are provided by the first and second weight matrices instead of inputs. So MLP blocks are key-value memories. 
\citet{knowledge_neurons} push forward by detecting how each key-value pair is related to each question exploiting activation magnitudes as well as their gradients, and providing a method to surgically manipulate answers for individual questions in Q\&A tasks. These works reorient research attention back to MLPs, which are previously shadowed by self-attention.

Recently, comprehensive experiments conducted by \citet{observation} demonstrate activation sparsity in MLPs is a prevailing phenomenon in various architectures and on various CV and NLP tasks. 
\citet{observation} also eliminate alternative explanations and attribute activation sparsity solely to training dynamics. 
The authors explain the sparsity theoretically with initialization and by calculating gradients, but their explanation is restricted to the last layer and the first step because in later steps the independence between weights and samples required by the explanation is broken. 
They also discover that some activation functions, such as $\tanh$, hinder the sparsity \citep[see][Fig B.3(c)]{observation}, but did not elaborate on it. 
Compared to their explanations, our explanation applies to all layers and large steps, and accounts for the activation functions' critical role in activation sparsity.

Following empirical discoveries by \citet{observation}, \citet{sharpness_aware} show that sharpness-aware (SA) optimization has a stronger bias toward activation sparsity. 
They explain theoretically by calculating gradients and finding that SA optimization imposes in gradients a component toward reducing norms of activations. However, their explanation is still conducted on shallow 2-layer pure MLPs and requires SA optimization, which is not included in standard training practice. Nevertheless, this explanation hints at the role of flatness in the emergence of activation sparsity. Inspired by them, we explain \emph{deep} networks trained by standard SGD or other stochastic trainers by substituting flat minima for SA optimization.

A more recent work by \citet{from_noises} holds a point that sparsity is a resistance to noises. However, noises are manually imposed and not included in standard data augmentations. We substitute gradient noise from SGD or other stochastic optimizers for them.
\citet{large_step} prove sparsity on 2-layer diagonal MLPs and conjecture similar things to happen in more general networks. Both works hint at the relation between noises (Gaussian sample noises and stochastic gradient noises) and activation sparsity, also leading to the flatness bias of stochastic optimization.

\citet{adversarial_of_moe} study the adversarial robustness of Mixture of Experts (MoE) models brought by architecture-imposed sparsity. They inspire us to relate sparsity with adversarial robustness, although we do it reversely. It is the major inspiration for our results.

To sum up, existing discoveries hint at the relation between activation sparsity and noises, flatness and activation functions but they are still restricted to shallow layers, small steps and special training. Inspired by them and filling their gaps, our explanation applies to deep networks and large training steps, and sticks to standard training procedures.

Although not devoting much to explaining the emergence of activation sparsity in CNNs, \citet{exploit_sparsity_in_CNN} boost activation sparsity through Hoyer regularization\citep{hoyer} and a new activation function FATReLU that uses dynamic thresholds between activation and deactivation. They also design algorithms to exploit this sparsity, leading to $\ge 1.75\mathrm{x}$ speedup in CNN's inference. Compared to their sparsity encouragement method that requires well-designed procedures to select thresholds, hyperparameters for our theoretically induced modifications can be easily selected. The discontinuity of FATReLU also bothers training from scratch\citep{exploit_sparsity_in_CNN}, while we recommend applying our modifications from scratch to enjoy better sparsity and additionally smaller \emph{training} costs. Regarding exploitation, we consider it out of the manuscript's scope.
\citet{L1_sparsity} encourages activation sparsity in CNN by explicit $L_1$ regularization. We intend to investigate the emergence of activation sparsity from implicit regularization as demonstrated by \citet{observation}, so we solely rely on implicit regularization boosted by modifications. Nevertheless, our methods are architecturally orthogonal and we believe applying both together can further boost activation sparsity.

There are other works that are not devoted to activation sparsity but are related. \citet{sparse_symbol} formulate, with Shapley value, and prove that there are sparse ``symbols'' as groups of patches that are the only major contributors to the output of any well-trained and masking-robust AIs. They provide a sparsity independent of training dynamics. Their theory focuses on symbols and sparsity in inputs, which is inherently different from ours.

In Primer \citep{primer}, several architectural changes given by architecture searching include a new activation function Squared-ReLU. In this work, we induce a similar squared $\relu$ activation but with the non-zero part shifted left and use it to guide the search for flat minima and gradient/activation sparsity. \cite{primer} demonstrate impressive improvements of Squared-ReLU in both ablation and addition experiments, and our work provides a potential explanation for this improvement.
\section{Limitations and Open Questions}
\label{sec:limitations}
Though we have proposed two effective non-``detect-then-describe'' methods for 3D dense captioning, the captions do not have much diversity because of the limited text annotations, beam search, and self-critical sequence training with the CiDEr reward.
% 
We believe that multi-modal pre-training on 3D vision-language tasks with more training data and the utilization of \textbf{L}arge \textbf{L}anguage \textbf{M}odels(LLM) trained on large corpus would increase the diversity of the generated captions.
% 
Additionally, other reward functions designed for 3D dense captioning will increase the diversity among object descriptions in the same scene.
% 
We will leave these topics for future study.


\section{Conclusions}
\label{sec:conclusion}
%
\whatsnew{
In this work, we decouple the caption generation from caption generation, and propose a set of two transformer-based approaches, namely Vote2Cap-DETR and Vote2Cap-DETR++, for 3D dense captioning.
%
Comparing with the sophisticated and explicit relation modules in conventional ``detect-then-describe'' pipelines, our proposed methods efficiently capture the object-object and object-scene relation through the attention mechanism.
%
The preliminary model, Vote2Cap-DETR, decouples the decoding process to generate captions and box estimations in parallel.
% 
We also propose vote queries for fast convergence, and develop a novel lightweight query-driven caption head for informative caption generation.
% 
In the advanced model, Vote2Cap-DETR++, we further decouple the queries to capture task-specific features for object localization and description generation.
% 
Additionally, we introduce an iterative spatial refinement strategy for vote queries, and insert 3D spatial information for more accurate captions.
%
Extensive experiments on two widely used datasets validate that both the proposed methods surpass prior ``detect-then-describe'' pipelines by a large margin.
}
%%%%%%%%%%%%%%%%%%%%%%%%%%%%%%%%%%%%%%%%%%%%%%%%%%%%%%%%%%%%
\bibliographystyle{plain}
\bibliography{main}
\appendix
\section*{Appendix}\label{sec:appendix}
\renewcommand{\thesubsection}{\Alph{subsection}}

\subsection{LLM Hallucination in the Real World}
During our testing of this app, we discovered a hallucination issue as shown in Figure~\ref{fig:hallucination}. We can observe that when we requested the app to output lines 5-10 from ``test.py'', the output was very peculiar, which raised our alertness. Further communication with the developer and code review confirmed that the issue was indeed caused by hallucination. The app hallucinated when we asked it to perform code execution, generating seemingly correct outputs without actually having the ability to execute code.
\begin{figure}[ht]
	\centering
	%\setlength{\abovecaptionskip}{0pt}
	\setlength{\belowcaptionskip}{0pt}
	\includegraphics[width=1.0\columnwidth]{figures/hallucination.png}
	\caption{LLM hallucination in a real-world app.} 
	\label{fig:hallucination}
	\vspace{-3mm}
\end{figure}


\subsection{Practical Real-World Attacks Against Other App Users}

% asdfasdlfjasdlfja \ref{fig:new_attack1}
% \lipsum[1-5]
Figure~\ref{fig:new_attack1} illustrates an instance of output hijacking in an real-world scenario. The attacker initiates by tampering with the application's source code, compelling the application to generate a specific output. This manipulation disrupts the normal experience of benign users, causing interference and potential harm.
\begin{figure}[ht]
	\centering
	%\setlength{\abovecaptionskip}{0pt}
	\setlength{\belowcaptionskip}{0pt}
	\includegraphics[width=1.0\linewidth]{figures/new_attack1.pdf}
	\caption{Output hijacking attack} 
	\label{fig:new_attack1}
	\vspace{-3mm}
\end{figure}

Figure~\ref{fig:new_attack2} illustrates an instance of OpenAI API Key stealing attack in an real-world scenario. 
The attacker first modifies the app's source code to enable automatic recording of users' OpenAI API keys after they input their keys. When an app user enters their OpenAI API key while using the app, the key is captured by the attacker without the victim being aware of the attack. This poses a significant threat. Additionally, the attacker can also steal other user information such as uploaded files and user prompts.
\begin{figure}[ht]
	\centering
	%\setlength{\abovecaptionskip}{0pt}
	\setlength{\belowcaptionskip}{0pt}
	\includegraphics[width=1.0\columnwidth]{figures/new_attack2.pdf}
	\caption{API key stealing attack} 
	\label{fig:new_attack2}
	\vspace{-3mm}
\end{figure}

Figure~\ref{fig:phishing_attack} illustrates an instance of phishing attack in an real-world scenario. For example, the attacker wants to trick the user to download and open its malware, it modifies the code first. Now here comes an app user, the modified app says every user should enter a secret token first to start using this app. and the secret token can be obtained by downloading the provided files (and actually the file is attacker’s malware). If the user trust the app, he will download the file and try to open it. Thus, the attacker tricks the user into opening its malware.
\begin{figure}[H]
	\centering
	\setlength{\belowcaptionskip}{0pt}
	\includegraphics[width=1.0\columnwidth]{figures/phishing2.pdf}
	\caption{Phishing attack} 
	\label{fig:phishing_attack}
	\vspace{-3mm}
\end{figure}


\subsection{App Search Details}
\label{sec:appendix:C}
\textbf{Vulnerable APIs used in app searching.} 
\begin{itemize}
    \item LangChain: \texttt{create\_csv\_agent, create\_pandas\_dataframe\_agent, create\_spark\_dataframe\_agent, PALChain}.
    \item PandasAI: \texttt{PandasAI}.
    \item LlamaIndex: \texttt{PandasQueryEngine}.
\end{itemize} 
\begin{table}[H]
\caption{Characteristics used to search black-box apps (number contains overlap between each characteristic.)}
\label{tab:cha}
\begin{tabular}{ll}
\toprule
\textbf{Characteristic (keywords)} & \textbf{\#Tested App} \\ \midrule
data analysis           & 16              \\
chat with ...           & 5               \\
csv                     & 6               \\
interperter             & 2               \\
math                    & 1               \\
data science            & 14              \\
langchain               & 5               \\
agent                   & 7               \\ \bottomrule
\end{tabular}
\end{table}

\end{document}