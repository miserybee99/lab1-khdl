\begin{figure}
    \centering
    \subfloat[OvA KL (MI)]{
        \fbox{ \includegraphics[height=0.1\paperheight,width=0.4\linewidth]{figs/mi_categorical.pdf}}
        \label{sfig:mi_categorical}
    }\hfil
    \subfloat[OvA MMD with linear kernel]{
        \fbox{ \includegraphics[height=0.1\paperheight,width=0.4\linewidth]{figs/mmd_categorical.pdf}}
        \label{sfig:mmd_categorical}
    }
    \caption{Clustering of a mixture of 3 Gaussian distributions with MI (left) and a GEMINI (right) using categorical distributions. The MI does not have insights on the data shape because of the model, and clusters points uniformly between the 3 clusters (black dots, red triangles and blue crosses) whereas the MMD is aware of the data shape through its kernel.}
    \label{fig:categorical_boundaries}
\end{figure}

We first took the most simple discriminative clustering model, where each cluster assignment according to the input datum follows a categorical distribution:
\begin{equation*}
    y|\vec{x}=\vec{x}_i \sim \text{Cat}(\theta_i^1,\theta_i^2,\cdots,\theta_i^K).
\end{equation*}
We generated $N=100$ samples from a simple mixture of $K=3$ Gaussian distributions. Each model thus only consists in $NK$ parameters to optimise. This is a simplistic way of describing the most flexible deep neural network. We then maximised on the one hand the OvA KL (MI) and on the other hand the OvA MMD. Both clustering results can be seen in Figure~\ref{fig:categorical_boundaries}. We concluded that without any function, e.g. a neural network, to link the parameters of the conditional distribution with $\vec{x}$, the MI struggles to find the correct decision boundaries. Indeed, the position of $\vec{x}$ in the 2D space plays no role and the decision boundary becomes only relevant with regards to cluster entropy maximisation: a uniform distribution between 3 clusters. However, it plays a major role in the kernel of the MMD-GEMINI thus solving correctly the problem.