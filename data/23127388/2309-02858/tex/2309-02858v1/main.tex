\documentclass[10pt, compsoc, peerreview, journal]{IEEEtran}

\title{Generalised Mutual Information: a Framework for Discriminative Clustering}

\author{Louis~Ohl,Pierre-Alexandre~Mattei, Charles~Bouveyron, Warith~Harchaoui, Micka\"el~Leclercq, Arnaud~Droit, Fr\'ed\'eric~Precioso
\IEEEcompsocitemizethanks{\IEEEcompsocthanksitem L.~Ohl, P-A.~Mattei, C.~Bouveyron and F.~Precioso are with the Universit\'e C\^ote d'Azur, INRIA Maasai team, CNRS
\IEEEcompsocthanksitem M.~Leclercq, A.~Droit and L.~Ohl are with the Universit\'e Laval, CHU de Qu\'ebec Research Centre
\IEEEcompsocthanksitem W.~Harchaoui is with Jellysmack, Artificial Intelligence Lab in Paris}%
\thanks{Manuscript under review}}
\usepackage[utf8]{inputenc} % allow utf-8 input
\usepackage[T1]{fontenc}    % use 8-bit T1 fonts
\usepackage[pdfencoding=auto]{hyperref}       % hyperlinks
\usepackage{url}            % simple URL typesetting
\usepackage{amsfonts}       % blackboard math symbols
\usepackage{nicefrac}       % compact symbols for 1/2, etc.
\usepackage{microtype}      % microtypography
\usepackage{wrapfig}      % microtypography
\usepackage{graphicx}
\usepackage{url}            % simple URL typesetting
\usepackage{booktabs}       % professional-quality tables
\usepackage{microtype}
\usepackage{amsmath}        % math symbols
\interdisplaylinepenalty=2500
\usepackage{amssymb}        % more math symbols
\usepackage{amsthm}         % theorems
\usepackage{mathtools}
\usepackage{xcolor}

\usepackage{multirow}       % multiple rows for a cell
\usepackage{multicol}       % multiple column in text mode

\usepackage[caption=false,font=normalsize,labelfont=sf,textfont=sf]{subfig}      %for multiple subfloats

%\usepackage{geometry}
%\geometry{margin=3cm}

\usepackage[nocompress]{cite}
\renewcommand{\vec}{\pmb}

%COMMANDS FOR SHORT MATHEMATICAL COMPREHENSION
\newcommand{\p}{p_\theta}
\newcommand{\argmax}{\textup{argmax}}
\newcommand{\E}{\mathbb{E}}
\newcommand{\I}{\mathcal{I}}

% THEOREMS
\newtheorem{proposition}{Proposition}[section]
\newtheorem{corollary}{Corollary}[section]

% Je sais que je pourrais utiliser des options, mais le but de ces multiples commandes est de vraiment simplifier la lisibilité des équations
\newcommand{\pdata}{p_\textup{data}(\vec{x})}
\newcommand{\px}{p(\vec{x})}
\newcommand{\ya}{y_a}
\newcommand{\yb}{y_b}
\newcommand{\py}{\p(y)}
\newcommand{\pya}{\p(\ya)}
\newcommand{\pyb}{\p(\yb)}
\newcommand{\pyx}{\p(y|\vec{x})}
\newcommand{\pyxa}{\p(\ya|\vec{x})}
\newcommand{\pyxb}{\p(\yb|\vec{x})}
\newcommand{\pxy}{\p(\vec{x}|y)}
\newcommand{\pxya}{\p(\vec{x}|\ya)}
\newcommand{\pxyb}{\p(\vec{x}|\yb)}

\newcommand{\pykx}{p(y=k|\vec{x})}\newcommand{\pxyk}{p(\vec{x}|y=k)}

% commands specific to mmd demonstration
\newcommand{\xa}{\vec{x}_a}
\newcommand{\xaP}{\vec{x}_a^\prime}
\newcommand{\xb}{\vec{x}_b}
\newcommand{\xbP}{\vec{x}_b^\prime}

\newcommand{\PX}{p(\vec{x})}
\newcommand{\PY}{p(y)}
\newcommand{\PYA}{p(y_1)}
\newcommand{\PYB}{p(y_2)}
\newcommand{\PXY}{p(\vec{x}|y)}
\newcommand{\PXYA}{p(\vec{x}|y_1)}
\newcommand{\PXYB}{p(\vec{x}|y_2)}
\newcommand{\PYX}{p(y|\vec{x})}
\newcommand{\PYXA}{p(y_1|\vec{x})}
\newcommand{\PYXB}{p(y_2|\vec{x})}
\newcommand{\Pjoint}{p(\vec{x},y)}

\begin{document}
\IEEEtitleabstractindextext{%
\begin{abstract}
% 
3D dense captioning requires a model to translate its understanding of an input 3D scene into several captions associated with different object regions.
% 
Existing methods adopt a sophisticated “detect-then-describe” pipeline, which builds explicit relation modules upon a 3D detector with numerous hand-crafted components.
% 
While these methods have achieved initial success, the cascade pipeline tends to accumulate errors because of duplicated and inaccurate box estimations and messy 3D scenes.
% 
In this paper, we first propose Vote2Cap-DETR, a simple-yet-effective transformer framework that decouples the decoding process of caption generation and object localization through parallel decoding.
% 
% We show that the sophisticated and explicit relation reasoning modules can be replaced by the attention mechanism to capture both object-object and object-scene relations.
% 
\whatsnew{
Moreover, we argue that object localization and description generation require different levels of scene understanding, which could be challenging for a shared set of queries to capture.
% 
To this end, we propose an advanced version, Vote2Cap-DETR++, which decouples the queries into localization and caption queries to capture task-specific features.
% 
Additionally, we introduce the iterative spatial refinement strategy to vote queries for faster convergence and better localization performance.
% 
We also insert additional spatial information to the caption head for more accurate descriptions.
% 
Without bells and whistles, extensive experiments on two commonly used datasets, ScanRefer and Nr3D, demonstrate Vote2Cap-DETR and Vote2Cap-DETR++ surpass conventional ``detect-then-describe'' methods by a large margin.
}
% 
Codes will be made available at \href{https://github.com/ch3cook-fdu/Vote2Cap-DETR}{https://github.com/ch3cook-fdu/Vote2Cap-DETR}.
\end{abstract}

\begin{IEEEkeywords}
Multi-modal Learning, 3D Scene Understanding, 3D Dense Captioning, Transformers.
\end{IEEEkeywords}

\begin{IEEEkeywords}
Clustering, discriminative clustering, unsupervised learning, information theory, mutual information, machine learning
\end{IEEEkeywords}}
\maketitle
\IEEEpeerreviewmaketitle

%------------------------------------------------------------------ MAIN MATTER

\IEEEraisesectionheading{\section{Introduction}
\label{sec:introduction}}
Navigating in unknown 3D environments is a crucial task for mobile robots.
Before being able to move, a robot must represent the environment in which it evolves.
A well-known and efficient way to map the environment is the occupancy grids introduced by \citet{elfes1989using}.
An occupancy grid provides the robot with information about the potential presence of an obstacle at a given position.
However, this information alone does not encompass all the capabilities of the robot to maneuver within its environment. 
For example, a wheeled robot can safely cross grasses, a curb, or a speed bump at low speed, as shown in \autoref{fig:intro}.
Therefore, a way to assess their hazardous nature is necessary to evolve in these environments.
Motivated by this fact, numerous recent works \cite{shi2023gridcentric} have contributed to risk-aware navigation.
%
However, \citet{laconte2019lambda} demonstrated that the Bayesian occupancy grid, which stores the probability of collision of a given position, is ill-suited to compute the risk over paths.
Indeed, the probability of collision, computed as the joint probability that every cell is free of obstacles, is highly dependent on the grid tessellation size.
To overcome this difficulty, they developed a novel framework called Lambda-Field to assess physics-based risks on occupancy grids.
\begin{figure}[t]
  \centering
  \includegraphics[width=\linewidth]{media/IMG_0217_rt.jpg}
  \caption{Example of a situation that a vehicle might encounter while navigating.
    Speed bumps are frequent obstacles an intelligent vehicle has to safely overcome.
    With our framework, the vehicle is able to make more effective decisions based on the level of risk it is allowed to take.
}
  \label{fig:intro}
  \vskip-.5em
\end{figure}
Moreover, in occupancy grids framework, most navigation methods \cite{PATLE2019582} use geometric and semantic reasoning to handle the obstacles present in the environment, such as curbs, traffic cones, speed bump, sidewalk, buildings, traffic circle and traffic lights shown in \autoref{fig:intro}.
\citet{laconte2019lambda} showed that path planning becomes more intuitive and meaningful when a physical risk metric is used.
However, their work has been developed solely for 2D environments.
%
Here, we adapt their work to 3D environments and use physical, risk-based reasoning to deal with the obstacles represented in \autoref{fig:intro}.

The main contributions of this paper are i) an extension of the Lambda-Field \cite{laconte2019lambda} to 3D environments; ii) a generalization of the risk that take into account traversable obstacles; iii) a mathematical formulation of a local risk-aware path planning algorithm in the Lambda-Field; iv) a demonstration of the applicability of the method, using data acquired in real urban environment.

%
%
%
%
%
%


\section{Is MI a good clustering objective?}
\label{sec:about_mi}
We consider in this section a dataset consisting in $N$ unlabelled samples $\mathcal{D}=\{\vec{x}_i\}_{i=1}^N$. We distinguish two major use cases of the mutual information: one where we measure the dependence between two continuous variables, as is the case in representation learning, and one where the random variable is discrete. In representation learning, the goal is to construct a continuous representation $\vec{z}$ extracted from the data $\vec{x}$ using a learnable distribution of parameters $\theta$. In clustering, samples $\vec{x}$ are assigned to the discrete variable $y$ through another learnable distribution.


%Each sample is associated to a cluster $y$ using a learnable distribution $\pyx$ of parameters $\theta$. The variable $y$ is discrete. For continuous case, e.g. representation learning, we write representations $\vec{z} \sim p_\theta(\vec{z}|\vec{x})$.
%\begin{equation*}\label{eq:mi_representation_learning}
%    \mathcal{I}(\vec{x},\vec{z})=\int\int  \log{\frac{p(\vec{x},\vec{z})}{p(\vec{x})p(\vec{z})}}d\vec{x}d\vec{z}\quad,
%\end{equation*}
%\begin{equation*}\label{eq:mi_clustering}
%    \mathcal{I}(\vec{x},y) = \sum_{i=1}^K \int p(y_i,\vec{x})\log{\frac{p(y_i,\vec{x})}{p(y_i)p(\vec{x})}}d\vec{x}
%\end{equation*}

\subsection{Representation learning}
\label{ssec:mi_representation}

Representation learning consists in finding high-level features $\vec{z}_i$ extracted from the data $\vec{x}_i$ in order to perform a \emph{downstream task}, e.g. clustering or classification. MI between $\vec{x}$ and $\vec{z}$ is a common choice for learning features\cite{hjelm_learning_2019}. However, estimating correctly MI between two random variables in continuous domains is often intractable when $p(\vec{x}|\vec{z})$ or $p(\vec{z}|\vec{x})$ is unknown, thus lower bounds are preferred, e.g. variational estimators such as MINE~\cite{belghazi_mine_2018}, $\I_\text{NCE}$\cite{van_den_oord_representation_2018},$\I_\text{BA}$\cite{barber_im_2003},$\I_\text{DV}$\cite{donsker_asymptotic_1983}. These bounds require more parameters: additional discriminator networks are also trained to make a distinction between data and features issued from the joint distribution or product of marginals. Most of these lower bounds often present high-variance such as $\I_\text{NJW}$\cite{nguyen_estimating_2010}. Poole et al.~\cite{poole_variational_2019} eventually bridged the gap between this high-variance estimators and the high-bias low-variance $\I_\text{NCE}$ by introducing $\I_\alpha$, an interpolated lower bound. Another common choice of loss function to train features are contrastive losses such as NT-XENT~\cite{chen_simple_2020} where the similarity between the features $\vec{z}_i$ from data $\vec{x}_i$ is maximised with the features $\tilde{\vec{z}}$ from a data-augmented $\tilde{\vec{x}}_i$ against any other features $\vec{z}_j$. Recently, Do et al.~\cite{do_clustering_2021} achieved excellent performances in single-stage methods by highlighting the link between the $\I_\text{NCE}$ estimator~\cite{van_den_oord_representation_2018} and contrastive learning losses. Representation learning therefore comes at the cost of a complex lower bound estimator on MI, which often requires data augmentation.
%While this often improves the performance of deep learning models, it can be a pitfall as it may lead to data overfitting in classification. Overall, data augmentation needs to be carefully designed when dealing with data where augmentations have a chance to change the class of the samples, such as medical data. {\bf TODO ref manquante} Thus, clustering tasks are even riskier when clusters are not known a priori and data augmentations can change the assignments.
Moreover, it was noticed that the MI is hardly predictive of downstream tasks~\cite{tschannen_mutual_2019} when the variable $y$ is continuous, i.e. a high value of MI does not clarify whether the discovered representations are insightful with regards to the target of the downstream task.

\subsection{Discriminative clustering}
\label{sssec:mi_clustering}

The MI has been first used as an objective for learning discriminative clustering models~\cite{bridle_unsupervised_1992}. Associated architectures went from simple logistic regression~\cite{krause_discriminative_2010} to deeper architectures~\cite{hu_learning_2017,ji_invariant_2019}. Beyond architecture improvement, the MI maximisation was also carried with several regularisations. These regularisations include penalty terms such as weight decay~\cite{krause_discriminative_2010} or Virtual Adversarial Training (VAT)~\cite{hu_learning_2017,miyato_virtual_2018}). Data augmentation was further used to provide invariances in clustering, as well as specific architecture designs like auxiliary clustering heads~\cite{ji_invariant_2019}. Rewriting the MI in terms of entropies:
\begin{equation}
\label{eq:mi_kl_entropies}
\I (\vec{x};y) = \mathcal{H}(y) - \mathcal{H}(y|\vec{x})
\end{equation}
highlights a requirement for balanced clusters, through the cluster entropy term $\mathcal{H}(y)$. Indeed, a uniform distribution maximises the entropy. This hints that an unregularised discrete mutual information for clustering can possibly produce uniformly distributed clusters among samples, regardless of how close they could be. We highlight this claim in section~\ref{ssec:local_maxima}. As an example of regularisation impact: maximising the MI with $\ell_2$ constraint can be equivalent to a soft and regularised K-Means in a feature space~\cite{jabi_deep_2019}. In clustering, the number of clusters to find is usually not known in advance. Therefore, an interesting clustering algorithm should be able to find a relevant number of clusters, i.e. perform model selection. However, model selection for parametric deep clustering models is expensive~\cite{ronen_deepdpm_2022}. Cluster selection through MI maximisation has been little studied in related works, since experiments usually tasked models to find the (supervised) classes of datasets. Furthermore, the literature diverged towards deep learning methods focusing mainly on images, yet rarely on other type of data such as tabular data~\cite{min_survey_2018}.

\subsection{Maximising the MI can lead to bad decision boundaries}
\label{ssec:local_maxima}

Maximising the MI directly can be a poor objective: a high MI value is not necessarily predictive of the quality of the features regarding downstream tasks~\cite{tschannen_mutual_2019} when $y$ is continuous. We support a similar argument for the case where the data $\vec{x}$ is a continuous random variable and the cluster assignment $y$ a categorical variable. Indeed, the MI can be maximised by setting appropriately a sharp decision boundary which partitions evenly the data, i.e. when the distribution $\pyx$ converges to a Dirac distribution. This reasoning can be seen in the entropy-based formulation of the MI (Eq.~\ref{eq:mi_kl_entropies}): any sharp decision boundary minimises the negative conditional entropy, while ensuring balanced clusters maximises the entropy of cluster proportions. Consider for example Figure~\ref{fig:example_good_odd_mi}, where a mixture of Gaussian distributions with equal variances is separated by a sharp decision boundary. We highlight that both models will have the same mutual information on condition that the misplaced decision boundary of Figure~\ref{sfig:odd_decision_boundary} splits evenly the dataset (see Appendix~\ref{app:mi_convergence}).

\begin{figure}
    \centering
    \subfloat[Good decision boundary]{
        \includegraphics[width=0.45\linewidth]{figs/good_boundary.pdf}
        \label{sfig:good_decision_boundary}
    }\hfil
    \subfloat[Misplaced boundary]{
        \includegraphics[width=0.45\linewidth]{figs/odd_decision_boundary.pdf}
        \label{sfig:odd_decision_boundary}
    }
    \caption{Example of maximised MI for a Gaussian mixture $\frac{1}{2} \mathcal{N}(\mu_0, \sigma^2)+\frac{1}{2}\mathcal{N}(\mu_1,\sigma^2)$. It is clear that figure \ref{sfig:good_decision_boundary} presents the best decision boundary and posterior between the two Gaussian distributions. Yet, as the posterior turns sharper, the difference between both MIs converges to 0.}
    \label{fig:example_good_odd_mi}
\end{figure}

Globally, MI misses the idea in clustering that any two points close to one another may be in the same cluster according to some chosen distance. Hence regularisations are required to ensure this constraint. An early sketch of these insights was mentioned by Bridle et al.~\cite{bridle_unsupervised_1992} or Corduneanu et al.~\cite{corduneanu_information_2012}. The non-predictiveness of MI was as well recently empirically highlighted by Zhang and Boykov~\cite{zhang_revisiting_2023} in discrete cases. This can be also be seen as a problem of invariance of the conditional distribution in low density areas~\cite{corduneanu_information_2012}.

\section{Extending the MI to the GEMINI}
\label{sec:gemini}
Given the identified limitations of MI, we now describe the discriminative clustering framework based on the expected distance equation of the mutual information. We then detail the different statistical distances we can use to extend MI to the Generalised Mutual Information (GEMINI).

\subsection{The discriminative clustering framework for GEMINIs}
\label{ssec:discriminative_clustering}

We only consider two random variables: the data $\vec{x}$ which can be continuous or discrete and the cluster assignment $y$ which is discrete. Instead of viewing the mutual information as a dependence-seeking objective (Eq.~\ref{eq:mutual_information_1}), we view it as a clustering objective that aims at separating the data distribution given cluster assignments $p(\vec{x}|y)$ from the data distribution $p(\vec{x})$ according to the KL divergence:
\begin{equation}\label{eq:base_mi}
    \I(\vec{x};y) = \E_{y\sim p(y)} \left[ D_\text{KL}(p(\vec{x}|y)\|p(\vec{x}))\right].
\end{equation}
To highlight the discriminative clustering design, we explicitly do not set any hypothesis on the data distribution by writing $\pdata$. The only part of the model that we design is a conditional distribution $\pyx$ that assigns a cluster $y$ to a sample $\vec{x}$ using the parameters $\theta$ of some learnable function $\psi$~\cite{minka_discriminative_2005}:
\begin{equation}\label{eq:conditional_model}
y|\vec{x} \sim \text{Categorical}(\psi_\theta(\vec{x})).
\end{equation}
This learnable function $\psi_\theta$ can typically be a neural network of adequate design regarding the data, e.g. a CNN, or a logistic regression. Consequently, the cluster proportions are controlled by $\theta$ because $\py=\mathbb{E}[\pyx]$ and so is the conditional distribution $\pxy$ even though intractable because we cannot compute the data distribution:

\begin{equation}
    \p(\vec{x},y) = \pdata \pyx.
\end{equation}

This questions how Eq. (\ref{eq:base_mi}) can be computed. Fortunately, well-known properties of MI can invert the distributions on which the KL divergence is computed~\cite{bridle_unsupervised_1992,krause_discriminative_2010} via Bayes' theorem:%The intractability of the likelihood $\pxy$ questions how Eq. (\ref{eq:base_mi}) can be computed. Fortunately, well-known properties of MI can invert the distributions between which the KL divergence is performed~\cite{bridle_unsupervised_1992,krause_discriminative_2010} via Bayes' theorem:
\begin{equation} \label{eq:tractable_discriminative_mi}
    \I(\vec{x};y) = \E_{\vec{x} \sim \pdata} \left[ D_\text{KL} (\pyx \| \py)\right],
\end{equation}
which is possible to estimate. Since we highlighted earlier that the KL divergence in the MI can lead to inappropriate decision boundaries, we are interested in replacing it by other distances or divergences. However, changing it in Eq. (\ref{eq:tractable_discriminative_mi}) would focus on the separation of cluster assignments from the cluster proportions which may be irrelevant to the data distribution. We rather alter Eq. (\ref{eq:base_mi}) to clearly show that we separate data distributions given clusters from the entire data distribution because it allows us to take into account the data space geometry.

\subsection{The GEMINI}
\label{ssec:gemini}

\subsubsection{Replacing the Kullback-Leibler divergence with other distances}

The goal of the GEMINI is to separate data distributions according to an arbitrary distance $D$, i.e. changing the KL divergence for another divergence or distance in the MI. This brings the definition of our first GEMINI, the \emph{One-vs-All} (OvA):
\begin{equation}\label{eq:gemini_ova}
    \I^\text{OvA}_D(\vec{x};y) = \E_{y \sim \py} \left[ D(\pxy\|\pdata)\right],
\end{equation}
as it compares the distance between the distribution of a specific cluster $\pxy$ against the entire data distribution $p(x)$.  There exist other distances than the KL to measure how far two distributions $p$ and $q$ are one from the other. We can make a clear distinction between two types of distances, Csiszar's $f$-divergences~\cite{csiszar_information-type_1967} and Integral Probability Metrics (IPM)~\cite{sriperumbudur_integral_2009}. Unlike $f$-divergences, IPM-derived distances like the Wasserstein distance or the Maximum Mean Discrepancy (MMD)~\cite{gneiting_strictly_2007,gretton_kernel_2012} bring knowledge about the data throughout either a distance $c$ or a kernel $\kappa$: these distances are geometry-aware. %We now review different statistical distances to consider in the GEMINI framework.



\subsubsection{\texorpdfstring{$f$}{f}-divergence GEMINIs} These divergences involve a convex function $f:\mathbb{R}^+\rightarrow\mathbb{R}$ such that $f(1)=0$. This function is applied to evaluate the ratio between two distributions $p$ and $q$, as in Eq.~(\ref{eq:f_divergences_definition}):
\begin{equation}
\label{eq:f_divergences_definition}
D_\text{f-div}(p,q) = \E_{\vec{z} \sim q(\vec{z})} \left[ f\left(\frac{p(\vec{z})}{q(\vec{z})}\right)\right].
\end{equation}
We will focus on three $f$-divergences: the KL divergence, the Total Variation (TV) distance and the squared Hellinger distance. While the KL divergence is the usual divergence for the MI, the TV and the squared Hellinger distance present different advantages among $f$-divergences. First of all, both of them are bounded between 0 and 1. It is consequently easy to check when any GEMINI using those is maximised contrarily to the MI that is bounded by the minimum of the entropies of $\vec{x}$ and $y$~\cite{gray_maximum_1977}. When used as distance between data conditional distribution $\pxy$ and data distribution $\pdata$, we can apply Bayes' theorem in order to get an estimable equation to maximise (see App.~\ref{app:deriving_geminis}), which only involves cluster assignment $\pyx$ and marginals $\py$ as summarised in Table~\ref{tab:all_geminis}, generalising thus the work of Bridle et al.~\cite{bridle_unsupervised_1992}. Note that all $f$-divergences are maximised when the two distributions $p$ and $q$ have disjoint supports~\cite{liese_divergences_2011}. Common $f$-divergence like the KL, the squared Hellinger or the Pearson $\chi^2$ divergence, except the total variation distance, are specific cases of the $\alpha$-divergence subclass. The convex function of $\alpha$-divergence is parameterized by a real number $\alpha$ with:

\begin{equation}\label{eq:alpha_divergence_function}
    f_\alpha(t) = \left\{\begin{array}{cc}
        \frac{t^\alpha-\alpha t+(\alpha-1)}{\alpha(\alpha-1)}, &  \alpha\neq0, \alpha\neq1,\\
        t\ln{t}, &\alpha =1,\\
        -\ln{t},&\alpha=0.
    \end{array}\right.
\end{equation}

However, this class of $\alpha$-divergence is inappropriate in some cases for clustering. Indeed, we show with Proposition~\ref{prop:alpha_div_maximisation} (proof in App.~\ref{app:proof_alpha_div_maximisation}) that the maximisation of $\alpha$-divergences can lead to any clustering of the data space with balanced clusters as the discriminative model $\pyx$ converges to a Dirac distribution.

\begin{proposition}\label{prop:alpha_div_maximisation}
Let $\{\mathcal{X}_k\}_{k=1}^K$ a partition of $\mathcal{X}$ such that $\mathbb{P}(\vec{x} \in \mathcal{X}_k) = \frac{1}{K}$. Then for any $\alpha$-divergence with $\alpha>0$, the OvA GEMINI is upper bounded by a function which only depends on the proportions of the clusters. If the clustering model follows a Dirac distribution: $\p(y=k|\pmb{x})=\mathbf{1}_{[\vec{x}\in\mathcal{X}_k]}$, then the upper bound is tight and the GEMINI cannot be improved.
%Let $D$ an $\alpha$-divergence and $p(y|\vec{x})$ a clustering model with $\vec{x}$ a continuous variable and $y$ and discrete variable taking $K$ values. If $\alpha>0$, then any Dirac model $p(y=k|\vec{x})=\pmb{1}_{[\vec{x}\in\mathcal{X}_k]}$ with the data space $\mathcal{X}$ split into supplementary disjoint spaces $\{\mathcal{X}_k\}_{k=1}^K$ maximises the OvA GEMINI $\I^\text{ova}_{D_\alpha}(\vec{x};y)$. The global maximum is reached when $p(y=k)=K^{-1},\forall k$.
\end{proposition}

It is worth mentioning in Proposition~\ref{prop:alpha_div_maximisation} that the proportions of the cluster $p(y=k)$ do not matter for the specific case of $\alpha=2$ to achieve the global maximum, i.e. for the Pearson $\chi^2$-divergence. We can infer from Proposition~\ref{prop:alpha_div_maximisation} the specific Corollary~\ref{cor:mi_maximisation} since the MI is a case of OvA $\alpha$-divergence-GEMINI with $\alpha=1$. We conclude that MI maximisation is a poor objective when a discriminative model can converge to a Dirac distribution.

\begin{corollary}\label{cor:mi_maximisation}
%Any Dirac model $p(y=k|\vec{x})=\pmb{1}_{[\vec{x}\in\mathcal{X}_k]}$ on partition $\{\mathcal{X}_k\}_{k=1}^K$ of the data space $\mathcal{X}$ with $p(\vec{x}\in\mathcal{X}_k)=\frac{1}{K}$ maximises the mutual information.
Let $\{\mathcal{X}_k\}_{k=1}^K$ a partition of $\mathcal{X}$. Then the mutual information of a discriminative distribution $p(y|\vec{x})$ is upper bounded by the entropy of $\vec{x}$ and the upper bound is tight if the distribution is a Dirac model $p(y=k|\vec{x})=\pmb{1}_{[\vec{x}\in\mathcal{X}_k]}$. The highest upper bound is reached when the partition is balanced.
\end{corollary}

\subsubsection{IPM GEMINIs} The IPM is another class of distance that incorporates knowledge from the data through a function $f$:

\begin{equation}\label{eq:ipm}
    D_\text{ipm}(p,q) = \sup_{f\in\mathcal{F}} \E_{\vec{z} \sim p(\vec{z})}[f(\vec{z})] -\E_{\vec{z}\sim q(\vec{z})}[f(\vec{z})],
\end{equation}
where $\mathcal{F}$ is a set of functions. As backpropagation through suprema could be intractable, we choose to focus on two specific variations of the IPM for the GEMINI: the MMD and the Wasserstein distance. Note however that not all Wasserstein distances are IPMs and while some of our propositions are formulated for IPMs, we consider as well the entire class of the Wasserstein distances.

The MMD corresponds to the distance between the respective expected embedding of samples from the distribution $p$ and the distribution $q$ in a reproducible kernel Hilbert space (RKHS) $\mathcal{H}$:
\begin{equation}\label{eq:mmd_definition}
    \text{MMD}(p\|q) = \| \E_{\vec{z} \sim p(\vec{z})} [\varphi(\vec{z})] - \E_{\vec{z}\sim q(\vec{z})} [\varphi(\vec{z})]\|_\mathcal{H},
\end{equation}
where $\varphi$ is the RKHS embedding. To compute this distance we can use the kernel trick~\cite{gretton_kernel_2012} by involving the kernel function $\kappa(\vec{a},\vec{b})=\langle\varphi(\vec{a}),\varphi(\vec{b})\rangle$. We use Bayes' theorem to uncover a version of the MMD that can be estimated through Monte Carlo using only the predictions $\pyx$.

The Wasserstein distance is an optimal transport distance. It corresponds to the minimal amount of energy to transform a distribution into another according to an energy function yielding the cost $c$ of moving the mass of a sample from one location to another:
\begin{equation}\label{eq:wasserstein_definition}
    \mathcal{W}_c^d(p,q) = \left(\inf_{\gamma \in \Gamma(p,q)} \E_{\vec{x}, \vec{z} \sim \gamma(\vec{x},\vec{z})}\left[c(\vec{x},\vec{z})^d \right]\right)^{\frac{1}{d}},
\end{equation}
where $\Gamma(p,q)$ is the set of all couplings between $p$ and $q$, $c$ a distance function in $\mathcal{X}$ and $d$ a real positive number. Computing the Wasserstein-$d$ distance between two distributions $\p(\vec{x}|y=k)$ and $\p(\vec{x})$ is difficult in our discriminative context because we only have access to a finite set of samples $N$. Note that in the remainder of the paper, we will focus on the Wasserstein-1 metric. The idea of an expected Wasserstein distance was first proposed by Harchaoui~\cite[Eq. 48]{harchaoui_2020_learning} under the one-vs-rest name with an additional cluster proportion factor. However, we found this additional factor to be not grounded enough. Moreover, we can show that for the Wasserstein-1 metric the one-vs-rest Wasserstein preliminary work~\cite{harchaoui_2020_learning} is equivalent to the one-vs-all Wasserstein-GEMINI. To achieve the Wasserstein-GEMINI, we instead use approximations of the distributions with weighted sums of Diracs:

\begin{multline}\label{eq:dirac_approximation}
\p(\vec{x}|y=k) \approx \sum_{i=1}^N m_i^k \delta_{\vec{x}_i} = p_N^k,\\\text{with}\quad m_i^k = \frac{\p(y=k|\vec{x}_i)}{\sum_{j=1}^N\p(y=k|\vec{x}_j)},
\end{multline}
where $\delta_{\vec{x}_i}$ is a Dirac located on sample location $\vec{x}_i\in\mathcal{X}$. For the distribution $\pdata$, we set all importance weights to $1/N$. We state in Prop.~\ref{prop:wasserstein_convergence} that this empirical estimate of the Wasserstein distance converges to the correct Wasserstein distance. These importance weights are compatible with the \verb+emd2+ function of the python optimal transport package~\cite{flamary_pot_2021} which gracefully supports automatic differentiation. We describe other maximisation strategies in Appendix~\ref{app:other_wasserstein_distances}.

\begin{proposition}\label{prop:wasserstein_convergence}
    Let $p(\vec{x}|y=k_1)$ and $p(\vec{x}|y=k_2)$ two cluster distributions that we empirically approximate with importance-weighed Dirac estimators $p_N^{k_1}= \sum_{i=1}^N m_i^{k_1}\delta_{\vec{x}_i}$, resp. and  $p_N^{k_2}= \sum_{i=1}^N m_i^{k_2}\delta_{\vec{x}_i}$. Then the Wasserstein distance between estimates converges to the Wasserstein distance between the cluster distributions.
\end{proposition}

We refer to Appendix~\ref{app:wasserstein_convergence} for proof of convergence.

\subsection{The One-vs-One GEMINI}

We question the relevance of evaluating a distance between the distribution of the data given a cluster assumption $\pxy$ and the entire data distribution $\pdata$ when the geometry is taken into account. We argue that it is intuitive in clustering to compare the distribution of one cluster against the distribution of \emph{another cluster} rather than the data distribution. Indeed, considering the geometry of the data space through a kernel in the case of the MMD or a distance in the case of the Wasserstein metric implies that we can effectively measure how two distributions are close to one another. In the formal design of the mutual information, the distribution of each cluster $p(\vec{x}|y)$ is compared to the complete data distribution $p(\vec{x})$. Therefore, if one distribution of a specific cluster $p(\vec{x}|y)$ were to look alike the data distribution $p(\vec{x})$, for example up to a constant in some areas of the space, then its distance to the data distribution could be 0, making it unnoticed when maximising the OvA GEMINI.

\begin{figure}
\centering
    \subfloat[OvA]{
        \includegraphics[width=0.45\linewidth]{figs/3_clusters_ova.pdf}
        \label{sfig:example_ova}
    }\hfil
    \subfloat[OvO]{
        \includegraphics[width=0.45\linewidth]{figs/3_clusters_ovo.pdf}
        \label{sfig:example_ovo}
    }
\caption{Here, 3 clusters of equal proportions from isotropic Gaussian distributions are located in -2, 0 and 2 on the x-axis, with small covariance. The complete data distribution hence has its expectation in 0 on the x-axis. Consequently, maximising the OvA MMD-GEMINI with a logistic regression led to 2 clusters whereas the same model with the OvO MMD-GEMINI is able to see all 3 clusters.}\label{fig:example_ova_ovo}

\end{figure}

Take the example of 3 distributions $\{p(\vec{x}|y=i)\}_{i=1}^3$ with respective different expectations $\{\mu_i\}_{i=1}^3$. We want to separate them using the OvA MMD-GEMINI with linear kernel. The mixture of the 3 distributions creates a data distribution with expectation $\mu=\sum_{i=1}^3 p(y=i)\mu_i$. However, if the distributions satisfy that this data expectation $\mu$ is equal to one of the sub-expectations $\mu_i$, then the associated distribution $i$ will have will not provide any information since its MMD to the data distribution is equal to 0. We illustrate this example in figure~\ref{fig:example_ova_ovo}.

To address this issue, we introduce the second GEMINI named \emph{one-vs-one} (OvO) in which we compare cluster distributions from independently drawn cluster assignments $\ya$ and $\yb$:
\begin{equation}\label{eq:gemini_ovo}
    \I^\text{OvO}_D(\vec{x};y) = \E_{\ya,\yb \sim \py} \left[ D(\pxya \| \pxyb)\right].
\end{equation}

The example of Figure~\ref{fig:example_ova_ovo} is tackled by the OvO GEMINI since the distance between each pair of the 3 clusters is non-null. Conceptually, the idea of optimising the OvO MMD-GEMINI in clustering can be found as well by França et al.~\cite{franca_kernel_2020} who derived a regularised squared MMD in a one-vs-one setting through restrictions to Dirac distributions. Note that for most distances, the OvO GEMINI is an upper bound of the OvA GEMINI; proof of Proposition~\ref{prop:ovo_greater_ova} in App.~\ref{app:proof_ovo_greater_ova}.

\begin{proposition}\label{prop:ovo_greater_ova}
Let $D$ be an $f$-divergence or an IPM and $\p(y|\vec{x})$ a clustering distribution. Then: $\mathcal{I}^\text{ova}_D(\vec{x},y) \leq \mathcal{I}^\text{ovo}_D(\vec{x},y)$.
%Let $D$ an $f$-divergence or an IPM and $p(\vec{x}|y)$ a distribution with $\vec{x}$ a continuous variable and $y$ and discrete variable taking $K$ values. Then: $\mathcal{I}^\text{ova}_D(\vec{x},y) \leq \mathcal{I}^\text{ovo}_D(\vec{x},y)$.
\end{proposition}

In the case of binary clustering, using an IPM distance implies equality between the OvA GEMINI and the OvO GEMINI; proof of Proposition~\ref{prop:equality_ova_ovo_ipm} in App.~\ref{app:proof_equality_ova_ovo_ipm}:

\begin{proposition}\label{prop:equality_ova_ovo_ipm}
Let $D$ be an IPM and $p(y|\vec{x})$ a clustering distribution $y$ taking $K=2$ values. Then: $\mathcal{I}^\text{ova}_D(\vec{x},y) = \mathcal{I}^\text{ovo}_D(\vec{x},y)$.
%Let $D$ an IPM and $p(\vec{x}|y)$ a distribution with $\vec{x}$ a continuous variable and $y$ and discrete variable taking $2$ values. Then: $\mathcal{I}^\text{ova}_D(\vec{x},y) = \mathcal{I}^\text{ovo}_D(\vec{x},y)$.
\end{proposition}

While the OvA GEMINI is maximised with Dirac clustering of the data space for some $\alpha$-divergence, we can extend Proposition~\ref{prop:alpha_div_maximisation} to all $f$-divergences for the OvO GEMINI with Proposition~\ref{prop:ovo_fdiv_maximisation} (Proof in App.~\ref{app:proof_ovo_fdiv_maximisation}). We notably conclude that for the total variation and the squared Hellinger distance, Dirac distributions on an even partition of the data space are the only optimal solutions.

\begin{proposition}\label{prop:ovo_fdiv_maximisation}
Let $D$ be an $f$-divergence, $\p(y|\vec{x})$ a clustering distribution such that $\p(y=k)=\frac{1}{K}$. The OvO GEMINI is then upper bounded by a function depending only on the cluster proportions. For the upper bound to be tight, a sufficient condition is to have disjoint supports between cluster distributions $\p(\vec{x}|y=k)$. The condition is necessary if the function $f$ satisfies $f(0) + g(0)< \infty$ where $g(t)=tf\left(\frac{1}{t}\right)$ is the convex conjugate of $f$.
%Let $D$ be an $f$-divergence, $\p(y|\vec{x})$ a clustering distribution such that $\p(y=k)=\frac{1}{K}$. The OvO GEMINI is then upper bounded. If the distributions $\p(\vec{x}|y=k)$ have disjoint supports, then the upper bound is tight. The opposite is true if the function $f$ satisfies $f(0) + g(0)< \infty$ where $g(t)=tf\left(\frac{1}{t}\right)$ is the convex conjugate of $f$.
%Let $D$ an $f$-divergence and $p(\vec{x}|y)$ a distribution with $\vec{x}$ a continuous variable and $y$ and discrete variable taking $K$ values. Then, the OvO GEMINI is maximised with proportions $p(y=k)=\frac{1}{K}$ if /iff $p(\vec{x}|y=k)\perp p(x|y=k^\prime), \forall k \neq k^\prime$. The latter holds if the $f$-divergence's function $f$ satisfies $f(0) + g(0)< \infty$ where $g(t)=tf\left(\frac{1}{t}\right)$ is the convex conjugate of $f$.
\end{proposition}

\subsection{Using GEMINIs}

\begin{table*}[!tb]
\centering
\caption{Definition of the GEMINI for $f$-divergences, MMD and the Wasserstein distance. We directly write here the equation that can be optimised to train a discriminative model $\pyx$ via stochastic gradient descent since they are expectations over the data.}
\label{tab:all_geminis}
\begin{tabular}{c c}
\toprule
Name&Equation\\\hline\\
KL OvA/MI& $\E_{\pdata}\left[D_\text{KL}(\pyx \|\py)\right]$\\
KL OvO& $\E_{\pdata}[D_\text{KL}(\pyx \| \py))+D_\text{KL}(\py \| \pyx))]$\\
\begin{minipage}{0.2\linewidth}\centering Squared Hellinger\\OvA\end{minipage} & $1-\E_{\pdata}[\E_{\py}[\sqrt{\frac{\pyx}{\py}}]]$\\
\begin{minipage}{0.2\linewidth}\centering Squared Hellinger\\OvO \end{minipage} & $\E_{\pdata}[\mathbb{V}_{\py}[\sqrt{\frac{\pyx}{\py}}]]$\\
TV OvA& $\E_{\pdata} [D_\text{TV} (\pyx \| \py) ]$\\
TV OvO& $\frac{1}{2}\E_{\pdata}[\E_{\ya,\yb\sim\py}[|\frac{ \pyxa }{ \pya } - \frac{\pyxb}{\pyb}|]]$\\
\midrule\\
MMD OvA& $\E_{\py} \left[ \E_{\xa,\xb \sim \pdata} \left[ k(\xa, \xb) \left( \frac{\p(y|\xa)\p(y|\xb)}{\py^2} + 1 - 2\frac{\p(y|\xa)}{\py}\right) \right]^{\frac{1}{2}}\right]$ \\
MMD OvO& \begin{minipage}{0.7\linewidth}\centering\begin{multline*}\E_{\ya,\yb \sim \py} \left[ \E_{\xa,\xb \sim \pdata} \left[ k(\xa, \xb) \left( \frac{\p(\ya|\xb) \p(\ya|\xb)}{\pya^2} \right.\right.\right.\\\left.\left.\left. + \frac{\p(\yb |\xa)\p(\yb|\xb)}{\pyb^2} - 2\frac{\p(\ya |\xa)\p(\yb|\xb)}{\pya\pyb}\right) \right]^{\frac{1}{2}}\right] \end{multline*}\end{minipage}\\
Wasserstein OvA&$\mathbb{E}_{\py}\left[\mathcal{W}_c\left(\sum_{i=1}^N m_i^y\delta_{\vec{x}_i},\sum_{i=1}^N \frac{1}{N}\delta_{\vec{x}_i}\right)\right]$\\
Wasserstein OvO&$\mathbb{E}_{\ya,\yb\sim \py}\left[\mathcal{W}_c\left(\sum_{i=1}^N m_i^{\ya}\delta_{\vec{x}_i},\sum_{i=1}^N m_i^{\yb}\delta_{\vec{x}_i}\right)\right]$\\
\bottomrule
\end{tabular}
\end{table*}

\subsubsection{Choosing a GEMINI}

We stated in Section~\ref{ssec:discriminative_clustering} that we present the GEMINI as a distance between distributions evaluated in the data space $\mathcal{X}$ so that the distance $D$ can take into account the topology of the data. In practice, we only design a discriminative model $\pyx$. Thus, we need to compute all formulas of the GEMINI through Bayes' theorem to get equations depending on $\pyx$ and $\py$. We summarise the equations from all aforementioned GEMINIs in Table~\ref{tab:all_geminis} (see Appendix~\ref{app:deriving_geminis} for derivations). We give details on the complexity of GEMINIs in Appendix~\ref{app:exp_complexity} to help choose one. We propose as well some theoretical speed-ups in App.~\ref{app:wasserstein_speedup}. It is also important to consider the experimental purposes and context to choose a GEMINI. Indeed, when it is easier to design a distance than a kernel, the Wasserstein-GEMINI is more compatible than the MMD-GEMINI and vice-versa. Moreover, the MMD-GEMINI inherently computes expectations in a Hilbert Space which allows computing centroids deemed representative of the clusters. This notion of centroid is less straightforward when using the Wasserstein metric. %\textbf{FIXME: Do I add a mention regarding the $f$-divergence gradient to help people choosing even though we discourage $f$-divergences?}

\begin{figure*}[!t]
    \centering
    \subfloat[OvA GEMINIs]{
        \includegraphics[width=0.45\linewidth]{figs/mse_ova.pdf}
        \label{sfig:mse_ova}
    }\hfil
    \subfloat[OvO GEMINIs]{
        \includegraphics[width=0.45\linewidth]{figs/mse_ovo.pdf}
        \label{sfig:mse_ovo}
    }
    \caption{Mean Squared Error (log scale) of estimates with varying batch sizes compared to the true value over a complete dataset of a 1000 samples. Each estimate was performed 50 times per batch size and GEMINI.}
    \label{fig:mse_gemini}
\end{figure*}

\subsubsection{Estimating a GEMINI}

All GEMINIs in Table~\ref{tab:all_geminis} can be estimated using Monte Carlo making them compatible with mini-batch learning, with batch sizes of a few hundred for large datasets similarly to prior works~\cite{hu_learning_2017,ji_invariant_2019, hjelm_learning_2019}. We highlight the importance of the batch size when using GEMINIs. With the use of mini-batch for training, the complete GEMINI is not evaluated on the entire dataset and hence a bias may rise from the empirical estimate. This bias then has consequences on the gradient, which in turn alters training. To illustrate this point, we generated 1000 predictions from a Dirichlet distribution with 10 clusters. These predictions are a proxy for the output of any discriminative model $\pyx$. We then compute the true GEMINI on all samples before evaluating it 50 times for different randomly sampled batches of increasing size. We report in Figure~\ref{fig:mse_gemini} the Mean Squared Error of all GEMINIs. We see that past 200 samples for both the OvA and the OvO models, the mean squared error is already close to or below $10^{-2}$, except for the OvO Wasserstein- and MMD-GEMINIs. This implies an upper bound of $10^{-2}$ for the bias of the estimates. We conclude that there is possibly a bias in GEMINIs estimates, but it remains small enough to be negligible.

\subsubsection{Code}

The original implementation for all experiments regarding GEMINI can be found here: \url{https://github.com/oshillou/GEMINI}. However, later works led to the development of python package for small-scale datasets on CPU: \emph{gemclus} at \url{https://gemini-clustering.github.io/}.


\section{Experiments}
\label{sec:experiments}
% SECTION Experiements %
\section{Experiments and Analysis} \label{Experiments}
\subsection{Experimental Design}
% This section outlines our methodology to propose a new classification model ViCGCN. Firstly, three benchmark datasets mentioned in Section \ref{Experiments/Datasets} are collected, and they be cleaned as described in the following. Consequently, the data after pre-processing is used to train our baselines and proposed model. With each model implemented, we fine-tune to find optimal hyper-parameters and improve their performance. Then, we evaluated the performance of models by Macro F1-score and Weighted F1-score deputed in Section \ref{Experiments/Metrics}. Section \ref{Experiments/Result} details the model evaluation results. To better understand our proposed model, we analyze and discuss the proposed model from various aspects: Impact of graph convolutional networks (see Section \ref{imapactGCN}) and impact of lambda (see Section \ref{impactlamda}). In addition, comparisons with previous studies were made to assess the achievement of our study correctly (see Section \ref{comparisonprestudies}). We also select the model that exhibits the most exceptional performance to conduct an error analysis on the inaccurate predictions detected within our proposed model (see Section \ref{erroranalysis}). At the end of the experiment, an ablation study was made to investigate the effectiveness and contribution of our proposed approach ViCGCN (see Section \ref{ablationstudy}). Figure \ref{fig::Experiments/Procedure/Overview} illustrates our methodology, including data preparation, fine-tuning baselines, proposed model, and performance analysis.

This section delineates our approach for introducing a novel classification model called ViCGCN. Initially, we gather three benchmark datasets, as mentioned in Section \ref{Experiments/Datasets}, and subject them to a cleaning process described subsequently. Subsequently, the pre-processed data is employed for training both our baseline models and the proposed model. We fine-tune each model to identify optimal hyperparameters and enhance their performance. The evaluation of model performance is conducted using Macro F1-score and Weighted F1-score, as discussed in Section \ref{Experiments/Metrics}. Detailed results of model evaluations are presented in Section \ref{Experiments/Result}.

To gain a deeper insight into our proposed model, we conduct a comprehensive analysis and discussion from various angles. This includes assessing the impact of graph convolutional networks (see Section \ref{imapactGCN}) and the influence of the lambda parameter (see Section \ref{impactlamda}). Additionally, we make comparisons with prior studies to accurately gauge the accomplishments of our research (see Section \ref{comparisonprestudies}). Furthermore, we select the model that exhibits the most outstanding performance to carry out an error analysis on the inaccuracies detected within our proposed model (see Section \ref{erroranalysis}). Towards the end of the experiment, we conduct an ablation study to investigate the effectiveness and contribution of our proposed approach, ViCGCN (see Section \ref{ablationstudy}). Figure \ref{fig::Experiments/Procedure/Overview} provides an overview of our methodology, encompassing data preparation, baseline fine-tuning, the proposed model, and performance analysis.

\begin{figure}[!hpbt]
    \centering
    \includegraphics[width=\textwidth]{Procedures.png}
    \caption{Overview of our experimental design.}
    \label{fig::Experiments/Procedure/Overview}
\end{figure}

\subsection{Baseline Models} \label{Experiments/Baseline}
Contextualized language models have been extensively used in various natural language processing tasks, including text classification. Additionally, since PhoBERT and viBERT are monolingual models specifically designed for the Vietnamese language, comparing their performance with a widely used and established model like mBERT is essential. Furthermore, as GCN has been shown to effectively capture the context and relationships between words in a text, integrating it with a contextualized language model could improve its performance in text classification tasks. Because of the following reasons, we compare our ViCGCN model with baseline models.
\subsubsection{Contextualized Language Models}
\begin{itemize}
    % \item \textbf{BERT\footnote{\url{https://github.com/google-research/bert}}}: BERT is a contextualized word representation model pre-trained using bidirectional transformers and based on a masked language model. BERT showed power in various NLP tasks. BERT and its variants are called the BERTology, two versions of BERT, base and large, respectively. Moreover, each version has two different versions: cased\footnote{\url{https://huggingface.co/bert-base-cased}} and uncased\footnote{\url{https://huggingface.co/bert-base-uncased}}, respectively. The only difference is that in \textit{BERT case uncased}, the text has been lowercase before the WordPiece tokenization step, while in the mBERT cased version, the text is the same as the input text.
    % \item \textbf{BERT (case uncased)}
    \item \textbf{Multilingual BERT (mBERT)\footnote{\url{https://github.com/google-research/bert}}}: mBERT, introduced by \citet{devlin-etal-2019-bert}, is a BERT-based model with specific characteristics. It consists of 12 layers, 768 hidden units, 12 attention heads, and a total of 110 million parameters. Remarkably, mBERT is designed to support 104 distinct languages, and it has been trained on and can be applied to text in these 104 languages using a combination of masked language modeling (MLM) and next sentence prediction objectives. This training corpus includes content from Wikipedia\footnote{\url{https://www.wikipedia.org/}}. \textit{mBERT} consists of two versions cased\footnote{\url{https://huggingface.co/bert-base-multilingual-cased}} and uncased\footnote{\url{https://huggingface.co/bert-base-multilingual-uncased}}.
    \item \textbf{RoBERTa\footnote{\url{https://huggingface.co/roberta-base}}}: \citet{DBLP:journals/corr/abs-1907-11692} proposed RoBERTa. They utilize a dynamic masking technique during the training process, instructing the model to predict intentionally hidden segments of text within unannotated language samples. RoBERTa, implemented using the PyTorch framework, makes critical adjustments to BERT's essential hyperparameters.
    \item \textbf{XLM-RoBERTa (XLM-R)\footnote{\url{https://github.com/facebookresearch/XLM}}}: \citet{XLMR}  proposed XLM-R a masked language model based on the transformer architecture. This model stands out as a multilingual powerhouse, having been pre-trained on text from a staggering 100 languages. What makes XLM-R particularly impressive is the extensive and careful curation of over 2.5TB of data from CommonCrawl. Among its notable contributions are the improvements made for low-resource languages through specialized training and vocabulary expansion. Moreover, XLM-R boasts a more expansive shared vocabulary and a substantial increase in its overall model capacity, incorporating a whopping 550 million parameters. XLM-R includes \textit{base}\footnote{\url{https://huggingface.co/xlm-roberta-base}} and \textit{large}\footnote{\url{https://huggingface.co/xlm-roberta-large}} version.
    \item \textbf{PhoBERT\footnote{\url{https://huggingface.co/vinai/phobert-base}}}: \citet{nguyen-tuan-nguyen-2020-phobert} introduced a set of large-scale monolingual language models specifically designed for the Vietnamese language. Among these models, PhoBERT stands out as the state-of-the-art contextualized language model for Vietnamese. PhoBERT's architecture is built upon the RoBERTa model, but it has been optimized for training on a substantial Vietnamese corpus to effectively handle Vietnamese text. PhoBERT comes in two versions: \textit{base} and the \textit{large} versions.
    \item \textbf{viBERT\footnote{\url{https://huggingface.co/FPTAI/vibert-base-cased}}}: \citet{viBERT} introduced viBERT, a pre-trained language model for Vietnamese based on the BERT architecture. The architecture of viBERT is similar to that of mBERT, and it has been pre-trained on a large corpus of 10GB of uncompressed Vietnamese text. However, unlike mBERT, viBERT excludes insufficient vocabulary due to the inclusion of languages other than Vietnamese in the mBERT vocabulary.
    \item \textbf{vELECTRA\footnote{\url{https://huggingface.co/FPTAI/velectra-base-discriminator-cased}}}: \citet{viBERT}  unveiled vELECTRA, a pre-trained language model tailored for Vietnamese that adheres to the ELECTRA framework. vELECTRA shares a parallel architectural structure with ELECTRA and has undergone pretraining on an extensive corpus comprising 60GB of uncompressed Vietnamese text.
\end{itemize}

\subsubsection{Other Graph Neural Networks}
Bert-GCN was introduced by \citet{BertGCN}, presenting a novel approach that harnesses the benefits of extensive pretraining alongside transductive learning for the purpose of text classification. Bert-GCN achieves this by constructing a diverse graph over the dataset, where documents are represented as nodes, all leveraging the embedding power of BERT. Consequently, this research undertakes the implementation of various BERT variations, such as multilingual and Vietnamese monolingual models, in conjunction with GCN-combined models to assess their effectiveness in text classification for Vietnamese tasks. Additionally, when compared to mBERT-GCN, RoBERTa-GCN, viBERT-GCN, and vELECTRA-GCN, our proposed ViCGCN model offers valuable insights into the impact of integrating both monolingual and multilingual Contextualized Language Models with GCN on three standardized benchmark datasets. 

\subsection{Benchmark Datasets} \label{Experiments/Datasets}
\subsubsection{Benchmark Datasets} \label{Experiments/Datasets/Data}
To verify the efficiency of our proposed approach to text classification on Vietnamese social media, we conducted our experiments on three widely used Vietnamese social media corpora, including Vietnamese Social Media Emotion Corpus (UIT-VSMEC) that was made available by Ho et al. \citet{DBLP:journals/corr/abs-1911-09339}, Vietnamese Students' Feedback Corpus (UIT-VSFC) built by \citet{VSFC}, and Vietnamese Constructive and Toxic Speech Detection (UIT-ViCTSD) introduced by \citet{DBLP:journals/corr/abs-2103-10069}.


\begin{itemize}
    \item \textbf{UIT-VSMEC \citet{DBLP:journals/corr/abs-1911-09339}}: UIT-VSMEC consists of 6,927 sentences that have been annotated with emotions to tackle the challenge of identifying emotions in Vietnamese social media comments. This dataset encompasses seven emotion categories: Enjoyment, Disgust, Sadness, Anger, Fear, Surprise, and Other.
    \item \textbf{UIT-VSFC \citet{VSFC}}: UIT-VSFC comprises 16,000 sentences that have been investigated for two distinct purposes: one related to sentiment analysis and the other related to topic classification. The sentiment analysis task involves categorizing sentences into three classes: Positive, Negative, and Neutral. Meanwhile, the topic classification task involves assigning sentences to one of four categories: Lecturer, Curriculum, Facility, or Others.
    \item \textbf{UIT-ViCTSD \cite{DBLP:journals/corr/abs-2103-10069}}: UIT-ViCTSD consists of 10,000 human-annotated comments on ten domains. Each comment is categorized into two tasks: constructiveness and toxicity in Vietnamese social media, which are binary classifications. Two categories are used to denote feedback: constructive and non-constructive. Similarly, comments can be labeled as toxic or non-toxic to identify harmful behavior.
\end{itemize}

\subsubsection{Pre-processing techniques}
A few efficient pre-processing techniques for Vietnamese text in general and Vietnamese social media text in particular were presented \cite{nguyen2020exploiting, PhoBERT-CNN}. However, we only follow some simple preprocessed techniques according to the quality of the three benchmark datasets mentioned in Section \ref{Experiments/Datasets/Data} and more essential to prove the outperform and efficiency of our model ViCGCN on Vietnamese social media raw text. Firstly, we removed stopwords defined in Vietnamese stopwords dict\footnote{\url{https://github.com/stopwords/vietnamese-stopwords}}. We, then, segment sentences into words by applying Word Segmenter of VnCoreNLP\footnote{\url{https://github.com/vncorenlp/VnCoreNLP}} for all of the models. Finally, the Regex\footnote{\url{https://docs.python.org/3/library/re.html}} library in Python is used to remove all punctuations in three benchmark datasets.

% remove stopwords, segmentation, remove punctuation 

The statistics of the pre-processed datasets are summarized in Table \ref{4/Dataset}.

% Experiments/Datasets/Table %
    
\begin{table}[!ht]
\centering
\caption{Statistics and descriptions of tasks of each dataset in this study.}
\resizebox{\linewidth}{!}{%
\begin{tabular}{lrrrlr} 
\hline
\textbf{Dataset}            & \multicolumn{1}{l}{\textbf{Train}} & \multicolumn{1}{l}{\textbf{Dev}} & \multicolumn{1}{l}{\textbf{Test}} & \multicolumn{1}{c}{\textbf{Task}}         & \multicolumn{1}{l}{\textbf{Classes}}  \\ 
\hline
\multicolumn{6}{c}{\textit{Binary text classification}}                                                                                                                                                                     \\ 
\hline
\multirow{2}{*}{UIT-ViCTSD} & 7,000                              & 2,000                            & 1,000                             & Constructive speech detection             & 2                                     \\
                            & 7,000                              & 2,000                            & 1,000                             & Toxic speech detection                    & 2                                     \\ 
\hline
\multicolumn{6}{c}{\textit{Multi-class text classification}}                                                                                                                                                                \\ 
\hline
\multirow{2}{*}{UIT-VSMEC}  & 5,548                              & 686                              & 693                               & Emotion recognition (with Other label)    & 7                                     \\
                            & 4,527                              & 583                              & 589                               & Emotion recognition (without Other label) & 6                                     \\ 
\cline{1-1}
\multirow{2}{*}{UIT-VSFC}   & 11,426                             & 1,583                            & 3,166                             & Sentiment-based classification            & 3                                     \\
                            & 11,426                             & 1,583                            & 3,166                             & Topic-based classification                & 4                                     \\
\hline
\end{tabular}}
\label{4/Dataset}
\end{table}


\subsection{Evaluation Metric} \label{Experiments/Metrics}
% This section describes the performance evaluation metrics employed in this study. The commonly used metric for classification tasks, particularly for the three datasets mentioned in this study, is the Average Macro F1-score (\%). However, owing to significantly imbalanced classes in the given datasets, the most suitable metric for this study is the average macro F1-score, which is the harmonic mean of Precision and Recall. Additionally, to facilitate comparisons with previous studies, we used the corresponding measure based on the metrics used in those studies, such as the average weighted F1-score (\%) for both UIT-VSMEC and UIT-VSFC.

This section outlines the performance evaluation criteria utilized in this research. In the realm of classification tasks, especially concerning the three datasets highlighted within this study, the conventional metric employed is the Average Macro F1-score (\%). However, given the significant class imbalances in the provided datasets, the most appropriate metric for this study is the Average Macro F1-score, which is derived as the harmonic mean of Precision and Recall. Furthermore, to facilitate comparisons with prior studies, we have also adopted relevant measures based on the metrics employed in those studies, such as the Average Weighted F1-score (\%) for both UIT-VSMEC and UIT-VSFC datasets.

 To compute the average macro F1-score, firstly, we calculate Precision and Recall by Equation (\ref{eq::Experiments/Metrics/Presision}) and Equation (\ref{eq::Experiments/Metrics/Recall}) respectively. Then, Equation (\ref{eq::Experiments/Metrics/F1-score}) is used to determine F1-score per class in the dataset. $tp$ are truly positive, $fp$ – false positive, $fn$ – false negative, and $tn$ – true negative counts, respectively.
\begin{equation}
    Precision = \frac{tp}{tp+fp} \label{eq::Experiments/Metrics/Presision}
\end{equation}
\begin{equation}
    Recall=\frac{tp}{tp+fn} \label{eq::Experiments/Metrics/Recall}
\end{equation}
\begin{equation}
    \textit{F1-score}=2\times\frac{Precision\times Recall}{Precision+Recall} \label{eq::Experiments/Metrics/F1-score}
\end{equation}

We compute the average macro F1-score (mF1) and weighted F1-score (wF1) after acquiring the F1 scores for all classes. Equation (\ref{eq::Experiments/Metrics/macro F1-score}) and Equation (\ref{eq::Experiments/Metrics/weighted F1-score}) present the macro F1-score and weighted F1-score, respectively, for multi-class classification for multi classes $C_{i}$, i $\in$ \{1, 2,... n\} (denoted for every class of the dataset). Where $\textit{F1-score}_{i}$ and $W_{i}$ are the \textit{F1-score} and weight of class \textit{i} of the dataset, respectively.

\begin{equation}
    \textit{mF1} = \frac{{\sum_{i=1}^{n} \textit{F1-score}_{i}}}{n} \label{eq::Experiments/Metrics/macro F1-score}
\end{equation}
\begin{equation}
    \textit{wF1} = \frac{\sum_{i=1}^{n} {\textit{F1-score}_{i} \times W_{i}}}{\sum_{i=1}^{n} W_{i}} \label{eq::Experiments/Metrics/weighted F1-score}
\end{equation}
\subsection{Experiment Configuration}
% In this study, we implemented many transfer learning models including $\text{BERT}_{base}$ \textit{cased}, $\text{BERT}_{base}$ \textit{uncased}, mBERT \textit{cased}, mBERT \textit{uncased}, $\text{RoBERTa}_{large}$, $\text{PhoBERT}_{base}$, $\text{PhoBERT}_{large}$. Besides, several combined models are conducted along with Text-GCN, Bert-GCN and mBERT-GCN. 
Section \ref{Experiments/Configuration/Baseline} and Section \ref{Experiments/Configuration/Proposed} provide our settings for both baselines and the proposed approach in detail.

\subsubsection{Basesline models' configuration} \label{Experiments/Configuration/Baseline}
We implemented many transfer learning models including mBERT both \textit{cased} and \textit{uncased}, $\text{RoBERTa}$, XLM-R, $\text{PhoBERT}_{base}$, $\text{PhoBERT}_{large}$, vELECTRA, and viBERT in this study. They run with their max sequence length of 256, batch size of 32, epoch of 10, and Adam optimizer \cite{https://doi.org/10.48550/arxiv.1412.6980} with a fixed learning rate of 2e-5.
 
\subsubsection{Our approach's configuration} \label{Experiments/Configuration/Proposed}
In our proposed approach, $\text{PhoBERT}_{base}$ is the output feature of the [CLS] token as the sentence node, followed by a feedforward layer to derive the final prediction. We use $\text{PhoBERT}_{base}$ pre-trained model from HuggingFace combined with a two-layer GCN to implement ViCGCN. We initialize Adam optimizer \cite{https://doi.org/10.48550/arxiv.1412.6980} with a fixed learning rate of 1e-3 and 1e-5 for the GCN and PhoBERT module, respectively. Moreover, PhoBERT runs with a 256 max sequence length.

\subsection{Experimental Results} \label{Experiments/Result}


% \begin{table}[!hpbt]
% \centering
% \caption{F1-score performances of models on the test sets of various Vietnamese social media textual datasets. Improvement (1) and Improvement (2) denoted the improvement over BERTology models and the improvement over BERTology integrated with GCN models, respectively}
% \label{tab::Experiments/Result}
% \resizebox{\linewidth}{!}{%
% \begin{tabular}{c|cc|cc|cc|cc|cc|cc} 
% \hline
% \textbf{Datasets}                & \multicolumn{4}{c|}{\textbf{UIT-VSMEC}}                                               & \multicolumn{4}{c|}{\textbf{UIT-ViCTSD}}                                                & \multicolumn{4}{c}{\textbf{UIT-VSFC}}                                                     \\ 
% \hline
% \textbf{Tasks}                   & \multicolumn{2}{c|}{\textbf{Seven labels}} & \multicolumn{2}{c|}{\textbf{Six labels}} & \multicolumn{2}{c|}{\textbf{Constructiveness}} & \multicolumn{2}{c|}{\textbf{Toxicity}} & \multicolumn{2}{c|}{\textbf{Sentiment-based}} & \multicolumn{2}{c}{\textbf{Topic-based}}  \\ 
% \hline
%                                  & \textbf{wF1}   & \textbf{mF1}              & \textbf{wF1}   & \textbf{mF1}            & \textbf{wF1}   & \textbf{mF1}                  & \textbf{wF1}   & \textbf{mF1}          & \textbf{wF1}   & \textbf{mF1}                 & \textbf{wF1}   & \textbf{mF1}             \\ 
% \hline
% BERT (cased)                     & 55.06          & 55.88                     & 66.34          & 65.31                   & 79.34          & 77.29                         & 87.45          & 64.49                 & 86.32          & 72.38                        & 86.17          & 73.62                    \\
% BERT (uncased)                   & 55.98          & 56.40                     & 58.55          & 56.65                   & 80.42          & 78.90                         & 85.21          & 63.28                 & 85.51          & 71.44                        & 85.98          & 72.56                    \\
% mBERT (\textit{cased)}           & 61.23          & 59.57                     & 66.72          & 66.72                   & 78.15          & 76.93                         & 86.71          & 64.36                 & 91.21          & 78.94                        & 88.06          & 77.63                    \\
% mBERT (\textit{uncased)}         & 58.31          & 57.03                     & 67.13          & 67.70                   & 78.11          & 76.63                         & 87.65          & 64.96                 & 89.52          & 76.60                        & 87.12          & 76.96                    \\
% RoBERTa                          & 56.51          & 56.22                     & 62.11          & 64.64                   & 78.92          & 77.71                         & 83.82          & 61.73                 & 90.11          & 76.98                        & 87.03          & 76.77                    \\
% XLM-R              & 69.55          & 68.12                     & 68.66          & 68.32                   & 81.97          & 80.02                         & 88.35          & 65.21                 & 91.02          & 76.95                        & 87.32          & 76.25                    \\
% PhoBERT \textit{base}            & 71.86          & 69.58                     & 75.19          & 74.32                   & 81.03          & 79.53                         & 88.83          & 65.61                 & 90.51          & 76.47                        & 87.84          & 76.98                    \\
% PhoBERT \textit{large}           & 72.87          & 70.22                     & 76.22          & 75.32                   & 83.21          & 80.22                         & 89.32          & 66.21                 & 91.81          & 77.81                        & 88.12          & 77.22                    \\
% viBERT                           & 67.55          & 65.32                     & 69.95          & 69.08                   & 82.27          & 80.12                         & 88.13          & 64.35                 & 89.77          & 76.25                        & 87.43          & 76.25                    \\ 
% \hline
% Bert-GCN (BERT \textit{cased)}   & 74.22          & 73.51                     & 78.25          & 77.02                   & 81.13          & 79.72                         & 88.10          & 64.52                 & 88.72          & 76.23                        & 88.15          & 76.25                    \\
% Bert-GCN (BERT \textit{uncased)} & 74.18          & 73.29                     & 80.54          & 78.31                   & 81.05          & 79.60                         & 88.67          & 64.96                 & 88.51          & 75.99                        & 87.75          & 76.06                    \\
% mBERT-GCN (mBERT cased)          & 75.12          & 74.83                     & 79.55          & 77.84                   & 82.15          & 80.33                         & 89.13          & 65.13                 & 91.89          & 79.84                        & 89.73          & 79.02                    \\
% mBERT-GCN (mBERT uncased)        & 74.56          & 73.98                     & 80.21          & 78.12                   & 82.88          & 81.12                         & 89.83          & 65.89                 & 91.72          & 79.64                        & 88.89          & 78.82                    \\
% RoBERTa-GCN                      & 74.82          & 74.22                     & 79.33          & 78.32                   & 83.47          & 82.77                         & 86.55          & 64.33                 & 91.12          & 79.32                        & 90.12          & 79.34                    \\
% viBERT-GCN                       & 78.25          & 78.37                     & 82.33          & 81.98                   & 86.12          & 85.02                         & 91.27          & 75.93                 & 93.27          & 87.52                        & 92.11          & 88.35                    \\ 
% \hline
% \textbf{ViCGCN}                  & \textbf{80.24} & \textbf{80.96}            & \textbf{84.91} & \textbf{83.27}          & \textbf{86.97} & \textbf{85.81}                & \textbf{91.95} & \textbf{76.29}        & \textbf{94.81} & \textbf{88.80}               & \textbf{93.91} & \textbf{89.61}           \\
% Improvement (1)                  & $\uparrow$7.37 & $\uparrow$10.74           & $\uparrow$8.69 & $\uparrow$7.95          & $\uparrow$3.76 & $\uparrow$5.59                & $\uparrow$2.63 & $\uparrow$9.98        & $\uparrow$3.00 & $\uparrow$9.86               & $\uparrow$5.69 & $\uparrow$11.98          \\
% Improvement (2)                  & $\uparrow$1.99 & $\uparrow$2.59            & $\uparrow$2.58 & $\uparrow$1.29          & $\uparrow$0.85 & $\uparrow$0.79                & $\uparrow$0.68 & $\uparrow$0.26        & $\uparrow$1.54 & $\uparrow$1.28               & $\uparrow$1.70 & $\uparrow$1.26           \\
% \hline
% \end{tabular}}
% \end{table}

\begin{table}[!ht]
\centering
\caption{F1-score performances of models on the test sets of various Vietnamese social media textual datasets. Improvement (1) and Improvement (2) denoted the improvement over BERTology models and the improvement over BERTology integrated with GCN models, respectively.}
\label{tab::Experiments/Result}
\resizebox{\linewidth}{!}{%
\begin{tabular}{c|cc|cc|cc|cc|cc|cc} 
\hline
\textbf{Datasets}        & \multicolumn{4}{c|}{\textbf{UIT-VSMEC}}                                               & \multicolumn{4}{c|}{\textbf{UIT-ViCTSD}}                                                & \multicolumn{4}{c}{\textbf{UIT-VSFC}}                                                     \\ 
\hline
\textbf{Tasks}           & \multicolumn{2}{c|}{\textbf{Seven labels}} & \multicolumn{2}{c|}{\textbf{Six labels}} & \multicolumn{2}{c|}{\textbf{Constructiveness}} & \multicolumn{2}{c|}{\textbf{Toxicity}} & \multicolumn{2}{c|}{\textbf{Sentiment-based}} & \multicolumn{2}{c}{\textbf{Topic-based}}  \\ 
\hline
                         & \textbf{wF1}   & \textbf{mF1}              & \textbf{wF1}   & \textbf{mF1}            & \textbf{wF1}   & \textbf{mF1}                  & \textbf{wF1}   & \textbf{mF1}          & \textbf{wF1}   & \textbf{mF1}                 & \textbf{wF1}   & \textbf{mF1}             \\ 
\hline
mBERT (\textit{cased)}   & 60.47          & 59.48                     & 65.02          & 62.65                   & 81.03          & 79.55                         & 88.32          & 65.63                 & 90.39          & 77.15                        & 87.32          & 77.93                    \\
mBERT (\textit{uncased)} & 60.17          & 59.18                     & 64.93          & 62.11                   & 80.89          & 79.47                         & 87.6           & 64.77                 & 89.95          & 77.8                         & 87.62          & 77.58                    \\
RoBERTa                  & 58.17          & 57.32                     & 63.32          & 59.97                   & 77.41          & 75.62                         & 85.85          & 59.71                 & 87.13          & 75.52                        & 86.77          & 75.30                    \\
XLM-R                    & 62.02          & 61.01                     & 68.19          & 63.70                   & 81.81          & 80.85                         & 89.92          & 73.09                 & 93.03          & 82.61                        & 89.67          & 79.25                    \\
PhoBERT \textit{base}    & 64.36          & 61.41                     & 69.02          & 64.12                   & 81.65          & 80.24                         & 89.58          & 72.12                 & 92.94          & 82.15                        & 88.29          & 78.54                    \\
PhoBERT \textit{large}   & 65.12          & 63.23                     & 71.13          & 65.12                   & 82.07          & 81.27                         & 90.12          & 73.32                 & 93.24          & 82.96                        & 88.72          & 79.12                    \\
vELECTRA                 & 63.58          & 61.38                     & 68.33          & 63.12                   & 82.41          & 80.82                         & 89.33          & 72.02                 & 91.89          & 82.01                        & 88.12          & 78.12                    \\
viBERT                   & 61.33          & 60.28                     & 68.48          & 62.09                   & 81.62          & 80.07                         & 89.14          & 71.87                 & 91.29          & 81.95                        & 88.22          & 78.35                    \\ 
\hline
mBERT-GCN (cased)        & 68.32          & 64.32                     & 69.32          & 66.18                   & 83.12          & 82.88                         & 90.32          & 69.42                 & 92.12          & 79.32                        & 88.32          & 79.42                    \\
mBERT-GCN (uncased)      & 67.98          & 64.11                     & 69.12          & 65.89                   & 82.32          & 82.01                         & 89.15          & 68.32                 & 91.01          & 79.02                        & 88.07          & 79.02                    \\
RoBERTa-GCN              & 66.17          & 62.12                     & 67.12          & 64.17                   & 81.33          & 80.96                         & 89.02          & 64.32                 & 90.12          & 78.42                        & 87.45          & 78.12                    \\
vELECTRA-GCN             & 69.42          & 65.44                     & 70.95          & 67.20                   & 84.62          & 84.62                         & 91.88          & 74.85                 & 93.56          & 83.12                        & 89.95          & 80.02                    \\
viBERT-GCN               & 69.32          & 65.12                     & 70.83          & 66.68                   & 84.32          & 83.12                         & 91.12          & 74.25                 & 93.12          & 82.47                        & 89.42          & 79.63                    \\ 
\hline
\textbf{ViCGCN (base)}   & \textbf{70.32} & \textbf{67.17}            & \textbf{71.02} & \textbf{67.48}          & \textbf{85.64} & \textbf{85.12}                & \textbf{92.22} & \textbf{75.32}        & \textbf{94.12} & \textbf{83.67}               & \textbf{90.12} & \textbf{80.11}           \\
\textbf{ViCGCN (large)}  & \textbf{71.33} & \textbf{67.82}            & \textbf{72.08} & \textbf{68.12}          & \textbf{86.12} & \textbf{85.88}                & \textbf{93.11} & \textbf{76.12}        & \textbf{94.83} & \textbf{84.23}               & \textbf{91.02} & \textbf{81.88}           \\ 
\hline
Improvement (1)          & $\uparrow$6.21 & $\uparrow$4.59            & $\uparrow$0.95 & $\uparrow$3.00          & $\uparrow$3.71 & $\uparrow$4.61                & $\uparrow$2.99 & $\uparrow$2.80        & $\uparrow$1.59 & $\uparrow$1.27               & $\uparrow$1.35 & $\uparrow$2.63           \\
Improvement (2)          & $\uparrow$1.91 & $\uparrow$2.38            & $\uparrow$1.13 & $\uparrow$0.92          & $\uparrow$1.50 & $\uparrow$1.26                & $\uparrow$1.23 & $\uparrow$1.27        & $\uparrow$1.27 & $\uparrow$1.11               & $\uparrow$1.07 & $\uparrow$1.86           \\
\hline
\end{tabular}}
\end{table}

To demonstrate the classification performance of our model ViCGCN, we compare it with other state-of-the-art and Integrated models as mentioned in Section \ref{Experiments/Baseline}. The F1-score results for both baseline and proposed models on the test sets of three Vietnamese social media text datasets are shown in Table \ref{tab::Experiments/Result} and we obtain the following observations.

Among BERTology models, RoBERTa and mBERT, including \textit{cased} and \textit{uncased}, have the most unfavorable performance of almost tasks of the three benchmark datasets. Moreover, the results show that monolingual models such as PhoBERT and viBERT perform better than other BERTology models. Additionally, through the execution of parallel computations for words, the problem of vanishing gradients is minimized, and PhoBERT archives the highest results in nearly all the tasks. However, in general, BERTology baseline models still find it hard to handle the complexity of social media: imbalanced and noisy data, which leads to poor performance compared to the integrated GCN model.
    
Our baseline integrated models can also benefit from graph structure by combining GCN as the final prediction module. Compared to BERTology baseline models, the performance boost from contextualized pre-trained language models with the GCN module is significant. Moreover, the multilingual and monolingual models integrated with GCN perform massively better than others. This explains the significance of incorporating both the Contextualized and GCN models into the integrated models can be attributed to their complementary nature in addressing the limitations of each other.

Compared to baseline models, our approach ViCGCN adopts large-scale, monolingual Vietnamese language model PhoBERT. Our integrated model ViCGCN obtains the ability to compute the new features of a node as the weighted average of itself and its second-order neighbors. In the context of imbalanced and noisy datasets, such as UIT-ViCTSD, the proposed ViCGCN model has demonstrated significant performance improvements compared to other baseline models, making it a promising approach for social media mining tasks. Moreover, Our proposed model demonstrated superior performance to the current state-of-the-art Vietnamese model, achieving improvements of 6.21\%, 4.61\%, and 2.63\% on three benchmark datasets. These results demonstrate the efficacy and validity of ViCGCN for Vietnamese text classification. As a result, our method achieves the best performance among all the tasks on three benchmark datasets in terms of UIT-VSMEC, UIT-ViCTSD, and UIT-VSFC, respectively.

\subsection{Analysis and Discussion}

\subsubsection{Impact of graph convolutional networks}
\label{imapactGCN}

Although we can implicitly infer the effectiveness of graph convolutional networks from Table \ref{tab::Experiments/Result}, we would like to discuss more the contribution of graph convolutional networks in contextualized language models. Table \ref{Result/Graph/VSMEC/table}, Table \ref{/Result/Graph/ViCTSD/table} and Table \ref{/Result/Graph/VSFC/table} display the comparisons between with and without GCN on three benchmark datasets as we can find that contextualized language model integrated with GCN outperformed all of the corresponding single models, respectively. As mentioned in Section \ref{Experiments/Result}, Contextualized Language Models have not performed well on three benchmark datasets. Integrating GCN with the BERTology model massively enhances the performance, which leads to improvements of up to 8.00\%, 7.99\%, 5.84\%, and 7.99\% of RoBERTa, viBERT, vELECTRA, and $\text{PhoBERT}_{base}$, respectively, on three benchmark datasets, UIT-VSMEC, UIT-ViCTSD, and UIT-VSFC, respectively. The average length of three datasets in UIT-VSMEC, UIT-ViCTSD, and UIT-VSFC is approximately 14. Additionally, the short sequence lengths can construct more dense graphs that provide richer contextual information, which may explain better performance by combining contextualized language models with GCN. This further demonstrates that Graph Convolutional Networks are essential in improving text classification performance.


% \begin{table}[!hp] 
% \centering
% \caption{Model performance on UIT-VSMEC.}
% \label{Result/Graph/VSMEC/table}
% %\centerline{
% \resizebox{\linewidth}{!}{
% \begin{tabular}{l|cc|cc} 
% \hline
% \textbf{Tasks}            & \multicolumn{2}{c|}{\textbf{Seven labels}}                                           & \multicolumn{2}{c}{\textbf{Six labels}}                                              \\ 
% \hline
%                           & \textbf{wF1}                             & \textbf{mF1}                              & \textbf{wF1}                             & \textbf{mF1}                              \\ 
% \hline
% BERT (cased)              & 55.06                                    & 55.88                                     & 66.34                                    & 65.31                                     \\
% Bert-GCN (BERT cased)     & 74.22 ($\uparrow$19.16)                  & 73.51 ($\uparrow$17.63)                   & 78.25 ($\uparrow$11.91)                  & 77.02 ($\uparrow$11.71)                   \\ 
% \hline
% BERT (uncased)            & 55.98                                    & 56.40                                     & 58.55                                    & 56.65                                     \\
% Bert-GCN (BERT uncased)   & 74.18 ($\uparrow$18.20)                  & 73.29 ($\uparrow$16.89)                   & 80.54 ($\uparrow$21.99)                  & 78.31 ($\uparrow$21.66)                   \\ 
% \hline
% mBERT (mBERT cased)       & 61.23                                    & 59.57                                     & 66.72                                    & 66.72                                     \\
% mBERT-GCN (mBERT cased)   & 75.12 ($\uparrow$13.89)                  & 74.83 ($\uparrow$15.26)                   & 79.55 ($\uparrow$12.83)                  & 77.84 ($\uparrow$11.12)                   \\ 
% \hline
% mBERT (uncased)           & 58.31                                    & 57.03                                     & 67.13                                    & 67.70                                     \\
% mBERT-GCN (mBERT uncased) & 74.56 ($\uparrow$16.25)                  & 73.98 ($\uparrow$16.95)                   & 80.21 ($\uparrow$13.08)                  & 78.12 ($\uparrow$10.42)                   \\ 
% \hline
% RoBERTa                   & 56.51                                    & 56.22                                     & 62.11                                    & 64.64                                     \\
% RoBERTa-GCN               & 74.82 ($\uparrow$18.31)                  & 74.22 ($\uparrow$18.00)                   & 79.33 ($\uparrow$17.22)                  & 78.32 ($\uparrow$13.68)                   \\ 
% \hline
% viBERT                    & 67.55                                    & 65.32                                     & 69.95                                    & 69.08                                     \\
% viBERT-GCN                & 78.25 ($\uparrow$10.70)                  & 78.37 ($\uparrow$13.05)                   & 82.33 ($\uparrow$12.38)                  & 81.98 ($\uparrow$12.90)                   \\ 
% \hline
% PhoBERT                   & 71.86                                    & 69.58                                     & 75.19                                    & 74.32                                     \\
% \textbf{ViCGCN (Ours)}    & \textbf{80.24 ($\uparrow$\textbf{8.38)}} & \textbf{80.96 ($\uparrow$\textbf{11.38)}} & \textbf{84.91 ($\uparrow$\textbf{9.72)}} & \textbf{83.27 ($\uparrow$\textbf{8.95)}}  \\
% \hline
% \end{tabular}}
% \end{table}

% \begin{table}[!hpbt] 
% \centering
% \caption{Model performance on UIT-ViCTSD.}
% \label{/Result/Graph/ViCTSD/table}
% \resizebox{\linewidth}{!}{%
% \begin{tabular}{l|cc|cc} 
% \hline
% \textbf{Tasks}            & \multicolumn{2}{c|}{\textbf{Constructiveness}}                    & \multicolumn{2}{c}{\textbf{Toxicity}}                               \\ 
% \hline
%                           & \textbf{wF1}                    & \textbf{mF1}                    & \textbf{wF1}                    & \textbf{mF1}                      \\ 
% \hline
% BERT (cased)              & 79.34                           & 77.29                           & 87.45                           & 64.49                             \\
% Bert-GCN (BERT cased)     & 81.13 ($\uparrow$1.79)          & 79.72 ($\uparrow$2.43)          & 88.10 ($\uparrow$0.65)          & 64.52 ($\uparrow$0.03)            \\ 
% \hline
% BERT (uncased)            & 80.42                           & 78.90                           & 85.21                           & 63.28                             \\
% Bert-GCN (BERT uncased)   & 81.05 ($\uparrow$0.63)          & 79.6 ($\uparrow$0.70)           & 88.67 ($\uparrow$3.46)          & 64.96 ($\uparrow$1.68)            \\ 
% \hline
% mBERT (mBERT cased)       & 78.15                           & 76.93                           & 86.71                           & 64.36                             \\
% mBERT-GCN (mBERT cased)   & 82.15 ($\uparrow$4.00)          & 80.33 ($\uparrow$3.40)          & 89.13 ($\uparrow$0.46)          & 65.13 ($\uparrow$0.77)            \\ 
% \hline
% mBERT (uncased)           & 78.11                           & 76.63                           & 87.65                           & 64.96                             \\
% mBERT-GCN (mBERT uncased) & 82.88 ($\uparrow$4.77)          & 81.12 ($\uparrow$4.49)          & 89.93 ($\uparrow$2.28)          & 65.89 ($\uparrow$0.93)            \\ 
% \hline
% RoBERTa                   & 78.92                           & 77.71                           & 83.82                           & 61.73                             \\
% RoBERTa-GCN               & 83.47 ($\uparrow$4.55)          & 82.77 ($\uparrow$5.06)          & 86.55 ($\uparrow$2.73)          & 64.33 ($\uparrow$2.60)            \\ 
% \hline
% viBERT                    & 82.27                           & 80.12                           & 88.13                           & 64.35                             \\
% viBERT-GCN                & 86.12 ($\uparrow$3.85)          & 85.02 ($\uparrow$4.90)          & 91.27 ($\uparrow$3.14)          & 75.93 ($\uparrow$11.58)           \\ 
% \hline
% PhoBERT                   & 81.03                           & 79.53                           & 88.83                           & 65.61                             \\
% \textbf{ViCGCN (Ours)}    & \textbf{86.97 ($\uparrow$5.94)} & \textbf{85.81 ($\uparrow$6.28)} & \textbf{91.95 ($\uparrow$3.12)} & \textbf{76.29 ($\uparrow$10.68)}  \\
% \hline
% \end{tabular}}
% \end{table}

% \begin{table}[!hpbt] 
% \centering
% \caption{Model performance on UIT-VSFC.}
% \label{/Result/Graph/VSFC/table}
% \resizebox{\linewidth}{!}{%
% \begin{tabular}{l|cc|cc} 
% \hline
% \textbf{Tasks}            & \multicolumn{2}{c|}{\textbf{Sentiment-based}}                      & \multicolumn{2}{c}{\textbf{Topic-based}}                                  \\ 
% \hline
%                           & \textbf{wF1}                    & \textbf{mF1}                     & \textbf{wF1}                    & \textbf{mF1}                            \\ 
% \hline
% BERT (cased)              & 86.32                           & 72.38                            & 86.17                           & 73.62                                   \\
% Bert-GCN (BERT cased)     & 88.72 ($\uparrow$2.40)          & 76.23 ($\uparrow$3.85)           & 88.15 ($\uparrow$1.98)          & 76.25 ($\uparrow$2.63)                  \\ 
% \hline
% BERT (uncased)            & 85.51                           & 71.44                            & 85.98                           & 72.56                                   \\
% Bert-GCN (BERT uncased)   & 88.51 ($\uparrow$3.00)          & 75.99 ($\uparrow$4.55)           & 87.75 ($\uparrow$1.77)          & 76.06 ($\uparrow$3.50)                  \\ 
% \hline
% mBERT (mBERT cased)       & 91.21                           & 78.94                            & 88.06                           & 77.63                                   \\
% mBERT-GCN (mBERT cased)   & 91.89 ($\uparrow$0.68)          & 79.84 ($\uparrow$0.90)           & 89.73 ($\uparrow$1.98)          & 79.02 ($\uparrow$1.39)                  \\ 
% \hline
% mBERT (uncased)           & 89.52                           & 76.60                            & 87.12                           & 76.96                                   \\
% mBERT-GCN (mBERT uncased) & 91.72 ($\uparrow$2.20)          & 79.64 ($\uparrow$3.04)           & 88.89 ($\uparrow$1.77)          & 78.82 ($\uparrow$1.86)                  \\ 
% \hline
% RoBERTa                   & 90.11                           & 76.98                            & 87.03                           & 76.77                                   \\
% RoBERTa-GCN               & 91.12 ($\uparrow$1.01)          & 79.32 ($\uparrow$2.34)           & 90.12 ($\uparrow$3.09)          & 79.34 ($\uparrow$2.57)                  \\ 
% \hline
% viBERT                    & 89.77                           & 76.25                            & 87.43                           & 76.55                                   \\
% viBERT-GCN                & 93.27 ($\uparrow$3.50)          & 87.52 ($\uparrow$11.27)          & 92.11 ($\uparrow$4.68)          & 88.35 ($\uparrow$11.80)  \\ 
% \hline
% PhoBERT                   & 90.51                           & 76.47                            & 87.84                           & 76.98                                   \\
% \textbf{ViCGCN (Ours)}    & \textbf{94.81 ($\uparrow$4.30)} & \textbf{88.80 ($\uparrow$12.33)} & \textbf{93.81 ($\uparrow$5.97)} & \textbf{89.61 ($\uparrow$12.63)}        \\
% \hline
% \end{tabular}}
% \end{table}

\begin{table}[!ht]
\centering
\caption{Model performance on UIT-VSMEC.}
\label{Result/Graph/VSMEC/table}
\resizebox{\linewidth}{!}{%
\begin{tabular}{l|cc|cc} 
\hline
\textbf{Tasks}          & \multicolumn{2}{c|}{\textbf{Seven labels}}                        & \multicolumn{2}{c}{\textbf{Six labels}}                            \\ 
\hline
                        & \textbf{wF1}                    & \textbf{mF1}                    & \textbf{wF1}                    & \textbf{mF1}                     \\ 
\hline
mBERT (cased)           & 60.47                           & 59.48                           & 65.02                           & 62.65                            \\
mBERT-GCN (cased)       & 68.32 ($\uparrow$7.85)          & 64.32 ($\uparrow$4.84)          & 69.32 ($\uparrow$4.30)          & 66.18 ($\uparrow$3.53)           \\ 
\hline
mBERT (uncased)         & 60.17                           & 59.18                           & 64.93                           & 62.11                            \\
mBERT-GCN (uncased)     & 67.98 ($\uparrow$7.81)          & 64.11 ($\uparrow$4.93)          & 69.12 ($\uparrow$4.90)          & 65.89 ($\uparrow$3.78)           \\ 
\hline
RoBERTa                 & 58.17                           & 57.32                           & 63.32                           & 59.97                            \\
RoBERTa-GCN             & 66.17 ($\uparrow$8.00)          & 62.12 ($\uparrow$4.80)          & 67.12 ($\uparrow$3.80)          & 64.17 ($\uparrow$4.20)           \\ 
\hline
viBERT                  & 61.33                           & 60.28                           & 68.48                           & 62.09                            \\
viBERT-GCN              & 69.32 ($\uparrow$7.99)          & 78.37 ($\uparrow$4.84)          & 82.33 ($\uparrow$2.35)          & 81.98 ($\uparrow$4.59)           \\ 
\hline
vELECTRA                & 63.58                           & 61.38                           & 68.33                           & 63.12                            \\
vELETRA-GCN             & 69.42 ($\uparrow$5.84)          & 65.44 ($\uparrow$4.06)          & 70.95 ($\uparrow$2.62)          & 67.20 ($\uparrow$4.08)           \\ 
\hline
PhoBERT (base)          & 64.36                           & 61.41                           & 69.02                           & 64.12                            \\
\textbf{ViCGCN (base)}  & \textbf{69.32 ($\uparrow$7.99)} & \textbf{65.12 ($\uparrow$4.84)} & \textbf{70.83 ($\uparrow$2.35)} & \textbf{66.68 ($\uparrow$4.59)}  \\ 
\hline
PhoBERT (large)         & 65.12                           & 71.13                           & 63.23                           & 65.12                            \\
\textbf{ViCGCN (large)} & \textbf{71.33 ($\uparrow$6.21)} & \textbf{72.08 ($\uparrow$0.95)} & \textbf{67.82 ($\uparrow$4.59)} & \textbf{68.12 ($\uparrow$3.00)}  \\
\hline
\end{tabular}}
\end{table}

\begin{table}[H]
\centering
\caption{Model performance on UIT-ViCTSD.}
\label{/Result/Graph/ViCTSD/table}
\resizebox{\linewidth}{!}{%
\begin{tabular}{l|cc|cc} 
\hline
\textbf{Tasks}            & \multicolumn{2}{c|}{\textbf{Constructiveness}}                    & \multicolumn{2}{c}{\textbf{Toxicity}}                              \\ 
\hline
                          & \textbf{wF1}                    & \textbf{mF1}                    & \textbf{wF1}                    & \textbf{mF1}                     \\ 
\hline
mBERT (mBERT cased)       & 81.03                           & 79.55                           & 88.32                           & 65.63                            \\
mBERT-GCN (mBERT cased)   & 83.12 ($\uparrow$2.09)          & 82.88 ($\uparrow$3.33)          & 90.32 ($\uparrow$2.00)          & 69.42 ($\uparrow$3.79)           \\ 
\hline
mBERT (uncased)           & 80.89                           & 79.47                           & 87.60                           & 64.77                            \\
mBERT-GCN (mBERT uncased) & 82.32 ($\uparrow$1.43)          & 82.01 ($\uparrow$2.54)          & 89.15 ($\uparrow$1.55)          & 68.32 ($\uparrow$3.55)           \\ 
\hline
RoBERTa                   & 77.41                           & 75.62                           & 85.85                           & 59.71                            \\
RoBERTa-GCN               & 81.33 ($\uparrow$3.92)          & 80.96 ($\uparrow$5.34)          & 89.02 ($\uparrow$3.17)          & 64.32 ($\uparrow$4.61)           \\ 
\hline
viBERT                    & 81.62                           & 80.07                           & 89.14                           & 71.87                            \\
viBERT-GCN                & 84.32 ($\uparrow$2.70)          & 83.12 ($\uparrow$3.05)          & 91.12 ($\uparrow$1.98)          & 74.25 ($\uparrow$2.38)           \\ 
\hline
vELECTRA                  & 82.41                           & 80.82                           & 89.33                           & 72.02                            \\
vELETRA-GCN               & 84.62 ($\uparrow$2.21)          & 84.62 ($\uparrow$3.80)          & 91.88 ($\uparrow$2.55)          & 74.85 ($\uparrow$2.83)           \\ 
\hline
PhoBERT (base)            & 81.65                           & 80.24                           & 89.58                           & 72.12                            \\
\textbf{ViCGCN (base)}    & \textbf{85.64 ($\uparrow$3.99)} & \textbf{85.12 ($\uparrow$4.88)} & \textbf{92.22 ($\uparrow$2.64)} & \textbf{75.32 ($\uparrow$3.20)}  \\ 
\hline
PhoBERT large             & 82.07                           & 90.12                           & 81.27                           & 73.32                            \\
\textbf{ViCGCN (large)}   & \textbf{86.12 ($\uparrow$4.05)} & \textbf{93.11 ($\uparrow$2.99)} & \textbf{85.88 ($\uparrow$4.61)} & \textbf{76.12 ($\uparrow$2.80)}  \\
\hline
\end{tabular}}
\end{table}

\begin{table}[!ht]
\centering
\caption{Model performance on UIT-VSFC.}
\label{/Result/Graph/VSFC/table}
\resizebox{\linewidth}{!}{%
\begin{tabular}{l|cc|cc} 
\hline
\textbf{Tasks}            & \multicolumn{2}{c|}{\textbf{Sentiment-based}}                                                                                          & \multicolumn{2}{c}{\textbf{Topic-based}}                                             \\ 
\hline
                          & \textbf{wF1}                                                                                & \textbf{mF1}                             & \textbf{wF1}                             & \textbf{mF1}                              \\ 
\hline
mBERT (cased)             & 90.39                                                                                       & 77.15                                    & 87.32                                    & 77.93                                     \\
mBERT-GCN (cased)         & 92.12 ($\uparrow$1.73)                                                                      & 79.32 ($\uparrow$2.17)                   & 88.32 ($\uparrow$1.00)                   & 79.42 ($\uparrow$1.49)                    \\
mBERT (uncased)           & 89.95                                                                                       & 77.80                                    & 87.62                                    & 77.58                                     \\
mBERT-GCN (mBERT uncased) & 91.01 ($\uparrow$1.06)                                                                      & 79.02 ($\uparrow$1.22)                   & 88.07 ($\uparrow$0.45)                   & 79.02 ($\uparrow$1.44)                    \\ 
\hline
RoBERTa                   & 87.13                                                                                       & 75.52                                    & 86.77                                    & 75.30                                     \\
RoBERTa-GCN               & 90.12 ($\uparrow$2.99)                                                                      & 78.42 ($\uparrow$2.90)                   & 87.45 ($\uparrow$0.68)                   & 78.12 ($\uparrow$2.82)                    \\ 
\hline
viBERT                    & 91.29                                                                                       & 81.95                                    & 88.22                                    & 78.35                                     \\
viBERT-GCN                & 93.12 ($\uparrow$1.83)                                                                      & 82.47 ($\uparrow$0.52)                   & 89.42 ($\uparrow$1.20)                   & 79.63 ($\uparrow$1.28)                    \\ 
\hline
vELECTRA                  & 91.89                                                                                       & 82.01                                    & 88.12                                    & 78.12                                     \\
vELETRA-GCN               & 93.56 ($\uparrow$1.67)                                                                      & 83.12 $(\uparrow$1.11)                   & 89.95 ($\uparrow$1.83)                   & 80.02 ($\uparrow$1.90)                    \\ 
\hline
PhoBERT (base)            & 92.94                                                                                       & 82.15                                    & 88.29                                    & 78.54                                     \\
\textbf{ViCGCN (base)}    & \begin{tabular}[c]{@{}c@{}}\textbf{94.12~}($\uparrow$\textbf{}\textbf{1.18)}\end{tabular} & \textbf{83.67~}($\uparrow$\textbf{1.52)} & \textbf{90.12~}($\uparrow$\textbf{1.83)} & \textbf{80.11~}($\uparrow$\textbf{1.57)}  \\ 
\hline
PhoBERT (large)           & 93.24                                                                                       & 88.72                                    & 82.96                                    & 79.12                                     \\
\textbf{ViCGCN (large)}   & \textbf{94.83~}($\uparrow$\textbf{1.59)}                                                    & \textbf{91.02~}($\uparrow$\textbf{2.3)}  & \textbf{84.23~}($\uparrow$\textbf{1.27)} & \textbf{81.88~}($\uparrow$\textbf{2.76)}  \\
\hline
\end{tabular}}
\end{table}

\subsubsection{Impact of lambda ($\lambda$)}
\label{impactlamda}

According to Equation \ref{equa::lambda}, the hyperparameter $\lambda$ controls the trade-off between two objectives, ViCGCN and PhoBERT, respectively. The optimal value of $\lambda$ may vary depending on the task. Therefore, extensive experiments on the dev set were conducted to determine the optimal value of $\lambda$. Figure \ref{fig::Experiments/Lamda/UIT-VSFC} shows the performances of ViCGCN on three benchmark datasets in terms of UIT-VSMEC, UIT-ViCTSD, and UIT-VSFC with different $\lambda$. On all three benchmark datasets, the F1-score is consistently higher with a more enormous $\lambda$ value. Moreover, taking only ViCGCN ($\lambda = 1$) as the final training objective consistently achieves a better performance than considering only PhoBERT ($\lambda = 0$). Setting $\lambda$ to a value from 0.6 to 0.8 is more desirable and can make the model reach its best when $\lambda = 0.6$ on all datasets. These observations indicate that the linear interpolation of the prediction from ViCGCN and the prediction from PhoBERT with higher ViCGCN weight can improve the Vietnamese social media text classification performance. On the other hand, the PhoBERT module is also indispensable.
% \begin{figure}[H]
%     \centering
%     \subfigure[Seven labels task]{\includegraphics[width=0.49\textwidth]{ldVSMEC_7.pdf}}
%     \subfigure[Six labels task]{\includegraphics[width=0.49\textwidth]{ldVSMEC_6.pdf}}
%     \caption{F1-score of ViCGCN when varying $\lambda$ on UIT-VSMEC dev set.}
%     \label{fig::Experiments/Lamda/VSMEC}
% \end{figure}
% \begin{figure}[H]
%     \centering
%     \subfigure[Constructiveness task]{\includegraphics[width=0.49\textwidth]{ldViCTSD_Constructiveness.pdf}}
%     \subfigure[Toxicity task]{\includegraphics[width=0.49\textwidth]{ldViCTSD_toxic.pdf}}
%     \caption{F1-score of ViCGCN when varying $\lambda$ on UIT-ViCTSD dev set.}
%     \label{fig::Experiments/Lamda/ViCTSD}
% \end{figure}
\begin{figure}[!hpt]
    \centering

    \subfigure[Seven labels task]{\includegraphics[width=0.49\textwidth]{ldVSMEC_7.pdf}}
    \subfigure[Six labels task]{\includegraphics[width=0.49\textwidth]{ldVSMEC_6.pdf}}

     \subfigure[Constructiveness task]{\includegraphics[width=0.49\textwidth]{ldViCTSD_Constructiveness.pdf}}
    \subfigure[Toxicity task]{\includegraphics[width=0.49\textwidth]{ldViCTSD_toxic.pdf}}
    
    \subfigure[Sentiment-based task]{\includegraphics[width=0.49\textwidth]{ldVSFC_sentiment.pdf}}
    \subfigure[Topic-based task]{\includegraphics[width=0.49\textwidth]{ldVSFC_topic.pdf}}
    % \caption{F1-score of ViCGCN when varying $\lambda$ on UIT-VSFC dev set.}
    
    \caption{F1-score of ViCGCN when varying $\lambda$ on the dev set.}
    
    \label{fig::Experiments/Lamda/UIT-VSFC}
\end{figure}

\subsubsection{Comparison with Previous Studies}
\label{comparisonprestudies}
% \subsubsection{UIT-VSMEC}
We conducted a number of surveys to evaluate how well our suggested technique performed in comparison to earlier studies. On the UIT-VSMEC, UIT-ViCTSD, and UIT-VSFC datasets, our method fared better than in any prior research. To provide for fair comparisons, similar evaluation metrics from earlier studies are employed. For all datasets used in this study, we use the average macro F1-score (\%) and average weighted F1-score (\%). 

Our integrated model ViCGCN outperformed the best results of each previous study on the VSMEC dataset by achieving 80.24\% weighted F1-score and 80.96\% macro F1-score on task Seven labels, which improves by 10.18\% and 13.93\% compared to the best previous study. Additionally, our model obtains 84.91\% weighted F1-score on the Six labels task as shown in Table \ref{tab::Experiments/Comparison/VSMEC}, increased by 13.92\% in comparison to the highest previous ones. Furthermore, Table \ref{tab::Experiments/Comparison/ViCTSD} deputed that ViCGCN achieves the best results, with a macro F1-score of 85.81\% for UIT-ViCTSD Constructiveness task, and 76.29\% macro F1-score for UIT-ViCSTD Toxicity task,  increased by 16.89\%. By obtaining 88.80\% macro F1-score and 94.81\% weighted F1-score, 89.61\% macro F1-score, and 93.81\% weighted F1-score on task Sentiment-based and Topic-based, respectively, our integrated model ViCGCN surpassed every previous study's top result on the UIT-VSFC dataset as describes in Table \ref{tab::Experiments/Comparison/VSFC}. In addition, our proposed approach reached new state-of-the-art performances on three Vietnamese benchmark social media datasets, UIT-VSMEC, UIT-ViCTSD, and UIT-VSFC, respectively. As a result, the proposed approach ViCGCN is significantly suitable and efficient for dealing with Vietnamese text in general and Vietnamese social media text classification tasks in particular.

\begin{table}[!hpt]
\centering
\caption{The comparison with previous studies on UIT-VSMEC.} \label{tab::Experiments/Comparison/VSMEC}
\resizebox{\linewidth}{!}{
\begin{tabular}{l|cc|cc} 
\hline
\textbf{Tasks}                                  & \multicolumn{2}{c|}{\textbf{Seven labels }} & \multicolumn{2}{c}{\textbf{Six labels }}  \\ 
\hline
                                                & \textbf{wF1}   & \textbf{mF1}               & \textbf{wF1}   & \textbf{mF1}             \\ 
\hline
CNN + Word2Vec                                  & 59.74          & -                          & 66.34          & -                        \\
MLR + TF-IDF Vectorizer + Key-clause extraction & 64.40          & -                          & -              & -                        \\
GRU + CNN + BiLSTM + LSTM                       & 65.79          & -                          & 70.99          & -                        \\
PhoBERT                                         & -              & 65.44                      & -              & -                        \\
XLM-R + VnEmolex                                & 70.06          & 67.03                      & -              & -                        \\ 
\hline
\textbf{ViCGCN (base)}                          & \textbf{70.32} & \textbf{67.17}             & \textbf{71.02} & \textbf{67.48}           \\
\textbf{ViCGCN (large)}                         & \textbf{71.33} & \textbf{67.82}             & \textbf{72.08} & \textbf{68.12}           \\
\hline
\end{tabular}}
\end{table}


% \subsubsection{ViCTSD}
\begin{table}[!ht] 
\centering
\caption{The comparison with previous studies on UIT-ViCTSD.}
\label{tab::Experiments/Comparison/ViCTSD}
\begin{tabular}{l|cc|cc} 
\hline
\textbf{Tasks}          & \multicolumn{2}{c|}{\textbf{Constructiveness }} & \multicolumn{2}{c}{\textbf{Toxicity }}  \\ 
\hline
                        & \textbf{wF1}   & \textbf{mF1}                   & \textbf{wF1}   & \textbf{mF1}           \\ 
\hline
PhoBERT                 & -              & 78.59                          & -              & 59.40                  \\
viBERT4news             & -              & 84.15                          & -              & -                      \\ 
\hline
\textbf{ViCGCN (base)}  & \textbf{85.64} & \textbf{85.12}                 & \textbf{92.22} & \textbf{75.32}         \\
\textbf{ViCGCN (large)} & \textbf{86.12} & \textbf{85.88}                 & \textbf{93.11} & \textbf{76.12}         \\
\hline
\end{tabular}
\end{table}

% \subsubsection{VSFC}
\begin{table}[!ht] 
\centering
\caption{The comparison with previous studies on UIT-VSFC.}
\label{tab::Experiments/Comparison/VSFC}
\resizebox{\linewidth}{!}{
\begin{tabular}{l|cc|cc} 
\hline
\textbf{Tasks}             & \multicolumn{2}{c|}{\textbf{Sentiment-based }} & \multicolumn{2}{c}{\textbf{Topic-based }}  \\ 
\hline
                           & \textbf{wF1}   & \textbf{mF1}                  & \textbf{wF1}   & \textbf{mF1}              \\ 
\hline
Maximum Entropy            & 87.64          & -                             & 84.03          & -                         \\
BiLSTM +Word2Vec~          & 92.03          & -                             & 89.62          & -                         \\
LD + SVM ()                & 92.20          & -                             & -              & -                         \\
BERT + CNN + BiLSTM + LSTM & 92.79          & -                             & 89.38          & -                         \\
BERT + CNN + BiLSTM        & 92.13          & -                             & 89.70          & -                         \\
XLM-R + VnEmoLex           & 93.97          & 83.40                         & -              & -                         \\ 
\hline
\textbf{ViCGCN (base)}     & \textbf{94.12} & \textbf{83.67}                & \textbf{90.12} & \textbf{80.11}            \\
\textbf{ViCGCN (large)}    & \textbf{94.83} & \textbf{84.23}                & \textbf{91.02} & \textbf{81.88}            \\
\hline
\end{tabular}}
\end{table}

% SECTION Errors and analysis %
\subsubsection{Errors Analysis}
\label{erroranalysis}

We utilize the error analysis of ViCGCN, our top-performing model, to analyze the errors observed in our proposed model. Figure\ref{fig::Experiments/CfMatrix/VSMEC}, Figure \ref{fig::Experiments/CfMatrix/ViCTSD} and Figure \ref{fig::Experiments/CfMatrix/VSFC}, respectively, show the confusion matrices for our best model's predictions on the test set for UIT-VSMEC, UIT-ViCTSD, and UIT-VSFC.

\begin{figure}[!ht]
    \centering
    \subfigure[Seven labels task]{\includegraphics[width=0.49\textwidth]{cfVSMEC_7.pdf}}
    \subfigure[Six labels task]{\includegraphics[width=0.49\textwidth]{cfVSMEC_6.pdf}}
    \caption{Error analysis of our proposed approach for UIT-VSMEC dataset.}
    \label{fig::Experiments/CfMatrix/VSMEC}
\end{figure}
\begin{figure}[!ht]
    \centering
    \subfigure[Constructiveness task]{\includegraphics[width=0.49\textwidth]{cfViCTSD_contructive.pdf}}
    \subfigure[Toxicity task]{\includegraphics[width=0.49\textwidth]{cfViCTSD_toxic.pdf}}
    \caption{Error analysis of our proposed approach for UIT-ViCTSD dataset.}
    \label{fig::Experiments/CfMatrix/ViCTSD}
\end{figure}
\begin{figure}[!ht]
    \centering
    \subfigure[Sentiment-based task]{\includegraphics[width=0.49\textwidth]{cfVSFC_sentiment.pdf}}
    \subfigure[Topic-based task]{\includegraphics[width=0.49\textwidth]{cfVSFC_topic.pdf}}
    \caption{Error analysis of our proposed approach for UIT-VSFC dataset.}
    \label{fig::Experiments/CfMatrix/VSFC}
\end{figure}

As described in Section \ref{Proposed model}, by incorporating contextualized language models such as BERT into GCN, ViCGCN can better capture the context and meaning of words and phrases, which can lead to more accurate identification of critical nodes. However, ViCGCN may not be able to explain why those nodes are essential or why specific nodes were not influential in the decision-making process. This can make it difficult for researchers to address specific issues in our proposed approach. Table \ref{fig:erroranalysissampleonViCTSD}, Table \ref{fig:erroranalysissampleonVSMEC}, and Table \ref{fig:erroranalysissampleonVSFC} contain a few illustrations of prediction errors. The results show that misclassifications were primarily due to the use of sarcasm, irony, and figurative language in social media comments. Furthermore, some misclassifications were due to the presence of multiple topics in a single comment, making it challenging to identify the primary intention. Additionally, ambiguity in identifying the labels of the datasets also leads to misclassifying of our proposed approach ViCGCN.

% \subsubsection{VSMEC dataset} \label{Errors/VSMEC}


\begin{table}[H]
\centering
\caption{Several examples of classification error on UIT-VSMEC dataset.}\label{fig:erroranalysissampleonVSMEC}
\resizebox{\linewidth}{!}{%
\begin{tblr}{
  row{1} = {c},
  cell{2}{2} = {c},
  cell{2}{3} = {c},
  cell{3}{2} = {c},
  cell{3}{3} = {c},
  hline{1-2,4} = {-}{},
}
\textbf{Comment}                                                       & \textbf{True Label} & \textbf{Predicted Label} \\
{mấy ai được như vậy\\(\textbf{English:} not many people can do that)} & other               & surprise               \\
{kinh khủng thật\\(\textbf{English:} it's terrible)}                   & fear                & sadness                
\end{tblr}}
\end{table}

% \subsubsection{ViCTSD dataset} \label{Errors/ViCTSD}

\begin{table}[!ht]
\centering
\caption{Several examples of classification error on UIT-ViCTSD dataset.}\label{fig:erroranalysissampleonViCTSD}
\resizebox{\linewidth}{!}{%
\begin{tblr}{
  row{1} = {c},
  cell{2}{2} = {c},
  cell{2}{3} = {c},
  cell{3}{2} = {c},
  cell{3}{3} = {c},
  hline{1-2,4} = {-}{},
}
\textbf{Comment}                                                                                                                                           & \textbf{True Label} & \textbf{Predicted Label} \\
{Người ăn không hết kẻ lần không ra\\(\textbf{English:} This man has much to eat but that \\
man finds no small piece.)}                                       & non\_constructive   & constructive           \\
{người trẻ còn sức khoẻ k lo làm ăn đi ăn trộm\\(\textbf{English:} Young people who are still healthy \\ don't worry about doing business but go to steal)} & non\_toxic          & toxic                  
\end{tblr}}
\end{table}


% % \subsubsection{UIT-VSFC dataset} \label{Errors/VSFC}

\begin{table}[!ht]
\centering
\caption{Several examples of classification error on UIT-VSFC dataset.}\label{fig:erroranalysissampleonVSFC}
\resizebox{\linewidth}{!}{%
% \centerline{
\begin{tblr}{
  row{1} = {c},
  cell{2}{2} = {c},
  cell{2}{3} = {c},
  cell{3}{2} = {c},
  cell{3}{3} = {c},
  hline{1-2,4} = {-}{},
}
\textbf{Comment}                                                                                                                                          & \textbf{True Label} & \textbf{Predicted Label} \\
{ví dụ phù hợp với nội dung kiến thức , hướng dẫn chi tiết\\(\textbf{English:}~Examples are consistent with content knowledge, \\ detailed instructions)} & neural          & positive           \\
{đảm bảo chất lượng tốt\\(\textbf{English:}~Good quality guarantee)}                                                                                               & others          & facility topic     
\end{tblr}}
% }
\end{table}
%%


\subsubsection{Ablation Study}
\label{ablationstudy}

\begin{table}[H]
\centering
\caption{Ablation test on our proposed approach. w/o GCN and w/o PhoBERT denoted the result of the ablation GCN and the result of the ablation PhoBERT, respectively}
\label{tab::Ablation}
\resizebox{\linewidth}{!}{%
% \centerline{
\begin{tabular}{l|cc|cc|cc|cc|cc|cc} 
\hline
\textbf{Datasets} & \multicolumn{4}{c|}{\textbf{VSMEC}}                                                                          & \multicolumn{4}{c|}{\textbf{ViCTSD}}                                                                         & \multicolumn{4}{c}{\textbf{VSFC}}                                                                \\ 
\hline
\textbf{Tasks}    & \multicolumn{2}{c|}{\textbf{Seven labels}}            & \multicolumn{2}{c|}{\textbf{Six labels}}             & \multicolumn{2}{c|}{\textbf{Constructiveness}}       & \multicolumn{2}{c|}{\textbf{Toxicity}}                & \multicolumn{2}{c|}{\textbf{Sentiment-based}}        & \multicolumn{2}{c}{\textbf{Topic-based}}  \\ 
\hline
                  & \multicolumn{1}{c|}{\textbf{wF1}} & \textbf{mF1}      & \multicolumn{1}{c|}{\textbf{wF1}} & \textbf{mF1}     & \multicolumn{1}{c|}{\textbf{wF1}} & \textbf{mF1}     & \multicolumn{1}{c|}{\textbf{wF1}} & \textbf{mF1}      & \multicolumn{1}{c|}{\textbf{wF1}} & \textbf{mF1}     & \textbf{wF1}     & \textbf{mF1}           \\ 
\hline
\multicolumn{13}{c}{\textbf{ViCGCN}}                                                                                                                                                                                                                                                                                                               \\ 
\hline
Performance       & \textbf{71.33}                    & \textbf{67.82}    & \textbf{72.08}                    & \textbf{68.12}   & \textbf{86.12}                    & \textbf{85.88}   & \textbf{93.11}                    & \textbf{76.12}    & \textbf{94.83}                    & \textbf{84.23}   & \textbf{91.02}   & \textbf{81.88}         \\ 
\hline
\multicolumn{13}{c}{\textbf{w/o GCN}}                                                                                                                                                                                                                                                                                                              \\ 
\hline
Performance       & 65.12                             & 63.23             & 71.13                             & 65.12            & 81.03                             & 79.53            & 90.12                             & 73.32             & 93.24                             & 82.96            & 88.72            & 79.12                  \\
Decrease          & $\downarrow$6.21                  & $\downarrow$4.59  & $\downarrow$0.95                  & $\downarrow$3.00 & $\downarrow$5.09                  & $\downarrow$6.35 & $\downarrow$2.99                  & $\downarrow$2.80  & $\downarrow$1.59                  & $\downarrow$1.27 & $\downarrow$2.30 & $\downarrow$2.76       \\ 
\hline
\multicolumn{13}{c}{\textbf{w/o PhoBERT}}                                                                                                                                                                                                                                                                                                          \\ 
\hline
Performance       & 52.32                             & 51.32             & 61.34                             & 58.42            & 79.63                             & 78.37            & 87.63                             & 64.32             & 88.32                             & 75.32            & 85.36            & 75.21                  \\
Decrease          & $\downarrow$19.01                 & $\downarrow$16.50 & $\downarrow$10.74                 & $\downarrow$9.70 & $\downarrow$6.49                  & $\downarrow$7.51 & $\downarrow$5.48                  & $\downarrow$11.80 & $\downarrow$6.51                  & $\downarrow$8.91 & $\downarrow$5.66 & $\downarrow$6.67       \\
\hline
\end{tabular}}
\end{table}

Our proposed method is considerably more effective than most current techniques for classifying text on social media. Ablation experiments were carried out on the proposed approach to prove the effectiveness of these two modules, PhoBERT and GCN. Table \ref{tab::Ablation} shows the ablation experiment results of the text classification module. For the model with GCN ablation, the experimental results are inferior to the model without ablation. While results of the \textit{w/o PhoBERT} model are not as good as those of the model with the contextualized pre-trained language model. The results of the ablation experiments demonstrate the effectiveness of the proposed importance of each module in general, as well as the combination of our proposed approach in particular. Our proposed approach, especially contextualized language models Integrated with graph neural networks, yield promising outcome for improving performance in further study. As a result, we conclude that all proposed modules are crucial in text classification on social media.

\section{Conclusions}
\label{sec:conclusions}
\section{Limitations and Open Questions}
\label{sec:limitations}
Though we have proposed two effective non-``detect-then-describe'' methods for 3D dense captioning, the captions do not have much diversity because of the limited text annotations, beam search, and self-critical sequence training with the CiDEr reward.
% 
We believe that multi-modal pre-training on 3D vision-language tasks with more training data and the utilization of \textbf{L}arge \textbf{L}anguage \textbf{M}odels(LLM) trained on large corpus would increase the diversity of the generated captions.
% 
Additionally, other reward functions designed for 3D dense captioning will increase the diversity among object descriptions in the same scene.
% 
We will leave these topics for future study.


\section{Conclusions}
\label{sec:conclusion}
%
\whatsnew{
In this work, we decouple the caption generation from caption generation, and propose a set of two transformer-based approaches, namely Vote2Cap-DETR and Vote2Cap-DETR++, for 3D dense captioning.
%
Comparing with the sophisticated and explicit relation modules in conventional ``detect-then-describe'' pipelines, our proposed methods efficiently capture the object-object and object-scene relation through the attention mechanism.
%
The preliminary model, Vote2Cap-DETR, decouples the decoding process to generate captions and box estimations in parallel.
% 
We also propose vote queries for fast convergence, and develop a novel lightweight query-driven caption head for informative caption generation.
% 
In the advanced model, Vote2Cap-DETR++, we further decouple the queries to capture task-specific features for object localization and description generation.
% 
Additionally, we introduce an iterative spatial refinement strategy for vote queries, and insert 3D spatial information for more accurate captions.
%
Extensive experiments on two widely used datasets validate that both the proposed methods surpass prior ``detect-then-describe'' pipelines by a large margin.
}

\subsubsection*{Acknowledgements}

This work has been supported by the French government, through the 3IA C\^ote d'Azur, Investment in the Future, project managed by the National Research Agency (ANR) with the reference number ANR-19-P3IA-0002. We would also like to thank the France Canada Research Fund (FFCR) for their contribution to the project. This work was partly supported by the Fonds de recherche du Québec – Santé (FRQS) and the Health-Data Hub, through the joint project AORTIC STENOSIS . The authors are grateful to the OPAL infrastructure from Université Côte d'Azur for providing resources and support.

\bibliographystyle{abbrv}
\bibliography{bib}\vfill
%[{\includegraphics[width=1in,height=1.25in,keepaspectratio]{lohl.png}}]
\begin{IEEEbiography}[{\includegraphics[width=1in,height=1.25in,keepaspectratio]{photos/lohl.jpeg}}]{Louis Ohl} %clip param removed
PhD Student at the Universit\'e C\^ote d'Azur and the Universit\'e Laval from 2021 to 2024 and computer science graduate from the INSA Lyon and KTH Royal Institute of Technology in 2021. His project mainly focuses on clustering and the choices of the decision boundaries to discriminatively separate samples with applications in medical research including identification of phenogroups in aortic stenosis.
\end{IEEEbiography}\vspace{-2\baselineskip}
\begin{IEEEbiography}[{\includegraphics[width=1in,height=1.25in,keepaspectratio]{photos/pamattei.png}}]{Pierre-Alexandre Mattei}  is a research scientist at Inria, in the Maasai team located in Sophia-Antipolis (France). He graduated from Ecole Normale Supérieure de Cachan in 2014, and received a PhD in applied mathematics from Université Paris Cité in 2017. His research interests mostly revolve around problems with latent variables: deep generative modelling, Bayesian inference, clustering, missing data.
\end{IEEEbiography}\vspace{-2\baselineskip}
%[{\includegraphics[width=1in,height=1.25in,keepaspectratio]{photos/cbouveyron.png}}]
\begin{IEEEbiographynophoto}{Charles Bouveyron} Professor of Statistics at Université Côte d'Azur, Nice, France. He holds a Chair on Artificial Intelligence and is the Director of the Institut 3IA Côte d'Azur and head of the research team MAASAI from INRIA. His research interests include statistical learning for clustering, classification and regression in high dimension, statistical learning on networks applied to different domains (medicine, image analysis, astrophysics, humanities...)    
\end{IEEEbiographynophoto}\vspace{-2\baselineskip}
\begin{IEEEbiography}[{\includegraphics[width=1in,height=1.25in,keepaspectratio]{photos/wharchaoui.png}}]{Warith Harchaoui} holds a Ph.D. in Applied Mathematics and is deeply engaged in the intersection of artificial intelligence (AI) and pattern analysis.
He commenced his academic journey at École Normale Supérieure de Cachan, where he received his M.Sc. in 2008. His initial forays into the industry revolved around Computer Vision applications in startups and global enterprises.
Harchaoui extended his academic pursuits at École Normale Supérieure de Paris, contributing to research at the globally recognized Willow laboratory. His scholarly interests encompass a range of topics within image, sound, video, and text processing, a domain where he has consistently leveraged mathematical models for innovative solutions.
From 2014 to 2020, Dr. Harchaoui also embraced a corporate role in Data Science, working with Oscaro.com the e-commerce leader in the automotive parts sector. Balancing industry demands with academic rigor, he completed his Ph.D. research between 2016 and 2020.
Today, Dr. Harchaoui serves at Jellysmack since 2021, where he combines his expertise to help internet video creators through every stage from content production, editing to social media promotion.
\end{IEEEbiography}\vspace{-2\baselineskip}
\begin{IEEEbiography}[{\includegraphics[width=1in,height=1.25in,keepaspectratio]{photos/mleclercq.jpg}}]{Mickaël Leclercq}
PhD in Computer Science from McGill University, Canada, in 2016. Since then, he's been working as a research associate in the computational biology laboratory of Prof. Droit, researcher at the CHU de Québec - Université Laval. His role is to develop and supervise bioinformatics research projects with an artificial intelligence component. He also contributes to the organization and coordination of the laboratory's research activities and supervises the work of students and young research professionals. He actively participates in the long-term vision of the laboratory's research themes as well as in the development of grant applications and manuscripts.
\end{IEEEbiography}\vfill
\begin{IEEEbiography}[{\includegraphics[width=1in,height=1.25in,keepaspectratio]{photos/adroit.png}}]{Arnaud Droit}
PhD graduate in 2007 in bioinformatics from Université Laval, Canada. Since 2012, he has been a Professor at Université Laval (Québec, Canada) and leads a bioinformatics and proteomics laboratory at the CHU de Québec. His research focuses on robust bioinformatics approaches, advanced statistical methods, and machine/deep learning strategies to extract information relevant to medical research from large biological datasets. Many of his projects are related to cancer research, where the aim is to discover biomarker signatures associated with various clinical outcomes. His expertise is essential to many research sectors requiring bioinformatics, so he built a large network of international collaborations, including the private sector. 
\end{IEEEbiography}\vspace{-2\baselineskip}
\begin{IEEEbiographynophoto}{Frederic Precioso} %clip param removed
Full Professor at the Universit\'e C\^ote d'Azur from 2011 and Affiliated Professor at Universit\'e Laval from 2022, his research interests cover active learning, foundation models, hybrid learning combining symbolic and non-symbolic models, clustering, MLOps, applied to different domains (health, biology, multimedia, autonomous vehicles, biodiversity).
\end{IEEEbiographynophoto}\vfill

\newpage
\onecolumn

\appendices
\section{Demonstration of the convergence to 0 of the MI for a Gaussian Mixture}
\label{app:mi_convergence}
\newcommand{\nmuA}{\mathcal{N}(x|\mu_0,\sigma^2)}
\newcommand{\nmuB}{\mathcal{N}(x|\mu_1,\sigma^2)}

\label{app:mi_boundaries}
\subsection{Models definition}
Let us consider a mixture of two Gaussian distributions, both with different means $\mu_0$ and $\mu_1$, s.t. $\mu_0<\mu_1$ and of same standard deviation $\sigma$:

\begin{equation*}
p(x|y=0) = \nmuA, p(x|y=1) = \nmuB ,
\end{equation*}

where $y$ is the cluster assignment. We take balanced clusters proportions, i.e. $p(y=0)=p(y=1)=\frac{1}{2}$. This first model is the basis that generated the complete dataset $p(x)$. When performing clustering with our discriminative model, we are not aware of the distribution. Consequently: we create other models. We want to compute the difference of mutual information between two decision boundaries that discriminative models $p_\theta(y|x)$ may yield.

We define two decision boundaries: one which splits evenly the data space called $p_A$ and another which splits it on a closed set $p_B$:

\begin{equation}\label{eq:p_a}
p_A(y=1|x) = \left\{ \begin{array}{c c}
1-\epsilon&x>\frac{\mu_1-\mu_0}{2}\\
\epsilon&\text{otherwise}
\end{array}\right. ,
\end{equation}

\begin{equation}\label{eq:p_b}
p_B(y=1|x) = \left\{ \begin{array}{c c}
1-\epsilon&x \in [\mu_0, \mu_1]\\
\epsilon&\text{otherwise}
\end{array}\right. .
\end{equation}

Our goal is to show that both models $p_A$ and $p_B$ will converge to the same value of mutual information as $\epsilon$ converges to 0.

\subsection{Computing cluster proportions}
\subsubsection{Cluster proportion of the correct decision boundary}

To compute the cluster proportions, we estimate with samples $x$ from the distribution $p_\text{data}(x)$. Since we are aware for this demonstration of the true nature of the data distribution, we can use $p(x)$ for sampling. Consequently, we can compute the two marginals:

\begin{align*}
p_A(y=1) &= \int_\mathcal{X} p(x) p_A(y=1|x)dx ,\\
&= \int_{-\infty}^{\frac{\mu_1-\mu_0}{2}}p(x)\epsilon dx + \int_{\frac{\mu_1-\mu_0}{2}}^{+\infty}p(x)(1-\epsilon)dx ,\\
&=\epsilon \left(\int_{-\infty}^{\frac{\mu_1-\mu_0}{2}}p(x)dx \right) + (1-\epsilon) \left( \int_{\frac{\mu_1-\mu_0}{2}}^{+\infty}p(x)dx\right) ,\\
&= \frac{1}{2} .
\end{align*}

\subsubsection{Cluster proportion of the misplaced decision boundary}
For the misplaced decision boundary, the marginal is different:

\begin{equation*}\begin{split}
p_B(y=1) &= \int_\mathcal{X} p(x) p_B(y=1|x)dx ,\\
&= \epsilon\left(\int_{-\infty}^{\mu_0}p(x)dx + \int_{\mu_1}^{+\infty}p(x)dx\right) + (1-\epsilon) \int_{\mu_0}^{\mu_1}p(x)dx ,\\
&=\epsilon \left(1-\int_{\mu_0}^{\mu_1}p(x)dx\right) + (1-\epsilon)\int_{\mu_0}^{\mu_1}p(x)dx .
\end{split}\end{equation*}

Here, we simply introduce a new variable named $\beta$ that will be a shortcut for noting the proportion of data between $\mu_0$ and $\mu_1$:

\begin{equation*}%\label{eq:beta_definition}
\beta = \int_{\mu_0}^{\mu_1}p(x)dx .
\end{equation*}

And so can we simply write the cluster proportion of decision boundary model B as:

\begin{equation*}\label{eq:py_b}
p_B(y=1) = \epsilon (1-\beta) + (1-\epsilon)\beta ,
\end{equation*}

Leading to the summary of proportions in Table~\ref{tab:proportions}. For convenience, we will write the proportions of model B using the shortcuts:

\begin{equation*}\label{eq:pi_b}
\pi_B = p_B(y=1) = \epsilon + \beta(1-2\epsilon) ,
\end{equation*}
\begin{equation*}%\label{eq:pibar_b}
\bar{\pi}_B = p_B(y=0) = 1-\epsilon - \beta(1-2\epsilon) .
\end{equation*}

\begin{table}[hbt]
\caption{Proportions of clusters for models A and B}\label{tab:proportions}
\centering
\begin{tabular}{c c c}
\toprule
$\mathcal{M}$&A&B\\
\cmidrule(lr){2-3}
$p_\mathcal{M}(y=1)$&$\frac{1}{2}$&$\epsilon+\beta(1-2\epsilon)$\\
$p_\mathcal{M}(y=0)$&$\frac{1}{2}$&$1-\epsilon-\beta(1-2\epsilon)$\\
\bottomrule
\end{tabular}
\end{table}

\subsection{Computing the KL divergences}
\subsubsection{Correct decision boundary}

We first start by computing the Kullback-Leibler divergence for some arbitrary value of $x\in\mathbb{R}$:

\begin{equation*}
D_\text{KL}(p_A(y|x)||p_A(y))= \sum_{i=0}^1 p_A(y=i|x) \log{\frac{p_A(y=i|x)}{p_A(y=i)}} .
\end{equation*}

We now need to detail the specific cases, for the value of $p(y=i|x)$ is dependent on $x$. We start $\forall x <\frac{\mu_1-\mu_0}{2}$:

\begin{align*}
D_\text{KL}(p_A(y|x)||p_A(y))&= p_A(y=0|x)\log{\frac{p_A(y=0|x)}{\frac{1}{2}}} + p_A(y=1|x)\log{\frac{p_A(y=1|x)}{\frac{1}{2}}} ,\\
&=(1-\epsilon)\log{2(1-\epsilon)} + \epsilon \log{2\epsilon} .
\end{align*}

The opposite case, $\forall x\geq \frac{\mu_1-\mu_0}{2}$ yields:
\begin{align*}
D_\text{KL}(p_A(y|x)||p_A(y))&= p_A(y=0|x)\log{\frac{p_A(y=0|x)}{\frac{1}{2}}} + p_A(y=1|x)\log{\frac{p_A(y=1|x)}{\frac{1}{2}}} ,\\
&=\epsilon\log{2\epsilon} + (1-\epsilon) \log{2(1-\epsilon)} .
\end{align*}

Since both cases are equal, we can write down:

\begin{equation}\label{eq:correct_kl_div}
D_\text{KL}(p_A(y|x)||p_A(y)) = \epsilon\log{2\epsilon} + (1-\epsilon)\log{2(1-\epsilon)} ,\forall x\in\mathbb{R} .
\end{equation}

\subsubsection{Misplaced boundary}

We proceed to the same detailing of the Kullback-Leibler divergence computation for the misplaced decision boundary. We start with the set $x\in [\mu_0,\mu_1]$:

\begin{align*}
D_\text{KL}(p_B(y|x)||p_B(y))&= p_B(y=0|x) \log{\frac{p_B(y=0|x)}{p_B(y=0)}} + p_B(y=1|x)\log{\frac{p_B(y=1|x)}{p_B(y=1)}} ,\\
&=\epsilon \log{\frac{\epsilon}{\bar{\pi}_B}} + (1-\epsilon) \log{\frac{1-\epsilon}{\pi_B}} .
\end{align*}

When $x$ is out of this set, the divergence becomes:

\begin{align*}
D_\text{KL}(p_B(y|x)||p_B(y))&=p_B(y=0|x) \log{\frac{p_B(y=0|x)}{p_B(y=0)}} + p_B(y=1|x)\log{\frac{p_B(y=1|x)}{p_B(y=1)}} ,\\
&=(1-\epsilon)\log{\frac{1-\epsilon}{\bar{\pi}_B}} + \epsilon \log{\frac{\epsilon}{\pi_B}} .
\end{align*}

To fuse the two results, we will write the KL divergence as such:

\begin{equation}\label{eq:odd_kl_div}
D_\text{KL}(p_B(y|x)||p_B(y)) = \epsilon\log{\epsilon}+ (1-\epsilon)\log(1-\epsilon) - C(x) ,\forall x\in\mathbb{R} ,
\end{equation}

where $C(x)$ is a constant term depending on $x$ defined by:

\begin{equation*}\label{eq:odd_constant}
C(x)=\left\{\begin{array}{c c}
\epsilon\log{\bar{\pi}_B} + (1-\epsilon)\log{\pi_B}&x\in[\mu_0,\mu_1]\\
\epsilon\log{\pi_B} + (1-\epsilon)\log{\bar{\pi}_B}&x\in\mathbb{R}\setminus[\mu_0,\mu_1]\\
\end{array}  .\right.
\end{equation*}

For simplicity of later writings, we will shorten the notations by:

\begin{equation*}
C(x)=\left\{ \begin{array}{c c}
\alpha_1&x\in[\mu_0,\mu_1]\\
\alpha_0&x\in\mathbb{R}\setminus[\mu_0,\mu_1]
\end{array}\right. .
\end{equation*}

\subsection{Evaluating the mutual information}

\subsubsection{Correct decision boundary}

We inject the value of the Kullback-Leibler divergence from Eq.~(\ref{eq:correct_kl_div}) inside an expectation performed over the data distribution $p_\text{data}(x)$:

\begin{align}
\mathcal{I}_A(x;y) &= \mathbb{E}_{x\sim p_\text{data}(x)}\left[ D_\text{KL}(p_A(y|x)||p_A(y))\right] ,\\
&=\int_\mathcal{X}p(x) \left(\epsilon \log(2\epsilon) + (1-\epsilon)\log(2(1-\epsilon)) \right)dx ,\\
&=\epsilon \log(2\epsilon) + (1-\epsilon)\log(2(1-\epsilon)) .\label{eq:mi_a}
\end{align}

Since the KL divergence was independent of $x$, we could leave the constant outside of the integral which is equal to 1.

We can assess the coherence of Eq.~(\ref{eq:mi_a}) since its limit as $\epsilon$ approaches 0 is $\log 2$. In terms of bits, this is the same as saying that the information on $X$ directly gives us information on the $Y$ of the cluster.

\subsubsection{Odd decision boundary}

We inject the value of the KL divergence from Eq.~(\ref{eq:odd_kl_div}) inside the expectation of the mutual information:

\begin{align*}
\mathcal{I}_B(x;y) &= \mathbb{E}_{x\sim p_\text{data}(x)}\left[ D_\text{KL}(p_B(y|x)||p_B(y))\right] ,\\
&=\int_{\mathcal{X}}p(x)\left(\epsilon\log\epsilon +(1-\epsilon)\log(1-\epsilon) -C(x)\right) ,dx\\
&=\epsilon\log\epsilon +(1-\epsilon)\log(1-\epsilon) - \int_{\mathcal{X}}p(x)C(x)dx .
\end{align*}

The first terms are constant with respect to $x$ and the integral of $p(x)$ over $\mathcal{X}$ adds up to 1. We finally need to detail the expectation of the constant $C(x)$ from Eq.~(\ref{eq:odd_constant}):

\begin{align*}
\mathbb{E}_x[C(x)] &= \int_{-\infty}^{\mu_0}C(x)p(x)dx + \int_{\mu_0}^{\mu_1}C(x)p(x)dx + \int_{\mu_1}^{+\infty}C(x)p(x)dx ,\\
&= \alpha_0\left(\int_{-\infty}^{\mu_0}p(x)dx + \int_{\mu_1}^{+\infty}p(x)dx \right) + \alpha_1\int_{\mu_0}^{\mu_1}p(x)dx ,\\
&=\alpha_0(1-\beta) + \alpha_1\beta .
\end{align*}

This can be further improved by unfolding the description of $\alpha_0$ and $\alpha_1$ from Eq.~(\ref{eq:odd_constant}):

\begin{align*}
\alpha_0(1-\beta)+\beta\alpha_1&= \alpha_0 +\beta(\alpha_1-\alpha_0),\\
&=\epsilon\log{\pi_B}+(1-\epsilon)\log{\bar{\pi}_B} + \beta\left[\epsilon\log{\bar{\pi}_B}+(1-\epsilon)\log{\pi_B}\right.\\&\quad\left.-\epsilon\log{\pi_B}- (1-\epsilon)\log{\bar{\pi}_B}\right],\\
&= \left[1-\epsilon+\beta\epsilon-\beta+\beta\epsilon\right]\log{\bar{\pi}_B} + \left[\epsilon+\beta-\beta\epsilon-\beta\epsilon\right]\log{\pi_B},\\
&=\log{\bar{\pi}_B} + \left[2\beta\epsilon-\beta-\epsilon\right]\log{\frac{\bar{\pi}_B}{\pi_B}} .
\end{align*}

We can finally write down the mutual information for the model B:

\begin{equation*}\label{eq:mi_b}
\mathcal{I}_B(x;y) = \epsilon\log{\epsilon} +(1-\epsilon)\log(1-\epsilon) -\log{\bar{\pi}_B} - \left[2\beta\epsilon-\beta-\epsilon\right]\log{\frac{\bar{\pi}_B}{\pi_B}} .
\end{equation*}

\subsection{Differences of mutual information}

Now that we have the exact value of both mutual informations, we can compute their differences:

\begin{align*}
\Delta_\mathcal{I} &= \mathcal{I}_A(x;y) - \mathcal{I}_B(x;y),\\
&= \epsilon\log(2\epsilon) + (1-\epsilon)\log(2(1-\epsilon)) - \epsilon\log{\epsilon} -(1-\epsilon)\log(1-\epsilon)\\&\quad+\log{\bar{\pi}_B} + \left[2\beta\epsilon-\beta-\epsilon\right]\log{\frac{\bar{\pi}_B}{\pi_B}} ,\\
&=\epsilon\log{2}+(1-\epsilon)\log2 +\log{\bar{\pi}_B} + \left[2\beta\epsilon-\beta-\epsilon\right]\log{\frac{\bar{\pi}_B}{\pi_B}} .
\end{align*}

We then deduce how the difference of mutual information evolves as the decision boundary becomes sharper, i.e. when $\epsilon$ approaches 0:

\begin{equation*}
\lim_{\epsilon\rightarrow 0}\Delta_\mathcal{I} = \log{2} + \log{\bar{\pi}_B} -\beta\log{\frac{\bar{\pi}_B}{\pi_B}} .
\end{equation*}

However, the cluster proportions by B $\pi_B$ also take a different value as $\epsilon$ approaches 0. Recalling Eq.~(\ref{eq:p_b}):

\begin{equation*}
\lim_{\epsilon\rightarrow 0} p_B(y=1) = \beta , \lim_{\epsilon\rightarrow 0} p_B(y=0) = 1-\beta .
\end{equation*}

And finally can we write that:

\begin{equation*}\begin{split}
\lim_{\epsilon\rightarrow 0} \Delta_\mathcal{I} &= \log{2} + \log(1-\beta) - \beta \log\frac{1-\beta}{\beta} ,\\
&=\log{2} + (1-\beta)\log(1-\beta) + \beta \log{\beta} ,\\
&=\log{2} -\mathcal{H}(\beta) .
\end{split}\end{equation*}

\begin{figure}[hbt]
    \centering
    \subfloat[Differences of MI between models A and B]{
        \includegraphics[width=0.45\linewidth]{figs/differences_mi.pdf}
        \label{sfig:mi_convergence_differences}
    }\hfil
    \subfloat[Gaussian mixture distribution $p(x)$ with proportion $\beta$ in between the two means $\mu_0$ and $\mu_1$]{
        \includegraphics[width=0.45\linewidth]{figs/gaussian_mixture_beta.pdf}
        \label{sfig:recap_gaussian_mixture}
    }
    \caption{Value of the mutual information for the two models splitting a Gaussian mixture depending on the distance between the two means $\mu_0$ and $\mu_1$ of the two generating Gaussian distributions. We estimate the MI by computing it 50 times on 1000 samples drawn from the Gaussian mixture.}
    \label{fig:mi_convergence_differences}
\end{figure}

To conclude, as the decision boundaries turn sharper, i.e. when $\epsilon$ approaches 0, the difference of mutual information between the two models is controlled by the entropy of proportion of data $\beta$ between the two means $\mu_0$ and $\mu_1$. We know that for binary entropies, the optimum is reached for $\beta=0.5$. In other words having $\mu_0$ and $\mu_1$ distant enough to ensure balance of proportions between the two clusters of model B leads to a difference of mutual information equal to 0. We experimentally highlight this convergence in Figure~\ref{fig:mi_convergence_differences} where we compute the mutual information of models A and B as the distance between the two means $\mu_0$ and $\mu_1$ increases in the Gaussian distribution mixture.

\section{Proof of Proposition~\ref{prop:alpha_div_maximisation}}
\label{app:proof_alpha_div_maximisation}
For the sake of clarity, we will use the notations $\pi_k\equiv p(y=k)$ during the demonstration.

\subsection{Modelisation of the conditional distribution}

We will consider two types of models. The first one is the \emph{generic} clustering model, where the cluster assignment follows a categorical distribution:

\begin{equation*}
y|\vec{x} \sim \text{Cat}(\psi_\theta(\vec{x})),
\end{equation*}
where $\psi_\theta : \mathcal{X} \rightarrow \Delta_{K}$ is a learnable function of parameters $\theta$ and $\Delta_{K}$ is a $K$-simplex.

The second model is a Dirac distribution where the data space $\mathcal{X}$ is divided into a partition $\mathcal{X}_k,\forall k \in \{1,\cdots,K\}$:

\begin{equation*}
\pykx = \pmb{1}_{[\vec{x}\in\mathcal{X}_k]},
\end{equation*}
with $\pmb{1}$ the indicator function. This is simply a sub-case of the generic model.

For both models, we consider that clusters are not empty and that the model is not degenerate, i.e. $\pi_k \in ]0,1[ \forall k \in \{1,K\}$.

\subsection{Value of the GEMINI OvA and upper bounds}

We first unfold the OvA GEMINI for the generic model and $\alpha\in \mathbb{R}\setminus\{0,1\}$:

\begin{align*}
\I^\text{ova}_{D_\alpha}(\vec{x};y) &= \E_{y\sim p(y)} \left[\E_{\vec{x} \sim \px} \left[f_\alpha\left(\frac{p(\vec{x}|y)}{\px}\right) \right] \right],\\
&=\sum_{k=1}^K \pi_k \int_{\mathcal{X}}\px \left(\frac{\pxyk^\alpha \px^{-\alpha}}{\alpha(\alpha-1)} - \frac{\pxyk \px^{-1}}{\alpha-1}+\frac{1}{\alpha}\right)d\vec{x},\\
&= \sum_{k=1}^K \pi_k \int_{\mathcal{X}} \left(\frac{\px\pykx^\alpha }{\pi_k^\alpha \alpha(\alpha-1)} - \frac{\pxyk}{\alpha-1} + \frac{\px}{\alpha}\right)d\vec{x}.
\end{align*}

After distributing the factor $\px$ in the integral, we can notice that the two last terms will be summed to 1, up to a factor depending on $\alpha$.

\begin{align*}
\I^\text{ova}_{D_\alpha}(\vec{x};y) &= \sum_{k=1}^K \pi_k \left(\frac{1}{\alpha}-\frac{1}{\alpha-1} + \frac{1}{\pi_k^\alpha \alpha(\alpha-1)}\mathbb{E}_{\vec{x}\sim \px}\left[\pykx^\alpha\right]\right)\\
&= \frac{-1}{\alpha(\alpha-1)} + \frac{1}{\alpha(\alpha-1)}\sum_{k=1}^K \pi_k^{1-\alpha} \E_{\vec{x}\sim \px}\left[\pykx^\alpha\right],\\
&= \left(\alpha(\alpha-1)\right)^\alpha \left[-1 + \sum_{k=1}^K \pi_k^{1-\alpha}\E_{\vec{x}\sim \px}\left[\pykx^\alpha\right] \right].
\end{align*}

Since we face a categorical distribution, we can affirm that $p(y|\vec{x}) \in [0,1]$. Therefore, depending on the value of $\alpha$, we have either $\pykx^\alpha \in [1, \infty[$ if $\alpha$ is negative, and $\pykx\in [0,1]$ for $\alpha$ positive. We now inspect these different cases.

\subsection{Case of \texorpdfstring{$\alpha \in ]1,+\infty[$}{alpha greater than 1}}

The upper bound we can get on the expectation involves the inequality $\pykx^\alpha \leq \pykx$. Owing to the linearity of the expectation, we can affirm that:

\begin{align*}
\E_{\vec{x}\sim \px} \left[\pykx^\alpha\right] &\leq \E_{\vec{x}\sim \px} [\pykx],\\
&\leq \pi_k.
\end{align*}

This allows us to derive the following upper bound on the OvA GEMINI:

\begin{align*}
\I^\text{ova}_{D_\alpha}(\vec{x};y) &=\left(\alpha(\alpha-1)\right)^{-1} \left[ -1 + \sum_{k=1}^K \pi_k^{1-\alpha} \E_{\vec{x}\sim \px}\left[\pykx^\alpha\right]\right],\\
&\leq \left(\alpha(\alpha-1)\right)^{-1}\left[-1+\sum_{k=1}^K\pi_k^{2-\alpha} \right],\\
&\leq \mathcal{B}_{]1,+\infty[}(\pi_1,\cdots,\pi_K).
\end{align*}

The upper bound $\mathcal{B}_{]1,+\infty[}(\pi_1,\cdots,\pi_K)$ is a convex function that is invariant to permutations of $\pi_k$. Interestingly, this upper bound only depends on the proportions of the clusters. Its maximimum is reached when $\pi_k=K^{-1}$. However, any solution is acceptable for the special case of $\alpha=2$, i.e. the Pearson $\chi^2$-divergence. In this case, the upper bound is a constant: $\mathcal{B}^+_2 = \frac{K-1}{2}$. In this situation, the proportions of the clusters do not matter. Only getting a Dirac model is sufficient to maximise the OvA GEMINI.

We can further conclude that the bound is tight when considering a Dirac model since $1^\alpha=1$ and $0^\alpha=0$, leading to:

\begin{align*}
\E_{\vec{x}\sim \px}\left[ \pmb{1}_{\left[\vec{x}\in\mathcal{X}_k\right]}^\alpha\right] &= \E_{\vec{x}\sim \px}\left[ \pmb{1}_{\left[\vec{x}\in\mathcal{X}_k\right]}\right],\\
&=\pi_k.
\end{align*}

And so do we conclude:

\begin{align*}
    \I^\text{ova}_{D_\alpha}(\vec{x};y) &=\left(\alpha(\alpha-1)\right)^{-1} \left[ -1 + \sum_{k=1}^K \pi_k^{1-\alpha} \E_{\vec{x}\sim \px}\left[\pykx^\alpha\right]\right],\\
    &= \left(\alpha(\alpha-1)\right)^{-1}\left[-1+\sum_{k=1}^K\pi_k^{2-\alpha} \right],\\
    &= \mathcal{B}_{]1,+\infty[}(\pi_1,\cdots,\pi_K)
\end{align*}

\subsection{Case of a \texorpdfstring{$\alpha \in ]0,1[$}{alpha between 0 and 1}}

In this case, the front factor $(\alpha(\alpha-1))^{-1}$ is negative. Thus, we are interested in minimising the second term and finding the lower bound. We can already infer:
\begin{align*}
\pykx &\leq \pykx^\alpha\\
\E_{\vec{x}\sim \px}[\pykx]&\leq \E_{\vec{x}\sim \px} \left[ \pykx^\alpha\right]\\
\pi_k&\leq \E_{\vec{x}\sim \px} \left[ \pykx^\alpha\right].
\end{align*}


This lower bound is tight for a Dirac model. We can finally compute for the OvA GEMINI that:

\begin{align*}
\I^\text{ova}_{D_\alpha}(\vec{x};y) &=\left(\alpha(\alpha-1)\right)^{-1} \left[ -1 + \sum_{k=1}^K \pi_k^{1-\alpha} \E_{\vec{x}\sim \px}\left[\pykx^\alpha\right]\right],\\
&\leq \left(\alpha(\alpha-1)\right)^{-1}\left[-1+\sum_{k=1}^K\pi_k^{2-\alpha} \right],\\
&\leq \mathcal{B}_{]0,1[}.
\end{align*}

Hence, we conclude that $\mathcal{B}_{]0,1[}=\mathcal{B}_{]1,+\infty[} = \mathcal{B}_{\mathbb{R}^{+*}\setminus \{1\}}$. In both cases, using a Dirac model implies that the OvA GEMINI reaches its upper bound.

\subsection{Case of a negative \texorpdfstring{$\alpha$}{alpha}}

The upper bound of the OvA GEMINI in this case is the infinity. Indeed, taking the example of the Dirac model is sufficient to consider regions of the data space $\mathcal{X}$ where the clustering distribution has no support. Thus, the expectation is undefined, or rather drifts towards infinity.

\subsection{Specific case of \texorpdfstring{$\alpha=1$}{alpha equal to 1}, the KL divergence}

In this case, we need to start the computations all over again using the definition $f(t)=t\log{t}$. Indeed, we can skip the term $-t +1$ since it does not affect the value of an $f$-divergence, i.e. for any convex function $f$ s.t. $f(1)=0$ and for any real constant $c$:

\begin{equation*}
D_{f(t)}(p\|q) = D_{f(t)+c(t-1)}(p\|q).
\end{equation*}

We thus get:

\begin{align*}
\I_{D_1}^\text{ova}(\vec{x};y) &= \E_{y\sim p(y)}\left[\E_{\vec{x} \sim \px}\left[\frac{\pxyk}{\px} \log\frac{\pxyk}{\px}\right]\right],\\
&= \sum_{k=1}^K \pi_k \int_\mathcal{X} \pxyk \log\left(\frac{\pxyk}{\px}\right)d\vec{x},\\
&= \sum_{k=1}^K \int_\mathcal{X} \pykx \px \log \left(\frac{\pykx}{\pi_k} \right)d\vec{x}.
\end{align*}

We can then separate the log term to make appear the two different entropies contributing to the mutual information:

\begin{align*}
\I_{D_1}^\text{ova}(\vec{x};y) &= \sum_{k=1}^K \int_\mathcal{X} \pykx \px\log(\pykx)d\vec{x} - \log{\pi_k}\int_\mathcal{X} \pykx \px d\vec{x},\\
&= \sum_{k=1}^K \int_\mathcal{X} \pykx \px\log(\pykx)d\vec{x} - \pi_k\log\pi_k
\end{align*}

We find once again an upper bound on the integral depending on $p(y|\vec{x})$. We know that the function $g: t\mapsto t\log{t}$ is convex and below 0 for $t\in [0,1]$. Hence:

\begin{equation*}
\pykx\log{\pykx} \leq 0,
\end{equation*}

with strict equality iff $\pykx \in \{0,1\}$. This implies that the Dirac model maximises the left integral. We deduce the upper bound of the mutual information:

\begin{align*}
\I_{D_1}^\text{ova}(\vec{x};y) &= \sum_{k=1}^K \int_\mathcal{X} \pykx \px\log(\pykx)d\vec{x} - \pi_k\log\pi_k,\\
&\leq \sum_{k=1}^K -\pi_k \log{\pi_k},\\
&\leq \mathcal{B}_1.
\end{align*}

This shows that the cluster proportion entropy is the upper bound of the mutual information. It is reached for any Dirac model.

\subsection{Last subcase: the null alpha}

In this case, the $\alpha$-divergence is defined by the function $f(t) = -\log{t}$. Let us derive again the OvA GEMINI:

\begin{align*}
\I_{D_0}^\text{ova}(\vec{x};y) &= \E_{y\sim p(y)}\left[\E_{\vec{x} \sim \px}\left[- \log\frac{\pxyk}{\px}\right]\right],\\
&= -\sum_{k=1}^K \pi_k \int_\mathcal{X} \px \log\left(\frac{\pxyk}{\px}\right)d\vec{x},\\
&= -\sum_{k=1}^K \pi_k\int_\mathcal{X} \px \log \left(\frac{\pykx}{\pi_k} \right)d\vec{x}.
\end{align*}

We expand again the logarithm and compute the integral over constant terms factorised by $\px$:

\begin{align*}
\I_{D_0}^\text{ova}(\vec{x};y) &= \sum_{k=1}^K \pi_k\log{\pi_k} - \pi_k\int_\mathcal{X} \px\log{\pykx}d\vec{x}.
\end{align*}

Now we can see that this OvA GEMINI may converge to infinity. Indeed, for the example of the Dirac model, we evaluate the integral with terms worth $\lim_{t\rightarrow 0}\log{t}$. We cannot conclude on the upper bounds of this case.

\subsection{Maximal upper bound}

We have shown so far that for $\alpha >0$, we can derive two different upper bounds that only depend on the proportions of the clusters $\pi_k$. These upper bounds can be reached by Dirac model of type $\pykx=\pmb{1}_{[\vec{x}\in\mathcal{X}_k]}$.

We can now question for the two upper bounds, $\mathcal{B}_{\mathbb{R}^{+*}\setminus\{1\}}$ and $\mathcal{B}_1$ what are the optimal cluster proportions $\pi_k$. By adding a Lagrangian constraint to enforce $\sum_{k=1}^K \pi_k=1$ in each upper bound, we can show that the maximal upper bound is reached iff $\pi_k=K^{-1} \forall k$. This concludes our proof.

\section{Proof of Proposition~\ref{prop:ovo_greater_ova}}
\label{app:proof_ovo_greater_ova}
Let us consider the value of the OvO GEMINI when the distance $D$ is an $f$-divergence or an IPM.

\subsection{Demonstration for \texorpdfstring{$f$}{f}-divergences}
We first need to highlight that $f$-divergences come with a conjugate convex function $g$. This conjugate enables the inversion of the arguments of the $f$-divergence:

\begin{equation*}
D_f(p\|q) = D_g(q\|p),
\end{equation*}
for any distributions $p$ and $q$. We can use this trick to revert first the $f$-divergence between the distribution $\pykx$ and $\px$:

\begin{equation*}
D_f(\pxyk\|\px) = D_g(\px\|\pxyk).
\end{equation*}

We then write $\px$ as a sum marginalising the $y$ variable. Using the convexity of the function $g$, we get a weighted upper bound of this divergence:

\begin{align*}
D_g(\px\|\pxyk)&= D_g\left(\sum_{k^\prime=1}^K p(y=k^\prime)p(\vec{x}|y=k^\prime) \| \left(\sum_{k^\prime=1}^K p(y=k^\prime)\right) \pxyk\right),\\
&\leq \sum_{k^\prime=1}^K p(y=k^\prime) D_g(p(\vec{x}|y=k^\prime) \| \pxyk),\\
&\leq \E_{k^\prime \sim p(y)} \left[ D_g(p(\vec{x}|y=k^\prime)\|\pxyk)\right].
\end{align*}

To retrieve the OvO form, we can compute the expectation of this inequality over all possible combinations of $p(y)$:

\begin{align*}
\E_{y \sim p(y)}\left[D_f(\px\|\pxyk)\right] &\leq \E_{y_1,y_2 \sim p(y)} \left[ D_g (p(\vec{x}|y_1)\|p(\vec{x}|y_2))\right],\\
\I^\text{ova}_{D_f}(\vec{x};y) &\leq \I^\text{ovo}_{D_g}(\vec{x};y).
\end{align*}

Then, owing to the conjugate convex functions, we can observe that for any $k, k^\prime \in \{1,\cdots, K\}$:

\begin{multline*}
D_g(\pxyk\|p(\vec{x}|y=k^\prime)) + D_g(p(\vec{x}|y=k^\prime)\|\pxyk) = D_f(p(\vec{x}|y=k^\prime)\|\pxyk)\\+ D_f(\pxyk\|p(\vec{x}|y=k^\prime)).
\end{multline*}

Consequently, the symmetry of OvO in its double expectation implies that:

\begin{equation*}
\I^\text{ovo}_{D_f}(\vec{x};y) = \I^\text{ovo}_{D_g}(\vec{x};y).
\end{equation*}

And so do we conclude that:

\begin{equation}
\I^\text{ova}_{D_f}(\vec{x};y) \leq \I^\text{ovo}_{D_f}(\vec{x};y).
\end{equation}

\subsection{Demonstration for IPMs}

For IPMs, we start from the OvA distance between the distribution of an arbitrary cluster $i$ among $K$:

\begin{equation*}
D_\text{IPM}(p(\vec{x}|y=i)\|\px) = \sup_{f\in\mathcal{F}} \E_{\vec{x}\sim p(\vec{x}|y=i)}[f(\vec{x})] - \E_{\vec{x}\sim \px}[f(\vec{x})].
\end{equation*}

We can extend the expectation of the data distribution over all clusters using the sum rules of probabilities:

\begin{align*}
D_\text{IPM}(p(\vec{x}|y=i)\|\px)&= \sup_{f\in\mathcal{F}} \E_{\vec{x}\sim p(\vec{x}|y=i)}[f(\vec{x})] - \sum_{k=1}^Kp(y=k)\E_{\vec{x}\sim \pxyk}[f(\vec{x})],\\
&=\sup_{f\in\mathcal{F}} \sum_{k=1}^K p(y=k) \left(\E_{\vec{x}\sim p(\vec{x}|y=i)}[f(\vec{x})] - \E_{\vec{x}\sim \pxyk}[f(\vec{x})]\right).
\end{align*}

We used in the second line the fact that the sum of $\sum_{k=1}^Kp(y=k)=1$ to factorise the first expectation. Finally, we can use the property that the supremum of a sum is lower or equal than a sum of suprema:

\begin{align*}
D_\text{IPM}(p(\vec{x}|y=i)\|\px) &\leq \sum_{k=1}^K \sup_{f\in\mathcal{F}} p(y=k)\left(\E_{\vec{x}\sim p(\vec{x}|y=i)}[f(\vec{x})] - \E_{\vec{x}\sim \pxyk}[f(\vec{x})]\right),\\
&\leq \sum_{k=1}^K p(y=k) D_\text{IPM} (p(\vec{x}|y=i)\|\pxyk).
\end{align*}

This upper bound corresponds to the expectation of the IPM between a specific cluster distribution $i$ and all other cluster distributions. Finally, by performing the expectation over all cluster of index $i$,  we can conclude that:

\begin{equation}
\I^\text{ova}_\text{IPM}(\vec{x};y) \leq \I^\text{ovo}_\text{IPM}(\vec{x};y).
\end{equation}

\section{Proof of Proposition~\ref{prop:equality_ova_ovo_ipm}}
\label{app:proof_equality_ova_ovo_ipm}
We will show here that when the cluster variable $y$ is binary, the OvA and OvO GEMINIs are equal if we use IPMs for distance between distributions. Indeed we can first unfold both equations:

\begin{equation}\label{eq:unfold_ova}\begin{split}
\I_\text{IPM}^\text{ova}(\vec{x},y) &= \E_{y\sim p(y)}\left[D_\text{IPM}(p(\vec{x}|y)\|\px\right]\\
&=p(y=0)D_\text{IPM}(p(\vec{x}|y=0)\|p(\vec{x})) + p(y=1)D_\text{IPM}(p(\vec{x}|y=1)\|\px)\\
&= p(y=0)\sup_{f\in\mathcal{F}} \{ \E_{\vec{x}\sim p(\vec{x}|y=0)}[f(\vec{x})] - \E_{\vec{x}\sim \px}[f(\vec{x})] \} \\&\quad+ p(y=1) \sup_{g\in\mathcal{F}} \{\E_{\vec{x}\sim p(\vec{x}|y=1)}[g(\vec{x})] - \E_{\vec{x}\sim \px}[g(\vec{x})]\},
\end{split}\end{equation}

and for the OvO, we simply use the symmetric property of IPMs:

\begin{equation}\label{eq:unfold_ovo}\begin{split}
\I_\text{IPM}^\text{ovo}(\vec{x},y) &=\E_{\ya,\yb \sim p(y)}\left[D_\text{IPM}(p(\vec{x}|\ya)\|p(\vec{x}|\yb))\right]\\
&=p(y=0)p(y=1)D_\text{IPM}(p(\vec{x}|y=0)||p(\vec{x}|y=1)) \\&\qquad+ p(y=1)p(y=0)D_\text{IPM}(p(\vec{x}|y=1)||p(\vec{x}|y=0))\\
&= 2 p(y=0)p(y=1) D_\text{IPM}(p(\vec{x}|y=0)||p(\vec{x}|y=1)).
\end{split}\end{equation}

Notice that in Eq. (\ref{eq:unfold_ovo}), we skipped the terms where both the random variables $y_1$ and $y_2$ are equal, since the implied distance is necessarily 0.

Now, to show the equivalence of both equations (\ref{eq:unfold_ova}) and (\ref{eq:unfold_ovo}), we simply need to write the sum rule of probabilities leading to the marginalisation of $\vec{x}$:

\begin{equation*}%\label{eq:marginalisation}
\px=p(\vec{x}|y=0)p(y=0) + p(\vec{x}|y=1)p(y=1).
\end{equation*}

Thus, we can rewrite the expectations depending on the distribution $\px$ with other distributions for any function $f$:

\begin{equation*}%\label{eq:rewritten_expectation_px}
\E_{\vec{x}\sim \px}[f(\vec{x})] = p(y=0)\E_{\vec{x} \sim p(\vec{x}|y=0)}[f(\vec{x})] + p(y=1)\E_{\vec{x}\sim p(\vec{x}|y=1)} [f(\vec{x})]],
\end{equation*}

which we can incorporate back into Eq. (\ref{eq:unfold_ova}) to get:

\begin{multline*}
\I_\text{IPM}^\text{ova}(\vec{x},y) = p(y=0)\sup_{f\in\mathcal{F}}\{(1-p(y=0))\E_{\vec{x}\sim p(\vec{x}|y=0)}[f(\vec{x})] - p(y=1)\E_{\vec{x}\sim p(\vec{x}|y=1)} [f(\vec{x})]]\}\\+p(y=1)\sup_{g\in\mathcal{F}} \{ (1-p(y=1))\E_{\vec{x}\sim p(\vec{x}|y=1)}[g(\vec{x})] - p(y=0)\E_{\vec{x} \sim p(\vec{x}|y=0)}[g(\vec{x})]\}.
\end{multline*}

Since we only use two clusters, we know that $p(y=1)=1-p(y=0)$. This helps us factorising terms inside the sup expressions:

\begin{equation*}\begin{split}
\I_\text{IPM}^\text{ova}(\vec{x},y) &= p(y=0)\sup_{f\in\mathcal{F}} \left\{p(y=1)\left[ \E_{\vec{x}\sim p(\vec{x}|y=0)}[f(\vec{x})] - \E_{\vec{x}\sim p(\vec{x}|y=1)}[f(\vec{x})]\right] \right\}\\
&\qquad p(y=1)\sup_{g\in\mathcal{F}} \left\{ p(y=0)\left[ \E_{\vec{x}\sim p(\vec{x}|y=1)}[g(\vec{x})] - \E_{\vec{x}\sim p(\vec{x}|y=0)}[g(\vec{x})]\right]\right\}.
\end{split}\end{equation*}

Eventually, the factors $p(y=0)$ and $p(y=1)$ do not depend on the functions $f$ and $g$, so we can pull them out of the supremum. The remaining expressions are then symmetric and can be thus factorised:

\begin{equation*}\label{eq:final_ova}\begin{split}
\I_\text{IPM}^\text{ova}(\vec{x},y) &= 2p(y=0)p(y=1)\sup_{f\in\mathcal{F}} \left\{ \E_{\vec{x}\sim p(\vec{x}|y=0)}[f(\vec{x})] - \E_{\vec{x}\sim p(\vec{x}|y=1)}[f(\vec{x})]\right\}\\
&= 2p(y=0)p(y=1)D_\text{IPM}(p(\vec{x}|y=0)||p(\vec{x}|y=1))\\
&=\I_\text{IPM}^\text{ovo}(\vec{x},y)
\end{split}\end{equation*} 

This concludes the proof.

\section{Proof of Proposition~\ref{prop:ovo_fdiv_maximisation}}
\label{app:proof_ovo_fdiv_maximisation}
For any $f$-divergence and two distribution $p$ and $q$ taking value in the space $\mathcal{X}$, then disjoint support between $p$ and $q$ implies the maximisation of the $f$-divergence. Indeed, the bounds of an $f$-divergence are:

\begin{equation*}
0 \leq D_f(p,q) \leq f(0)+g(0),
\end{equation*}
where the upper bound can be infinity depending on $f$ and its convex conjugate $g: t\longrightarrow tf(1/t)$. Thus, for any two different clusters $k\neq k^\prime$:

\begin{equation*}
0\leq D_f(p(\vec{x}|y=k)\|p(\vec{x}|y=k^\prime)) \leq f(0)+g(0).
\end{equation*}

However, distributions for the same clusters have an $f$-divergence of 0. We can therefore sum all the terms and their respective upper bounds:

\begin{align*}
\sum_{k\neq k^\prime}^K p(y=k)p(y=k^\prime) D_f(p(\vec{x}|y=k)\|p(\vec{x}|y=k^\prime)) &\leq \sum_{k\neq k^\prime}^K p(y=k)p(y=k^\prime) (f(0)+g(0))\\
\mathcal{I}_{D_f}^\text{ovo}(\vec{x},y) &\leq \sum_{k=1}^K p(y=k)p(y\neq k)(f(0)+g(0)),
\end{align*}

Following~\cite[theorem 5]{caglar2014divergence}, disjoint supports between the distribution $p(\vec{x}|y=k)$ and $p(\vec{x}|y=k^\prime)$ implies the equality with the upper bound. Assume that the data space $\mathcal{X}$ is separated into $K$ disjoint and supplementary spaces $\mathcal{X}_k$. To each subspace $\mathcal{X}_k$ corresponds a cluster distribution $p(\vec{x}|y=k)$. This disjoint supports are achieved for any model of the form:

\begin{equation*}
p(y=k|\vec{x}) = \pmb{1}[\vec{x}\in\mathcal{X}_k],
\end{equation*}
which implies the disjoint distributions:

\begin{equation*}
p(\vec{x}|y=k) \propto \pmb{1}[\vec{x}\in\mathcal{X}_k]
\end{equation*}

Each of these spaces control the proportion of data in the cluster $k$, and hence controls $p(y=k)$. Thus, the OvO GEMINI is equal to its upper bound owing to disjoint supports:

\begin{equation*}
\mathcal{I}^\text{ovo}_{D_f}(\vec{x},y) = \sum_{k=1}^K p(y=k)p(y\neq k)(f(0)+g(0))
\end{equation*}

We need to maximise the upper bound. This will be the maximum value of OvO GEMINI reachable for models with disjoint supports. Adding a Lagrangian term to respect the contraint of $\sum_{k=1}^K p(y=k)=1$ leads to the optimal solution $p(y=k)=\frac{1}{K}$. This concludes the proof.

\section{Deriving GEMINIs}
\label{app:deriving_geminis}
We show in this appendix how to derive all estimable forms of the GEMINI.

\subsection{\texorpdfstring{$f$}{f}-divergence GEMINI}

We detail here the derivation for 3 $f$-divergences that we previously chose: the KL divergence, the TV distance and the squared Hellinger distance, as well as the generic scenario for any function $f$.

\subsubsection{Generic scenario}

First, we recall that the definition of an $f$-divergence involves a convex function:

\begin{align*}
    f:\mathbb{R}^+ &\rightarrow \mathbb{R}\\
    x&\rightarrow f(x)\\
    \text{s.t.}\quad f(1)&=0,
\end{align*}

between two distributions $p$ and $q$ as described:

\begin{equation*}%\label{eq:app_fdiv}
    D_\text{f-div}(p,q) = \E_{\vec{x} \sim q} \left[ f\left(\frac{p(\vec{x})}{q(\vec{x})}\right)\right].
\end{equation*}

We simply inject this definition in the OvA GEMINI and directly obtain both an expectation on the cluster assignment $y$ and on the data variable $\vec{x}$. We then merge the writing of the two expectations for the sake of clarity.

\begin{align*}
    \mathcal{I}^\text{ova}_\text{f-div}(x;y) &= \E_{\py} \left[ D_\text{f-div}(\pxy || \pdata)\right],\\
    &= \E_{\py} \left[ \E_{\pdata}\left[ f\left(\frac{\pxy}{\pdata}\right)\right]\right],\\
    &= \E_{\py,\pdata} \left[ f\left(\frac{\pyx}{\py}\right)\right].\\
\end{align*}

Injecting the $f$-divergence in the OvO GEMINI first yields:

\begin{align*}
    \mathcal{I}^\text{ovo}_\text{f-div}(x;y) &= \E_{\pya,\pyb} \left[ D_\text{f-div}(\pxya || \pxyb)\right],\\
    &= \E_{\pya,\pyb} \left[ \E_{\pxyb}\left[ f\left(\frac{\pxya}{\pxyb}\right)\right]\right].\\
\end{align*}

Now, by using Bayes theorem, we can perform the inner expectation over the data distribution. We then merge the expectations for the sake of clarity.

\begin{align*}
    \mathcal{I}^\text{ovo}_\text{f-div}(x;y) &= \E_{\pya,\pyb} \left[ \E_{\pdata} \left[ \frac{\pyxb}{\pyb}f\left(\frac{\pxya}{\pxyb}\right)\right]\right],\\
    &= \E_{\pya,\pyb,\pdata} \left[ \frac{\pyxb}{\pyb}f\left(\frac{\pyxa\pyb}{\pyxb\pya}\right)\right].\\
\end{align*}

Notice that we also changed the ratio of conditional distributions inside the function by a ratio of posteriors through Bayes' theorem, weighted by the relative cluster proportions.

Now, we can derive into details these equations for the 3 $f$-divergences we focused on: the KL divergence, the TV distance and the squared Hellinger distance.

\subsubsection{Kullback-Leibler divergence}

The function for Kullback-Leibler is $f(t) = t\log t$. We do not need to write the OvA equation: it is straightforwardly the usual MI. For the OvO, we inject the function definition by replacing:

\begin{equation*}
    t=\frac{\pyxa\pyb}{\pyxb\pya},
\end{equation*}

in order to get:

\begin{align*}
    \mathcal{I}^\text{ovo}_\text{KL}(x;y) &=  \E_{\pya,\pyb,\pdata} \left[ \frac{\pyxb}{\pyb}\times\frac{\pyxa\pyb}{\pyxb\pya}\log{\frac{\pyxa\pyb}{\pyxb\pya}}\right].
\end{align*}

We can first simplify the factors outside of the logs:

\begin{align*}
    \mathcal{I}^\text{ovo}_\text{KL}(x;y)&=\E_{\pya,\pyb,\pdata} \left[ \frac{\pyxa}{\pya}\log{\frac{\pyxa\pyb}{\pyxb\pya}}\right].
\end{align*}

If we use the properties of the log, we can separate the inner term in two sub-expressions:

\begin{align*}
    \mathcal{I}^\text{ovo}_\text{KL}(x;y) = \E_{\pya,\pyb,\pdata} \left[ \frac{\pyxa}{\pya}\log{\frac{\pyxa}{\pya}} + \frac{\pyxa}{\pya}\log{\frac{\pyb}{\pyxb}}\right].
\end{align*}

Hence, we can use the linearity of the expectation to separate the two terms above. The first term is constant w.r.t. $\yb$, so we can remove this variable from the expectation among the subscripts:

\begin{equation*}
    \mathcal{I}^\text{ovo}_\text{KL}(\vec{x},y) = \E_{\pya,\pdata} \left[ \frac{\pyxa}{\pya}\log{\frac{\pyxa}{\pya}} \right] + \E_{\pya,\pyb,\pdata} \left[ \frac{\pyxa}{\pya}\log{\frac{\pyb}{\pyxb}} \right].
\end{equation*}

Since the variables $\ya$ and $\yb$ are independent, we can use the fact that:

\begin{equation*}
    \E_{\pya}\left[\frac{\pyxa}{\pya}\right] = \int \pya \frac{\pyxa}{\pya} d\ya = 1,
\end{equation*}

inside the second term to reveal the final form of the equation:

\begin{align*}
    \mathcal{I}^\text{ovo}_\text{KL}(x;y) = \E_{\pdata,\py} \left[ \frac{\pyx}{\py}\log{\frac{\pyx}{\py}} \right] + \E_{\pdata,\py} \left[ \log{\frac{\py}{\pyx}}\right].
\end{align*}

Notice that since both terms did not compare one cluster assignment $\ya$ against another $\yb$, we can switch to the same common variable $y$. Both terms are in fact KL divergences depending on the cluster assignment $y$. The first is the reverse of the second. This sum of KL divergences is sometimes called the \emph{symmetric} KL, and so can we write in two ways the OvO KL-GEMINI:

\begin{align*}
    \mathcal{I}^\text{ovo}_\text{KL}(x;y) &= \E_{\pdata} \left[ D_\text{KL} (\pyx || \py)\right] + \E_{\pdata} \left[ D_\text{KL} (\py || \pyx)\right],\\
    &= \E_{\pdata} \left[ D_\text{KL-sym}(\pyx || \py)\right].\\
\end{align*}

We can also think of this equation as the usual MI with an additional term based on the reversed KL divergence.

\subsubsection{Total Variation distance}

For the total variation, the function is $f(t)=\frac{1}{2} |t-1|$. Thus, the OvA GEMINI is:

\begin{equation*}
    \mathcal{I}^\text{ova}_\text{TV}(x;y) = \frac{1}{2}\E_{\py,\pdata} \left[ |\frac{\pyx}{\py}-1|\right].
\end{equation*}

And the OvO is:

\begin{align*}
    \mathcal{I}^\text{ovo}_\text{TV}(x;y) &=\frac{1}{2}\E_{\pya,\pyb,\pdata} \left[ \frac{\pyxb}{\pyb}|\frac{\pyxa\pyb}{\pyxb\pya}-1|\right],\\
    &=\frac{1}{2}\E_{\pya,\pyb,\pdata} \left[ |\frac{\pyxa}{\pya} - \frac{\pyxb}{\pyb} | \right].
\end{align*}

We did not find any further simplification of these equations.

\subsubsection{Squared Hellinger distance}

Finally, the squared Hellinger distance is based on $f(t)=2(1-\sqrt{t})$. Hence the OvA unfolds as:

\begin{align*}
    \mathcal{I}^\text{ova}_{\text{H}^2}(x;y)&= \E_{\py,\pdata} \left[ 2\left(1-\sqrt{\frac{\pyx}{\py}}\right)\right],\\
    &= 2-2\E_{\pdata,\py} \left[\sqrt{\frac{\pyx}{\py}}\right].
\end{align*}

The idea of the squared OvA Hellinger-GEMINI is therefore to minimise the expected square root of the relative certainty between the posterior and cluster proportion.

For the OvO setting, the definition yields:

\begin{equation*}
    \mathcal{I}^\text{ovo}_{\text{H}^2}(x;y)=\E_{\pya,\pyb,\pdata} \left[ \frac{\pyxb}{\pyb}\times 2 \times\left(1-\sqrt{\frac{\pyxa\pyb}{\pyxb\pya}}\right)\right],
\end{equation*}

which we can already simplify by putting the constant 2 outside of the expectation, and by inserting all factors inside the square root before simplifying and separating the expectation:

\begin{align*}
    \mathcal{I}^\text{ovo}_{\text{H}^2}(x;y) &= 2\E_{\pya,\pyb,\pdata} \left[ \frac{\pyxb}{\pyb} - \frac{\pyxb}{\pyb}\sqrt{\frac{\pyxa\pyb}{\pya\pyxb}}\right],\\
    &= 2\E_{\pya,\pyb,\pdata} \left[ \frac{\pyxb}{\pyb} \right] - 2\E_{\pya,\pyb,\pdata} \left[ \sqrt{\frac{\pyxa \pyxb}{\pya \pyb}}\right].
\end{align*}

We can replace the first term by the constant 1, as shown for the OvO KL derivation. Since we can split the square root into the product of two square roots, we can apply twice the expectation over $\ya$ and $\yb$ because these variables are independent:

\begin{equation*}
    \mathcal{I}^\text{ovo}_{\text{H}^2}(x;y) = 2-2\E_{\pdata} \left[ \E_{\py} \left[\sqrt{\frac{\pyx}{\py}}\right]^2\right].
\end{equation*}

To avoid computing this squared expectation, we use the equation of the variance $\mathbb{V}$ to replace it. Thus:

\begin{align*}
    \mathcal{I}^\text{ovo}_{\text{H}^2}(x;y) &= 2-2\E_{\pdata} \left[ \E_{\py}\left[\frac{\pyx}{\py}\right] - \mathbb{V}_{\py}\left[\sqrt{\frac{\pyx}{\py}}\right]\right],\\
    &= 2 - 2\E_{\pdata}\left[ \E_{\py} \left[ \frac{\pyx}{\py}\right]\right] + 2 \E_{\pdata} \left[ \mathbb{V}_{\py} \left[ \sqrt{\frac{\pyx}{\py}}\right]\right].
\end{align*}

Then, for the same reason as before, the second term is worth 1, which cancels the first constant. We therefore end up with:

\begin{equation*}
    \mathcal{I}^\text{ovo}_{\text{H}^2}(x;y)= 2\E_{\pdata} \left[ \mathbb{V}_{\py}\left[\sqrt{\frac{\pyx}{\py}}\right]\right].
\end{equation*}

Similar to the OvO KL case, the OvO squared Hellinger converges to an OvA setting, i.e. we only need information about the cluster distribution itself without comparing it to another. Furthermore, the idea of maximising the variance of the cluster assignments is straightforward for clustering.

\subsection{Maximum Mean Discrepancy}

When using an IPM with a family of functions that project an input of $\mathcal{X}$ to the unit ball of an RKHS $\mathcal{H}$, the IPM becomes the MMD distance.

\begin{align*}
\text{MMD}(p,q) &= \sup_{f: ||f||_\mathcal{H}\leq 1} \E_{\xa\sim p}[f(\xa)] - \E_{\xb \sim q} [f(\xb)],\\
&= \| \E_{\xa\sim p} [\varphi(\xa)] - \E_{\xb \sim q}[\varphi(\xb)]\|_{\mathcal{H}},\\
\end{align*}

where $\varphi$ is a embedding function of the RKHS.

By using a kernel function $\kappa(\xa,\xb) = <\varphi(\xa), \varphi(\xb)>$, we can express the square of this distance thanks to inner product space properties~\cite{gretton_kernel_2012}:

\begin{align*}
\text{MMD}^2 (p,q) &= \E_{\xa, \xa^\prime \sim p}[\kappa(\xa, \xa^\prime)] + \E_{\xb,\xb^\prime \sim q}[\kappa(\xb, \xb^\prime)] - 2 \E_{\xa\sim p, \xb\sim q}[\kappa(\xa, \xb)].
\end{align*}

%For the sake of clarity in the above equation, we keep the notation $\xa$ for data samples originating from the left-hand-side distribution of the MMD: $p$. Respectively, $\xb$ is the random variable from the right-hand-side distribution of the MMD $q$. We note with $\xa^\prime$ and $\xb^\prime$ similar variables independently drawn from the same respective distributions.

Now, we can derive each term of this equation using our distributions $p\equiv\pxy$ and $q\equiv \pdata$ for the OvA case, and $p\equiv\pxya, q\equiv\pxyb$ for the OvO case. In both scenarios, we aim at finding an expectation over the data variable $x$ using only the respectively known and estimable terms $\pyx$ and $\py$.

\paragraph{OvA scenario}

For the first term, we use Bayes' theorem twice to get an expectation over two variables $\xa$ and $\xb$ drawn from the data distribution.

\begin{equation*}
    \begin{split}
        \E_{\xa, \xa^\prime \sim p}&= \E_{\xa, \xa^\prime \sim \pxy} \left[ \kappa(\xa,\xa^\prime)\right],\\
        &= \E_{\xa,\xa^\prime \sim \pdata} \left[ \frac{\p(y|\xa)\p(y|\xa^\prime)}{\py^2} \kappa(\xa,\xa^\prime)\right].
    \end{split}
\end{equation*}

For the second term, we do not need to perform anything particular as we directy get an expectation over the data variabes $\xa$ and $\xb$.

\begin{equation*}
    \E_{\xb, \xb^\prime \sim q}= \E_{\xb, \xb^\prime \sim \pdata} \left[ \kappa(\xb,\xb^\prime)\right].
\end{equation*}

The last term only needs Bayes theorem once, for the distribution $q$ is directly replaced by the data distribution $\pdata$:

\begin{align*}
    \E_{\xa\sim p, \xb \sim q}&=\E_{\xa \sim \pxy, \xb \sim \pdata}\left[ \kappa(\xa,\xb)\right],\\
    &= \E_{\xa, \xb \sim px} \left[ \frac{\p(y|\xa)}{\py} \kappa(\xa,\xb)\right].
\end{align*}

Note that for the last term, we could replace $\p(y|\xa)$ by $\p(y|\xb)$; that would not affect the result since $\xa$ and $\xb$ are independently drawn from $\pdata$. We thus replace all terms, and do not forget to put a square root on the entire sum since we have computed so far the squared MMD:

\begin{align*}
        \mathcal{I}_\text{MMD}^\text{ova}(x;y) &= \E_{\py} \left[ \text{MMD}(\pxy,\pdata)\right],\\
        &=\E_{\py} \left[ \left(\E_{\xa,\xa^\prime \sim \pdata} \left[ \frac{\p(y|\xa)\p(y|\xa^\prime)}{\py^2} \kappa(\xa,\xa^\prime)\right] \right.\right.\\
        &\left.\left.\qquad+ \E_{\xb, \xb^\prime \sim \pdata} \left[ \kappa(\xb,\xb^\prime)\right] -2 \E_{\xa, \xb \sim \pdata} \left[ \frac{\p(y|\xa)}{\py} \kappa(\xa,\xb)\right] \right)^{\frac{1}{2}}  \right].\\
\end{align*}

Since all variables $\xa$, $\xa^\prime$, $\xb$ and $\xb^\prime$ are independently drawn from the same distribution $\pdata$, we can replace all of them by the variables $\vec{x}$ and $\vec{x}^\prime$. We then use the linearity of the expectation and factorise by the kernel $\kappa(\vec{x},\vec{x}^\prime)$:

\begin{equation*}
        \mathcal{I}_\text{MMD}^\text{ova}(x;y) = \E_{\py} \left[ \E_{\vec{x},\vec{x}^\prime \sim \pdata}\left[ \kappa(\vec{x},\vec{x}^\prime) \left( \frac{\p(y|\vec{x})\p(y|\vec{x}^\prime)}{\py^2} + 1 - 2\frac{\p(y|\vec{x})}{\py}\right) \right]^{\frac{1}{2}}\right].
\end{equation*}

\paragraph{OvO scenario}

The two first terms of the OvO MMD are the same as the first term of the OvA setting, with a simple subscript $a$ or $b$ at the appropriate place. Only the negative term changes. We once again use Bayes' theorem twice:

\begin{align*}
        \E_{\xa \sim p, \xb \sim q} [\kappa(\xa, \xb)]&= \E_{\xa \sim \pxya, \xb \pxyb} \left[ \kappa(\xa,\xb) \right],\\
        &= \E_{\xa, \xb \sim \pdata} \left[\frac{\p(\ya|\xa)}{\pya}\frac{\p(\yb|\xb)}{\pyb} \kappa(\xa,\xb) \right].
\end{align*}

The final sum is hence similar to the OvA:

\begin{align*}
        \mathcal{I}_\text{MMD}^\text{ovo}(x;y) &= \E_{\pya, \pyb} \left[ \text{MMD} (\pxya, \pxyb)\right],\\
        &= \E_{\pya, \pyb} \left[ \E_{\vec{x},\vec{x}^\prime \sim \pdata} \left[ \kappa(\vec{x},\vec{x}^\prime) \left( \frac{\pyxa \p(\ya|\vec{x}^\prime)}{\pya^2} + \frac{\pyxb \p(\yb|\vec{x}^\prime)}{\pyb^2}\right.\right.\right.\\&\quad\left.\left.\left.-2 \frac{\pyxa \p(\yb|\vec{x}^\prime) }{\pya \pyb} \right) \right]^{\frac{1}{2}}\right].
\end{align*}

\subsection{Wasserstein distance (Proof of Prop.~\ref{prop:wasserstein_convergence})}
\label{app:wasserstein_convergence}

To compute the Wasserstein distance between the distributions $\p(\vec{x}|y=k)$, we estimate it using approximate distributions. We replace $\p(\vec{x}|y=k)$ by a weighted sum of Dirac measures on specific samples $\vec{x}_i$: $p_N^k$:

\begin{equation*}
    \p(\vec{x}|y=k) \approx \sum_{i=1}^N m_i^k\delta_{\vec{x}_i} = p_N^k,
\end{equation*}
where $\{m_i^k\}_{i=1}^N$ is the set of weights. We now show that computing the Wasserstein distance between these approximates converges to the correct distance. We first need to show that $p_N^k$ weakly converges to $p$. To that end, we will use the Portmanteau theorem~\cite{billingsley_convergence_2013}. Let $f$ be any bounded and continuous function. Computing the expectation of such through $\p$ is:

\begin{equation*}
\E_{\vec{x}\sim \p(\vec{x}|y=k)}[f(\vec{x})] = \int_{\mathcal{X}}f(\vec{x})\p(\vec{x}|y=k)d\vec{x},
\end{equation*}
which can be estimated using self-normalised importance sampling~\cite[Chapter 9]{owen_monte_2009}. The proposal distribution we take for sampling is $\pdata$. Although we cannot evaluate both $\pxy$ and $\pdata$ up to a constant, we can evaluate their ratio up to a constant which is sufficient:

\begin{align*}
\E_{\vec{x}\sim \p(\vec{x}|y=k)}[f(\vec{x})]&= \int_{\mathcal{X}}f(\vec{x})\frac{\p(\vec{x}|y=k)}{\pdata}\pdata d\vec{x},\\
&= \int_{\mathcal{X}}f(\vec{x})\frac{\p(y=k|\vec{x})}{\p(y=k)}\pdata d\vec{x},\\
&\approx \sum_{i=1}^N f(\vec{x}_i) \frac{\p(y=k|\vec{x}=\vec{x}_i)}{\sum_{j=1}^N \p(y=k|\vec{x}=\vec{x}_j)}.
\end{align*}

Now, by noticing in the last line that the importance weights are self normalised and add up to 1, we can identify them as the point masses of our previous Dirac approximations:

\begin{equation*}
m_i^k = \frac{\p(y=k|\vec{x}=\vec{x}_i)}{\sum_{j=1}^N \p(y=k|\vec{x}=\vec{x}_j)}.
\end{equation*}

This allows to write that the Monte Carlo estimation through importance sampling of the expectation w.r.t $\p(\vec{x}|y=k)$ is directly the expectation taken on the discrete approximation $p_N^k$. We can conclude that their is a convergence between the two expectations owing to the law of large numbers:

\begin{equation*}
\lim_{N\rightarrow +\infty}\E_{\vec{x}\sim p_N^k}[f(\vec{x})]  = \E_{\vec{x}\sim \p(\vec{x}|y=k)}[f(\vec{x})].
\end{equation*}

Since $f$ is bounded and continuous, the portmanteau theorem~\cite{billingsley_convergence_2013} states that $p_N^k$ weakly converges to $\p(\vec{x}|y=k)$ when defining the importance weights as the normalised predictions cluster-wise.

To conclude, when two series of measures $p_N$ and $q_N$ weakly converge respectively to $p$ and $q$, so does their Wasserstein distance ~\cite[Corollary 6.9]{villani_optimal_2009}, hence:

\begin{equation}
    \lim_{N\rightarrow+\infty}\mathcal{W}_c(p_N^{k_1},p_N^{k_2})= \mathcal{W}_c\left(\p(\vec{x}|y=k_1)\|\p(\vec{x}|y=k_2)\right).
\end{equation}

For the one-vs-all Wasserstein GEMINI, we simply need to set the second distribution to the empirical data distribution: $m_i=1/N$.

%The Wasserstein is a subcase of the IPM when the family function $\mathcal{F}$ consists in a set of 1-Lipschitz functions: it is the dual representation. We focus on this specific distance in the OvO setting. First of all, we remind that the equations for IPMs in GEMINIs are:

%\begin{equation*}
%    \mathcal{I}^\text{ova}_\text{IPM} = \mathbb{E}_{y \sim p_\theta(y)} \left[ \sup_{f\in\mathcal{F}} \mathbb{E}_{\pmb{x}\sim p_\text{data}(\pmb{x})}[f(\pmb{x})] - \mathbb{E}_{\pmb{x} \sim p_\theta(\pmb{x}|y)} [f(\pmb{x})]\right],
%\end{equation*}

%and

%\begin{equation*}\mathcal{I}^\text{ovo}_\text{IPM} = \mathbb{E}_{y_1,y_2 \sim p_\theta(y)} \left[ \sup_{f\in\mathcal{F}} \mathbb{E}_{\pmb{x}\sim p_\theta(\pmb{x}|y_1)}[f(\pmb{x})] - \mathbb{E}_{\pmb{x} \sim p_\theta(\pmb{x}|y_2)} [f(\pmb{x})]\right].\end{equation*}

%Since we made no assumption on the data distribution, we are unable to sample data points according to the conditional distributions $\pxy$. However, we can perform \emph{importance sampling}:

%\begin{equation*}\mathbb{E}_{\vec{x}\sim \pxy}[f(\vec{x})] \approx \frac{1}{N} \sum_{i=1}^N \tilde{\pi}_i f(\pmb{x}_i).\end{equation*}

%We then need a proposal distribution $q$ from which we can easily sample in order to estimate $\pxy$. We simply use Bayes' theorem:

%\begin{equation*}
%\int p_\theta(\pmb{x}|y) f(\pmb{x}) d\pmb{x} = \int \frac{p_\theta (y|\pmb{x})}{p_\theta(y)}p_\text{data}(\pmb{x})d\pmb{x}.    
%\end{equation*}

%Hence, the proposal distribution is simply the data distribution $\pdata$. Even though we are in fact not able to sample from it, we have at our hand a dataset which makes the above integral estimable through Monte Carlo. Notice that contrary to the conditions of importance sampling, we are not able to evaluate $\pdata$. Still, this allows to us compute properly self-normalised weights $\tilde{\pi}_i^k$ for each sample $\vec{x}_i$ out of $N$ in the cluster $k$:

%\begin{align*}
%\tilde{\pi}_i^k &= \frac{p_\theta(\vec{x}_i|y=k)/p_\text{data}(\vec{x}_i)}{\sum_{j=1}^N p_\theta(\vec{x}_j|y=k)/p_\text{data}(\vec{x}_j)}\\
%&= \frac{p_\theta(y=k|\vec{x}_i)/p_\theta(y=k)}{\sum_{j=1}^N p_\theta(y=k|\vec{x}_j)/p_\theta(y=k)}\\
%&= \frac{p_\theta(y=k|\vec{x}_i)}{\sum_{j=1}^N p_\theta(y=k|\vec{x}_j)}.
%\end{align*}

%We use these self-normalised weights to subsequently compute the Wasserstein distance between a cluster distribution $\p(\vec{x}|y=k)$ and another $\p(\vec{x}|y=k^\prime)$ in the OvO setting. For the OvA setting and the data distribution $\pdata$, we simply consider all self-normalised weights to be equal to $\frac{1}{N}$.

\section{Other maximisations of the Wasserstein distances}
\label{app:other_wasserstein_distances}
The Wasserstein-1 metric can be considered as an IPM defined over a set of 1-Lipschitz functions. Indeed, such writing is the dual representation of the Wasserstein-1 metric:

\begin{equation*}
    W_c(p\|q) = \sup_{f, \|f\|_L\leq 1} \E_{x\sim p}[f(x)] - \E_{z\sim q}[f(z)].
\end{equation*}

Yet, evaluating a supremum as an objective to maximise is hardly compatible with the usual backpropagation in neural networks. This definition was used in attempts to stabilise GAN training~\cite{arjovsky_wasserstein_2017} by using 1-Lipschitz neural networks~\cite{gouk_regularisation_2021}. However, the Lipschitz continuity was achieved at the time by weight clipping, whereas other methods such as spectral normalisation~\cite{miyato_spectral_2018} now allow arbitrarily large weights. The restriction of 1-Lipscthiz functions to 1-Lipschitz neural networks does not equal the true Wasserstein distance, and the term "neural net distance" is sometimes preferred~\cite{arora_generalization_2017}. Still, estimating the Wasserstein distance using a set of Lipschitz functions derived from neural networks architectures brings more difficulties to actually leverage the true distance according to the energy cost $c$ of Eq.~\ref{eq:wasserstein_definition}.

Globally, we hardly experimented the generic IPM for GEMINIs. Our efforts for defining a set of 1-Lipschitz critics, one per cluster for OvA or one per pair of clusters for OvO, to perform the neural net distance~\cite{arora_generalization_2017} were not fruitful. This is mainly because such an objective requires the definition of one neural network for the posterior distribution $\pyx$ and $K$ (resp. $K(K-1)/2$) other 1-Lipschitz neural networks for the OvA (resp. OvO) critics, i.e. a large number of parameters. Moreover, this brings the problem of designing not only one but many neural networks while the design of one accurate architecture is already a sufficient problem.

\section{Choosing a GEMINI}
\label{app:exp_complexity}
The complexity of GEMINI increases with the distances previously mentionned depending on the number of clusters $K$ and the number of samples per batch $N$. It ranges from $\mathcal{O}(NK)$ for the usual MI to $\mathcal{O}(K^2N^3\log{N})$ for the OvO Wasserstein-GEMINI. As an example, we show in Figure~\ref{fig:time_performances} the average time of GEMINI as the number of tasked clusters increases for both 10 samples per batch (Figure~\ref{sfig:time_performances_batch10}) and 500 samples (Figure~\ref{sfig:time_performances_batch500}). The batches consists in randomly generated prediction and distances or kernel between randomly generated data.

\begin{figure}
    \centering
    \subfloat[10 samples per batch]{
        \includegraphics[width=0.48\linewidth]{figs/time_performances_batch_10.pdf}
        \label{sfig:time_performances_batch10}
    }\hfil
    \subfloat[500 samples per batch]{
        \includegraphics[width=0.48\linewidth]{figs/time_performances_batch_500.pdf}
        \label{sfig:time_performances_batch500}
    }
    \caption{Average performance time (in seconds) of GEMINIs as the number of tasked clusters grows for batches of size 10 and 500 samples.}
    \label{fig:time_performances}
\end{figure}

The OvO Wasserstein is the most complex, and so its usage should remain for 10 clusters or less overall. The second most time-consuming loss is the OvA Wasserstein, however its tendency in optimisation to only find 2 clusters makes it a inappropriate. The main difference also to notice between the following MMD is regarding their memory complexity. The OvA MMD requires only $\mathcal{O}(KN^2)$ while the OvO MMD requires $\mathcal{O}(K^2N^2)$. This memory complexity should be the major guide to choosing one MMD-GEMINI or the other. Thus, the minimal time-consuming and resource-demanding GEMINI is the OvA MMD if we consider GEMINIs that incorporates knowledge of data through kernels and distances. Other versions involving $f$-divergences have in fact the same complexity as MI in our implementations, apart from the OvO TV which reaches $\mathcal{O}(K^2N)$ in our implementation.

\section{Further speed-ups for the Wasserstein-OvO}
\label{app:wasserstein_speedup}
The complexity of the Wasserstein metric is $\mathcal{O}(N^3\log{N})$ for a batch of size $N$. Consequently, the complexity of the OvO Wasserstein reaches $\mathcal{O}(K^2N^3\log{N})$ for $K$ clusters. This implies that this GEMINI is hardly usable for a high number of clusters. To tackle this complexity, we propose instead to sample $M$ independent pairs among the $K(K-1)/2$ pairs of clusters to compare. We evaluate the OvO Wasserstein on these pairs and scale it to the same order as if performed on all pairs. This is stochastically equivalent to maximising the metric for all pairs:

\begin{equation}
    \hat{\I}^\text{ovo}_\mathcal{W}(\vec{x};y) = \frac{2M}{K(K-1)}\times\sum_{k,k^\prime \in I}\p(y=k)\p(y=k^\prime)\times \mathcal{W}_\delta\left(\{m_i^k\}_{i=1}^N, \{m_i^{k^\prime}\}_{i=1}^N \right),
\end{equation}

with

$$I = \{(k_n, k^\prime_n)\}_{n=1}^M\quad; (k_n,k^\prime_n) \sim \mathcal{U}\left(\{1, \cdots, K\}^2 \right),$$ a set of uniformly drawn pairs of clusters.

This optimisation requires however longer training time as a tradeoff for a controlled complexity of $\mathcal{O}(MN^3)$. Note that the same optimisation can be applied to the OvA Wasserstein-GEMINI.

\section{All pair shortest paths distance}
\label{app:fw_distance}
Sometimes, using distances such as the $\ell_2$ may not capture well the shape of manifolds. To do so, we derive a metric using the all pair shortest paths. Simply put, this metric consists in considering the number of closest neighbors that separates two data samples. To compute it, we first use a sub-metric that we note $d$, say the $\ell_2$ norm. This allows us to compute all distances $d_{ij}$ between every sample $i$ and $j$. From this matrix of sub-distances, we can build a graph adjacency matrix $W$ following the rules:

\begin{equation*}
    W_{ij} = \left\{\begin{array}{cr}
        1 & d_{ij}\leq \epsilon \\
        0 & d_{ij}> \epsilon
    \end{array}\right.,
\end{equation*}

where $\epsilon$ is a chosen threshold such that the graph has sparse edges. Our typical choice for $\epsilon$ is the 5\% quantile of all $d_{ij}$.

We chose the graph adjacency matrix to be undirected, owing to the symmetry of $d_{ij}$ and unweighted. Indeed, solving the all-pairs shortest paths involves the Floyd-Warshall algorithm~\cite{warshall_theorem_1962,roy_transitivite_1959} which complexity $\mathcal{O}(n^3)$ is not affordable when the number of samples $n$ becomes large. An undirected and unweighted graph leverages performing $n$ times the breadth-first-search algorithm, yielding a total complexity of $\mathcal{O}(n^2+ne)$ where $e$ is the number of edges. Consequently, setting a good threshold $\epsilon$ controls the complexity of the shortest paths to finds. Our final distance between two nodes $i$ and $j$ is eventually:

\begin{equation}
    c_{ij} = \left\{\begin{array}{c r}\text{Shortest-path}^W(i,j)&\text{if it exists.}\\
    n&\text{otherwise}
    \end{array}\right..
\end{equation}

This metric $c$ can then be incorporated inside the Wasserstein-GEMINI.

\section{Packages for experiments}
\label{app:requirements}
For the implementation details, we use several packages with a python 3.8 version.
\begin{itemize}
    \item We use PyTorch~\cite{pytorch} for all deep learning models and automatic differentiation, as well as NumPy~\cite{numpy} for arrays handling.
    \item We use Python Optimal Transport's function \verb+emd2+\cite{flamary_pot_2021} to compute the Wasserstein distances between weighted sums of Diracs.
    \item We used the implementation of SIMCLR from PyTorch Lightning~\cite{pytorch_lightning}.
    \item Small datasets such as isotropic Gaussian Mixture of score computations are performed using scikit-learn\cite{scikit-learn}.
\end{itemize}


\end{document}
