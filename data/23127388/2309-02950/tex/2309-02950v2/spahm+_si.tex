\documentclass[aip,jcp,reprint,amsmath,amssymb]{revtex4-2}
\bibliographystyle{apsrev4-2}
\usepackage[utf8]{inputenc}
\usepackage[T1]{fontenc}
\usepackage{graphicx}
\usepackage{multirow}
\usepackage{dcolumn}
\usepackage{hyperref}
\usepackage{braket}
\usepackage{mathtools}
\usepackage[shortlabels]{enumitem}
\usepackage[version=4]{mhchem}
\usepackage{siunitx}
\usepackage{ifthen}

\usepackage{xspace}
\makeatletter
\def\eg{\textit{e.g.}\@\xspace}
\def\ie{\textit{i.e.}\@\xspace}
\makeatother
\newcommand{\SPAHMinner}{SPA\textsuperscript{H}M}
\newcommand\SPAHM[1][9]{\SPAHMinner\ifthenelse{\equal{#1}{9}}{\xspace}{(#1)}}
\renewcommand{\vec}[1]{\mathrm{\mathbf{#1}}}
\newcommand{\de}{\,{\mathrm{d}}}
\newcommand{\transp}{^\intercal}
\newcommand{\phantomtransp}{^{\vphantom{\intercal}}}
\newcommand{\kovlp}[1]{K^\mathrm{overlap}_{#1}}
\newcommand{\becomes}[1]{\enskip\xRightarrow{\text{#1}}\enskip}
\newcommand{\hiddendd}[3]{\frac{#1#2}{#1#3}}
\newcommand{\pdd}[2]{\hiddendd{\partial}{#1}{#2}}

\renewcommand{\thesection}{S\arabic{section}}
\renewcommand{\thepage}{S\arabic{page}}
\renewcommand{\thetable}{S\arabic{table}}
\renewcommand{\thefigure}{S\arabic{figure}}
\renewcommand{\theequation}{S\arabic{equation}}
\renewcommand{\bibnumfmt}[1]{$^{\rm S#1}$}
\renewcommand{\citenumfont}[1]{{\rm S#1}}

\graphicspath{{fig_si/}}

\begin{document}

\title{{\sc Supplementary Information}\texorpdfstring{\\}{}
\texorpdfstring{\SPAHM[a,b]}{SPAHM(a,b)}:
Encoding the density information from guess Hamiltonian in quantum machine learning representations}

\author{Ksenia R. Briling}
\affiliation{Laboratory for Computational Molecular Design, Institute of Chemical Sciences and Engineering,
\'{E}cole Polytechnique F\'{e}d\'{e}rale de Lausanne, 1015 Lausanne, Switzerland}
\author{Yannick Calvino Alonso}
\affiliation{Laboratory for Computational Molecular Design, Institute of Chemical Sciences and Engineering,
\'{E}cole Polytechnique F\'{e}d\'{e}rale de Lausanne, 1015 Lausanne, Switzerland}
\author{Alberto Fabrizio}
\affiliation{Laboratory for Computational Molecular Design, Institute of Chemical Sciences and Engineering,
\'{E}cole Polytechnique F\'{e}d\'{e}rale de Lausanne, 1015 Lausanne, Switzerland}
\affiliation{National Centre for Computational Design and Discovery of Novel Materials (MARVEL),
\'{E}cole Polytechnique F\'{e}d\'{e}rale de Lausanne, 1015 Lausanne, Switzerland}
\author{Clemence Corminboeuf}
\email{clemence.corminboeuf@epfl.ch}
\affiliation{Laboratory for Computational Molecular Design, Institute of Chemical Sciences and Engineering,
\'{E}cole Polytechnique F\'{e}d\'{e}rale de Lausanne, 1015 Lausanne, Switzerland}
\affiliation{National Centre for Computational Design and Discovery of Novel Materials (MARVEL),
\'{E}cole Polytechnique F\'{e}d\'{e}rale de Lausanne, 1015 Lausanne, Switzerland}

\date{\today}

\maketitle
\onecolumngrid
\tableofcontents
\clearpage

%%%%%%%%%%%%%%%%%%%%%%%%%%%%%%%%%%%%%%%%%%%%%%%%%%%%%%%%%%%%%%%%%%%%%%%%%%%%%%%%%%%%%%%%%%%%%%%%%%%%
\section{Derivation of the \texorpdfstring{\SPAHM[a,b]}{SPAHM(a,b)} overlap kernels}
\label{sec:derivation}
\subsection{Atom density [\texorpdfstring{\SPAHM[a]}{SPAHM(a)}]}
\label{sec:derivation-a}

Let us consider two atoms, $A$ and $B$.
Each atomic density $\rho_I(\vec r)$ is represented
as a linear combination of atom-centered spherical Gaussian basis functions $\{\phi_{n\ell m}\}$,
labeled with their radial channel number $n$ and angular $\ell$ and magnetic $m$ quantum numbers,
\begin{gather}
\rho_A(\vec r) = \sum_{n\ell m} c_{n\ell m}^A \phi_{n\ell m}(\vec r), \quad
\rho_B(\vec r) = \sum_{n'\ell'm'} c_{n'\ell'm'}^B \phi_{n'\ell'm'}(\vec r),
\end{gather}
and each nucleus is virtually positioned at the origin.

The overlap kernel $\kovlp{A,B}$ between atoms $A$ and $B$ is the squared overlap
of $\rho_A$ and $\rho_B$ averaged over all possible relative orientations $\hat R$,
\begin{equation}
\kovlp{A,B}
= \frac{1}{8\pi^2}\int \left|\braket{\rho_A|\hat R|\rho_B}\right|^2 \de \hat R
= \frac{1}{8\pi^2}\int \left|k_{AB}(\hat R)\right|^2 \de \hat R.
\end{equation}

For a given orientation, the overlap $k_{AB}(\hat R)$ is
\begin{equation}
\begin{split}
k_{AB}(\hat R)
&= \braket{\rho_A|\hat R|\rho_B}
=\int \rho_A(\vec r) \hat R\, \rho_B(\vec r) \de^3 \vec r
\\&=\sum_{n\ell m} c_{n\ell m}^A \sum_{n'\ell'm'} c_{n'\ell'm'}^B
\braket{\phi_{n\ell m}| \hat R\, \phi_{n'\ell'm'}}
\\&=\sum_{n\ell m} c_{n\ell m}^A \sum_{n'\ell'm'} c_{n'\ell'm'}^B
\Braket{\phi_{n\ell m}| \sum_{m''} \phi_{n'\ell'm''} D_{m''m'}^{\ell'}(\hat R)}
\\&=\sum_\ell\sum_{nm} \sum_{n'm'} c_{n\ell m}^A c_{n'\ell m'}^B
A_{nn'}^\ell D_{mm'}^{\ell}(\hat R),
\end{split}
\end{equation}
where $\vec D$ are Wigner D-matrices for \emph{real} spherical harmonics\cite{VMK1988}
and $A_{nn'}^\ell = \braket{\phi_{n\ell m} |\phi_{n'\ell m}} \ \forall m$.

The kernel becomes
\begin{equation}
\begin{split}
\kovlp{A,B}
&=\frac{1}{8\pi^2} \int \Big|
\sum_\ell\sum_{nm}\sum_{n'm'} c_{n\ell m}^A c_{n'\ell m'}^B
A_{nn'}^\ell D^{\ell}_{mm'}(\hat R)
\Big|^2 \de \hat R
\\&=
\frac{1}{8\pi^2}\sum_{\substack{{\ell_1\,n_1m_1\,n'_1m'_1}\\{\ell_2\,n_2m_2\,n'_2m'_2}}}
c_{n_1\ell_1m_1}^A c_{n'_1\ell_1m'_1}^B
c_{n_2\ell_2m_2}^A c_{n'_2\ell_2m'_2}^B
A_{n_1n'_1}^{\ell_1} A_{n_2n'_2}^{\ell_2}
\cdot\int
D^{\ell_1}_{m_1m'_1}(\hat R)\,
D^{\ell_2}_{m_2m'_2}(\hat R)\,\de \hat R.
\end{split}
\end{equation}
Thanks to orthogonality of the real Wigner D-matrices,\cite{VMK1988} \ie,
\begin{equation}
\int
D^{\ell_1}_{m_1m_1'}(\hat R)\,
D^{\ell_2}_{m_2m_2'}(\hat R)\,
\de \hat R
=
\frac{8\pi^2}{2\ell_1+1}\,\delta_{\ell_1\ell_2}\delta_{m_1m_2}\delta_{m'_1m'_2},
\end{equation}
the kernel is further simplified to
\begin{equation}
\label{eq:atom-kernel-full}
\kovlp{A,B} =
\sum_{\ell}  \sum_{\substack{n_1n'_1 \\ n_2n'_2 }}
\underbrace{
\left(\sum_m    c_{n_1\ell m}^A   c_{n_2\ell m}^A\right)
}_{u^A_p}
\underbrace{
\left(\frac{A_{n_1n'_1}^{\ell} A_{n_2n'_2}^{\ell}}{2\ell+1}\right)
}_{M_{pq}}
\underbrace{
\left(\sum_m c_{n'_1\ell m}^B c_{n'_2\ell m}^B\right)
}_{u^B_q}.
\end{equation}
With $p=(n_1,n_2,\ell), q=(n'_1,n'_2,\ell)$ it can be rewritten as a dot product
\begin{equation}
\label{eq:atom-kernel-brief}
\kovlp{A,B}
= \sum_{pq} u_p^A M_{pq} u_q^B = \vec u_A\transp \vec M \vec u_B \phantomtransp
= (\vec M^{1/2} \vec u_A)\transp (\vec M^{1/2} \vec u_B) = \vec v_A\transp \vec v_B\phantomtransp,
\end{equation}
where $\vec v_I$ is the representation of an atomic electron density $\rho_I(\vec r)$
and is an analog of the power spectrum of atomic neighbor density.\cite{BKC2013}

\clearpage
%%%%%%%%%%%%%%%%%%%%%%%%%%%%%%%%%%%%%%%%%%%%%%%%%%%%%%%%%%%%%%%%%%%%%%%%%%%%%%%%%%%%%%%%%%%%%%%%%%%%
\subsection{Bond density [\texorpdfstring{\SPAHM[b]}{SPAHM(b)}]}
\label{sec:derivation-b}

Now let us consider two bonds, $AB$ and $XY$.
The (L\"owdin) bond densities $\rho_{AB}(\vec r)$ and $\rho_{XY}(\vec r)$ are
decomposed onto basis sets centered in the middle of each bond,
\begin{equation}
\label{eq:bond-fit}
\rho_{AB}(\vec r) = \sum_{i} c_{i} \phi_i(\vec r), \quad
\rho_{XY}(\vec r) = \sum_{j} c_{j} \phi_j(\vec r),
\end{equation}
where a function $\phi_i$ is defined by a radial channel number $n_i$ and angular $\ell_i$ and magnetic $m_i$ quantum numbers.
Both bonds are aligned along the $z$-axis and their midpoints are put at the origin.

The overlap kernel $\kovlp{AB,XY} = \mathcal I_1$ between the two bonds $AB$ and $XY$ is defined
as a overlap integral $\mathcal I_2(\varphi)$
squared averaged over rotations $\hat \varphi_z$ around the $z$-axis,
\begin{equation}
\mathcal I_1
= \frac1{2\pi}\int_0^{2\pi} \de \varphi \left| \braket{\rho_{AB}| \hat \varphi_z| \rho_{XY}} \right|^2
= \frac1{2\pi}\int_0^{2\pi} \de \varphi \left| \mathcal I_2(\varphi) \right|^2.
\end{equation}

With the decomposition~\eqref{eq:bond-fit}, the overlap integral $\mathcal I_2(\varphi)$
is rewritten with overlap of the basis functions,
\begin{equation}
\mathcal I_2(\varphi) = \braket{ \rho_{AB} | \hat \varphi_z | \rho_{XY} }
= \sum_{ij} c_{i} c_{j} \braket{\phi_i | \hat \varphi_z | \phi_j }
= \sum_{ij} c_{i} c_{j} \mathcal I_3^{ij}(\varphi),
\end{equation}
as well as the kernel $\mathcal I_1$,
\begin{equation}
\mathcal I_1
= \frac{1}{2\pi} \sum_{iji'j'} c_{i} c_{j} c_{i'} c_{j'}
\int \mathcal I_3^{ij}(\varphi)\mathcal I_3^{i'j'}(\varphi) \de \varphi
= \sum_{iji'j'} c_{i} c_{j} c_{i'} c_{j'} \mathcal I_6^{iji'j'}.
\end{equation}

With the rules for rotation of real spherical harmonics around the quantization axis,
the overlap $\mathcal I_3(\varphi)$ becomes
\begin{equation}
\mathcal I_3^{ij}(\varphi)
= \braket{\phi_i | \hat \varphi_z | \phi_j}
= \braket{\phi_i | \phi_{j}     }\cos {m_j\varphi}
+ \braket{\phi_i | \phi_{\bar \jmath}}\sin {m_j\varphi}
= S_{ij     }\cos {m_j\varphi} + S_{i\bar\jmath}\sin {m_j\varphi},
\end{equation}
where $\phi_{\bar \jmath}$ is the same basis function as $\phi_{j}$ but with an opposite phase
(\ie $m_j = -m_{\bar \jmath}$).
The integral over rotation $\mathcal I_6$ is simplified to
\begin{equation}
\mathcal I_6^{iji'j'} =
\delta_{|m_j|,|m_{j'}|} (S_{ij} S_{i'j'} + S_{i\bar\jmath} S_{i'\bar\jmath'}(1-\delta_{m_j0}) ),
\end{equation}
and the overlap kernel $\mathcal I_1$ --- to
\begin{equation}
\mathcal I_1
= \sum_{iji'j'} c_{i} c_{j} c_{i'} c_{j'}
\delta_{|m_j|,|m_{j'}|} (S_{ij} S_{i'j'} + S_{i\bar\jmath} S_{i'\bar\jmath'}(1-\delta_{m_j0}) ).
\end{equation}

When $p$ and $q$ are centered at the same point,
$S_{pq} = \delta_{\ell_p \ell_q} \delta_{m_p m_q} A_{n_p n_q}^{\ell_p}$.
Thus $\mathcal I_1$ is further simplified to
\begin{equation}
\label{eq:bond-kernel-full}
\mathcal I_1
= \sum_{ii'jj'}
\underbrace{(c_{i} c_{i'} \delta_{|m_i|,|m_{i'}|} )}_{u^{AB}_{ii'}}
\underbrace{
\delta_{\ell_{i}  \ell_{j}}    A_{n_{i}  n_{j}} ^{\ell_{i}}
\delta_{\ell_{i'} \ell_{j'}}   A_{n_{i'} n_{j'}}^{\ell_{i'}}
( \delta_{m_{i}  m_{j}}  \delta_{m_{i'} m_{j'}}
+  \delta_{m_{i} , -m_{j }} \delta_{m_{i'}, -m_{j'}} (1-\delta_{m_j0}) )}_{M_{ii',jj'}}
\underbrace{(c_{j} c_{j'} \delta_{|m_j|,|m_{j'}|} )}_{u^{XY}_{jj'}},
\end{equation}
which can be rewritten as a dot product in the same spirit as the atom-density kernel,
\begin{equation}
\label{eq:bond-kernel-brief}
\kovlp{AB,XY}
= \sum_{ii'jj'} u_{ii'}^{AB} M_{ii',jj'} u_{jj'}^{XY}
= \vec u_{AB}\transp \vec M \vec u_{XY} \phantomtransp
= (\vec M^{1/2} \vec u_{AB})\transp (\vec M^{1/2} \vec u_{XY})
= \vec v_{AB}\transp \vec v_{XY} \phantomtransp,
\end{equation}
where $\vec v_{IJ}$ is the representation of a bond density $\rho_{IJ}(\vec r)$.

\clearpage

%%%%%%%%%%%%%%%%%%%%%%%%%%%%%%%%%%%%%%%%%%%%%%%%%%%%%%%%%%%%%%%%%%%%%%%%%%%%%%%%%%%%%%%%%%%%%%%%%%%%

\section{Learning curves}
\label{sec:lc}

\subsection{QM7 and its derivatives}
\label{sec:lc-qm7}
\begin{figure}[h]
\centering
\includegraphics[width=1.0\linewidth]{QM7_full.pdf}
\caption{
Learning curves of atomic charges and spins
for the QM7, QM7/2-RC, and QM7/2+QM7/2-RC datasets.
}
\label{fig:lc-qm7}
\end{figure}

\clearpage %%%%%%%%%%%%%%%%%%%%%%%%%%%%%%%%%%%%%%%%%%%%%%%%%%%%%%%%%%%%%%%%%%%%%%%%%%%%%%%%%%%%%%%%%

\subsection{APS-RC and APS}
\label{sec:lc-azoswitch}

\begin{figure}[h!]
\centering
\includegraphics[width=0.45\linewidth]{azoswitch_full.pdf}
\caption{
Learning curves of atomic charges and spins for the APS-RC dataset.
}
\label{fig:lc-aps-rc}
\end{figure}

\clearpage

\begin{figure}[ht]
\centering
\includegraphics[width=0.48\linewidth]{azoswitch-ex_full.pdf}
\caption{
Learning curves of atomic contributions
to the hole and particle densities of the productive $\pi\to\pi^*$ state
for the APS dataset;
($+$) [dashed line] and ($-$) [solid line]
indicate \SPAHM computed for radical cations and anions, respectively.
}
\label{fig:lc-aps-ex}
\end{figure}

\clearpage

\begin{figure}[ht]
\centering
\includegraphics[width=0.38\linewidth]{azoswitch-ex_hist_full.pdf}
\caption{
Histograms of the full training set errors
of atomic contributions
to the hole and particle densities of the productive $\pi\to\pi^*$ state
for the APS dataset;
($+$) [dashed line] and ($-$) [solid line]
indicate \SPAHM computed for radical cations and anions, respectively.
}
\label{fig:lc-aps-ex-hist}
\end{figure}

\clearpage

%%%%%%%%%%%%%%%%%%%%%%%%%%%%%%%%%%%%%%%%%%%%%%%%%%%%%%%%%%%%%%%%%%%%%%%%%%%%%%%%%%%%%%%%%%%%%%%%%%%%

\section{Out-of-sample system}

\subsection{APS-RC}
\label{sec:oos-aps-rc}

Analysis of an individual system clearly illustrates the relevance of our models.
From the APS-RC dataset we selected an out-of-sample structure and
used previously trained \SPAHM[a,b] models to predict the atomic charges of its radical cation.
Fig.~\ref{fig:oos-aps-rc} compares the predicted and computed values
of atomic charges for a selection of atoms included in the $\pi$-conjugated system.
For \SPAHM[b], the predicted values accurately reproduce the computed ones within \SI{0.01}{{a.u.}},
thus verifying its performance.
However, by taking the changes in atomic charges for all the constituing atoms
and summing them up (\ie $\sum_{k}(q_k^\mathrm{cation}-q_k^\mathrm{neutral})$)
we obtain a total molecular charge $\sim 0.9$,
approximately yielding the removed electron.

\begin{figure}[ht]
\centering
\includegraphics[width=.5\linewidth]{oos-aps-rc.pdf}
\caption{
Predicted by \SPAHM[a] (red) and \SPAHM[b] (blue) and computed (black) atomic charges
for a radical cation of an out-of-sample structure on a selection of atoms (highlighted).
}
\label{fig:oos-aps-rc}
\end{figure}

\subsection{APS}
\label{sec:oos-aps-ex}
\begin{figure}[ht]
\centering
\includegraphics[width=.85\linewidth]{oos-aps-ex-num.pdf}
\caption{
Predicted by \SPAHM[a,b] atomic contributions
to the hole and particle densities of the productive $\pi\to\pi^*$ state
for an out-of-sample structure.
}
\label{fig:oos-aps-ex}
\end{figure}

\clearpage

%%%%%%%%%%%%%%%%%%%%%%%%%%%%%%%%%%%%%%%%%%%%%%%%%%%%%%%%%%%%%%%%%%%%%%%%%%%%%%%%%%%%%%%%%%%%%%%%%%%%

\section{Comparison of different atom-density-based models}
\label{sec:models}

In this section we describe and compare four models
used to post-process the guess density matrix.
The key elements of all of them are
density fitting\cite{BER1973,W1973,ETOHA1995} (DF),
\ie decomposition of the electron density onto an atom-centered basis set,
\begin{equation}
\vec c = \vec J^{-1} \vec w,
\quad
w_i = \sum_{pq} D_{pq} (\chi_p \chi_ q| \phi_i),
\end{equation}
where
$\vec D$ is a density matrix,
$\{\chi_p\}$ is the atomic orbital basis,
$\{\phi_i\}$ is the density-fitting basis,
$J_{ij} = (\phi_i | \phi_j)$,
and $(\cdots|\cdots)$ is a two-electron integral in chemists' notation,
and a subsequent symmetrization described in Sec.~\ref{sec:derivation-a}.

\begin{itemize}
\item The \emph{pure} model simply consists of fitting the guess density
      and partitioning of the resulting vector according to the nuclei centers
      following by symmetrization,
      \begin{equation}
      \vec D^\mathrm{guess}
      \becomes{DF} \vec c
      \becomes{part.} \{\vec c_I\}
      \becomes{sym.} \{\vec v_I\},
      \end{equation}
      (\ie $c_i \in \vec c_I$ if $\phi_i$ is centered on the nuclei $I$).

\item The \emph{diff} model consists of the same steps except that
      the difference between the guess density and the superposition of atomic densities (SAD) is used,
      \begin{equation}
      \vec D^\mathrm{guess}-\vec D^\mathrm{SAD}
      \becomes{} \vec c
      \becomes{} \{\vec c_I\}
      \becomes{} \{\vec v_I\}.
      \end{equation}
\end{itemize}

Both the \emph{short} and \emph{long} models follow the L\"owdin population analysis\cite{L1950}
to partition the molecular density matrix into atomic contributions $\{\vec D_{(I)}\}$,
\begin{equation}
\vec D^\mathrm{guess}
\becomes{}  \tilde{\vec D} = \vec S^{1/2} \vec D \vec S^{1/2}
\becomes{}  \{\tilde{\vec D}_{(I)}\}
\becomes{}  \{\vec D_{(I)} = \vec S^{-1/2} \tilde{\vec D}_{(I)} \vec S^{-1/2}\},
\end{equation}
where $\vec S$ is the atomic orbitals overlap matrix.
The resulting atomic density matrices $\{\vec D_{(I)}\}$
are individually subject to density fitting and symmetrization.

\begin{itemize}
\item The \emph{long} version
      includes the coefficients related to other atom centers
      as a long-range contribution to the atomic density:
      \begin{equation}
      \vec D^\mathrm{guess}
      \becomes{L\"owdin} \vec D^\mathrm{guess}_{(I)}
      \becomes{DF} \vec c_{(I)}
      \becomes{part.} \{\vec c_{J(I)}\}
      \becomes{sym.} \{\vec v_{J(I)}\}
      \quad \forall\,I.
      \end{equation}
      To construct the final representation for atom $I$, the vectors $\{\vec v_{J(I)}\}$
      are grouped according to the nuclear charge of $J$, summed up, and concatenated,
      but it is not the only possible way to proceed.

\item The \emph{short} version only retains the coefficients directly related
      to the basis functions centered on the atom of interest,
      \begin{equation}
      \vec D^\mathrm{guess}
      \becomes{} \vec D^\mathrm{guess}_I
      \becomes{} \vec c_{(I)}
      \becomes{part.} \vec c_{I(I)}
      \becomes{sym.} \vec v_{I(I)}
      \quad \forall\,I.
      \end{equation}
\end{itemize}

The learning curves for the models are shown on Fig.~\ref{fig:atom-models}.
Overall, the \emph{long} model shows the best overall performance
and was selected as default to be used hereinafter.

\begin{figure}[ht]
\centering
\includegraphics[width=0.5\linewidth]{models.pdf}
\caption{
Learning curves of atomic charges and shielding constants
for the QM7 dataset.
The color code reflects the different models used to construct the \SPAHM[a] representation
from the rotationally-invariant vectors.
}
\label{fig:atom-models}
\end{figure}

\clearpage

%%%%%%%%%%%%%%%%%%%%%%%%%%%%%%%%%%%%%%%%%%%%%%%%%%%%%%%%%%%%%%%%%%%%%%%%%%%%%%%%%%%%%%%%%%%%%%%%%%%%

\section{Generalization to open-shell systems}
\label{sec:alphabeta}

We considered three ways to generalize the model to open-shell systems:
\begin{enumerate}[1)]
\item
concatenation of representation vectors $\vec x$
obtained from $\rho_\alpha$ and $\rho_\beta$ separately
(``$\alpha\beta$'');
\item
representation vector obtained from
$\rho = \rho_\alpha+\rho_\beta$,
the total electron density as in case of closed-shell systems~(``$+$'');
\item
concatenation of representation vectors
obtained from $\rho= \rho_\alpha+\rho_\beta$ and $\rho_m = \rho_\alpha-\rho_\beta$ separately
(``$+-$'').
\end{enumerate}
They were tested on the QM7/2-RC dataset with the \SPAHM[b] representation.
The results are shown on Fig.~\ref{fig:alphabeta}.
As expected, in most cases the ``$+$'' model,
having no information on the spin density, performed the worst,
whereas the ``$\alpha\beta$'' model showed the best results and was chosen as the default option.

\begin{figure}[h]
\centering
\includegraphics[width=0.4\linewidth]{open-shell.pdf}
\caption{
Learning curves of atomic charges and spins
for the QM7/2-RC dataset and the \SPAHM[b] representation.
The color code reflects the different models used to generalize the representation to open-shell systems.
}
\label{fig:alphabeta}
\end{figure}

\clearpage
%%%%%%%%%%%%%%%%%%%%%%%%%%%%%%%%%%%%%%%%%%%%%%%%%%%%%%%%%%%%%%%%%%%%%%%%%%%%%%%%%%%%%%%%%%%%%%%%%%%%

\section{Basis set for the bond-density-based representation}

\subsection{Optimization}
\label{sec:basis}

The decomposition of the bond density onto a midbond-centered basis set
required optimization of a suitable basis.
First, we followed the procedure described in Ref.~\onlinecite{FBGC2020b}
used to optimize a basis to fit the on-top pair density.

\bigskip

For each bond of interest in a molecule, we search for the set of coefficients $\{ c_i \}$
that approximates the bond density in the least-squares sense,
\begin{equation}
\rho_{AB}(\vec r) \approx \sum_i c_i \phi_i(\vec r),
\qquad
\vec c = \vec S^{-1} \vec b,
\end{equation}
where $\vec S$ is the overlap matrix,
$b_i = \braket{\rho_{AB}|\phi_i}$, and
the decomposition error is
\begin{equation}
\mathcal E=
\int \Big(\rho_{AB}(\vec r) - \sum_i c_i \phi_i(\vec r) \Big)^2 \de^3 \vec r =
\braket{\rho_{AB} | \rho_{AB}} - \vec b\transp \vec S^{-1} \vec b.
\end{equation}
Thus,
to optimize the exponents, we minimize the sum of decomposition errors $\mathcal E$
for the molecules chosen for the bond of interest.
The exponents $\{\alpha_\mu\}$ for all the angular momenta are optimized simultaneously.
The exponents are parameterized as $\alpha_\mu = \exp(p_\mu)$,
and the first derivatives of the loss functions~$\mathcal E$ with respect to the exponents are computed as follows,
\begin{equation}
\pdd{\mathcal E}{\alpha_\mu} =
\vec c\transp \left( \pdd{\vec S}{\alpha_\mu}\, \vec c - 2\,\pdd{\vec b}{\alpha_\mu} \right),
\end{equation}
with the overlap integrals and their derivatives taken numerically.

All the bonds were treated separately.
For each bond (or atom pair) presented in the QM7 and APS datasets
we chose representative molecules containing it
(\eg, \ce{H2} for \ce{H}--\ce{H};
\ce{C2H2}, \ce{C2H4}, and \ce{C2H6} for \ce{C}--\ce{C};
\ce{H2O} and \ce{H2O2} for \ce{H}--\ce{O}),
and the sum of the molecular decomposition errors was minimized.
The maximum angular momentum $\ell_\mathrm{max}$ and the number of functions $n_\ell$ for each $\ell$
were gradually increased and optimized on each step,
until addition of further radial functions or
angular momenta did not provide any significant decrease of error.
The optimized exponents are available separately in \texttt{Q-stack}
(\url{https://github.com/lcmd-epfl/Q-stack}).

\bigskip

However, for some of the bonds the fitting errors were huge (up to 20\%)
due to the fact that largest fraction of the bond density is still localized on participating nuclei,
thus the fine-tuning of the fitting basis could not improve much.
This could be solved with adding a single Gaussian centered in the midbond as a weight function.
Our tests showed that, however the fitting error significantly decreased,
the quality of learning was almost the same.

On Fig.~\ref{fig:bond-basis} we compare the performance of \SPAHM[b] computed
using the fully-optimized basis for each bond (``normal'')
and using the same (\ce{C}--\ce{C}) basis for every bond (``same basis'').
It is clear that the representation quality does not depend on the exponents of the basis
thus their optimization can be omitted.
(the role of angular momenta is discussed in Sec.~\ref{sec:simplebond}).

\begin{figure}[ht]
\centering
\includegraphics[width=0.5\linewidth]{bond-basis.pdf}
\caption{
Learning curves of atomic charges and shielding constants
for the QM7 dataset.
The color code reflects the different basis sets used to generate the \SPAHM[b] representations:
``normal'':        fully-optimized basis for each bond;
``same basis'':    the same (\ce{C}--\ce{C}) basis for every bond;
``only $\ell=0$'': optimized basis with $s$-orbitals only;
``only $m=0$'':    optimized basis with $m\neq0$ orbitals excluded.
}
\label{fig:bond-basis}
\end{figure}

\clearpage

%%%%%%%%%%%%%%%%%%%%%%%%%%%%%%%%%%%%%%%%%%%%%%%%%%%%%%%%%%%%%%%%%%%%%%%%%%%%%%%%%%%%%%%%%%%%%%%%%%%%

\subsection{Simplified models}
\label{sec:simplebond}

We also tested two approaches to simplify the bond-based representation,
which reduce the effort for both the two-electron integral evaluation and vector symmetrization.

\smallskip

The first one is to use only the $s$-orbitals.
The learning curves for the QM7 dataset
for the representation based on the fully-optimized basis truncated to the functions with $\ell=0$
are shown on Fig.~\ref{fig:bond-basis}.
Its peformance is significantly deteriorated and it is clear that higher angular momenta are necessary.

\smallskip

Another option is to use the orbitals with $m=0$, \ie, symmetric with respect to rotation around the bond.
Then Eq.~\ref{eq:bond-kernel-full} is simplified to
\begin{equation}
\kovlp{AB,XY}
=
\sum_{\ell \ell'}  \sum_{\substack{n_1n'_1 \\ n_2n'_2 }}
\underbrace{c_{n_1 \ell 0} c_{ n_1' \ell' 0} }_{u^{AB}_{p}}
\underbrace{
A_{n_1  n_2 }^{\ell}
A_{n_1' n_2'}^{\ell'}
}_{M_{pq}}
\underbrace{c_{ n_2 \ell 0} c_{n_2' \ell' 0 } }_{u^{XY}_{q}}.
\end{equation}
In the current implementation,
the bond density is first projected onto the DF basis set
and then rotated so the bond is aligned with the $z$-axis
and the DF coefficients are transformed accordingly.
This is why in our tests the density is fitted with the ``full'' basis set
and only the final representation is truncated to have only products of cylindrically-symmetric orbitals.

The learning curves for QM7 and for APS-RC
comparing the truncated representation with the full one
are shown on Fig.~\ref{fig:bond-basis} and Fig.~\ref{fig:azoswitch-m0}, respectively.
For QM7, the truncated representation yields the same or slightly worse performance,
whereas for a more challenging APS-RC it even improves the learning in some cases.

\smallskip

While functions with $\ell=0$ are not sufficient to construct a good representation,
the representation built from $m=0$ only
performs very well on simple organic molecules
and at least in the case of the APS-RC dataset
the part of the density that seems to be orthogonal to the aromatic ring
is well enough captured by \eg $d_{z^2}$-orbital.
This simplification of \SPAHM[b] is promising in terms of both
performance and potential optimizations and should be studied further.

\begin{figure}[ht]
\centering
\includegraphics[width=0.46\linewidth]{azoswitch_m0.pdf}
\caption{
Learning curves of atomic charges and spins
for the APS-RC dataset.
The color code reflects the different basis sets used to generate the \SPAHM[b] representations:
``normal'': fully-optimized basis for each bond;
``only $m=0$'': optimized basis with $m\neq0$ orbitals excluded.
Learning curves for SLATM are given for comparison.
}
\label{fig:azoswitch-m0}
\end{figure}

\clearpage
%%%%%%%%%%%%%%%%%%%%%%%%%%%%%%%%%%%%%%%%%%%%%%%%%%%%%%%%%%%%%%%%%%%%%%%%%%%%%%%%%%%%%%%%%%%%%%%%%%%%

\section{Effect of the Hamiltonian}
\label{sec:potentials}

We compared the \SPAHM[a,b] representations built upon the density matrices obtained from
the H\"uckel guess\cite{H1963,L2019}, the LB\cite{LB2020} guess (default),
and a converged PBE0\cite{AB1999} computation.
The learning curves are shown on Fig.~\ref{fig:potentials}.
As expected, the worst approximation, the H\"uckel guess, gives the worst regression results.
In contrast to the eigenvalue \SPAHM,\cite{FBC2022},
the converged density makes the best representation,
sometimes overperforming SLATM, which opens the way to improvement of \SPAHM[a,b]
through improvement of the underlying guess Hamiltonian.

\begin{figure}[h]
\centering
\includegraphics[width=1.0\linewidth]{potential.pdf}
\caption{
Learning curves of atomic charges and shielding constants
for the QM7 dataset.
The color code reflects the different Hamiltonians used to generate the \SPAHM[a,b] representations.
}
\label{fig:potentials}
\end{figure}

\clearpage
%%%%%%%%%%%%%%%%%%%%%%%%%%%%%%%%%%%%%%%%%%%%%%%%%%%%%%%%%%%%%%%%%%%%%%%%%%%%%%%%%%%%%%%%%%%%%%%%%%%%

\section{Comparison with the KDFA representation}
\label{sec:mr2021}

Recently the kernel density functional approximation\cite{MR2021} (KDFA) was proposed,
similar in construction to our \SPAHM[a] model.

In KDFA, the representation vector for an atom
is also built from the density-fitting coefficients of the functions centered on its nucleus.
Instead of the coefficients themselves, rotationally-invariant sums $\sum_m |c_{nlm}|^2$ are used.
This could be seen as a simplification of Eq.~\ref{eq:atom-kernel-full}
with a combination of Kronecker deltas instead of $M_{pq}$,
omitting the cross-products of different radial basis functions,
\begin{equation}
K^\mathrm{KDFA}_{A,B} =
\sum_{\substack{\ell \\ n_1n'_1 \\ n_2n'_2 }}
\underbrace{
\left(\sum_m    c_{n_1\ell m}^A   c_{n_2\ell m}^A\right)
}_{u^A_p}
\underbrace{
\vphantom{\Bigg|}  \delta_{n_1 n_2} \delta_{n'_1 n'_2} \delta_{n_1 n'_1}
}_{M_{pq}}
\underbrace{
\left(\sum_{m} c_{n'_1\ell m}^B c_{n'_2\ell m}^B\right)
}_{u^B_q}
=
\sum_{n \ell}
\underbrace{
\left(\sum_m | c_{n\ell m}^A |^2\right)
}_{v^A_p}
\underbrace{
\left(\sum_{m} | c_{n\ell m}^B |^2\right)
}_{v^B_q}.
\end{equation}

The learning curves comparing the performance of the KDFA representation
with our \emph{pure} and \emph{long} models (see Sec.~\ref{sec:models})
are shown of Fig.~\ref{fig:mr2021}.
Overall, the performance of the KDFA representation is close to the \emph{pure} model.
However, the \emph{long} model is consistently better,
presumably due to inclusion of ``long-range'' contributions to the atomic density.

\begin{figure}[ht]
\centering
\includegraphics[width=0.5\linewidth]{MR2021.pdf}
\caption{
Learning curves of atomic charges and shielding constants
for the QM7 dataset.
The color code reflects the different representations.
``MR2021'' stands for the KDFA\cite{MR2021} representation.
}
\label{fig:mr2021}
\end{figure}

%%%%%%%%%%%%%%%%%%%%%%%%%%%%%%%%%%%%%%%%%%%%%%%%%%%%%%%%%%%%%%%%%%%%%%%%%%%%%%%%%%%%%%%%%%%%%%%%%%%%

\section*{References}

\bibliography{spahm+.bib}

\end{document}
