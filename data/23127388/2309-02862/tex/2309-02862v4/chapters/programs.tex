\section{Concurrent Programs}

\subsection{Syntax}

Let $\valset$ be a finite data domain and $\varset$ be a finite set of shared variables over $\valset$.
We define the \emph{instruction set}
\begin{align*}
\instrs :=\ & \set{ \rd\of\xd, \wr\of\xd \mid \xvar \in \varset, \dval \in \valset } \\
	 \cup\ & \set{ \arw\of\xdd \mid \xvar \in \varset, \dval, \dval' \in \valset } \\
	 \cup\ & \set{ \nop, \mf }
\end{align*}
which are called \emph{read}, \emph{write}, \emph{atomic read-write}, \emph{skip} and \emph{memory fence}, respectively.
A process is represented by a finite state labelled transition system.
It is given as the triple $\process = \tuple{ \stateset, \instrs, \transition }$, where $\stateset$ is a finite set of \emph{local states} and $\transition \subseteq \stateset \times \instrs \times \stateset$ is the transition relation.
As with transition systems, we write $\state_1 \to[\instr] \state_2$ if $\tuple{ \state_1, \instr, \state_2} \in \transition$ and $\state_1 \to \state_2$ if there exists some $\instr$ such that $\state_1 \to[\instr] \state_2$.

A \emph{concurrent program} is a tuple of processes $\program = \tuple{ \process^\pid }_{\pid \in \indexset}$, where $\indexset$ is a finite set of process identifiers.
For each $\pid \in \indexset$ we have $\process^\pid = \tuple{ \stateset^\pid, \instrs, \transition^\pid }$.
A \emph{global} state of $\program$ is a function $\statemap: \indexset \to \bigcup_{\pid \in \indexset} \stateset^\pid$ that maps each process to its local state, i.e $\statemap(\pid) \in \stateset^\pid$.
\autoref{fig:concurrent-program} shows a simple example of a concurrent program $\program$ consisting of two processes $\process^1$ and $\process^2$.
\begin{figure}
\centering
\begin{tikzpicture}[xscale=2,yscale=-2]

    \node at (-1, 0) {$\process^1$:};
    \node at ( 2, 0) {$\process^2$:};

    \node (q1) at (0, 0) {$\state_1$};
    \node (q2) at (0, 1) {$\state_2$};
    \node (r1) at (3, 0) {$\rstate_1$};
    \node (r2) at (3, 1) {$\rstate_2$};

    \draw[->] (q1) -- node[right] {$\wr(\xvar, 1)$} (q2);
    \draw[->] (r1) -- node[right] {$\rd(\xvar, 1)$} (r2);

    \draw[->] (q2) to [out=45,in=-45,loop] node[right] {$\nop$} (q2);
    \draw[->] (r1) to [out=45,in=-45,loop] node[right] {$\nop$} (r1);
    \draw[->] (r2) to [out=45,in=-45,loop] node[right] {$\nop$} (r2);

\end{tikzpicture}
\caption{Concurrent program $\program = \tuple{\process^1, \process^2}$}
\label{fig:concurrent-program}
\end{figure}

\subsection{SC Semantics}

Under SC (Sequential Consistency) semantics, all processes of a concurrent program interact with the shared memory directly, and all operations appear to be executed in some sequential order that is consistent with the program order of each individual process.

Formally, an SC \emph{configuration} is a tuple $\conf = \tuple{ \statemap, \memorymap }$, where $\statemap: \indexset \to \bigcup_{\pid \in \indexset} \stateset^\pid$ is a global state of $\program$ and $\memorymap: \varset \to \valset$ represents the memory state of each shared variable.
Given a configuration $\conf$, we write $\statemap\of\conf$ and $\memorymap\of\conf$ for the global program state and memory state of $\conf$.

The semantics of a concurrent program running under SC is defined by a transition system $\TS^\SC_\program = \tuple{ \confset^\SC_\program, \lblset^\SC_\program, \to^\SC_\program }$, where $\confset^\SC_\program$ is the set of all SC configurations of $\program$ and $\lblset^\SC_\program := \set{ \instr_\pid \mid \instr \in \instrs, \pid \in \indexset }$ is the set of labels.
The transition relation $\to$ (we usually drop the indices ${}^\SC_\program$ whenever they are clear from context) is given by the following rules (see \autoref{fig:sc-semantics}):
\begin{itemize}
	\item \textbf{Read:} If $\statemap(\pid) = \state$, $\memorymap(\xvar) = \dval$ and $\state \to[\rd\of\xd] \state'$, then $\tuple{ \statemap, \memorymap } \to[\rd\of\xd_\pid] \tuple{ \statemap', \memorymap }$, where $\statemap'(\pid) = \state'$ and $\statemap'(\pid') = \statemap(\pid')$ for all $\pid' \neq \pid$.
	\item \textbf{Write:} If $\statemap(\pid) = \state$ and $\state \to[\wr\of\xd] \state'$, then $\tuple{ \statemap, \memorymap } \to[\wr\of\xd_\pid] \tuple{ \statemap', \memorymap' }$, where $\statemap'(\pid) = \state'$, $\statemap'(\pid') = \statemap(\pid')$ for all $\pid' \neq \pid$, $\memorymap'(\xvar) = \dval$ and $\memorymap'(\xvar') = \memorymap(\xvar')$ for all $\xvar' \neq \xvar$.
	\item \textbf{ARW:} If $\statemap(\pid) = \state$, $\memorymap(\xvar) = \dval$ and $\state \to[\arw\of\xdd] \state'$, then $\tuple{ \statemap, \memorymap }\to[\arw\of\xdd_\pid] \tuple{ \statemap', \memorymap' }$, where $\statemap'(\pid) = \state'$, $\statemap'(\pid') = \statemap(\pid')$ for all $\pid' \neq \pid$, $\memorymap'(\xvar) = \dval$ and $\memorymap'(\xvar') = \memorymap(\xvar')$ for all $\xvar' \neq \xvar$.
	\item \textbf{Skip:} If $\statemap(\pid) = \state$ and $\state \to[\nop] \state'$, then $\tuple{ \statemap, \memorymap } \to[\nop_\pid] \tuple{ \statemap', \memorymap }$, where $\statemap'(\pid) = \state'$ and $\statemap'(\pid') = \statemap(\pid')$ for all $\pid' \neq \pid$.
\end{itemize}
The memory fence operation $\mf$ is not used in programs running under SC semantics.
If needed, it can be formally defined as following the same semantics as $\nop$.
\input{figures/SC_semantics}

A global state $\statemap_F$ is \emph{reachable} from an initial configuration $\conf_0$, if there is a configuration $\conf_F$ with $\statemap(\conf_F) = \statemap_F$ such that $\conf_F$ is reachable from $\conf_0$ in $\TS^\SC_\program$. The \textbf{state reachability problem} of SC is, given a program $\program$, an initial configuration $\conf_0$ and a final global state $\statemap_F$, to decide whether $\statemap_F$ is reachable from $\conf_0$ in $\TS^\SC_\program$.

\subsection{TSO Semantics}

Under TSO semantics, the processes of a concurrent program do not interact with the shared memory directly, but indirectly through FIFO \emph{store buffers} instead.
When performing a \emph{write} instruction $\wr\of\xd$, the process adds a new message $\tuple\xd$ to the tail of its own store buffer.
A \emph{read} instruction $\rd\of\xd$ works differently depending on the current buffer content of the process.
If its buffer contains a write message on variable $\xvar$, the value $\dval$ must correspond to the value of the most recent such message.
Otherwise, the value is read directly from memory.
The \emph{atomic read-write} instruction $\arw\of\xdd$ is only enabled if the buffer of the process is empty and the value of $\xvar$ in the memory is $\dval$.
When executed, this instruction directly changes the value of $\xvar$ in the memory to $\dval'$.
A \emph{skip} instruction only changes the local state of the process.
The \emph{memory fence} instruction is disabled, i.e. it cannot be executed, unless the buffer of the process is empty.
Additionally, at any point during the execution, the process can \emph{update} the write message at the head of its buffer to the memory.
For example, if the oldest message in the buffer is $\tuple\xd$, it will be removed from the buffer and the memory value of variable $\xvar$ will be updated to contain the value $\dval$.
This happens in a non-deterministic manner.

Formally, we introduce a TSO \emph{configuration} as a tuple $\conf = \tuple{ \statemap, \buffermap, \memorymap }$, where:
\begin{itemize}
	\item $\statemap: \indexset \to \bigcup_{\pid \in \indexset} \stateset^\pid$ is a global state of $\program$.
	\item $\buffermap: \indexset \to (\varset \times \valset)\kstar$ represents the buffer state of each process.
	\item $\memorymap: \varset \to \valset$ represents the memory state of each shared variable.
\end{itemize}
Given a configuration $\conf$, we write $\statemap\of\conf$, $\buffermap\of\conf$ and $\memorymap\of\conf$ for the global program state, buffer state and memory state of $\conf$.
The semantics of a concurrent program running under TSO is defined by a transition system $\TS^\TSO_\program = \tuple{ \confset^\TSO_\program, \lblset^\TSO_\program, \to^\TSO_\program }$,
where $\confset^\TSO_\program$ is the set of all possible TSO configurations
and $\lblset^\TSO_\program := \set{ \instr_\pid \mid \instr \in \instrs, \pid \in \indexset } \cup \set{ \up_\iota \mid \pid \in \indexset }$ is the set of labels.
The transition relation $\to$ (i.e. $\to^\TSO_\program$) is given by the rules in \autoref{fig:tso-semantics}, where we use $\buffermap\of\pid|_{\set\xvar \times \valset}$ to denote the restriction of $\buffermap\of\pid$ to write messages on the variable $\xvar$.

\begin{figure}
\centering
\begin{equation*}
\begin{array}{lc}

\inference{read-own-write}
	{\state \to[\rd\of\xd] \state' \qquad \statemap\of\pid = \state \qquad \buffermap\of\pid|_{\set\xvar \times \valset} = \tuple\xd \cdot \word}
	{\tuple{ \statemap, \buffermap, \memorymap} \to[\rd\of\xd_\pid] \tuple{ \statemap[\pid \leftarrow \state'], \buffermap, \memorymap}}
\bigskip\\
\inference{read-from-memory}
	{\state \to[\rd\of\xd] \state' \qquad \statemap\of\pid = \state \qquad \buffermap\of\pid|_{\set\xvar \times \valset} = \varepsilon \qquad \memorymap\of\xvar = \dval}
	{\tuple{ \statemap, \buffermap, \memorymap} \to[\rd\of\xd_\pid] \tuple{ \statemap[\pid \leftarrow \state'], \buffermap, \memorymap}}
\bigskip\\
\inference{write}
	{\state \to[\wr\of\xd] \state' \qquad \statemap\of\pid = \state}
	{\tuple{ \statemap, \buffermap, \memorymap} \to[\wr\of\xd_\pid] \tuple{ \statemap[\pid \leftarrow \state'], \buffermap[\pid \leftarrow \tuple\xd \cdot \buffermap\of\pid], \memorymap}}
\bigskip\\
\inference{arw}
	{\state \to[\arw\of\xdd] \state' \qquad \statemap\of\pid = \state \qquad \buffermap\of\pid = \varepsilon \qquad \memorymap\of\xvar = \dval}
	{\tuple{ \statemap, \buffermap, \memorymap} \to[\arw\of\xdd_\pid] \tuple{ \statemap[\pid \leftarrow \state'], \buffermap, \memorymap[\xvar \leftarrow \dval']}}
\bigskip\\
\inference{skip}
	{\state \to[\nop] \state' \qquad \statemap\of\pid = \state}
	{\tuple{ \statemap, \buffermap, \memorymap} \to[\nop_\pid] \tuple{ \statemap[\pid \leftarrow \state'], \buffermap, \memorymap}}
\bigskip\\
\inference{memory-fence}
	{\state \to[\mf] \state' \qquad \statemap\of\pid = \state \qquad \buffermap\of\pid = \varepsilon}
	{\tuple{ \statemap, \buffermap, \memorymap} \to[\mf_\pid] \tuple{ \statemap[\pid \leftarrow \state'], \buffermap, \memorymap}}
\bigskip\\
\inference{update}
	{\buffermap\of\pid = \word \cdot \tuple\xd}
	{\tuple{ \statemap, \buffermap, \memorymap} \to[\up_\pid] \tuple{ \statemap, \buffermap[\pid \leftarrow \word], \memorymap[\xvar \leftarrow \dval]}}

\end{array}
\end{equation*}
\caption{TSO semantics}
\label{fig:tso-semantics}
\end{figure}


A global state $\statemap_F$ is \emph{reachable} from an initial configuration $\conf_0$, if there is a configuration $\conf_F$ with $\statemap(\conf_F) = \statemap_F$ such that $\conf_F$ is reachable from $\conf_0$ in $\TS_\program$.
The \textbf{state reachability problem} of TSO is, given a program $\program$, an initial configuration $\conf_0$ and a final global state $\statemap_F$, to decide whether $\statemap_F$ is reachable from $\conf_0$ in $\TS_\program$.

We define $\up\kstar$ to be the transitive closure of $\set{ \up_\iota \mid \pid \in \indexset }$, i.e. $\conf_1 \to[\up\kstar]_\program \conf_2$ if and only if $\conf_2$ can be obtained from $\conf_1$ by some amount of buffer updates.
