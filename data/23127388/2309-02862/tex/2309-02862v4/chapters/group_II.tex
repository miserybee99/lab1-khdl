\section{Group II}

This group contains TSO games where both players are allowed to update the buffer \emph{only} before their own move.
Let player X be the player that has a winning strategy and player Y her opponent.
Note that this differs from the previous section, in which the players X and Y were defined based on their updating capabilities.

Similar to the argumentation for Group I, we want to show that player X also has a winning strategy where she empties the buffer in each move.
But, in contrast to before, this time there is an exception:
Since the player has to update the buffer \emph{before} her move, by updating a memory variable she might disable a read transition that she intended to execute.
Thus, we do not require her to empty the buffer in that case.

Formally, let $\game = \tuple{ \confset, \confset_A, \confset_B, \to, \confset_F}$ be a TSO game where both players are allowed to perform buffer updates exactly before their own moves.
Suppose $\sigma_X$ is a winning strategy for player X and some configuration $\conf_0$.
We construct another strategy $\bar\sigma_X$ for player X.
Let $\conf \in \confset_X$, $\conf' = \sigma_X(\conf)$ and $\bar\conf$ as in the previous section, i.e. a (non-unique) configuration such that $\conf \to[\up\kstar] \bar\conf$ and the buffers of $\bar\conf$ are empty.
Suppose that $\conf \to[\instr_\pid] \conf'$, where $\instr_\pid$ is not a read or atomic read-write instruction.
Then, starting from $\conf$, updating all buffer messages does not change that the transition from $\statemap(\conf)\of\pid$ to $\statemap(\conf')\of\pid$ is enabled.
Thus, $\instr_\pid$ can also be executed from $\bar\conf$.
We call the resulting configuration $\tilde\conf'$ and observe that $\bar\conf \to \tilde\conf'$ and $\conf' \to[\up\kstar] \tilde\conf'$.
We define $\bar\sigma_X(\conf) := \tilde\conf'$.
This can be seen in \autoref{fig:group-II}.
Note that $\tilde\conf'$ may have at most one message in its buffers.
In the other case, where there is no transition from $\conf$ to $\conf'$ other than read or atomic read-write instructions, we define $\bar\sigma_X(\conf) := \sigma_X(\conf) = \conf'$ and observe that $\conf'$ cannot have more buffer messages than $\conf$.

\begin{figure}
\centering
\begin{tikzpicture}[xscale=3,yscale=-1]
    \node	at (0,-1)	(c0) {$\conf$};
    \node	at (0,1)	(c0b){$\bar\conf$};
    \node	at (1,-1)	(c1) {$\conf'$};
    \node	at (1,1)	(c1t){$\tilde\conf'$};
    \node	at (2,-1)	(c2) {$\conf''$};

    \draw[->] (c0) -- node[above] {$\sigma_X$}
                      node[below] {$\instr_\pid$}   (c1);
    \draw[->] (c0) -- node[below] {$\bar\sigma_X$}  (c1t);
    \draw[->] (c0) -- node[right] {$\up\kstar$}     (c0b);
    \draw[->] (c0b)-- node[below] {$\instr_\pid$}   (c1t);
    \draw[->] (c1) -- node[right] {$\up\kstar$}     (c1t);
    \draw[->] (c1) -- node[above] {$\bar\sigma_Y$}  (c2);
    \draw[->] (c1t)-- node[below] {$\sigma_Y$}      (c2);

    % \draw[thick,decoration={brace},decorate] (c1b.north -| c2.east) -- node[right]{\ empty buffers} (c1b.south -| c2.east);
\end{tikzpicture}
\caption{Commutative diagram of strategies in games of group II, in the case where $\instr_\pid \neq \rd\of\xd$ and $\instr_\pid \neq \arw\of\xdd$.}
\label{fig:group-II}
\end{figure}


\begin{clm}
\label{claim:bb1}
    $\bar\sigma_X$ is a winning strategy for $\conf_0$.
\end{clm}
\begin{proof}
    First, suppose that $\conf_0 \in \confset_X$ and let $\sigma_Y$ be an arbitrary strategy of player Y.
    We define another (non-positional) strategy $\bar\sigma_Y$, that depends on the last two configurations, by $\bar\sigma_Y(\conf, \conf') := \sigma_Y(\bar\sigma_X(\conf))$.
    We observe that for all $\conf \in \confset_X$, it holds that $\bar\sigma_Y(\conf, \sigma_X(\conf)) = \sigma_Y(\bar\sigma_X(\conf))$.
    It follows that the play $\play_1$ induced by $\sigma_X$ and $\bar\sigma_Y$ and the play $\play_2$ induced by $\bar\sigma_X$ and $\sigma_Y$ agree on every second configuration, i.e. the configurations in $\confset_X$.
    In particular, the sequence of visited global TSO configurations is the same in both plays.
    Since $\sigma_X$ is winning, it means that $\play_1$ is winning for player X and thus also $\play_2$ is winning.
    Because we chose $\sigma_Y$ arbitrarily, it follows that $\bar\sigma_X$ is a winning strategy.

    Otherwise, if $\conf_0 \in \confset_Y$, we consider the successors of $\conf_0$ instead.
    We note that $\bar\sigma_X$ must also be a winning strategy for each $\conf \in \post(\conf_0)$.
    But then, we can apply the previous arguments to each of those configurations and conclude that $\bar\sigma_X$ is a winning strategy for all of them.
    Thus, it is also a winning strategy for $\conf_0$.
\end{proof}

We conclude that if player X has a winning strategy $\sigma_X$, then she also has a winning strategy $\bar\sigma_X$ where she empties the buffers before every turn in which she does not perform a read operation.
By symmetry, the same holds true for player Y.
Thus, we can limit our analysis to this type of strategies.
We see that the number of messages in the buffers is bounded:
Suppose that the game is in configuration $\conf \in \confset_X$.
Then, $\bar\sigma_X$ either empties the buffer and adds at most one new message, or it performs a transition due to a read instruction, which does not increase the size of the buffers.
The analogous argumentation holds for player Y.
Hence, we can reduce the game to a game on bounded buffers, which is finite state and thus decidable.

Given the configuration $\conf_0$ as above, we construct a finite game $\game' = \tuple{ \confset', \confset_X', \confset_Y', \to', \confset_F'}$ as follows.
The set $\confset_X'$ contains all configurations from $\confset_X$ which have at most as many buffer messages than $\conf_0$ (or at most one message, if $\conf_0$ has empty buffers):
$$\confset_X' := \Set{ \conf \in \confset_X \mid \sizeof{\buffermap\of\conf} \leq \max\set{1, \sizeof{\buffermap\of{\conf_0}}} } \qquad \text{where} \qquad \sizeof\buffermap = \sum_{\pid\in\indexset} \sizeof{\buffermap\of\pid}$$
The set $\confset_Y'$ is defined accordingly.
Note that both sets are finite.
Lastly, $\to'$ is defined as the restriction of $\to$ to configurations of $\game'$, and $\confset_F' := \confset_F \cap \confset_A'$.
We define $\bar\sigma_X'$ to be the restriction of $\bar\sigma_X$ to $\confset_X'$.
Since $\bar\sigma_X'(\conf) \in \confset_Y'$ for all $\conf \in \confset_X'$, $\bar\sigma_X'$ is indeed a valid strategy for $\game'$.
In particular, it is the restriction of $\bar\sigma_X$ to $\game'$.

\begin{clm}
\label{claim:bb2}
    $\bar\sigma_X'$ is a winning strategy for $\conf_0$ in $\game'$.
\end{clm}
\begin{proof}
    First, consider the case where $\conf_0 \in \confset_X$.
    Let $\sigma_Y'$ be a strategy for player Y in $\game'$ and let $\sigma_Y$ be an arbitrary extension of $\sigma_Y'$ to $\game$.
    The play $\play$ induced by $\bar\sigma_X$ and $\sigma_Y$ in $\game$ is the same as the play $\play'$ induced by $\bar\sigma_X'$ and $\sigma_Y'$ in $\game'$.
    Since $\bar\sigma_X$ is a winning strategy, $\play$ is a winning play.
    It follows that $\play'$ must also be a winning strategy.
    Since $\sigma_Y'$ was arbitrary, it follows that $\bar\sigma_X'$ is a winning strategy and $\conf_0$ is winning in $\game'$.
\end{proof}

\begin{thm}
    The safety problem for games of group II is \exptime-complete.
\end{thm}
\begin{proof}
    By \autoref{claim:bb1} and \autoref{claim:bb2}, if a configuration $\conf_0$ is winning for player A in game $\game$, then it is also winning in $\game'$.
    The same holds true for player B.
    Thus, the safety problem for $\game$ is equivalent to the safety problem for $\game'$.
    Similar to the games of group I, $\game'$ is finite and has exponentially many configurations.
    By \autoref{lem:finite} and \autoref{cor:complexity}, we can again conclude that the safety problem is \exptime-complete.
\end{proof}
