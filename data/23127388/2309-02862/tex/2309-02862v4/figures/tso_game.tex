\begin{figure}
    \centering
    \begin{tikzpicture}[xscale=8,yscale=-3]
        \definecolor{up}{RGB}{170,0,0}
        \definecolor{fin}{RGB}{0,170,0}
        \definecolor{ins}{RGB}{0,0,170}
        \tikzset{up/.style={draw=up}}
        \tikzset{fin/.style={color=fin}}
        \tikzset{ins/.style={color=ins}}
    
        \node (q1r1-A) at (0, 0) {$\tuple{ (\state_1, \rstate_1), (\varepsilon, \varepsilon), \set{ x \mapsto 0 } }_A$};
        \node (q1r1-B) at (1, 0) {$\tuple{ (\state_1, \rstate_1), (\varepsilon, \varepsilon), \set{ x \mapsto 0 } }_B$};
    
        \node (q2r1-B) at (0, 1) {$\tuple{ (\state_2, \rstate_1), (\tuple{\xvar, 1}, \varepsilon), \set{ x \mapsto 0 } }_B$};
        \node (q2r1-A) at (1, 1) {$\tuple{ (\state_2, \rstate_1), (\tuple{\xvar, 1}, \varepsilon), \set{ x \mapsto 0 } }_A$};
    
        \node (q2r1-B') at (0, 2) {$\tuple{ (\state_2, \rstate_1), (\varepsilon, \varepsilon), \set{ x \mapsto 1 } }_B$};
        \node (q2r1-A') at (1, 2) {$\tuple{ (\state_2, \rstate_1), (\varepsilon, \varepsilon), \set{ x \mapsto 1 } }_A$};

        \node[fin] (q2r2-A) at (0, 3) {$\tuple{ (\state_2, \rstate_2), (\varepsilon, \varepsilon), \set{ x \mapsto 1 } }_A$};
        \node (q2r2-B) at (1, 3) {$\tuple{ (\state_2, \rstate_2), (\varepsilon, \varepsilon), \set{ x \mapsto 1 } }_B$};
    
        \draw[->] ([yshift= 1pt] q1r1-A.east) -- node[below, ins] {$\nop$} ([yshift= 1pt] q1r1-B.west);
        \draw[->] ([yshift=-1pt] q1r1-B.west) -- node[above, ins] {$\nop$} ([yshift=-1pt] q1r1-A.east);
    
        \draw[->] (q1r1-A) -- node[right, ins] {$\wr(\xvar, 1)$} (q2r1-B);
        \draw[->] (q1r1-B) -- node[right, ins] {$\wr(\xvar, 1)$} (q2r1-A);
    
        \draw[->] ([yshift= 1pt] q2r1-B.east) -- node[below, ins] {$\nop$} ([yshift= 1pt] q2r1-A.west);
        \draw[->] ([yshift=-1pt] q2r1-A.west) -- node[above, ins] {$\nop$} ([yshift=-1pt] q2r1-B.east);

        \draw[->, up] (q1r1-A.south west) to[bend left=10] node[below right,xshift=5pt,yshift=-25pt, ins] {$\wr(\xvar, 1);\up$} (q2r1-B'.north west);
        \draw[->, up] (q2r1-A) -- node[below right, ins] {$\nop;\up$ or $\up;\nop$} (q2r1-B');
        \draw[->, up] (q2r1-A.south east) to[bend right=10] node[above left,yshift=25pt, ins] {$\up;\rd(\xvar, 1)$} (q2r2-B.north east);
        
        \draw[->] ([yshift= 1pt] q2r1-B'.east) -- node[below, ins] {$\nop$} ([yshift= 1pt] q2r1-A'.west);
        \draw[->] ([yshift=-1pt] q2r1-A'.west) -- node[above, ins] {$\nop$} ([yshift=-1pt] q2r1-B'.east);
    
        \draw[->] (q2r1-B') -- node[right, ins] {$\rd(\xvar, 1)$} (q2r2-A);
        \draw[->] (q2r1-A') -- node[right, ins] {$\rd(\xvar, 1)$} (q2r2-B);
    
        \draw[->] ([yshift= 1pt] q2r2-A.east) -- node[below, ins] {$\nop$} ([yshift= 1pt] q2r2-B.west);
        \draw[->] ([yshift=-1pt] q2r2-B.west) -- node[above, ins] {$\nop$} ([yshift=-1pt] q2r2-A.east);
    
\end{tikzpicture}
\caption{
    The transition relation of the TSO game $\game^\TSO(\program, \set{\rstate_2})$ induced by the program $\program$ from \autoref{fig:concurrent-program}, in the case where player A is allowed to update before and after her turn, but player B is not allowed to update buffer messages.
    Note that only configurations reachable from $\tuple{ (\state_1, \rstate_1), (\varepsilon, \varepsilon), \set{ x \mapsto 0 } }_A$ are shown.
    The labels in blue are not formally part of the game definition, but are included to indicate which instruction of $\program$ gives rise to the transition.
    The transitions in red are due to updates; they correspond to both an instruction and an update operation.
    The configuration in green is the final state induced by the set of final local states $\stateset_F^\program := \set{\rstate_2}$.
}
\label{fig:tso-game}
\end{figure}