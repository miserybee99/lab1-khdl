\section{Undecidability of the B-TSO Game}
\label{apx:btso}

In this section we prove the undecidability of the B-TSO game, which is the TSO game where player B has full control over the buffer updates.
Similar to what was presented in the A-TSO section, we construct a program consisting of two processes $\program = \tuple{\process^1, \process^2}$.
As previously, $\process^1$ starts out with all states of $\channelstateset$ and is then augmented by auxiliary states and transitions to simulate the PCS behaviour.
This is sketched in \autoref{fig:b-reduction-1}.
The construction of $\process^2$ is shown in \autoref{fig:b-reduction-2}.
Note that each state of $\process^1$ labelled as $\hstate$ is supposed to be distinct for each transition.
However, all other states with the same label, i.e. $\hstate_L$, $\hstate_F$, $\state_L$, $\state_F$ and $\state_1$, are actually representing the same state, respectively.
The visual separation was done just for clarity.
Note that $\process^2$ is deadlock-free, which makes a dedicated $\process^3$ superfluous.
The set of final states is defined as $\stateset_F^\program := \set{\state_F}$.

\begin{figure}
\centering

% skip
\begin{subfigure}[b]{0.45\linewidth}
\centering
\begin{tikzpicture}[
    state/.style={},
    xscale=1.5,yscale=-1.5
]
    \node[state]	at (0,0)	(q1) {$\channelstate$};
    \node[state]	at (0,1)	(h1) {$\hstate$};
    \node[state]	at (0,2)	(q2) {$\channelstate'$};

    \draw[->] (q1) -- node[right] {$\nop$} (h1);
    \draw[->] (h1) -- node[right] {$\nop$} (q2);

    \node[state]	at (3,1)	(hL) {$\hstate_L$};
    \draw[->] (h1) -- node[above right] {$\rd(\xrd, \channelmessage')$} (hL);
    \draw[->] (hL) to [out=315,in=45,loop] node[right] {$\nop$} (hL);
\end{tikzpicture}
\bigskip
\caption{skip operation $\channelstate \to[\nop]_\channelsystem \channelstate'$}
\end{subfigure}
\hfill
%
% send
\begin{subfigure}[b]{0.45\linewidth}
\centering
\begin{tikzpicture}[
    state/.style={},
    xscale=1.5,yscale=-1.5
]
    \node[state]	at (0,0)	(q1) {$\channelstate$};
    \node[state]	at (0,1)	(h1) {$\hstate$};
    \node[state]	at (0,2)	(q2) {$\channelstate'$};

    \draw[->] (q1) -- node[right] {$\wr(\xwr,\channelmessage)$} (h1);
    \draw[->] (h1) -- node[right] {$\wr(\yvar,1)$} (q2);

    \node[state]	at (3,1)	(hL) {$\hstate_L$};
    \draw[->] (h1) -- node[above right] {$\rd(\xrd, \channelmessage')$} (hL);
    \draw[->] (hL) to [out=315,in=45,loop] node[right] {$\nop$} (hL);
\end{tikzpicture}
\bigskip
\caption{send operation $\channelstate \to[!\channelmessage]_\channelsystem \channelstate'$}
\end{subfigure}
\bigskip
\bigskip

% receive
\begin{subfigure}[b]{0.45\linewidth}
\centering
\begin{tikzpicture}[
    state/.style={},
    xscale=1.5,yscale=-1.5
]
    \node[state]	at (0,0)	(q1) {$\channelstate$};
    \node[state]	at (0,1)	(h1) {$\hstate$};
    \node[state]	at (0,2)	(q2) {$\channelstate'$};

    \draw[->] (q1) -- node[right] {$\rd(\xrd,\channelmessage)$} (h1);
    \draw[->] (h1) -- node[right] {$\rd(\xrd,\bot)$} (q2);
\end{tikzpicture}
\bigskip
\caption{receive operation $\channelstate \to[?\channelmessage]_\channelsystem \channelstate'$}
\end{subfigure}
\hfill
%
% final gadget
\begin{subfigure}[b]{0.45\linewidth}
\centering
\begin{tikzpicture}[
    state/.style={},
    xscale=1.5,yscale=-1.5
]
    \node[state]	at (0,0)	(q1) {$\channelstate_F$};
    \node[state]	at (0,1)	(h1) {$\hstate_F$};

    \draw[->] (q1) -- node[right] {$\wr(\xwr,\top)$} (h1);
    \draw[->] (h1) to [out=45,in=135,loop] node[below] {$\nop$} (h1);
\end{tikzpicture}
\bigskip
\caption{construction for each $\channelstate_F \in \channelstateset_F$}
\end{subfigure}

\caption{$\process^1$ of the B-TSO reduction from PCS}
\label{fig:b-reduction-1}
\end{figure}

\begin{figure}
\centering
\begin{tikzpicture}[
    state/.style={},
    xscale=1.3,yscale=-1.3
]
    \node[state] at (0,0) (q1) {$\state_1$};
    \node[state] at (0,1) (qm) {$\state_\channelmessage$};
    \node[state] at (0,2) (q3) {$\state_3$};
    \node[state] at (0,3) (q4) {$\state_4$};
    \node[state] at (0,4) (q5) {$\state_5$};
    \node[state] at (0,5) (q6) {$\state_6$};
    \node[state] at (0,6) (q7) {$\state_7$};
    \node[state] at (0,7) (q8) {$\state_8$};
    \node[state] at (0,8) (q9) {$\state_9$};
    \node[state] at (0,9) (q10) {$\state_{10}$};
    \node[state] at (0,10) (q11) {$\state_{11}$};
    \node[state] at (0,11) (q12) {$\state_{12}$};
    \node[state] at (0,12) (q13) {$\state_{13}$};
    \node[state] at (0,13) (q14) {$\state_{14}$};
    \node[state] at (0,14) (q1') {$\state_{1}$};

    \draw[->] (q1) -- node[right] {$\rd(\xwr,\channelmessage)$} (qm);
    \draw[->] (qm) -- node[right] {$\wr(\xrd,\channelmessage)$} (q3);
    \draw[->] (q3) -- node[right] {$\mf$} (q4);
    \draw[->] (q4) -- node[right] {$\wr(\xwr,\bot)$} (q5);
    \draw[->] (q5) -- node[right] {$\nop$} (q6);
    \draw[->] (q6) -- node[right] {$\mf$} (q7);
    \draw[->] (q7) -- node[right] {$\rd(\yvar,0)$} (q8);
    \draw[->] (q8) -- node[right] {$\rd(\yvar,1)$} (q9);
    \draw[->] (q9) -- node[right] {$\wr(\yvar,0)$} (q10);
    \draw[->] (q10) -- node[right] {$\mf$} (q11);
    \draw[->] (q11) -- node[right] {$\rd(\xwr,\bot)$} (q12);
    \draw[->] (q12) -- node[right] {$\wr(\xrd,\bot)$} (q13);
    \draw[->] (q13) -- node[right] {$\nop$} (q14);
    \draw[->] (q14) -- node[right] {$\mf$} (q1');

    \node at (-1.5,0) {$\nop$};
    \node[state] at (-3,0)(h1) {$\hstate_1$};
    \node[state] at (-5,0)(f1) {$\state_F$};
    \draw[->] (q1) to [bend left=15] (h1);
    \draw[->] (h1) -- node[above] {$\nop$} (f1);
    \draw[->] (f1) to [out=225,in=135,loop] node[left] {$\nop$} (f1);
    \draw[->] (h1) to [bend left=15] (q1);

    \node at (-1.5,2) {$\nop$};
    \node[state] at (-3,2)(h2) {$\hstate_2$};
    \node[state] at (-5,2)(f2) {$\state_F$};
    \draw[->] (q3) to [bend left=15] (h2);
    \draw[->] (h2) -- node[above] {$\nop$} (f2);
    \draw[->] (f2) to [out=225,in=135,loop] node[left] {$\nop$} (f2);
    \draw[->] (h2) to [bend left=15] (q3);

    % \node at (-1.5,3) {$\nop$};
    % \node[state] at (-3,3)(h3) {$\hstate_3$};
    % \node[state] at (-5,3)(f3) {$\state_F$};
    % \draw[->] (q4) to [bend left=15] (h3);
    % \draw[->] (h3) -- node[above] {$\nop$} (f3);
    % \draw[->] (f3) to [out=225,in=135,loop] node[left] {$\nop$} (f3);
    % \draw[->] (h3) to [bend left=15] (q4);

    \node at (-1.5,7) {$\nop$};
    \node[state] at (-3,7)(h4) {$\hstate_3$};
    \node[state] at (-5,7)(f4) {$\state_F$};
    \draw[->] (q8) to [bend left=15] (h4);
    \draw[->] (h4) -- node[above] {$\nop$} (f4);
    \draw[->] (f4) to [out=225,in=135,loop] node[left] {$\nop$} (f4);
    \draw[->] (h4) to [bend left=15] (q8);

    \node at (-1.5,9) {$\nop$};
    \node[state] at (-3,9)(h5) {$\hstate_4$};
    \node[state] at (-5,9)(f5) {$\state_F$};
    \draw[->] (q10) to [bend left=15] (h5);
    \draw[->] (h5) -- node[above] {$\nop$} (f5);
    \draw[->] (f5) to [out=225,in=135,loop] node[left] {$\nop$} (f5);
    \draw[->] (h5) to [bend left=15] (q10);

    \node at (-1.5,13) {$\nop$};
    \node[state] at (-3,13)(h5) {$\hstate_5$};
    \node[state] at (-5,13)(f5) {$\state_F$};
    \draw[->] (q14) to [bend left=15] (h5);
    \draw[->] (h5) -- node[above] {$\nop$} (f5);
    \draw[->] (f5) to [out=225,in=135,loop] node[left] {$\nop$} (f5);
    \draw[->] (h5) to [bend left=15] (q14);


    \node[state] at (3,0) (qf) {$\state_L$};
    \draw[->] (q1) -- node[above right] {$\rd(\xwr, \channelmessage)$} (qf);
    \draw[->] (qf) to [out=315,in=45,loop] node[right] {$\nop$} (qf);

    \node[state] at (3,1) (qf) {$\state_F$};
    \draw[->] (qm) -- node[above right] {$\nop$} (qf);
    \draw[->] (qf) to [out=315,in=45,loop] node[right] {$\nop$} (qf);

    \node[state] at (3,2) (qf) {$\state_L$};
    \draw[->] (q3) -- node[above right] {$\nop$} (qf);
    \draw[->] (qf) to [out=315,in=45,loop] node[right] {$\nop$} (qf);

    \node[state] at (3,3) (qf) {$\state_F$};
    \draw[->] (q4) -- node[above right] {$\nop$} (qf);
    \draw[->] (qf) to [out=315,in=45,loop] node[right] {$\nop$} (qf);

    \node[state] at (3,5) (qf) {$\state_L$};
    \draw[->] (q6) -- node[above right] {$\nop$} (qf);
    \draw[->] (qf) to [out=315,in=45,loop] node[right] {$\nop$} (qf);

    \node[state] at (3,6) (qf) {$\state_L$};
    \draw[->] (q7) -- node[above right] {$\rd(\yvar,1)$} (qf);
    \draw[->] (qf) to [out=315,in=45,loop] node[right] {$\nop$} (qf);

    \node[state] at (3,10) (qf) {$\state_L$};
    \draw[->] (q11) -- node[above right] {$\rd(\xwr,\channelmessage)$} (qf);
    \draw[->] (qf) to [out=315,in=45,loop] node[right] {$\nop$} (qf);

    \node[state] at (3,13) (qf) {$\state_L$};
    \draw[->] (q14) -- node[above right] {$\rd(\xwr,\channelmessage)$} (qf);
    \draw[->] (qf) to [out=315,in=45,loop] node[right] {$\nop$} (qf);

    \node[state] at (3,14) (qf) {$\state_F$};
    \draw[->] (q1') -- node[above right] {$\rd(\xwr,\top)$} (qf);
    \draw[->] (qf) to [out=315,in=45,loop] node[right] {$\nop$} (qf);

    \draw[thick,decoration={brace, mirror},decorate] (q1.south west) -- node[left]{for all $\channelmessage \in \channelmessageset$} (q3.north west);
\end{tikzpicture}
\bigskip
\caption{$\process^2$ of the B-TSO reduction from PCS}
\label{fig:b-reduction-2}
\end{figure}


This construction, which will be described in more detail in the following, also works out for the variants of B-TSO where player B is allowed to update either only before or only after her move.
In the former case, this is clear, since the construction never requires her to perform an update between her move and the next one of player A.
In the latter case, this is due to the fact that the construction gives her enough opportunities to update during auxiliary $\nop$ transitions, which would not be needed in the B-TSO game with full buffer control.

In particular, consider the transition $\state_1 \to[\rd(\xwr, \channelmessage)] \state_\channelmessage$ in $\process^2$ as an example.
If player B wants to take this transition while the $\tuple{\xwr, \channelmessage}$ message is still in the buffer, she can instead move to $\hstate_1$ and perform an update after her move.
Player A then immediately needs to respond with going back to $\state_1$, since $\state_F$ is winning for player B.
For the same reason, player A cannot transition to $\hstate_1$ herself, which prevents the potential of entering an infinite loop.

Furthermore, consider $\state_L$ in $\process^2$.
Note that it is immediately losing for player B, since the final state $\state_F$ is not reachable from there.

\begin{thm}
\label{thm:equivalence-btso-pcs}
    Consider the B-TSO game or one of its variants.
    The set of final states $\channelstateset_F$ of $\channelsystem$ is reachable from $\channelstate_0 \in \channelstateset$ if and only if player B wins the game $\game^\TSO(\program,\stateset_F^\program)$ starting from the configuration $\conf_0 := \tuple{ \tuple{\channelstate_0, \state_1}, \tuple{\varepsilon, \varepsilon}, \set{ \xwr \mapsto \bot, \xrd \mapsto \bot, \yvar \mapsto 0} }_B \in \confset_B$.
\end{thm}

\subsection{From $\channelsystem$-Reachability to a Winning Strategy in B-TSO}

Proceeding as in the case of A-TSO games, suppose $\channelstateset_F$ is reachable from $\channelstate_0$.
Given a run $\conf_0^\channelsystem, \dots, \conf_n^\channelsystem$ in the PCS, recall the definitions of $\conf_k^\channelsystem$ in \autoref{apx:atso}.
For all $k = 0, \dots, n$, we define a game configuration $\conf_k = \tuple{ \statemap_k, \buffermap_k, \memorymap_k }_B$:
\begin{enumerate}
    \item\ $\statemap_k := \tuple{ \channelstate_k, \state_1 }$
    \item\ $\buffermap_k := \tuple{ \tuple{\tuple{\yvar,1}, \tuple{\xwr,\word_k[1]}, \dots, \tuple{\yvar,1}, \tuple{\xwr,\word_k[\sizeof{\word_k}]}, \tuple{\yvar,1}}, \varepsilon }$
    \item\ $\memorymap_k := \set{ \xwr \mapsto \bot, \xrd \mapsto \bot, \yvar \mapsto 0 }$
\end{enumerate}

We describe a strategy for player B that forces the play into visiting all configurations $\conf_1, \dots, \conf_n$.
Since $\conf_n \in \confset_F$, this means that the strategy is winning.
To achieve this, we show by induction over k that starting from $\conf_0$, Player B can force a play that visits either $\conf_k$ or $\confset_F$.
For $k = 0$, this is true since $\conf_0$ is the initial configuration.
Let the induction hypothesis be that the claim holds true for some arbitrary but fixed $0 \leq k < n$.
For the induction step, we can assume that the game is in configuration $\conf_k$.

In the following, we assume that player A avoids immediate defeats.
This does not lose generality, since otherwise the game would be in $\confset_F$ and the induction would still hold.

If $\channeloperation_{k+1} = \nop$, then player B first takes the transition $\channelstate_k \to[\nop] \hstate$ in $\process^1$.
Since all other outgoing transitions are disabled, player A has to follow up with the transition $\hstate \to[\nop] \channelstate_{k+1}$.
As there is no change in the states of $\process^2$, the buffers, or the memory, we can conclude that the current configuration of the game is $\conf_{k+1}$.

Next, consider the case where $\channeloperation_{k+1} = !\channelmessage$, which is similar to the previous one.
Player B takes $\channelstate_k \to[\wr(\xwr,\channelmessage)] \hstate$ and the only option for player A is $\hstate \to[\wr(\yvar,1)] \channelstate_{k+1}$.
The buffer of $\process^1$ has now two additional pending write operations, which correspond to the channel message $\channelmessage$, i.e. the last letter of $\word_{k+1}$.
Thus, the game is in configuration $\conf_{k+1}$.

The last case is where $\channeloperation_{k+1} = ?\channelmessage$, which is more complicated.
Player B starts by updating the message $\tuple{\xwr, \channelmessage}$ and taking the transition $\state_1 \to[\rd(\xwr, \channelmessage)] \state_\channelmessage$ in $\process^2$ (recall the remark about the B-TSO variants above -- she might need to take a detour over $\hstate_1$).
Player A has to respond with $\state_\channelmessage \to[\wr(\xrd, \channelmessage)] \state_3$, since otherwise player B could move to $\state_F$ and win.
Now, player B is in a similar situation, where she needs to leave $\state_3$, or else player A can move to $\state_L$ which prevents the game from ever reaching $\state_F$.
So, player B empties the buffer of $\process^2$ (possibly using $\hstate_2$) and takes the transition $\state_3 \to[\mf] \state_4$.
Player A has to respond with $\state_4 \to[\wr(\xwr,\bot)] \state_5$.
At this point, player B cannot further proceed in $\process^2$, since player A could respond with $\state_6 \to[\nop] \state_L$ and win.
Thus, she has to continue in $\process^1$ instead, where the only enabled transition is $\channelstate_k \to[\rd(\xrd, \channelmessage)] \hstate$.
Now, $\process^1$ is disabled and the play continues in $\process^2$.
Eventually, it will go through all states $\state_6, \dots, \state_{14}$.
Note that player B has to update the messages $\tuple{\xwr,\bot}$, $\tuple{\yvar,1}$, $\tuple{\yvar,0}$ and $\tuple{\xrd,\bot}$ along the way, and has enough opportunities to do so.
Finally, player B takes the transition $\state_{14} \to[\mf] \state_1$ in $\process^2$ and player A is forced to respond with $\hstate \to[\rd(\xrd, \bot)] \channelstate_{k+1}$ in $\process^1$.
Note that the last transition is enabled since the $\tuple{\xrd,\bot}$-message has certainly reached the memory.

To summarise the changes:
$\process^1$ transitioned from $\channelstate_k$ to $\channelstate_{k+1}$, while $\process^2$ is back in state $\state_1$.
The buffer of $\process^1$ has updated its two oldest messages, which correspond to the last letter in $\word_k$, and the buffer of $\process^2$ has been emptied.
All variables have their default values.

We conclude that the game is in configuration $\conf_{k+1}$, which completes the proof by induction.
In summary, starting from $\conf_0$, the game either reaches the configuration $\conf_n \in \confset_F$, or another configuration in $\confset_F$.
In the latter case, player B already won, but in the former case, she takes the transition $\channelstate_n \to[\wr(\xwr, \top)] \hstate_F$, updates this message, and finally reaches $\state_1 \to[\rd(\xwr, \top)] \state_F \in \stateset_F^\program$ in her next turn.

\subsection{From a Winning Strategy in B-TSO to $\channelsystem$-Reachability}

For the other direction, suppose that player B has a winning strategy.
Consider a strategy of player A that avoids reaching $\state_F \in \stateset^2 \cap \stateset_F^\program$ whenever possible.
Furthermore, to support our argumentation we can assume arbitrary behaviour of player A, since the strategy of player B must be winning in any case.

Let $\channelstate_0, \dots, \channelstate_n$ be the sequence of channel states that are visited by $\process^1$ during the run induced by these two strategies.
Since player B uses a winning strategy, this sequence is finite.
We will show by induction that for each $k = 0, \dots, n$, the run contains the game configuration $\conf_k = \tuple{ \statemap_k, \buffermap_k, \memorymap_k }_B \in \confset_B$, where:
\begin{enumerate}
    \item\ $\statemap_k := \tuple{ \channelstate_k, \state_1 }$
    \item\ $\buffermap_k := \tuple{ \tuple{\tuple{\yvar,1}, \tuple{\xwr,\word_k[1]}, \dots, \tuple{\yvar,1}, \tuple{\xwr,\word_k[\sizeof{\word_k}]}}, \varepsilon }$, where $\word_k \in \channelmessageset\kstar$
    \item\ $\memorymap_k := \set{ \xwr \mapsto \bot, \xrd \mapsto \bot, \yvar \mapsto 0 }$
    \item\ If $k > 0$, there is a label $\channeloperation_k$ such that $\tuple{ \channelstate_{k-1}, \word_{k-1} } \to[\channeloperation_k] \tuple{ \channelstate_k, \word_k }$.
\end{enumerate}
The induction base case clearly holds true, since $\conf_0$ is the initial configuration of the game.
So, suppose that the claim holds for some arbitrary but fixed $0 \leq k < n$.
By the induction hypothesis, we can assume that the game is in configuration $\conf_k$.
Consider the different moves player B can make.

First, she may take a transition $\channelstate_k \to[\nop] \hstate$.
She surely does not update any buffer messages during that move, since this would enable player A to take $\state_1 \to[\rd(\xwr, \channelmessage)] \state_L$ and win.
So, we assume that player A moves along $\hstate \to[\nop] \channelstate'$.
By the construction of $\process^1$, the path $\channelstate_k \to[\nop] \hstate \to[\nop] \channelstate'$ implies the existence of a transition $\channelstate_k \to[\nop] \channelstate'$.
We define $\channelstate_{k+1} := \channelstate'$ and $\word_{k+1} := \word_k$ and conclude that the game is now in configuration $\conf_{k+1}$.

Another option for player A is to take the transition $\channelstate_k \to[\wr(\xwr, \channelmessage)] \hstate$.
The argumentation continues exactly as in the previous paragraph, the only difference is that $\channelstate_k \to[!\channelmessage] \channelstate'$ and $\word_{k+1} := \channelmessage \bullet \word_k$.

Since all other transitions in $\process^1$ are disabled, the last possibility for player B to move is in $\process^2$.
She first has to update a $\tuple{\xwr, \channelmessage}$-message to the memory (potentially using the loop through $\hstate_1$), and then she is able to perform $\state_1 \to[\rd(\xwr, \channelmessage)] \state_\channelmessage$.
Assume that player A responds with $\state_\channelmessage \to[\wr(\xrd, \channelmessage)] \state_3$.
Since staying in $\state_3$ is losing for player B, she needs to update the $\tuple{\xrd, \channelmessage}$-message to the memory, which enables $\state_3 \to[\nop] \state_4$ in $\process^2$.
Player A is now forced to proceed from there and takes $\state_4 \to[\wr(\xwr, \bot)] \state_5$.

At this point, player B cannot continue in $\process^2$, since moving to $\state_6$ opens up the opportunity for player A to immediately force her win.
However, also in $\process^1$ she has limited possibilities.
In case she starts to simulate a skip or send operation, player A can react with $\hstate \to[\rd(\xrd, \channelmessage)] \state_L$ and win.
Since player B employs a winning strategy, we conclude that at least one transition $\channelstate_k \to[\rd(\xrd, \channelmessage)] \hstate$ is enabled, which player B takes.
Player A then has to proceed in $\process^2$ and the play continues there until reaching $\state_{14}$.

We note the following observations, based on the fact that player A cannot force a win.
First, in $\state_7$, the variable $\yvar$ apparently has value $0$.
This means that up to this point, no message from the buffer of $\process^1$ has been updated (other than the single $\tuple{\xwr, \channelmessage}$ from before).
Second, between $\state_8$ and $\state_9$, the value of $\yvar$ is $1$.
We conclude that at least one $\tuple{\yvar, 1}$-message has been updated.
Third, in $\state_{11}$, player A reads $\bot$ from $\xwr$.
This shows that since the transition $\state_6 \to[\mf] \state_7$, no further $\tuple{\xwr, \channelmessage'}$-message has been updated.
In summary, we conclude that during the whole process, exactly the two messages $\tuple{\xwr, \channelmessage}$ and $\tuple{\yvar, 1}$ have been updated to the memory.

We continue with the execution from $\state_{14}$.
Player B empties the buffer of $\process^2$ and takes either $\hstate \to[\rd(\xrd, \bot)] \channelstate'$ in $\process^1$ or $\state_{14} \to[\mf] \state_1$ in $\process^2$.
We assume that player A takes the other one, respectively.

We define $\channelstate_{k+1} := \channelstate'$ and $\word_{k+1}$ by $\word_{k+1} \bullet \channelmessage := \word_k$.
The path $\channelstate_k \to[\rd(\xrd, \channelmessage)] \hstate \to[\rd(\xrd, \bot)] \channelstate_{k+1}$ implies that $\tuple{\channelstate_k, \word_k} \to[?\channelmessage] \tuple{\channelstate_{k+1}, \word_{k+1}}$.
Furthermore, the observations above imply that the game is in configuration $\conf_{k+1}$ as defined in the beginning. This concludes the proof by induction.

Now, we investigate how the run can continue after $\conf_n$.
Player B cannot start to simulate a skip or send operation, since we can assume that player A will move to $\channelstate'$, which contradicts the initial assumption that $\channelstate_n$ is the last channel state visited in this run.
The same holds true for a receive operation:
Following the argumentation from above, we have already seen that player A can either force a win, or lead the game into completing the simulation of the receive operation.
The only possibility left is that $\channelstate_n$ is a final channel state and player B takes the transition $\channelstate_n \to[\wr(\xwr, \top)] \hstate_F$.

In summary, we constructed a path through $\channelsystem$ from $\tuple{\channelstate_0, \varepsilon}$ to $\tuple{\channelstate_n, \word_n}$, where $\channelstate_n \in \channelstateset_F$.
This shows that the set of final states of $\channelsystem$ is reachable.

\subsection{Undecidability}

\begin{thm}
\label{thm:btso}
    The safety problem for the B-TSO game is undecidable.
\end{thm}
\begin{proof}
    This follows directly from \autoref{thm:equivalence-btso-pcs} and the undecidability of the state reachability problem for perfect channel systems \cite{DBLP:journals/jacm/BrandZ83}.
\end{proof}
