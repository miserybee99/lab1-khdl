\section{Conclusion and Future Work}
In this work we have addressed for the first time the game problem for programs running under weak memory models in general and TSO in particular.
Surprisingly, our results show that depending on when the updates take place, the problem can turn out to be undecidable or decidable.
In fact, there is a subtle difference between the decidable (group I, II and IV) and undecidable (group III) TSO games.
For the former games, when a player is taking a turn, the system does not know who was responsible for the last update.
But for the latter games, the last update can be attributed to a specific player.
Another surprising finding is the complexity of the game problem for the groups I, II and IV which is \exptime-complete in contrast with the non-primitive recursive complexity of the reachability problem for programs running under TSO and the undecidability of the repeated reachability problem.

In future work, the games where exactly one player has control over the buffer seem to be the most natural ones to expand on.
In particular, the A-TSO game (where player A can update before and after her move) and the B-TSO game (same, but for player B).
On the other hand, the games of groups I, II and IV seem to be degenerate cases and therefore rather uninteresting.
In particular, we have shown that they are not more powerful than games on programs that follow SC semantics.

Another direction for future work is considering other memory models, such as the partial store ordering semantics, the release-acquire semantics, and the ARM semantics.
It is also interesting to define stochastic games for programs running under TSO as extension of the probabilistic TSO semantics \cite{DBLP:conf/esop/AbdullaAAGK22}.
