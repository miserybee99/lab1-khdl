
% Comments and provisional text
\newcommand{\AC}[1]{\textcolor{blue!80!gray}{\texttt{[AE: #1]}}}
\newcommand{\AT}[1]{\textcolor{blue!60!black}{#1}}

\newcommand{\LC}[1]{\textcolor{red!50!gray}{\texttt{[LE: #1]}}}
\newcommand{\LT}[1]{\textcolor{red!60!black}{\texttt{#1}}}

\newcommand{\CC}[1]{\textcolor{green!50!gray}{\texttt{[CD: #1]}}}
\newcommand{\CT}[1]{\textcolor{green!60!black}{\texttt{#1}}}

\newcommand{\rt}[1]{\textcolor{red}{#1}}

\newcommand{\NeedRef}{\textcolor{red}{(REFERENCES!!!)}}

% Cite and Reference commands
\newcommand{\chapref}[1]{Chap.(\ref{#1})}
\newcommand{\secref}[1]{Sec.(\ref{#1})}
\newcommand{\figref}[1]{Fig.\ref{#1}}
\newcommand{\tabref}[1]{Tab.\ref{#1}}
\newcommand{\appendref}[1]{Appendix (\ref{#1})}


% Short-cut text types commands
% Text
\newcommand{\tit}[1]{\textit{#1}} % Italic Font
\newcommand{\tbf}[1]{\textbf{#1}} % Bold Font
\newcommand{\ttt}[1]{\texttt{#1}}


\newcommand{\GR}{\textit{GR }}
\newcommand{\NR}{\textit{NR}}
\newcommand{\dft}{\tit{DFT }}
\newcommand{\dftop}{\mathcal{F}}
\newcommand{\ie}{\textit{i.e }}
\newcommand{\LCDM}{$\Lambda$CDM }

% Library names
\newcommand{\numpy}{\ttt{NumPy}}

% Mathematica Text
\newcommand{\mcal}[1]{\mathcal{#1}}
\newcommand{\mbf}[1]{\mathbf{#1}}
\newcommand{\mbb}[1]{\mathbb{#1}}
\newcommand{\bsym}[1]{\boldsymbol{#1}}

% Equations Location
\newcommand{\mathleft}{\@fleqntrue\@mathmargin0pt}
\newcommand{\mathcenter}{\@fleqnfalse}

% Mathematical symbols
\newcommand{\ii}{\hat{\imath}}
\newcommand{\defeq}{\coloneq}
\newcommand{\eval}[1]{\biggr\rvert_{#1}}

% New operators
\newcommand*{\dt}[1]{\accentset{\mbox{\large\bfseries .}}{#1}} % over dot bigger
\newcommand{\overo}[1]{\mathring{#1}}
\newcommand{\CD}{{}^{(4)}\nabla} % 4-dim Cov-Deriv
\newcommand{\Cd}{\nabla} % 3-dim Cov-Deriv
\newcommand{\cd}{\mbf{D}} % 2-dim Cov-Deriv
\newcommand{\Ld}[1]{\mcal{L}_{\bsym{#1}}} % 2-dim Cov-Deriv
\newcommand{\pd}[1]{\partial_{#1}} % 2-dim Cov-Deriv

% New mathematical variables
\newcommand{\ko}{\mathring{k}} % Traceless part of the tangential projection of the extrinsic curvature K_{ab}
\newcommand{\normalM}{n}
\newcommand{\normal}{\hat{n}}
\newcommand{\ndot}{\dot{\normal}}
\newcommand{\lapseM}{\alpha} % The lapse of the 3+1 foliation
\newcommand{\shiftM}{\beta} % The shift of the 3+1 foliation
\newcommand{\lapse}{\hat{\alpha}} % The lapse of the 2+1 foliation
\newcommand{\shift}{\hat{\beta}} % The shift of the 2+1 foliation
\newcommand{\JT}{J^{(\perp)}} % Normal part of the current vector
\newcommand{\Jp}{J^{(||)}} % Tangential part of the current vector

% Other Symbols
\newcommand{\cmark}{\textcolor{green!60!black}{\ding{51}}}%
\newcommand{\xmark}{\textcolor{red}{\ding{55}}}%
