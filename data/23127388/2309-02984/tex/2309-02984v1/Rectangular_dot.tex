%% 
%% Copyright 2007-2020 Elsevier Ltd
%% 
%% This file is part of the 'Elsarticle Bundle'.
%% ---------------------------------------------
%% 
%% It may be distributed under the conditions of the LaTeX Project Public
%% License, either version 1.2 of this license or (at your option) any
%% later version.  The latest version of this license is in
%%    http://www.latex-project.org/lppl.txt
%% and version 1.2 or later is part of all distributions of LaTeX
%% version 1999/12/01 or later.
%% 
%% The list of all files belonging to the 'Elsarticle Bundle' is
%% given in the file `manifest.txt'.
%% 

%% Template article for Elsevier's document class `elsarticle'
%% with numbered style bibliographic references
%% SP 2008/03/01
%%
%% 
%%
%% $Id: elsarticle-template-num.tex 190 2020-11-23 11:12:32Z rishi $
%%
%%
%% \documentclass[preprint,12pt]{elsarticle}

%% Use the option review to obtain double line spacing
%% \documentclass[authoryear,preprint,review,12pt]{elsarticle}

%% Use the options 1p,twocolumn; 3p; 3p,twocolumn; 5p; or 5p,twocolumn
%% for a journal layout:
%% \documentclass[final,1p,times]{elsarticle}
%% \documentclass[final,1p,times,twocolumn]{elsarticle}
%% \documentclass[final,3p,times]{elsarticle}
%% \documentclass[final,3p,times,twocolumn]{elsarticle}
%% \documentclass[final,5p,times]{elsarticle}
\documentclass[final,5p,times,twocolumn]{elsarticle}

%% For including figures, graphicx.sty has been loaded in
%% elsarticle.cls. If you prefer to use the old commands
%% please give \usepackage{epsfig}

%% The amssymb package provides various useful mathematical symbols
\usepackage{amssymb}
\usepackage[utf8]{inputenc} % usually not needed (loaded by default)
\usepackage[T1]{fontenc}
\usepackage{siunitx} %numbers and SI units //Added GC
\usepackage{xparse} % //Added GC
\usepackage{physics} %physcis notation //Added GC
\usepackage{graphicx}% Include figure files
\usepackage{natbib,hyperref}
%% The amsthm package provides extended theorem environments
%% \usepackage{amsthm}

%% The lineno packages adds line numbers. Start line numbering with
%% \begin{linenumbers}, end it with \end{linenumbers}. Or switch it on
%% for the whole article with \linenumbers.
%% \usepackage{lineno}

\journal{Journal of Magnetism and Magnetic Materials}

\begin{document}

\begin{frontmatter}

%% Title, authors and addresses

%% use the tnoteref command within \title for footnotes;
%% use the tnotetext command for theassociated footnote;
%% use the fnref command within \author or \address for footnotes;
%% use the fntext command for theassociated footnote;
%% use the corref command within \author for corresponding author footnotes;
%% use the cortext command for theassociated footnote;
%% use the ead command for the email address,
%% and the form \ead[url] for the home page:
%% \title{Title\tnoteref{label1}}
%% \tnotetext[label1]{}
%% \author{Name\corref{cor1}\fnref{label2}}
%% \ead{email address}
%% \ead[url]{home page}
%% \fntext[label2]{}
%% \cortext[cor1]{}
%% \affiliation{organization={},
%%             addressline={},
%%             city={},
%%             postcode={},
%%             state={},
%%             country={}}
%% \fntext[label3]{}

%\title{Shaping spin waves by changing boundary conditions in a square magnetic dot}
\title{Shaping magnetization dynamics in a planar square dot by adjusting its surface anisotropy}

%% use optional labels to link authors explicitly to addresses:
%% \author[label1,label2]{}
%% \affiliation[label1]{organization={},
%%             addressline={},
%%             city={},
%%             postcode={},
%%             state={},
%%             country={}}
%%
%% \affiliation[label2]{organization={},
%%             addressline={},
%%             city={},
%%             postcode={},
%%             state={},
%%             country={}}

\author[inst1]{Grzegorz Centa{\l}a}
\ead{grzcen@amu.edu.pl}
\affiliation[inst1]{organization={Institute of Spintronics and Quantum Information, Faculty of Physics, Adam Mickiewicz University, Pozna{\'n}},%Department and Organization
            addressline={Uniwersytetu~Pozna{\'n}skiego~2}, 
            city={Pozna{\'n}},
            postcode={61-614}, 
            country={Poland}}
            
\author[inst1]{Jaros{\l}aw W. K{\l}os}

\begin{abstract}
%% Text of abstract

A planar square dot is one of the simplest structures confined to three dimensions. Despite its geometrical simplicity, the description of the spin wave modes in this structure is not trivial due to the competition of dipolar and exchange interactions. An additional factor that makes this description challenging are the boundary conditions depend both on non-local dipolar interactions and local surface parameters such as surface anisotropy. In the presented work, we showed how the surface anisotropy applied at the lateral faces of the dot can tune the frequency of fundamental mode in the planar CoFeB dot, magnetized in an out-of-plane direction. Moreover, we analyzed the spin wave profile of the fundamental mode and the corresponding dynamic stray field. We showed that the asymmetric application of surface anisotropy produces an asymmetric profile of dynamic stray field for square dot and can be used to tailor inter-dot coupling. The calculations were performed with the use of the finite-element method.

\end{abstract}

%%Graphical abstract
%\begin{graphicalabstract}
%\includegraphics{fig_toc.pdf}
%\end{graphicalabstract}

%%Research highlights
%\begin{highlights}
%\item Surface anisotropy allows control of magnetization distribution 
%\item Surface anisotropy allows for increased interaction between nanodots
%\item Changing the boundary conditions allows to change the frequency of the spin waves
%\end{highlights}

\begin{keyword}
%% keywords here, in the form: keyword \sep keyword
square dot \sep boundary conditions \sep spin waves \sep finite-element method
%% PACS codes here, in the form: \PACS code \sep code
%% SPRAWDZIĆ!
%\PACS 0000 \sep 1111
%% MSC codes here, in the form: \MSC code \sep code
%% or \MSC[2008] code \sep code (2000 is the default)
%% SPRAWDZIĆ
%\MSC 0000 \sep 1111
\end{keyword}

\end{frontmatter}

%% \linenumbers

%% main text
%% INTRODUCTION
\section{Introduction}
\label{sec:intro}
Flat dots and stripes are magnonic structures with non-trivial geometry, which can be relatively easily produced by patterning the magnetic layer \cite{Barman_2020,Rychly_2022,Gubiotti_2012,di_bona_magnetic_2007,Wang2010}. The magnetization dynamics in planar nanoelements (in flat dots and stripes) is clearly different from that observed in a infinite continuous film \cite{Pirro2021}. The obvious difference is the restriction of the propagation of the spin wave to one direction (in stripes) or its full confinement (in dots). The introduction of these geometrical constraints leads to the quantization of the wave vector in one or two dimensions, which qualitatively modifies the spectrum of spin waves \cite{Kra14_krawczyk2014review,Demokritov_magnonics_2013}. In the case of stripes, we are dealing with a spectrum in the form of a quasi-one-dimensional dispersion relation, where the propagation along the stripes is mediated by a series of mods resulting from the quantization of the wave vector across the stripes \cite{bessonov2015magnonic}. In dots, on the other hand, the dynamical magnetization is completely confined and the spectrum has a form of the set of discrete frequencies \cite{GUBBIOTTI_2007_sd}. In addition to the properties described above, which characterize any wave excitations, spin waves in nanostructures have specific properties. Their dynamics is limited only to the magnetic medium, and the boundary conditions at the surfaces with a non-magnetic medium result not only from the magnetic properties of the surface but are interdependent with the dynamics of magnetization inside the nanoelement \cite{Guslienko_2005}.
%\textcolor{red}{The demagnetization field depends on both volume and surface magnetic charges. The presence of volume charges is due to the inhomogeneity of magnetization in the sample volume while the presence of surface charges is due to the presence of surfaces. In our work, we focused on the modification of surface charges.} 
The parameter describing the freedom of magnetization precession on the surface of the magnetic body is the surface anisotropy \cite{Puszkarski1979} expressed by the surface anisotropy constant $K_{\rm s}$. 
%sprawdzone poniższy tekst
%For a magnetically saturated sample, positive value of uniaxial surface anisotropy constant reduces (increases) the dynamics of the spin wave at the surface when the magnetic field is applied tangentially (perpendicularly) to the surface where the surface anisotropy is applied. 
This local parameter affects boundary conditions for dynamic component  $\mathbf{m}(\mathbf{r},t)=\mathbf{m}(\mathbf{r})e^{i\omega t}$ of magnetization $\mathbf{M}(\textbf{r},t)~=~\mathbf{M}_0(\mathbf{r})~+~\mathbf{m}(\mathbf{r},t)$ which take a general form (for exchange dominated spin waves)\cite{Gurevich1996}:
%Wzory z Gurevicha – Melkova dotyczące wymiennych warunków brzegowych w obecności anizotropii powierzchniowej w SI, wraz o opisem symboli
% Sprawdzone dla naszego przypadku 
\begin{equation}\label{eq:BC}
%    \frac{\partial m_{y}}{\partial n}+\xi m_{y}\cos{2\Theta}=0,~~~~\frac{\partial m_{x}}{\partial n}+\xi m_{x}\cos^{2}{\Theta}=0,
\hat{\mathbf{z}}\times\frac{\partial \mathbf{m}}{\partial \hat{\mathbf{n}}}+\frac{K_{\rm s}}{A}\left(\left(\hat{\mathbf{n}}\cdot\mathbf{m}\right)\hat{\mathbf{n}}\times\hat{\mathbf{z}}+\left(\hat{\mathbf{n}}\cdot\hat{\mathbf{z}}\right)\hat{\mathbf{n}}\times\mathbf{m}\right)=0
\end{equation}
%\begin{equation}\label{eq:BC}
%\mathbf{M}\times\frac{\partial \mathbf{M}}{\partial \hat{\mathbf{n}}}+\frac{K_{\rm s}}{A}\left(\left(\hat{\mathbf{n}}\cdot\mathbf{M}\right)\hat{\mathbf{n}}\times\mathbf{M}\right)=0
%\end{equation}
%\begin{equation}
%    \frac{\partial \mathbf{m}}{\partial \hat{\mathbf{n}}}=0
%\end{equation}
%\begin{equation}
%    K_{\rm s}=0
%\end{equation}
%\begin{equation}
%    K_{\rm s}\ne 0
%\end{equation}
%\begin{equation}
%    \theta=0
%\end{equation}
%\begin{equation}
%    \theta=\pi/2
%\end{equation}
%\begin{equation}
%    \mathbf{M}=\mathbf{M}_0+\mathbf{m}e^{i\omega t}
%\end{equation}
%\begin{equation}
%  \frac{\partial m_{x,y}}{\partial \hat{\mathbf{n}}}+\frac{K_{\rm s}}{A}m_{x,y}=0
%\end{equation}
%\begin{equation}
%  \frac{\partial m_{x}}{\partial \hat{\mathbf{n}}}-\frac{K_{\rm s}}{A}m_{x}=0
%\end{equation}
%\begin{equation}
%  \frac{\partial m_{y}}{\partial \hat{\mathbf{n}}}=0
%\end{equation}
%\begin{equation}
%    -\lambda_x-\lambda_z=1
%\end{equation}
and can be used to modify the amplitude and ellipticity of spin wave precession at the surface.
For saturated sample in out-of-plane direction $\mathbf{M}_0=M_{\rm S}\hat{\mathbf{z}}$, where $M_{\rm S}$ saturation magnetization. The unit vector $\hat{\mathbf{n}}$ is normal to the surface and determines the axis of uniaxial anisotropy. The symbol $A$ denotes the exchange stiffness constant.
%where $\Theta$ is the angle between magnetization direction and normal to surface $\hat{\mathbf{n}}$ ($\Theta~=~0$ - magnetization normal to surface, $\Theta~=~\pi/2$ - magnetization tangential to surface), $n$~-~normal to surface, $\xi$ - pinning parameter equal to $K_{s}/A$, $A$ - exchange stiffness constant.
The sources of magnetic anisotropy (including surface  anisotropy) are attributed to spin-orbit interactions and are related to the deformation of electron shells by the crystal field in the bulk (and at the surface/interface). The anisotropy can be tuned by applying an electric field to the sample or by combining the ferromagnetic material with another material (e.g., heavy metal). Exemplary values of surface anisotropy constant can be found in \cite{GRADMANN_1986,Maruyama_2009}.

Formula (\ref{eq:BC}) does not take into account the influence of dipolar interactions, which, being long-range, are dependent on the size and shape of the nanostructure. Dynamic dipolar interactions result from both the inhomogenity of precession amplitude (volume charges) and the presence of surfaces (surface charges). In the case of planar structures, a reduction in the amplitude of the spin wave (so-called dipolar pinning) is observed at the side faces of the structure, depending on the width-to-thickness ratio. The dipolar pinning thus translates into boundary conditions for the spin waves \cite{Guslienko_2005}.

A peculiarity of planar magnonic structures is that the spin wave spectrum can be shaped by controlling the dipolar pinning at the surfaces of the structure through the selection of structural parameters \cite{centala2019influence, Guslienko_2005}, in addition to modifying the surface parameters (i.e., the surface anisotropy ). It is worth noting that in a system of nanoelements (e.g., the planar arrangement of strip or dot systems) separated by a non-magnetic material, a demagnetizing field ensures the coupling between them \cite{centala2019influence}. In magnetically saturated nanostructures, the dynamic demagnetizing field produced by fundamental mode (where the magnetization precesses in phase in the volume of the sample) significantly depends on the amplitude of magnetization near the surface, and this amplitude is usually reduced due to dipolar pinning, which is an obstacle to the design of dipolarly-coupled magnonic systems and devices of planar geometry \cite{Rana2013,Tacci2010,Keatley2008}.

In our work, we show that dipolar pinning and the associated effects: (i) an increase in the frequency of the fundamental mode and (ii) a reduction in the dynamic demagnetizing field, can be partially compensated by introducing uniaxial surface anisotropy on the lateral faces of the structure. We performed numerical studies for a planar square dot with a strong field applied perpendicularly to its plane. Our goal was also to show that by modifying the surface anisotropy on only one pair of lateral faces, it is possible to selectively reduce dipolar pinning and shape the dynamic demagnetizing field produced by the square dot independently in two perpendicular directions. This will allow the dot to dynamically couple to neighboring nanoelements with different strengths in two geometrically equivalent directions.

The layout of the work is as follows. In the subsequent section, we describe the considered structure and present the computational technique we used. Next, we discuss the results. We show that strong surface anisotropy on the lateral faces of the dot, can increase the precession amplitude at the edge of the dot and significantly reduce the frequency of the fundamental mode. We finish the results with a discussion concerning the spatial distribution and ellipticity of the dynamic demagnetization field produced by the fundamental mode. The paper concludes with a summary.

 
%% END INTRODUCTION

%% MATERIALS AND METHODS
\section{Structure and methods}
\label{sec:MatMed}
%For a magnetically saturated sample, positive value of uniaxial surface anisotropy constant reduces (increases) the dynamics of the spin wave at the surface when the magnetic field is applied tangentially (perpendicularly) to the surface where the surface anisotropy is applied. 
The considered structure is a flat magnetic dot of square shape, made of cobalt iron boron alloy (CoFeB) with uniaxial surface anisotropy applied to the lateral surfaces for which $y$-direction is the normal direction (see inset Fig.~\ref{Fig:disprel_struct}). We consider the structure with the dimensions of 80~nm x 80~nm x 30~nm magnetized in a forward volume configuration. The material parameters used to perform the simulations were: magnetization saturation $\mu_0 M_{s}~=~1571$~mT, exchange stiffness constant $A~=~15$~pJ/m~\cite{Conca_2013}, and gyromagnetic ratio $\gamma~=~187$~rad/T/ns. The external magnetic field used for the simulation $\mu_0 H_{0}~=~2000$~mT is applied in the direction $\hat{\mathbf{z}}$, normal to the plane of the flat dot.

%% Fig 1: STRUCTURE %%
\begin{figure}[!hb]
\includegraphics[width=0.95\linewidth]{fig_disprel_struct.pdf}
\caption{
The frequency dependence on the surface anisotropy constant $K_{\rm s}$. The horizontal dashed line (magenta) indicates the frequency for the CoFeB layer in the forward volume configuration. Circle, square, and rhomb mark the points where the surface anisotropy ($K_{\rm s}$) takes values of 0, 2, and 4.735~mJ/m$^{2}$, respectively, considered further in the manuscript. 
An inset shows the square magnetic dot made of CoFeB with surface anisotropy ($K_{\rm s}$) on lateral sides for which the $y$-direction is normal. On the right of the considered structure, the cone of the magnetization precession
$\textbf{M}$ around the equilibrium position, given by static effective field  $\textbf{H}_{\rm eff}$. Magnetization spins at a circular frequency ($\omega$) and is forced by a torque equal $-\textbf{M}\times\textbf{H}_{eff}$.
}
\label{Fig:disprel_struct}
\end{figure}
%% END FIG. 1. STRUCTURE %%

We obtained the results numerically using the finite element method. For this purpose, we used a semi-classical approach where the Landau-Lifshitz equation (\ref{eq:LLE}) describes the dynamics of the magnetization vector $\textbf{M}(\textbf{r},t)$ \cite{Gurevich1996}.
\begin{equation}\label{eq:LLE}
\frac{d\textbf{M}}{dt}=-\gamma\mu_{0}[ \textbf{M} \times \textbf{H}_{\rm eff} + \frac{\alpha}{M_{S}} \textbf{M} \times (\textbf{M} \times \textbf{H}_{\rm eff})]:
\end{equation}
where the symbol $\mu_{0}$ denotes the vaccum permeability,  $\alpha$ represents the damping constant, and $\textbf{H}_{\rm eff}(\textbf{r},t)$ is the effective magnetic field. In our studies, we neglect the damping constant since it is small $\alpha=0.0042$ ~\cite{Conca_2013}. 

We wanted to investigate the possibility of tuning the frequency of the dot fundamental mode by adjusting the surface anisotropy on its lateral faces. We also aimed to investigate the extreme case in which the fundamental modulus frequency is reduced to the frequency of ferromagnetic resonance in a uniform plane, even if the required values of the surface anisotropy constant are shifted beyond the experimentally verified range.

The effective magnetic field $\textbf{H}_{\rm eff}(\textbf{r},t)$ contains the following components: the external field  $H_{0}\hat{\mathbf{z}}$, exchange field $\textbf{H}_{\rm ex}(\textbf{r},t)= \left( 2A/\mu_{0} M^2_{s} \right)\laplacian\textbf{M}(\textbf{r},t)$, 
%surface uniaxial anisotropy field $\textbf{H}_{\rm S}(\textbf{r},t)$, 
and dipolar field $\textbf{H}_{\rm d}(\textbf{r},t)=-\nabla 
\varphi (\mathbf{r},t)$~\cite{szulc2022}. The magnetostatic potential $\varphi$ is found using the Gauss law for magnetism and magnetostatic approximation \cite{Rychly_2022}.
We model the surface anisotropy by boundary conditions considering only the first component describing the uniaxial surface anisotropy, being in agreement with Eq.~\ref{eq:BC}.
%Therefore, the field $\textbf{H}_{\rm S}(\textbf{r},t)$ can be expressed by the formula \cite{Gurevich1996}:
%wysyłałem kiedyś Panu taką pracę gdzie była dośc ofólna formuła na pole anisotropii
%\begin{equation}\label{eq:SurfAni}
% \textbf{H}_{\rm S}=\frac{2K_{\rm U}}{\mu_{0}M^{2}_{S}}(\textbf{M} \cdot \textbf{n}_{0})\textbf{n}_{0},
%\end{equation}
% where $\textbf{n}_{0}$ is a unit vector normal to the boundary of the ferromagnet and pointed outward magnetic medium. $K_{U}$?
 
% Using the magnetostatic approximation, the dipolar term of the effective magnetic field 
 %The exchange field is defined in standard way for micromagnetism, as a Laplacian acting on magnetization distribution \cite{Krawczyk2012exchange}.

%% END MATERIALS AND METHODS

%% RESULTS
\section{Results}
\label{sec:Results}

For the considered system, we decided to use a surface anisotropy constant exceeding typical experimental values ($K_{s}>1$~mJ/m$^{2}$ \cite{coey_2010}). We took this step to test whether the fundamental spin wave mode in a flat dot can be reduced to the FMR frequency of the unconstrained layer made of the same material. Using the Comsol Multiphysics, we have plotted the dependence of spin wave frequency on the anisotropy constant  (Fig.~\ref{Fig:disprel_struct}). The data indicate that such a reduction of frequency is theoretically possible. However, the required value of $K_{s}$  significantly exceeds the values known from experimental studies.

The relationship shown in Fig.~\ref{Fig:disprel_struct} cannot be considered in separation from the profiles shown in Fig.~\ref{Fig:mprofiles_struct}. For exchange dominated system, the absence of surface anisotropy $K_{\rm s}=0$ makes the magnetization free at the surface and the boundary  (\ref{eq:BC}) reduced to the form $\partial \textbf{m}/\partial \hat{\mathbf{n}}=0$. However, due to the presence of dipolar pinning \cite{Guslienko_2005}, the reduction of magnetization dynamics is observed at the lateral faces of flat dot and the spin waves profile is not uniform anymore -- see the top profile on Fig.~\ref{Fig:mprofiles_struct}.

%% Fig 2: STRUCTURE %%
\begin{figure}[!ht]
\includegraphics[width=0.95\linewidth]{fig_mprofiles_struct.pdf}
\caption{
Amplitude profiles of the dynamic components of magnetization ($|m_{\alpha}|$, where $\alpha=x,y$) obtained by cutting the structure along the dotted line (see inset). Surface anisotropy was applied to the side walls of the structure for which the $y$-direction is normal. Profiles were obtained for surface anisotropy constant $K_{\rm s}= 0, 2, 4.735~{\rm mJ/m}^{2}$ -- see top, middle, and bottom plots, respectively. 
}
\label{Fig:mprofiles_struct}
\end{figure}
%% END FIG. 2. STRUCTURE %%

By increasing the surface anisotropy constant, we release the normal component of dynamic magnetization at the edges to counteract dipolar pinning. The increase of the normal component of magnetization enhances the strength of the dynamic demagnetizing field, due to the increase of amplitude of surface charges \cite{centala2019influence}.
%Due to the effect through the field expressed by the formula Eq.~\ref{eq:SurfAni}, we release the dynamic component of magnetization perpendicular ($m_{y}$) to the lateral boundary of square dot modified by the surface anisotropy. 
The tangential component of dynamic magnetization is also changed by the coupling resulting from the Landau-Lifshitz equation (see middle and bottom profiles on Fig.~\ref{Fig:mprofiles_struct}).

It is worth noting that the most rapid frequency drop in Fig.~\ref{Fig:disprel_struct} occurs for $K_{s}>3$~mJ/m$^{2}$. In this range,  the flatness of profile of the normal component of magnetization $m_y$ is not improving anymore. The limit of the FMR frequency of the film (dashed line in Fig.~\ref{Fig:disprel_struct}) is reached for highly non-uniform profile $m_y(y)$ with the amplitude of precession significantly uplifted at the edges of the dot -- see the bottom plot in Fig.~\ref{Fig:mprofiles_struct}).

% This is due to the dominance of mods concentrated near the sides of the square dot. When the lowest mod frequency in the square dot is equal to the FMR frequency of the CoFeB layer, the concentration of the amplitude at the center of the dot is not visible and thus the effect associated with dipolar pinning is canceled.

For $K_{\rm s}=0$, a circular ellipticity (see Fig.~\ref{Fig:hmprofiles_struct}) at the center of the square dot is evident for both the dynamic components of magnetization \textbf{m} and the demagnetization field \textbf{h}$_{\rm d}$. However, a discrepancy in ellipticity is apparent at the sides of the dot due to the demagnetizing effects related to the presence of the magnetic surface charges. The ellipse of the dynamic component of the demagnetization field is aligned with the long axis parallel to the nearest walls while the long axis of the magnetization ellipse is aligned perpendicular to the nearest walls.

%% Fig 3: STRUCTURE %%
\begin{figure}[!ht]
\includegraphics[width=0.95\linewidth]{fig_hm_elip.pdf}
\caption{
The ellipticity change associated with the modification of surface anisotropy $K_{\rm s}$ demonstrated for the dynamic components of magnetization $\textbf{m}$ and the dynamic magnetic field derived from dipolar interactions $\textbf{h}_{d}$. The results are presented for the values of surface anisotropy equal to 0, 2, and 4.735~mJ/m$^{2}$ for the upper, middle, and lower rows, respectively. The side walls of the structure on which the boundary conditions are affected by surface anisotropy $K_{\rm s}>0$ are marked with a bold blue line. The areas where the procession is weakly visible are magnified 10 times and presented in black circles.
}
\label{Fig:hmprofiles_struct}
\end{figure}
%% END FIG. 3. STRUCTURE %%

For a surface anisotropy constant $K_{s}=2$~mJ/m$^{2}$, the ellipticity at the center of the dot for \textbf{m} and \textbf{h}$_{\rm d}$ remains almost circular while it changes near the edges of the dot. Near these edges where surface anisotropy has been applied, the long axis of the elliptical precession for \textbf{m} is elongated with respect to the $K_{s}=0$ state. At faces without surface anisotropy, this axis is shortened. When considering \textbf{h}$_{\rm d}$, please note that the direction of the long axis of ellipticity at the edges of the dot is oriented in parallel to the $y$-direction. 

For the case when $K_{\rm s}=4.735$~mJ/m$^{2}$ the long axes of both \textbf{m} and \textbf{h}$_{d}$ are directed parallel to the $y$ axis. Moreover, the ellipticity for \textbf{m} as well as \textbf{h}$_{d}$ at the edge of the dot where the surface anisotropy is applied becomes extremely large demonstrating the significant increase of the $m_{y}$ and $h_{{\rm d},y}$ components.

%%% Fig 4: STRUCTURE %%
\begin{figure}[!ht]
\includegraphics[width=0.95\linewidth]{fig_hprofiles_struct.pdf}
\caption{
Profiles of the dynamic magnetic field component from dipolar interactions ($h_{{\rm d},\alpha}$, where $\alpha=x,y$) along $\alpha$-direction (see inset). The colors of dotted lines in the inset correspond to the color of $h_{{\rm d},\alpha}$ on the profiles. Profiles were made for the same value of surface anisotropy ($K_{\rm s}=2$~mJ/m$^{2}$) applied to the lateral surfaces perpendicular to $y$-direction. The lines on the green background correspond to the component of the dynamic magnetic field $h_{{\rm d},\alpha}$ inside the dot. The lines on the white background visualize the stray field. Bold blue lines represent surface anisotropy.
}
\label{Fig:hprofiles_struct}
\end{figure}
%% END FIG. 4. STRUCTURE %%

%\textcolor{red}{The demagnetization field depends on both volume and surface magnetic charges. The presence of volume charges is due to the inhomogeneity of magnetization in the sample volume while the presence of surface charges is due to the presence of surfaces. In our work, we focused on the modification of surface charges.} 

The dynamic components of magnetization and the dynamic components of the demagnetization field (inside the dot) increase the stray field (outside the dot) -- see Fig.~\ref{Fig:hprofiles_struct}. We can see the directional increase the stray field: $|h_{{\rm d},y}|>|h_{{\rm d},x}|$. As a result, such a dot, when placed in the square lattice, will interact differently in $x-$ and $y-$direction with its neighbors. In such a system, we should observe the difference in spin wave propagation in two principal directions $x$ and $y$, even if its geometry does not exhibit such asymmetry. 

%% END RESULTS

%% CONCLUSIONS
\section{Conclusions}
\label{sec:Conclusions}
The magnetization in a confined structure is shaped by the interplay of dipolar and exchange interactions. However, for the structure of planar geometry, the surface anisotropy applied on lateral faces is an additional important factor which forms the profiles of confined modes, due to its impact on boundary conditions for spin waves. Moreover, it allows to shape the external dynamic magnetic field and thus the interaction with other dots of such kind.

%% END CONCLUSIONS

% Acknowledgements
\section{Acknowledgements}  
\label{sec:Acknowledgements}
This work has received funding from National Science Centre Poland grants  UMO-2020/39/O/ST5/02110,\break UMO-2021/43/I/ST3/00550 and support from the Polish National Agency for Academic Exchange grant BPN/PRE/2022/1/00014/U/00001.
%% END Acknowledgements






%% Fig 5: STRUCTURE %%
%\begin{figure}
%\includegraphics[width=0.9\linewidth]{fig_hmelip.pdf}
%\caption{
%Ellipticity of the normal components of the demagnetization field inside the square dot for the surface anisotropy constant Ks = 0 mJ/m$^{2}$, Ks = 3 mJ/m$^{2}$ and Ks = 4.3 mJ/m$^{2}$. 
%}
%\label{Fig:helip}
%\end{figure}
%% END FIG. 5. STRUCTURE %%

%Lorem ipsum dolor sit amet, consectetur adipiscing elit, sed do eiusmod tempor incididunt ut labore et dolore magna aliqua. Ut enim ad minim veniam, quis nostrud exercitation ullamco laboris nisi ut aliquip ex ea commodo consequat. Duis aute irure dolor in reprehenderit in voluptate velit esse cillum dolore eu fugiat nulla pariatur. Excepteur sint occaecat cupidatat non proident, sunt in culpa qui officia deserunt mollit anim id est laborum see appendix~\ref{sec:sample:appendix}.

%% The Appendices part is started with the command \appendix;
%% appendix sections are then done as normal sections
%\appendix

%\section{Sample Appendix Section}
%\label{sec:sample:appendix}
%Lorem ipsum dolor sit amet, consectetur adipiscing elit, sed do eiusmod tempor section \ref{sec:intro} incididunt ut labore et dolore magna aliqua. Ut enim ad minim veniam, quis nostrud exercitation ullamco laboris nisi ut aliquip ex ea commodo consequat. Duis aute irure dolor in reprehenderit in voluptate velit esse cillum dolore eu fugiat nulla pariatur. Excepteur sint occaecat cupidatat non proident, sunt in culpa qui officia deserunt mollit anim id est laborum.

%% If you have bibdatabase file and want bibtex to generate the
%% bibitems, please use
%%
%\vfill\null
\bibliographystyle{elsarticle-num} 
\bibliography{bibs.bib}



%% else use the following coding to input the bibitems directly in the
%% TeX file.

% \begin{thebibliography}{00}

% %% \bibitem{label}
% %% Text of bibliographic item

% \bibitem{}

% \end{thebibliography}
\end{document}
\endinput
%%
%% End of file `elsarticle-template-num.tex'.
