Navigating in unknown 3D environments is a crucial task for mobile robots.
Before being able to move, a robot must represent the environment in which it evolves.
A well-known and efficient way to map the environment is the occupancy grids introduced by \citet{elfes1989using}.
An occupancy grid provides the robot with information about the potential presence of an obstacle at a given position.
However, this information alone does not encompass all the capabilities of the robot to maneuver within its environment. 
For example, a wheeled robot can safely cross grasses, a curb, or a speed bump at low speed, as shown in \autoref{fig:intro}.
Therefore, a way to assess their hazardous nature is necessary to evolve in these environments.
Motivated by this fact, numerous recent works \cite{shi2023gridcentric} have contributed to risk-aware navigation.
%
However, \citet{laconte2019lambda} demonstrated that the Bayesian occupancy grid, which stores the probability of collision of a given position, is ill-suited to compute the risk over paths.
Indeed, the probability of collision, computed as the joint probability that every cell is free of obstacles, is highly dependent on the grid tessellation size.
To overcome this difficulty, they developed a novel framework called Lambda-Field to assess physics-based risks on occupancy grids.
\begin{figure}[t]
  \centering
  \includegraphics[width=\linewidth]{media/IMG_0217_rt.jpg}
  \caption{Example of a situation that a vehicle might encounter while navigating.
    Speed bumps are frequent obstacles an intelligent vehicle has to safely overcome.
    With our framework, the vehicle is able to make more effective decisions based on the level of risk it is allowed to take.
}
  \label{fig:intro}
  \vskip-.5em
\end{figure}
Moreover, in occupancy grids framework, most navigation methods \cite{PATLE2019582} use geometric and semantic reasoning to handle the obstacles present in the environment, such as curbs, traffic cones, speed bump, sidewalk, buildings, traffic circle and traffic lights shown in \autoref{fig:intro}.
\citet{laconte2019lambda} showed that path planning becomes more intuitive and meaningful when a physical risk metric is used.
However, their work has been developed solely for 2D environments.
%
Here, we adapt their work to 3D environments and use physical, risk-based reasoning to deal with the obstacles represented in \autoref{fig:intro}.

The main contributions of this paper are i) an extension of the Lambda-Field \cite{laconte2019lambda} to 3D environments; ii) a generalization of the risk that take into account traversable obstacles; iii) a mathematical formulation of a local risk-aware path planning algorithm in the Lambda-Field; iv) a demonstration of the applicability of the method, using data acquired in real urban environment.

%
%
%
%
%
%
