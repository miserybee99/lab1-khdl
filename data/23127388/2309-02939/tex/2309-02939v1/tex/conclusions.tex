In this paper, we presented a method of risk-aware navigation in unknown 3D environments.
We propose a generalization of the work of \citet{laconte2019lambda}, taking into account the traversal of small obstacles such as road curbs or speed bumps.
We showed how to compute the associated Lambda-Field, using 3D lidar measurements.
The resulting map is used to evaluate the expected maximum potential energy that the robot's wheels will absorb in the event of a collision along a given path.
Finally, we presented a path planning formulation that take the risk into account as a hard constraint.
Using our formulation, we showed that the risk function is well-fitted for risk-aware navigation in urban environment.
While being able to mimic a standard path planning approach by setting the risk threshold to zero, our framework can also generate a path going over obstacles if the risk is tolerable.
This work focused on explaining the theoretical framework and its applicability.

Future works will focus on improving the path planning part and demonstrating the pertinence of the framework in larger-scale experiments.
%
The exploration of alternative risk metrics, such as \ac{CVaR}, will be pursued to account for tail events.
We intend to extend our framework to enable the robot to perform long-term missions.
Indeed, providing mobile robots with such capabilities will augment the autonomy of intelligent vehicles in rural environments.
A global path planner based on OpenStreetMap \cite{open_street_map} will be included to the framework.
Furthermore, we will conduct extensive experiments on the framework, incorporating quantitative evaluations.
Finally, we will investigate the possibility of adding several risks to further constrain the path-planning algorithm, such as the risk of crossing a continuous lane marking.