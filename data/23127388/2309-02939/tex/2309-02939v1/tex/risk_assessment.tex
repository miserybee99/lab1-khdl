In order to navigate in the Lambda-Field, we define a risk function that reflects the potential obstacles that the robot can face. We choose to describe the risk by the maximum potential energy absorbed by the wheels.

We illustrate in \autoref{fig:elevation_risk} the modeling of the wheel.
The wheel is modeled as a deformable disk, see \autoref{fig:elevation_risk}.
When the wheel crosses an elevated obstacle, it deforms at the point where the two objects collide.
This deformation is approximated with the deformation of a spring of stiffness $k_r$.
\begin{figure}[!htbp]
  \centering
  \includegraphics[width=\linewidth]{media/tikz_risk.pdf}
  \caption{Modeling of the wheel of radius $R$ in collision with the curb (in blue) of height $H$, with a speed $v$ and angle $\Psi$.
  The deformation of the wheel due to the collision is approximated with the deformation of a spring of stiffness $k_r$.}
  \label{fig:elevation_risk}
  \vskip-.5em
\end{figure}
In order to find the maximum amplitude $l_{m}$ of the tire compression due to the collision, we solve the following differential equation:
\begin{equation}
    \label{eq:eq_diff_spring}
    \ddot{l} + \omega^2l = 0 \quad \text{with } \omega = \sqrt{\frac{k_r}{m}}
\end{equation}
where $l$ is the compression at the contact point and $m$ is the mass of the vehicle.
As such, the maximum amplitude $l_{m}$ of the tire compression is
\begin{equation}
    \label{eq:max_defflection_spring}
    l_{m} = \frac{v\cos(\Psi)}{\omega}
\end{equation}
where $v$ is the linear velocity of the robot and ${\Psi=\arcsin(\frac{R-\min(H,R)}{R})}$ is the angle of attack of the collision.
The angle $\Psi$ depends on the elevation of the obstacle and the radius $R$ of the wheel.
To be conservative, we take the maximum difference in elevation $H$ of the obstacles that lie in the transverse axis of the robot path $\mathcal{P}$.
Finally, the risk function is defined as the maximum potential energy that will be absorbed in the wheels at the time of collision and is computed by
\begin{equation}
    \label{eq:risk_function}
    r(X) = \dfrac{1}{2} k_r  l_{m}^2
\end{equation}
As such, the risk function $r(X)$ has a clear physical meaning and is expressed in Joule.
%
%
%
In the next section, we use this risk assessment to construct safe paths in 3D environments.