\NewDocumentCommand{\uut}{}{u^p \left(u^p\right)^\transpose}
\NewDocumentCommand{\utau}{O{u}}{\left(#1^p\right)^\transpose A^p #1^p}
\NewDocumentCommand{\xtax}{}{\utau[x]}

\section{Proof of \cref{theorem:spectral_of_accumulated}}

\NewDocumentCommand{\sti}{}{Stieltjes}
\RenewDocumentCommand{\uut}{O{k}}{u_{#1} u_{#1}^\transpose}
\NewDocumentCommand{\xxt}{O{k}}{x_{#1} x_{#1}^\transpose}

More notations and preliminary information are used in the proof. They are listed in \cref{appendix:more_preliminary}.

To have a clear goal, we first point out how the ratios in \cref{theorem:spectral_of_accumulated} emerge. 
In random matrix theory (RMT), a powerful tool is \sti{} transform when it comes to spectral distribution. 
\sti{} transform of a distribution $\mu$ is 
\begin{align}
    s^\mu(z) \defeq \int_{\reals} \frac{1}{\lambda - z} \mu(\dd \lambda), \forall z \in \positivecomplex,
\end{align}
where $\positivecomplex \defeq \set{u + v i: u, v \in \reals, v > 0} \subset \complexes$.
It can be seen as the expected value of inverses. So if the \sti{} transform of the spectral distribution' bound is computed and multiplied with the average eigenvalue that can be computed by trace as the sum of eigenvalues, then the expected fraction is bounded. \cref{lemma:sti_and_eigenvalue_ratio} fulfills this intuition.

\NewDocumentCommand{\approxquadratic}{m}{\sqrt{\frac{2}{v} \left(c + #1\right)}}
\begin{lemma}[\sti{} transform and eigenvalue ratio]\label{lemma:sti_and_eigenvalue_ratio}
    For any density $\mu$ over real numbers that is only supported on positive numbers, there is
        \begin{align}
            \rpart{s^{\mu}(v i)}
            =&  \int \frac{1}{\lambda + \frac{v^2}{\lambda}} \mu(\dd \lambda) = \ex{\frac{1}{\lambda + \frac{v^2}{\lambda}}},\\
            \ipart{s^{\mu}(v i)}
            =&  \int \frac{1}{\left(\frac{\lambda}{\sqrt{v}}\right)^2  + v} \mu(\dd \lambda) = \ex{\frac{1}{\left(\frac{\lambda}{\sqrt{v}}\right)^2 + v}},
        \end{align}
    where $v \in \reals^+$.
    As a result, if upperbound or mean of $\lambda$ is obtained, one can estimate the averaged ratio between eigenvalues.
\end{lemma}
\begin{proof}
    By definition,
    \begin{align}
        s^{\mu}(z)
        \defeq& \int \frac{1}{\lambda - z} \mu(\dd \lambda)
        =  \int \frac{\lambda - \bar{z}}{\left(\lambda - z\right)\left(\lambda - \bar{z}\right)} \mu(\dd \lambda)
        =  \int \frac{\lambda - u + v i}{\lambda^2 - 2 u \lambda + u^2 + v^2} \mu(\dd \lambda), \\
        \rpart{s^{\mu}(z)}
        =&  \int \frac{\lambda - u}{\lambda^2 - 2 u \lambda + u^2 + v^2} \mu(\dd \lambda)
        =   \int \frac{1}{\lambda - u + \frac{v^2}{\lambda - u}} \mu(\dd \lambda), \\
        \ipart{s^{\mu}(z)}
        =&  \int \frac{v}{(\lambda - u)^2 + v^2} \mu(\dd \lambda).
    \end{align}
    By setting $z = 0 + v i$, there is
    \begin{align}
        \rpart{s^{\mu}(z)}
        =&  \int \frac{1}{\lambda + \frac{v^2}{\lambda}} \mu(\dd \lambda),
        \ipart{s^{\mu}(z)}
        =   \int \frac{1}{\left(\frac{\lambda}{\sqrt{v}}\right)^2  + v} \mu(\dd \lambda).
    \end{align}
\end{proof}

We adapt the proof of Theorem 2.1 from \citet{mp_quadratic_form}, where a quadratic equation $\frac{\ex{S_p}}{1 + \ex{S_p}} - z \ex{S_p} = c + o(1) = \frac{p}{b T} + o(1)$ bridges the \sti{} transform of the spectral distribution and value $c$. We follow a similar path, but in a non-asymptotic and dependent scenario where it is hard to obtain an $o(1)$ residual. So we turn to a generalized form where the residual term is bounded. \cref{lemma:bound_of_sti} explores how we can bound $\ex{S_p}$, which is scaled \sti{} transform of the empirical spectral distribution, by $c$ and the magnitude of the residual term.

\begin{lemma}[Bounds on continuous roots of a equation]\label{lemma:bound_of_sti}
    Suppose $S$ is a function of $z, p, b, T$ such that
    \begin{align}
        \frac{S}{1 + S} - z S = c + t,
    \end{align}
    where $z = 0 + v i \in \positivecomplex (v \in \reals^+), c = p / b T \in [0, 1]$, and $t \in \complexes$ is also a function of $z, p, b, T$.
    Assume $S$ is continuous w.r.t. $z$ and further assume $S$ always has a non-negative real part.
    Assume $\abs{t}$ is upperbounded by function $\tau(p, b, T) / v$ where $\tau$ is constant w.r.t. $v$.
    If for some $v_0 \ge 2 c$, there is 
    \begin{align}
        \frac{v_0 + (1 - c)}{\sqrt{2}} > \frac{\tau}{v_0} + 2 \sqrt{c v_0} + 2 \sqrt{\tau}
    \end{align}
    then for all $v \ge v_0$, there is
    \begin{align}
        \abs{\ipart{S}}, \abs{\rpart{S}} \le \abs{S} \le&  \sqrt{\frac{2}{v} \abs{c + t}}.
    \end{align}
\end{lemma}
\begin{proof}
    Assume $p, b, T$ is fixed and $v \ge v_0$. 
        
    First of all, since $t = \frac{S}{1 + S} - z S - c$, where $S$ is continuous w.r.t. $z$, when well defined $t$ is also continuous w.r.t. $z$. Since $S$ is well defined for all $z = v i$ and $S$ has a non-negative real part, meaning $1 + S \neq 0$, $t$ is defined and continuous for all $z = v i$.

    $\frac{S}{1 + S} - z S = c + t$ implies
    \begin{align}
        S^2 + \frac{z - 1 + c + t}{z} S + \frac{c + t}{z} = (S + b)^2 - b^2 + \frac{c + t}{z} = 0,
    \end{align}
    where $b = \frac{z - 1 + c + t}{2 z}$.
    Consider a modified version of this quadratic equation by altering the zeroth order term
    \begin{align}
        \left(R + b\right)^2 - b^2 + \frac{c + t}{z} u = 0. \label{eq:modified_quadratic}
    \end{align}
    where $R = R(z, p, T, u)$ is a solution of the modified equation. Since $z, t, u$  depend on $z, u$ in a continuous manner and there is no singular point given $\rpart{z} > 0$, there exist two continuous solutions $R_1$ and $R_2$ to the equation that is well defined for all $z=v i$ and $u$ where $v \in \reals^+, u \in [0, 1]$. Conversely, all possible solutions $R$ are contained in the union of manifolds given by $R_1$ and $R_2$.
    Since
    \begin{align}
        &\forall u \in [0,1], - b^2 + \frac{c + t}{z} u \neq 0
        \impliedby  \abs{b^2} > \abs{\frac{c + t}{z}}\\
        \impliedby& \abs{v i - 1 + c + t} > 2 \sqrt{v \abs{c + t}}
        \impliedby  \sqrt{v^2 + (1 - c)^2} - \abs{t} > 2 \sqrt{v} \sqrt{c + \abs{t}}\\
        \impliedby& \sqrt{v^2 + (1 - c)^2} > \abs{t} + 2 \sqrt{v} \sqrt{c + \abs{t}}
        \impliedby  \sqrt{v^2 + (1 - c)^2} > \frac{\tau}{v} + 2 \sqrt{c v + \tau}\\
        \impliedby& \frac{v + (1 - c)}{\sqrt{2}} > \frac{\tau}{v} + 2 \sqrt{c v} + 2 \sqrt{\tau} \quad\quad\left(\text{LHS: concavity of $\sqrt{x}$}\right)\\
        \impliedby& \frac{v_0 + (1-c)}{\sqrt{2}} > \frac{\tau}{v_0} + 2 \sqrt{c v_0} + 2 \sqrt{\tau} \\&    \land \forall v \ge v_0,
            \frac{\dd \left(\frac{v + (1 - c)}{\sqrt{2}} - \left(\frac{\tau}{v} + 2 \sqrt{c v} + 2 \sqrt{\tau}\right)\right)}{\dd v} = \frac{1}{\sqrt{2}} + \frac{\tau}{v^2} - \frac{\sqrt{c}}{\sqrt{v}} \ge \frac{1}{\sqrt{2}} - \frac{\sqrt{c}}{\sqrt{v}} \ge 0\\
        \impliedby& \frac{v_0 + (1 - c)}{\sqrt{2}} > \frac{\tau}{v_0} + 2 \sqrt{c v_0} + 2 \sqrt{\tau} \land v_0 \ge 2 c,
    \end{align}
    $R_1$ and $R_2$ has no intersection when $v \ge v_0$. As a result, continuous $S$, as $R(u=1)$ for some $R$, can only be found in exactly one of $R_1$ and $R_2$ and thus is either $R_1(u=1)$ or $R_2(u=1)$.

    Consider $R$s' behaviours when $u=0$. $R_1$ and $R_2$ are either $0$ or $-2b$. By their being continuous and disjoint, $R_1$ and $R_2$ can only be universally one of them otherwise there would be discontinuity. Without loss of generality, let
    \begin{align}
        R_1(z, p, T, 0) =&  0,\\
        R_2(z, p, T, 0) =&  -2 b =  \frac{1 - z - c - t}{z} = \frac{-i - v + c i + t i}{v}.
    \end{align}
        
    Assume $S = R_2(u=1)$ for contradiction. If so, $\lim_{v \to \infty} \rpart{R_2(u=1)} \ge 0$. Consider the following trajectory of $(z, u)$ to conclude that $\lim_{v \to \infty} \rpart{R_2(u=1)} \le -0.5$:
    \begin{enumerate}
        \item Start from $z=i, u=0$, where $R_2 = -\frac{i}{v} - 1 + \frac{c + t}{v} i$.
        \item Increase $v$ alone until $v \ge 100 (c + \tau(z, b, p, T) / v)$. After this step, term $\frac{c + t}{v} i$ will not flip the sign of $-1$ given that $\abs{\frac{c + t}{v} i} \le \frac{c + \tau / v}{v} \le \frac{1}{100}$. Therefore, after this step $\rpart{R_2} \le -0.99$.
        \item Increase $u$ alone to $1$. During this step, the term $\frac{c + t}{z} u$ in \cref{eq:modified_quadratic}, which appears under squared root in $R_2$, changes $R_2$ at most by $\sqrt{\abs{\frac{c + t}{z}}} \le \sqrt{\frac{c + \tau}{v}} \le \frac{1}{10}$. Therefore, it will not flip the sign and $\rpart{R_2} \le -0.89$. 
        \item Increase $v$ to infinity. The influences of $\frac{c + t}{v}$ and $\frac{c + t}{v} u$ will further drop and become ignorable. So $R_2 \le -0.5$ in later parts of the trajectory.
    \end{enumerate}
    Note since $\tau/v$ monotonically decreases as $v$ increases, step (3) will happen when $v < \infty$ and the limit following this trajectory is equal to $\lim_{v \to \infty} R_2(z)$.
    Therefore, $\lim_{v \to \infty} R_2(z) \le -0.5$, $S \neq R_2$ and $S = R_1(u=1)$. As a result, $S$ can be approximated from $R_1=0$ at $u=0$.

    \NewDocumentCommand{\derivativenorm}{}{\frac{\abs{c + t}}{2 \sqrt{\abs{v^2 b^2 - v(c + t) u}}}}
    To this end, taking differentiation (the derivative w.r.t. $u$ exists except for a finite number of points given the elementary dependence on it) gives
    \begin{align}
        2(R(u) + b) \dd R + \frac{c + t}{z} \dd u = 0
    \end{align}
    or
    \begin{align}
        \abs{\frac{\dd R}{\dd u}} = \abs{-\frac{c + t}{2 z (R(u) + b)}} = \frac{\abs{c + t}}{2 v \sqrt{\abs{b^2 - \frac{c + t}{z} u}}} \le \derivativenorm.
    \end{align}
    Starting from $R(u=0)=0$, there is
    \begin{align}
        &\abs{S - R(u=0)}
        =  \abs{\int_{0}^1 \frac{\dd S}{\dd u} \dd u}\\
        \le&    \int_{0}^1 \derivativenorm \dd u
        \le     \frac{1}{2 v}\int_{0}^1 \frac{v \abs{c + t}}{\sqrt{\abs{v^2 \abs{b^2} - v \abs{c + t} u}}} \dd u\\
        =&      \frac{1}{2 v}\int_{v^2 \abs{b}^2 - v \abs{c + t}}^{v^2 \abs{b}^2} \frac{1}{\sqrt{\abs{w}}} \dd w
        \le     \frac{1}{2 v}\int_{- v \abs{c + t}/2}^{v \abs{c + t} /2} \frac{1}{\sqrt{\abs{w}}} \dd w
        =       \sqrt{\frac{2}{v} \abs{c + t}},
    \end{align}
    where the second inequality follows the triangle inequality for subtracting edges. The third inequality is done by moving $[v^2 \abs{b}^2 - v \abs{c + t}, v^2 \abs{b^2}]$ towards zero until it is symmetric, during which the end with the largest absolute value is moved to the other end and its absolute value is decreased, enlarging the inverted square root.
\end{proof}

As a result, if the residual term is bounded, so are the \sti{} transform and the expected fraction. In the proof of \citet{mp_quadratic_form} this quadratic equation is extracted by convergence, which is hard in non-asymptotic analysis. We extract it with approximation, whose error is bounded by vector norm bound $\alpha / p$, anisotropy bound $\beta / p$. 

Some matrices are also involved in the approximation, where a technical lemma by \citet{mp_quadratic_form} is very useful. \cref{lemma:3.1_from_mp_quadratic_form} states that a special matrix, which we will encounter many times in the following proof, has a bounded spectral norm. Combined with Hölder's inequality, bounds on matrices' spectral norms and vector norms ($\alpha/p$) will suppress the approximation error.

\begin{lemma}[Lemma 3.1 by \citet{mp_quadratic_form}]\label{lemma:3.1_from_mp_quadratic_form}
    Let $C \in \reals^{p \times p}$ be a real symmetric positive semi-definite matrix and $x \in \reals^p$. If $z \in \complexes$ is such that $v = \ipart{z} > 0$, then
    \begin{enumerate}
        \item\label{lemma:spectral_bound_of_reverse} 
            $\norm{\left(C - z I\right)^{-1}} \le 1 / v$ ;
        \item\label{lemma:x_means_little_in_reversed_sample_covariance}
            $\abs{\trace{C + x x^\transpose - z I}^{-1} - \trace{C - z I}^{-1}} \le 1 / v$;
        \item $\abs{x^\transpose\left(C + x x^\transpose - z I\right)^{-1} x} \le 1 + \abs{z} / v$;
        \item $\ipart{z + z \trace{C - z I}^{-1}} \ge v$ and $\ipart{\trace{C - z I}^{-1}} > 0$;
        \item $\ipart{z + z x^\transpose (C - z I)^{-1} x} \ge v$.
    \end{enumerate}
\end{lemma}

With these three lemmas, we can begin the proof of \cref{theorem:spectral_of_accumulated}. Following \citet{mp_quadratic_form}, we relate \sti{} transform of the spectral distribution with the quadratic equation. The quadratic equation is formed by approximation, whose error becomes $t$ in \cref{lemma:bound_of_sti}. After concluding the bounds of the \sti{} transform, applying \cref{lemma:sti_and_eigenvalue_ratio} gives the final conclusion.

\def\StateSpectralOfAccumulated{proof}
\ifdefstring{\StateSpectralOfAccumulated}{display}{

\begin{restatable}[Spectral concentration of accumulated steps]{theorem}{SpectralOfAccumulated}
    \label{theorem:spectral_of_accumulated}
    Let $X^p = X^{p, b T} \in \reals^{p \times b T}$ be a random matrix that forms a Batch Dependence Model as in \cref{def:batch_model} with batch size $b$ and step count $T$, whose columns are $x^p_j = X^{p, b T}_{\cdot, j} \in \reals^{p}$. 
    Columns in $X^p = X^{p, b T} \in \reals^{p \times b T}$ are \emph{not} necessarily independent.
        
    Let $x_k^p \defeq X^{p}_{\cdot, k}$ be the $k$-th column of $X^p$ and $x_{t, l}^p \defeq X^{t}_{\cdot, l}$ be the $l$-th column of batch $t$ or equivalently the $k=\left((t-1)*b + l\right)$-th column $x_k$ in $X^{p}$. Superscription $p$ may be dropped for convenience.

    Let $S^{p} \defeq \frac{1}{b T} X^p \left(X^p\right)^\transpose = \frac{1}{b T} \sum_{k=1}^{b T} x_k x_k^\transpose$ be the empirical covariance matrix of all random vectors, and $I_p$ be the compatible identity matrix.
    Assume $x_{t, l}$s' norm is bounded, say by $1$, and scale it with 
    \begin{align}
        u^p_{t, l} \defeq \sqrt{\alpha} x^p_{t, l},
    \end{align}
    obtaining $U^p = U^{p, b} \defeq \begin{bmatrix} U^1 & \cdots &  U^t & \cdots  & U^T \end{bmatrix} = \sqrt{a} \cdot X^p$, where $a \defeq \frac{\trace{S^p}}{\trace{S^p S^p}}$. Let 
    \begin{align}
        T^{p} \defeq \frac{1}{b T} U^p \left(U^p\right)^\transpose = \frac{1}{b T} \sum_{k=1}^{b T} u_{k} u_{k}^\transpose = a \cdot S^{p}
    \end{align}
    be the empirical covariance matrix of all $u^p_{k}$s. 

    Assume $a$ is bounded by $\alpha(p)$ and $\ex{\sqrt{p \trace{\left(T^{p} - I^p\right)\left(T^{t} - I^p\right)}}}$ is also upperbounded by $\beta(p)$.

    Further assume that the following function of $z \in \positivecomplex$ and $p, b, T \in \nats^+$
    \begin{align}
        \ex{\sum_{i} \frac{1}{\lambda_i\left(U U^\transpose\right) - z}}
    \end{align}
    is always continuous w.r.t. $z$ for any $p, b, T$.

    If the above assumptions are satisfied, the non-zero eigenvalue concentrates. To be more specific, let $\overline{\lambda^{>0}}$ be the mean of non-zero eigenvalues of $\frac{1}{b T} U^p \left(U^p\right)^\transpose$ and use $\ex{\overline{\lambda^{>0}}}^2$ to represent the overall situation of non-zero eigenvalues. Then there is
    \begin{align}
        \ex{\frac{\ex{\overline{\lambda^{>0}} / \sqrt{v}}^2}{\left(\frac{\lambda}{\sqrt{v}}\right)^2 + v}}
        \le&    \frac{\sqrt 2}{c \sqrt{v}} \frac{\alpha^2}{v \cdot \min(b T, p)^2}  \sqrt{c + \frac{\left(2 \sqrt{2} + 2\right) c \alpha}{v p} + \frac{c \beta}{v p} + \frac{c}{v p}} \label{eq:im_bound},\\
        \ex{\frac{\ex{\overline{\lambda^{>0}}}}{\lambda + \frac{v^2}{\lambda}}}
        \le&    \frac{\sqrt 2}{c \sqrt{v}} \frac{\alpha}{\min(b T, p)}  \sqrt{c + \frac{\left(2 \sqrt{2} + 2\right) c \alpha}{v p} + \frac{c \beta}{v p} + \frac{c}{v p}} \label{eq:re_bound}.
        \end{align}
    for any $v \ge v_0$, where $\lambda$ is a randomly selected eigenvalue of $T^p \defeq \frac{1}{b T} U^p \left(U^p\right)^\transpose$ and $c = p / b T \in [0, 1]$, and $v_0 \ge 2 c$ satisfying 
    \begin{align}
        \frac{v_0 + (1 - c)}{\sqrt{2}} > \frac{\tau}{v_0} + 2 \sqrt{c v_0} + 2 \sqrt{\tau} \label{eq:hard_condition}
    \end{align}
    with $\tau \defeq \frac{c}{p} \left(1 + \beta + 2\left(\sqrt{2} + 1\right) \alpha\right)$.
\end{restatable}

}{

% \arxivonly{\SpectralOfAccumulated*}
\begin{proof}[Proof of \cref{theorem:spectral_of_accumulated}]\label{proof:spectral_of_accumulated}

The proof is adapted from \citet{mp_quadratic_form} where independence conditions are replaced with Batch Dependence model and new regularities.


Cauchy-\sti{} transform method is used. 
When applied to empirical spectral density, by definition there is
\begin{align}
    s^{F^A}(z) = \trace{A - z I}^{-1} / p \defeq \trace{\left(A - z I\right)^{-1}} / p.
\end{align}
for positive semi-definite $A \in \reals^{p \times p}$.
Specifically, $\frac{1}{b T} U^p \left(U^p\right)^\transpose = \frac{1}{b T} U U^\transpose$'s \sti{} transform is
\begin{align}
    s_p(z) = \trace{\frac{1}{b T} U U^\transpose - z I}^{-1} / p = b T / p \trace{U U^\transpose - z b T I}^{-1}.
\end{align}
% By \sti{} continuity theorem \citep{RMT_book}, it is sufficient to show that $s_p(s) \asto s(z)$ for all $z \in \positivecomplex$, where $s$ is the \sti{} transform of Marchenko-Pastur distribution with parameter $p$ and $n$. To this end, typical steps include the following steps \citep{mp_proof_sketch}:
% \begin{itemize}
    % \item $s_p(z) - \ex{s_p(z)} \asto 0$, by a martingale argument;
    % \item $\ex{s_p(z)} \to s(z)$.
% \end{itemize}

\NewDocumentCommand{\boundedby}{m}{\varXi\left(#1\right)}
To ease presentation, we define $\boundedby{g}$ to indicate (complex) functions whose magnitudes are bounded by positive real function $g$, i.e.,
\begin{align}
    h \in \boundedby{g} \iff \forall x, y, \abs{h(x, y)} \le g(x),
\end{align}
where $y$ indicates variables other than $x$ that $h$ relies.
$\boundedby{\cdot}$ will be used combined with ``$=$'' imitating $O(\cdot)$. Since $\boundedby{\cdot}$ does not hide constant scaling factors and biases in it, unlike $O(\cdot)$ it can be freely added, averaged, multiplied and divided, i.e.,
\begin{align}
    \boundedby{g_1} + \boundedby{g_2} \in& \boundedby{g_1 + g_2},
    \frac{1}{n} \sum_{i=1}^n \boundedby{g_i} \in \boundedby{\frac{1}{n}\sum_{i=1}^n g_i},\\
    \boundedby{g_1} \cdot \boundedby{g_2} \in& \boundedby{g_1 \cdot g_2},
    \frac{\boundedby{g_1}}{g_2} \in \boundedby{\frac{g_1}{g_2}}.
\end{align}

% For $s_p(z) - \ex{s_p(z)}$'s almost sure convergence, \citet{mp_quadratic_form} refer to \citet{mp_proof_sketch}, which we will repeat here to examine and replace independence assumptions within.

% \NewDocumentCommand{\condiex}{O{k} m}{\ex[#1]{#2}}
% Let $\condiex[k]{\cdot} \defeq \ex{\cdot \mid U^p_{\cdot, 1: k}}$ denote the conditional expectation given $U^p_{\cdot, 1}, \dots, U^p_{\cdot, k}$. Then $s_p(z) = \condiex[b]{s_p(z)}$ and $\ex{s_p(z)} = \condiex[0]{s_p(z)}$, and there is
% \begin{align}
    % s_p(z) - \ex{s_p(z)}
    % =&  \sum_{k=1}^{b} \left(\condiex[k]{s_p(z)} - \condiex[k-1]{s_p(z)}\right)
    % =  \sum_{k=1}^{b} \gamma_k,
% \end{align}
% where $\gamma_k \defeq \condiex[k]{s_p(z)} - \condiex[k-1]{s_p(z)}$. Note that $\set{\condiex[k]{s_p(z)}}_k$ is already a Doob martingale, so $\set{\gamma_k}_k$, as its difference, is a sequence of martingale differences.
        
% \NewDocumentCommand{\invR}{O{1}}{\left(R^p_k\right)^{-#1}}
% Let $R^p_{k} \defeq \frac{1}{b T} \sum_{k'} \uut[k'] - z I - \frac{1}{b T}u_k u_k^\transpose$. By \cref{lemma:3.1_from_mp_quadratic_form}(1), $R^p_{k}$ is invertible, and by Sherman-Morrison formula there is
% \begin{align}
    % s_p(z)
    % =&  \trace{R^p_k + \frac{1}{b T} u_k u_k^\transpose }^{-1} / p
    % =  \frac{1}{p} \trace{\invR - \frac{\invR u_k u_k^\transpose \invR / b T}{1 + u_k^\transpose \invR u_k / b T}}\\
    % =&  \frac{1}{p} \left(\trace{\invR} - \frac{u_k^\transpose \invR[2] u_k}{b T + u_k^\transpose \invR u_k}\right).
% \end{align}
% By minor single sample assumption, $\condiex[k]{\trace{\invR}} = \condiex[k-1]{\trace{\invR}}$ is bounded by $\beta(z)$. Regarding the second term,
% \begin{align}
    % &   \abs{\frac{u_k^\transpose \invR[2] u_k}{b T + u_k^\transpose \invR u_k}}\\
    % =&      \frac{\abs{\trace{u_k u_k^\transpose \invR \invR}}}{\abs{b T + u_k^\transpose \invR u_k}}
    % \le    \frac{\norm{u_k u_k^\transpose \invR \invR}_1}{b T + u_k^\transpose \invR u_k}\\
    % \le&   \frac{\norm{u_k u_k^\transpose \invR}_1 \norm{\invR}_{\infty}}{b T + u_k^\transpose \invR u_k}
    % \le    \frac{1}{v} \frac{u_k^\transpose \invR u_k}{b T + u_k^\transpose \invR u_k}
    % \le \frac{1}{v},
% \end{align}
% where the third inequality follows \cref{lemma:3.1_from_mp_quadratic_form}(1).
% Therefore $\gamma_k$ can be bounded by
% \begin{align}
    % \gamma_k \le& \frac{2}{v} \frac{1}{p} + \beta(z) < \infty.
% \end{align}
% By Azuma's inequality, there is
% \begin{align}
    % \prob{\abs{s_p(z) - \ex{s_p(z)}} > \epsilon} \le 2 \exp\left(-\frac{\epsilon^2}{2 p \left(\frac{2}{v} \frac{1}{p}\right)^2}\right) \le 2 \exp\left(-\epsilon^2 v^2 p / 8\right).
% \end{align}
% Given that $\prob{\abs{s_p(z) - \ex{s_p(z)}}}$ decays exponentially with $p$, for every $\epsilon > 0$, there is
% \begin{align}
    % \sum_{p=0}^\infty \prob{\abs{s_p(z) - \ex{s_p(z)}} > \epsilon} < \infty
% \end{align}
% which implies $s_p(z) - \ex{s_p(z)} \asto 0$ (Theorem 7.5 by \citet{exponential_decay_to_as}).

Consistent with final conclusion, fix $z = 0 + v i$ ($v \in \reals^+$) such that $v \ge v_0$ throughout the proof.
Define $A^p \defeq \sum_{k} \uut$.
Sample an auxiliary vector $u_{T, b+1} = u_{b T  + 1} \in \reals^p$ so that it is sampled from the conditional distribution given the first $T-1$ batches but it is conditionally independent with other samples in $U^T$, i.e., an extra sample for the last batch. This dependence relation can be expressed by only adding edges $U^{1: T-1} \to u_{T, b+1}$ to the SCMs of the Batch Dependence Model. With the auxiliary vector, define $B^p \defeq A^p +  \uut[T, b+1]$. %Define $C^b_t \defeq B^p - u^t \left(u^t\right)^\transpose$

By \cref{lemma:3.1_from_mp_quadratic_form}(1), $B^p - z b T I$ is non-degenerate and
\begin{align}
    p 
    =&  \trace{\left(B^p - z b T I\right) \left(B^p - z b T I\right)^{-1}}\\
    =&   \sum_{t=1}^{T} \sum_{l=1}^{b + \indic{t = T}} u_{t, l}^\transpose \left(B^p - z b T I\right)^{-1} u_{t, l} - z b T \trace{B^p - z b T I}^{-1}.
\end{align}
Taking expectations and using the exchangeability within each batch give
\begin{align}
    p = \sum_{t=1}^{T} (b + \indic{t = T}) \ex{u_t^\transpose \left(B^p - z b T I\right)^{-1} u_t} - z b T \ex{\trace{B^p - z b T I}^{-1}} \label{eq:a3}.
\end{align}

Define $S_p(z) \defeq \trace{A^p - z b T I}^{-1}$ and note that $S_p(z) = (p / b T) s_p(z)$. 

By \cref{lemma:3.1_from_mp_quadratic_form}(2), there is
\begin{align}
    \ex{\trace{B^p - z b T I}^{-1}} =& \ex{S_p(z)} + \boundedby{1/ v b T} = \ex{S_p(z)} + \boundedby{c / v p} \label{eq:a1}.
\end{align}

\NewDocumentCommand{\approxmatrix}{O{\boundedby} O{2}}{#1{\frac{#2 \sqrt{2} c \alpha}{v p}}}
We now prove
\begin{align}
    \frac{1}{T} \sum_{t=1}^{T} \ex{u_t^\transpose \left(B^p - z b T I\right)^{-1} u_t} = \frac{\ex{S_p(z)}}{1 + \ex{S_p(z)}} + t \label{eq:claim}, 
\end{align}
where $\abs{t}$ is bounded by a function of $c, \alpha, \beta, v, p$.

\NewDocumentCommand{\approxfunction}{O{\boundedby}}{#1{1}}
A complex function $\frac{x}{1 + x} = 1 - \frac{1}{x + 1}$ emerges many times. We will approximate it to the first order so its complex derivative should be computed and bounded.
\begin{align}
    \abs{\left(\frac{x}{1 + x}\right)'}
    =&  \abs{\frac{1}{(x+1)^2}} 
    = \frac{1}{\abs{x + 1}^2}
\end{align}
Therefore, if $x_1, x_2$ both stay away from $-1$, then $\abs{\left(\frac{x'}{1 + x'}\right)'} = \approxfunction$ on the line connecting $x_1, x_2$ and we can approximate $\frac{x_2}{1 + x_2}$ by $\frac{x_1}{1 + x_1} + \boundedby{1} \cdot \Delta x = \frac{x_1}{1 + x_1} + \boundedby{\Delta x}$, where $\Delta x = x_2 - x_1$.
In latter application, $x$, both the start and the end of approximation, is often of form $\frac{1}{n} \sum_{i=1}^n \ex{u_i^\transpose \left(C - z b T I\right)^{-1} u_i}$ possibly with averaging or expectation missing, where $C$ is real symmetric positive semi-definite and $u_i$ is a real vector. The eigenvalues in $\left(C - z b T I\right)^{-1}$ are
\begin{align}
    \frac{1}{\lambda_i(C) - v b T i}
    =&  \frac{\lambda_i(C) + v b T i}{\lambda_i(C)^2 + (v b T)^2},
\end{align}
whose real part is
\begin{align}
    \rpart{\frac{1}{\lambda_i(C) - v b T i}}
    =&  \frac{\lambda_i(C)}{\lambda_i(C)^2 + (v b T)^2} \ge 0.
\end{align}
As a result, the real part of inner products is always non-negative and $x$ stays away from $-1$, and the magnitude of derivatives is $\approxfunction$.

Another approximation is done between $C^p_k$ and $A_p$, whose difference is the outer products of a constant number of random vectors, and it should be minor considering there are $b T$ of them. Formally, for real symmetric positive semi-definite $C$ with eigenvalue decomposition $C = V \Lambda V^\transpose$ by real matrices $V$ and $\Lambda$, $(C - z I)$ can be decomposed to $(C - z I) = V \left(\Lambda - z I\right) V^\transpose$, and non-degenerate $\left(C - z I\right)^{-1}$ to $\left(C - z I\right)^{-1} = V \left(\Lambda - z I\right)^{-1} V^\transpose \defto V \Sigma \Sigma V^\transpose$ where $\Sigma \defeq \sqrt{\left(\Lambda - z I\right)^{-1}}$. Let $S \defeq V \Sigma V^\transpose$ to have $S^\transpose S = S S = \left(C - z I\right)^{-1}$. After that, there is
\begin{align}
    &   \abs{y^\transpose \left(C + x x^\transpose - z I\right)^{-1}y - y^\transpose \left(C - z I\right)^{-1} y}
    =  \abs{y^\transpose \left(\left(C + x x^\transpose - z I\right)^{-1} - \left(C - z I\right)^{-1} \right) y}\\
    =&  \abs{\frac{
            y^\transpose \left(C - z I\right)^{-1} x x^\transpose \left(C - z I\right)^{-1} y
        }{1 + x^\transpose \left(C - z I\right)^{-1} x}}
    =   \abs{\frac{
            \left(y^\transpose S^\transpose S x\right) \left(x^\transpose S^\transpose S y\right)
        }{1 + x^\transpose \left(C - z I\right)^{-1} x}}
    =  \abs{\frac{
            \left(a^{\transpose} \bar{b}\right) \left(\bar{b}^\transpose a\right)
        }{1 + x^\transpose \left(C - z I\right)^{-1} x}}\\
    =&   \frac{
            \abs{a^* b} \abs{a^* b}
        }{\abs{1 + x^\transpose \left(C - z I\right)^{-1} x}}
    \le \frac{
            \norm{a^*}_2 \norm{b}_2 \norm{a^*}_2 \norm{b}_2
        }{\abs{1 + x^\transpose \left(C - z I\right)^{-1} x}}
    =   \frac{
            \abs{a^* a} \abs{b^* b}
        }{\abs{1 + x^\transpose \left(C - z I\right)^{-1} x}}\\
    =&  \frac{
            \abs{\trace{y y^\transpose S^* S}} \abs{b^* b}
        }{\abs{1 + x^\transpose \left(C - z I\right)^{-1} x}}
    \le \frac{
            \norm{y y^\transpose S^* S}_1 \abs{b^* b}
        }{\abs{1 + x^\transpose \left(C - z I\right)^{-1} x}}
    \le \frac{
            \norm{y y^\transpose}_1 \norm{S^* S}_\infty \abs{b^* b}
        }{\abs{1 + b^\transpose b}}
    =   \frac{\norm{y}_2^2}{\ipart{z}} \frac{
            \abs{b^* b}
        }{\abs{1 + b^\transpose b}},
\end{align}
where $a \defeq S y, b \defeq \bar{S} \bar{x} = \bar{S} x$, the second step is from Sherman-Morrison formula, and the second last inequality is due to \cref{lemma:abs_trace_and_schatten_1}. The fact, that $S^* S$ is positive semi-definite whose largest eigenvalue is smaller than the upperbound $\frac{1}{v}$ of $S^\transpose S$'s eigenvalue magnitude, is also used.  To bound the fraction between $\abs{b^* b}$ and $\abs{1 + b^\transpose b}$, recall the eigenvalue decomposition on $\left(C - z I\right)^{-1}$ 
\begin{align}
    \left(C - z I\right)^{-1} = V \Sigma \Sigma^\transpose V^\transpose
\end{align}
and $S = V \Sigma^\transpose V^\transpose$. Then 
\begin{align}
    b^\transpose b &= v^\transpose \Sigma \Sigma v,
    b^* b = v^\transpose \bar{\Sigma} \Sigma v,
\end{align}
where $v \defeq V^\transpose x$ is a real vector. Notice that $\Sigma \Sigma = \diag{\frac{1}{\lambda_i(C) - v i}} = \diag{\frac{\lambda_i(C) + v i}{\lambda_i(C)^2 + v^2}}$ where both real and imaginary parts are non-negative, and that $\Sigma^* \Sigma = \diag{\frac{\abs{\lambda_i(C) + v i}}{\lambda_i(C)^2 + v^2}}$. With this, the inner products are simplified to
\begin{align}
    b^\transpose b &= \sum_{i} \frac{v_i^2 \lambda_i(C)}{\lambda_i(C)^2 + v^2} + i \sum_{i} \frac{v_i^2 v}{\lambda_i(C)^2 + v^2},
    b^* b = \sum_{i} \abs{\frac{v_i^2}{\lambda_i(C)^2 + v^2} \lambda_i(C) + i \frac{v_i^2 v}{\lambda_i(C)^2 + v^2}}
\end{align}
Representing complex numbers by 2-dimensional vectors $w_i \defeq \begin{bmatrix} \frac{v_i^2 \lambda_i(C)}{\lambda_i(C)^2 + v^2} & \frac{v_i^2 v}{\lambda_i(C)^2 + v^2} \end{bmatrix}^\transpose$, there are
\begin{align}
    \abs{b^\transpose b} &= \norm{\sum_i w_i}_2,
    \abs{b^* b} = \sum_i \norm{w_i}_2.
\end{align}
Noting that all entries of $w_i$'s are non-negative, there is
\begin{align}
    \abs{b^* b}
    &=  \sum_i \norm{w_i}_2
    \le \sum_{i} \norm{w_i}_1 
    =   \norm{\sum_i w_i}_1 
    \le \sqrt{2} \norm{\sum_i w_i}_2 = \sqrt{2} \abs{b^\transpose b}.
\end{align}
So $\frac{\abs{b^* b}}{\abs{b^\transpose b}} \le \sqrt{2}$. Given that the real part of $b^\transpose b$ is non-negative, adding $1$ will only increase its magnitude. As a result, there is
\begin{align}
    &   \abs{y^\transpose \left(C + x x^\transpose - z I\right)^{-1}y - y^\transpose \left(C - z I\right)^{-1} y}
    \le  \frac{\sqrt{2} \norm{y}_2^2}{\ipart{z}},
\end{align}

When $z b T$ is substituted, there is
\begin{align}
    &   \abs{y^\transpose \left(C + x x^\transpose - z b T I\right)^{-1}y - y^\transpose \left(C - z b T I\right)^{-1} y}
    =  \frac{\sqrt{2} \norm{y}_2^2}{v b T}.
\end{align}
In later use, $y$ is instantiated by $u_k$ and there is $\norm{u_k}_2^2 = a \norm{x}_2^2 \le \alpha$ for any $t$, so by assumption the approximation error is always bounded by 
\begin{align}
    \abs{u_k^\transpose \left(C + x x^\transpose - z b T I\right)^{-1} u_k - u_k^\transpose \left(C - z b T I\right)^{-1} u_k} \le \approxmatrix[\boundedby][].
\end{align}

\NewDocumentCommand{\innersum}{}{\frac{1}{b T}\sum_{k}}
\NewDocumentCommand{\outersum}{}{}
\NewDocumentCommand{\innerapproximator}{O{\left(A^p - z b T I\right)^{-1}} O{k}}{ u_{#2}^\transpose #1 u_{#2} }
\NewDocumentCommand{\innerouterproduct}{O{}}{u_{k} #1 u_{k}^\transpose}
\NewDocumentCommand{\biasapproximator}{}{\frac{\ex{\innersum \innerapproximator}}{1 + \ex{\innersum \innerapproximator}}}
\NewDocumentCommand{\diffapproximator}{}{\innerapproximator - \ex{\innersum \innerapproximator}}
With these two approximation techniques, we first approximate the LHS of \cref{eq:claim}.
To this end, let $C^p_k \defeq B^p - u_k u_k^\transpose$ and by Sherman-Morrison formula there is
\begin{align}
    &   u_k^\transpose \left(B^p - z b T I\right)^{-1} u_k
    =   u_k^\transpose \left(C^p_k + u_k u_k^\transpose - z b T I\right)^{-1} u_k\\
    =&  u_k^\transpose \left(\left(C^p_k - z b T I\right)^{-1} - \frac{\left(C^p_k - z b T I\right)^{-1} u_k u_k^\transpose \left(C^p_k - z b T I\right)^{-1}}{1 + u_k^\transpose\left(C^p_k - z b T I\right)^{-1} u_k}\right) u_k\\
    =&  \frac{u_k^\transpose \left(C^p_k - z b T I\right)^{-1}u_k}{1 + u_t^\transpose\left(C^p_k - z b T I\right)^{-1} u_k}
    =  \frac{u_k^\transpose \left(A^p - z b T I\right)^{-1}u_k}{1 + u_k^\transpose\left(A^p - z b T I\right)^{-1} u_k} + \approxfunction \cdot \approxmatrix.
\end{align}
After that, there is
\begin{align}
    &   \frac{1}{T} \sum_{t=1}^T \ex{u_t^\transpose (B^p - z b T I)^{-1} u_t}\\
    =&  \frac{1}{T} \sum_{t=1}^T \frac{1}{b} \sum_{l=1}^b \ex{\frac{u_{t, l}^\transpose \left(A^p - z b T I\right)^{-1} u_{t, l}}{1 + u_{t, l}^\transpose \left(A^p - z b T I\right)^{-1} u_{t, l}}} + \approxfunction \cdot \approxmatrix\\
    =&  \outersum \innersum  \ex{\biasapproximator} + \approxmatrix \\
        &+ \outersum \innersum \ex{\approxfunction \abs{\diffapproximator}}\\
    =&  \outersum \biasapproximator + \approxmatrix\\ 
        &+ \approxfunction  \outersum \innersum \ex{\abs{\diffapproximator}}\\
    =&  \outersum \biasapproximator \label{eq:term1}\\ 
        &+ \approxfunction  \ex{\abs{u_r^\transpose \left(A^p - z b T I\right)^{-1} u_r - \ex{u_r^\transpose \left(A^p - z b T I\right)^{-1} u_r}}}  \label{eq:term2} \\&+ \approxmatrix,
\end{align}
where $r$ in the last line is a uniformly randomly selected index from $\set{1, \dots, b T}$ independently to the training process.

\NewDocumentCommand{\approxbias}{O{\boundedby}}{#1{\frac{c \beta}{v p}}}
Note that $S_p(z) = \trace{A^p - z b T I}^{-1}, \ex{S_p(z)} = \ex{\trace{A^p - z b T I}^{-1}}$. So for the term in \cref{eq:term1} we proceed by proving $\frac{1}{b} \sum_{l=1}^{b} \ex{u_{t, l}^\transpose \left(A^p - z b T I\right)^{-1} u_{t, l}}$ approximates $\ex{\trace{A^p -z b T I}^{-1}}$. For convenience let $D \defeq b T\left(A^p - z b T I\right)^{-1}$ be an alias to it, whose spectral norm satisfies $\norm{D}_{\infty} \le b T \frac{1}{ v b T} = \frac{1}{v}$, then
\begin{align}
    &       \abs{\ex{\innersum \innerapproximator} - \ex{S_p(z)}}\\
    =&      c\abs{\frac{\ex{\innersum \innerapproximator[b T \left(A^p - z b T I\right)^{-1}]}}{p} - \frac{ \ex{b T S_p(z)}}{p}}\\
    =&      \frac{c}{p} \abs{\innersum \ex{\innerapproximator[D]} - \ex{\trace{D}}}
    =      \frac{c}{p} \abs{\ex{\trace{\left(\innersum \innerouterproduct - I\right) D}}}\\
    \le&    \frac{c}{p} \ex{\norm{\left(\innersum \innerouterproduct - I\right) D}_1}
    \le    \frac{c}{p} \ex{\norm{\left(\innersum \innerouterproduct - I\right)}_1 \norm{D}_{\infty}}\\
    \le&    \frac{c}{v p}\ex{\norm{\left(\innersum \innerouterproduct - I\right)}_1}
    \le    \frac{c}{v p}  \ex{\sqrt{p \trace{\left(T^p_t - I\right)^\transpose \left(T^p_t - I\right)}}}\\
    =&      \approxbias,
\end{align}
where the last inequality is because
\begin{align}
    \norm{A}_1 =& \sum_{i=1}^{p} \abs{\lambda_i(A)} = \norm{\begin{bmatrix}
        \lambda_1(A) & \cdots & \lambda_i(A) & \cdots \lambda_p(A)
    \end{bmatrix}^\transpose}_1\\
    \le&    \sqrt{p} \norm{\begin{bmatrix}
        \lambda_1(A) & \cdots & \lambda_i(A) & \cdots \lambda_p(A)
    \end{bmatrix}^\transpose}_2\\
    =&  \sqrt{p} \norm{A}_2 = \sqrt{p \trace{A^\transpose A}},
\end{align}
given that $A = \left( \left(\innersum \innerouterproduct - I\right) \right)$ is symmetric so that its singular values are absolute eigenvalues.
With approximation on complex function $\frac{x}{1 + x}$, this $\approxbias$-boundedness implies $\approxfunction \cdot \approxbias = \approxbias$ approximation of in \cref{eq:term1}.

\NewDocumentCommand{\approxvariance}{O{\boundedby}}{#1{\frac{c \alpha}{v p}}}
\NewDocumentCommand{\utdu}{}{u_t^\transpose D u_t}
\NewDocumentCommand{\xtdx}{}{x_r^\transpose D x_r}
For the difference term in \cref{eq:term2}, we prove its diminishment by $\frac{1}{v}$-bounded variance of $u_{t, l}^\transpose (A^p - z b T I)^{-1} u_{t, l}$, or formally
\begin{align}
    \ex{\abs{X - \ex{X}}^2} - \ex{\abs{X - \ex{X}}}^2 =& \ex{\left(\abs{X - \ex{X}} - \ex{\abs{X - \ex{X}}}\right)^2} \ge 0\\
    \ex{\abs{X - \ex{X}}} \le&  \sqrt{\ex{\abs{X - \ex{X}}^2}} = \sqrt{\var{X}},
\end{align}
and 
\begin{align}
    &   \var{\innerapproximator[\left(A^p - z b T I\right)^{-1}][r]}
    =  \frac{c^2}{p^2} \var{\innerapproximator[D][r]}\\
    =&  \frac{c^2}{p^2} \left(\ex{\trace{\innerapproximator[D][r] \innerapproximator[D][r]}} - \ex{\trace{\innerapproximator[D][r]}}^2\right)
    \le    \frac{c^2}{p^2} \alpha^2 \left(\ex{\trace{\xtdx \xtdx}}\right)\\
    \le&   \frac{c^2 \alpha^2}{p^2} \ex{\trace{\xtdx \xtdx}}
    \le    \frac{c^2 \alpha^2}{p^2} \ex{\norm{x^p}_2^4 \norm{D}_\infty^2}
    =  \frac{c^2 \alpha^2}{v^2 p^2},
\end{align}
where the last step follows that $x^p$'s norm is bounded and that $\norm{D}$ is also uniformly bounded. 

\NewDocumentCommand{\approxlargest}{O{\boundedby}}{#1{\approxmatrix[] + \approxbias[] + \approxvariance[]}}
To sum up, we have obtained 
\begin{align}
    \frac{1}{T} \sum_{t=1}^T \ex{u_t^\transpose \left(B^p - z b T I\right)^{-1} u_t} = \frac{\ex{S_p(z)}}{1 + \ex{S_p(z)}} + t,
\end{align}
where $\abs{t} = \approxlargest$.

\NewDocumentCommand{\approxall}{O{\boundedby}}{#1{\approxlargest[] + \frac{c}{v p} + \frac{c \alpha }{v p}}}

With \cref{eq:claim}, \cref{eq:a1}, one can reduce \cref{eq:a3} to
\begin{align}
    p =& T (b + O(1)) \left(\frac{\ex{S_p(z)}}{1 + \ex{S_p(x)}} + t\right) - z b T \left(\ex{S_p(z)} + \boundedby{c / v p}\right),
\end{align}
and
\begin{align}
    \frac{\ex{S_p(z)}}{1 + \ex{S_p(x)}} - z \ex{S_p(z)} =& \frac{p}{b T} + s = c + s,
\end{align}
where $s = \approxall$.


$\ex{S_p(p)}$ always have a non-negative real part because real parts of eigenvalues of $\left(U U^\transpose - z b T I\right)$ are always non-negative by an argument similar to previous ones. Since $\alpha, \beta, c$ and $p$ depend only on $p, b, T$ instead of $v$, $s = \approxall$ is bounded by $\tau / v$ which satisfies $\tau$ is constant w.r.t $v$, $\frac{v_0 + (1 - c)}{\sqrt{2}} > \frac{\tau}{v_0} + 2 \sqrt{c v_0} + 2 \sqrt{\tau}$ and $v \ge v_0 \ge 2 c$. Given $c \in [0, 1]$, by \cref{lemma:bound_of_sti}, there is
\begin{align}
    &   \abs{\ipart{\ex{S_p(v i)}}} \le \abs{\ex{S_p(v i)}} \\
    \le& \approxquadratic{\approxall[]}
\end{align}
for $v \ge v_0$. Bounds using the real part are similarly obtained.

The similar bound for $s_p(z) = (b T / p) S_p(z) = \frac{1}{c} S_p(z)$ is
\begin{align}
    \abs{\ipart{\ex{s_p(v i)}}}
    \le& \frac{1}{c}\approxquadratic{\approxall[]}.
\end{align}

The expected mean $\ex{\overline{\lambda^{ > 0}}}$ of $\frac{1}{b T} U^p \left(U^p\right)^\transpose$'s non-zero eigenvalue is
\begin{align}
    &   \ex{\frac{\trace{\frac{1}{b T} U^p \left(U^p\right)^\transpose}}{\min(b T, p)}}
    =  \frac{\frac{1}{b T}\sum_{k=1}^{b T}\ex{\trace{u_{k} u_k^\transpose}}}{\min(b T, p)}
    =  \frac{\frac{1}{b T}\sum_{k=1}^{b T}\ex{\norm{u_k}_2^2}}{\min(b T, p)}
    \le   \frac{\alpha}{\min(b T, p)}.
\end{align}

Finally, the desired conclusion is obtained through \cref{lemma:sti_and_eigenvalue_ratio} by
\begin{align}
        \ex{\frac{\ex{\overline{\lambda^{>0}} / \sqrt{v}}^2}{\left(\frac{\lambda}{\sqrt{v}}\right)^2 + v}}
    \le    \frac{1}{v} \ex{\overline{\lambda^{>0}}}^2 \abs{\ipart{s_p(v i)}},
    % \le&    \frac{1}{v c} \frac{\alpha^2}{\min(b T, p)^2}  \approxquadratic{\approxall[]},\\
        \ex{\frac{\ex{\overline{\lambda^{>0}}}}{\lambda + \frac{v^2}{\lambda}}}
    \le    \ex{\overline{\lambda^{>0}}} \abs{\rpart{s_p(v i)}}.
    % \le&    \frac{1}{c} \frac{\alpha}{\min(b T, p)}  \approxquadratic{\approxall[]}.
\end{align}

\end{proof}
}


