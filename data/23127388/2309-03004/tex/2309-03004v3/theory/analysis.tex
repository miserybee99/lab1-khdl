\subsection{Gradients w.r.t. Weight Matrices and Zeroth Biases}\label{sec:gradients}

Since we are interested in the flat minima, gradients will be heavily involved. Therefore, we will compute gradients and updates to weight matrices in this subsection. We will also define the notion of effective gradient sparsity and argue its practical and theoretical importance, justifying our theories that are built on effective gradient sparsity.

Let $\mlp_*^l$ be any sublayer of the block $\mlp^l$, whose weight matrix is $W^l$. Let $g^{l}_{*, i}$ be the gradient w.r.t. the output of $\mlp_*^l$ on sample $(x_s, y_s)$ at a token $u_i^{l}$, where $u_i^{l}$ abstracts $x^{l-1}$, $x^{l-1} + d^l$ or $\alpha^l$. Let $G^{l}_{*}$ be the stacked version of $g^{l}_{*, i}$ and $U^l$ be that of $u_i^l$.

One can compute the gradient w.r.t. $W^l$ on a token input $u^{l}_i$ by
\begin{align}
    \left(g^{l}_{*, i} \hadamard \sigma'_i \right) \left(u^{l}_i\right)^{\transpose},
\end{align}
where $\sigma'_i = \gamma^l_i$ for weight matrices, while $\sigma'$ is $[1]_{i \in \set{1, \dots, d}}$ for value matrices since there are no activation functions in value layers.
Summing updates from all tokens to obtain gradients of sample $(x_s, y_s)$ gives
\begin{align}
    \derivatives{\loss(\theta, (x_s, y_s))}{W^l}
    \defeq&  \sum_{i} \left(g^{l}_{*, i} \hadamard \sigma'_i \right) \left(u^{l}_{i}\right)^\transpose
    =      \left(G^{l}_* \hadamard \Sigma'\right) \left(U^l\right)^\transpose.
\end{align}

The gradients w.r.t. zeroth biases, if they exist, are also computed.
\begin{align}
    \derivatives{\loss(\theta, (x_s, y_s))}{d^l_i}
    =& \left(J_{\loss}\left(\alpha^l\right) J_{\alpha^l}\left(K^l \left(x^l_i + d^l_i\right) + b_K^l\right) J_{K^l \left(x^l_i + d^l_i\right) + b_K^l}\left(d^l_i\right)\right)^\transpose\\
    =& \left(\left(g^{l}_{K, i}\right)^\transpose \diag{\gamma^l_i} K^l\right)^\transpose
    =   \left(K^l\right)^\transpose \left(g^l_{K, i} \hadamard \gamma^l_i\right).
\end{align}


Instantiating these results on key and value matrices obtains \cref{lemma:gradients}.
\begin{lemma}[Gradients w.r.t. weight matrices]\label{lemma:gradients}
    The gradients w.r.t. weight matrices and zeroth biases in $\dbmlp^l$ of sample $(x_s, y_s)$ are
    \begin{align}
        \derivatives{\loss(\theta, (x_s, y_s))}{K^l}
        =& \left(G^{l}_K \hadamard \Gamma^l\right) \left(X^{l-1} + D^l\right)^\transpose,
        \derivatives{\loss(\theta, (x_s, y_s))}{V^l}
        = G^{l}_V \left(\Alpha^{l}\right)^\transpose,\\
        \derivatives{\loss(\theta, (x_s, y_s))}{D^l}
        =&  \left(K^l\right)^\transpose \left(G^l_K \hadamard \Gamma^l\right).
    \end{align}
    Particularly if hidden features are single tokens, there are
    \begin{align}
        \derivatives{\loss(\theta, (x_s, y_s))}{K^l}
        =& \left(g^{l}_K \hadamard \gamma^l\right) \left(x^{l-1} + d^l\right)^\transpose,
        \derivatives{\loss(\theta, (x_s, y_s))}{V^l}
        = g^{l}_V \left(\alpha^{l}\right)^\transpose,\\
        \derivatives{\loss(\theta, (x_s, y_s))}{d^l}
        =&  \left(K^l\right)^\transpose \left(g_{K}^l \hadamard \gamma^l\right).
    \end{align}
\end{lemma}

\cref{lemma:gradients} introduces a Hadamard product between gradients from deeper layers and the derivatives of the activation in weight layers. We define it as the effective gradient pattern.
\begin{definition}[Effective gradient sparsity]\label{def:effective_gradient_sparsity}
    Define effective gradient patterns of $\mlp^l$ on sample $(x_s, y_s)$ at token $x^{l-1}$ to be 
    \begin{align}
        \eta^l \defeq \diag{g^l_K} \gamma^l = g^l_K \hadamard \gamma^l. 
    \end{align}
    Let $\Eta^l = G^l_K \hadamard \Gamma^l \in \reals^{n \times k}$ (capitalized ``$\eta$'' instead of capitalized ``$h$'') be its stacked version when there are $k$ tokens.

    Effective gradient sparsity is that most elements in $\eta^l$ are near zero for most samples and tokens.
    For mathematical convenience, define $\norm{\eta^l}_2^2 = \norm{\diag{g^l_K} \gamma^l}_2^2 = \norm{g^l_K \hadamard \gamma^l}_2^2$ to be the effective gradient sparsity measured in squared $L_2$ norm.
\end{definition}

This notion first simplifies \cref{lemma:gradients}.
\begin{lemma}[Gradients w.r.t. weight matrices, restated with $\eta$ and $\Eta$]\label{lemma:gradients_with_eff}
    The gradients w.r.t. weight matrices and zeroth biases in $\dbmlp^l$ of sample $(x_s, y_s)$ are
    \begin{align}
        \derivatives{\loss(\theta, (x_s, y_s))}{K^l}
        =& \Eta^l \left(X^{l-1} + D^l\right)^\transpose,
        \derivatives{\loss(\theta, (x_s, y_s))}{V^l}
        = G^{l}_V \left(\Alpha^{l}\right)^\transpose,\\
        \derivatives{\loss(\theta, (x_s, y_s))}{D^l}
        =&  \left(K^l\right)^\transpose \Eta^l.
    \end{align}
    Particularly if hidden features are single tokens, there are
    \begin{align}
        \derivatives{\loss(\theta, (x_s, y_s))}{K^l}
        =& \eta^l \left(x^{l-1} + d^l\right)^\transpose,
        \derivatives{\loss(\theta, (x_s, y_s))}{V^l}
        = g^{l}_V \left(\alpha^{l}\right)^\transpose,
        \derivatives{\loss(\theta, (x_s, y_s))}{d^l}
        =  \left(K^l\right)^\transpose \eta^l.
    \end{align}
\end{lemma}

This sparsity inherits sparsity in $\gamma$s but also allows more ``sparsity'' due to $g^l_K$s. Sparsity in $g^l_K$ is also meaningful in the sense that if the $i$-th entry $g^l_i$ of $g^l_K$ is small in magnitude, then 1) there is little contribution of $i$-th neuron in gradients to shallower layers during the backward propagation and 2) $\alpha^l_i$ does not influence the output much in forward propagation if the activation is also near-zero. Therefore, the $i$-th neuron can also be pruned during backward propagation and possibly during inference with minor cost in accuracy, and thus this notion is of even more practical value, although $g^l_K$ cannot be known before backward propagation. 
The notion of effective gradient sparsity in $\eta^l$ considers the two kinds of sparsity in a combined manner.
This incorporation of gradients w.r.t. activations also reminds us of the improved knowledge attribution method proposed by \citet{knowledge_neurons} where not only activation magnitude but also the gradients of model output w.r.t. the activation are exploited. 
Last but not least, the effective sparsity hides the complexity of deeper layers and attributes its emergence solely to $K^l$, which is somehow shallow despite there being dozens of deeper modules and allows easier theoretical manipulations.

More importantly, effective gradient sparsity patterns are what key layers try to memorize in columns, given in \cref{obs:memorizing_eff}. 
\begin{observation}[$\eta$s are memorized in key matrices columns]\label{obs:memorizing_eff}
    Consider the update of one sample to the key matrix given by \cref{lemma:gradients_with_eff}
    \begin{align}
            &\derivatives{\loss(\theta, (x_s, y_s))}{K^l}
            = \Eta^l \left(X^{l-1} + D^l\right)^\transpose
            = \sum_{i} \eta^l_i \left(x^{l-1}_i + d^l_i\right)^\transpose\\
            =&  \begin{bmatrix}
                \sum_i \left(X^{l-1}_{i, 1} + D^{l}_{i, 1}\right)\eta^l_i & \cdots & \sum_{i} \left(X^{l-1}_{i, j} + D^{l}_{i, j}\right) \eta^l_i & \cdots & \sum_{i} \left(X^{l-1}_{i, d} + D^{l}_{i, d}\right) \eta^l_i
            \end{bmatrix}.
    \end{align}
    In the update of each column, a mixture of effective gradient sparsity patterns is borne into the key matrix with different weights given by the input.
\end{observation}
Taking a transposed view, key layers also memorize $x^{l-1}_i$s, under the control of $\eta^l_i$s.
\begin{observation}[$\eta$s control row memorization in key matrices]\label{obs:memorizing_inputs}
    Consider the update of one sample to the key matrix given by \cref{lemma:gradients_with_eff}
    \begin{align}
            \derivatives{\loss(\theta, (x_s, y_s))}{K^l}
            =& \Eta^l \left(X^{l-1} + D^l\right)^\transpose
            = \sum_{i} \eta^l_i \left(x^{l-1}_i + d^l_i\right)^\transpose
            =  \begin{bmatrix}
                \sum_i \Eta^l_{j, i} \left(x^{l-1}_i + d^l_i\right)^\transpose
            \end{bmatrix}_j.
    \end{align}
    In the update of each row, a mixture of inputs is borne into the key matrix, weighted by entries of effective gradient sparsity pattern.
\end{observation}

Similar things also happen in value matrices but with $g^{l}_V$s and $\alpha^l$s memorized. 
It is interesting to see that linear layers are trying to resemble the gradients back-propagated to them. 
This observation may lead to a notion of pseudo gradients which can be calculated as the forward propagation sweeps by. Maybe a portion of samples can be trained by these pseudo gradients to save computation budgets. However, this is out of our scope and is left for future works to explore.

Unfortunately, effective gradient sparsity is not strictly related to activation sparsity even under common activation functions, so we keep the notion of gradient sparsity as well. 
The gap is due to the Hadamard product with $g^l_K$, which can cause smaller $\norm{\eta^l}_2^2$ without gradient sparsity on $\gamma^l$ by 1) reducing the norm of itself, or 2) misaligning itself with $\gamma^l$, i.e. multiplying small entries in $g^l_K$ with the derivatives of activated neurons and leaving large gradients to non-activated neurons. The $L_2$ modeling of sparsity also hinders the direct relation with sparsity measured in $L_0$ norms.
The first possibility can be eliminated by the phenomenon of parameter growth already discovered by \citet{parameter_growth}. This indicates that the norm of Transformers' parameters will increase even under weight decay and normalization. Since gradients are obtained by multiplying parameters and hidden features, $\norm{g^l_K}_2^2$ is also likely to increase. We empirically examine it in \cref{sec:t_exp:spectral_increase} where $\trace{M^l_i} \defeq \trace{g_{K, i}^l \left(g_{K, i}^l\right)^\transpose}= \norm{g_{K, i}^l}_2^2$ is observed to increase, at least in ViTs.
For the second possibility and the disconnection between $L_2$ and $L_0$ norm, under $\relu$-activation, similarly to \cref{remark:L2_and_L0}, $\norm{\eta^l}_2 = \norm{g^l \hadamard \gamma^l}_2$ can be seen as the $L_0$ norm of activations, but weighted by entries in $g^l_K$. In \cref{sec:t_exp:align_eff}, we will empirically demonstrate that $\gamma^l$ aligns with $g^l_K$ very well, i.e., the distribution of squared values of entries in $g^l_K$ that corresponds to non-zero entries in $\gamma^l$ is similar or even righter-shifted compared to the distribution of all entries' squared values in $g^l_K$, and the former avoids the long tail of the latter with small magnitudes. This indicates that it is not the case that effective gradient sparsity measured in $L_2$ norms is achieved by adversarially aligning non-zero derivatives of activation functions to small gradients and aligning zero derivatives to large gradients. Therefore, the weighting by $g^l$ to the $L_0$ norm is quite moderate, and a considerable portion of $0$-$1$ entries in $\gamma^l$ are attached to large weights that can be approximated by $\norm{g^l_K}_2^2 / d$ (which is increasing under parameter growth), so at least this portion of entries enjoy the approximate connection from effective gradient sparsity to gradient sparsity, and finally to activation sparsity. This intuition leads to \cref{lemma:eff_and_sparsity} that fully exploits coincidences in $\relu$ and the piecewise constancy of $L_0$ norm.
\begin{lemma}[Relation between $\eta^l$ and $\gamma^l$ for $\relu$ networks]\label{lemma:eff_and_sparsity}
    Let $\set{\gamma^l_i}_{i \in \set{1, \dots, n}} \subseteq \set{0, 1}^d$ be a set of $n$ $0$-$1$ vectors  and $\set{g^l_{K, i}}_{i \in \set{1, \dots, n}} \subseteq \reals^d$ be a set of $n$ real vectors. Let $\eta^l_i \defeq g^l_{K, i} \hadamard \gamma^l_i$, resembling \cref{def:effective_gradient_sparsity}.
    For any distribution $D$ over subscripts $i \in \set{1, \dots, n}$, there is
    \begin{align}
        &   \ex[i \sim D]{\norm{\eta^l_i}_2^2}\\
        =&  \ex[i \sim D]{d \cdot \ex[j \sim U[1, \dots, d]]{\left(g^l_{K, i, j}\right)^2 \left(\gamma^l_{i, j}\right)^2}}
        =   d \cdot \ex[i, j]{\left(g^l_{K, i, j}\right)^2 \gamma^l_{i, j}}\\
        =&  d \cdot \prob{\gamma^l_{i, j} = 1} \cdot 1 \cdot \ex{\left(g^l_{K, i, j}\right)^2 \mid \gamma^l_{i, j} = 1}
            + d \cdot \prob{\gamma^l_{i, j} = 0} \cdot 0 \cdot \ex{\left(g^l_{K, i, j}\right)^2 \mid \gamma^l_{i, j} = 0}\\
        =&  d \cdot \prob{\gamma^l_{i, j} = 1} \cdot 1 \cdot \ex{\left(g^l_{K, i, j}\right)^2 \mid \gamma^l_{i, j} = 1}
            + d \cdot \prob{\gamma^l_{i, j} = 0} \cdot 0 \cdot \ex{\left(g^l_{K, i, j}\right)^2 \mid \gamma^l_{i, j} = 1}\\
        =&  d \cdot \ex{\left(g^l_{K, i, j}\right)^2 \mid \gamma^l_{i, j} = 1} \cdot \left(\prob{\gamma^l_{i, j} = 1} \cdot 1 + \prob{\gamma^l_{i, j} = 0} \cdot 0\right)\\
        =&  \ex[i \sim D, j \sim U[1,\dots, d]]{\left(g^l_{K, i, j}\right)^2 \mid \gamma^l_{i, j} = 1} \cdot \ex[i \sim D]{\norm{\gamma^l_i}_0},
    \end{align}
    where $U[1, \dots, d]$ stands for uniform distribution among $\set{1,2, \dots, d}$, $g^l_{K, i, j}$ and $\gamma^l_{i, j}$ stand for the $j$-th entry of the $i$-th vectors.
\end{lemma}
In \cref{sec:t_exp:align_eff} we will demonstrate that $\ex{\left(g^l_{K, i, j}\right)^2 \mid \gamma^l_{i, j} = 1}$ is comparable to or even larger than $\ex{\left(g^l_{K, i, j}\right)^2}$ that is increasing due to parameter growth.
For other activation functions with jump discontinuity between deactivation and activation like $\jrelu$, the alignment is also moderate and the weighted $L_2$ norm first pushes activations towards zero, after which $L_2$ norms become closer to $L_0$ norm due to derivatives' jump discontinuity at $0$. Following a similar argument of \cref{lemma:eff_and_sparsity}, effective gradient sparsity then approximates sparsity measured in $L_0$ norms as well.
Another very informal and heuristic argument for the alignment as well as $\eta^l$'s connection to activation and gradient sparsity is that, if the $i$-th entry in $\eta^l$ is near zero, then during the row memorization described by \cref{obs:memorizing_inputs}, little change is imposed by $x^{l-1}$ to the $i$-th row of $K^l$. Next time $x^{l-1}$ arrives, although with changes due to shallower layers, it is more likely the $i$-th row in $K^l$ forgets $x^{l-1}$ and $\left(K^l x^{l-1}\right)_i = \inner{K^l_i}{x^{l-1}}$ or $\left(K^l \left(x^{l-1} + d^l\right)\right)_i = \inner{K^l_i}{x^{l-1} + d^l}$ are closer to $0$ or negative values, leading to smaller possibility of activation, followed by activation sparsity in $\alpha^l$, and thus gradient sparsity in $\gamma^l$ under common activation functions.
To sum up, effective gradient sparsity measured by $L_2$ norm can be connected to activation or gradient sparsity measured in $L_0$ norms, although the result requires empirical assumptions and is heuristic and informal for activation functions other than $\relu$. Therefore, we will base our theorems on $\norm{\eta^l}_2^2$ considering the connection discussed above and the mathematical convenience brought by $L_2$ norm and the Hadamard product.

\subsection{Flat Minima, Implicit Adversarial Robustness and Effective Gradient Sparsity}\label{sec:three_elements}

\NewDocumentCommand{\ce}{}{\loss_{\mathrm{CE}}}
As discussed in \cref{sec:illustration}, we start from $\trace{\hessian}$ and relate it to implicit adversarial robustness w.r.t. some hidden feature $x^l$, or $\ex[(X, Y)]{\norm{\derivatives{\loss}{x^l}}_2^2}$. 
Cross Entropy loss is assumed under classification tasks. Note that aside from explicit classification tasks, next-token classification and Cross Entropy loss form the basis for many self-supervision objectives in NLP pretraining, including causal language modeling and masking language modeling. Therefore this assumption can be applied across broad practical scenarios. Assuming $\loss = \ce = -\log f(y \mid \theta, x)$, the Hessian writes
\begin{align}
    \nabla_\theta^2 \ex[(X, Y)]{-\log f(Y \mid \theta, X)}
    =& -\ex[(X, Y)]{\nabla_\theta^2 \log f(Y \mid \theta, X)},
\end{align}
which, together with $\ex[(X, Y)]{\norm{\derivatives{\loss}{x^l}}_2^2}$, reminds us of the famous equality of Fisher's Information Matrix, i.e.,
\begin{align}
    \mathcal{I}(\theta)
    = -\ex[X]{\nabla_\theta^2 \log g_\theta(X)}
    = \ex[X]{\left(\derivatives{g_\theta(X)}{\theta}\right)\left(\derivatives{g_\theta(X)}{\theta}\right)^\transpose},
\end{align}
the trace of RHS of which is exactly the expected squared norm of the gradients. So adapting the classic proof of this equality, we connect flatness measured by Hessian trace to the norm of gradients in \cref{lemma:flatness_and_grad_norm}.
\begin{lemma}[Flatness and samplewise gradient norm]
    \label{lemma:flatness_and_grad_norm}
    Assume $f_\theta$ is a neural network parameterized by $\theta$, trained by Cross Entropy loss $\ce$. Given $\hessian$ being the Hessian matrix of loss at $\theta$, there is
    \begin{align}
        \trace{\hessian}
        =   \ex[(X, Y) \sim \dataset]{\norm{\nabla_\theta \ce(\theta, (X, Y))}_2^2} - \ex[(X, Y) \sim \dataset]{\frac{\trace{\nabla^2_\theta f(Y \mid \theta, X)}}{f(Y \mid \theta, X)}}.
    \end{align}
    Further for well learned models, i.e., $f_\theta(Y \mid X) \approx \dataset(Y \mid X)$ for all training samples, there is
    \begin{align}
        \trace{\hessian}
        \approx   \ex[(X, Y) \sim \dataset]{\norm{\nabla_\theta \ce(\theta, (X, Y))}_2^2}.
    \end{align}
\end{lemma} 

The proof of it can be found in \cref{proof:flatness_and_grad_norm}. This lemma invokes a samplewise point of view on Hessian and flatness, which is closer to adversarial attacks because they are added in a samplewise manner. Aside from implication in the context of sparsity, \cref{lemma:flatness_and_grad_norm} indicates that a flat minimum also provides solutions that are locally good for most individual samples. 
We acknowledge that similar results with gradient outer products under abstract losses have been given by \citet{empirical_hessian}, but we believe under Cross Entropy loss the result becomes more direct, intriguing and implicative. 
As an aside, Cross Entropy seems an interesting loss function, for example, \citet{ib_disentangle} rewrite CE loss expected among all training traces into mutual information between training data and final parameters.

After that, we move perturbations from parameters to hidden features in \cref{lemma:grad_norm_and_adversarial}.
\begin{lemma}[Gradient norm and implicit adversarial robustness]\label{lemma:grad_norm_and_adversarial}
    Let $f_\theta$ be a neural network parameterized by $\theta$, and the parameters for the $l$-th layer is $\theta_l$. Let $\dbmlp^l$ be the $l$-th doubly biased MLP in $f_\theta$ whose input is $X^{l-1}$, zeroth bias is $D^l$. Then there is
    \begin{align}
        \norm{\nabla_{\theta_l} \ce(\theta, (x, y))}_2^2 \ge \norm{\nabla_{X^{l-1}} \ce(\theta, (x, y))}_2^2 + \norm{\HXT}_2^2 + \norm{\GAlphaT}_2^2.
    \end{align}
    If $\mlp^l$ is not doubly biased, then the first term in RHS simply disappears.
\end{lemma}
\begin{proof}
    By noticing $\theta_l$ contains at least $K^l, V^l$ and $D^l$, there is
    \begin{align}
        \norm{\nabla_{\theta_l} \ce}_2^2 \ge \norm{\nabla_{D^l} \ce}_2^2 + \norm{\nabla_{K^l} \ce}_2^2 + \norm{\nabla_{V^l} \ce}_2^2.
    \end{align}
    Consider how $X^{l-1}$ is processed in $\dbmlp^l$: It is added with $D^l$ before any other operations. So there is $\nabla_{D^l} \ce = \nabla_{X^{l-1}} \ce = \nabla_{X^{l-1} + D^l} \ce$. 
    Combining \cref{lemma:gradients_with_eff} for the second and the third terms, the lemma follows.
\end{proof}

From the proof of \cref{lemma:grad_norm_and_adversarial} we can see that zeroth biases avoid tedious linear or non-linear operations and drastically ease our analysis. This implies that one can design theoretically oriented architecture that allows easier theoretical analyses.

We then connect implicit adversarial robustness to gradient sparsity.
\begin{lemma}[Implicit adversarial robustness and gradient sparsity]\label{lemma:adversarial_and_sparsity}
    Under the same condition of \cref{lemma:grad_norm_and_adversarial}, together with the assumption that in $\dbmlp^l$ the weight matrix is $K^l \in \reals^{n \times d}$ and $\gamma^l$ is the entrywise derivatives of activations, there is
    \begin{align}
        \norm{\nabla_{x^{l-1}} \ce(\theta, (x, y))}_2^2 = \left(\gamma^l\right)^\transpose \left(\left(K^l (K^l)^\transpose\right) \hadamard M^l\right) \gamma^l = \left(\eta^l\right)^\transpose \kkT \eta^l,
    \end{align}
    if hidden features are single tokens, where $M^l \defeq g^l_K \left(g^l_K\right)^\transpose$ is a symmetric positive semi-definite matrix of rank at most $1$, $\hadamard$ denotes Hadamard product, i.e., entrywise product.
    
    If hidden features are matrices, then there is
    \begin{align}
        \norm{\nabla_{X^{l-1}} \ce(\theta, (x, y))}_2^2 = \trace{\Eta^l \left(\Eta^l\right)^\transpose \kkT}.
    \end{align}
\end{lemma}
\begin{proof}
    By \cref{lemma:gradients_with_eff}, the gradients w.r.t. $d^l$ is 
    \begin{align}
        \derivatives{\ce(\theta, (x, y))}{d^l} = \left(K^l\right)^\transpose \eta^l.
    \end{align}
    Similar to the proof of \cref{lemma:grad_norm_and_adversarial}, $x^{l-1}$ and $d^l$ share the same gradient, so
    \begin{align}
        \nabla_{x^{l-1}} \ce = \derivatives{\ce(\theta, (x, y))}{x^{l-1}} = \left(K^l\right)^\transpose \eta^l.
    \end{align}

    Now we can compute the squared norm of gradients by its relation with trace
    \begin{align}
        \norm{\nabla_{x^{l-1}} \ce}_2^2
        =& \trace{\left(\nabla_{x^{l-1}} \ce\right)^\transpose \nabla_{x^{l-1}} \ce}
        =  \trace{\left(\eta^l\right)^\transpose \kkT \eta^l}\\
        =&  \left(\eta^l\right)^\transpose \kkT \eta^l.
    \end{align}
    To see how $M^l \defeq g^l_{K} \left(g^l_K\right)^\transpose$ emerges, expand the definition of $\eta^l$ and obtain
    \begin{align}
        \norm{\nabla_{x^{l-1}} \ce}_2^2
        =&  \left(\eta^l\right)^\transpose \kkT \eta^l
        =   \left(\diag{g^l_{K}} \gamma^l\right)^\transpose \kkT \left(\diag{g^l_K} \gamma^l\right)\\
        =&  \left(\gamma^l\right)^\transpose \diag{g^l_K} \kkT \diag{g^l_k} \gamma^l\\
        =&  \left(\gamma^l\right)^\transpose \left(g^l_K \left(g^l_K\right)^\transpose \hadamard \kkT \right) \gamma^l
        =  \left(\gamma^l\right)^\transpose \left(M \hadamard \kkT \right) \gamma^l,
    \end{align}
    where the second last step is to apply \cref{eq:hadamard_and_diagonal}.
    Note that $M^l$'s rank is at most $1$.

    When hidden features are matrices, $\norm{\nabla_{X^{l-1}} \ce}_2^2$ sums the gradient norms for all tokens, which leads to
    \begin{align}
        \norm{\nabla_{X^{l-1}}\ce}_2^2
        =&  \sum_i \norm{\nabla_{x^{l-1}_i} \ce}_2^2
        =  \trace{\sum_i \left(\eta^l_i\right)^\transpose \kkT \eta^l_i}\\
        =&  \trace{\left(\sum_i \eta^l_i \left(\eta^l_i\right)^\transpose\right) \kkT}
        =  \trace{\Eta^l \left(\Eta^l\right)^\transpose \kkT}.
    \end{align}
\end{proof}

Combining \cref{lemma:flatness_and_grad_norm}, \cref{lemma:grad_norm_and_adversarial} and \cref{lemma:adversarial_and_sparsity} together, we have the main theorem.
\begin{theorem}[Flatness, implicit adversarial robustness and sparsity]\label{theorem:main}
    Let $f_\theta$ be a well-learned neural network parameterized by $\theta$, trained under Cross Entropy loss $\ce$. Let $\hessian$ be the Hessian matrix w.r.t. parameters at $\theta$. Let $\dbmlp^l$ be the $l$-th doubly biased MLP in $f_\theta$ whose input is $x^{l-1}$. There is
    \begin{align}
        &\trace{\hessian}\\
        \ge& \sum_{l} \left(
            \ex{\norm{\nabla_{X^{l-1}} \ce(\theta, (X, Y))}_2^2} 
            +   \ex{\norm{\HXT}_2^2} + \ex{\norm{\GAlphaT}_2^2}\right)\\
        =& \sum_{i, l} \ex{\left(\eta^l_i\right)^\transpose \kkT \eta^l_i} + \sum_{l} \ex{\norm{\HXT}_2^2} + \sum_l \ex{\norm{\GAlphaT}_2^2}. \label{eq:main}
    \end{align}
    The first term can also be expressed by $\left(\gamma^l_i\right)^\transpose \left(\kkT \hadamard M^l_i\right) \gamma^l_i$, where $M^l$ is a symmetric positive semi-definite matrix of rank at most $1$.
    Further by Schur's Theorem, $\left(K^l \left(K^l\right)^\transpose\right) \hadamard M^l_i$ is also positive semi-definite.

    If vanilla $\mlp$s are used, then the first term in RHS simply disappears.
\end{theorem}


The chained upperbounds connect flatness and implicit adversarial robustness to effective gradient sparsity (the first two terms in Equation \ref{eq:main}) and as well as activation sparsity (the last term in Equation \ref{eq:main}), indicating that both gradient and activation sparsity can be sources of implicit adversarial robustness and flatness. If flatness is achieved, then it is possibly done through (effective) gradient sparsity and activation sparsity. Note that this bound is very tight because $\mlp$s take a large portion of parameters \citep{knowledge_neurons} even in Transformers, so by Cauchy's Interlace Theorem most large eigenvalues of $\hessian$ are retained in the submatrix of $\mlp$ parameters. Therefore to achieve flatness, the terms in \cref{eq:main} must be suppressed.

\subsection{Discussions on \cref{theorem:main}}\label{sec:discussion}

In this section, we discuss the implications of \cref{theorem:main} under several particular settings, including pure MLPs, pure LayerNorm-ed MLPs, Transformers and Transformers with hypothetical massive perturbation training. 
We point out their tendency toward effective gradient sparsity, which leads to gradient and activation sparsity as discussed in \cref{sec:gradients}, among which effective gradient sparsity is more stable.

\subsubsection{Pure MLPs}

The last two terms in \cref{eq:main} have similar forms, so we inspect them together. To have a clearer understanding on them, first consider the situations where models use single-token hidden features in \cref{corollary:main_with_hidden_vectors}
\begin{corollary}[Flatness and sparsity in pure MLPs]\label{corollary:main_with_hidden_vectors}.
    Inherit the assumptions of \cref{theorem:main}. Assume additionally that the model uses hidden features of single tokens, then there is
    \begin{align}
        \trace{\hessian} 
        \ge& \sum_{l} \ex{\left(\eta^l\right)^\transpose \kkT \eta^l} + \sum_{l} \ex{\norm{x^{l-1} + d^l}_2^2 \norm{\eta^l}_2^2} + \sum_l \ex{\norm{g^l_V}_2^2 \norm{\alpha^l}_2^2} \label{eq:main_with_hidden_vectors}.
    \end{align}
\end{corollary}
\begin{proof}
    With single-token hidden features, $\norm{\HXT}_2^2$ reduces to 
    \begin{align}
        &   \norm{\HXT}_2^2
        =   \norm{\eta^l \left(x^{l-1} + d^l\right)^\transpose}_2^2
        =   \trace{\left(\eta^l \left(x^{l-1} + d^l\right)^\transpose\right)^\transpose \eta^l \left(x^{l-1} + d^l\right)^\transpose}\\
        =&  \trace{\left(x^{l-1} + d^l\right) \left(\eta^l\right)^\transpose \eta^l \left(x^{l-1} + d^l\right)^\transpose}
        =  \trace{\left(\eta^l\right)^\transpose \eta^l \left(x^{l-1} + d^l\right)^\transpose \left(x^{l-1} + d^l\right)}\\
        =& \norm{\eta^l}_2^2 \norm{x^{l-1} + d^l}_2^2.
    \end{align}
    $\norm{\GAlphaT}_2^2$ can be reduced similarly.
\end{proof}
If normalization layers are imposed, for example, LayerNorm layers before MLP blocks, then $\norm{x^{l-1}}_2^2 = d$ will not change during training, eliminating all other sources of suppressing the second term aside from effective gradient sparsity. 
Since key matrices also take a large portion of parameters, flatness in these parameters must be achieved as well and $\trace{\hessian[\theta_K]}$ is not too small from $\trace{\hessian}$, the second term alone will have a strong tendency to decrease. Therefore, a rigorously proved, strong and stable tendency of pure LayerNorm-ed MLPs toward effective gradient sparsity is presented in \cref{theorem:main_with_hidden_vectors_and_layernorm}.
\begin{theorem}[Flatness and sparsity in pure MLPs with LayerNorms]\label{theorem:main_with_hidden_vectors_and_layernorm}
    Inherit the assumptions of \cref{theorem:main}. Assume additionally that the model uses vector hidden features and LayerNorm layers, with affine transformation turned off, are imposed before every MLP block. Temporarily assume non-$\dbmlp$ models are used, then there is
    \begin{align}
        \trace{\hessian} 
        \ge& d \sum_{l} \ex{\norm{\eta^l}_2^2} + \sum_l \ex{\norm{g^l_V}_2^2 \norm{\alpha^l}_2^2}\label{eq:main_with_hidden_vectors_and_layernorm}.
    \end{align}

    If $\dbmlp$s are used and LayerNorm layers are placed \emph{before} zeroth biases, by clipping the norm of columns in zeroth biases to $c$, there will be
    \begin{align}
        \trace{\hessian} 
        \ge& \sum_{l} \ex{\left(\eta^l\right)^\transpose \kkT \eta^l} + \left(\sqrt{d} - c\right)^2 \sum_{l} \ex{\norm{\eta^l}_2^2} + \sum_l \ex{\norm{g^l_V}_2^2 \norm{\alpha^l}_2^2} \label{eq:main_with_hidden_vectors_and_layernorm_dbmlp}.
    \end{align}
    By \cref{lemma:eff_and_sparsity}, for $\relu$ networks, there is further
    \begin{align}
        \trace{\hessian} 
        \ge& \sum_{l} \ex{\left(\eta^l\right)^\transpose \kkT \eta^l} \\&+ \left(\sqrt{d} - c\right)^2 \cdot \ex[X, l, i]{\left(g^l_{K, i}\right)^2 \mid \alpha^l_i > 0} \cdot \sum_{l} \ex{\norm{\alpha^l}_0} + \sum_l \ex{\norm{g^l_V}_2^2 \norm{\alpha^l}_2^2}
    \end{align}
    for $\dbmlp$ networks, and similar results can be obtained for architectures without zeroth biases.
\end{theorem}
LayerNorm places quite strong and stable drives towards effective gradient sparsity, if they are placed right before (DB-)MLP blocks and affine factors are turned off to avoid their reduction due to updates or weight decay. \cref{theorem:main_with_hidden_vectors_and_layernorm} can be one explanation of the benefits of LayerNorms and the practice to exclude their parameters from weight decay.

The last term in \cref{eq:main_with_hidden_vectors} relating activation sparsity is less ensured than the second term. In experiments (although conducted with Transformers) we observe $\norm{g_V^l}_2^2$ becomes small in deep layers, indicating that effective gradient sparsity is the main cause of activation sparsity in deep layers.

\subsubsection{Transformers and Other Architectures}

When stacked hidden features are used, for example in Transformers, the discussion is more tricky. It is possible that gradients of different tokens cancel each other in 
\begin{align}
    \HXT = \sum_{i} \eta^l_i \left(x^{l-1}_i + d^l_i\right)^\transpose.
\end{align}
We can only rigorously conclude a possibly loose lowerbound with $\norm{\eta^l_i}_2^2$ in \cref{corollary:main_with_minimum_eigenvalue} using \cref{eq:trace_product_and_eigenvalue}.
\begin{corollary}[Flatness and sparsity in Transformers]\label{corollary:main_with_minimum_eigenvalue}
    Inherit assumptions from \cref{theorem:main}, then there is
    \NewDocumentCommand{\mineigenbound}{m m}{\sum_{l} \ex{\lambda_{\min}\left(\left(#1\right)^\transpose #1\right) \sum_i \norm{#2}_2^2}}
    \begin{align}
        \trace{\hessian} 
        \ge&
            \sum_{i, l} \ex{\left(\eta^l_i\right)^\transpose \kkT \eta^l_i}\\
            &+ \mineigenbound{\left(X^{l-1} + D^l\right)}{\eta^l_i}\\
            &+ \mineigenbound{G^{l}_V}{\alpha^l_i},
    \end{align}
    where $\lambda_{\min}\left(\cdot\right)$ indicates the minimum eigenvalue of a matrix.
\end{corollary}
\arxivonly{\begin{proof}
    \begin{align}
        \HXT 
        =&  \trace{\left(\HXT\right)^\transpose \HXT}\\
        =&  \trace{\left(X^{l-1}\right)^\transpose X^{l-1} \left(H^l\right)^\transpose H^l}.
    \end{align}
    The last term can be similarly rearranged. Applying \cref{eq:trace_product_and_eigenvalue} to both of them and noticing $\trace{\left(H^l\right)^\transpose H^l} = \sum_i \norm{\eta^l_i}_2^2$ finish the proof.
\end{proof}}

There are tricky ways to bypass the canceling, however. For example, consider augments conducted on hidden features such as dropout. 
They effectively duplicate the parameter into $k$ views and perturb each view independently (using dropout, rows of weight matrices are randomly pruned) if there are $k$ tokens. 
If flatness can be extended to these effective duplicated parameters, i.e., if there is still flatness when we really duplicate the weight matrices and assign one token for each matrix, then each view is only handling one token and we can repeat \cref{corollary:main_with_hidden_vectors}. 
However, traditional dropout may hinder the sparsity by eliminating activations and forcing the model to back up the representation. Additionally, its perturbations are not local enough, hindering theoretical analyses. 
A soft dropout by slightly perturbing before activation functions is more preferable. 
Moreover, the perturbation should better be conducted on weight matrices in an entrywise manner to avoid summing and canceling gradients.
Under this hypothetical synapse perturbation \citep{synaptic_noise_1,synaptic_noise_2} but in a tokenwise manner, we assume flatness can be obtained w.r.t. the duplicated parameters because, in the real model, losses are suppressed even under independent perturbation so the effective model is not sensitive to independent changes in individual parameter duplicates. This intuition leads to \cref{lemma:flatness_of_perturbed_model}.

\NewDocumentCommand{\proxytheta}{}{\hat{\theta}}
\NewDocumentCommand{\dupW}{}{\tilde{W}}
\begin{lemma}[Flatnesses of perturbed model and perturbation-induced effective model]\label{lemma:flatness_of_perturbed_model}
    Assume weight matrices are perturbed by Gaussian noise \emph{independently} for each token, i.e., the perturbed $\mlp_*^l$ outputs
    \begin{align}
        \activation\left(\begin{bmatrix}
            \proxyW^{l, 1} x_1 &   \cdots & \proxyW^{l, i} x_i & \cdots & \proxyW^{l, k} x_k
        \end{bmatrix} + b^l\right) \label{eq:massive_perturbation}
    \end{align}
    for input hidden matrix $X = \begin{bmatrix} x_1 & \cdots & x_i \cdots & x_k \end{bmatrix}$, where $\proxyW^{l, i} \defeq W^l + \Epsilon^{l, i}$ , and $\Epsilon^{l, 1}, \dots, \Epsilon^{l, k}$ are $k \times n \times d$ independent centered Gaussian variables with variance $\sigma^2$. 
    Let random variable $\Epsilon$ denote the collection of all perturbations.
    Let $\proxytheta$ be the collection of proxied parameters, where $\proxyW^{l, i}$s are taken into consideration instead of $W^{l}$, while other parameters are inherited from $\theta$.

    Let $g_{\efftheta}$ be the effective parameter by duplicating each weight matrix $W^l$ for $k$ times into $\dupW^{l, 1}, \dots, \dupW^{l, k}$, each of which deals with exactly one hidden vector $x_i$ during inference.

    Then
    \begin{align}
        &   \frac{1}{\sigma^2} \ex[(X, Y), E]{\left(\loss(f_\theta, E, (X, Y)) - \loss(f_\theta, 0, (X, Y))\right)^2} + n d L k \cdot o\left(1\right)\\
        %=& \ex{\norm{\derivatives{\loss(f_\theta, 0, (X, Y))}{\proxytheta}}_2^2}
        %= \ex{\norm{\derivatives{\loss(g_{\efftheta}, (X, Y))}{\efftheta}}_2^2}\\
        =& \sum_{l} \sum_{W \in \set{K, V}} \sum_{i}  \ex{\norm{\derivatives{\loss(f_\theta, 0, (X, Y))}{\proxyW^{l, i}}}_2^2}
        = \sum_{l} \sum_{W \in \set{K, V}} \sum_{i} \ex{\norm{\derivatives{\loss(g_{\efftheta}, (X, Y))}{\dupW^{l, i}}}_2^2},
    \end{align}
    where $\loss(f_\theta, E, (X, Y))$ indicates the loss of $f_\theta$ on sample $(X, Y)$ when perturbation is $E$, $L$ is the number of $\mlp$ layers.
    If Cross Entropy loss is assumed and $f_\theta$ is well learned when perturbations are removed then by \cref{lemma:flatness_and_grad_norm} applied to $g_{\efftheta}$,
    \begin{align}
        &   \frac{1}{\sigma^2} \ex[(X, Y), E]{\left(\ce(f_\theta, E, (X, Y)) - \ce(f_\theta, 0, (X, Y))\right)^2} + n d L k \cdot o\left(1\right)\\
        =& \sum_{l} \sum_{W \in \set{K, V}} \sum_{i}  \ex{\norm{\derivatives{\ce(f_\theta, 0, (X, Y))}{\proxyW^{l, i}}}_2^2}
        = \sum_{l} \sum_{W \in \set{K, V}} \sum_{i} \ex{\norm{\derivatives{\ce(g_{\efftheta}, (X, Y))}{\dupW^{l, i}}}_2^2}\\
        \le& \ex{\norm{\derivatives{\ce(f_\theta, 0, (X, Y))}{\proxytheta}}_2^2} 
        = \ex{\norm{\derivatives{\ce(g_{\efftheta}, (X, Y))}{\efftheta}}_2^2} \approx \trace{\hessian[\efftheta]}\\
    \end{align}
\end{lemma}
\begin{proof}
    By construction of $g_{\efftheta}$, gradients w.r.t. $\proxyW^{l, i}$ and $\dupW^{l, i}$ share the same path in $f_\theta$ and $g_{\efftheta}$. If the same sample is used and the perturbation is removed, then they are equal.


    We can approximate $\loss(f_\theta, \Epsilon, (x, y))$ from $\loss(f_\theta, 0, (x, y))$ by
    \begin{align}
        \left(\loss(f_\theta, \Epsilon, (x, y)) - \loss(f_\theta, 0, (x, y))\right)^2
        =&  \left(\left(\nabla_{\vectorize{\Epsilon}} \loss(f_\theta, 0, (x, y)) \right)^\transpose \vectorize{E} +  o\left(\norm{\vectorize{E}}_2 \right)\right)^2\\
        =&  \left(\left(\nabla_{\vectorize{\Epsilon}} \loss(f_\theta, 0, (x, y)) \right)^\transpose \vectorize{E}\right)^2 +o\left(\norm{\vectorize{E}}_2^2 \right)\\
    \end{align}
    Since Gaussian $\vectorize{E}$ has covariance $\sigma^2 I$, $\nabla_{\vectorize{\Epsilon}}^\transpose \loss \times  \vectorize{E}$ is also Gaussian whose variance is 
    \begin{align}
        \sigma^2 \left(\nabla_{\vectorize{\Epsilon}} \loss(f_\theta, 0, (x, y))\right)^\transpose \left(\nabla_{\vectorize{\Epsilon}} \loss(f_\theta, 0, (x, y))\right) = \sigma^2 \norm{\nabla_{\vectorize{\Epsilon}} \loss}_2^2.    
    \end{align}
    Taking expectation over noises $E$, there is
    \begin{align}
            \ex[E]{\left(\loss(f_\theta, \Epsilon, (x, y)) - \loss(f_\theta, 0, (x, y))\right)^2}
        =& \sigma^2 \norm{\nabla_{\vectorize{\Epsilon}} \loss(f_\theta, 0, (X, Y))}_2^2 + o(n d L k \sigma^2)\\
        =&  \sigma^2 \norm{\nabla_{\vectorize{\proxyW}} \loss(f_\theta, 0, (X, Y))}_2^2 + o(n d L k \sigma^2),
    \end{align}
    where $\dupW$ denote the collection of all $\dupW^{l, i}$s. 
    The rest of the proof is easy according to the equivalence between $f_\theta$ with perturbations removed and $g_{\efftheta}$.
\end{proof}
\begin{remark}
    Although $\trace{\hessian[\efftheta]}$ is involved by an inequality, considering the large portion of $\dupW$ parameters, $\sum_{l} \sum_{W \in \set{K, V}} \sum_{i} \ex{\norm{\derivatives{\ce(g_{\efftheta}, (X, Y))}{\dupW^{l, i}}}_2^2}$ can in fact represent $\trace{\hessian[\efftheta]}$ well. If this argument is not satisfying, then perturb all parameters in the same way so that ``$\le$'' becomes ``$=$''.
\end{remark}

So by training a weight-perturbed non-pure-MLP network to have low losses, we are helping its pure-MLP equivalence reaching flat minima, where effective gradient sparsity can be directly obtained in \cref{theorem:main_with_effective_duplication}. If we assume 
\begin{align}
    \frac{1}{\sigma^2} \ex[(X, Y), E]{\left(\ce(f_\theta, E, (X, Y)) - \ce(f_\theta, 0, (X, Y))\right)^2} \le \trace{\hessian[\efftheta]}
\end{align}
is indeed suppressed during training because losses are suppressed to near-zero values, then \cref{theorem:main_with_effective_duplication} is meaningful.

\begin{theorem}[Flatness and sparsity under tokenwise synapse noise perturbations]\label{theorem:main_with_effective_duplication}
    Inherit the assumptions of \cref{theorem:main} as well as notations in \cref{lemma:flatness_of_perturbed_model}. Further assume that weight matrices are independently perturbed before multiplying with any individual tokens during training, then
    \begin{align}
        &   \frac{1}{\sigma^2} \ex[(X, Y), E]{\left(\ce(f_\theta, E, (X, Y)) - \ce(f_\theta, 0, (X, Y))\right)^2} + n d L k \cdot o\left(1\right) \\& + \sum_{i, l} \ex{\left(\eta^l_i\right)^\transpose \kkT \eta^l_i}\\
        =& \sum_{i, l} \ex{\left(\eta^l_i\right)^\transpose \kkT \eta^l_i} + \sum_{i, l} \ex{\norm{x^{l-1}_i + d^l_i}_2^2 \norm{\eta^l_i}_2^2} + \sum_{i, l} \ex{\norm{g^l_{V, i}}_2^2 \norm{\alpha^l_i}_2^2} \\
        \le&    \trace{\hessian[\tilde{\theta}]} + \trace{\hessian[\theta_D]}
        \label{eq:main_with_effective_duplication},
    \end{align}
    where $\tilde{\theta}$ stands for the perturbation-induced effective parameter where weight matrices are really duplicated so that each of them serves one token and $\theta_D$ is the collection of parameters in all zeroth biases.
    If no-affine LayerNorms are applied, there is further
    \begin{align}
        &   \frac{1}{\sigma^2} \ex[(X, Y), E]{\left(\ce(f_\theta, E, (X, Y)) - \ce(f_\theta, 0, (X, Y))\right)^2} + n d L k \cdot o\left(1\right)\\
        \\& + \sum_{i, l} \ex{\left(\eta^l_i\right)^\transpose \kkT \eta^l_i}\\
        =&  \sum_{i, l} \ex{\left(\eta^l_i\right)^\transpose \kkT \eta^l_i} + \left(\sqrt{d} - c\right)^2 \sum_{i, l} \ex{\norm{\eta^l_i}_2^2} + \sum_{i, l} \ex{\norm{g^l_{V, i}}_2^2 \norm{\alpha^l_i}_2^2}\\
        \le&   \trace{\hessian[\tilde{\theta}]} + \trace{\hessian[\theta_D]},\\
    \end{align}
    where $c=0$ if non-$\dbmlp$s are used, otherwise $c$ is the norm bound of columns in zeroth biases.
    
    By \cref{lemma:eff_and_sparsity}, for $\relu$ networks, $\left(\sqrt{d} - c\right)^2 \sum_{i, l} \ex{\norm{\eta^l_i}_2^2}$ terms in the above equations can be replaced by $\left(\sqrt{d} - c\right)^2 \cdot \ex[X, l, i, j]{\left(g^l_{K, i, j}\right)^2 \mid \alpha^l_{i, j} > 0} \cdot \sum_{i, l} \ex{\norm{\alpha^l_i}_0}$ to have a direct relation with activation sparsity.
\end{theorem}
Aside from Transformers, tokenwisely perturbed CNNs, channel mixing layers of MLP-Mixers and other potential architectures apply, as long as they have MLP blocks and the perturbed loss is small enough.
Additionally, this bound is also very tight by the tightness of \cref{lemma:flatness_of_perturbed_model} or by counting parameters, so perturbed error's reduction or the flatness of the effective model inevitably leads to a reduction in sparsity.
\arxivonly{A simple algorithm $\magic$ is immediate after \cref{lemma:flatness_of_perturbed_model} and \cref{theorem:main_with_effective_duplication}, which is listed in \cref{appendix:magic} in order not to disrupt the presentation of major theoretical results.}

\subsubsection{The First Term from Zeroth Biases}

Now we look back to the first term in \cref{eq:main}. Although the first term seems to have minor weight compared to others, either by counting parameters or by decomposing eigenvalues of $\kkT$ (see \cref{fig:eigenvalues_of_kkT}), an investigation is still worthy since it leads to another phenomenon of spectral concentration in $\kkT$ and introduces random matrix theory to reason about training dynamics. 

In the first term, since the gradient of the inner product w.r.t. $\eta^l$ is 
\begin{align}
    2 \kkT \eta^l,
\end{align}
there is an \emph{overall} positive tendency toward sparsity if $\trace{\hessian}$ is suppressed, because $\left(\eta^l\right)^\transpose 2 \kkT \eta^l$ is non-negative, i.e., partial derivatives w.r.t. $\eta^l$ are always overall positive if they are weighted by values in $\eta^l$ themselves. 

It is better to reach a non-overall conclusion. Moreover, there are two possibilities that can also achieve implicit adversarial robustness: reducing the norm of $\kkT$, or misaligning the non-null space of $\kkT$ with $\eta^l$. 
The first alternative can already be eliminated by parameter growth observed by \citet{parameter_growth}, where Transformers are observed to have increasing parameter norms during training. We also empirically verify this phenomenon under CV and NLP settings and show that the trace, or the sum of eigenvalues, will not decrease drastically during training in \cref{sec:t_exp:spectral_increase}, even under weight decay. 
Another possible elimination is normalization layers, which make parameters scale-invariant and introduce normalized effective parameters \citep{zhang_three_2018} with moderate norms. However, this requires a total reform of the theory to utilize effective weight matrices so we simply hide normalization layers in $M$ and leave it, and especially its interaction with weight decay, for future works.

The other alternative is dealt with in the next two subsections, where single-token features are used in proofs but the theories apply to stacked hidden features.
To give a brief account, in the following subsections we will prove that non-zero eigenvalues of $\kkT$ have similar values. So if gradients have moderate projections in the non-null subspace of $\kkT$, then $\left(\eta^l\right)^\transpose \kkT \eta$ can be lowerbounded by $\lambda r \left(\eta^l\right)^\transpose \left(\eta^l\right)$, where $r \le 1$ indicates how much of $\eta^l$ falls in the non-null space of $\kkT$ and $\lambda$ is the smallest non-zero eigenvalue or some averaged non-zero eigenvalues of $\kkT$, which is not too small compared to the largest one. Assuming gradients are still back-propagated to shallower layers, $\lambda r \left(\eta^l\right)^\transpose \eta^l$ can only be suppressed by decreasing $\norm{\eta^l}_2^2$ given that $\lambda$ increases with the trace of $\kkT$.
In \cref{sec:spectral_init} we prove this phenomenon at initialization in \cref{theorem:initial_spectral_properties} that the largest eigenvalue is initially at most $9$ times larger than the smallest non-zero one in Base-sized Transformers. The proof is based on ubiquitous Xavier or Kaiming initializations and Marchenko-Pastur distribution from random matrix theory.
In \cref{sec:spectral_training}, we theoretically discuss its re-emergence during stochastic training. We first rewrite the updates to the weight matrix $K^l$ into two large random matrices, whose shape is hidden dimension times the number of samples used in training. We then extend Marchenko-Pastur distribution in \cref{theorem:spectral_of_accumulated} under the practical inter-batch dependence, intra-batch independence and non-asymptotic scenario to prove an upperbound on the fraction between the largest and smallest non-zero eigenvalues. Conditions and assumptions of the theorem are verified empirically in \cref{sec:t_exp:anisotropy}. There are still gaps in combining the two random matrices, so we measure the spectral concentration in $\kkT$ empirically in \cref{sec:t_exp:spectral_concentration} and leave a more rigorous discussion for future works.

\subsection{Spectral Concentration at Initialization}\label{sec:spectral_init}

In this section, we prove that $K^l \left(K^l\right)^\transpose$ has eigenvalues that are close to each other, at least at initialization. With effective gradient sparsity measured by $\norm{\eta^l}_2^2$, this spectral concentration allows us to approximate the first term in RHS of \cref{eq:main} with $\lambda r \left(\eta^l\right)^\transpose \eta^l$, which is almost directly effective gradient sparsity measured in $L_2$ norms. 

To reach this goal, recall Marchenko–Pastur distribution in \cref{theorem:singular_value_of_product_of_random_matrices} that reveals the asymptotic spectral distribution of random matrices' product.
Applying \cref{theorem:singular_value_of_product_of_random_matrices} to $K^l\left(K^l\right)^\transpose$ initialized by Xavier or Kaiming initialization, we obtain \cref{theorem:initial_spectral_properties_of_kkT}.

\begin{theorem}[Initial spectral concentration of $\kkT$]\label{theorem:initial_spectral_properties_of_kkT}
    Assume $n \neq d$. Let $K^l \in \reals^{n \times d}$ be the weight matrix initialized by (Gaussian, uniform, or other distribution-based) Xavier or Kaiming initialization. When $n, d \to \infty$ with $d / n = 1 / c$, the ratio between the largest and smallest \emph{non-zero} eigenvalues of $\kkT$ converges weakly to
    \begin{align}
        \frac{\lambda_1\left(\kkT\right)}{\min_{k: \lambda_k\left(\kkT\right) > 0} \lambda_k\left(\kkT\right)} \le \frac{\left(1 + \sqrt{c}\right)^2}{\left(1 - \sqrt{c}\right)^2}.
    \end{align}
    Regarding zero eigenvalues, if $d > n$, there is no zero eigenvalue, and if $d \le n$, the expected portion of zero eigenvalues is $1 - \frac{1}{c} = 1 - \frac{d}{n}$.
    
\end{theorem}
\begin{proof}
    The initialization methods utilize centered distribution, and thus there is $\ex{K^l_{i, j}} = 0$.

    $\kkT$ differs from $S^p$ of \cref{theorem:singular_value_of_product_of_random_matrices} in 1) the shared standard variance of entries not being $1$, and 2) the scaling factor $\frac{1}{d}$. Since we are only interested in the ratio between eigenvalues, these differences of simultaneous scaling can be ignored.

    By \cref{eq:mp_density}, we can see that the support of eigenvalues is restricted to $[a, b] \cup \set{0}$. As a result, non-zero eigenvalues can only be found in $[a, b]$.

    When $c < 1$, i.e., $d > n$, the support degenerates to $[a, b]$. 
    When $c \ge 1$, the probability to pick a zero eigenvalue is $F(0) = \lim_{u \to 0^+} \int_{0}^{u} f(x) \mathrm{d}x = \lim_{u \to 0^+} \int_{0}^{u} \left(1 - \frac{1}{c}\right) \delta(x) \mathrm{d} x = 1 - \frac{1}{c}$ .
\end{proof}

Note that \cref{theorem:initial_spectral_properties_of_kkT} applies to uniform or other base initialization distribution as long as it is centered and entrywisely independent with the same variance. Since $n, d$ are generally large, even in small model sizes like Small and Base, we believe this lemma applies to common practice. In Base-sized Transformers, it is usually the case where $n = 3072, d = 768$, indicating $1 - \frac{768}{3072} = \frac{3}{4}$ of eigenvalues are $0$, while the rest of them varies up to the ratio of $\frac{\left(1 + \sqrt{\frac{1}{4}}\right)^2}{\left(1 - \sqrt{\frac{1}{4}}\right)^2} = 9$. This is a surprisingly small value compared to the number of dimensions.

Effective gradient sparsity patterns have a great affinity to $\kkT$, allowing \cref{theorem:initial_spectral_properties}.

\begin{theorem}[Implication of spectral concentration of $\kkT$ at initialization]\label{theorem:initial_spectral_properties}
    Assume $d \neq n$ and they are sufficiently large. $K^l \in \reals^{n \times d}$ is initialized as in \cref{theorem:initial_spectral_properties_of_kkT}. Let $M^l = g^l_K \left(g^l_K\right)^\transpose$, $\gamma^l$ and $\eta^l$ be those defined previously. 
    
    If $d > n$ then there is
    \begin{align}
        \left(\gamma^l\right)^\transpose \left(\left(K^l \left(K^l\right)^\transpose\right) \hadamard M^l\right) \gamma^l
        =& \left(\eta^l\right)^\transpose \kkT \eta^l\\
        \ge&    \lambda_{n}\left(\kkT\right) \cdot \norm{\eta^l}_2^2,
    \end{align}
    where $n$-th eigenvalue $\lambda_n\left(\kkT\right)$ is moderate and cannot be arbitrarily small because
    \begin{align}
        \frac{\lambda_1}{\lambda_{n}} \le \left(\frac{1 + \sqrt{c}}{1 - \sqrt{c}}\right)^2,
    \end{align}
    where $c = d / n$.

    If $n > d$, let $K^l = U \Sigma V^\transpose$ be the singular value decomposition of $K^l$. The result is restricted to the projection to the subspace expanded by $\left(U^\transpose\right)_{1:d}$, i.e.,
    \begin{align}
        \left(\gamma^l\right)^\transpose \left(\left(K^l \left(K^l\right)^\transpose\right) \hadamard M^l\right) \gamma^l
        =& \left(\eta^l\right)^\transpose \kkT \eta^l\\
        \ge&    \lambda_{d}\left(\kkT\right) \cdot \norm{\left(U^{T}\right)_{1: d}\eta^l}_2^2,
    \end{align}
    where $d$-th eigenvalue $\lambda_d\left(\kkT\right)$ satisfies
    \begin{align}
        \frac{\lambda_1}{\lambda_{d}} \le \left(\frac{1 + \sqrt{c}}{1 - \sqrt{c}}\right)^2,
    \end{align}
    where $c = d / n$.

    For demonstration, when $\set{n, d} = \set{3072, 768}$, the ratio upperbound is $9$.
\end{theorem}
\begin{proof}
    The proof is straightforward after \cref{theorem:initial_spectral_properties_of_kkT}, by noting that when $n > d$ there are exactly $\left(1 - \frac{1}{c}\right) n = n - d$ zero eigenvalues in $\kkT$, or equivalently $d$ non-zero eigenvalues in $\kkT$.
\end{proof}

There are still gaps between $\lambda \left(\eta^l\right)^\transpose \eta^l$ and current practices where $d < n$ and there are a lot of zero eigenvalues in $\kkT$, but \cref{theorem:initial_spectral_properties} is perfectly useful for wide MLPs where $d > n$. Therefore, we propose a drastic architectural modification called wide MLP where $d > n$, i.e., the model dimension is larger than the hidden dimension in MLP blocks. Aside from more powerful motivation toward sparsity, wide MLPs also allow rows in $K^l$ to be mutually orthogonal and permit perfect sparsity, which is impossible when $n > d$.
 
In non-wide MLPs, we believe that since $\kkT$ are randomly initialized and samples are randomly selected, there are moderate projections of $\eta^l$ into the non-null subspace of $\kkT$. Another supporting intuition is that although $\sigma'$ is not identity or linear, the derivatives of common activation functions are often monotonically increasing and form an approximation to its inputs. This approximation is better when part of the activation derivatives are linear, as in the case of Squared-$\relu$\citep{primer} and our $\jrelu$. Therefore, taking $\sigma'$ to $K^l x + b^l$, which already falls near the subspace expanded by $\left(U^\transpose\right)_{1: d}$, does not deviate far from the subspace. 
\cref{obs:memorizing_eff} also supports this moderate projection dynamically because if $\eta^l$ is in the null space, it will be borne into every column of the key matrix and next time it will have non-zero projections if $\eta^l$ does not change too much after one epoch and the column memory is not blurred too severely. Repeatedly memorizing different $\eta^l$ will make the non-null space of $\kkT$ a mixture of the majority $\eta^l$s provided by the training data set. 
The empirical evidence for this is that $\left(\eta^l\right)^\transpose \kkT \eta^l$, according to the derivation in \cref{lemma:adversarial_and_sparsity}, is actually the norm of gradients back propagated to shallower layers. Extreme cases where $\eta^l$ are contained only in the null space of $\kkT$ result in zero gradients for shallower layers, which rarely happens. If no residual connection is involved this insight strongly augments the spectral explanation. The detailed and formal analysis of the zero eigenvalues especially when there are residual connections is left for future empirical and theoretical works. For now, we can simply cover the gap with wide MLPs.

\subsection{Spectral Concentration during Stochastic Training}\label{sec:spectral_training}

In this subsection, we discuss how spectral concentration re-emerges during later stochastic training. 

First recall the Marchenko-Pastur distribution in \cref{theorem:singular_value_of_product_of_random_matrices}. The condition of random centered matrices in \cref{theorem:singular_value_of_product_of_random_matrices} invites another randomness other than initialization to the party, i.e., stochastic gradient noise (SGN) brought by stochastic optimizers.
After $t$ updates, $K^{l}$ can be written as the sum of random initialization, stochastic gradient noises and full-batch gradients that are not as stochastic as the two former terms, i.e.
\begin{align}
    K^{l, t} = \underbrace{K^{l, 0} - \sum_{i=1}^{t} U^{i}_{K^l}}_{\text{stochastic, centered}} - \sum_{i=1}^t \derivatives{\loss(\theta^t)}{K^l}
\end{align}
As discussed in \cref{sec:flat_minima}, $U^{i}_{K^l}$ is by definition centered. If it can be assumed that the $\sas$ SGN with large variance and noise norm shadows full-batch gradient, then $K^{l, t}$ is the sum of two centered random matrix with a slight non-stochastic bias, to which Marchenko-Pastur distribution would approximately apply if further entries in $U^{i}_{K^l}$ shared similar variance and were sampled independently. \citet{relax_mp_distribution} and works cited by them have tried to relax the independence condition of \cref{theorem:singular_value_of_product_of_random_matrices}, but it is still far from applying relaxed Marchenko-Pastur distribution here. Aside from waiting for this mathematical progress, we build empirical basis in \cref{sec:t_exp:spectral_concentration} where at all steps, in $\kkT$ of ViT and the decoder of T5, there is a stable portion of near-zero eigenvalues as in \cref{theorem:initial_spectral_properties} across all layers, and the majority of non-zero ones, with significant gap with near-zero ones, vary up to a ratio of $<100$ for most of the time. It is not surprising that this effect empirically sustains and even becomes stronger at the end of training because the model is well-learned by then and the full-batch gradient is of a smaller norm.

\NewDocumentCommand{\DKDKT}{O{t}}{\DK{#1}\left(\DK{#1}\right)^\transpose}
\NewDocumentCommand{\DKTDK}{O{t}}{\left(\DK{#1}\right)^\transpose \DK{#1}}

To have a more satisfying discussion, we propose an extended version of Marchenko-Pastur distribution and find a re-directed view on stochastic gradients to apply it. To this end, what conditions and assumptions can stochastic training provide must be figured out first.

Observing the structure of $\mlp$ or $\dbmlp$ layers, the most essential operation involving weight matrix $K^l$ is
\begin{align}
    z^l \defeq K^l x^l,
\end{align}
where $x$ is abused to represent anything that is multiplied with $K^l$, abstracting $x^l$ or $x^l + d^l$, while $z^l$ is the vector passed to the vanilla bias, activation function or later layers. This structure gives birth to the update of a sample $(x_s, y_s)$ to $K^l$ that writes
\NewDocumentCommand{\DK}{m}{\Delta K^{l, #1}}
\begin{align}
    \DK{s}
    =&  -\eta_{\mathrm{lr}} \cdot \derivatives{\loss(\theta, (x_s, y_s))}{z^{l, s}} \times (x^{l, s})^\transpose
    =  -\eta_{\mathrm{lr}} \cdot \eta^{l, s} \left(x^{l, s}\right)^\transpose,
\end{align}
where $\eta_{\mathrm{lr}}$ is the learning rate, assuming no scheduling is used.
At step $t$ with batch $B_t$, the update on $K^l$ averages these samplewise differences, i.e.
\begin{align}
    \DK{t}
    \defeq& - \frac{\eta_{\mathrm{lr}}}{\size{B_t}} \sum_{s \in B_t} \eta^{l, s} \left(x^{l, s}\right)^\transpose
    =  - \frac{\eta_{\mathrm{lr}}}{\size{B_t}} H^t \left(X^t\right)^\transpose,
\end{align}
where $\Eta^t \in \reals^{n \times \size{B_t}}$ (capitalized ``$\eta$'') is the matrix consisting of column vectors $\eta^{l,s}$ for sample $s$ in the batch $B_t$, and $X^t \in \reals^{d \times \size{B_t}}$ is similarly constructed with $x^{l,s}, s \in B_t$. 
Note that $X^t$ and $\Eta^t$ are random matrices because samples are independently randomly selected and gradients are also random variables as functions of variables.
Taking a similar view throughout the training, there is
\begin{align}
    K^{l, T} - K^{l, 0}
    =&  -\eta_{\mathrm{lr}} \sum_{t=1}^T \frac{1}{\size{B_t}} \sum_{s \in B_t} \eta^{l, s} \left(x^{l, s}\right)^\transpose
    =  -\frac{\eta_{\mathrm{lr}}}{b} \Eta^{1: T} \left(X^{1:T}\right)^\transpose \label{eq:update_and_large_matrices},
\end{align}
where $T$ is the number of batches, $b$ is batch size, and $\Eta^{1:T} \defeq \begin{bmatrix}  \Eta^1 & \cdots & \Eta^t & \cdots & \Eta^T  \end{bmatrix}$, and $X^{1:T} \defeq \begin{bmatrix} X^1 & \cdots & X^t & \cdots & X^T \end{bmatrix}$.
Another product of large random matrices emerges in the empirical covariance matrix of the difference, i.e.,
\begin{align}
    \left(K^{l, T} - K^{l, 0}\right) \left(K^{l, T} - K^{l, 0}\right)^\transpose = \frac{\eta_{\mathrm{lr}}^2}{b^2} \Eta^{1: T} \left(X^{1:T}\right)^\transpose X^{1:T} \left(\Eta^{1:T}\right)^\transpose,
\end{align}
where $X^{1:T}$ and $\Eta^{1:T}$ are random matrices in the sense that samples or gradient vectors in each batch are independently randomly sampled, if conditioned on the model state.

Since we are interested in spectral distribution and that cycling a matrix product does \emph{not} change non-zero eigenvalues, a more desirable form is
\begin{align}
    \left(\Eta^{1:T}\right)^\transpose \Eta^{1:T} \left(X^{1:T}\right)^\transpose X^{1:T},
\end{align}
and we intend to separately investigate the spectral distributions of 
\begin{align}
    &\text{$\left(\Eta^{1:T}\right)^\transpose \Eta^{1:T}$ or spectrally equivalent $\Eta^{1:T} \left(\Eta^{1:T}\right)^\transpose$},\\
    \text{and, }&\text{$\left(X^{1:T}\right)^\transpose X^{1:T}$ or spectrally equivalent $X^{1:T} \left(X^{1:T}\right)^\transpose$}.
\end{align}

After these transforms and dividing-and-conquering, the empirical covariance matrices look ready for Marchenko-Pastur law. However, there are dependencies between previous and later batches through model states, hindering the direct application of independence conditions of Marchenko-Pastur law. Fortunately, there is still conditional independence \emph{within} a batch. This mixture of dependence and independence is captured by \cref{def:batch_model}.
\begin{definition}[Batch Dependence Model]\label{def:batch_model}
    Let $U^{1: T} \in \reals^{p \times (b T)}$ be a random matrices. Decompose it into blocks with batch size $b$, i.e., 
        \begin{align}
            U^{1: T} = 
                \begin{bmatrix}
                    U^1 & \cdots & U^t & \cdots & U^{T-1} & U^{T}
                \end{bmatrix},
        \end{align}
    where $U^t \in \reals^{p \times b}$.
    If the dependence between elements can be described by SCMs
    \begin{align}
        \set{u^t_k \defeq g^t\left(U^{1}, U^{2}, \dots, U^{t-1}, \epsilon^t_k\right): t \in [1, T], k \in [1, b]}
    \end{align}
    or SCMs that resemble the notions of samples and model state
    \begin{align}
        \set{u^t_k \defeq g^t\left(m^{t-1}, \epsilon^t_k\right) : t \in [1, T], k \in [1, b]} \cup \set{m^t \defeq h^t\left(m^{t-1}, U^{t}\right)}
    \end{align}
    where $u^t_k$s are columns in $U^t$, $m^{t}$ is the model state (parameters, momentum, etc.) after step $t$, $\epsilon^{t}_k$ is I.I.D. random noises, then $U^{1: T}$ is a random matrix with batch dependence.
    \begin{remark}
        Within each batch (i.e., when conditioned on all previous batches), samples are I.I.D. sampled to simulate batch sampling. However, previous samples have trained the parameters and will shift the distribution of shallow layers's output as well as back-propagated gradients. Therefore, the current batch depends on previous batches.
    \end{remark}
\end{definition}

There are works re-establishing Marchenko-Pastur law with independence conditions relaxed to martingale conditions \citep{mp_martingale}, but some conditions in it require entrywise conditional independence. There are also Marchenko-Pastur laws for time series \citep{mp_linear_time_series1,mp_linear_time_series2}, but restricted to linear dependence. 
We adapt proofs by \citet{mp_quadratic_form}, and use anisotropy condition in all samples or gradients to extend Marchenko-Pastur distribution under the batch dependence in $X^{1:T}$ and $\Eta^{1:T}$, leading to \cref{theorem:spectral_of_accumulated}.
In later formal definitions, theorems and proofs, $X$ is abstractly used to represent both $\Eta^{1: T}$ or $X^{1:T}$, $p$ indicates the height of $X^{1:T}$ or $\Eta^{1: T}$, i.e., the hidden dimensions $d$ or $n$, while $b$ and $T$ keep their meanings as batch size and total number of batches.

\def\StateSpectralOfAccumulated{display}
\ifdefstring{\StateSpectralOfAccumulated}{display}{

\begin{restatable}[Spectral concentration of accumulated steps]{theorem}{SpectralOfAccumulated}
    \label{theorem:spectral_of_accumulated}
    Let $X^p = X^{p, b T} \in \reals^{p \times b T}$ be a random matrix that forms a Batch Dependence Model as in \cref{def:batch_model} with batch size $b$ and step count $T$, whose columns are $x^p_j = X^{p, b T}_{\cdot, j} \in \reals^{p}$. 
    Columns in $X^p = X^{p, b T} \in \reals^{p \times b T}$ are \emph{not} necessarily independent.
        
    Let $x_k^p \defeq X^{p}_{\cdot, k}$ be the $k$-th column of $X^p$ and $x_{t, l}^p \defeq X^{t}_{\cdot, l}$ be the $l$-th column of batch $t$ or equivalently the $k=\left((t-1)*b + l\right)$-th column $x_k$ in $X^{p}$. Superscription $p$ may be dropped for convenience.

    Let $S^{p} \defeq \frac{1}{b T} X^p \left(X^p\right)^\transpose = \frac{1}{b T} \sum_{k=1}^{b T} x_k x_k^\transpose$ be the empirical covariance matrix of all random vectors, and $I_p$ be the compatible identity matrix.
    Assume $x_{t, l}$s' norm is bounded, say by $1$, and scale it with 
    \begin{align}
        u^p_{t, l} \defeq \sqrt{\alpha} x^p_{t, l},
    \end{align}
    obtaining $U^p = U^{p, b} \defeq \begin{bmatrix} U^1 & \cdots &  U^t & \cdots  & U^T \end{bmatrix} = \sqrt{a} \cdot X^p$, where $a \defeq \frac{\trace{S^p}}{\trace{S^p S^p}}$. Let 
    \begin{align}
        T^{p} \defeq \frac{1}{b T} U^p \left(U^p\right)^\transpose = \frac{1}{b T} \sum_{k=1}^{b T} u_{k} u_{k}^\transpose = a \cdot S^{p}
    \end{align}
    be the empirical covariance matrix of all $u^p_{k}$s. 

    Assume $a$ is bounded by $\alpha(p)$ and $\ex{\sqrt{p \trace{\left(T^{p} - I^p\right)\left(T^{t} - I^p\right)}}}$ is also upperbounded by $\beta(p)$.

    Further assume that the following function of $z \in \positivecomplex$ and $p, b, T \in \nats^+$
    \begin{align}
        \ex{\sum_{i} \frac{1}{\lambda_i\left(U U^\transpose\right) - z}}
    \end{align}
    is always continuous w.r.t. $z$ for any $p, b, T$.

    If the above assumptions are satisfied, the non-zero eigenvalue concentrates. To be more specific, let $\overline{\lambda^{>0}}$ be the mean of non-zero eigenvalues of $\frac{1}{b T} U^p \left(U^p\right)^\transpose$ and use $\ex{\overline{\lambda^{>0}}}^2$ to represent the overall situation of non-zero eigenvalues. Then there is
    \begin{align}
        \ex{\frac{\ex{\overline{\lambda^{>0}} / \sqrt{v}}^2}{\left(\frac{\lambda}{\sqrt{v}}\right)^2 + v}}
        \le&    \frac{\sqrt 2}{c \sqrt{v}} \frac{\alpha^2}{v \cdot \min(b T, p)^2}  \sqrt{c + \frac{\left(2 \sqrt{2} + 2\right) c \alpha}{v p} + \frac{c \beta}{v p} + \frac{c}{v p}} \label{eq:im_bound},\\
        \ex{\frac{\ex{\overline{\lambda^{>0}}}}{\lambda + \frac{v^2}{\lambda}}}
        \le&    \frac{\sqrt 2}{c \sqrt{v}} \frac{\alpha}{\min(b T, p)}  \sqrt{c + \frac{\left(2 \sqrt{2} + 2\right) c \alpha}{v p} + \frac{c \beta}{v p} + \frac{c}{v p}} \label{eq:re_bound}.
        \end{align}
    for any $v \ge v_0$, where $\lambda$ is a randomly selected eigenvalue of $T^p \defeq \frac{1}{b T} U^p \left(U^p\right)^\transpose$ and $c = p / b T \in [0, 1]$, and $v_0 \ge 2 c$ satisfying 
    \begin{align}
        \frac{v_0 + (1 - c)}{\sqrt{2}} > \frac{\tau}{v_0} + 2 \sqrt{c v_0} + 2 \sqrt{\tau} \label{eq:hard_condition}
    \end{align}
    with $\tau \defeq \frac{c}{p} \left(1 + \beta + 2\left(\sqrt{2} + 1\right) \alpha\right)$.
\end{restatable}

}{

% \arxivonly{\SpectralOfAccumulated*}
\begin{proof}[Proof of \cref{theorem:spectral_of_accumulated}]\label{proof:spectral_of_accumulated}

The proof is adapted from \citet{mp_quadratic_form} where independence conditions are replaced with Batch Dependence model and new regularities.


Cauchy-\sti{} transform method is used. 
When applied to empirical spectral density, by definition there is
\begin{align}
    s^{F^A}(z) = \trace{A - z I}^{-1} / p \defeq \trace{\left(A - z I\right)^{-1}} / p.
\end{align}
for positive semi-definite $A \in \reals^{p \times p}$.
Specifically, $\frac{1}{b T} U^p \left(U^p\right)^\transpose = \frac{1}{b T} U U^\transpose$'s \sti{} transform is
\begin{align}
    s_p(z) = \trace{\frac{1}{b T} U U^\transpose - z I}^{-1} / p = b T / p \trace{U U^\transpose - z b T I}^{-1}.
\end{align}
% By \sti{} continuity theorem \citep{RMT_book}, it is sufficient to show that $s_p(s) \asto s(z)$ for all $z \in \positivecomplex$, where $s$ is the \sti{} transform of Marchenko-Pastur distribution with parameter $p$ and $n$. To this end, typical steps include the following steps \citep{mp_proof_sketch}:
% \begin{itemize}
    % \item $s_p(z) - \ex{s_p(z)} \asto 0$, by a martingale argument;
    % \item $\ex{s_p(z)} \to s(z)$.
% \end{itemize}

\NewDocumentCommand{\boundedby}{m}{\varXi\left(#1\right)}
To ease presentation, we define $\boundedby{g}$ to indicate (complex) functions whose magnitudes are bounded by positive real function $g$, i.e.,
\begin{align}
    h \in \boundedby{g} \iff \forall x, y, \abs{h(x, y)} \le g(x),
\end{align}
where $y$ indicates variables other than $x$ that $h$ relies.
$\boundedby{\cdot}$ will be used combined with ``$=$'' imitating $O(\cdot)$. Since $\boundedby{\cdot}$ does not hide constant scaling factors and biases in it, unlike $O(\cdot)$ it can be freely added, averaged, multiplied and divided, i.e.,
\begin{align}
    \boundedby{g_1} + \boundedby{g_2} \in& \boundedby{g_1 + g_2},
    \frac{1}{n} \sum_{i=1}^n \boundedby{g_i} \in \boundedby{\frac{1}{n}\sum_{i=1}^n g_i},\\
    \boundedby{g_1} \cdot \boundedby{g_2} \in& \boundedby{g_1 \cdot g_2},
    \frac{\boundedby{g_1}}{g_2} \in \boundedby{\frac{g_1}{g_2}}.
\end{align}

% For $s_p(z) - \ex{s_p(z)}$'s almost sure convergence, \citet{mp_quadratic_form} refer to \citet{mp_proof_sketch}, which we will repeat here to examine and replace independence assumptions within.

% \NewDocumentCommand{\condiex}{O{k} m}{\ex[#1]{#2}}
% Let $\condiex[k]{\cdot} \defeq \ex{\cdot \mid U^p_{\cdot, 1: k}}$ denote the conditional expectation given $U^p_{\cdot, 1}, \dots, U^p_{\cdot, k}$. Then $s_p(z) = \condiex[b]{s_p(z)}$ and $\ex{s_p(z)} = \condiex[0]{s_p(z)}$, and there is
% \begin{align}
    % s_p(z) - \ex{s_p(z)}
    % =&  \sum_{k=1}^{b} \left(\condiex[k]{s_p(z)} - \condiex[k-1]{s_p(z)}\right)
    % =  \sum_{k=1}^{b} \gamma_k,
% \end{align}
% where $\gamma_k \defeq \condiex[k]{s_p(z)} - \condiex[k-1]{s_p(z)}$. Note that $\set{\condiex[k]{s_p(z)}}_k$ is already a Doob martingale, so $\set{\gamma_k}_k$, as its difference, is a sequence of martingale differences.
        
% \NewDocumentCommand{\invR}{O{1}}{\left(R^p_k\right)^{-#1}}
% Let $R^p_{k} \defeq \frac{1}{b T} \sum_{k'} \uut[k'] - z I - \frac{1}{b T}u_k u_k^\transpose$. By \cref{lemma:3.1_from_mp_quadratic_form}(1), $R^p_{k}$ is invertible, and by Sherman-Morrison formula there is
% \begin{align}
    % s_p(z)
    % =&  \trace{R^p_k + \frac{1}{b T} u_k u_k^\transpose }^{-1} / p
    % =  \frac{1}{p} \trace{\invR - \frac{\invR u_k u_k^\transpose \invR / b T}{1 + u_k^\transpose \invR u_k / b T}}\\
    % =&  \frac{1}{p} \left(\trace{\invR} - \frac{u_k^\transpose \invR[2] u_k}{b T + u_k^\transpose \invR u_k}\right).
% \end{align}
% By minor single sample assumption, $\condiex[k]{\trace{\invR}} = \condiex[k-1]{\trace{\invR}}$ is bounded by $\beta(z)$. Regarding the second term,
% \begin{align}
    % &   \abs{\frac{u_k^\transpose \invR[2] u_k}{b T + u_k^\transpose \invR u_k}}\\
    % =&      \frac{\abs{\trace{u_k u_k^\transpose \invR \invR}}}{\abs{b T + u_k^\transpose \invR u_k}}
    % \le    \frac{\norm{u_k u_k^\transpose \invR \invR}_1}{b T + u_k^\transpose \invR u_k}\\
    % \le&   \frac{\norm{u_k u_k^\transpose \invR}_1 \norm{\invR}_{\infty}}{b T + u_k^\transpose \invR u_k}
    % \le    \frac{1}{v} \frac{u_k^\transpose \invR u_k}{b T + u_k^\transpose \invR u_k}
    % \le \frac{1}{v},
% \end{align}
% where the third inequality follows \cref{lemma:3.1_from_mp_quadratic_form}(1).
% Therefore $\gamma_k$ can be bounded by
% \begin{align}
    % \gamma_k \le& \frac{2}{v} \frac{1}{p} + \beta(z) < \infty.
% \end{align}
% By Azuma's inequality, there is
% \begin{align}
    % \prob{\abs{s_p(z) - \ex{s_p(z)}} > \epsilon} \le 2 \exp\left(-\frac{\epsilon^2}{2 p \left(\frac{2}{v} \frac{1}{p}\right)^2}\right) \le 2 \exp\left(-\epsilon^2 v^2 p / 8\right).
% \end{align}
% Given that $\prob{\abs{s_p(z) - \ex{s_p(z)}}}$ decays exponentially with $p$, for every $\epsilon > 0$, there is
% \begin{align}
    % \sum_{p=0}^\infty \prob{\abs{s_p(z) - \ex{s_p(z)}} > \epsilon} < \infty
% \end{align}
% which implies $s_p(z) - \ex{s_p(z)} \asto 0$ (Theorem 7.5 by \citet{exponential_decay_to_as}).

Consistent with final conclusion, fix $z = 0 + v i$ ($v \in \reals^+$) such that $v \ge v_0$ throughout the proof.
Define $A^p \defeq \sum_{k} \uut$.
Sample an auxiliary vector $u_{T, b+1} = u_{b T  + 1} \in \reals^p$ so that it is sampled from the conditional distribution given the first $T-1$ batches but it is conditionally independent with other samples in $U^T$, i.e., an extra sample for the last batch. This dependence relation can be expressed by only adding edges $U^{1: T-1} \to u_{T, b+1}$ to the SCMs of the Batch Dependence Model. With the auxiliary vector, define $B^p \defeq A^p +  \uut[T, b+1]$. %Define $C^b_t \defeq B^p - u^t \left(u^t\right)^\transpose$

By \cref{lemma:3.1_from_mp_quadratic_form}(1), $B^p - z b T I$ is non-degenerate and
\begin{align}
    p 
    =&  \trace{\left(B^p - z b T I\right) \left(B^p - z b T I\right)^{-1}}\\
    =&   \sum_{t=1}^{T} \sum_{l=1}^{b + \indic{t = T}} u_{t, l}^\transpose \left(B^p - z b T I\right)^{-1} u_{t, l} - z b T \trace{B^p - z b T I}^{-1}.
\end{align}
Taking expectations and using the exchangeability within each batch give
\begin{align}
    p = \sum_{t=1}^{T} (b + \indic{t = T}) \ex{u_t^\transpose \left(B^p - z b T I\right)^{-1} u_t} - z b T \ex{\trace{B^p - z b T I}^{-1}} \label{eq:a3}.
\end{align}

Define $S_p(z) \defeq \trace{A^p - z b T I}^{-1}$ and note that $S_p(z) = (p / b T) s_p(z)$. 

By \cref{lemma:3.1_from_mp_quadratic_form}(2), there is
\begin{align}
    \ex{\trace{B^p - z b T I}^{-1}} =& \ex{S_p(z)} + \boundedby{1/ v b T} = \ex{S_p(z)} + \boundedby{c / v p} \label{eq:a1}.
\end{align}

\NewDocumentCommand{\approxmatrix}{O{\boundedby} O{2}}{#1{\frac{#2 \sqrt{2} c \alpha}{v p}}}
We now prove
\begin{align}
    \frac{1}{T} \sum_{t=1}^{T} \ex{u_t^\transpose \left(B^p - z b T I\right)^{-1} u_t} = \frac{\ex{S_p(z)}}{1 + \ex{S_p(z)}} + t \label{eq:claim}, 
\end{align}
where $\abs{t}$ is bounded by a function of $c, \alpha, \beta, v, p$.

\NewDocumentCommand{\approxfunction}{O{\boundedby}}{#1{1}}
A complex function $\frac{x}{1 + x} = 1 - \frac{1}{x + 1}$ emerges many times. We will approximate it to the first order so its complex derivative should be computed and bounded.
\begin{align}
    \abs{\left(\frac{x}{1 + x}\right)'}
    =&  \abs{\frac{1}{(x+1)^2}} 
    = \frac{1}{\abs{x + 1}^2}
\end{align}
Therefore, if $x_1, x_2$ both stay away from $-1$, then $\abs{\left(\frac{x'}{1 + x'}\right)'} = \approxfunction$ on the line connecting $x_1, x_2$ and we can approximate $\frac{x_2}{1 + x_2}$ by $\frac{x_1}{1 + x_1} + \boundedby{1} \cdot \Delta x = \frac{x_1}{1 + x_1} + \boundedby{\Delta x}$, where $\Delta x = x_2 - x_1$.
In latter application, $x$, both the start and the end of approximation, is often of form $\frac{1}{n} \sum_{i=1}^n \ex{u_i^\transpose \left(C - z b T I\right)^{-1} u_i}$ possibly with averaging or expectation missing, where $C$ is real symmetric positive semi-definite and $u_i$ is a real vector. The eigenvalues in $\left(C - z b T I\right)^{-1}$ are
\begin{align}
    \frac{1}{\lambda_i(C) - v b T i}
    =&  \frac{\lambda_i(C) + v b T i}{\lambda_i(C)^2 + (v b T)^2},
\end{align}
whose real part is
\begin{align}
    \rpart{\frac{1}{\lambda_i(C) - v b T i}}
    =&  \frac{\lambda_i(C)}{\lambda_i(C)^2 + (v b T)^2} \ge 0.
\end{align}
As a result, the real part of inner products is always non-negative and $x$ stays away from $-1$, and the magnitude of derivatives is $\approxfunction$.

Another approximation is done between $C^p_k$ and $A_p$, whose difference is the outer products of a constant number of random vectors, and it should be minor considering there are $b T$ of them. Formally, for real symmetric positive semi-definite $C$ with eigenvalue decomposition $C = V \Lambda V^\transpose$ by real matrices $V$ and $\Lambda$, $(C - z I)$ can be decomposed to $(C - z I) = V \left(\Lambda - z I\right) V^\transpose$, and non-degenerate $\left(C - z I\right)^{-1}$ to $\left(C - z I\right)^{-1} = V \left(\Lambda - z I\right)^{-1} V^\transpose \defto V \Sigma \Sigma V^\transpose$ where $\Sigma \defeq \sqrt{\left(\Lambda - z I\right)^{-1}}$. Let $S \defeq V \Sigma V^\transpose$ to have $S^\transpose S = S S = \left(C - z I\right)^{-1}$. After that, there is
\begin{align}
    &   \abs{y^\transpose \left(C + x x^\transpose - z I\right)^{-1}y - y^\transpose \left(C - z I\right)^{-1} y}
    =  \abs{y^\transpose \left(\left(C + x x^\transpose - z I\right)^{-1} - \left(C - z I\right)^{-1} \right) y}\\
    =&  \abs{\frac{
            y^\transpose \left(C - z I\right)^{-1} x x^\transpose \left(C - z I\right)^{-1} y
        }{1 + x^\transpose \left(C - z I\right)^{-1} x}}
    =   \abs{\frac{
            \left(y^\transpose S^\transpose S x\right) \left(x^\transpose S^\transpose S y\right)
        }{1 + x^\transpose \left(C - z I\right)^{-1} x}}
    =  \abs{\frac{
            \left(a^{\transpose} \bar{b}\right) \left(\bar{b}^\transpose a\right)
        }{1 + x^\transpose \left(C - z I\right)^{-1} x}}\\
    =&   \frac{
            \abs{a^* b} \abs{a^* b}
        }{\abs{1 + x^\transpose \left(C - z I\right)^{-1} x}}
    \le \frac{
            \norm{a^*}_2 \norm{b}_2 \norm{a^*}_2 \norm{b}_2
        }{\abs{1 + x^\transpose \left(C - z I\right)^{-1} x}}
    =   \frac{
            \abs{a^* a} \abs{b^* b}
        }{\abs{1 + x^\transpose \left(C - z I\right)^{-1} x}}\\
    =&  \frac{
            \abs{\trace{y y^\transpose S^* S}} \abs{b^* b}
        }{\abs{1 + x^\transpose \left(C - z I\right)^{-1} x}}
    \le \frac{
            \norm{y y^\transpose S^* S}_1 \abs{b^* b}
        }{\abs{1 + x^\transpose \left(C - z I\right)^{-1} x}}
    \le \frac{
            \norm{y y^\transpose}_1 \norm{S^* S}_\infty \abs{b^* b}
        }{\abs{1 + b^\transpose b}}
    =   \frac{\norm{y}_2^2}{\ipart{z}} \frac{
            \abs{b^* b}
        }{\abs{1 + b^\transpose b}},
\end{align}
where $a \defeq S y, b \defeq \bar{S} \bar{x} = \bar{S} x$, the second step is from Sherman-Morrison formula, and the second last inequality is due to \cref{lemma:abs_trace_and_schatten_1}. The fact, that $S^* S$ is positive semi-definite whose largest eigenvalue is smaller than the upperbound $\frac{1}{v}$ of $S^\transpose S$'s eigenvalue magnitude, is also used.  To bound the fraction between $\abs{b^* b}$ and $\abs{1 + b^\transpose b}$, recall the eigenvalue decomposition on $\left(C - z I\right)^{-1}$ 
\begin{align}
    \left(C - z I\right)^{-1} = V \Sigma \Sigma^\transpose V^\transpose
\end{align}
and $S = V \Sigma^\transpose V^\transpose$. Then 
\begin{align}
    b^\transpose b &= v^\transpose \Sigma \Sigma v,
    b^* b = v^\transpose \bar{\Sigma} \Sigma v,
\end{align}
where $v \defeq V^\transpose x$ is a real vector. Notice that $\Sigma \Sigma = \diag{\frac{1}{\lambda_i(C) - v i}} = \diag{\frac{\lambda_i(C) + v i}{\lambda_i(C)^2 + v^2}}$ where both real and imaginary parts are non-negative, and that $\Sigma^* \Sigma = \diag{\frac{\abs{\lambda_i(C) + v i}}{\lambda_i(C)^2 + v^2}}$. With this, the inner products are simplified to
\begin{align}
    b^\transpose b &= \sum_{i} \frac{v_i^2 \lambda_i(C)}{\lambda_i(C)^2 + v^2} + i \sum_{i} \frac{v_i^2 v}{\lambda_i(C)^2 + v^2},
    b^* b = \sum_{i} \abs{\frac{v_i^2}{\lambda_i(C)^2 + v^2} \lambda_i(C) + i \frac{v_i^2 v}{\lambda_i(C)^2 + v^2}}
\end{align}
Representing complex numbers by 2-dimensional vectors $w_i \defeq \begin{bmatrix} \frac{v_i^2 \lambda_i(C)}{\lambda_i(C)^2 + v^2} & \frac{v_i^2 v}{\lambda_i(C)^2 + v^2} \end{bmatrix}^\transpose$, there are
\begin{align}
    \abs{b^\transpose b} &= \norm{\sum_i w_i}_2,
    \abs{b^* b} = \sum_i \norm{w_i}_2.
\end{align}
Noting that all entries of $w_i$'s are non-negative, there is
\begin{align}
    \abs{b^* b}
    &=  \sum_i \norm{w_i}_2
    \le \sum_{i} \norm{w_i}_1 
    =   \norm{\sum_i w_i}_1 
    \le \sqrt{2} \norm{\sum_i w_i}_2 = \sqrt{2} \abs{b^\transpose b}.
\end{align}
So $\frac{\abs{b^* b}}{\abs{b^\transpose b}} \le \sqrt{2}$. Given that the real part of $b^\transpose b$ is non-negative, adding $1$ will only increase its magnitude. As a result, there is
\begin{align}
    &   \abs{y^\transpose \left(C + x x^\transpose - z I\right)^{-1}y - y^\transpose \left(C - z I\right)^{-1} y}
    \le  \frac{\sqrt{2} \norm{y}_2^2}{\ipart{z}},
\end{align}

When $z b T$ is substituted, there is
\begin{align}
    &   \abs{y^\transpose \left(C + x x^\transpose - z b T I\right)^{-1}y - y^\transpose \left(C - z b T I\right)^{-1} y}
    =  \frac{\sqrt{2} \norm{y}_2^2}{v b T}.
\end{align}
In later use, $y$ is instantiated by $u_k$ and there is $\norm{u_k}_2^2 = a \norm{x}_2^2 \le \alpha$ for any $t$, so by assumption the approximation error is always bounded by 
\begin{align}
    \abs{u_k^\transpose \left(C + x x^\transpose - z b T I\right)^{-1} u_k - u_k^\transpose \left(C - z b T I\right)^{-1} u_k} \le \approxmatrix[\boundedby][].
\end{align}

\NewDocumentCommand{\innersum}{}{\frac{1}{b T}\sum_{k}}
\NewDocumentCommand{\outersum}{}{}
\NewDocumentCommand{\innerapproximator}{O{\left(A^p - z b T I\right)^{-1}} O{k}}{ u_{#2}^\transpose #1 u_{#2} }
\NewDocumentCommand{\innerouterproduct}{O{}}{u_{k} #1 u_{k}^\transpose}
\NewDocumentCommand{\biasapproximator}{}{\frac{\ex{\innersum \innerapproximator}}{1 + \ex{\innersum \innerapproximator}}}
\NewDocumentCommand{\diffapproximator}{}{\innerapproximator - \ex{\innersum \innerapproximator}}
With these two approximation techniques, we first approximate the LHS of \cref{eq:claim}.
To this end, let $C^p_k \defeq B^p - u_k u_k^\transpose$ and by Sherman-Morrison formula there is
\begin{align}
    &   u_k^\transpose \left(B^p - z b T I\right)^{-1} u_k
    =   u_k^\transpose \left(C^p_k + u_k u_k^\transpose - z b T I\right)^{-1} u_k\\
    =&  u_k^\transpose \left(\left(C^p_k - z b T I\right)^{-1} - \frac{\left(C^p_k - z b T I\right)^{-1} u_k u_k^\transpose \left(C^p_k - z b T I\right)^{-1}}{1 + u_k^\transpose\left(C^p_k - z b T I\right)^{-1} u_k}\right) u_k\\
    =&  \frac{u_k^\transpose \left(C^p_k - z b T I\right)^{-1}u_k}{1 + u_t^\transpose\left(C^p_k - z b T I\right)^{-1} u_k}
    =  \frac{u_k^\transpose \left(A^p - z b T I\right)^{-1}u_k}{1 + u_k^\transpose\left(A^p - z b T I\right)^{-1} u_k} + \approxfunction \cdot \approxmatrix.
\end{align}
After that, there is
\begin{align}
    &   \frac{1}{T} \sum_{t=1}^T \ex{u_t^\transpose (B^p - z b T I)^{-1} u_t}\\
    =&  \frac{1}{T} \sum_{t=1}^T \frac{1}{b} \sum_{l=1}^b \ex{\frac{u_{t, l}^\transpose \left(A^p - z b T I\right)^{-1} u_{t, l}}{1 + u_{t, l}^\transpose \left(A^p - z b T I\right)^{-1} u_{t, l}}} + \approxfunction \cdot \approxmatrix\\
    =&  \outersum \innersum  \ex{\biasapproximator} + \approxmatrix \\
        &+ \outersum \innersum \ex{\approxfunction \abs{\diffapproximator}}\\
    =&  \outersum \biasapproximator + \approxmatrix\\ 
        &+ \approxfunction  \outersum \innersum \ex{\abs{\diffapproximator}}\\
    =&  \outersum \biasapproximator \label{eq:term1}\\ 
        &+ \approxfunction  \ex{\abs{u_r^\transpose \left(A^p - z b T I\right)^{-1} u_r - \ex{u_r^\transpose \left(A^p - z b T I\right)^{-1} u_r}}}  \label{eq:term2} \\&+ \approxmatrix,
\end{align}
where $r$ in the last line is a uniformly randomly selected index from $\set{1, \dots, b T}$ independently to the training process.

\NewDocumentCommand{\approxbias}{O{\boundedby}}{#1{\frac{c \beta}{v p}}}
Note that $S_p(z) = \trace{A^p - z b T I}^{-1}, \ex{S_p(z)} = \ex{\trace{A^p - z b T I}^{-1}}$. So for the term in \cref{eq:term1} we proceed by proving $\frac{1}{b} \sum_{l=1}^{b} \ex{u_{t, l}^\transpose \left(A^p - z b T I\right)^{-1} u_{t, l}}$ approximates $\ex{\trace{A^p -z b T I}^{-1}}$. For convenience let $D \defeq b T\left(A^p - z b T I\right)^{-1}$ be an alias to it, whose spectral norm satisfies $\norm{D}_{\infty} \le b T \frac{1}{ v b T} = \frac{1}{v}$, then
\begin{align}
    &       \abs{\ex{\innersum \innerapproximator} - \ex{S_p(z)}}\\
    =&      c\abs{\frac{\ex{\innersum \innerapproximator[b T \left(A^p - z b T I\right)^{-1}]}}{p} - \frac{ \ex{b T S_p(z)}}{p}}\\
    =&      \frac{c}{p} \abs{\innersum \ex{\innerapproximator[D]} - \ex{\trace{D}}}
    =      \frac{c}{p} \abs{\ex{\trace{\left(\innersum \innerouterproduct - I\right) D}}}\\
    \le&    \frac{c}{p} \ex{\norm{\left(\innersum \innerouterproduct - I\right) D}_1}
    \le    \frac{c}{p} \ex{\norm{\left(\innersum \innerouterproduct - I\right)}_1 \norm{D}_{\infty}}\\
    \le&    \frac{c}{v p}\ex{\norm{\left(\innersum \innerouterproduct - I\right)}_1}
    \le    \frac{c}{v p}  \ex{\sqrt{p \trace{\left(T^p_t - I\right)^\transpose \left(T^p_t - I\right)}}}\\
    =&      \approxbias,
\end{align}
where the last inequality is because
\begin{align}
    \norm{A}_1 =& \sum_{i=1}^{p} \abs{\lambda_i(A)} = \norm{\begin{bmatrix}
        \lambda_1(A) & \cdots & \lambda_i(A) & \cdots \lambda_p(A)
    \end{bmatrix}^\transpose}_1\\
    \le&    \sqrt{p} \norm{\begin{bmatrix}
        \lambda_1(A) & \cdots & \lambda_i(A) & \cdots \lambda_p(A)
    \end{bmatrix}^\transpose}_2\\
    =&  \sqrt{p} \norm{A}_2 = \sqrt{p \trace{A^\transpose A}},
\end{align}
given that $A = \left( \left(\innersum \innerouterproduct - I\right) \right)$ is symmetric so that its singular values are absolute eigenvalues.
With approximation on complex function $\frac{x}{1 + x}$, this $\approxbias$-boundedness implies $\approxfunction \cdot \approxbias = \approxbias$ approximation of in \cref{eq:term1}.

\NewDocumentCommand{\approxvariance}{O{\boundedby}}{#1{\frac{c \alpha}{v p}}}
\NewDocumentCommand{\utdu}{}{u_t^\transpose D u_t}
\NewDocumentCommand{\xtdx}{}{x_r^\transpose D x_r}
For the difference term in \cref{eq:term2}, we prove its diminishment by $\frac{1}{v}$-bounded variance of $u_{t, l}^\transpose (A^p - z b T I)^{-1} u_{t, l}$, or formally
\begin{align}
    \ex{\abs{X - \ex{X}}^2} - \ex{\abs{X - \ex{X}}}^2 =& \ex{\left(\abs{X - \ex{X}} - \ex{\abs{X - \ex{X}}}\right)^2} \ge 0\\
    \ex{\abs{X - \ex{X}}} \le&  \sqrt{\ex{\abs{X - \ex{X}}^2}} = \sqrt{\var{X}},
\end{align}
and 
\begin{align}
    &   \var{\innerapproximator[\left(A^p - z b T I\right)^{-1}][r]}
    =  \frac{c^2}{p^2} \var{\innerapproximator[D][r]}\\
    =&  \frac{c^2}{p^2} \left(\ex{\trace{\innerapproximator[D][r] \innerapproximator[D][r]}} - \ex{\trace{\innerapproximator[D][r]}}^2\right)
    \le    \frac{c^2}{p^2} \alpha^2 \left(\ex{\trace{\xtdx \xtdx}}\right)\\
    \le&   \frac{c^2 \alpha^2}{p^2} \ex{\trace{\xtdx \xtdx}}
    \le    \frac{c^2 \alpha^2}{p^2} \ex{\norm{x^p}_2^4 \norm{D}_\infty^2}
    =  \frac{c^2 \alpha^2}{v^2 p^2},
\end{align}
where the last step follows that $x^p$'s norm is bounded and that $\norm{D}$ is also uniformly bounded. 

\NewDocumentCommand{\approxlargest}{O{\boundedby}}{#1{\approxmatrix[] + \approxbias[] + \approxvariance[]}}
To sum up, we have obtained 
\begin{align}
    \frac{1}{T} \sum_{t=1}^T \ex{u_t^\transpose \left(B^p - z b T I\right)^{-1} u_t} = \frac{\ex{S_p(z)}}{1 + \ex{S_p(z)}} + t,
\end{align}
where $\abs{t} = \approxlargest$.

\NewDocumentCommand{\approxall}{O{\boundedby}}{#1{\approxlargest[] + \frac{c}{v p} + \frac{c \alpha }{v p}}}

With \cref{eq:claim}, \cref{eq:a1}, one can reduce \cref{eq:a3} to
\begin{align}
    p =& T (b + O(1)) \left(\frac{\ex{S_p(z)}}{1 + \ex{S_p(x)}} + t\right) - z b T \left(\ex{S_p(z)} + \boundedby{c / v p}\right),
\end{align}
and
\begin{align}
    \frac{\ex{S_p(z)}}{1 + \ex{S_p(x)}} - z \ex{S_p(z)} =& \frac{p}{b T} + s = c + s,
\end{align}
where $s = \approxall$.


$\ex{S_p(p)}$ always have a non-negative real part because real parts of eigenvalues of $\left(U U^\transpose - z b T I\right)$ are always non-negative by an argument similar to previous ones. Since $\alpha, \beta, c$ and $p$ depend only on $p, b, T$ instead of $v$, $s = \approxall$ is bounded by $\tau / v$ which satisfies $\tau$ is constant w.r.t $v$, $\frac{v_0 + (1 - c)}{\sqrt{2}} > \frac{\tau}{v_0} + 2 \sqrt{c v_0} + 2 \sqrt{\tau}$ and $v \ge v_0 \ge 2 c$. Given $c \in [0, 1]$, by \cref{lemma:bound_of_sti}, there is
\begin{align}
    &   \abs{\ipart{\ex{S_p(v i)}}} \le \abs{\ex{S_p(v i)}} \\
    \le& \approxquadratic{\approxall[]}
\end{align}
for $v \ge v_0$. Bounds using the real part are similarly obtained.

The similar bound for $s_p(z) = (b T / p) S_p(z) = \frac{1}{c} S_p(z)$ is
\begin{align}
    \abs{\ipart{\ex{s_p(v i)}}}
    \le& \frac{1}{c}\approxquadratic{\approxall[]}.
\end{align}

The expected mean $\ex{\overline{\lambda^{ > 0}}}$ of $\frac{1}{b T} U^p \left(U^p\right)^\transpose$'s non-zero eigenvalue is
\begin{align}
    &   \ex{\frac{\trace{\frac{1}{b T} U^p \left(U^p\right)^\transpose}}{\min(b T, p)}}
    =  \frac{\frac{1}{b T}\sum_{k=1}^{b T}\ex{\trace{u_{k} u_k^\transpose}}}{\min(b T, p)}
    =  \frac{\frac{1}{b T}\sum_{k=1}^{b T}\ex{\norm{u_k}_2^2}}{\min(b T, p)}
    \le   \frac{\alpha}{\min(b T, p)}.
\end{align}

Finally, the desired conclusion is obtained through \cref{lemma:sti_and_eigenvalue_ratio} by
\begin{align}
        \ex{\frac{\ex{\overline{\lambda^{>0}} / \sqrt{v}}^2}{\left(\frac{\lambda}{\sqrt{v}}\right)^2 + v}}
    \le    \frac{1}{v} \ex{\overline{\lambda^{>0}}}^2 \abs{\ipart{s_p(v i)}},
    % \le&    \frac{1}{v c} \frac{\alpha^2}{\min(b T, p)^2}  \approxquadratic{\approxall[]},\\
        \ex{\frac{\ex{\overline{\lambda^{>0}}}}{\lambda + \frac{v^2}{\lambda}}}
    \le    \ex{\overline{\lambda^{>0}}} \abs{\rpart{s_p(v i)}}.
    % \le&    \frac{1}{c} \frac{\alpha}{\min(b T, p)}  \approxquadratic{\approxall[]}.
\end{align}

\end{proof}
}

Its proof is left in \cref{proof:spectral_of_accumulated}. This concludes that spectral concentration also happens in $\Eta^{1: T} \left(\Eta^{1:T}\right)^\transpose$ and $X^{1:T} \left(X^{1:T}\right)^\transpose$, as long as the samples in one batch are independently sampled and all model-state-specific samples involved during training are diverse enough to have low anisotropy and norm bounds. To see that the conditions are satisfied, note the only hard conditions are that of continuity, which is hard to verify and thus simply assumed, as well as the choice of $v$ and $v_0$. We assume enough steps have been trained so $c$ can be very small, for example effectively $\frac{768}{32 \times 50,042} \approx 4.8 \times 10^{-4}$ under strong weight decay (it is even smaller when weight decay weakens or disappears) as we shall see in latter paragraphs. So $v_0 = 10^{-3} > 2 c$ can be chosen. In experiments in \cref{sec:t_exp:anisotropy} we will see $\beta / p$ is always smaller than $1$ and $\alpha / p$ can be less than $0.05$ when weight decay is strong and dimension $p$ is large enough. Under these conditions, it can be verified that LHS of \cref{eq:hard_condition} is larger than RHS by at least $0.061$. So the applicability of the theorem depends on how good the bound is. To empirically verify the applicability of \cref{theorem:spectral_of_accumulated}, one needs to compute anisotropy $\ex{\sqrt{p \trace{\left(T^{p} - I\right)^\transpose\left(T^{p} - I\right)}}}$ and $\alpha$ from $x^p$'s marginal distribution, i.e., by mixing hidden features or back-propagated gradients, which is conducted in \cref{sec:t_exp:anisotropy}.

In contrast with conventional asymptotic results with $p \to \infty$ and $p / b T \to c$ in RMT, our theorem gives bounds for non-limiting scenarios. One of its benefits in the context of machine learning is that one often is more interested in training behaviors when the total number $b T$ of training samples increases as the training proceeds while $p$ is held still. 
Nevertheless, the co-increasing scenarios are also of interest, especially in the era of large models \citep{scaling_law}.
In our result, the spectral concentration is measured by the expected fraction between eigenvalues and the expected eigenvalue, but with an extra shadowing parameter $v$ that may shadow small eigenvalues and decreasing $v$ to suppress the disturbance loosens the bound. Fortunately, most $v$ as dominators show up together with $c$ as numerators, the ratio between hidden dimension and number of training samples used which is often extremely small. If $\frac{\alpha}{p}$ is also small and decreases as $p$ is enlarged, as we will show in the experiments, then the bound will be controlled.

It is frustrating to see that the bound diverges as the training step increases due to factor $\frac{1}{c} = \frac{b T}{p}$. However, we consider weight decay as an alleviation to this issue. Weight decay effectively introduces a sliding window which reduces the effective $T$, because vectors out of the window are exponentially decayed and they have little contribution to the sample covariance matrix. For example, let $w$ be the parameter of weight decay, then each column in $\Eta^{1: T}$ or $X^{1:T}$ becomes $\left(\sqrt{1 - \eta_{\mathrm{lr}} w}\right)^{T - t} \eta_{t, l}$ or $\left(\sqrt{1 - \eta_{\mathrm{lr}} w}\right)^{T - t} u_{t, l}$, where $\eta_{\mathrm{lr}}$ is learning rate. Setting $r = 1 - \eta_{\mathrm{lr}} w$, if we consider the tail whose weights' sum is smaller than a threshold $\tau$ as those out of the window, then the window size $k$ needs to satisfy $\sum_{i={k+1}}^{\infty} r^i = r^{k+1} / (1 - r) \le \tau$, where $r$ is used to obtain tighter bounds instead of $\sqrt{r} = \sqrt{1 - \eta_{\mathrm{lr}} w}$ by arguments in \cref{appendix:effective_window_size}. One sufficient condition for this is $k \ge \frac{\ln \tau (1 - r)}{\ln r} - 1$. When $\tau=10^{-3}, \eta=10^{-3}, w = 10^{-1}$, the effective window size is about $161,172$. When $w$ increases to $0.3$, the effective window size becomes $50,057$. As a result, $\frac{1}{c}$ is upperbounded by a constant as the training proceeds if weight decay presents.

There are still gaps between our results and the spectral distribution of $\kkT$ during stochastic training. For example, spectral concentration of $\Eta \Eta^\transpose$ and $X X^\transpose$ hints similar phenomena in their interlaced product $\left(K^{l, T} - K^{l, 0}\right) \left(K^{l, T} - K^{l, 0}\right)^\transpose$, which, however, is not yet formally proved. Moreover, applying \cref{theorem:spectral_of_accumulated} assumes SGD instead of adaptive optimizers is used due to \cref{eq:update_and_large_matrices}. Filling these gaps is left for future works because we have established spectral concentration's empirical supports in \cref{sec:t_exp:spectral_concentration}. 

Finally, we have discussed the spectral concentration of $\kkT$. To put everything together, assuming moderate projection of $\eta^l$ into non-null space of $\kkT$, and considering the increasing trace of $\kkT$ and spectral concentration in $\kkT$, the only way to suppress 
\begin{align}
    \trace{\hessian[\theta_D]} \ge \left(\gamma^l\right)^\transpose \left(\kkT \hadamard M^l\right) \gamma^l = \left(\eta^l\right)^\transpose \kkT \eta^l \ge \lambda r \left(\eta^l\right)^\transpose \eta^l
\end{align}
is to reduce $\norm{\eta^l}_2^2 = \norm{g^l_K \hadamard \gamma^l}_2^2$. To see gradient-sparsity-induced activation sparsity's emergence, given empirically that $\trace{M^l} = \trace{\left(g^l_K\right)^\transpose g^l_K} = \norm{g^l_K}_2^2$ will not decrease during training, the only way to suppress $\norm{\eta^l}_2^2$ is to decrease $\gamma^l$, at least at entries where $g^l_K$ has large magnitudes.
