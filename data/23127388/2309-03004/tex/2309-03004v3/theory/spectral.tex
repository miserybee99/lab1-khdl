\subsection{Spectral Concentration at Initialization}\label{sec:spectral_init}

In this section, we prove that $K^l \left(K^l\right)^\transpose$ has eigenvalues that are close to each other, at least at initialization. With effective gradient sparsity measured by $\norm{\eta^l}_2^2$, this spectral concentration allows us to approximate the first term in RHS of \cref{eq:main} with $\lambda r \left(\eta^l\right)^\transpose \eta^l$, which is almost directly effective gradient sparsity measured in $L_2$ norms. 

To reach this goal, recall Marchenko–Pastur distribution in \cref{theorem:singular_value_of_product_of_random_matrices} that reveals the asymptotic spectral distribution of random matrices' product.
Applying \cref{theorem:singular_value_of_product_of_random_matrices} to $K^l\left(K^l\right)^\transpose$ initialized by Xavier or Kaiming initialization, we obtain \cref{theorem:initial_spectral_properties_of_kkT}.

\begin{theorem}[Initial spectral concentration of $\kkT$]\label{theorem:initial_spectral_properties_of_kkT}
    Assume $n \neq d$. Let $K^l \in \reals^{n \times d}$ be the weight matrix initialized by (Gaussian, uniform, or other distribution-based) Xavier or Kaiming initialization. When $n, d \to \infty$ with $d / n = 1 / c$, the ratio between the largest and smallest \emph{non-zero} eigenvalues of $\kkT$ converges weakly to
    \begin{align}
        \frac{\lambda_1\left(\kkT\right)}{\min_{k: \lambda_k\left(\kkT\right) > 0} \lambda_k\left(\kkT\right)} \le \frac{\left(1 + \sqrt{c}\right)^2}{\left(1 - \sqrt{c}\right)^2}.
    \end{align}
    Regarding zero eigenvalues, if $d > n$, there is no zero eigenvalue, and if $d \le n$, the expected portion of zero eigenvalues is $1 - \frac{1}{c} = 1 - \frac{d}{n}$.
    
\end{theorem}
\begin{proof}
    The initialization methods utilize centered distribution, and thus there is $\ex{K^l_{i, j}} = 0$.

    $\kkT$ differs from $S^p$ of \cref{theorem:singular_value_of_product_of_random_matrices} in 1) the shared standard variance of entries not being $1$, and 2) the scaling factor $\frac{1}{d}$. Since we are only interested in the ratio between eigenvalues, these differences of simultaneous scaling can be ignored.

    By \cref{eq:mp_density}, we can see that the support of eigenvalues is restricted to $[a, b] \cup \set{0}$. As a result, non-zero eigenvalues can only be found in $[a, b]$.

    When $c < 1$, i.e., $d > n$, the support degenerates to $[a, b]$. 
    When $c \ge 1$, the probability to pick a zero eigenvalue is $F(0) = \lim_{u \to 0^+} \int_{0}^{u} f(x) \mathrm{d}x = \lim_{u \to 0^+} \int_{0}^{u} \left(1 - \frac{1}{c}\right) \delta(x) \mathrm{d} x = 1 - \frac{1}{c}$ .
\end{proof}

Note that \cref{theorem:initial_spectral_properties_of_kkT} applies to uniform or other base initialization distribution as long as it is centered and entrywisely independent with the same variance. Since $n, d$ are generally large, even in small model sizes like Small and Base, we believe this lemma applies to common practice. In Base-sized Transformers, it is usually the case where $n = 3072, d = 768$, indicating $1 - \frac{768}{3072} = \frac{3}{4}$ of eigenvalues are $0$, while the rest of them varies up to the ratio of $\frac{\left(1 + \sqrt{\frac{1}{4}}\right)^2}{\left(1 - \sqrt{\frac{1}{4}}\right)^2} = 9$. This is a surprisingly small value compared to the number of dimensions.

Effective gradient sparsity patterns have a great affinity to $\kkT$, allowing \cref{theorem:initial_spectral_properties}.

\begin{theorem}[Implication of spectral concentration of $\kkT$ at initialization]\label{theorem:initial_spectral_properties}
    Assume $d \neq n$ and they are sufficiently large. $K^l \in \reals^{n \times d}$ is initialized as in \cref{theorem:initial_spectral_properties_of_kkT}. Let $M^l = g^l_K \left(g^l_K\right)^\transpose$, $\gamma^l$ and $\eta^l$ be those defined previously. 
    
    If $d > n$ then there is
    \begin{align}
        \left(\gamma^l\right)^\transpose \left(\left(K^l \left(K^l\right)^\transpose\right) \hadamard M^l\right) \gamma^l
        =& \left(\eta^l\right)^\transpose \kkT \eta^l\\
        \ge&    \lambda_{n}\left(\kkT\right) \cdot \norm{\eta^l}_2^2,
    \end{align}
    where $n$-th eigenvalue $\lambda_n\left(\kkT\right)$ is moderate and cannot be arbitrarily small because
    \begin{align}
        \frac{\lambda_1}{\lambda_{n}} \le \left(\frac{1 + \sqrt{c}}{1 - \sqrt{c}}\right)^2,
    \end{align}
    where $c = d / n$.

    If $n > d$, let $K^l = U \Sigma V^\transpose$ be the singular value decomposition of $K^l$. The result is restricted to the projection to the subspace expanded by $\left(U^\transpose\right)_{1:d}$, i.e.,
    \begin{align}
        \left(\gamma^l\right)^\transpose \left(\left(K^l \left(K^l\right)^\transpose\right) \hadamard M^l\right) \gamma^l
        =& \left(\eta^l\right)^\transpose \kkT \eta^l\\
        \ge&    \lambda_{d}\left(\kkT\right) \cdot \norm{\left(U^{T}\right)_{1: d}\eta^l}_2^2,
    \end{align}
    where $d$-th eigenvalue $\lambda_d\left(\kkT\right)$ satisfies
    \begin{align}
        \frac{\lambda_1}{\lambda_{d}} \le \left(\frac{1 + \sqrt{c}}{1 - \sqrt{c}}\right)^2,
    \end{align}
    where $c = d / n$.

    For demonstration, when $\set{n, d} = \set{3072, 768}$, the ratio upperbound is $9$.
\end{theorem}
\begin{proof}
    The proof is straightforward after \cref{theorem:initial_spectral_properties_of_kkT}, by noting that when $n > d$ there are exactly $\left(1 - \frac{1}{c}\right) n = n - d$ zero eigenvalues in $\kkT$, or equivalently $d$ non-zero eigenvalues in $\kkT$.
\end{proof}

There are still gaps between $\lambda \left(\eta^l\right)^\transpose \eta^l$ and current practices where $d < n$ and there are a lot of zero eigenvalues in $\kkT$, but \cref{theorem:initial_spectral_properties} is perfectly useful for wide MLPs where $d > n$. Therefore, we propose a drastic architectural modification called wide MLP where $d > n$, i.e., the model dimension is larger than the hidden dimension in MLP blocks. Aside from more powerful motivation toward sparsity, wide MLPs also allow rows in $K^l$ to be mutually orthogonal and permit perfect sparsity, which is impossible when $n > d$.
 
In non-wide MLPs, we believe that since $\kkT$ are randomly initialized and samples are randomly selected, there are moderate projections of $\eta^l$ into the non-null subspace of $\kkT$. Another supporting intuition is that although $\sigma'$ is not identity or linear, the derivatives of common activation functions are often monotonically increasing and form an approximation to its inputs. This approximation is better when part of the activation derivatives are linear, as in the case of Squared-$\relu$\citep{primer} and our $\jrelu$. Therefore, taking $\sigma'$ to $K^l x + b^l$, which already falls near the subspace expanded by $\left(U^\transpose\right)_{1: d}$, does not deviate far from the subspace. 
\cref{obs:memorizing_eff} also supports this moderate projection dynamically because if $\eta^l$ is in the null space, it will be borne into every column of the key matrix and next time it will have non-zero projections if $\eta^l$ does not change too much after one epoch and the column memory is not blurred too severely. Repeatedly memorizing different $\eta^l$ will make the non-null space of $\kkT$ a mixture of the majority $\eta^l$s provided by the training data set. 
The empirical evidence for this is that $\left(\eta^l\right)^\transpose \kkT \eta^l$, according to the derivation in \cref{lemma:adversarial_and_sparsity}, is actually the norm of gradients back propagated to shallower layers. Extreme cases where $\eta^l$ are contained only in the null space of $\kkT$ result in zero gradients for shallower layers, which rarely happens. If no residual connection is involved this insight strongly augments the spectral explanation. The detailed and formal analysis of the zero eigenvalues especially when there are residual connections is left for future empirical and theoretical works. For now, we can simply cover the gap with wide MLPs.

\subsection{Spectral Concentration during Stochastic Training}\label{sec:spectral_training}

In this subsection, we discuss how spectral concentration re-emerges during later stochastic training. 

First recall the Marchenko-Pastur distribution in \cref{theorem:singular_value_of_product_of_random_matrices}. The condition of random centered matrices in \cref{theorem:singular_value_of_product_of_random_matrices} invites another randomness other than initialization to the party, i.e., stochastic gradient noise (SGN) brought by stochastic optimizers.
After $t$ updates, $K^{l}$ can be written as the sum of random initialization, stochastic gradient noises and full-batch gradients that are not as stochastic as the two former terms, i.e.
\begin{align}
    K^{l, t} = \underbrace{K^{l, 0} - \sum_{i=1}^{t} U^{i}_{K^l}}_{\text{stochastic, centered}} - \sum_{i=1}^t \derivatives{\loss(\theta^t)}{K^l}
\end{align}
As discussed in \cref{sec:flat_minima}, $U^{i}_{K^l}$ is by definition centered. If it can be assumed that the $\sas$ SGN with large variance and noise norm shadows full-batch gradient, then $K^{l, t}$ is the sum of two centered random matrix with a slight non-stochastic bias, to which Marchenko-Pastur distribution would approximately apply if further entries in $U^{i}_{K^l}$ shared similar variance and were sampled independently. \citet{relax_mp_distribution} and works cited by them have tried to relax the independence condition of \cref{theorem:singular_value_of_product_of_random_matrices}, but it is still far from applying relaxed Marchenko-Pastur distribution here. Aside from waiting for this mathematical progress, we build empirical basis in \cref{sec:t_exp:spectral_concentration} where at all steps, in $\kkT$ of ViT and the decoder of T5, there is a stable portion of near-zero eigenvalues as in \cref{theorem:initial_spectral_properties} across all layers, and the majority of non-zero ones, with significant gap with near-zero ones, vary up to a ratio of $<100$ for most of the time. It is not surprising that this effect empirically sustains and even becomes stronger at the end of training because the model is well-learned by then and the full-batch gradient is of a smaller norm.

\NewDocumentCommand{\DKDKT}{O{t}}{\DK{#1}\left(\DK{#1}\right)^\transpose}
\NewDocumentCommand{\DKTDK}{O{t}}{\left(\DK{#1}\right)^\transpose \DK{#1}}

To have a more satisfying discussion, we propose an extended version of Marchenko-Pastur distribution and find a re-directed view on stochastic gradients to apply it. To this end, what conditions and assumptions can stochastic training provide must be figured out first.

Observing the structure of $\mlp$ or $\dbmlp$ layers, the most essential operation involving weight matrix $K^l$ is
\begin{align}
    z^l \defeq K^l x^l,
\end{align}
where $x$ is abused to represent anything that is multiplied with $K^l$, abstracting $x^l$ or $x^l + d^l$, while $z^l$ is the vector passed to the vanilla bias, activation function or later layers. This structure gives birth to the update of a sample $(x_s, y_s)$ to $K^l$ that writes
\NewDocumentCommand{\DK}{m}{\Delta K^{l, #1}}
\begin{align}
    \DK{s}
    =&  -\eta_{\mathrm{lr}} \cdot \derivatives{\loss(\theta, (x_s, y_s))}{z^{l, s}} \times (x^{l, s})^\transpose
    =  -\eta_{\mathrm{lr}} \cdot \eta^{l, s} \left(x^{l, s}\right)^\transpose,
\end{align}
where $\eta_{\mathrm{lr}}$ is the learning rate, assuming no scheduling is used.
At step $t$ with batch $B_t$, the update on $K^l$ averages these samplewise differences, i.e.
\begin{align}
    \DK{t}
    \defeq& - \frac{\eta_{\mathrm{lr}}}{\size{B_t}} \sum_{s \in B_t} \eta^{l, s} \left(x^{l, s}\right)^\transpose
    =  - \frac{\eta_{\mathrm{lr}}}{\size{B_t}} H^t \left(X^t\right)^\transpose,
\end{align}
where $\Eta^t \in \reals^{n \times \size{B_t}}$ (capitalized ``$\eta$'') is the matrix consisting of column vectors $\eta^{l,s}$ for sample $s$ in the batch $B_t$, and $X^t \in \reals^{d \times \size{B_t}}$ is similarly constructed with $x^{l,s}, s \in B_t$. 
Note that $X^t$ and $\Eta^t$ are random matrices because samples are independently randomly selected and gradients are also random variables as functions of variables.
Taking a similar view throughout the training, there is
\begin{align}
    K^{l, T} - K^{l, 0}
    =&  -\eta_{\mathrm{lr}} \sum_{t=1}^T \frac{1}{\size{B_t}} \sum_{s \in B_t} \eta^{l, s} \left(x^{l, s}\right)^\transpose
    =  -\frac{\eta_{\mathrm{lr}}}{b} \Eta^{1: T} \left(X^{1:T}\right)^\transpose \label{eq:update_and_large_matrices},
\end{align}
where $T$ is the number of batches, $b$ is batch size, and $\Eta^{1:T} \defeq \begin{bmatrix}  \Eta^1 & \cdots & \Eta^t & \cdots & \Eta^T  \end{bmatrix}$, and $X^{1:T} \defeq \begin{bmatrix} X^1 & \cdots & X^t & \cdots & X^T \end{bmatrix}$.
Another product of large random matrices emerges in the empirical covariance matrix of the difference, i.e.,
\begin{align}
    \left(K^{l, T} - K^{l, 0}\right) \left(K^{l, T} - K^{l, 0}\right)^\transpose = \frac{\eta_{\mathrm{lr}}^2}{b^2} \Eta^{1: T} \left(X^{1:T}\right)^\transpose X^{1:T} \left(\Eta^{1:T}\right)^\transpose,
\end{align}
where $X^{1:T}$ and $\Eta^{1:T}$ are random matrices in the sense that samples or gradient vectors in each batch are independently randomly sampled, if conditioned on the model state.

Since we are interested in spectral distribution and that cycling a matrix product does \emph{not} change non-zero eigenvalues, a more desirable form is
\begin{align}
    \left(\Eta^{1:T}\right)^\transpose \Eta^{1:T} \left(X^{1:T}\right)^\transpose X^{1:T},
\end{align}
and we intend to separately investigate the spectral distributions of 
\begin{align}
    &\text{$\left(\Eta^{1:T}\right)^\transpose \Eta^{1:T}$ or spectrally equivalent $\Eta^{1:T} \left(\Eta^{1:T}\right)^\transpose$},\\
    \text{and, }&\text{$\left(X^{1:T}\right)^\transpose X^{1:T}$ or spectrally equivalent $X^{1:T} \left(X^{1:T}\right)^\transpose$}.
\end{align}

After these transforms and dividing-and-conquering, the empirical covariance matrices look ready for Marchenko-Pastur law. However, there are dependencies between previous and later batches through model states, hindering the direct application of independence conditions of Marchenko-Pastur law. Fortunately, there is still conditional independence \emph{within} a batch. This mixture of dependence and independence is captured by \cref{def:batch_model}.
\begin{definition}[Batch Dependence Model]\label{def:batch_model}
    Let $U^{1: T} \in \reals^{p \times (b T)}$ be a random matrices. Decompose it into blocks with batch size $b$, i.e., 
        \begin{align}
            U^{1: T} = 
                \begin{bmatrix}
                    U^1 & \cdots & U^t & \cdots & U^{T-1} & U^{T}
                \end{bmatrix},
        \end{align}
    where $U^t \in \reals^{p \times b}$.
    If the dependence between elements can be described by SCMs
    \begin{align}
        \set{u^t_k \defeq g^t\left(U^{1}, U^{2}, \dots, U^{t-1}, \epsilon^t_k\right): t \in [1, T], k \in [1, b]}
    \end{align}
    or SCMs that resemble the notions of samples and model state
    \begin{align}
        \set{u^t_k \defeq g^t\left(m^{t-1}, \epsilon^t_k\right) : t \in [1, T], k \in [1, b]} \cup \set{m^t \defeq h^t\left(m^{t-1}, U^{t}\right)}
    \end{align}
    where $u^t_k$s are columns in $U^t$, $m^{t}$ is the model state (parameters, momentum, etc.) after step $t$, $\epsilon^{t}_k$ is I.I.D. random noises, then $U^{1: T}$ is a random matrix with batch dependence.
    \begin{remark}
        Within each batch (i.e., when conditioned on all previous batches), samples are I.I.D. sampled to simulate batch sampling. However, previous samples have trained the parameters and will shift the distribution of shallow layers's output as well as back-propagated gradients. Therefore, the current batch depends on previous batches.
    \end{remark}
\end{definition}

There are works re-establishing Marchenko-Pastur law with independence conditions relaxed to martingale conditions \citep{mp_martingale}, but some conditions in it require entrywise conditional independence. There are also Marchenko-Pastur laws for time series \citep{mp_linear_time_series1,mp_linear_time_series2}, but restricted to linear dependence. 
We adapt proofs by \citet{mp_quadratic_form}, and use anisotropy condition in all samples or gradients to extend Marchenko-Pastur distribution under the batch dependence in $X^{1:T}$ and $\Eta^{1:T}$, leading to \cref{theorem:spectral_of_accumulated}.
In later formal definitions, theorems and proofs, $X$ is abstractly used to represent both $\Eta^{1: T}$ or $X^{1:T}$, $p$ indicates the height of $X^{1:T}$ or $\Eta^{1: T}$, i.e., the hidden dimensions $d$ or $n$, while $b$ and $T$ keep their meanings as batch size and total number of batches.

\def\StateSpectralOfAccumulated{display}
\ifdefstring{\StateSpectralOfAccumulated}{display}{

\begin{restatable}[Spectral concentration of accumulated steps]{theorem}{SpectralOfAccumulated}
    \label{theorem:spectral_of_accumulated}
    Let $X^p = X^{p, b T} \in \reals^{p \times b T}$ be a random matrix that forms a Batch Dependence Model as in \cref{def:batch_model} with batch size $b$ and step count $T$, whose columns are $x^p_j = X^{p, b T}_{\cdot, j} \in \reals^{p}$. 
    Columns in $X^p = X^{p, b T} \in \reals^{p \times b T}$ are \emph{not} necessarily independent.
        
    Let $x_k^p \defeq X^{p}_{\cdot, k}$ be the $k$-th column of $X^p$ and $x_{t, l}^p \defeq X^{t}_{\cdot, l}$ be the $l$-th column of batch $t$ or equivalently the $k=\left((t-1)*b + l\right)$-th column $x_k$ in $X^{p}$. Superscription $p$ may be dropped for convenience.

    Let $S^{p} \defeq \frac{1}{b T} X^p \left(X^p\right)^\transpose = \frac{1}{b T} \sum_{k=1}^{b T} x_k x_k^\transpose$ be the empirical covariance matrix of all random vectors, and $I_p$ be the compatible identity matrix.
    Assume $x_{t, l}$s' norm is bounded, say by $1$, and scale it with 
    \begin{align}
        u^p_{t, l} \defeq \sqrt{\alpha} x^p_{t, l},
    \end{align}
    obtaining $U^p = U^{p, b} \defeq \begin{bmatrix} U^1 & \cdots &  U^t & \cdots  & U^T \end{bmatrix} = \sqrt{a} \cdot X^p$, where $a \defeq \frac{\trace{S^p}}{\trace{S^p S^p}}$. Let 
    \begin{align}
        T^{p} \defeq \frac{1}{b T} U^p \left(U^p\right)^\transpose = \frac{1}{b T} \sum_{k=1}^{b T} u_{k} u_{k}^\transpose = a \cdot S^{p}
    \end{align}
    be the empirical covariance matrix of all $u^p_{k}$s. 

    Assume $a$ is bounded by $\alpha(p)$ and $\ex{\sqrt{p \trace{\left(T^{p} - I^p\right)\left(T^{t} - I^p\right)}}}$ is also upperbounded by $\beta(p)$.

    Further assume that the following function of $z \in \positivecomplex$ and $p, b, T \in \nats^+$
    \begin{align}
        \ex{\sum_{i} \frac{1}{\lambda_i\left(U U^\transpose\right) - z}}
    \end{align}
    is always continuous w.r.t. $z$ for any $p, b, T$.

    If the above assumptions are satisfied, the non-zero eigenvalue concentrates. To be more specific, let $\overline{\lambda^{>0}}$ be the mean of non-zero eigenvalues of $\frac{1}{b T} U^p \left(U^p\right)^\transpose$ and use $\ex{\overline{\lambda^{>0}}}^2$ to represent the overall situation of non-zero eigenvalues. Then there is
    \begin{align}
        \ex{\frac{\ex{\overline{\lambda^{>0}} / \sqrt{v}}^2}{\left(\frac{\lambda}{\sqrt{v}}\right)^2 + v}}
        \le&    \frac{\sqrt 2}{c \sqrt{v}} \frac{\alpha^2}{v \cdot \min(b T, p)^2}  \sqrt{c + \frac{\left(2 \sqrt{2} + 2\right) c \alpha}{v p} + \frac{c \beta}{v p} + \frac{c}{v p}} \label{eq:im_bound},\\
        \ex{\frac{\ex{\overline{\lambda^{>0}}}}{\lambda + \frac{v^2}{\lambda}}}
        \le&    \frac{\sqrt 2}{c \sqrt{v}} \frac{\alpha}{\min(b T, p)}  \sqrt{c + \frac{\left(2 \sqrt{2} + 2\right) c \alpha}{v p} + \frac{c \beta}{v p} + \frac{c}{v p}} \label{eq:re_bound}.
        \end{align}
    for any $v \ge v_0$, where $\lambda$ is a randomly selected eigenvalue of $T^p \defeq \frac{1}{b T} U^p \left(U^p\right)^\transpose$ and $c = p / b T \in [0, 1]$, and $v_0 \ge 2 c$ satisfying 
    \begin{align}
        \frac{v_0 + (1 - c)}{\sqrt{2}} > \frac{\tau}{v_0} + 2 \sqrt{c v_0} + 2 \sqrt{\tau} \label{eq:hard_condition}
    \end{align}
    with $\tau \defeq \frac{c}{p} \left(1 + \beta + 2\left(\sqrt{2} + 1\right) \alpha\right)$.
\end{restatable}

}{

% \arxivonly{\SpectralOfAccumulated*}
\begin{proof}[Proof of \cref{theorem:spectral_of_accumulated}]\label{proof:spectral_of_accumulated}

The proof is adapted from \citet{mp_quadratic_form} where independence conditions are replaced with Batch Dependence model and new regularities.


Cauchy-\sti{} transform method is used. 
When applied to empirical spectral density, by definition there is
\begin{align}
    s^{F^A}(z) = \trace{A - z I}^{-1} / p \defeq \trace{\left(A - z I\right)^{-1}} / p.
\end{align}
for positive semi-definite $A \in \reals^{p \times p}$.
Specifically, $\frac{1}{b T} U^p \left(U^p\right)^\transpose = \frac{1}{b T} U U^\transpose$'s \sti{} transform is
\begin{align}
    s_p(z) = \trace{\frac{1}{b T} U U^\transpose - z I}^{-1} / p = b T / p \trace{U U^\transpose - z b T I}^{-1}.
\end{align}
% By \sti{} continuity theorem \citep{RMT_book}, it is sufficient to show that $s_p(s) \asto s(z)$ for all $z \in \positivecomplex$, where $s$ is the \sti{} transform of Marchenko-Pastur distribution with parameter $p$ and $n$. To this end, typical steps include the following steps \citep{mp_proof_sketch}:
% \begin{itemize}
    % \item $s_p(z) - \ex{s_p(z)} \asto 0$, by a martingale argument;
    % \item $\ex{s_p(z)} \to s(z)$.
% \end{itemize}

\NewDocumentCommand{\boundedby}{m}{\varXi\left(#1\right)}
To ease presentation, we define $\boundedby{g}$ to indicate (complex) functions whose magnitudes are bounded by positive real function $g$, i.e.,
\begin{align}
    h \in \boundedby{g} \iff \forall x, y, \abs{h(x, y)} \le g(x),
\end{align}
where $y$ indicates variables other than $x$ that $h$ relies.
$\boundedby{\cdot}$ will be used combined with ``$=$'' imitating $O(\cdot)$. Since $\boundedby{\cdot}$ does not hide constant scaling factors and biases in it, unlike $O(\cdot)$ it can be freely added, averaged, multiplied and divided, i.e.,
\begin{align}
    \boundedby{g_1} + \boundedby{g_2} \in& \boundedby{g_1 + g_2},
    \frac{1}{n} \sum_{i=1}^n \boundedby{g_i} \in \boundedby{\frac{1}{n}\sum_{i=1}^n g_i},\\
    \boundedby{g_1} \cdot \boundedby{g_2} \in& \boundedby{g_1 \cdot g_2},
    \frac{\boundedby{g_1}}{g_2} \in \boundedby{\frac{g_1}{g_2}}.
\end{align}

% For $s_p(z) - \ex{s_p(z)}$'s almost sure convergence, \citet{mp_quadratic_form} refer to \citet{mp_proof_sketch}, which we will repeat here to examine and replace independence assumptions within.

% \NewDocumentCommand{\condiex}{O{k} m}{\ex[#1]{#2}}
% Let $\condiex[k]{\cdot} \defeq \ex{\cdot \mid U^p_{\cdot, 1: k}}$ denote the conditional expectation given $U^p_{\cdot, 1}, \dots, U^p_{\cdot, k}$. Then $s_p(z) = \condiex[b]{s_p(z)}$ and $\ex{s_p(z)} = \condiex[0]{s_p(z)}$, and there is
% \begin{align}
    % s_p(z) - \ex{s_p(z)}
    % =&  \sum_{k=1}^{b} \left(\condiex[k]{s_p(z)} - \condiex[k-1]{s_p(z)}\right)
    % =  \sum_{k=1}^{b} \gamma_k,
% \end{align}
% where $\gamma_k \defeq \condiex[k]{s_p(z)} - \condiex[k-1]{s_p(z)}$. Note that $\set{\condiex[k]{s_p(z)}}_k$ is already a Doob martingale, so $\set{\gamma_k}_k$, as its difference, is a sequence of martingale differences.
        
% \NewDocumentCommand{\invR}{O{1}}{\left(R^p_k\right)^{-#1}}
% Let $R^p_{k} \defeq \frac{1}{b T} \sum_{k'} \uut[k'] - z I - \frac{1}{b T}u_k u_k^\transpose$. By \cref{lemma:3.1_from_mp_quadratic_form}(1), $R^p_{k}$ is invertible, and by Sherman-Morrison formula there is
% \begin{align}
    % s_p(z)
    % =&  \trace{R^p_k + \frac{1}{b T} u_k u_k^\transpose }^{-1} / p
    % =  \frac{1}{p} \trace{\invR - \frac{\invR u_k u_k^\transpose \invR / b T}{1 + u_k^\transpose \invR u_k / b T}}\\
    % =&  \frac{1}{p} \left(\trace{\invR} - \frac{u_k^\transpose \invR[2] u_k}{b T + u_k^\transpose \invR u_k}\right).
% \end{align}
% By minor single sample assumption, $\condiex[k]{\trace{\invR}} = \condiex[k-1]{\trace{\invR}}$ is bounded by $\beta(z)$. Regarding the second term,
% \begin{align}
    % &   \abs{\frac{u_k^\transpose \invR[2] u_k}{b T + u_k^\transpose \invR u_k}}\\
    % =&      \frac{\abs{\trace{u_k u_k^\transpose \invR \invR}}}{\abs{b T + u_k^\transpose \invR u_k}}
    % \le    \frac{\norm{u_k u_k^\transpose \invR \invR}_1}{b T + u_k^\transpose \invR u_k}\\
    % \le&   \frac{\norm{u_k u_k^\transpose \invR}_1 \norm{\invR}_{\infty}}{b T + u_k^\transpose \invR u_k}
    % \le    \frac{1}{v} \frac{u_k^\transpose \invR u_k}{b T + u_k^\transpose \invR u_k}
    % \le \frac{1}{v},
% \end{align}
% where the third inequality follows \cref{lemma:3.1_from_mp_quadratic_form}(1).
% Therefore $\gamma_k$ can be bounded by
% \begin{align}
    % \gamma_k \le& \frac{2}{v} \frac{1}{p} + \beta(z) < \infty.
% \end{align}
% By Azuma's inequality, there is
% \begin{align}
    % \prob{\abs{s_p(z) - \ex{s_p(z)}} > \epsilon} \le 2 \exp\left(-\frac{\epsilon^2}{2 p \left(\frac{2}{v} \frac{1}{p}\right)^2}\right) \le 2 \exp\left(-\epsilon^2 v^2 p / 8\right).
% \end{align}
% Given that $\prob{\abs{s_p(z) - \ex{s_p(z)}}}$ decays exponentially with $p$, for every $\epsilon > 0$, there is
% \begin{align}
    % \sum_{p=0}^\infty \prob{\abs{s_p(z) - \ex{s_p(z)}} > \epsilon} < \infty
% \end{align}
% which implies $s_p(z) - \ex{s_p(z)} \asto 0$ (Theorem 7.5 by \citet{exponential_decay_to_as}).

Consistent with final conclusion, fix $z = 0 + v i$ ($v \in \reals^+$) such that $v \ge v_0$ throughout the proof.
Define $A^p \defeq \sum_{k} \uut$.
Sample an auxiliary vector $u_{T, b+1} = u_{b T  + 1} \in \reals^p$ so that it is sampled from the conditional distribution given the first $T-1$ batches but it is conditionally independent with other samples in $U^T$, i.e., an extra sample for the last batch. This dependence relation can be expressed by only adding edges $U^{1: T-1} \to u_{T, b+1}$ to the SCMs of the Batch Dependence Model. With the auxiliary vector, define $B^p \defeq A^p +  \uut[T, b+1]$. %Define $C^b_t \defeq B^p - u^t \left(u^t\right)^\transpose$

By \cref{lemma:3.1_from_mp_quadratic_form}(1), $B^p - z b T I$ is non-degenerate and
\begin{align}
    p 
    =&  \trace{\left(B^p - z b T I\right) \left(B^p - z b T I\right)^{-1}}\\
    =&   \sum_{t=1}^{T} \sum_{l=1}^{b + \indic{t = T}} u_{t, l}^\transpose \left(B^p - z b T I\right)^{-1} u_{t, l} - z b T \trace{B^p - z b T I}^{-1}.
\end{align}
Taking expectations and using the exchangeability within each batch give
\begin{align}
    p = \sum_{t=1}^{T} (b + \indic{t = T}) \ex{u_t^\transpose \left(B^p - z b T I\right)^{-1} u_t} - z b T \ex{\trace{B^p - z b T I}^{-1}} \label{eq:a3}.
\end{align}

Define $S_p(z) \defeq \trace{A^p - z b T I}^{-1}$ and note that $S_p(z) = (p / b T) s_p(z)$. 

By \cref{lemma:3.1_from_mp_quadratic_form}(2), there is
\begin{align}
    \ex{\trace{B^p - z b T I}^{-1}} =& \ex{S_p(z)} + \boundedby{1/ v b T} = \ex{S_p(z)} + \boundedby{c / v p} \label{eq:a1}.
\end{align}

\NewDocumentCommand{\approxmatrix}{O{\boundedby} O{2}}{#1{\frac{#2 \sqrt{2} c \alpha}{v p}}}
We now prove
\begin{align}
    \frac{1}{T} \sum_{t=1}^{T} \ex{u_t^\transpose \left(B^p - z b T I\right)^{-1} u_t} = \frac{\ex{S_p(z)}}{1 + \ex{S_p(z)}} + t \label{eq:claim}, 
\end{align}
where $\abs{t}$ is bounded by a function of $c, \alpha, \beta, v, p$.

\NewDocumentCommand{\approxfunction}{O{\boundedby}}{#1{1}}
A complex function $\frac{x}{1 + x} = 1 - \frac{1}{x + 1}$ emerges many times. We will approximate it to the first order so its complex derivative should be computed and bounded.
\begin{align}
    \abs{\left(\frac{x}{1 + x}\right)'}
    =&  \abs{\frac{1}{(x+1)^2}} 
    = \frac{1}{\abs{x + 1}^2}
\end{align}
Therefore, if $x_1, x_2$ both stay away from $-1$, then $\abs{\left(\frac{x'}{1 + x'}\right)'} = \approxfunction$ on the line connecting $x_1, x_2$ and we can approximate $\frac{x_2}{1 + x_2}$ by $\frac{x_1}{1 + x_1} + \boundedby{1} \cdot \Delta x = \frac{x_1}{1 + x_1} + \boundedby{\Delta x}$, where $\Delta x = x_2 - x_1$.
In latter application, $x$, both the start and the end of approximation, is often of form $\frac{1}{n} \sum_{i=1}^n \ex{u_i^\transpose \left(C - z b T I\right)^{-1} u_i}$ possibly with averaging or expectation missing, where $C$ is real symmetric positive semi-definite and $u_i$ is a real vector. The eigenvalues in $\left(C - z b T I\right)^{-1}$ are
\begin{align}
    \frac{1}{\lambda_i(C) - v b T i}
    =&  \frac{\lambda_i(C) + v b T i}{\lambda_i(C)^2 + (v b T)^2},
\end{align}
whose real part is
\begin{align}
    \rpart{\frac{1}{\lambda_i(C) - v b T i}}
    =&  \frac{\lambda_i(C)}{\lambda_i(C)^2 + (v b T)^2} \ge 0.
\end{align}
As a result, the real part of inner products is always non-negative and $x$ stays away from $-1$, and the magnitude of derivatives is $\approxfunction$.

Another approximation is done between $C^p_k$ and $A_p$, whose difference is the outer products of a constant number of random vectors, and it should be minor considering there are $b T$ of them. Formally, for real symmetric positive semi-definite $C$ with eigenvalue decomposition $C = V \Lambda V^\transpose$ by real matrices $V$ and $\Lambda$, $(C - z I)$ can be decomposed to $(C - z I) = V \left(\Lambda - z I\right) V^\transpose$, and non-degenerate $\left(C - z I\right)^{-1}$ to $\left(C - z I\right)^{-1} = V \left(\Lambda - z I\right)^{-1} V^\transpose \defto V \Sigma \Sigma V^\transpose$ where $\Sigma \defeq \sqrt{\left(\Lambda - z I\right)^{-1}}$. Let $S \defeq V \Sigma V^\transpose$ to have $S^\transpose S = S S = \left(C - z I\right)^{-1}$. After that, there is
\begin{align}
    &   \abs{y^\transpose \left(C + x x^\transpose - z I\right)^{-1}y - y^\transpose \left(C - z I\right)^{-1} y}
    =  \abs{y^\transpose \left(\left(C + x x^\transpose - z I\right)^{-1} - \left(C - z I\right)^{-1} \right) y}\\
    =&  \abs{\frac{
            y^\transpose \left(C - z I\right)^{-1} x x^\transpose \left(C - z I\right)^{-1} y
        }{1 + x^\transpose \left(C - z I\right)^{-1} x}}
    =   \abs{\frac{
            \left(y^\transpose S^\transpose S x\right) \left(x^\transpose S^\transpose S y\right)
        }{1 + x^\transpose \left(C - z I\right)^{-1} x}}
    =  \abs{\frac{
            \left(a^{\transpose} \bar{b}\right) \left(\bar{b}^\transpose a\right)
        }{1 + x^\transpose \left(C - z I\right)^{-1} x}}\\
    =&   \frac{
            \abs{a^* b} \abs{a^* b}
        }{\abs{1 + x^\transpose \left(C - z I\right)^{-1} x}}
    \le \frac{
            \norm{a^*}_2 \norm{b}_2 \norm{a^*}_2 \norm{b}_2
        }{\abs{1 + x^\transpose \left(C - z I\right)^{-1} x}}
    =   \frac{
            \abs{a^* a} \abs{b^* b}
        }{\abs{1 + x^\transpose \left(C - z I\right)^{-1} x}}\\
    =&  \frac{
            \abs{\trace{y y^\transpose S^* S}} \abs{b^* b}
        }{\abs{1 + x^\transpose \left(C - z I\right)^{-1} x}}
    \le \frac{
            \norm{y y^\transpose S^* S}_1 \abs{b^* b}
        }{\abs{1 + x^\transpose \left(C - z I\right)^{-1} x}}
    \le \frac{
            \norm{y y^\transpose}_1 \norm{S^* S}_\infty \abs{b^* b}
        }{\abs{1 + b^\transpose b}}
    =   \frac{\norm{y}_2^2}{\ipart{z}} \frac{
            \abs{b^* b}
        }{\abs{1 + b^\transpose b}},
\end{align}
where $a \defeq S y, b \defeq \bar{S} \bar{x} = \bar{S} x$, the second step is from Sherman-Morrison formula, and the second last inequality is due to \cref{lemma:abs_trace_and_schatten_1}. The fact, that $S^* S$ is positive semi-definite whose largest eigenvalue is smaller than the upperbound $\frac{1}{v}$ of $S^\transpose S$'s eigenvalue magnitude, is also used.  To bound the fraction between $\abs{b^* b}$ and $\abs{1 + b^\transpose b}$, recall the eigenvalue decomposition on $\left(C - z I\right)^{-1}$ 
\begin{align}
    \left(C - z I\right)^{-1} = V \Sigma \Sigma^\transpose V^\transpose
\end{align}
and $S = V \Sigma^\transpose V^\transpose$. Then 
\begin{align}
    b^\transpose b &= v^\transpose \Sigma \Sigma v,
    b^* b = v^\transpose \bar{\Sigma} \Sigma v,
\end{align}
where $v \defeq V^\transpose x$ is a real vector. Notice that $\Sigma \Sigma = \diag{\frac{1}{\lambda_i(C) - v i}} = \diag{\frac{\lambda_i(C) + v i}{\lambda_i(C)^2 + v^2}}$ where both real and imaginary parts are non-negative, and that $\Sigma^* \Sigma = \diag{\frac{\abs{\lambda_i(C) + v i}}{\lambda_i(C)^2 + v^2}}$. With this, the inner products are simplified to
\begin{align}
    b^\transpose b &= \sum_{i} \frac{v_i^2 \lambda_i(C)}{\lambda_i(C)^2 + v^2} + i \sum_{i} \frac{v_i^2 v}{\lambda_i(C)^2 + v^2},
    b^* b = \sum_{i} \abs{\frac{v_i^2}{\lambda_i(C)^2 + v^2} \lambda_i(C) + i \frac{v_i^2 v}{\lambda_i(C)^2 + v^2}}
\end{align}
Representing complex numbers by 2-dimensional vectors $w_i \defeq \begin{bmatrix} \frac{v_i^2 \lambda_i(C)}{\lambda_i(C)^2 + v^2} & \frac{v_i^2 v}{\lambda_i(C)^2 + v^2} \end{bmatrix}^\transpose$, there are
\begin{align}
    \abs{b^\transpose b} &= \norm{\sum_i w_i}_2,
    \abs{b^* b} = \sum_i \norm{w_i}_2.
\end{align}
Noting that all entries of $w_i$'s are non-negative, there is
\begin{align}
    \abs{b^* b}
    &=  \sum_i \norm{w_i}_2
    \le \sum_{i} \norm{w_i}_1 
    =   \norm{\sum_i w_i}_1 
    \le \sqrt{2} \norm{\sum_i w_i}_2 = \sqrt{2} \abs{b^\transpose b}.
\end{align}
So $\frac{\abs{b^* b}}{\abs{b^\transpose b}} \le \sqrt{2}$. Given that the real part of $b^\transpose b$ is non-negative, adding $1$ will only increase its magnitude. As a result, there is
\begin{align}
    &   \abs{y^\transpose \left(C + x x^\transpose - z I\right)^{-1}y - y^\transpose \left(C - z I\right)^{-1} y}
    \le  \frac{\sqrt{2} \norm{y}_2^2}{\ipart{z}},
\end{align}

When $z b T$ is substituted, there is
\begin{align}
    &   \abs{y^\transpose \left(C + x x^\transpose - z b T I\right)^{-1}y - y^\transpose \left(C - z b T I\right)^{-1} y}
    =  \frac{\sqrt{2} \norm{y}_2^2}{v b T}.
\end{align}
In later use, $y$ is instantiated by $u_k$ and there is $\norm{u_k}_2^2 = a \norm{x}_2^2 \le \alpha$ for any $t$, so by assumption the approximation error is always bounded by 
\begin{align}
    \abs{u_k^\transpose \left(C + x x^\transpose - z b T I\right)^{-1} u_k - u_k^\transpose \left(C - z b T I\right)^{-1} u_k} \le \approxmatrix[\boundedby][].
\end{align}

\NewDocumentCommand{\innersum}{}{\frac{1}{b T}\sum_{k}}
\NewDocumentCommand{\outersum}{}{}
\NewDocumentCommand{\innerapproximator}{O{\left(A^p - z b T I\right)^{-1}} O{k}}{ u_{#2}^\transpose #1 u_{#2} }
\NewDocumentCommand{\innerouterproduct}{O{}}{u_{k} #1 u_{k}^\transpose}
\NewDocumentCommand{\biasapproximator}{}{\frac{\ex{\innersum \innerapproximator}}{1 + \ex{\innersum \innerapproximator}}}
\NewDocumentCommand{\diffapproximator}{}{\innerapproximator - \ex{\innersum \innerapproximator}}
With these two approximation techniques, we first approximate the LHS of \cref{eq:claim}.
To this end, let $C^p_k \defeq B^p - u_k u_k^\transpose$ and by Sherman-Morrison formula there is
\begin{align}
    &   u_k^\transpose \left(B^p - z b T I\right)^{-1} u_k
    =   u_k^\transpose \left(C^p_k + u_k u_k^\transpose - z b T I\right)^{-1} u_k\\
    =&  u_k^\transpose \left(\left(C^p_k - z b T I\right)^{-1} - \frac{\left(C^p_k - z b T I\right)^{-1} u_k u_k^\transpose \left(C^p_k - z b T I\right)^{-1}}{1 + u_k^\transpose\left(C^p_k - z b T I\right)^{-1} u_k}\right) u_k\\
    =&  \frac{u_k^\transpose \left(C^p_k - z b T I\right)^{-1}u_k}{1 + u_t^\transpose\left(C^p_k - z b T I\right)^{-1} u_k}
    =  \frac{u_k^\transpose \left(A^p - z b T I\right)^{-1}u_k}{1 + u_k^\transpose\left(A^p - z b T I\right)^{-1} u_k} + \approxfunction \cdot \approxmatrix.
\end{align}
After that, there is
\begin{align}
    &   \frac{1}{T} \sum_{t=1}^T \ex{u_t^\transpose (B^p - z b T I)^{-1} u_t}\\
    =&  \frac{1}{T} \sum_{t=1}^T \frac{1}{b} \sum_{l=1}^b \ex{\frac{u_{t, l}^\transpose \left(A^p - z b T I\right)^{-1} u_{t, l}}{1 + u_{t, l}^\transpose \left(A^p - z b T I\right)^{-1} u_{t, l}}} + \approxfunction \cdot \approxmatrix\\
    =&  \outersum \innersum  \ex{\biasapproximator} + \approxmatrix \\
        &+ \outersum \innersum \ex{\approxfunction \abs{\diffapproximator}}\\
    =&  \outersum \biasapproximator + \approxmatrix\\ 
        &+ \approxfunction  \outersum \innersum \ex{\abs{\diffapproximator}}\\
    =&  \outersum \biasapproximator \label{eq:term1}\\ 
        &+ \approxfunction  \ex{\abs{u_r^\transpose \left(A^p - z b T I\right)^{-1} u_r - \ex{u_r^\transpose \left(A^p - z b T I\right)^{-1} u_r}}}  \label{eq:term2} \\&+ \approxmatrix,
\end{align}
where $r$ in the last line is a uniformly randomly selected index from $\set{1, \dots, b T}$ independently to the training process.

\NewDocumentCommand{\approxbias}{O{\boundedby}}{#1{\frac{c \beta}{v p}}}
Note that $S_p(z) = \trace{A^p - z b T I}^{-1}, \ex{S_p(z)} = \ex{\trace{A^p - z b T I}^{-1}}$. So for the term in \cref{eq:term1} we proceed by proving $\frac{1}{b} \sum_{l=1}^{b} \ex{u_{t, l}^\transpose \left(A^p - z b T I\right)^{-1} u_{t, l}}$ approximates $\ex{\trace{A^p -z b T I}^{-1}}$. For convenience let $D \defeq b T\left(A^p - z b T I\right)^{-1}$ be an alias to it, whose spectral norm satisfies $\norm{D}_{\infty} \le b T \frac{1}{ v b T} = \frac{1}{v}$, then
\begin{align}
    &       \abs{\ex{\innersum \innerapproximator} - \ex{S_p(z)}}\\
    =&      c\abs{\frac{\ex{\innersum \innerapproximator[b T \left(A^p - z b T I\right)^{-1}]}}{p} - \frac{ \ex{b T S_p(z)}}{p}}\\
    =&      \frac{c}{p} \abs{\innersum \ex{\innerapproximator[D]} - \ex{\trace{D}}}
    =      \frac{c}{p} \abs{\ex{\trace{\left(\innersum \innerouterproduct - I\right) D}}}\\
    \le&    \frac{c}{p} \ex{\norm{\left(\innersum \innerouterproduct - I\right) D}_1}
    \le    \frac{c}{p} \ex{\norm{\left(\innersum \innerouterproduct - I\right)}_1 \norm{D}_{\infty}}\\
    \le&    \frac{c}{v p}\ex{\norm{\left(\innersum \innerouterproduct - I\right)}_1}
    \le    \frac{c}{v p}  \ex{\sqrt{p \trace{\left(T^p_t - I\right)^\transpose \left(T^p_t - I\right)}}}\\
    =&      \approxbias,
\end{align}
where the last inequality is because
\begin{align}
    \norm{A}_1 =& \sum_{i=1}^{p} \abs{\lambda_i(A)} = \norm{\begin{bmatrix}
        \lambda_1(A) & \cdots & \lambda_i(A) & \cdots \lambda_p(A)
    \end{bmatrix}^\transpose}_1\\
    \le&    \sqrt{p} \norm{\begin{bmatrix}
        \lambda_1(A) & \cdots & \lambda_i(A) & \cdots \lambda_p(A)
    \end{bmatrix}^\transpose}_2\\
    =&  \sqrt{p} \norm{A}_2 = \sqrt{p \trace{A^\transpose A}},
\end{align}
given that $A = \left( \left(\innersum \innerouterproduct - I\right) \right)$ is symmetric so that its singular values are absolute eigenvalues.
With approximation on complex function $\frac{x}{1 + x}$, this $\approxbias$-boundedness implies $\approxfunction \cdot \approxbias = \approxbias$ approximation of in \cref{eq:term1}.

\NewDocumentCommand{\approxvariance}{O{\boundedby}}{#1{\frac{c \alpha}{v p}}}
\NewDocumentCommand{\utdu}{}{u_t^\transpose D u_t}
\NewDocumentCommand{\xtdx}{}{x_r^\transpose D x_r}
For the difference term in \cref{eq:term2}, we prove its diminishment by $\frac{1}{v}$-bounded variance of $u_{t, l}^\transpose (A^p - z b T I)^{-1} u_{t, l}$, or formally
\begin{align}
    \ex{\abs{X - \ex{X}}^2} - \ex{\abs{X - \ex{X}}}^2 =& \ex{\left(\abs{X - \ex{X}} - \ex{\abs{X - \ex{X}}}\right)^2} \ge 0\\
    \ex{\abs{X - \ex{X}}} \le&  \sqrt{\ex{\abs{X - \ex{X}}^2}} = \sqrt{\var{X}},
\end{align}
and 
\begin{align}
    &   \var{\innerapproximator[\left(A^p - z b T I\right)^{-1}][r]}
    =  \frac{c^2}{p^2} \var{\innerapproximator[D][r]}\\
    =&  \frac{c^2}{p^2} \left(\ex{\trace{\innerapproximator[D][r] \innerapproximator[D][r]}} - \ex{\trace{\innerapproximator[D][r]}}^2\right)
    \le    \frac{c^2}{p^2} \alpha^2 \left(\ex{\trace{\xtdx \xtdx}}\right)\\
    \le&   \frac{c^2 \alpha^2}{p^2} \ex{\trace{\xtdx \xtdx}}
    \le    \frac{c^2 \alpha^2}{p^2} \ex{\norm{x^p}_2^4 \norm{D}_\infty^2}
    =  \frac{c^2 \alpha^2}{v^2 p^2},
\end{align}
where the last step follows that $x^p$'s norm is bounded and that $\norm{D}$ is also uniformly bounded. 

\NewDocumentCommand{\approxlargest}{O{\boundedby}}{#1{\approxmatrix[] + \approxbias[] + \approxvariance[]}}
To sum up, we have obtained 
\begin{align}
    \frac{1}{T} \sum_{t=1}^T \ex{u_t^\transpose \left(B^p - z b T I\right)^{-1} u_t} = \frac{\ex{S_p(z)}}{1 + \ex{S_p(z)}} + t,
\end{align}
where $\abs{t} = \approxlargest$.

\NewDocumentCommand{\approxall}{O{\boundedby}}{#1{\approxlargest[] + \frac{c}{v p} + \frac{c \alpha }{v p}}}

With \cref{eq:claim}, \cref{eq:a1}, one can reduce \cref{eq:a3} to
\begin{align}
    p =& T (b + O(1)) \left(\frac{\ex{S_p(z)}}{1 + \ex{S_p(x)}} + t\right) - z b T \left(\ex{S_p(z)} + \boundedby{c / v p}\right),
\end{align}
and
\begin{align}
    \frac{\ex{S_p(z)}}{1 + \ex{S_p(x)}} - z \ex{S_p(z)} =& \frac{p}{b T} + s = c + s,
\end{align}
where $s = \approxall$.


$\ex{S_p(p)}$ always have a non-negative real part because real parts of eigenvalues of $\left(U U^\transpose - z b T I\right)$ are always non-negative by an argument similar to previous ones. Since $\alpha, \beta, c$ and $p$ depend only on $p, b, T$ instead of $v$, $s = \approxall$ is bounded by $\tau / v$ which satisfies $\tau$ is constant w.r.t $v$, $\frac{v_0 + (1 - c)}{\sqrt{2}} > \frac{\tau}{v_0} + 2 \sqrt{c v_0} + 2 \sqrt{\tau}$ and $v \ge v_0 \ge 2 c$. Given $c \in [0, 1]$, by \cref{lemma:bound_of_sti}, there is
\begin{align}
    &   \abs{\ipart{\ex{S_p(v i)}}} \le \abs{\ex{S_p(v i)}} \\
    \le& \approxquadratic{\approxall[]}
\end{align}
for $v \ge v_0$. Bounds using the real part are similarly obtained.

The similar bound for $s_p(z) = (b T / p) S_p(z) = \frac{1}{c} S_p(z)$ is
\begin{align}
    \abs{\ipart{\ex{s_p(v i)}}}
    \le& \frac{1}{c}\approxquadratic{\approxall[]}.
\end{align}

The expected mean $\ex{\overline{\lambda^{ > 0}}}$ of $\frac{1}{b T} U^p \left(U^p\right)^\transpose$'s non-zero eigenvalue is
\begin{align}
    &   \ex{\frac{\trace{\frac{1}{b T} U^p \left(U^p\right)^\transpose}}{\min(b T, p)}}
    =  \frac{\frac{1}{b T}\sum_{k=1}^{b T}\ex{\trace{u_{k} u_k^\transpose}}}{\min(b T, p)}
    =  \frac{\frac{1}{b T}\sum_{k=1}^{b T}\ex{\norm{u_k}_2^2}}{\min(b T, p)}
    \le   \frac{\alpha}{\min(b T, p)}.
\end{align}

Finally, the desired conclusion is obtained through \cref{lemma:sti_and_eigenvalue_ratio} by
\begin{align}
        \ex{\frac{\ex{\overline{\lambda^{>0}} / \sqrt{v}}^2}{\left(\frac{\lambda}{\sqrt{v}}\right)^2 + v}}
    \le    \frac{1}{v} \ex{\overline{\lambda^{>0}}}^2 \abs{\ipart{s_p(v i)}},
    % \le&    \frac{1}{v c} \frac{\alpha^2}{\min(b T, p)^2}  \approxquadratic{\approxall[]},\\
        \ex{\frac{\ex{\overline{\lambda^{>0}}}}{\lambda + \frac{v^2}{\lambda}}}
    \le    \ex{\overline{\lambda^{>0}}} \abs{\rpart{s_p(v i)}}.
    % \le&    \frac{1}{c} \frac{\alpha}{\min(b T, p)}  \approxquadratic{\approxall[]}.
\end{align}

\end{proof}
}

Its proof is left in \cref{proof:spectral_of_accumulated}. This concludes that spectral concentration also happens in $\Eta^{1: T} \left(\Eta^{1:T}\right)^\transpose$ and $X^{1:T} \left(X^{1:T}\right)^\transpose$, as long as the samples in one batch are independently sampled and all model-state-specific samples involved during training are diverse enough to have low anisotropy and norm bounds. To see that the conditions are satisfied, note the only hard conditions are that of continuity, which is hard to verify and thus simply assumed, as well as the choice of $v$ and $v_0$. We assume enough steps have been trained so $c$ can be very small, for example effectively $\frac{768}{32 \times 50,042} \approx 4.8 \times 10^{-4}$ under strong weight decay (it is even smaller when weight decay weakens or disappears) as we shall see in latter paragraphs. So $v_0 = 10^{-3} > 2 c$ can be chosen. In experiments in \cref{sec:t_exp:anisotropy} we will see $\beta / p$ is always smaller than $1$ and $\alpha / p$ can be less than $0.05$ when weight decay is strong and dimension $p$ is large enough. Under these conditions, it can be verified that LHS of \cref{eq:hard_condition} is larger than RHS by at least $0.061$. So the applicability of the theorem depends on how good the bound is. To empirically verify the applicability of \cref{theorem:spectral_of_accumulated}, one needs to compute anisotropy $\ex{\sqrt{p \trace{\left(T^{p} - I\right)^\transpose\left(T^{p} - I\right)}}}$ and $\alpha$ from $x^p$'s marginal distribution, i.e., by mixing hidden features or back-propagated gradients, which is conducted in \cref{sec:t_exp:anisotropy}.

In contrast with conventional asymptotic results with $p \to \infty$ and $p / b T \to c$ in RMT, our theorem gives bounds for non-limiting scenarios. One of its benefits in the context of machine learning is that one often is more interested in training behaviors when the total number $b T$ of training samples increases as the training proceeds while $p$ is held still. 
Nevertheless, the co-increasing scenarios are also of interest, especially in the era of large models \citep{scaling_law}.
In our result, the spectral concentration is measured by the expected fraction between eigenvalues and the expected eigenvalue, but with an extra shadowing parameter $v$ that may shadow small eigenvalues and decreasing $v$ to suppress the disturbance loosens the bound. Fortunately, most $v$ as dominators show up together with $c$ as numerators, the ratio between hidden dimension and number of training samples used which is often extremely small. If $\frac{\alpha}{p}$ is also small and decreases as $p$ is enlarged, as we will show in the experiments, then the bound will be controlled.

It is frustrating to see that the bound diverges as the training step increases due to factor $\frac{1}{c} = \frac{b T}{p}$. However, we consider weight decay as an alleviation to this issue. Weight decay effectively introduces a sliding window which reduces the effective $T$, because vectors out of the window are exponentially decayed and they have little contribution to the sample covariance matrix. For example, let $w$ be the parameter of weight decay, then each column in $\Eta^{1: T}$ or $X^{1:T}$ becomes $\left(\sqrt{1 - \eta_{\mathrm{lr}} w}\right)^{T - t} \eta_{t, l}$ or $\left(\sqrt{1 - \eta_{\mathrm{lr}} w}\right)^{T - t} u_{t, l}$, where $\eta_{\mathrm{lr}}$ is learning rate. Setting $r = 1 - \eta_{\mathrm{lr}} w$, if we consider the tail whose weights' sum is smaller than a threshold $\tau$ as those out of the window, then the window size $k$ needs to satisfy $\sum_{i={k+1}}^{\infty} r^i = r^{k+1} / (1 - r) \le \tau$, where $r$ is used to obtain tighter bounds instead of $\sqrt{r} = \sqrt{1 - \eta_{\mathrm{lr}} w}$ by arguments in \cref{appendix:effective_window_size}. One sufficient condition for this is $k \ge \frac{\ln \tau (1 - r)}{\ln r} - 1$. When $\tau=10^{-3}, \eta=10^{-3}, w = 10^{-1}$, the effective window size is about $161,172$. When $w$ increases to $0.3$, the effective window size becomes $50,057$. As a result, $\frac{1}{c}$ is upperbounded by a constant as the training proceeds if weight decay presents.

There are still gaps between our results and the spectral distribution of $\kkT$ during stochastic training. For example, spectral concentration of $\Eta \Eta^\transpose$ and $X X^\transpose$ hints similar phenomena in their interlaced product $\left(K^{l, T} - K^{l, 0}\right) \left(K^{l, T} - K^{l, 0}\right)^\transpose$, which, however, is not yet formally proved. Moreover, applying \cref{theorem:spectral_of_accumulated} assumes SGD instead of adaptive optimizers is used due to \cref{eq:update_and_large_matrices}. Filling these gaps is left for future works because we have established spectral concentration's empirical supports in \cref{sec:t_exp:spectral_concentration}. 

Finally, we have discussed the spectral concentration of $\kkT$. To put everything together, assuming moderate projection of $\eta^l$ into non-null space of $\kkT$, and considering the increasing trace of $\kkT$ and spectral concentration in $\kkT$, the only way to suppress 
\begin{align}
    \trace{\hessian[\theta_D]} \ge \left(\gamma^l\right)^\transpose \left(\kkT \hadamard M^l\right) \gamma^l = \left(\eta^l\right)^\transpose \kkT \eta^l \ge \lambda r \left(\eta^l\right)^\transpose \eta^l
\end{align}
is to reduce $\norm{\eta^l}_2^2 = \norm{g^l_K \hadamard \gamma^l}_2^2$. To see gradient-sparsity-induced activation sparsity's emergence, given empirically that $\trace{M^l} = \trace{\left(g^l_K\right)^\transpose g^l_K} = \norm{g^l_K}_2^2$ will not decrease during training, the only way to suppress $\norm{\eta^l}_2^2$ is to decrease $\gamma^l$, at least at entries where $g^l_K$ has large magnitudes.
