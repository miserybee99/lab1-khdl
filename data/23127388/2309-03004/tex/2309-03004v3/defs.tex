\makeatletter
\newenvironment{breakablealgorithm}
  {% \begin{breakablealgorithm}
   \begin{center}
     \refstepcounter{algorithm}% New algorithm
     \hrule height.8pt depth0pt \kern2pt% \@fs@pre for \@fs@ruled
     \renewcommand{\caption}[2][\relax]{% Make a new \caption
       {\raggedright\textbf{\ALG@name~\thealgorithm} ##2\par}%
       \ifx\relax##1\relax % #1 is \relax
         \addcontentsline{loa}{algorithm}{\protect\numberline{\thealgorithm}##2}%
       \else % #1 is not \relax
         \addcontentsline{loa}{algorithm}{\protect\numberline{\thealgorithm}##1}%
       \fi
       \kern2pt\hrule\kern2pt
     }
  }{% \end{breakablealgorithm}
     \kern2pt\hrule\relax% \@fs@post for \@fs@ruled
   \end{center}
  }
\makeatother


\newcommand{\set}[1]{\{#1\}}
\newcommand{\setName}[1]{{\textbb{#1}}}
\newcommand{\floor}[1]{{\left\lfloor #1 \right\rfloor}}
\newcommand{\ceil}[1]{{\left\lceil #1 \right\rceil}}
\newcommand{\srt}[2]{{\texttt{#1}(#2)}}
\newcommand{\srtName}[1]{{\texttt{#1}}}
\newcommand{\nil}{{\textrm{NIL}}}
\newcommand{\gh}{{\mbox{-}}}
\newcommand{\seg}[3]{{#1[#2\dots#3]}}
\newcommand{\arr}[2]{{#1[#2]}}
% \newcommand{\interval}[2]{{\left[#1,#2\right]}}
\newcommand{\sqrket}[2]{{\left[#1,#2\right]}}
\newcommand{\angBracket}[1]{{\left\langle#1\right\rangle}}
\newcommand{\size}[1]{{\left|{#1}\right|}}
\newcommand{\abs}[1]{\size{#1}}
\newcommand{\mathCmt}[1]{{\text{(#1)}}}
\newcommand{\mn}[1]{{\min\set{#1}}}
\newcommand{\mx}[1]{{\max\set{#1}}}
\newcommand{\event}[1]{{\set{#1}}}
\newcommand{\bigO}[1]{{O\left(#1\right)}}
\newcommand{\bigTheta}[1]{{\Theta\left(#1\right)}}
\newcommand{\bigOmg}[1]{{\Omega\left(#1\right)}}
\newcommand{\diam}[1]{{\mathop{\mathrm{diam}}\left(#1\right)}}
\newcommand{\dfsWhite}{{\textrm{WHITE}}}
\newcommand{\dfsGray}{{\textrm{GRAY}}}
\newcommand{\dfsBlack}{{\textrm{BLACK}}}
\newcommand{\lcm}{{\mathop{\mathrm{lcm}}}}
\newcommand{\intGroup}[1]{{\mathbb{Z}_{#1}}}
\newcommand{\substitute}[2]{{\left[\frac{#1}{#2}\right]}}
\newcommand{\bvec}[1]{\mathbb{#1}}
\newcommand{\Null}{\textrm{Null}}
% \newcommand{\unit}[1]{{\,\,\mathrm{#1}}}
\newcommand{\mathsc}[1]{\text{\sc{#1}}}
\newcommand{\dom}{\mathrm{dom}}
\newcommand{\pr}[1]{{\mathrm{Pr}\left[#1\right]}}
\newcommand{\var}[1]{{\mathbf{Var}\left[{#1}\right]}}
% \newcommand{\fact}[1]{{{#1}!}}
\newcommand{\indic}[1]{{\mathrm{I}\left[{#1}\right]}}
\newcommand{\mathcomment}[1]{\text{(#1)}}
\newcommand{\norm}[1]{\left\lVert#1\right\rVert}
\newcommand{\mat}[1]{\mathbf{#1}}
\newcommand{\inlinecode}[1]{\texttt{#1}}
\newcommand{\cardinality}[1]{{\left|{#1}\right|}}
\newcommand{\argmin}{{\mathrm{argmin}}}
\newcommand{\funcmodify}[3]{{{#1}\left\{{#2} \leadsto {#3}\right\}}}
\newcommand{\sgn}{\textrm{sgn}}
\NewDocumentCommand{\reals}{}{\mathbb{R}}
\NewDocumentCommand{\nats}{}{\mathbb{N}}
\NewDocumentCommand{\complexes}{}{\mathbb{C}}
\NewDocumentCommand{\dd}{}{\mathrm{d}}
\NewDocumentCommand{\rpart}{m}{\mathrm{Re}\left(#1\right)}
\NewDocumentCommand{\ipart}{m}{\mathrm{Im}\left(#1\right)}

\DeclareMathOperator{\supportinner}{supp}

\NewDocumentCommand{\mset}{m}{\set{#1}}
\NewDocumentCommand{\opt}{O{}}{\mathrm{OPT}_{\mathrm{#1}}}
\NewDocumentCommand{\sol}{O{}}{\mathrm{SOL}_{\mathrm{#1}}}
\NewDocumentCommand{\support}{m}{\supportinner{#1}}
\NewDocumentCommand{\ex}{O{} m}{{\mathbf{E}_{#1}\left[#2\right]}}
\DeclareMathOperator{\rangeinner}{ran}
\NewDocumentCommand{\range}{m}{\rangeinner{#1}}
\NewDocumentCommand{\defeq}{}{\triangleq}
\NewDocumentCommand{\mutualinfo}{O{} O{} m m}{I_{#1}^{#2}(#3; #4)}
\NewDocumentCommand{\entropy}{O{} O{} m m}{H_{#1}^{#2}(#3)}
\NewDocumentCommand{\prob}{O{} O{} m}{\mathrm{P}^{#1}_{#2}\left[#3\right]}
\NewDocumentCommand{\transpose}{}{T}
\NewDocumentCommand{\trace}{m}{\mathrm{tr}\left(#1\right)}
\NewDocumentCommand{\hadamard}{}{\odot}
\NewDocumentCommand{\diag}{m}{\mathrm{diag}\left(#1\right)}
\NewDocumentCommand{\gaussian}{m m}{\mathcal{N}\left(#1, #2\right)}
\NewDocumentCommand{\chisquared}{m}{\chi^2_{#1}}

% define theorem-like environments
\newenvironment{cntrProofTree}{\begin{center}\begin{prooftree}}{\end{prooftree}\end{center}}
\DeclareMathOperator*{\argmax}{arg\,max}

% \newtheorem{theorem}{Theorem}
% \newtheorem{lemma}[theorem]{Lemma}
% \theoremsymbol{\ensuremath{\color{lightgray}\blacksquare}}
% \newtheorem*{proof}{Proof}

% \newtheorem{theorem}{Theorem}
% \newtheorem{corollary}{Corollary}
% \newtheorem{lemma}{Lemma}
% \newtheorem{definition}{Definition}
% \newtheorem{remark}{Remark}
% \newtheorem{observation}{Observation}
% \newtheorem{conjecture}{Conjecture}

\declaretheorem[name=Theorem]{thm}
\declaretheorem[name=Corollary]{restatablecorollary}
