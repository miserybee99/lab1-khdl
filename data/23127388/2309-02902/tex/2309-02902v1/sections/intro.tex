% SECTION Introduction %
\section{Introduction}
\label{introduction}

Social media processing is the essential task in natural language processing (NLP) \cite{Natural-Language-Processing-for-Social-Media, injadat2016data}, utilized in many world applications over the last few decades, such as topic detection \cite{topic-detection}, spam detection \cite{Spam-detection-in-twitter}, opinion mining \cite{Opinion-Mining-on-Social-Media-Data}, and text classification \cite{A-survey-on-text-mining-in-social-networks} in general. Along with information science and engineering development, the world has witnessed exponential growth in social media platforms in many countries, including Vietnam. According to the Digital 2021 report\footnote{\url{https://datareportal.com/reports/digital-2021-global-overview-report}} by We Are Social and Hootsuite (2021), there were 4.2 billion active social media users worldwide in January 2021, an increase of 13\% compared to the previous year. Facebook\footnote{\url{https://www.facebook.com/}} alone had 2.7 billion active users in the same month, making it the largest social media platform in the world. Instagram\footnote{\url{https://www.instagram.com/}}, owned by Facebook, had over 1 billion active users in 2020, while Twitter\footnote{\url{https://twitter.com/}} had 353 million monthly active users in the same year. Furthermore, the report shows that people spend an average of 2 hours and 25 minutes daily on social media and messaging apps. These statistics show social media platforms' widespread use and growing popularity worldwide. Therefore, social media platforms (e.g., Facebook\footnote{\url{https://www.facebook.com/}}, Twitter\footnote{\url{https://twitter.com/}}, and Instagram\footnote{\url{https://www.instagram.com/}}) have become an integral part of modern communication and is a rich source of information for various applications, including sentiment analysis, opinion mining, and recommendation systems. However, social media mining in Vietnamese poses unique challenges due to the complexity of the language and the informal nature of social media text. Furthermore, Vietnamese social media text often includes slang, misspellings, and other non-standard language features, making applying traditional natural language processing (NLP) techniques difficult. Therefore, the need for effective Social media processing techniques on Vietnamese social media has become more crucial to improving their performance.

According to the essentials of Social media processing on social media, several pieces of research have been studied. In Vietnamese particular, \citet{VSFC} presented a Vietnamese Students’ Feedback Corpus (UIT-VSFC), a free and high-quality corpus with two distinct tasks, namely sentiment-based and topic-based classifications. \citet{DBLP:journals/corr/abs-1911-09339} introduced a standard Vietnamese Social Media Emotion Corpus (UIT-VSMEC), contributing to emotion recognition in Vietnamese. \citet{DBLP:journals/corr/abs-2103-10069} proposed a novel dataset for identifying the constructiveness and toxicity of Vietnamese comments on social media, the Vietnamese Constructive and Toxic Speech Detection dataset (UIT-ViCTSD). A range of Social media processing worked on these Vietnamese social media datasets \cite{huynh-etal-2020-simple,https://doi.org/10.48550/arxiv.2209.10482,Doan_2022,nguyen2020exploiting}. However, these tasks need to be further improved due to the limitations of the models. Furthermore, because of the complexity of social media comments, the current field still faces many challenges due to social media comments' colossal volume and variety in both immensity and topics. One of the primary challenges is dealing with imbalanced data. Social media platforms generate massive amounts of data, often not equally distributed across all classes. This results in an imbalanced dataset, where several classes may have significantly fewer examples than others. Another challenge is handling noisy data, which is common in social media. Social media data often contain many emojis, abbreviations, and non-standard language, making it difficult for Social media processing models to classify the text accurately. Social media comments may also contain sarcasm and irony, which can be challenging to interpret.

Graph Convolutional Networks (GCN) can address the issues of imbalanced and noisy data in Social media processing by taking advantage of the graph structure of the data. By incorporating the graph structure into the model, GCN can effectively capture the relationships and dependencies among the data, which helps to reduce the impact of noise and imbalance. Additionally, Contextualized Language Models can be fine-tuned on specific tasks like text classification, which enables them to adapt to the specific characteristics of the data, such as imbalanced or noisy data. Therefore, our research contributes to text classification on social media in Vietnamese. Our research proposes a novel approach integrating Graph Convolutional Networks with the Vietnamese state-of-the-art Contextualized Language Model, PhoBERT. Our research can handle these issues, including noisy and imbalanced datasets. 

The primary contributions of this study can be outlined as follows: We proposed a novel Vietnamese text classification model ViCGCN by jointly training the large-scale language pre-trained language model PhoBERT and Graph Convolutional Networks (GCN) modules for evaluating social media text classification or social media mining in Vietnamese. Various experiments were conducted with two approaches: BERTology and its Integrated model (BERTology Integrated with GCN), including the Vietnamese state-of-the-art model PhoBERT on three Vietnamese benchmark datasets. Compared to the Vietnamese state-of-the-art PhoBERT model and previous studies done on the datasets, our integrated model ViCGCN achieves significantly better performances. A survey on single large-scale and integrated models was carried out to demonstrate the efficacy of GCN on large-scale pre-trained language models. Additionally, our integrated model ViCGCN successfully addressed the imbalanced and noisy problem of social media datasets. Finally, jointly training BERTology and GCN modules significantly improves performances on Vietnamese social-domain text classification tasks.

The rest of this paper is structured in the following manner. Section \ref{Related work} surveys several current works on Vietnamese text classification. Our proposed approach is presented in detail through Section \ref{Proposed model}. Section \ref{Experiments} briefly looks at the used datasets for our experiments, illustrates processes for implementing models and our experimental results on each task, and describes the result analysis and discussion of the proposed approach. In summary, Section \ref{Conclusion} serves as the final part of our research and outlines our conclusions and any potential areas for future exploration.