 %% 
%% Copyright 2019-2020 Elsevier Ltd
%% 
%% This file is part of the 'CAS Bundle'.
%% --------------------------------------
%% 
%% It may be distributed under the conditions of the LaTeX Project Public
%% License, either version 1.2 of this license or (at your option) any later version.  The latest version of this license is in http://www.latex-project.org/lppl.txt and version 1.2 or later is part of all distributions of LaTeX version 1999/12/01 or later.

%% The list of all files belonging to the 'CAS Bundle' is given in the file `manifest.txt'.
%% 
%% Template article for cas-dc documentclass for double column output.

% \documentclass[a4paper,fleqn,longmktitle]{cas-sc}
%\documentclass{elsarticle}
\documentclass[final,1p,times]{elsarticle}
% \documentclass[preprint,12pt]{elsarticle}

%\graphicspath{ {./figures/} }
%\usepackage[numbers]{natbib}
% \usepackage{epsfig}
\usepackage{hyperref}
% \usepackage[utf8]{vietnam}%
\usepackage[T5]{fontenc} %c
\usepackage{latexsym} %c
\usepackage{graphicx} %c
\usepackage{pgf-pie} %c   
\usepackage{soul}%c
\usepackage{color}%c
\usepackage{tablefootnote}%c
\usepackage{url}%c
\usepackage{caption}%c
% \usepackage{subcaption}%C

% \usepackage[english]{babel} %
\usepackage{float}
\usepackage{subfigure}
\usepackage{tabto}
\usepackage{tabularx}
\usepackage{tabularray}
\usepackage{mathtools}
\usepackage{amsmath}
\usepackage{adjustbox}
\usepackage{lscape}
\usepackage{amsfonts}
%\usepackage{verbatim} %comments
%\usepackage{apalike}
\usepackage{multirow}

% \restylefloat{figure}
% \restylefloat{table}

\usepackage{comment}

\newcommand{\tabitem}{\textbullet~~}
\newcommand{\cmark}{\ding{51}}
\newcommand{\xmark}{\ding{55}}

% \bibliographystyle{model5-names}\biboptions{authoryear}

%% The lineno packages adds line numbers. Start line numbering with
%% \begin{linenumbers}, end it with \end{linenumbers}. Or switch it on
%% for the whole article with \linenumbers.
%% \usepackage{lineno}

\begin{document}
\let\WriteBookmarks\relax
\def\floatpagepagefraction{1}
\def\textpagefraction{.001}

\begin{frontmatter}


\title{ViCGCN: Graph Convolutional Network with Contextualized Language Models for Social Media Mining in Vietnamese}

%% use optional labels to link authors explicitly to addresses:
%% \author[label1,label2]{}
%% \affiliation[label1]{organization={},
%%             addressline={},
%%             city={},
%%             postcode={},
%%             state={},
%%             country={}}
%%
%% \affiliation[label2]{organization={},
%%             addressline={},
%%             city={},
%%             postcode={},
%%             state={},
%%             country={}}

\author[label1,label2]{Chau-Thang~Phan}
\ead{20520929@gm.uit.edu.vn}    
\author[label1,label2]{Quoc-Nam~Nguyen}
\ead{20520644@gm.uit.edu.vn}
\author[label1,label2]{Chi-Thanh~Dang}
\ead{20520761@gm.uit.edu.vn}
\author[label1,label2]{Trong-Hop~Do}
\ead{hopdt@uit.edu.vn}
\author[label1,label2]{Kiet~Van~Nguyen\corref{cor1}}
\ead{kietnv@uit.edu.vn}
\cortext[cor1]{Corresponding author at the University of Information Technology, Vietnam National University, Ho Chi Minh City, Vietnam.}

\affiliation[label1]{organization={Faculty of Information Science and Engineering, University of Information Technology},
            city={Ho Chi Minh city},
            country={Vietnam}}
\affiliation[label2]{organization={Vietnam National University},
            city={Ho Chi Minh city},
            country={Vietnam}}

% \journal{Expert Systems with Applications}
\journal{Neurocomputing}

\begin{abstract}
Social media processing is a fundamental task in natural language processing (NLP) with numerous applications. As Vietnamese social media and information science have grown rapidly, the necessity of information-based mining on Vietnamese social media has become crucial. However, state-of-the-art research faces several significant drawbacks, including imbalanced data and noisy data on social media platforms. Imbalanced and noisy are two essential issues that need to be addressed in Vietnamese social media texts. Graph Convolutional Networks can address the problems of imbalanced and noisy data in text classification on social media by taking advantage of the graph structure of the data. This study presents a novel approach based on contextualized language model (PhoBERT) and graph-based method (Graph Convolutional Networks). In particular, the proposed approach, ViCGCN, jointly trained the power of \textbf{C}ontextualized embeddings with the ability of Graph Convolutional Networks, \textbf{GCN}, to capture more syntactic and semantic dependencies to address those drawbacks. Extensive experiments on various Vietnamese benchmark datasets were conducted to verify our approach. The observation shows that applying GCN to BERTology models as the final layer significantly improves performance. Moreover, the experiments demonstrate that ViCGCN outperforms 13 powerful baseline models, including BERTology models, fusion BERTology and GCN models, other baselines, and state-of-the-art methods on three benchmark social media datasets. Our proposed ViCGCN approach demonstrates a significant improvement of up to 6.21\%, 4.61\%, and 2.63\% over the best Contextualized Language Models, including multilingual and monolingual, on three benchmark datasets, UIT-VSMEC (for Vietnamese emotion recognition), UIT-ViCTSD (for Vietnamese constructive and toxic analysis), and UIT-VSFC (for Vietnamese sentiment analysis), respectively. Additionally, our integrated model ViCGCN achieves the best performance compared to other BERTology integrated with GCN models. The code\footnote{\url{https://github.com/phanchauthang/ViCGCN}} is publicly available for research purposes.
\end{abstract}

% %%Graphical abstract
% \begin{graphicalabstract}
% %\includegraphics{grabs}
% \end{graphicalabstract}

% %%Research highlights
% \begin{highlights}
% \item Research highlight 1
% \item Research highlight 2
% \end{highlights}

%{\bf OpenViVQA} - A new benchmark dataset for Vietnamese visual question answering

%\begin{highlights}
%    \item Proposing a novel Graph Convolutional Network integrated with Contextualized Language Model ViCGCN firstly for Vietnamese social media texts
%    \item Addressing challenges in Vietnamese social media: imbalanced and noisy
%    \item ViCGCN captures more syntactic and semantic dependencies than language models
%    \item Reaching new SOTA performances on Vietnamese social media benchmarks
%\end{highlights}

\begin{keyword}
Graph \sep Social Media Mining \sep Social Media Processing \sep Graph Convolutional Networks \sep Language Model \sep Contextualized Language Model
\end{keyword} 

\end{frontmatter}

%% \linenumbers

%% main text

\section{Introduction}

Text-Attributed Graphs (TAGs) are a type of graph that have textual data as node attributes. 
These types of graphs are prevalent in the real world, such as in citation networks \cite{hu2020open} where the node attribute is the paper's abstract. TAGs have diverse potential applications, including paper classification \cite{chien2021node} and user profiling\cite{kim2020multimodal}. 
However, studying TAGs presents a significant challenge: how to model the intricate interplay between graph structures and textual features. 
This issue has been extensively explored in several fields, including natural language processing, information extraction, and graph representation learning. 

% Text-Attributed Graphs (TAGs) are a type of graph that is widely present in the real world. 
% In practical applications, many node features can be composed of text. For example, in citation networks, the node feature is the abstract of a paper, and in social networks, the node feature is the user's profile. 
% TAGs have broad potential application values, such as paper classification and user identification. 
% Modeling TAGs involves techniques from multiple fields, including information extraction, natural language processing, and graph representation learning, making it a hot academic topic currently.

An idealized approach involves combining pre-trained language models (PLMs) \cite{he2020deberta,liu2019roberta} with graph neural networks and jointly training them \cite{zhao2022learning,mavromatis2023train}. Nevertheless, this method requires fine-tuning the PLMs, which demands substantial computational resources. Additionally, trained models are hard to be reused in other tasks because finetuning PLM may bring catastrophic forgetting\cite{chen2020recall}. 

Therefore, a more commonly used and efficient approach is the two-stage process \cite{yang2021bert,zhang2022stance,malhotra2020classification}: (1) utilizing pre-trained language models (PLMs) for unsupervised modeling of the nodes' textual features. 
(2) supervised learning using Graph Neural Networks (GNNs). 
Compared to joint training of PLMs and GNNs, this approach offers several advantages in practical applications. 
For example, it can be combined with numerous GNN frameworks or PLMs, and this approach does not require fine-tuning PLMs for every downstream task.
However, PLMs are unable to fully leverage the wealth of information contained in the graph structure, which represents significant information. 
To overcome these limitations, some works propose self-supervised fine-tuning PLMs using graph information to extract graph-aware node features \cite{chien2021node}. Such methods have achieved significant success across various benchmark datasets\cite{hu2020open}. 
% Unsupervised modeling of nodes' textual features by language models (LM) and subsequent supervised learning of the graph feature by Graph Neural Networks (GNNs) is a classical and effective approach for processing TAGs. 
% However, the generated node representation is untrainable in downstream tasks, a unsuitable representation may affect the performance of subsequent GNNs learning. 
% To address limitations, many works merged recently, which investigate how to better utilize pre-trained language models in TAGs modeling. 
% A method is joint PLMs with GNNs by knowledge distillation. 
% and self-supervised fine-tuning PLMs to adapt graph data.   
% First, PLMs are fine-tuned by self-supervised tasks related to graphs, enabling them to capture and comprehend graph information. Then, the fine-tuned PLM is used to generate node representations.
% This approach has achieved significant results in numerous public datasets.


% However, these SSL-based node feature extraction methods suffer from the few-shot challenge. are based on graphs with over 100,000 nodes. 
% This means that during the self-supervised training phase, there are enough samples, and downstream task training samples are also abundant. 
% For example, in Ogbn-arxiv, there are over 70,000 training samples (60\%). 
% However, this situation poses a significant gap from the real world. 
% Firstly, training labels are often expensive, and secondly, there exist many small graphs in the real world.  

However, both self-supervised methods and using language models directly to process TAG suffer from a fundamental drawback. They cannot incorporate downstream task information, which results in identical representations being generated for all downstream tasks. This is evidently counterintuitive as the required information may vary for different tasks. For example, height is useful information in predicting a user's weight but fails to accurately predict age. This issue can be resolved by utilizing task-specific prompts combined with language models \cite{petroni2019language} to extract downstream task-related representations. For example, suppose we have a paper's abstract $\{\mathbf{Abstract}\}$ in a citation network, and the task is to classify the subject of the paper. We can add some prompts to a node's sentence:
$
    \{This, is, a, paper, of, [\mathbf{mask}], subject, its, abstract, is,:, \mathbf{Abstract}\}
$. And then use the embedding corresponding to the [mask] token generated by PLMs as the node feature. Yet this approach fails to effectively integrate graph information. 

To better integrate task-specific information and knowledge within graph structure, this paper proposes a novel framework called G-Prompt. G-Prompt combines a graph adapter and task-specific prompts to extract node features. Specifically, G-Prompt contains a graph adapter that helps PLMs become aware of graph structures. This graph adapter is self-supervised and trained by fill-mask tasks on specific TAGs. G-Prompt then incorporates task-specific prompts to obtain interpretable node representations for downstream tasks.



% However, we observe the SSL-based methods are in the small-sample scenario and found that: \\
% 1. The representations generated by large-scale language models perform similarly to word2vec in small-sample situations. This is clearly counterintuitive, as numerous experiments have shown that pre-trained language models can learn rich knowledge from massive text. \\
% 2. The representation of entire BERT models finetuned on graph self-supervised tasks such as GIANT performs similarly to the frozen language model's representation through GAE pre-training in extremely small sample sizes. However, overall, it outperforms graph-free representations. \\
% 3. Since using PLM-generated representations did not yield good results, we experimented with RoBERTa-based representations with task prompts, which performed the best in small-sample scenarios.

% This implies that both Graph-aware and Task-aware representations are crucial for node representation. 
% However, current methods \textbf{can not effectively combine} the two because current unsupervised node feature generation methods do not consider downstream tasks. 
% Meanwhile, pre-trained models cannot be task-specifically transformed. 
% There is a significant gap between self-supervised tasks and BERT's own pre-training tasks. 
% Directly finetuning BERT would destroy the prior knowledge learned from massive text data.

% Furthermore, current methods generate node features that \textbf{lack interpretability}. 
% The features generated by current methods are continuous and lack interpretability. 
% It is challenging to explain why a particular representation works, and it is difficult to manually select a few features for downstream tasks.
% Meanwhile, the current state-of-the-art methods require finetuning of pre-trained language models (PLMs). However, with the increasing size of PLMs, the computational cost of finetuning has become prohibitively high, often requiring a substantial amount of data to achieve good performance. Thus, it is challenging to integrate these methods with even more powerful language models.

% Therefore, this paper aims to explore the possibility of generating task-aware and graph-aware representations with BERT without finetuning. For the former, a naive method is to use prompts, which are manually input task-related hints, along with text features to generate corresponding words using a language model. For example, for citation networks, we can add prompt information before the abstract: "This is a paper published on <mask> subject, its abstract is [content]." We then use the word distribution after decoding the <mask> as a node feature. However, incorporating graph information into the prompt is challenging. To address this issue, we propose a new framework called GPrompt. This framework combines graph adapters and prompts to extract node features. The graph adapter operates on the last linear transformation layer that predicts words in the LM, i.e., a learnable graph neural network is added to that layer. The goal of the GNN is to help the language model perceive neighbor information of nodes and better predict the masked word. The graph adapter is trained through the language model's native fill-mask task. After the adapter is trained, GPrompt incorporates task-related prompts based on the fill-mask framework of the language model, combined with the graph adapter, to generate task-related representations that are interpretable and perceive graph information.


% pithc on parameter-efficient tuning, cite lora/adaptor
% However, replacing the linear transformation with GNN imposes huge computational costs, and it is not feasible to aggregate neighbors once for each token of every word. To speed up the training process, we adopt DecoupleGNN and use geometric mean to aggregate information from each neighbor. The geometric mean is equivalent to training neighbor nodes with the target node's label in the cross-entropy loss function, so there is no need to globally aggregate neighbor information during GraphAdapter training. This strategy accelerates training effectively through global edge sampling.

We conduct extensive experiments on three real-world datasets in the domains of few-shot and zero-shot learning, in order to demonstrate the effectiveness of our proposed method. The results of our experiments show that G-Prompt achieves state-of-the-art performance in few-shot learning, with an average improvement of \textit{avg.} 4.1\% compared to the best baseline. Besides, our G-Prompt embeddings are also highly robust in zero-shot settings, outperforming PLMs by \textit{avg.} 2.7\%. Furthermore, we conduct an analysis of the representations generated by G-Prompt and found that they have high interpretability with respect to task performance.








\section{Related Works}

In this section, we list works on the same topic as ours. \cref{sec:preliminary} contains works on different topics that our explanation depends on, we omit their details for simplicity here.

In our point of view, the research of activation sparsity in MLP modules starts from the discovery of the relation between MLP and knowledge gained during training. \citet{mlp_as_database} first rewrite MLPs in Transformers into an unnormalized attention mechanism where queries are inputs to the MLP block while keys and values are provided by the first and second weight matrices instead of inputs. So MLP blocks are key-value memories. 
\citet{knowledge_neurons} push forward by detecting how each key-value pair is related to each question exploiting activation magnitudes as well as their gradients, and providing a method to surgically manipulate answers for individual questions in Q\&A tasks. These works reorient research attention back to MLPs, which are previously shadowed by self-attention.

Recently, comprehensive experiments conducted by \citet{observation} demonstrate activation sparsity in MLPs is a prevailing phenomenon in various architectures and on various CV and NLP tasks. 
\citet{observation} also eliminate alternative explanations and attribute activation sparsity solely to training dynamics. 
The authors explain the sparsity theoretically with initialization and by calculating gradients, but their explanation is restricted to the last layer and the first step because in later steps the independence between weights and samples required by the explanation is broken. 
They also discover that some activation functions, such as $\tanh$, hinder the sparsity \citep[see][Fig B.3(c)]{observation}, but did not elaborate on it. 
Compared to their explanations, our explanation applies to all layers and large steps, and accounts for the activation functions' critical role in activation sparsity.

Following empirical discoveries by \citet{observation}, \citet{sharpness_aware} show that sharpness-aware (SA) optimization has a stronger bias toward activation sparsity. 
They explain theoretically by calculating gradients and finding that SA optimization imposes in gradients a component toward reducing norms of activations. However, their explanation is still conducted on shallow 2-layer pure MLPs and requires SA optimization, which is not included in standard training practice. Nevertheless, this explanation hints at the role of flatness in the emergence of activation sparsity. Inspired by them, we explain \emph{deep} networks trained by standard SGD or other stochastic trainers by substituting flat minima for SA optimization.

A more recent work by \citet{from_noises} holds a point that sparsity is a resistance to noises. However, noises are manually imposed and not included in standard data augmentations. We substitute gradient noise from SGD or other stochastic optimizers for them.
\citet{large_step} prove sparsity on 2-layer diagonal MLPs and conjecture similar things to happen in more general networks. Both works hint at the relation between noises (Gaussian sample noises and stochastic gradient noises) and activation sparsity, also leading to the flatness bias of stochastic optimization.

\citet{adversarial_of_moe} study the adversarial robustness of Mixture of Experts (MoE) models brought by architecture-imposed sparsity. They inspire us to relate sparsity with adversarial robustness, although we do it reversely. It is the major inspiration for our results.

To sum up, existing discoveries hint at the relation between activation sparsity and noises, flatness and activation functions but they are still restricted to shallow layers, small steps and special training. Inspired by them and filling their gaps, our explanation applies to deep networks and large training steps, and sticks to standard training procedures.

Although not devoting much to explaining the emergence of activation sparsity in CNNs, \citet{exploit_sparsity_in_CNN} boost activation sparsity through Hoyer regularization\citep{hoyer} and a new activation function FATReLU that uses dynamic thresholds between activation and deactivation. They also design algorithms to exploit this sparsity, leading to $\ge 1.75\mathrm{x}$ speedup in CNN's inference. Compared to their sparsity encouragement method that requires well-designed procedures to select thresholds, hyperparameters for our theoretically induced modifications can be easily selected. The discontinuity of FATReLU also bothers training from scratch\citep{exploit_sparsity_in_CNN}, while we recommend applying our modifications from scratch to enjoy better sparsity and additionally smaller \emph{training} costs. Regarding exploitation, we consider it out of the manuscript's scope.
\citet{L1_sparsity} encourages activation sparsity in CNN by explicit $L_1$ regularization. We intend to investigate the emergence of activation sparsity from implicit regularization as demonstrated by \citet{observation}, so we solely rely on implicit regularization boosted by modifications. Nevertheless, our methods are architecturally orthogonal and we believe applying both together can further boost activation sparsity.

There are other works that are not devoted to activation sparsity but are related. \citet{sparse_symbol} formulate, with Shapley value, and prove that there are sparse ``symbols'' as groups of patches that are the only major contributors to the output of any well-trained and masking-robust AIs. They provide a sparsity independent of training dynamics. Their theory focuses on symbols and sparsity in inputs, which is inherently different from ours.

In Primer \citep{primer}, several architectural changes given by architecture searching include a new activation function Squared-ReLU. In this work, we induce a similar squared $\relu$ activation but with the non-zero part shifted left and use it to guide the search for flat minima and gradient/activation sparsity. \cite{primer} demonstrate impressive improvements of Squared-ReLU in both ablation and addition experiments, and our work provides a potential explanation for this improvement.
% SECTION Proposed PhoBERT-GCN model %
\section{ViCGCN:  Contextualized Graph Convolution Networks for Vietnamese Social Media}
\label{Proposed model}
% Method body (Explain PhoBERT-GCN in detail)

% PhoBERT\footnote{https://huggingface.co/vinai/phobert-base} (\cite{nguyen-tuan-nguyen-2020-phobert}), with its versions $\text{PhoBERT}_{base}$ and $\text{PhoBERT}_{large}$, respectively, is the first public large-scale monolingual language models for Vietnamese, designed for natural language processing (NLP) task. It is based on the popular Bidirectional Encoder Representations from Transformers\footnote{https://github.com/google-research/bert}, aslo called BERT, architecture, which uses a transformer network to encode the input text and generate high-quality representations of the text.

% \begin{figure}[!h]
%     \centering
%     \includegraphics[width=\textwidth]{figures/ProposedModel/PhoBERT.pdf}
%     \caption{The process of representing the input of PhoBERT model}
%     \label{fig::Proposed/PhoBERT}
% \end{figure}

% More specifically PhoBERT's layers, it has 12 transformer layers, each with a hidden size of 768, and 12 attention heads. The total number of parameters in the model is around 125 million. The input to these layers is tokenized text, which is then converted into embeddings using the embedding layer. These embeddings are then processed through the transformer blocks to generate contextualized word representation.


% In addition to the transformer layers, PhoBERT also includes a pre-processing layer which is responsible for tokenization, sentence segmentation, and special token handling. There is also a pooling layer at the end of the model that generates a fixed-size representation of the input text.

% This architecture and pre-training allows PhoBERT to learn a deep understanding of the structure of Vietnamese language and to have a facility to handle downstream NLP tasks, such as text classification, named entity recognition, and machine translation. As a result, PhoBERT is considered a state-of-the-art NLP model. 

% Generally, according to \cite{Graph-convolutional-networks}, Graph Convolutional Networks (GCN) are a type of graph neural network architecture that can leverage the graph structure of the data and aggregate node information from the neighborhoods in a convolutional fashion. There are two main types of GCN: spectral-based and spatial-based GCN. Spectral-based GCN operates by performing a Fourier transform on the adjacency matrix to obtain its spectrum. Spatial-based GCN, on the other hand, operate directly on the graph structure by propagating messages from node to node in each layer.

% TextGCN\footnote{https://github.com/yao8839836/text{\_}gcn} (\cite{Graph-Convolutional-Networks-for-Text-Classification}) is a variant of spectral-based GCNs and designed specifically for text data. It uses the bag-of-words representation and term frequency-inverse document frequency (TF-IDF) weighting scheme to represent the text data as a weighted graph. 

% TextGCN model consists of two main components: a graph convolutional network (GCN) and a multi-layer perceptron (MLP). The GCN operates on the graph representation of the documents and words to learn the node representations that capture the relationships between them. 

% On the other hand, TextGCN, like other models, has a loss function. It use the cross-entropy loss, which is a multi-class classification loss function, measures the difference between predicted and actual class probabilities. The cross-entropy loss for a single instance is defined as:
% \begin{equation}
% \centering
%     L = - \sum_{i=1}^{c} y_i \log(\hat{y_i})
% \end{equation}

% where $c$ is the number of classes, $y_i$ is the true class label (one-hot encoded), and $\hat{y_i}$ is the predicted probability of class $i$. TextGCN model is trained by minimizing the cross-entropy and the model parameters are learned by backpropagation through the GCN and MLP layers.

% \begin{figure}[!htb] \label{fig::ProposedModel/TextGCN}
%     \centering
%     \includegraphics[width=\textwidth]{figures/ProposedModel/TextGCN.pdf}
%     \caption{Schematic of Text GCN. Example taken from UIT-ViCTSD dataset}
%     \label{fig::Proposed/TextGCN}
% \end{figure}
 
 % \textbf{Graph Neural Network with Contextualized Language Models}
%%% day la phan cua pct
%%% Ly do chon contextualized language models
%%% Ly do chon PhoBERT

Contextualized language models like BERT have shown impressive performance in a wide range of NLP tasks, especially in tasks that require a deep understanding of the meaning of language, such as text classification, sentiment analysis, and named entity recognition. The explanation for this phenomenon is contextual models can capture the contextual meaning of words based on their surrounding words, which is crucial for many NLP tasks. On the other hand, Graph Convolutional Networks (GCN) is a type of graph neural network that can handle graph-structured data, such as text-based dependency graphs, commonly used in Vietnamese language processing. Additionally, GCN is more suitable for semi-supervised learning tasks where the training data is limited and noisy. As a result, the combination of contextualized language model and GCN allows for better modeling of the text data, capturing the complex relationships between words and sentences in a text corpus, leading to improved or showed state-of-the-art (SOTA) performance on a variety of NLP tasks. In this study, we proposed ViCGCN Integrated model and evaluated its efficacy in social media processing for Vietnamese. The ViCGCN architecture consists of two layers, namely the PhoBERT layer and the GCN layer, respectively. Figure \ref{fig::ProposedModel/PhoBERT-GCN} presents an overview of our proposed approach's architecture.

 \begin{figure}[!ht] 
     \centering
     \includegraphics[width=\textwidth]{PhoBERT-GCN.png}
     \caption{An overview of our proposed approach ViCGCN architecture.}
     \label{fig::ProposedModel/PhoBERT-GCN}
 \end{figure}

Firstly, we present the architecture of PhoBERT\footnote{\url{https://huggingface.co/vinai/phobert-base}} (\cite{nguyen-tuan-nguyen-2020-phobert}) and how the PhoBERT model performs as the first layer of our proposed approach. PhoBERT was chosen because PhoBERT is specifically designed for the Vietnamese language, making it highly effective for Vietnamese language processing tasks. 
% PhoBERT with its versions $\text{PhoBERT}_{base}$ and $\text{PhoBERT}_{large}$, respectively, is the first public large-scale monolingual language models for Vietnamese, designed for natural language processing (NLP) task. 
PhoBERT architecture is based on the popular Bidirectional Encoder Representations from Transformers\footnote{\url{https://github.com/google-research/bert}}, also called BERT, architecture, which uses a transformer network to encode the input text and generate high-quality representations of the text.

\begin{figure}[!ht]
    \centering
    \includegraphics[width=\textwidth]{PhoBERT.pdf}
    \caption{The process of representing the contextualized language model's input. "Harry Maguire là một cầu thủ giỏi. Tôi rất thích anh ấy" is "Harry Maguire is a good player. I very like him" in English.}
    \label{fig::Proposed/PhoBERT}
\end{figure}

% More specifically PhoBERT's layers, it has 12 transformer layers, each with a hidden size of 768, and 12 attention heads. The total number of parameters in the model is around 125 million. 
The input to these layers is tokenized text, which is then converted into embeddings using the embedding layer as illustrated in Figure \ref{fig::Proposed/PhoBERT}. These embeddings are then processed through the transformer blocks to generate contextualized word representation. In addition to the transformer layers, PhoBERT also includes a pre-processing layer which is responsible for tokenization, sentence segmentation, and special token handling. In this study, PhoBERT is accountable for processing the input text. It takes in the raw text input and applies a series of transformer-based layers. This produces a contextualized embedding for each word in the input. Then, these contextualized embeddings are fed into the GCN layer. The output of the PhoBERT layer represents the contextualized embeddings for each word in the input.

% \subsection{Graph Neural Networks}

% Generally, according to \cite{Graph-convolutional-networks}, Graph Convolutional Networks (GCN) are a type of graph neural network architecture that can leverage the graph structure of the data and aggregate node information from the neighborhoods in a convolutional fashion. There are two main types of GCN: spectral-based and spatial-based GCN. Spectral-based GCN operates by performing a Fourier transform on the adjacency matrix to obtain its spectrum. Spatial-based GCN, on the other hand, operate directly on the graph structure by propagating messages from node to node in each layer.

% TextGCN\footnote{https://github.com/yao8839836/text{\_}gcn} (\cite{Graph-Convolutional-Networks-for-Text-Classification}) is a variant of spectral-based GCNs and designed specifically for text data. It uses the bag-of-words representation and term frequency-inverse document frequency (TF-IDF) weighting scheme to represent the text data as a weighted graph. 

% TextGCN model consists of two main components: a graph convolutional network (GCN) and a multi-layer perceptron (MLP). The GCN operates on the graph representation of the documents and words to learn the node representations that capture the relationships between them. 

% On the other hand, TextGCN, like other models, has a loss function. It use the cross-entropy loss, which is a multi-class classification loss function, measures the difference between predicted and actual class probabilities. The cross-entropy loss for a single instance is defined as:
% \begin{equation}
% \centering
%     L = - \sum_{i=1}^{c} y_i \log(\hat{y_i})
% \end{equation}

% where $c$ is the number of classes, $y_i$ is the true class label (one-hot encoded), and $\hat{y_i}$ is the predicted probability of class $i$. TextGCN model is trained by minimizing the cross-entropy and the model parameters are learned by backpropagation through the GCN and MLP layers.

% \begin{figure}[!htb] \label{fig::ProposedModel/TextGCN}
%     \centering
%     \includegraphics[width=\textwidth]{figures/ProposedModel/TextGCN.pdf}
%     \caption{Schematic of TextGCN. Example taken from UIT-ViCTSD dataset}
%     \label{fig::Proposed/TextGCN}
% \end{figure}
 
 % In the first layer, a large language model - PhoBERT - is responsible for processing the input text. More specifically, it takes in the raw text input and applies a series of transformer-based layers. This produces a contextualized embedding for each word in the input. Then, these contextualized embeddings are fed into the GCN layer. The output of the PhoBERT layer represents the contextualized embeddings for each word in the input. 

 The second layer, the GCN layer, on the other hand, takes the output of the BERT layer, which is a sequence of contextualized word embeddings, as input, and applies graph convolution operations to aggregate information from the surrounding words in a sentence. To be more specific, we create a heterogeneous graph that comprises both document nodes and word nodes, following the TextGCN\footnote{\url{https://github.com/yao8839836/text_gcn}} \cite{Graph-Convolutional-Networks-for-Text-Classification}. Figure \ref{fig::Proposed/TextGCN} schematically presents the overall GCN layer of our integrated model.

\begin{figure}[!htb]
    \centering
    \includegraphics[width=\textwidth]{TextGCN.pdf}
    \caption{Schematic of GCN layer in ViCGCN. A social media example is taken from UIT-ViCTSD dataset. "Bu*i", "C*c", "Giỏi", "Hay", and "Tốt" are "D*ck", "C*ck", "excellent", "Good", and "Nice" in English, respectively.}
    \label{fig::Proposed/TextGCN}
\end{figure}

% For more in-depth, the GCN layer uses the dependency parse tree of the sentence to construct a graph where the words' sentences are represented as nodes and their syntactic relationships are represented as edges. In the case of the ViCGCN model, the graph represents the relationship between words and sentences in a text document. To be more specific, we create a heterogeneous graph that includes nodes for both words and documents using the approach of TextGCN\footnote{\url{https://github.com/yao8839836/text_gcn}} \cite{Graph-Convolutional-Networks-for-Text-Classification}. To establish connections between word and document (sentences in our situation) nodes, we utilize the term frequency-inverse document frequency (TF-IDF) to define edges between word-document pairs and positive point-wise mutual information (PPMI) to define edges between word-word pairs. The weight of an edge
% between two nodes, \textit{i} and \textit{j} defined as: 

% For more in-depth, the GCN layer uses the dependency parse tree of the sentence to construct a graph where the words' sentences are represented as nodes and their syntactic relationships are represented as edges. In the context of the ViCGCN model, the graph depicts the associations between words and sentences within a textual document. To elaborate further, we construct a diverse graph encompassing nodes for both words and documents, drawing inspiration from the TextGCN\footnote{\url{https://github.com/yao8839836/text_gcn}} \cite{Graph-Convolutional-Networks-for-Text-Classification}. To establish links between word and document nodes, we employ the term frequency-inverse document frequency (TF-IDF) to define connections between word-document pairs. We also employ positive point-wise mutual information (PPMI) to define connections between word-word pairs. The weight of an edge linking two nodes, denoted as \textit{i} and \textit{j}, is defined as follows:

To provide a more comprehensive understanding, the GCN layer in our approach utilizes the dependency parse tree of the sentence to create a graphical representation. In this graph, the sentences' words are represented as nodes, and their syntactic relationships are captured as edges. Within the ViCGCN model, this graph serves to illustrate the relationships among words and sentences within a given text document.

To delve deeper into the methodology, we establish a diverse graph that encompasses nodes representing both words and entire documents, drawing inspiration from TextGCN\footnote{\url{https://github.com/yao8839836/text_gcn}} \cite{Graph-Convolutional-Networks-for-Text-Classification}. To establish connections between word and document nodes, we make use of the term frequency-inverse document frequency (TF-IDF) measure, which helps define the associations between word-document pairs. Additionally, we employ positive point-wise mutual information (PPMI) to establish connections between word-word pairs. The weight of an edge connecting two nodes, denoted as \textit{i} and \textit{j}, is defined as follows:
\begin{equation}
    A_{i, j} = 
    \begin{cases}
        PPMI(i, j), & \textit{i, j are words and i} \neq j \\
        TF-IDF(i, j), & \textit{i is document, j is word} \\
        1, & i = j \\
        0, & otherwise
    \end{cases}
\end{equation}

% In ViCGCN, the PhoBERT model obtains the document embeddings and treats them as input representations for document nodes.  Document node embeddings are denoted by $X_{doc} \in \mathbb{R}^{n_{doc} \times d}$, where $n_{doc}$ is is the number of document nodes, $n_{word}$ is the number of word nodes (including both training and testing), $d$ is the embedding dimensionality. As a result, the initial node feature matrix is given by:

In ViCGCN, the contextual language model PhoBERT is responsible for acquiring document embeddings and considering them input representations for document nodes. These document node embeddings are represented as $X_{doc} \in \mathbb{R}^{n_{doc} \times d}$, where $n_{doc}$ signifies the count of document nodes, $n_{word}$ represents the count of word nodes (comprising both training and testing), and $d$ denotes the dimensionality of the embeddings. Consequently, the initial matrix of node features is formulated as follows:

\begin{equation}
    X = 
    \begin{pmatrix}
        X_{doc} \\
        0
    \end{pmatrix}_{(n_{doc}+n_{word}) \times d}
\end{equation}

 % Then X is fed into a series of Graph Convolutional Networks layers, where each layer aggregates information from the neighbors of each node to refine its representation. Specifically, the output feature matrix of the $i$-th GCN layer $L^{(i)}$ is computed as where $f$ is an activation function, $\tilde{A}$ is the normalized adjacency matrix and $W^{(i)} \in \mathbb{R}^{d_{i-1} \times d_i}$ is a weight matrix of the layer. $L^{(0)} = X$ is the input feature matrix of the model. 

 Then X is fed into a series of Graph Convolutional Networks layers, where each layer aggregates information from the neighbors of each node to refine its representation. More precisely, the output feature matrix for the $i$-th GCN layer, denoted as $L^{(i)}$, is calculated as follows: it involves an activation function represented by $f$, utilizes the normalized adjacency matrix denoted as $\tilde{A}$, and incorporates a weight matrix $W^{(i)} \in \mathbb{R}^{d_{i-1} \times d_i}$ specific to that layer. The initial input feature matrix of the model is denoted as $L^{(0)} = X$.

\begin{equation}
    L^{(i)} = f(\tilde{A}L^{(i-1)}W^{(i})
\end{equation}
 
 The output of the GCN layer is a set of updated embeddings, which capture the interactions among the words in the sentence and pass through a softmax activation layer to obtain the final predictions, where $g$ represents the GCN model: 
 \begin{equation}
     \textbf{Z}_{\text{GCN}} = softmax(g(X, A))
 \end{equation}

Moreover, an auxiliary classifier on BERT embeddings is conducted by directly feeding document embeddings (denoted by $X$) to a dense layer with softmax activation.
\begin{equation}
    \textbf{Z}_{\text{BERT}} = softmax(WX)
\end{equation}

 To combine the output embeddings of the PhoBERT and GCN layers and obtain the best classification performance, we propose to use a hyperparameter $\lambda$ to control the trade-off between them in the final classification. Specifically, we compute a weighted sum of the two embeddings using the following equation:
\begin{equation}
    \mathbf{Z} = \lambda \mathbf{Z}_{\text{GCN}} + (1-\lambda) \mathbf{Z}_{\text{PhoBERT}}
    \label{equa::lambda}
\end{equation}
 % $$\mathbf{Z} = \lambda \mathbf{Z}_{\text{GCN}} + (1-\lambda) \mathbf{Z}_{\text{PhoBERT}}$$

 where $\mathbf{Z}_{\text{GCN}}$ is the output embedding of the GCN layer and $\mathbf{Z}_{\text{PhoBERT}}$ is the output embedding of the PhoBERT layer. The softmax function normalizes the output and produces class probabilities for text classification. Moreover, comprehensive experiments were conducted on the three benchmarks UIT-VSMEC, UIT-ViCTSD, and UIT-VSFC to determine the optimal lambda value for the ViCGCN model in Section \ref{imapactGCN}.

 By combining the power of PhoBERT's contextualized embeddings with the ability of GCN to capture syntactic and semantic dependencies, the ViCGCN can achieve better performance on social media processing tasks, especially those that require an understanding of semantic relationships between words. Furthermore, the ViCGCN model can also handle a broader range of text inputs, including more extended and more complex sentences, due to its ability to capture the contextualized meaning of words and the syntactic and semantic dependencies between them. This makes it a highly effective tool for natural language processing tasks, especially text classification and social media processing tasks.
% SECTION Experiements %
\section{Experiments and Analysis} \label{Experiments}
\subsection{Experimental Design}
% This section outlines our methodology to propose a new classification model ViCGCN. Firstly, three benchmark datasets mentioned in Section \ref{Experiments/Datasets} are collected, and they be cleaned as described in the following. Consequently, the data after pre-processing is used to train our baselines and proposed model. With each model implemented, we fine-tune to find optimal hyper-parameters and improve their performance. Then, we evaluated the performance of models by Macro F1-score and Weighted F1-score deputed in Section \ref{Experiments/Metrics}. Section \ref{Experiments/Result} details the model evaluation results. To better understand our proposed model, we analyze and discuss the proposed model from various aspects: Impact of graph convolutional networks (see Section \ref{imapactGCN}) and impact of lambda (see Section \ref{impactlamda}). In addition, comparisons with previous studies were made to assess the achievement of our study correctly (see Section \ref{comparisonprestudies}). We also select the model that exhibits the most exceptional performance to conduct an error analysis on the inaccurate predictions detected within our proposed model (see Section \ref{erroranalysis}). At the end of the experiment, an ablation study was made to investigate the effectiveness and contribution of our proposed approach ViCGCN (see Section \ref{ablationstudy}). Figure \ref{fig::Experiments/Procedure/Overview} illustrates our methodology, including data preparation, fine-tuning baselines, proposed model, and performance analysis.

This section delineates our approach for introducing a novel classification model called ViCGCN. Initially, we gather three benchmark datasets, as mentioned in Section \ref{Experiments/Datasets}, and subject them to a cleaning process described subsequently. Subsequently, the pre-processed data is employed for training both our baseline models and the proposed model. We fine-tune each model to identify optimal hyperparameters and enhance their performance. The evaluation of model performance is conducted using Macro F1-score and Weighted F1-score, as discussed in Section \ref{Experiments/Metrics}. Detailed results of model evaluations are presented in Section \ref{Experiments/Result}.

To gain a deeper insight into our proposed model, we conduct a comprehensive analysis and discussion from various angles. This includes assessing the impact of graph convolutional networks (see Section \ref{imapactGCN}) and the influence of the lambda parameter (see Section \ref{impactlamda}). Additionally, we make comparisons with prior studies to accurately gauge the accomplishments of our research (see Section \ref{comparisonprestudies}). Furthermore, we select the model that exhibits the most outstanding performance to carry out an error analysis on the inaccuracies detected within our proposed model (see Section \ref{erroranalysis}). Towards the end of the experiment, we conduct an ablation study to investigate the effectiveness and contribution of our proposed approach, ViCGCN (see Section \ref{ablationstudy}). Figure \ref{fig::Experiments/Procedure/Overview} provides an overview of our methodology, encompassing data preparation, baseline fine-tuning, the proposed model, and performance analysis.

\begin{figure}[!hpbt]
    \centering
    \includegraphics[width=\textwidth]{Procedures.png}
    \caption{Overview of our experimental design.}
    \label{fig::Experiments/Procedure/Overview}
\end{figure}

\subsection{Baseline Models} \label{Experiments/Baseline}
Contextualized language models have been extensively used in various natural language processing tasks, including text classification. Additionally, since PhoBERT and viBERT are monolingual models specifically designed for the Vietnamese language, comparing their performance with a widely used and established model like mBERT is essential. Furthermore, as GCN has been shown to effectively capture the context and relationships between words in a text, integrating it with a contextualized language model could improve its performance in text classification tasks. Because of the following reasons, we compare our ViCGCN model with baseline models.
\subsubsection{Contextualized Language Models}
\begin{itemize}
    % \item \textbf{BERT\footnote{\url{https://github.com/google-research/bert}}}: BERT is a contextualized word representation model pre-trained using bidirectional transformers and based on a masked language model. BERT showed power in various NLP tasks. BERT and its variants are called the BERTology, two versions of BERT, base and large, respectively. Moreover, each version has two different versions: cased\footnote{\url{https://huggingface.co/bert-base-cased}} and uncased\footnote{\url{https://huggingface.co/bert-base-uncased}}, respectively. The only difference is that in \textit{BERT case uncased}, the text has been lowercase before the WordPiece tokenization step, while in the mBERT cased version, the text is the same as the input text.
    % \item \textbf{BERT (case uncased)}
    \item \textbf{Multilingual BERT (mBERT)\footnote{\url{https://github.com/google-research/bert}}}: mBERT, introduced by \citet{devlin-etal-2019-bert}, is a BERT-based model with specific characteristics. It consists of 12 layers, 768 hidden units, 12 attention heads, and a total of 110 million parameters. Remarkably, mBERT is designed to support 104 distinct languages, and it has been trained on and can be applied to text in these 104 languages using a combination of masked language modeling (MLM) and next sentence prediction objectives. This training corpus includes content from Wikipedia\footnote{\url{https://www.wikipedia.org/}}. \textit{mBERT} consists of two versions cased\footnote{\url{https://huggingface.co/bert-base-multilingual-cased}} and uncased\footnote{\url{https://huggingface.co/bert-base-multilingual-uncased}}.
    \item \textbf{RoBERTa\footnote{\url{https://huggingface.co/roberta-base}}}: \citet{DBLP:journals/corr/abs-1907-11692} proposed RoBERTa. They utilize a dynamic masking technique during the training process, instructing the model to predict intentionally hidden segments of text within unannotated language samples. RoBERTa, implemented using the PyTorch framework, makes critical adjustments to BERT's essential hyperparameters.
    \item \textbf{XLM-RoBERTa (XLM-R)\footnote{\url{https://github.com/facebookresearch/XLM}}}: \citet{XLMR}  proposed XLM-R a masked language model based on the transformer architecture. This model stands out as a multilingual powerhouse, having been pre-trained on text from a staggering 100 languages. What makes XLM-R particularly impressive is the extensive and careful curation of over 2.5TB of data from CommonCrawl. Among its notable contributions are the improvements made for low-resource languages through specialized training and vocabulary expansion. Moreover, XLM-R boasts a more expansive shared vocabulary and a substantial increase in its overall model capacity, incorporating a whopping 550 million parameters. XLM-R includes \textit{base}\footnote{\url{https://huggingface.co/xlm-roberta-base}} and \textit{large}\footnote{\url{https://huggingface.co/xlm-roberta-large}} version.
    \item \textbf{PhoBERT\footnote{\url{https://huggingface.co/vinai/phobert-base}}}: \citet{nguyen-tuan-nguyen-2020-phobert} introduced a set of large-scale monolingual language models specifically designed for the Vietnamese language. Among these models, PhoBERT stands out as the state-of-the-art contextualized language model for Vietnamese. PhoBERT's architecture is built upon the RoBERTa model, but it has been optimized for training on a substantial Vietnamese corpus to effectively handle Vietnamese text. PhoBERT comes in two versions: \textit{base} and the \textit{large} versions.
    \item \textbf{viBERT\footnote{\url{https://huggingface.co/FPTAI/vibert-base-cased}}}: \citet{viBERT} introduced viBERT, a pre-trained language model for Vietnamese based on the BERT architecture. The architecture of viBERT is similar to that of mBERT, and it has been pre-trained on a large corpus of 10GB of uncompressed Vietnamese text. However, unlike mBERT, viBERT excludes insufficient vocabulary due to the inclusion of languages other than Vietnamese in the mBERT vocabulary.
    \item \textbf{vELECTRA\footnote{\url{https://huggingface.co/FPTAI/velectra-base-discriminator-cased}}}: \citet{viBERT}  unveiled vELECTRA, a pre-trained language model tailored for Vietnamese that adheres to the ELECTRA framework. vELECTRA shares a parallel architectural structure with ELECTRA and has undergone pretraining on an extensive corpus comprising 60GB of uncompressed Vietnamese text.
\end{itemize}

\subsubsection{Other Graph Neural Networks}
Bert-GCN was introduced by \citet{BertGCN}, presenting a novel approach that harnesses the benefits of extensive pretraining alongside transductive learning for the purpose of text classification. Bert-GCN achieves this by constructing a diverse graph over the dataset, where documents are represented as nodes, all leveraging the embedding power of BERT. Consequently, this research undertakes the implementation of various BERT variations, such as multilingual and Vietnamese monolingual models, in conjunction with GCN-combined models to assess their effectiveness in text classification for Vietnamese tasks. Additionally, when compared to mBERT-GCN, RoBERTa-GCN, viBERT-GCN, and vELECTRA-GCN, our proposed ViCGCN model offers valuable insights into the impact of integrating both monolingual and multilingual Contextualized Language Models with GCN on three standardized benchmark datasets. 

\subsection{Benchmark Datasets} \label{Experiments/Datasets}
\subsubsection{Benchmark Datasets} \label{Experiments/Datasets/Data}
To verify the efficiency of our proposed approach to text classification on Vietnamese social media, we conducted our experiments on three widely used Vietnamese social media corpora, including Vietnamese Social Media Emotion Corpus (UIT-VSMEC) that was made available by Ho et al. \citet{DBLP:journals/corr/abs-1911-09339}, Vietnamese Students' Feedback Corpus (UIT-VSFC) built by \citet{VSFC}, and Vietnamese Constructive and Toxic Speech Detection (UIT-ViCTSD) introduced by \citet{DBLP:journals/corr/abs-2103-10069}.


\begin{itemize}
    \item \textbf{UIT-VSMEC \citet{DBLP:journals/corr/abs-1911-09339}}: UIT-VSMEC consists of 6,927 sentences that have been annotated with emotions to tackle the challenge of identifying emotions in Vietnamese social media comments. This dataset encompasses seven emotion categories: Enjoyment, Disgust, Sadness, Anger, Fear, Surprise, and Other.
    \item \textbf{UIT-VSFC \citet{VSFC}}: UIT-VSFC comprises 16,000 sentences that have been investigated for two distinct purposes: one related to sentiment analysis and the other related to topic classification. The sentiment analysis task involves categorizing sentences into three classes: Positive, Negative, and Neutral. Meanwhile, the topic classification task involves assigning sentences to one of four categories: Lecturer, Curriculum, Facility, or Others.
    \item \textbf{UIT-ViCTSD \cite{DBLP:journals/corr/abs-2103-10069}}: UIT-ViCTSD consists of 10,000 human-annotated comments on ten domains. Each comment is categorized into two tasks: constructiveness and toxicity in Vietnamese social media, which are binary classifications. Two categories are used to denote feedback: constructive and non-constructive. Similarly, comments can be labeled as toxic or non-toxic to identify harmful behavior.
\end{itemize}

\subsubsection{Pre-processing techniques}
A few efficient pre-processing techniques for Vietnamese text in general and Vietnamese social media text in particular were presented \cite{nguyen2020exploiting, PhoBERT-CNN}. However, we only follow some simple preprocessed techniques according to the quality of the three benchmark datasets mentioned in Section \ref{Experiments/Datasets/Data} and more essential to prove the outperform and efficiency of our model ViCGCN on Vietnamese social media raw text. Firstly, we removed stopwords defined in Vietnamese stopwords dict\footnote{\url{https://github.com/stopwords/vietnamese-stopwords}}. We, then, segment sentences into words by applying Word Segmenter of VnCoreNLP\footnote{\url{https://github.com/vncorenlp/VnCoreNLP}} for all of the models. Finally, the Regex\footnote{\url{https://docs.python.org/3/library/re.html}} library in Python is used to remove all punctuations in three benchmark datasets.

% remove stopwords, segmentation, remove punctuation 

The statistics of the pre-processed datasets are summarized in Table \ref{4/Dataset}.

% Experiments/Datasets/Table %
    
\begin{table}[!ht]
\centering
\caption{Statistics and descriptions of tasks of each dataset in this study.}
\resizebox{\linewidth}{!}{%
\begin{tabular}{lrrrlr} 
\hline
\textbf{Dataset}            & \multicolumn{1}{l}{\textbf{Train}} & \multicolumn{1}{l}{\textbf{Dev}} & \multicolumn{1}{l}{\textbf{Test}} & \multicolumn{1}{c}{\textbf{Task}}         & \multicolumn{1}{l}{\textbf{Classes}}  \\ 
\hline
\multicolumn{6}{c}{\textit{Binary text classification}}                                                                                                                                                                     \\ 
\hline
\multirow{2}{*}{UIT-ViCTSD} & 7,000                              & 2,000                            & 1,000                             & Constructive speech detection             & 2                                     \\
                            & 7,000                              & 2,000                            & 1,000                             & Toxic speech detection                    & 2                                     \\ 
\hline
\multicolumn{6}{c}{\textit{Multi-class text classification}}                                                                                                                                                                \\ 
\hline
\multirow{2}{*}{UIT-VSMEC}  & 5,548                              & 686                              & 693                               & Emotion recognition (with Other label)    & 7                                     \\
                            & 4,527                              & 583                              & 589                               & Emotion recognition (without Other label) & 6                                     \\ 
\cline{1-1}
\multirow{2}{*}{UIT-VSFC}   & 11,426                             & 1,583                            & 3,166                             & Sentiment-based classification            & 3                                     \\
                            & 11,426                             & 1,583                            & 3,166                             & Topic-based classification                & 4                                     \\
\hline
\end{tabular}}
\label{4/Dataset}
\end{table}


\subsection{Evaluation Metric} \label{Experiments/Metrics}
% This section describes the performance evaluation metrics employed in this study. The commonly used metric for classification tasks, particularly for the three datasets mentioned in this study, is the Average Macro F1-score (\%). However, owing to significantly imbalanced classes in the given datasets, the most suitable metric for this study is the average macro F1-score, which is the harmonic mean of Precision and Recall. Additionally, to facilitate comparisons with previous studies, we used the corresponding measure based on the metrics used in those studies, such as the average weighted F1-score (\%) for both UIT-VSMEC and UIT-VSFC.

This section outlines the performance evaluation criteria utilized in this research. In the realm of classification tasks, especially concerning the three datasets highlighted within this study, the conventional metric employed is the Average Macro F1-score (\%). However, given the significant class imbalances in the provided datasets, the most appropriate metric for this study is the Average Macro F1-score, which is derived as the harmonic mean of Precision and Recall. Furthermore, to facilitate comparisons with prior studies, we have also adopted relevant measures based on the metrics employed in those studies, such as the Average Weighted F1-score (\%) for both UIT-VSMEC and UIT-VSFC datasets.

 To compute the average macro F1-score, firstly, we calculate Precision and Recall by Equation (\ref{eq::Experiments/Metrics/Presision}) and Equation (\ref{eq::Experiments/Metrics/Recall}) respectively. Then, Equation (\ref{eq::Experiments/Metrics/F1-score}) is used to determine F1-score per class in the dataset. $tp$ are truly positive, $fp$ – false positive, $fn$ – false negative, and $tn$ – true negative counts, respectively.
\begin{equation}
    Precision = \frac{tp}{tp+fp} \label{eq::Experiments/Metrics/Presision}
\end{equation}
\begin{equation}
    Recall=\frac{tp}{tp+fn} \label{eq::Experiments/Metrics/Recall}
\end{equation}
\begin{equation}
    \textit{F1-score}=2\times\frac{Precision\times Recall}{Precision+Recall} \label{eq::Experiments/Metrics/F1-score}
\end{equation}

We compute the average macro F1-score (mF1) and weighted F1-score (wF1) after acquiring the F1 scores for all classes. Equation (\ref{eq::Experiments/Metrics/macro F1-score}) and Equation (\ref{eq::Experiments/Metrics/weighted F1-score}) present the macro F1-score and weighted F1-score, respectively, for multi-class classification for multi classes $C_{i}$, i $\in$ \{1, 2,... n\} (denoted for every class of the dataset). Where $\textit{F1-score}_{i}$ and $W_{i}$ are the \textit{F1-score} and weight of class \textit{i} of the dataset, respectively.

\begin{equation}
    \textit{mF1} = \frac{{\sum_{i=1}^{n} \textit{F1-score}_{i}}}{n} \label{eq::Experiments/Metrics/macro F1-score}
\end{equation}
\begin{equation}
    \textit{wF1} = \frac{\sum_{i=1}^{n} {\textit{F1-score}_{i} \times W_{i}}}{\sum_{i=1}^{n} W_{i}} \label{eq::Experiments/Metrics/weighted F1-score}
\end{equation}
\subsection{Experiment Configuration}
% In this study, we implemented many transfer learning models including $\text{BERT}_{base}$ \textit{cased}, $\text{BERT}_{base}$ \textit{uncased}, mBERT \textit{cased}, mBERT \textit{uncased}, $\text{RoBERTa}_{large}$, $\text{PhoBERT}_{base}$, $\text{PhoBERT}_{large}$. Besides, several combined models are conducted along with Text-GCN, Bert-GCN and mBERT-GCN. 
Section \ref{Experiments/Configuration/Baseline} and Section \ref{Experiments/Configuration/Proposed} provide our settings for both baselines and the proposed approach in detail.

\subsubsection{Basesline models' configuration} \label{Experiments/Configuration/Baseline}
We implemented many transfer learning models including mBERT both \textit{cased} and \textit{uncased}, $\text{RoBERTa}$, XLM-R, $\text{PhoBERT}_{base}$, $\text{PhoBERT}_{large}$, vELECTRA, and viBERT in this study. They run with their max sequence length of 256, batch size of 32, epoch of 10, and Adam optimizer \cite{https://doi.org/10.48550/arxiv.1412.6980} with a fixed learning rate of 2e-5.
 
\subsubsection{Our approach's configuration} \label{Experiments/Configuration/Proposed}
In our proposed approach, $\text{PhoBERT}_{base}$ is the output feature of the [CLS] token as the sentence node, followed by a feedforward layer to derive the final prediction. We use $\text{PhoBERT}_{base}$ pre-trained model from HuggingFace combined with a two-layer GCN to implement ViCGCN. We initialize Adam optimizer \cite{https://doi.org/10.48550/arxiv.1412.6980} with a fixed learning rate of 1e-3 and 1e-5 for the GCN and PhoBERT module, respectively. Moreover, PhoBERT runs with a 256 max sequence length.

\subsection{Experimental Results} \label{Experiments/Result}


% \begin{table}[!hpbt]
% \centering
% \caption{F1-score performances of models on the test sets of various Vietnamese social media textual datasets. Improvement (1) and Improvement (2) denoted the improvement over BERTology models and the improvement over BERTology integrated with GCN models, respectively}
% \label{tab::Experiments/Result}
% \resizebox{\linewidth}{!}{%
% \begin{tabular}{c|cc|cc|cc|cc|cc|cc} 
% \hline
% \textbf{Datasets}                & \multicolumn{4}{c|}{\textbf{UIT-VSMEC}}                                               & \multicolumn{4}{c|}{\textbf{UIT-ViCTSD}}                                                & \multicolumn{4}{c}{\textbf{UIT-VSFC}}                                                     \\ 
% \hline
% \textbf{Tasks}                   & \multicolumn{2}{c|}{\textbf{Seven labels}} & \multicolumn{2}{c|}{\textbf{Six labels}} & \multicolumn{2}{c|}{\textbf{Constructiveness}} & \multicolumn{2}{c|}{\textbf{Toxicity}} & \multicolumn{2}{c|}{\textbf{Sentiment-based}} & \multicolumn{2}{c}{\textbf{Topic-based}}  \\ 
% \hline
%                                  & \textbf{wF1}   & \textbf{mF1}              & \textbf{wF1}   & \textbf{mF1}            & \textbf{wF1}   & \textbf{mF1}                  & \textbf{wF1}   & \textbf{mF1}          & \textbf{wF1}   & \textbf{mF1}                 & \textbf{wF1}   & \textbf{mF1}             \\ 
% \hline
% BERT (cased)                     & 55.06          & 55.88                     & 66.34          & 65.31                   & 79.34          & 77.29                         & 87.45          & 64.49                 & 86.32          & 72.38                        & 86.17          & 73.62                    \\
% BERT (uncased)                   & 55.98          & 56.40                     & 58.55          & 56.65                   & 80.42          & 78.90                         & 85.21          & 63.28                 & 85.51          & 71.44                        & 85.98          & 72.56                    \\
% mBERT (\textit{cased)}           & 61.23          & 59.57                     & 66.72          & 66.72                   & 78.15          & 76.93                         & 86.71          & 64.36                 & 91.21          & 78.94                        & 88.06          & 77.63                    \\
% mBERT (\textit{uncased)}         & 58.31          & 57.03                     & 67.13          & 67.70                   & 78.11          & 76.63                         & 87.65          & 64.96                 & 89.52          & 76.60                        & 87.12          & 76.96                    \\
% RoBERTa                          & 56.51          & 56.22                     & 62.11          & 64.64                   & 78.92          & 77.71                         & 83.82          & 61.73                 & 90.11          & 76.98                        & 87.03          & 76.77                    \\
% XLM-R              & 69.55          & 68.12                     & 68.66          & 68.32                   & 81.97          & 80.02                         & 88.35          & 65.21                 & 91.02          & 76.95                        & 87.32          & 76.25                    \\
% PhoBERT \textit{base}            & 71.86          & 69.58                     & 75.19          & 74.32                   & 81.03          & 79.53                         & 88.83          & 65.61                 & 90.51          & 76.47                        & 87.84          & 76.98                    \\
% PhoBERT \textit{large}           & 72.87          & 70.22                     & 76.22          & 75.32                   & 83.21          & 80.22                         & 89.32          & 66.21                 & 91.81          & 77.81                        & 88.12          & 77.22                    \\
% viBERT                           & 67.55          & 65.32                     & 69.95          & 69.08                   & 82.27          & 80.12                         & 88.13          & 64.35                 & 89.77          & 76.25                        & 87.43          & 76.25                    \\ 
% \hline
% Bert-GCN (BERT \textit{cased)}   & 74.22          & 73.51                     & 78.25          & 77.02                   & 81.13          & 79.72                         & 88.10          & 64.52                 & 88.72          & 76.23                        & 88.15          & 76.25                    \\
% Bert-GCN (BERT \textit{uncased)} & 74.18          & 73.29                     & 80.54          & 78.31                   & 81.05          & 79.60                         & 88.67          & 64.96                 & 88.51          & 75.99                        & 87.75          & 76.06                    \\
% mBERT-GCN (mBERT cased)          & 75.12          & 74.83                     & 79.55          & 77.84                   & 82.15          & 80.33                         & 89.13          & 65.13                 & 91.89          & 79.84                        & 89.73          & 79.02                    \\
% mBERT-GCN (mBERT uncased)        & 74.56          & 73.98                     & 80.21          & 78.12                   & 82.88          & 81.12                         & 89.83          & 65.89                 & 91.72          & 79.64                        & 88.89          & 78.82                    \\
% RoBERTa-GCN                      & 74.82          & 74.22                     & 79.33          & 78.32                   & 83.47          & 82.77                         & 86.55          & 64.33                 & 91.12          & 79.32                        & 90.12          & 79.34                    \\
% viBERT-GCN                       & 78.25          & 78.37                     & 82.33          & 81.98                   & 86.12          & 85.02                         & 91.27          & 75.93                 & 93.27          & 87.52                        & 92.11          & 88.35                    \\ 
% \hline
% \textbf{ViCGCN}                  & \textbf{80.24} & \textbf{80.96}            & \textbf{84.91} & \textbf{83.27}          & \textbf{86.97} & \textbf{85.81}                & \textbf{91.95} & \textbf{76.29}        & \textbf{94.81} & \textbf{88.80}               & \textbf{93.91} & \textbf{89.61}           \\
% Improvement (1)                  & $\uparrow$7.37 & $\uparrow$10.74           & $\uparrow$8.69 & $\uparrow$7.95          & $\uparrow$3.76 & $\uparrow$5.59                & $\uparrow$2.63 & $\uparrow$9.98        & $\uparrow$3.00 & $\uparrow$9.86               & $\uparrow$5.69 & $\uparrow$11.98          \\
% Improvement (2)                  & $\uparrow$1.99 & $\uparrow$2.59            & $\uparrow$2.58 & $\uparrow$1.29          & $\uparrow$0.85 & $\uparrow$0.79                & $\uparrow$0.68 & $\uparrow$0.26        & $\uparrow$1.54 & $\uparrow$1.28               & $\uparrow$1.70 & $\uparrow$1.26           \\
% \hline
% \end{tabular}}
% \end{table}

\begin{table}[!ht]
\centering
\caption{F1-score performances of models on the test sets of various Vietnamese social media textual datasets. Improvement (1) and Improvement (2) denoted the improvement over BERTology models and the improvement over BERTology integrated with GCN models, respectively.}
\label{tab::Experiments/Result}
\resizebox{\linewidth}{!}{%
\begin{tabular}{c|cc|cc|cc|cc|cc|cc} 
\hline
\textbf{Datasets}        & \multicolumn{4}{c|}{\textbf{UIT-VSMEC}}                                               & \multicolumn{4}{c|}{\textbf{UIT-ViCTSD}}                                                & \multicolumn{4}{c}{\textbf{UIT-VSFC}}                                                     \\ 
\hline
\textbf{Tasks}           & \multicolumn{2}{c|}{\textbf{Seven labels}} & \multicolumn{2}{c|}{\textbf{Six labels}} & \multicolumn{2}{c|}{\textbf{Constructiveness}} & \multicolumn{2}{c|}{\textbf{Toxicity}} & \multicolumn{2}{c|}{\textbf{Sentiment-based}} & \multicolumn{2}{c}{\textbf{Topic-based}}  \\ 
\hline
                         & \textbf{wF1}   & \textbf{mF1}              & \textbf{wF1}   & \textbf{mF1}            & \textbf{wF1}   & \textbf{mF1}                  & \textbf{wF1}   & \textbf{mF1}          & \textbf{wF1}   & \textbf{mF1}                 & \textbf{wF1}   & \textbf{mF1}             \\ 
\hline
mBERT (\textit{cased)}   & 60.47          & 59.48                     & 65.02          & 62.65                   & 81.03          & 79.55                         & 88.32          & 65.63                 & 90.39          & 77.15                        & 87.32          & 77.93                    \\
mBERT (\textit{uncased)} & 60.17          & 59.18                     & 64.93          & 62.11                   & 80.89          & 79.47                         & 87.6           & 64.77                 & 89.95          & 77.8                         & 87.62          & 77.58                    \\
RoBERTa                  & 58.17          & 57.32                     & 63.32          & 59.97                   & 77.41          & 75.62                         & 85.85          & 59.71                 & 87.13          & 75.52                        & 86.77          & 75.30                    \\
XLM-R                    & 62.02          & 61.01                     & 68.19          & 63.70                   & 81.81          & 80.85                         & 89.92          & 73.09                 & 93.03          & 82.61                        & 89.67          & 79.25                    \\
PhoBERT \textit{base}    & 64.36          & 61.41                     & 69.02          & 64.12                   & 81.65          & 80.24                         & 89.58          & 72.12                 & 92.94          & 82.15                        & 88.29          & 78.54                    \\
PhoBERT \textit{large}   & 65.12          & 63.23                     & 71.13          & 65.12                   & 82.07          & 81.27                         & 90.12          & 73.32                 & 93.24          & 82.96                        & 88.72          & 79.12                    \\
vELECTRA                 & 63.58          & 61.38                     & 68.33          & 63.12                   & 82.41          & 80.82                         & 89.33          & 72.02                 & 91.89          & 82.01                        & 88.12          & 78.12                    \\
viBERT                   & 61.33          & 60.28                     & 68.48          & 62.09                   & 81.62          & 80.07                         & 89.14          & 71.87                 & 91.29          & 81.95                        & 88.22          & 78.35                    \\ 
\hline
mBERT-GCN (cased)        & 68.32          & 64.32                     & 69.32          & 66.18                   & 83.12          & 82.88                         & 90.32          & 69.42                 & 92.12          & 79.32                        & 88.32          & 79.42                    \\
mBERT-GCN (uncased)      & 67.98          & 64.11                     & 69.12          & 65.89                   & 82.32          & 82.01                         & 89.15          & 68.32                 & 91.01          & 79.02                        & 88.07          & 79.02                    \\
RoBERTa-GCN              & 66.17          & 62.12                     & 67.12          & 64.17                   & 81.33          & 80.96                         & 89.02          & 64.32                 & 90.12          & 78.42                        & 87.45          & 78.12                    \\
vELECTRA-GCN             & 69.42          & 65.44                     & 70.95          & 67.20                   & 84.62          & 84.62                         & 91.88          & 74.85                 & 93.56          & 83.12                        & 89.95          & 80.02                    \\
viBERT-GCN               & 69.32          & 65.12                     & 70.83          & 66.68                   & 84.32          & 83.12                         & 91.12          & 74.25                 & 93.12          & 82.47                        & 89.42          & 79.63                    \\ 
\hline
\textbf{ViCGCN (base)}   & \textbf{70.32} & \textbf{67.17}            & \textbf{71.02} & \textbf{67.48}          & \textbf{85.64} & \textbf{85.12}                & \textbf{92.22} & \textbf{75.32}        & \textbf{94.12} & \textbf{83.67}               & \textbf{90.12} & \textbf{80.11}           \\
\textbf{ViCGCN (large)}  & \textbf{71.33} & \textbf{67.82}            & \textbf{72.08} & \textbf{68.12}          & \textbf{86.12} & \textbf{85.88}                & \textbf{93.11} & \textbf{76.12}        & \textbf{94.83} & \textbf{84.23}               & \textbf{91.02} & \textbf{81.88}           \\ 
\hline
Improvement (1)          & $\uparrow$6.21 & $\uparrow$4.59            & $\uparrow$0.95 & $\uparrow$3.00          & $\uparrow$3.71 & $\uparrow$4.61                & $\uparrow$2.99 & $\uparrow$2.80        & $\uparrow$1.59 & $\uparrow$1.27               & $\uparrow$1.35 & $\uparrow$2.63           \\
Improvement (2)          & $\uparrow$1.91 & $\uparrow$2.38            & $\uparrow$1.13 & $\uparrow$0.92          & $\uparrow$1.50 & $\uparrow$1.26                & $\uparrow$1.23 & $\uparrow$1.27        & $\uparrow$1.27 & $\uparrow$1.11               & $\uparrow$1.07 & $\uparrow$1.86           \\
\hline
\end{tabular}}
\end{table}

To demonstrate the classification performance of our model ViCGCN, we compare it with other state-of-the-art and Integrated models as mentioned in Section \ref{Experiments/Baseline}. The F1-score results for both baseline and proposed models on the test sets of three Vietnamese social media text datasets are shown in Table \ref{tab::Experiments/Result} and we obtain the following observations.

Among BERTology models, RoBERTa and mBERT, including \textit{cased} and \textit{uncased}, have the most unfavorable performance of almost tasks of the three benchmark datasets. Moreover, the results show that monolingual models such as PhoBERT and viBERT perform better than other BERTology models. Additionally, through the execution of parallel computations for words, the problem of vanishing gradients is minimized, and PhoBERT archives the highest results in nearly all the tasks. However, in general, BERTology baseline models still find it hard to handle the complexity of social media: imbalanced and noisy data, which leads to poor performance compared to the integrated GCN model.
    
Our baseline integrated models can also benefit from graph structure by combining GCN as the final prediction module. Compared to BERTology baseline models, the performance boost from contextualized pre-trained language models with the GCN module is significant. Moreover, the multilingual and monolingual models integrated with GCN perform massively better than others. This explains the significance of incorporating both the Contextualized and GCN models into the integrated models can be attributed to their complementary nature in addressing the limitations of each other.

Compared to baseline models, our approach ViCGCN adopts large-scale, monolingual Vietnamese language model PhoBERT. Our integrated model ViCGCN obtains the ability to compute the new features of a node as the weighted average of itself and its second-order neighbors. In the context of imbalanced and noisy datasets, such as UIT-ViCTSD, the proposed ViCGCN model has demonstrated significant performance improvements compared to other baseline models, making it a promising approach for social media mining tasks. Moreover, Our proposed model demonstrated superior performance to the current state-of-the-art Vietnamese model, achieving improvements of 6.21\%, 4.61\%, and 2.63\% on three benchmark datasets. These results demonstrate the efficacy and validity of ViCGCN for Vietnamese text classification. As a result, our method achieves the best performance among all the tasks on three benchmark datasets in terms of UIT-VSMEC, UIT-ViCTSD, and UIT-VSFC, respectively.

\subsection{Analysis and Discussion}

\subsubsection{Impact of graph convolutional networks}
\label{imapactGCN}

Although we can implicitly infer the effectiveness of graph convolutional networks from Table \ref{tab::Experiments/Result}, we would like to discuss more the contribution of graph convolutional networks in contextualized language models. Table \ref{Result/Graph/VSMEC/table}, Table \ref{/Result/Graph/ViCTSD/table} and Table \ref{/Result/Graph/VSFC/table} display the comparisons between with and without GCN on three benchmark datasets as we can find that contextualized language model integrated with GCN outperformed all of the corresponding single models, respectively. As mentioned in Section \ref{Experiments/Result}, Contextualized Language Models have not performed well on three benchmark datasets. Integrating GCN with the BERTology model massively enhances the performance, which leads to improvements of up to 8.00\%, 7.99\%, 5.84\%, and 7.99\% of RoBERTa, viBERT, vELECTRA, and $\text{PhoBERT}_{base}$, respectively, on three benchmark datasets, UIT-VSMEC, UIT-ViCTSD, and UIT-VSFC, respectively. The average length of three datasets in UIT-VSMEC, UIT-ViCTSD, and UIT-VSFC is approximately 14. Additionally, the short sequence lengths can construct more dense graphs that provide richer contextual information, which may explain better performance by combining contextualized language models with GCN. This further demonstrates that Graph Convolutional Networks are essential in improving text classification performance.


% \begin{table}[!hp] 
% \centering
% \caption{Model performance on UIT-VSMEC.}
% \label{Result/Graph/VSMEC/table}
% %\centerline{
% \resizebox{\linewidth}{!}{
% \begin{tabular}{l|cc|cc} 
% \hline
% \textbf{Tasks}            & \multicolumn{2}{c|}{\textbf{Seven labels}}                                           & \multicolumn{2}{c}{\textbf{Six labels}}                                              \\ 
% \hline
%                           & \textbf{wF1}                             & \textbf{mF1}                              & \textbf{wF1}                             & \textbf{mF1}                              \\ 
% \hline
% BERT (cased)              & 55.06                                    & 55.88                                     & 66.34                                    & 65.31                                     \\
% Bert-GCN (BERT cased)     & 74.22 ($\uparrow$19.16)                  & 73.51 ($\uparrow$17.63)                   & 78.25 ($\uparrow$11.91)                  & 77.02 ($\uparrow$11.71)                   \\ 
% \hline
% BERT (uncased)            & 55.98                                    & 56.40                                     & 58.55                                    & 56.65                                     \\
% Bert-GCN (BERT uncased)   & 74.18 ($\uparrow$18.20)                  & 73.29 ($\uparrow$16.89)                   & 80.54 ($\uparrow$21.99)                  & 78.31 ($\uparrow$21.66)                   \\ 
% \hline
% mBERT (mBERT cased)       & 61.23                                    & 59.57                                     & 66.72                                    & 66.72                                     \\
% mBERT-GCN (mBERT cased)   & 75.12 ($\uparrow$13.89)                  & 74.83 ($\uparrow$15.26)                   & 79.55 ($\uparrow$12.83)                  & 77.84 ($\uparrow$11.12)                   \\ 
% \hline
% mBERT (uncased)           & 58.31                                    & 57.03                                     & 67.13                                    & 67.70                                     \\
% mBERT-GCN (mBERT uncased) & 74.56 ($\uparrow$16.25)                  & 73.98 ($\uparrow$16.95)                   & 80.21 ($\uparrow$13.08)                  & 78.12 ($\uparrow$10.42)                   \\ 
% \hline
% RoBERTa                   & 56.51                                    & 56.22                                     & 62.11                                    & 64.64                                     \\
% RoBERTa-GCN               & 74.82 ($\uparrow$18.31)                  & 74.22 ($\uparrow$18.00)                   & 79.33 ($\uparrow$17.22)                  & 78.32 ($\uparrow$13.68)                   \\ 
% \hline
% viBERT                    & 67.55                                    & 65.32                                     & 69.95                                    & 69.08                                     \\
% viBERT-GCN                & 78.25 ($\uparrow$10.70)                  & 78.37 ($\uparrow$13.05)                   & 82.33 ($\uparrow$12.38)                  & 81.98 ($\uparrow$12.90)                   \\ 
% \hline
% PhoBERT                   & 71.86                                    & 69.58                                     & 75.19                                    & 74.32                                     \\
% \textbf{ViCGCN (Ours)}    & \textbf{80.24 ($\uparrow$\textbf{8.38)}} & \textbf{80.96 ($\uparrow$\textbf{11.38)}} & \textbf{84.91 ($\uparrow$\textbf{9.72)}} & \textbf{83.27 ($\uparrow$\textbf{8.95)}}  \\
% \hline
% \end{tabular}}
% \end{table}

% \begin{table}[!hpbt] 
% \centering
% \caption{Model performance on UIT-ViCTSD.}
% \label{/Result/Graph/ViCTSD/table}
% \resizebox{\linewidth}{!}{%
% \begin{tabular}{l|cc|cc} 
% \hline
% \textbf{Tasks}            & \multicolumn{2}{c|}{\textbf{Constructiveness}}                    & \multicolumn{2}{c}{\textbf{Toxicity}}                               \\ 
% \hline
%                           & \textbf{wF1}                    & \textbf{mF1}                    & \textbf{wF1}                    & \textbf{mF1}                      \\ 
% \hline
% BERT (cased)              & 79.34                           & 77.29                           & 87.45                           & 64.49                             \\
% Bert-GCN (BERT cased)     & 81.13 ($\uparrow$1.79)          & 79.72 ($\uparrow$2.43)          & 88.10 ($\uparrow$0.65)          & 64.52 ($\uparrow$0.03)            \\ 
% \hline
% BERT (uncased)            & 80.42                           & 78.90                           & 85.21                           & 63.28                             \\
% Bert-GCN (BERT uncased)   & 81.05 ($\uparrow$0.63)          & 79.6 ($\uparrow$0.70)           & 88.67 ($\uparrow$3.46)          & 64.96 ($\uparrow$1.68)            \\ 
% \hline
% mBERT (mBERT cased)       & 78.15                           & 76.93                           & 86.71                           & 64.36                             \\
% mBERT-GCN (mBERT cased)   & 82.15 ($\uparrow$4.00)          & 80.33 ($\uparrow$3.40)          & 89.13 ($\uparrow$0.46)          & 65.13 ($\uparrow$0.77)            \\ 
% \hline
% mBERT (uncased)           & 78.11                           & 76.63                           & 87.65                           & 64.96                             \\
% mBERT-GCN (mBERT uncased) & 82.88 ($\uparrow$4.77)          & 81.12 ($\uparrow$4.49)          & 89.93 ($\uparrow$2.28)          & 65.89 ($\uparrow$0.93)            \\ 
% \hline
% RoBERTa                   & 78.92                           & 77.71                           & 83.82                           & 61.73                             \\
% RoBERTa-GCN               & 83.47 ($\uparrow$4.55)          & 82.77 ($\uparrow$5.06)          & 86.55 ($\uparrow$2.73)          & 64.33 ($\uparrow$2.60)            \\ 
% \hline
% viBERT                    & 82.27                           & 80.12                           & 88.13                           & 64.35                             \\
% viBERT-GCN                & 86.12 ($\uparrow$3.85)          & 85.02 ($\uparrow$4.90)          & 91.27 ($\uparrow$3.14)          & 75.93 ($\uparrow$11.58)           \\ 
% \hline
% PhoBERT                   & 81.03                           & 79.53                           & 88.83                           & 65.61                             \\
% \textbf{ViCGCN (Ours)}    & \textbf{86.97 ($\uparrow$5.94)} & \textbf{85.81 ($\uparrow$6.28)} & \textbf{91.95 ($\uparrow$3.12)} & \textbf{76.29 ($\uparrow$10.68)}  \\
% \hline
% \end{tabular}}
% \end{table}

% \begin{table}[!hpbt] 
% \centering
% \caption{Model performance on UIT-VSFC.}
% \label{/Result/Graph/VSFC/table}
% \resizebox{\linewidth}{!}{%
% \begin{tabular}{l|cc|cc} 
% \hline
% \textbf{Tasks}            & \multicolumn{2}{c|}{\textbf{Sentiment-based}}                      & \multicolumn{2}{c}{\textbf{Topic-based}}                                  \\ 
% \hline
%                           & \textbf{wF1}                    & \textbf{mF1}                     & \textbf{wF1}                    & \textbf{mF1}                            \\ 
% \hline
% BERT (cased)              & 86.32                           & 72.38                            & 86.17                           & 73.62                                   \\
% Bert-GCN (BERT cased)     & 88.72 ($\uparrow$2.40)          & 76.23 ($\uparrow$3.85)           & 88.15 ($\uparrow$1.98)          & 76.25 ($\uparrow$2.63)                  \\ 
% \hline
% BERT (uncased)            & 85.51                           & 71.44                            & 85.98                           & 72.56                                   \\
% Bert-GCN (BERT uncased)   & 88.51 ($\uparrow$3.00)          & 75.99 ($\uparrow$4.55)           & 87.75 ($\uparrow$1.77)          & 76.06 ($\uparrow$3.50)                  \\ 
% \hline
% mBERT (mBERT cased)       & 91.21                           & 78.94                            & 88.06                           & 77.63                                   \\
% mBERT-GCN (mBERT cased)   & 91.89 ($\uparrow$0.68)          & 79.84 ($\uparrow$0.90)           & 89.73 ($\uparrow$1.98)          & 79.02 ($\uparrow$1.39)                  \\ 
% \hline
% mBERT (uncased)           & 89.52                           & 76.60                            & 87.12                           & 76.96                                   \\
% mBERT-GCN (mBERT uncased) & 91.72 ($\uparrow$2.20)          & 79.64 ($\uparrow$3.04)           & 88.89 ($\uparrow$1.77)          & 78.82 ($\uparrow$1.86)                  \\ 
% \hline
% RoBERTa                   & 90.11                           & 76.98                            & 87.03                           & 76.77                                   \\
% RoBERTa-GCN               & 91.12 ($\uparrow$1.01)          & 79.32 ($\uparrow$2.34)           & 90.12 ($\uparrow$3.09)          & 79.34 ($\uparrow$2.57)                  \\ 
% \hline
% viBERT                    & 89.77                           & 76.25                            & 87.43                           & 76.55                                   \\
% viBERT-GCN                & 93.27 ($\uparrow$3.50)          & 87.52 ($\uparrow$11.27)          & 92.11 ($\uparrow$4.68)          & 88.35 ($\uparrow$11.80)  \\ 
% \hline
% PhoBERT                   & 90.51                           & 76.47                            & 87.84                           & 76.98                                   \\
% \textbf{ViCGCN (Ours)}    & \textbf{94.81 ($\uparrow$4.30)} & \textbf{88.80 ($\uparrow$12.33)} & \textbf{93.81 ($\uparrow$5.97)} & \textbf{89.61 ($\uparrow$12.63)}        \\
% \hline
% \end{tabular}}
% \end{table}

\begin{table}[!ht]
\centering
\caption{Model performance on UIT-VSMEC.}
\label{Result/Graph/VSMEC/table}
\resizebox{\linewidth}{!}{%
\begin{tabular}{l|cc|cc} 
\hline
\textbf{Tasks}          & \multicolumn{2}{c|}{\textbf{Seven labels}}                        & \multicolumn{2}{c}{\textbf{Six labels}}                            \\ 
\hline
                        & \textbf{wF1}                    & \textbf{mF1}                    & \textbf{wF1}                    & \textbf{mF1}                     \\ 
\hline
mBERT (cased)           & 60.47                           & 59.48                           & 65.02                           & 62.65                            \\
mBERT-GCN (cased)       & 68.32 ($\uparrow$7.85)          & 64.32 ($\uparrow$4.84)          & 69.32 ($\uparrow$4.30)          & 66.18 ($\uparrow$3.53)           \\ 
\hline
mBERT (uncased)         & 60.17                           & 59.18                           & 64.93                           & 62.11                            \\
mBERT-GCN (uncased)     & 67.98 ($\uparrow$7.81)          & 64.11 ($\uparrow$4.93)          & 69.12 ($\uparrow$4.90)          & 65.89 ($\uparrow$3.78)           \\ 
\hline
RoBERTa                 & 58.17                           & 57.32                           & 63.32                           & 59.97                            \\
RoBERTa-GCN             & 66.17 ($\uparrow$8.00)          & 62.12 ($\uparrow$4.80)          & 67.12 ($\uparrow$3.80)          & 64.17 ($\uparrow$4.20)           \\ 
\hline
viBERT                  & 61.33                           & 60.28                           & 68.48                           & 62.09                            \\
viBERT-GCN              & 69.32 ($\uparrow$7.99)          & 78.37 ($\uparrow$4.84)          & 82.33 ($\uparrow$2.35)          & 81.98 ($\uparrow$4.59)           \\ 
\hline
vELECTRA                & 63.58                           & 61.38                           & 68.33                           & 63.12                            \\
vELETRA-GCN             & 69.42 ($\uparrow$5.84)          & 65.44 ($\uparrow$4.06)          & 70.95 ($\uparrow$2.62)          & 67.20 ($\uparrow$4.08)           \\ 
\hline
PhoBERT (base)          & 64.36                           & 61.41                           & 69.02                           & 64.12                            \\
\textbf{ViCGCN (base)}  & \textbf{69.32 ($\uparrow$7.99)} & \textbf{65.12 ($\uparrow$4.84)} & \textbf{70.83 ($\uparrow$2.35)} & \textbf{66.68 ($\uparrow$4.59)}  \\ 
\hline
PhoBERT (large)         & 65.12                           & 71.13                           & 63.23                           & 65.12                            \\
\textbf{ViCGCN (large)} & \textbf{71.33 ($\uparrow$6.21)} & \textbf{72.08 ($\uparrow$0.95)} & \textbf{67.82 ($\uparrow$4.59)} & \textbf{68.12 ($\uparrow$3.00)}  \\
\hline
\end{tabular}}
\end{table}

\begin{table}[H]
\centering
\caption{Model performance on UIT-ViCTSD.}
\label{/Result/Graph/ViCTSD/table}
\resizebox{\linewidth}{!}{%
\begin{tabular}{l|cc|cc} 
\hline
\textbf{Tasks}            & \multicolumn{2}{c|}{\textbf{Constructiveness}}                    & \multicolumn{2}{c}{\textbf{Toxicity}}                              \\ 
\hline
                          & \textbf{wF1}                    & \textbf{mF1}                    & \textbf{wF1}                    & \textbf{mF1}                     \\ 
\hline
mBERT (mBERT cased)       & 81.03                           & 79.55                           & 88.32                           & 65.63                            \\
mBERT-GCN (mBERT cased)   & 83.12 ($\uparrow$2.09)          & 82.88 ($\uparrow$3.33)          & 90.32 ($\uparrow$2.00)          & 69.42 ($\uparrow$3.79)           \\ 
\hline
mBERT (uncased)           & 80.89                           & 79.47                           & 87.60                           & 64.77                            \\
mBERT-GCN (mBERT uncased) & 82.32 ($\uparrow$1.43)          & 82.01 ($\uparrow$2.54)          & 89.15 ($\uparrow$1.55)          & 68.32 ($\uparrow$3.55)           \\ 
\hline
RoBERTa                   & 77.41                           & 75.62                           & 85.85                           & 59.71                            \\
RoBERTa-GCN               & 81.33 ($\uparrow$3.92)          & 80.96 ($\uparrow$5.34)          & 89.02 ($\uparrow$3.17)          & 64.32 ($\uparrow$4.61)           \\ 
\hline
viBERT                    & 81.62                           & 80.07                           & 89.14                           & 71.87                            \\
viBERT-GCN                & 84.32 ($\uparrow$2.70)          & 83.12 ($\uparrow$3.05)          & 91.12 ($\uparrow$1.98)          & 74.25 ($\uparrow$2.38)           \\ 
\hline
vELECTRA                  & 82.41                           & 80.82                           & 89.33                           & 72.02                            \\
vELETRA-GCN               & 84.62 ($\uparrow$2.21)          & 84.62 ($\uparrow$3.80)          & 91.88 ($\uparrow$2.55)          & 74.85 ($\uparrow$2.83)           \\ 
\hline
PhoBERT (base)            & 81.65                           & 80.24                           & 89.58                           & 72.12                            \\
\textbf{ViCGCN (base)}    & \textbf{85.64 ($\uparrow$3.99)} & \textbf{85.12 ($\uparrow$4.88)} & \textbf{92.22 ($\uparrow$2.64)} & \textbf{75.32 ($\uparrow$3.20)}  \\ 
\hline
PhoBERT large             & 82.07                           & 90.12                           & 81.27                           & 73.32                            \\
\textbf{ViCGCN (large)}   & \textbf{86.12 ($\uparrow$4.05)} & \textbf{93.11 ($\uparrow$2.99)} & \textbf{85.88 ($\uparrow$4.61)} & \textbf{76.12 ($\uparrow$2.80)}  \\
\hline
\end{tabular}}
\end{table}

\begin{table}[!ht]
\centering
\caption{Model performance on UIT-VSFC.}
\label{/Result/Graph/VSFC/table}
\resizebox{\linewidth}{!}{%
\begin{tabular}{l|cc|cc} 
\hline
\textbf{Tasks}            & \multicolumn{2}{c|}{\textbf{Sentiment-based}}                                                                                          & \multicolumn{2}{c}{\textbf{Topic-based}}                                             \\ 
\hline
                          & \textbf{wF1}                                                                                & \textbf{mF1}                             & \textbf{wF1}                             & \textbf{mF1}                              \\ 
\hline
mBERT (cased)             & 90.39                                                                                       & 77.15                                    & 87.32                                    & 77.93                                     \\
mBERT-GCN (cased)         & 92.12 ($\uparrow$1.73)                                                                      & 79.32 ($\uparrow$2.17)                   & 88.32 ($\uparrow$1.00)                   & 79.42 ($\uparrow$1.49)                    \\
mBERT (uncased)           & 89.95                                                                                       & 77.80                                    & 87.62                                    & 77.58                                     \\
mBERT-GCN (mBERT uncased) & 91.01 ($\uparrow$1.06)                                                                      & 79.02 ($\uparrow$1.22)                   & 88.07 ($\uparrow$0.45)                   & 79.02 ($\uparrow$1.44)                    \\ 
\hline
RoBERTa                   & 87.13                                                                                       & 75.52                                    & 86.77                                    & 75.30                                     \\
RoBERTa-GCN               & 90.12 ($\uparrow$2.99)                                                                      & 78.42 ($\uparrow$2.90)                   & 87.45 ($\uparrow$0.68)                   & 78.12 ($\uparrow$2.82)                    \\ 
\hline
viBERT                    & 91.29                                                                                       & 81.95                                    & 88.22                                    & 78.35                                     \\
viBERT-GCN                & 93.12 ($\uparrow$1.83)                                                                      & 82.47 ($\uparrow$0.52)                   & 89.42 ($\uparrow$1.20)                   & 79.63 ($\uparrow$1.28)                    \\ 
\hline
vELECTRA                  & 91.89                                                                                       & 82.01                                    & 88.12                                    & 78.12                                     \\
vELETRA-GCN               & 93.56 ($\uparrow$1.67)                                                                      & 83.12 $(\uparrow$1.11)                   & 89.95 ($\uparrow$1.83)                   & 80.02 ($\uparrow$1.90)                    \\ 
\hline
PhoBERT (base)            & 92.94                                                                                       & 82.15                                    & 88.29                                    & 78.54                                     \\
\textbf{ViCGCN (base)}    & \begin{tabular}[c]{@{}c@{}}\textbf{94.12~}($\uparrow$\textbf{}\textbf{1.18)}\end{tabular} & \textbf{83.67~}($\uparrow$\textbf{1.52)} & \textbf{90.12~}($\uparrow$\textbf{1.83)} & \textbf{80.11~}($\uparrow$\textbf{1.57)}  \\ 
\hline
PhoBERT (large)           & 93.24                                                                                       & 88.72                                    & 82.96                                    & 79.12                                     \\
\textbf{ViCGCN (large)}   & \textbf{94.83~}($\uparrow$\textbf{1.59)}                                                    & \textbf{91.02~}($\uparrow$\textbf{2.3)}  & \textbf{84.23~}($\uparrow$\textbf{1.27)} & \textbf{81.88~}($\uparrow$\textbf{2.76)}  \\
\hline
\end{tabular}}
\end{table}

\subsubsection{Impact of lambda ($\lambda$)}
\label{impactlamda}

According to Equation \ref{equa::lambda}, the hyperparameter $\lambda$ controls the trade-off between two objectives, ViCGCN and PhoBERT, respectively. The optimal value of $\lambda$ may vary depending on the task. Therefore, extensive experiments on the dev set were conducted to determine the optimal value of $\lambda$. Figure \ref{fig::Experiments/Lamda/UIT-VSFC} shows the performances of ViCGCN on three benchmark datasets in terms of UIT-VSMEC, UIT-ViCTSD, and UIT-VSFC with different $\lambda$. On all three benchmark datasets, the F1-score is consistently higher with a more enormous $\lambda$ value. Moreover, taking only ViCGCN ($\lambda = 1$) as the final training objective consistently achieves a better performance than considering only PhoBERT ($\lambda = 0$). Setting $\lambda$ to a value from 0.6 to 0.8 is more desirable and can make the model reach its best when $\lambda = 0.6$ on all datasets. These observations indicate that the linear interpolation of the prediction from ViCGCN and the prediction from PhoBERT with higher ViCGCN weight can improve the Vietnamese social media text classification performance. On the other hand, the PhoBERT module is also indispensable.
% \begin{figure}[H]
%     \centering
%     \subfigure[Seven labels task]{\includegraphics[width=0.49\textwidth]{ldVSMEC_7.pdf}}
%     \subfigure[Six labels task]{\includegraphics[width=0.49\textwidth]{ldVSMEC_6.pdf}}
%     \caption{F1-score of ViCGCN when varying $\lambda$ on UIT-VSMEC dev set.}
%     \label{fig::Experiments/Lamda/VSMEC}
% \end{figure}
% \begin{figure}[H]
%     \centering
%     \subfigure[Constructiveness task]{\includegraphics[width=0.49\textwidth]{ldViCTSD_Constructiveness.pdf}}
%     \subfigure[Toxicity task]{\includegraphics[width=0.49\textwidth]{ldViCTSD_toxic.pdf}}
%     \caption{F1-score of ViCGCN when varying $\lambda$ on UIT-ViCTSD dev set.}
%     \label{fig::Experiments/Lamda/ViCTSD}
% \end{figure}
\begin{figure}[!hpt]
    \centering

    \subfigure[Seven labels task]{\includegraphics[width=0.49\textwidth]{ldVSMEC_7.pdf}}
    \subfigure[Six labels task]{\includegraphics[width=0.49\textwidth]{ldVSMEC_6.pdf}}

     \subfigure[Constructiveness task]{\includegraphics[width=0.49\textwidth]{ldViCTSD_Constructiveness.pdf}}
    \subfigure[Toxicity task]{\includegraphics[width=0.49\textwidth]{ldViCTSD_toxic.pdf}}
    
    \subfigure[Sentiment-based task]{\includegraphics[width=0.49\textwidth]{ldVSFC_sentiment.pdf}}
    \subfigure[Topic-based task]{\includegraphics[width=0.49\textwidth]{ldVSFC_topic.pdf}}
    % \caption{F1-score of ViCGCN when varying $\lambda$ on UIT-VSFC dev set.}
    
    \caption{F1-score of ViCGCN when varying $\lambda$ on the dev set.}
    
    \label{fig::Experiments/Lamda/UIT-VSFC}
\end{figure}

\subsubsection{Comparison with Previous Studies}
\label{comparisonprestudies}
% \subsubsection{UIT-VSMEC}
We conducted a number of surveys to evaluate how well our suggested technique performed in comparison to earlier studies. On the UIT-VSMEC, UIT-ViCTSD, and UIT-VSFC datasets, our method fared better than in any prior research. To provide for fair comparisons, similar evaluation metrics from earlier studies are employed. For all datasets used in this study, we use the average macro F1-score (\%) and average weighted F1-score (\%). 

Our integrated model ViCGCN outperformed the best results of each previous study on the VSMEC dataset by achieving 80.24\% weighted F1-score and 80.96\% macro F1-score on task Seven labels, which improves by 10.18\% and 13.93\% compared to the best previous study. Additionally, our model obtains 84.91\% weighted F1-score on the Six labels task as shown in Table \ref{tab::Experiments/Comparison/VSMEC}, increased by 13.92\% in comparison to the highest previous ones. Furthermore, Table \ref{tab::Experiments/Comparison/ViCTSD} deputed that ViCGCN achieves the best results, with a macro F1-score of 85.81\% for UIT-ViCTSD Constructiveness task, and 76.29\% macro F1-score for UIT-ViCSTD Toxicity task,  increased by 16.89\%. By obtaining 88.80\% macro F1-score and 94.81\% weighted F1-score, 89.61\% macro F1-score, and 93.81\% weighted F1-score on task Sentiment-based and Topic-based, respectively, our integrated model ViCGCN surpassed every previous study's top result on the UIT-VSFC dataset as describes in Table \ref{tab::Experiments/Comparison/VSFC}. In addition, our proposed approach reached new state-of-the-art performances on three Vietnamese benchmark social media datasets, UIT-VSMEC, UIT-ViCTSD, and UIT-VSFC, respectively. As a result, the proposed approach ViCGCN is significantly suitable and efficient for dealing with Vietnamese text in general and Vietnamese social media text classification tasks in particular.

\begin{table}[!hpt]
\centering
\caption{The comparison with previous studies on UIT-VSMEC.} \label{tab::Experiments/Comparison/VSMEC}
\resizebox{\linewidth}{!}{
\begin{tabular}{l|cc|cc} 
\hline
\textbf{Tasks}                                  & \multicolumn{2}{c|}{\textbf{Seven labels }} & \multicolumn{2}{c}{\textbf{Six labels }}  \\ 
\hline
                                                & \textbf{wF1}   & \textbf{mF1}               & \textbf{wF1}   & \textbf{mF1}             \\ 
\hline
CNN + Word2Vec                                  & 59.74          & -                          & 66.34          & -                        \\
MLR + TF-IDF Vectorizer + Key-clause extraction & 64.40          & -                          & -              & -                        \\
GRU + CNN + BiLSTM + LSTM                       & 65.79          & -                          & 70.99          & -                        \\
PhoBERT                                         & -              & 65.44                      & -              & -                        \\
XLM-R + VnEmolex                                & 70.06          & 67.03                      & -              & -                        \\ 
\hline
\textbf{ViCGCN (base)}                          & \textbf{70.32} & \textbf{67.17}             & \textbf{71.02} & \textbf{67.48}           \\
\textbf{ViCGCN (large)}                         & \textbf{71.33} & \textbf{67.82}             & \textbf{72.08} & \textbf{68.12}           \\
\hline
\end{tabular}}
\end{table}


% \subsubsection{ViCTSD}
\begin{table}[!ht] 
\centering
\caption{The comparison with previous studies on UIT-ViCTSD.}
\label{tab::Experiments/Comparison/ViCTSD}
\begin{tabular}{l|cc|cc} 
\hline
\textbf{Tasks}          & \multicolumn{2}{c|}{\textbf{Constructiveness }} & \multicolumn{2}{c}{\textbf{Toxicity }}  \\ 
\hline
                        & \textbf{wF1}   & \textbf{mF1}                   & \textbf{wF1}   & \textbf{mF1}           \\ 
\hline
PhoBERT                 & -              & 78.59                          & -              & 59.40                  \\
viBERT4news             & -              & 84.15                          & -              & -                      \\ 
\hline
\textbf{ViCGCN (base)}  & \textbf{85.64} & \textbf{85.12}                 & \textbf{92.22} & \textbf{75.32}         \\
\textbf{ViCGCN (large)} & \textbf{86.12} & \textbf{85.88}                 & \textbf{93.11} & \textbf{76.12}         \\
\hline
\end{tabular}
\end{table}

% \subsubsection{VSFC}
\begin{table}[!ht] 
\centering
\caption{The comparison with previous studies on UIT-VSFC.}
\label{tab::Experiments/Comparison/VSFC}
\resizebox{\linewidth}{!}{
\begin{tabular}{l|cc|cc} 
\hline
\textbf{Tasks}             & \multicolumn{2}{c|}{\textbf{Sentiment-based }} & \multicolumn{2}{c}{\textbf{Topic-based }}  \\ 
\hline
                           & \textbf{wF1}   & \textbf{mF1}                  & \textbf{wF1}   & \textbf{mF1}              \\ 
\hline
Maximum Entropy            & 87.64          & -                             & 84.03          & -                         \\
BiLSTM +Word2Vec~          & 92.03          & -                             & 89.62          & -                         \\
LD + SVM ()                & 92.20          & -                             & -              & -                         \\
BERT + CNN + BiLSTM + LSTM & 92.79          & -                             & 89.38          & -                         \\
BERT + CNN + BiLSTM        & 92.13          & -                             & 89.70          & -                         \\
XLM-R + VnEmoLex           & 93.97          & 83.40                         & -              & -                         \\ 
\hline
\textbf{ViCGCN (base)}     & \textbf{94.12} & \textbf{83.67}                & \textbf{90.12} & \textbf{80.11}            \\
\textbf{ViCGCN (large)}    & \textbf{94.83} & \textbf{84.23}                & \textbf{91.02} & \textbf{81.88}            \\
\hline
\end{tabular}}
\end{table}

% SECTION Errors and analysis %
\subsubsection{Errors Analysis}
\label{erroranalysis}

We utilize the error analysis of ViCGCN, our top-performing model, to analyze the errors observed in our proposed model. Figure\ref{fig::Experiments/CfMatrix/VSMEC}, Figure \ref{fig::Experiments/CfMatrix/ViCTSD} and Figure \ref{fig::Experiments/CfMatrix/VSFC}, respectively, show the confusion matrices for our best model's predictions on the test set for UIT-VSMEC, UIT-ViCTSD, and UIT-VSFC.

\begin{figure}[!ht]
    \centering
    \subfigure[Seven labels task]{\includegraphics[width=0.49\textwidth]{cfVSMEC_7.pdf}}
    \subfigure[Six labels task]{\includegraphics[width=0.49\textwidth]{cfVSMEC_6.pdf}}
    \caption{Error analysis of our proposed approach for UIT-VSMEC dataset.}
    \label{fig::Experiments/CfMatrix/VSMEC}
\end{figure}
\begin{figure}[!ht]
    \centering
    \subfigure[Constructiveness task]{\includegraphics[width=0.49\textwidth]{cfViCTSD_contructive.pdf}}
    \subfigure[Toxicity task]{\includegraphics[width=0.49\textwidth]{cfViCTSD_toxic.pdf}}
    \caption{Error analysis of our proposed approach for UIT-ViCTSD dataset.}
    \label{fig::Experiments/CfMatrix/ViCTSD}
\end{figure}
\begin{figure}[!ht]
    \centering
    \subfigure[Sentiment-based task]{\includegraphics[width=0.49\textwidth]{cfVSFC_sentiment.pdf}}
    \subfigure[Topic-based task]{\includegraphics[width=0.49\textwidth]{cfVSFC_topic.pdf}}
    \caption{Error analysis of our proposed approach for UIT-VSFC dataset.}
    \label{fig::Experiments/CfMatrix/VSFC}
\end{figure}

As described in Section \ref{Proposed model}, by incorporating contextualized language models such as BERT into GCN, ViCGCN can better capture the context and meaning of words and phrases, which can lead to more accurate identification of critical nodes. However, ViCGCN may not be able to explain why those nodes are essential or why specific nodes were not influential in the decision-making process. This can make it difficult for researchers to address specific issues in our proposed approach. Table \ref{fig:erroranalysissampleonViCTSD}, Table \ref{fig:erroranalysissampleonVSMEC}, and Table \ref{fig:erroranalysissampleonVSFC} contain a few illustrations of prediction errors. The results show that misclassifications were primarily due to the use of sarcasm, irony, and figurative language in social media comments. Furthermore, some misclassifications were due to the presence of multiple topics in a single comment, making it challenging to identify the primary intention. Additionally, ambiguity in identifying the labels of the datasets also leads to misclassifying of our proposed approach ViCGCN.

% \subsubsection{VSMEC dataset} \label{Errors/VSMEC}


\begin{table}[H]
\centering
\caption{Several examples of classification error on UIT-VSMEC dataset.}\label{fig:erroranalysissampleonVSMEC}
\resizebox{\linewidth}{!}{%
\begin{tblr}{
  row{1} = {c},
  cell{2}{2} = {c},
  cell{2}{3} = {c},
  cell{3}{2} = {c},
  cell{3}{3} = {c},
  hline{1-2,4} = {-}{},
}
\textbf{Comment}                                                       & \textbf{True Label} & \textbf{Predicted Label} \\
{mấy ai được như vậy\\(\textbf{English:} not many people can do that)} & other               & surprise               \\
{kinh khủng thật\\(\textbf{English:} it's terrible)}                   & fear                & sadness                
\end{tblr}}
\end{table}

% \subsubsection{ViCTSD dataset} \label{Errors/ViCTSD}

\begin{table}[!ht]
\centering
\caption{Several examples of classification error on UIT-ViCTSD dataset.}\label{fig:erroranalysissampleonViCTSD}
\resizebox{\linewidth}{!}{%
\begin{tblr}{
  row{1} = {c},
  cell{2}{2} = {c},
  cell{2}{3} = {c},
  cell{3}{2} = {c},
  cell{3}{3} = {c},
  hline{1-2,4} = {-}{},
}
\textbf{Comment}                                                                                                                                           & \textbf{True Label} & \textbf{Predicted Label} \\
{Người ăn không hết kẻ lần không ra\\(\textbf{English:} This man has much to eat but that \\
man finds no small piece.)}                                       & non\_constructive   & constructive           \\
{người trẻ còn sức khoẻ k lo làm ăn đi ăn trộm\\(\textbf{English:} Young people who are still healthy \\ don't worry about doing business but go to steal)} & non\_toxic          & toxic                  
\end{tblr}}
\end{table}


% % \subsubsection{UIT-VSFC dataset} \label{Errors/VSFC}

\begin{table}[!ht]
\centering
\caption{Several examples of classification error on UIT-VSFC dataset.}\label{fig:erroranalysissampleonVSFC}
\resizebox{\linewidth}{!}{%
% \centerline{
\begin{tblr}{
  row{1} = {c},
  cell{2}{2} = {c},
  cell{2}{3} = {c},
  cell{3}{2} = {c},
  cell{3}{3} = {c},
  hline{1-2,4} = {-}{},
}
\textbf{Comment}                                                                                                                                          & \textbf{True Label} & \textbf{Predicted Label} \\
{ví dụ phù hợp với nội dung kiến thức , hướng dẫn chi tiết\\(\textbf{English:}~Examples are consistent with content knowledge, \\ detailed instructions)} & neural          & positive           \\
{đảm bảo chất lượng tốt\\(\textbf{English:}~Good quality guarantee)}                                                                                               & others          & facility topic     
\end{tblr}}
% }
\end{table}
%%


\subsubsection{Ablation Study}
\label{ablationstudy}

\begin{table}[H]
\centering
\caption{Ablation test on our proposed approach. w/o GCN and w/o PhoBERT denoted the result of the ablation GCN and the result of the ablation PhoBERT, respectively}
\label{tab::Ablation}
\resizebox{\linewidth}{!}{%
% \centerline{
\begin{tabular}{l|cc|cc|cc|cc|cc|cc} 
\hline
\textbf{Datasets} & \multicolumn{4}{c|}{\textbf{VSMEC}}                                                                          & \multicolumn{4}{c|}{\textbf{ViCTSD}}                                                                         & \multicolumn{4}{c}{\textbf{VSFC}}                                                                \\ 
\hline
\textbf{Tasks}    & \multicolumn{2}{c|}{\textbf{Seven labels}}            & \multicolumn{2}{c|}{\textbf{Six labels}}             & \multicolumn{2}{c|}{\textbf{Constructiveness}}       & \multicolumn{2}{c|}{\textbf{Toxicity}}                & \multicolumn{2}{c|}{\textbf{Sentiment-based}}        & \multicolumn{2}{c}{\textbf{Topic-based}}  \\ 
\hline
                  & \multicolumn{1}{c|}{\textbf{wF1}} & \textbf{mF1}      & \multicolumn{1}{c|}{\textbf{wF1}} & \textbf{mF1}     & \multicolumn{1}{c|}{\textbf{wF1}} & \textbf{mF1}     & \multicolumn{1}{c|}{\textbf{wF1}} & \textbf{mF1}      & \multicolumn{1}{c|}{\textbf{wF1}} & \textbf{mF1}     & \textbf{wF1}     & \textbf{mF1}           \\ 
\hline
\multicolumn{13}{c}{\textbf{ViCGCN}}                                                                                                                                                                                                                                                                                                               \\ 
\hline
Performance       & \textbf{71.33}                    & \textbf{67.82}    & \textbf{72.08}                    & \textbf{68.12}   & \textbf{86.12}                    & \textbf{85.88}   & \textbf{93.11}                    & \textbf{76.12}    & \textbf{94.83}                    & \textbf{84.23}   & \textbf{91.02}   & \textbf{81.88}         \\ 
\hline
\multicolumn{13}{c}{\textbf{w/o GCN}}                                                                                                                                                                                                                                                                                                              \\ 
\hline
Performance       & 65.12                             & 63.23             & 71.13                             & 65.12            & 81.03                             & 79.53            & 90.12                             & 73.32             & 93.24                             & 82.96            & 88.72            & 79.12                  \\
Decrease          & $\downarrow$6.21                  & $\downarrow$4.59  & $\downarrow$0.95                  & $\downarrow$3.00 & $\downarrow$5.09                  & $\downarrow$6.35 & $\downarrow$2.99                  & $\downarrow$2.80  & $\downarrow$1.59                  & $\downarrow$1.27 & $\downarrow$2.30 & $\downarrow$2.76       \\ 
\hline
\multicolumn{13}{c}{\textbf{w/o PhoBERT}}                                                                                                                                                                                                                                                                                                          \\ 
\hline
Performance       & 52.32                             & 51.32             & 61.34                             & 58.42            & 79.63                             & 78.37            & 87.63                             & 64.32             & 88.32                             & 75.32            & 85.36            & 75.21                  \\
Decrease          & $\downarrow$19.01                 & $\downarrow$16.50 & $\downarrow$10.74                 & $\downarrow$9.70 & $\downarrow$6.49                  & $\downarrow$7.51 & $\downarrow$5.48                  & $\downarrow$11.80 & $\downarrow$6.51                  & $\downarrow$8.91 & $\downarrow$5.66 & $\downarrow$6.67       \\
\hline
\end{tabular}}
\end{table}

Our proposed method is considerably more effective than most current techniques for classifying text on social media. Ablation experiments were carried out on the proposed approach to prove the effectiveness of these two modules, PhoBERT and GCN. Table \ref{tab::Ablation} shows the ablation experiment results of the text classification module. For the model with GCN ablation, the experimental results are inferior to the model without ablation. While results of the \textit{w/o PhoBERT} model are not as good as those of the model with the contextualized pre-trained language model. The results of the ablation experiments demonstrate the effectiveness of the proposed importance of each module in general, as well as the combination of our proposed approach in particular. Our proposed approach, especially contextualized language models Integrated with graph neural networks, yield promising outcome for improving performance in further study. As a result, we conclude that all proposed modules are crucial in text classification on social media.
\section{Limitations and Open Questions}
\label{sec:limitations}
Though we have proposed two effective non-``detect-then-describe'' methods for 3D dense captioning, the captions do not have much diversity because of the limited text annotations, beam search, and self-critical sequence training with the CiDEr reward.
% 
We believe that multi-modal pre-training on 3D vision-language tasks with more training data and the utilization of \textbf{L}arge \textbf{L}anguage \textbf{M}odels(LLM) trained on large corpus would increase the diversity of the generated captions.
% 
Additionally, other reward functions designed for 3D dense captioning will increase the diversity among object descriptions in the same scene.
% 
We will leave these topics for future study.


\section{Conclusions}
\label{sec:conclusion}
%
\whatsnew{
In this work, we decouple the caption generation from caption generation, and propose a set of two transformer-based approaches, namely Vote2Cap-DETR and Vote2Cap-DETR++, for 3D dense captioning.
%
Comparing with the sophisticated and explicit relation modules in conventional ``detect-then-describe'' pipelines, our proposed methods efficiently capture the object-object and object-scene relation through the attention mechanism.
%
The preliminary model, Vote2Cap-DETR, decouples the decoding process to generate captions and box estimations in parallel.
% 
We also propose vote queries for fast convergence, and develop a novel lightweight query-driven caption head for informative caption generation.
% 
In the advanced model, Vote2Cap-DETR++, we further decouple the queries to capture task-specific features for object localization and description generation.
% 
Additionally, we introduce an iterative spatial refinement strategy for vote queries, and insert 3D spatial information for more accurate captions.
%
Extensive experiments on two widely used datasets validate that both the proposed methods surpass prior ``detect-then-describe'' pipelines by a large margin.
}

% \bibliographystyle{model5-names}\biboptions{authoryear}
% \bibliography{cas-refs.bib}


%% The Appendices part is started with the command \appendix;
%% appendix sections are then done as normal sections
%% \appendix

%% \section{}
%% \label{}

%% For citations use: 
%%       \citet{<label>} ==> Jones et al. [21]
%%       \citep{<label>} ==> [21]
%%

%% If you have bibdatabase file and want bibtex to generate the
%% bibitems, please use
%%
\bibliographystyle{elsarticle-num-names} 
\bibliography{cas-refs.bib}

%% else use the following coding to input the bibitems directly in the
%% TeX file.


\end{document}

