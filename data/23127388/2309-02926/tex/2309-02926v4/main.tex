%%
%% This is file `sample-sigconf.tex',
%% generated with the docstrip utility.
%%
%% The original source files were:
%%
%% samples.dtx  (with options: `sigconf')
%% 
%% IMPORTANT NOTICE:
%% 
%% For the copyright see the source file.
%% 
%% Any modified versions of this file must be renamed
%% with new filenames distinct from sample-sigconf.tex.
%% 
%% For distribution of the original source see the terms
%% for copying and modification in the file samples.dtx.
%% 
%% This generated file may be distributed as long as the
%% original source files, as listed above, are part of the
%% same distribution. (The sources need not necessarily be
%% in the same archive or directory.)
%%
%%
%% Commands for TeXCount
%TC:macro \cite [option:text,text]
%TC:macro \citep [option:text,text]
%TC:macro \citet [option:text,text]
%TC:envir table 0 1
%TC:envir table* 0 1
%TC:envir tabular [ignore] word
%TC:envir displaymath 0 word
%TC:envir math 0 word
%TC:envir comment 0 0
%%
%%
%% The first command in your LaTeX source must be the \documentclass
%% command.
%%
%% For submission and review of your manuscript please change the
%% command to \documentclass[manuscript, screen, review]{acmart}.
%%
%% When submitting camera ready or to TAPS, please change the command
%% to \documentclass[sigconf]{acmart} or whichever template is required
%% for your publication.
%%
%%
\documentclass[sigconf]{acmart}
% \documentclass[sigconf, authorversion, nonacm]{acmart}

% to be able to draw some self-contained figs
\usepackage{amsmath}

% inlined bib file
%\usepackage{filecontents}

\usepackage{verbatim} % for \begin{comment} command 
\usepackage{booktabs} % for toprule midrule bottomrule command...
%\usepackage{comment}
\usepackage{enumitem}
\usepackage{multicol}
\usepackage{listings}
\usepackage{url}
\def\UrlBreaks{\do\A\do\B\do\C\do\D\do\E\do\F\do\G\do\H\do\I\do\J
\do\K\do\L\do\M\do\N\do\O\do\P\do\Q\do\R\do\S\do\T\do\U\do\V
\do\W\do\X\do\Y\do\Z\do\[\do\\\do\]\do\^\do\_\do\`\do\a\do\b
\do\c\do\d\do\e\do\f\do\g\do\h\do\i\do\j\do\k\do\l\do\m\do\n
\do\o\do\p\do\q\do\r\do\s\do\t\do\u\do\v\do\w\do\x\do\y\do\z
\do\.\do\@\do\\\do\/\do\!\do\_\do\|\do\;\do\>\do\]\do\)\do\,
\do\?\do\'\do+\do\=\do\#}
%\usepackage{soul}
\usepackage{threeparttable}
\usepackage{tikz}
\usepackage{xcolor}
\usepackage{xspace}

% \usepackage{hyperref}

\usepackage{multirow}
\usepackage{bbding}

% clean mode, revision mode, and new mode
% use \rlegend to see the usage.
% \rchange{old}{new}
% \usepackage[clean]{revdiff} 

\usepackage{siunitx}
%\usepackage{ulem}
\usepackage{colortbl}
\usepackage{pifont}
\usepackage{seqsplit}
\usepackage[ruled,vlined,linesnumbered]{algorithm2e}

\usepackage{float}

\usepackage{lipsum} 
%\SetAlFnt{\small}

\definecolor{codegray}{rgb}{0.5,0.5,0.5}
\definecolor{codepurple}{rgb}{0.58,0,0.82}
\definecolor{backcolour}{rgb}{0.95,0.95,0.92}
\definecolor{keyword_color}{RGB}{176,1,75}
\definecolor{id_color}{RGB}{52,5,255}
\definecolor{comment_color}{RGB}{64,128,128}

\definecolor{githubred}{RGB}{255,235,233}
\definecolor{githubgreen}{RGB}{230,255,236}
\definecolor{hpwgray}{RGB}{239,241,243}
\definecolor{codegreen}{RGB}{0,115,0}


\definecolor{hgblue}{RGB}{138,200,224}
\definecolor{hgred}{RGB}{245,138,143}

\definecolor{gold}{RGB}{221, 196, 65}
\definecolor{silver}{RGB}{215, 215, 215}
\definecolor{bronze}{RGB}{126, 66, 5}

\definecolor{mygray}{gray}{.9}

\lstdefinestyle{mystyle}{
	backgroundcolor=\color{backcolour}, 
	commentstyle=\color{codegreen},
	keywordstyle=\color{keyword_color}\bfseries,
	numberstyle=\tiny\color{codegray},
	stringstyle=\color{codepurple},
	% keywordstyle=\color{blue!80!black},
	identifierstyle=\color{id_color},
	basicstyle=\ttfamily\footnotesize,
	breakatwhitespace=false,         
	breaklines=true,                 
	captionpos=b,                    
	keepspaces=true,                 
	numbers=left,                    
	numbersep=5pt,                  
	showspaces=false,                
	showstringspaces=false,
	showtabs=false,                  
	tabsize=2,
	xleftmargin=1.5em,
	xrightmargin=0.5em, 
	aboveskip=1em,
	escapeinside={\%*}{*)}
}
\lstset{style=mystyle}



\def\X#1{\ding{\numexpr181+#1}}

\newtheorem{mydef}{Definition}

\def\BibTeX{{\rm B\kern-.05em{\sc i\kern-.025em b}\kern-.08em
		T\kern-.1667em\lower.7ex\hbox{E}\kern-.125emX}}
	
\newcommand\old[1]{\ignorespaces} % ignore spaces and texts.
\newcommand\re[1]{\textcolor{black}{#1}}
\newcommand\red[1]{\textcolor{red}{#1}}
% \newcommand\tool{\textsc{LLMVulnX}\xspace}
\newcommand\tool{\textsc{LLMSmith}\xspace}
\newcommand\dzz[1]{\textcolor{brown}{ZZ: #1}}
\newcommand\guozhu[1]{\textcolor{purple}{GZ: #1}}
\newcommand\lyk[1]{\textcolor{orange}{lyk: #1}}
\newcommand\todo[1]{\textcolor{black}{#1}}
\newcommand\lt[1]{\textcolor{red}{LT: #1}}
\newcommand\kai[1]{\textcolor{blue}{kai: #1}}



\newcommand\etal{\textit{et al.}\xspace}
\newcommand\eg{\textit{e.g.}\xspace}
\newcommand\Eg{\textit{E.g.}\xspace}
\newcommand\ie{\textit{i.e.}\xspace}

\renewcommand{\texttt}[1]{\textsf{#1}}


% Draw the circle
\newcommand*\emptycirc[1][1ex]{\tikz\draw (0,0) circle (#1);} 
\newcommand*\halfcirc[1][1ex]{%
	\begin{tikzpicture}
	\draw[fill] (0,0)-- (90:#1) arc (90:270:#1) -- cycle ;
	\draw (0,0) circle (#1);
	\end{tikzpicture}}
\newcommand*\fullcirc[1][1ex]{\tikz\fill (0,0) circle (#1);} 

% \settopmatter{printfolios=true}
\pagestyle{plain}
\pagenumbering{arabic}

%%
%% \BibTeX command to typeset BibTeX logo in the docs
\AtBeginDocument{%
  \providecommand\BibTeX{{%
    Bib\TeX}}}

%% ------------------ use package -----------------------
\usepackage{enumitem}
\usepackage[ruled,vlined,linesnumbered]{algorithm2e}
% \usepackage{algorithmic}
% \usepackage{algorithm}
% \renewcommand{\algorithmiccomment}[1]{\hfill $\triangleright$ #1}
%% ------------------------------------------------------

%% Rights management information.  This information is sent to you
%% when you complete the rights form.  These commands have SAMPLE
%% values in them; it is your responsibility as an author to replace
%% the commands and values with those provided to you when you
%% complete the rights form.
% \setcopyright{acmcopyright}
% \copyrightyear{2024}
% \acmYear{2024}
% \acmDOI{XXXXXXX.XXXXXXX}

% \copyrightyear{2024}
% \acmYear{2024}
% \setcopyright{acmcopyright}
% \acmConference[CCS '24] {Proceedings of the 2024 ACM SIGSAC Conference on Computer and Communications Security}{October 14-18, 2024}{Salt Lake City, USA.}
% \acmBooktitle{Proceedings of the 2024 ACM SIGSAC Conference on Computer and Communications Security (CCS '24), October 14-18, 2024, Salt Lake City, USA.}
% \acmPrice{15.00}
% \acmISBN{xxxx}
% \acmDOI{10.1145/3658644.3690338}

\copyrightyear{2024} 
\acmYear{2024}
\setcopyright{rightsretained}
\acmConference[CCS '24]{Proceedings of the 2024 ACM SIGSAC Conference on Computer and Communications Security}{October 14--18, 2024}{Salt Lake City, UT, USA}
\acmBooktitle{Proceedings of the 2024 ACM SIGSAC Conference on Computer and Communications Security (CCS '24), October 14--18, 2024, Salt Lake City, UT, USA}
\acmDOI{10.1145/3658644.3690338}
\acmISBN{979-8-4007-0636-3/24/10}

%% These commands are for a PROCEEDINGS abstract or paper.
%\acmConference[Conference acronym 'XX]{Make sure to enter the correct
%  conference title from your rights confirmation emai}{June 03--05,
 % 2018}{Woodstock, NY}
%%
%%  Uncomment \acmBooktitle if the title of the proceedings is different
%%  from ``Proceedings of ...''!
%%
%%\acmBooktitle{Woodstock '18: ACM Symposium on Neural Gaze Detection,
%%  June 03--05, 2018, Woodstock, NY}
%\acmPrice{15.00}
%\acmISBN{978-1-4503-XXXX-X/18/06}


%%
%% Submission ID.
%% Use this when submitting an article to a sponsored event. You'll
%% receive a unique submission ID from the organizers
%% of the event, and this ID should be used as the parameter to this command.
%%\acmSubmissionID{123-A56-BU3}

%%
%% For managing citations, it is recommended to use bibliography
%% files in BibTeX format.
%%
%% You can then either use BibTeX with the ACM-Reference-Format style,
%% or BibLaTeX with the acmnumeric or acmauthoryear sytles, that include
%% support for advanced citation of software artefact from the
%% biblatex-software package, also separately available on CTAN.
%%
%% Look at the sample-*-biblatex.tex files for templates showcasing
%% the biblatex styles.
%%

%%
%% The majority of ACM publications use numbered citations and
%% references.  The command \citestyle{authoryear} switches to the
%% "author year" style.
%%
%% If you are preparing content for an event
%% sponsored by ACM SIGGRAPH, you must use the "author year" style of
%% citations and references.
%% Uncommenting
%% the next command will enable that style.
%%\citestyle{acmauthoryear}


%%Remove copyright
\settopmatter{printacmref=true}
% \setcopyright{none}
% \renewcommand\footnotetextcopyrightpermission[1]{}
% \pagestyle{plain}
\settopmatter{printfolios=false}
%%
%% end of the preamble, start of the body of the document source.
\begin{document}

%%
%% The "title" command has an optional parameter,
%% allowing the author to define a "short title" to be used in page headers.
\title{Demystifying RCE Vulnerabilities in LLM-Integrated Apps}

%%
%% The "author" command and its associated commands are used to define
%% the authors and their affiliations.
%% Of note is the shared affiliation of the first two authors, and the
%% "authornote" and "authornotemark" commands
%% used to denote shared contribution to the research.
\author{Tong Liu}
% \orcid{0009-0004-5804-6551}
% \affiliation{%
%     \institution{Chinese Academy of Sciences}
%     \department{Institute of Information Engineering}
%     }
% \affiliation{%
%     \institution{University of Chinese Academy of Sciences}
%     \department{School of Cyber Security}
%     \city{Beijing} 
%     \country{China}}
\affiliation{%
    \institution{IIE, CAS$^{\dag}$}
    \institution{School of Cyber Security, UCAS$^{\ddag}$}
    \city{Beijing} 
    \country{China}}
\email{liutong@iie.ac.cn}

\author{Zizhuang Deng}
%\orcid{xxxx}
\affiliation{%
    \institution{School of Cyber Science and Technology, Shandong University}
    \city{Qingdao}
    \country{China}}
\email{dengzz@sdu.edu.cn}

\author{Guozhu Meng$^*$}
% \orcid{xxx}
\affiliation{%
    \institution{IIE, CAS$^{\dag}$}
    \institution{School of Cyber Security, UCAS$^{\ddag}$}
    \city{Beijing} 
    \country{China}}
\email{mengguozhu@iie.ac.cn}

\author{Yuekang Li}
% \orcid{xxx}
\affiliation{%
    \institution{University of New South Wales}
    \city{Sydney} 
    \country{Australia}}
\email{yuekang.li@unsw.edu.au}

\author{Kai Chen}
% \orcid{xxx}
\affiliation{%
    \institution{IIE, CAS$^{\dag}$}
    \institution{School of Cyber Security, UCAS$^{\ddag}$}
    \city{Beijing} 
    \country{China}}
\email{chenkai@iie.ac.cn}

\thanks{$*$ Corresponding authors}
\thanks{${\dag}$ Institute of Information Engineering,  Chinese Academy of Sciences}
\thanks{${\ddag}$ University of Chinese Academy of Sciences}

%%
%% By default, the full list of authors will be used in the page
%% headers. Often, this list is too long, and will overlap
%% other information printed in the page headers. This command allows
%% the author to define a more concise list
%% of authors' names for this purpose.

%%
%% The abstract is a short summary of the work to be presented in the
%% article.
%-------------------------------------------------------------------------------
\begin{abstract}
%-------------------------------------------------------------------------------

Large Language Models (LLMs) show promise in transforming software development, with a growing interest in integrating them into more intelligent apps. Frameworks like LangChain aid LLM-integrated app development, offering code execution utility/APIs for custom actions. However, these capabilities theoretically introduce Remote Code Execution (RCE) vulnerabilities, enabling remote code execution through prompt injections. No prior research systematically investigates these frameworks' RCE vulnerabilities or their impact on applications and exploitation consequences. Therefore, there is a huge research gap in this field.



In this study, we propose \tool to detect, validate and exploit the RCE vulnerabilities in LLM-integrated frameworks and apps. To achieve this goal, we develop two novel techniques, including 1) a lightweight static analysis to construct call chains to identify RCE vulnerabilities in frameworks; 2) a systematical prompt-based exploitation method to verify and exploit the found vulnerabilities in LLM-integrated apps. This technique involves various strategies to control LLM outputs, trigger RCE vulnerabilities and launch subsequent attacks. Our research has uncovered a total of 20 vulnerabilities in 11 LLM-integrated frameworks, comprising 19 RCE vulnerabilities and 1 arbitrary file read/write vulnerability. Of these, 17 have been confirmed by the framework developers, with 13 vulnerabilities being assigned CVE IDs, 6 of which have a CVSS score of 9.8, and we were also awarded a bug bounty of \$1350. For the 51 apps potentially affected by RCE, we successfully executed attacks on 17 apps, 16 of which are vulnerable to RCE and 1 to SQL injection. Furthermore, we conduct a comprehensive analysis of these vulnerabilities and construct practical attacks to demonstrate the hazards in reality, \eg, app output hijacking, user data leakage, even the potential to take full control of systems. Last, we propose several mitigation measures for both framework and app developers to counteract such attacks. 
\end{abstract}

%%
%% The code below is generated by the tool at http://dl.acm.org/ccs.cfm.
%% Please copy and paste the code instead of the example below.
%%

\begin{CCSXML}
<ccs2012>
   <concept>
       <concept_id>10002978.10003022</concept_id>
       <concept_desc>Security and privacy~Software and application security</concept_desc>
       <concept_significance>500</concept_significance>
       </concept>
 </ccs2012>
\end{CCSXML}

\ccsdesc[500]{Security and privacy~Software and application security}

\keywords{Large Language Model, LLM-integrated Applications, RCE}

%%
%% This command processes the author and affiliation and title
%% information and builds the first part of the formatted document.
\maketitle

%----------------------------------INTRO-------------------------------------------
\section{Introduction}

Text-Attributed Graphs (TAGs) are a type of graph that have textual data as node attributes. 
These types of graphs are prevalent in the real world, such as in citation networks \cite{hu2020open} where the node attribute is the paper's abstract. TAGs have diverse potential applications, including paper classification \cite{chien2021node} and user profiling\cite{kim2020multimodal}. 
However, studying TAGs presents a significant challenge: how to model the intricate interplay between graph structures and textual features. 
This issue has been extensively explored in several fields, including natural language processing, information extraction, and graph representation learning. 

% Text-Attributed Graphs (TAGs) are a type of graph that is widely present in the real world. 
% In practical applications, many node features can be composed of text. For example, in citation networks, the node feature is the abstract of a paper, and in social networks, the node feature is the user's profile. 
% TAGs have broad potential application values, such as paper classification and user identification. 
% Modeling TAGs involves techniques from multiple fields, including information extraction, natural language processing, and graph representation learning, making it a hot academic topic currently.

An idealized approach involves combining pre-trained language models (PLMs) \cite{he2020deberta,liu2019roberta} with graph neural networks and jointly training them \cite{zhao2022learning,mavromatis2023train}. Nevertheless, this method requires fine-tuning the PLMs, which demands substantial computational resources. Additionally, trained models are hard to be reused in other tasks because finetuning PLM may bring catastrophic forgetting\cite{chen2020recall}. 

Therefore, a more commonly used and efficient approach is the two-stage process \cite{yang2021bert,zhang2022stance,malhotra2020classification}: (1) utilizing pre-trained language models (PLMs) for unsupervised modeling of the nodes' textual features. 
(2) supervised learning using Graph Neural Networks (GNNs). 
Compared to joint training of PLMs and GNNs, this approach offers several advantages in practical applications. 
For example, it can be combined with numerous GNN frameworks or PLMs, and this approach does not require fine-tuning PLMs for every downstream task.
However, PLMs are unable to fully leverage the wealth of information contained in the graph structure, which represents significant information. 
To overcome these limitations, some works propose self-supervised fine-tuning PLMs using graph information to extract graph-aware node features \cite{chien2021node}. Such methods have achieved significant success across various benchmark datasets\cite{hu2020open}. 
% Unsupervised modeling of nodes' textual features by language models (LM) and subsequent supervised learning of the graph feature by Graph Neural Networks (GNNs) is a classical and effective approach for processing TAGs. 
% However, the generated node representation is untrainable in downstream tasks, a unsuitable representation may affect the performance of subsequent GNNs learning. 
% To address limitations, many works merged recently, which investigate how to better utilize pre-trained language models in TAGs modeling. 
% A method is joint PLMs with GNNs by knowledge distillation. 
% and self-supervised fine-tuning PLMs to adapt graph data.   
% First, PLMs are fine-tuned by self-supervised tasks related to graphs, enabling them to capture and comprehend graph information. Then, the fine-tuned PLM is used to generate node representations.
% This approach has achieved significant results in numerous public datasets.


% However, these SSL-based node feature extraction methods suffer from the few-shot challenge. are based on graphs with over 100,000 nodes. 
% This means that during the self-supervised training phase, there are enough samples, and downstream task training samples are also abundant. 
% For example, in Ogbn-arxiv, there are over 70,000 training samples (60\%). 
% However, this situation poses a significant gap from the real world. 
% Firstly, training labels are often expensive, and secondly, there exist many small graphs in the real world.  

However, both self-supervised methods and using language models directly to process TAG suffer from a fundamental drawback. They cannot incorporate downstream task information, which results in identical representations being generated for all downstream tasks. This is evidently counterintuitive as the required information may vary for different tasks. For example, height is useful information in predicting a user's weight but fails to accurately predict age. This issue can be resolved by utilizing task-specific prompts combined with language models \cite{petroni2019language} to extract downstream task-related representations. For example, suppose we have a paper's abstract $\{\mathbf{Abstract}\}$ in a citation network, and the task is to classify the subject of the paper. We can add some prompts to a node's sentence:
$
    \{This, is, a, paper, of, [\mathbf{mask}], subject, its, abstract, is,:, \mathbf{Abstract}\}
$. And then use the embedding corresponding to the [mask] token generated by PLMs as the node feature. Yet this approach fails to effectively integrate graph information. 

To better integrate task-specific information and knowledge within graph structure, this paper proposes a novel framework called G-Prompt. G-Prompt combines a graph adapter and task-specific prompts to extract node features. Specifically, G-Prompt contains a graph adapter that helps PLMs become aware of graph structures. This graph adapter is self-supervised and trained by fill-mask tasks on specific TAGs. G-Prompt then incorporates task-specific prompts to obtain interpretable node representations for downstream tasks.



% However, we observe the SSL-based methods are in the small-sample scenario and found that: \\
% 1. The representations generated by large-scale language models perform similarly to word2vec in small-sample situations. This is clearly counterintuitive, as numerous experiments have shown that pre-trained language models can learn rich knowledge from massive text. \\
% 2. The representation of entire BERT models finetuned on graph self-supervised tasks such as GIANT performs similarly to the frozen language model's representation through GAE pre-training in extremely small sample sizes. However, overall, it outperforms graph-free representations. \\
% 3. Since using PLM-generated representations did not yield good results, we experimented with RoBERTa-based representations with task prompts, which performed the best in small-sample scenarios.

% This implies that both Graph-aware and Task-aware representations are crucial for node representation. 
% However, current methods \textbf{can not effectively combine} the two because current unsupervised node feature generation methods do not consider downstream tasks. 
% Meanwhile, pre-trained models cannot be task-specifically transformed. 
% There is a significant gap between self-supervised tasks and BERT's own pre-training tasks. 
% Directly finetuning BERT would destroy the prior knowledge learned from massive text data.

% Furthermore, current methods generate node features that \textbf{lack interpretability}. 
% The features generated by current methods are continuous and lack interpretability. 
% It is challenging to explain why a particular representation works, and it is difficult to manually select a few features for downstream tasks.
% Meanwhile, the current state-of-the-art methods require finetuning of pre-trained language models (PLMs). However, with the increasing size of PLMs, the computational cost of finetuning has become prohibitively high, often requiring a substantial amount of data to achieve good performance. Thus, it is challenging to integrate these methods with even more powerful language models.

% Therefore, this paper aims to explore the possibility of generating task-aware and graph-aware representations with BERT without finetuning. For the former, a naive method is to use prompts, which are manually input task-related hints, along with text features to generate corresponding words using a language model. For example, for citation networks, we can add prompt information before the abstract: "This is a paper published on <mask> subject, its abstract is [content]." We then use the word distribution after decoding the <mask> as a node feature. However, incorporating graph information into the prompt is challenging. To address this issue, we propose a new framework called GPrompt. This framework combines graph adapters and prompts to extract node features. The graph adapter operates on the last linear transformation layer that predicts words in the LM, i.e., a learnable graph neural network is added to that layer. The goal of the GNN is to help the language model perceive neighbor information of nodes and better predict the masked word. The graph adapter is trained through the language model's native fill-mask task. After the adapter is trained, GPrompt incorporates task-related prompts based on the fill-mask framework of the language model, combined with the graph adapter, to generate task-related representations that are interpretable and perceive graph information.


% pithc on parameter-efficient tuning, cite lora/adaptor
% However, replacing the linear transformation with GNN imposes huge computational costs, and it is not feasible to aggregate neighbors once for each token of every word. To speed up the training process, we adopt DecoupleGNN and use geometric mean to aggregate information from each neighbor. The geometric mean is equivalent to training neighbor nodes with the target node's label in the cross-entropy loss function, so there is no need to globally aggregate neighbor information during GraphAdapter training. This strategy accelerates training effectively through global edge sampling.

We conduct extensive experiments on three real-world datasets in the domains of few-shot and zero-shot learning, in order to demonstrate the effectiveness of our proposed method. The results of our experiments show that G-Prompt achieves state-of-the-art performance in few-shot learning, with an average improvement of \textit{avg.} 4.1\% compared to the best baseline. Besides, our G-Prompt embeddings are also highly robust in zero-shot settings, outperforming PLMs by \textit{avg.} 2.7\%. Furthermore, we conduct an analysis of the representations generated by G-Prompt and found that they have high interpretability with respect to task performance.








%----------------------------------------------------------------------------------

%-----------------------------------BG---------------------------------------------
\section{Background}
\subsection{Text-Attributed Graph}

Let $G = \{V,A\}$ denotes a text-attributed graph (TAG), where $V$ is the node set and $A$ is the adjacency matrix. Each node $i \in V$ is associated with a sentence $S_i = \{s_{i,0},s_{i,1},...,s_{i,|S_i|}\}$, which represents the textual feature of the node. In most cases, the first token in each sentence (i.e., $s_{i,0}$) is $[\mathbf{cls}]$, indicating the beginning of the sentence. This paper focuses on how to unsupervised extract high-quality node features on TAGs for various downstream tasks.

\subsection{Pretrained Language Models}

Before we introduce G-Prompt, we require some basic concepts of pre-trained language models.

\textbf{Framework of PLMs}. A PLM consists of a multi-layer transformer encoder that takes a sentence $S_i$ as input and outputs the hidden states of each token:
\begin{equation}
    \mathbf{PLM}(\{s_{i,0}, s_{i,1},...,s_{i,|S_i|}\}) = \{h_{i,0}, h_{i,1},...,h_{i,|S_i|}\},
\end{equation}
where $h_{i,k}$ is the dense hidden state of $s_{i,k}$.

\textbf{Pretraining of PLMs}. The fill-mask task is commonly used to pre-train PLMs \cite{devlin2018bert,liu2019roberta,he2020deberta}. Given a sentence $S_i$, the mask stage involves randomly selecting some tokens and replacing them with either $[\mathbf{mask}]$ or random tokens, resulting in a modified sentence $\hat{S}_i = \{s_{i,0}, s_{i,1},...,\hat{s}_{i,k},...,s_{i,|S_i|}\}$, where $\hat{s}_{i,k}$ represents the masked token. In the filling stage, $\hat{S}_i$ is passed through the transformer encoder, which outputs the hidden states of each token. We denote the hidden state of the masked token $\hat{s}_{i,k}$ as $\hat{h}_{i,k}$, which is used to predict the ID of the masked token:
\begin{equation}
    \hat{y}_{i,k} = f_{\rm{LM}}(\hat{h}_{i,k}),
\end{equation}
where $f_{LM}$ is a linear transformation with softmax fuction, $\hat{y}_{i,k} \in \mathbb{N}^{1\times T}$, and $T$ is the size of the vocabulary. The loss function of the fill-mask task is defined as $\mathcal{L} = \rm{CE}(\hat{y}_{i,k}, y_{i,k})$, where $y_{i,k}$ is the ID of the masked token, and $\rm{CE}(\cdot,\cdot)$ is the cross-entropy loss.

\textbf{Sentence Embedding}. The hidden state of the $[\mathbf{cls}]$ token ($h_{i,0}$) and the mean-pooling of all hidden states are commonly used as sentence embeddings \cite{reimers2019sentence, gao2021simcse}.

\textbf{Prompting on PLMs}. Sentence embedding and token embedding are simultaneously pre-trained in many PLMs. However, due to the gap between pretraining tasks and downstream tasks, sentence embedding always requires fine-tuning for specific tasks. To address this issue, some studies utilize prompts to extract sentence features \cite{jiang2022promptbert}. For example, suppose we have a paper's abstract $\{\mathbf{Abstract}\}$, and the task is to classify the subject of it. We can add some prompts to the sentence:
\begin{equation}
    \{This, is, a, paper, of, [\mathbf{mask}], subject, its, abstract, is,:, \mathbf{Abstract}\}
\end{equation} 
Then this sentence is encoded by PLMs, and we let $h_{i|p}$ denote the hidden state of the $[\mathbf{mask}]$ token in prompts. Extensive experiment shows that using prompts can shorten the gap between PLMs and downstream tasks and maximize the utilization of the knowledge PLMs learned during pretraining.

\subsection{Graph Neural Networks}

Graph Neural Networks (GNNs) have achieved remarkable success in modeling graph-structured data\cite{velivckovic2017graph,gasteiger2018predict}. The message-passing framework is a commonly used architecture of GNN. At a high level, GNNs take a set of node features $X^0$ and an adjacency matrix $A$ as input and iteratively capture neighbors' information via message-passing. More specifically, for a given node $i \in V$, each layer of message-passing can be expressed as:
\begin{equation}
    x_i^{k} = \mathbf{Pool}\{f_\theta(x^{k-1}_j) | j\in \mathcal{N}_i\}
\end{equation} 
where $\mathbf{Pool}\{\cdot\}$ is an aggregation function that combines the features of neighboring nodes, such as mean-pooling. And $\mathcal{N}_i$ denotes the set of neighbors of node $i$. 
%Different GNN architectures employ different aggregation methods; for instance, GraphSAGE utilizes mean-pool while GAT incorporates an attention mechanism.


% \subsection{Modeling TAGs}
% Most GNNs are designed to operate on continuous node features and cannot handle textual features directly. As a result, modeling TAGs requires combining LMs and GNNs. The most straightforward approach is to join the structure of GNNs and LMs and then end-to-end train them. However, most current LMs are based on Transformers with enormous trainable parameters, so end-to-end training requires significant computing resources.

% Recently, impressive results have been achieved by combining LM and GNNs using the soft connection, e.g., knowledge distillation, and expectation-maximization framework. However, this approach involves fine-tuning LM, which is also extremely computationally expensive. Furthermore, the finetuned model is task-specific, are hard to employ in other downstream tasks.

% A convenient framework commonly used in various applications involves using PLMs to unsupervised convert the textual features of nodes into continuous representations. Then, the extracted node representation and graph structure can be input into GNNs for end-to-end training. It's worth noting that the converted node feature is reusable for many downstream tasks.


% \hxw{problem}
%----------------------------------------------------------------------------------

%-----------------------------------METHOD-----------------------------------------
\section{Method: G-Prompt}
Utilizing the information of downstream tasks and graphs is crucial for generating high-quality node representations. 
The term ``high quality'' is inherently task-specific, as exemplified by the fact that height is a useful feature in predicting user weight but fails to accurately predict age. 
Besides,  the valuable topological information of TAGs can significantly enhance the understanding of textual features in TAGs. 
However, extracting node features using both task and graph information simultaneously is significantly challenging. 
Current PLMs used for handling textual features are graph-free, while current graph-based methods employed to extract node features are primarily task-free. Therefore, this paper proposes a novel self-supervised method, G-Prompt, capable of extracting task-specific and graph-aware node representations. 

\begin{figure*}[t]
	\centering
	\includegraphics[width=0.95\textwidth]{./picture/Model.pdf}
	\caption{Framework of G-Prompt}
	\label{fig:exp}
\end{figure*}

\subsection{Overview}

While previous works have frequently employed PLMs to process TAGs, these investigations have been constrained in extracting a broad node representation from the text-based characteristics and have not incorporated task-specific prior knowledge. 
Consequently, additional learning supervision via GNNs is needed to enable the effective adaptation of these node representations to downstream tasks. 
To address this limitation, the paper suggests incorporating prompts and PLMs into the process of extracting task-relevant node features from TAGs.
%\hkq{which should be highlighted in the next sentence }
Nevertheless, PLMs only utilize contextual information to generate the prompts-related output, which may be insufficient for handling TAGs.
Graph structures often contain essential information that can facilitate a better understanding of textual features.
For instance, in a citation network, a masked sentence such as \textit{``This paper focuses on [MASK] learning in AI domain''} could have multiple candidate tokens based solely on context.
However, if many papers related to graphs are cited, we can infer with greater confidence that the masked token is likely \textit{``graph''}. 
At present, PLMs operate solely based on context, and their structure is graph-free. 
Directly incorporating graph information into PLMs by prompts is not feasible because limited prompts cannot describe the entire topological structure adequately.

Therefore, the proposed G-Prompt leverages a self-supervised based graph adapter and prompts to make PLMs aware of the graph information and downstream task. Given a specific TAG, the pipeline of G-Prompt is as follows: 
(1) Training an adapter on the given TAG to make PLMs graph-aware. 
Specifically, we propose a graph adapter that operates on the prediction layer of PLMs to assist in capturing graph information, which is fine-tuned by the fill-mask task based on the textual data contained by the given TAG. 
(2) Employing task-specific prompts and fine-tuned graph adapters to generate task-aware and graph-aware node features.

\subsection{Fine-Tuning PLMs with the Graph Adapter}

% Currently, PLMs that have undergone large-scale text data pre-training have strong contextual understanding abilities and generalization abilities which form the basis for us to extract specific task information using prompts. 
Using adapters to enable PLMs to perceive graph information is a straightforward idea. 
However, unlike adapters used for downstream task fine-tuning \cite{hu2021lora,liu2022few}, the graph adapter is used to combine prompts in order to extract task-relevant node representations. 
This is an unsupervised process, which means that the graph adapter only receives self-supervised training on given TAGs. 
Consequently, the most challenging aspect of graph adapters is how to assist PLMs in perceiving graph information while also maintaining their contextual understanding capability. 
Additionally, the graph adapter is only trained on a given TAG, generalizing to prompt tokens can also be quite difficult.
Next, we introduce the graph adapter and discuss how it overcomes these challenges in detail.

% The focus of this paper is on promoting the PLM to extract node features of TAGs, which is essentially a fill-mask task. Therefore, this paper proposes Graph Adapter, which \textbf{targets maximally retaining LMs' contextual modeling ability while enabling them to incorporate graph information during the fill-mask process.}

\textbf{Context-friendly adapter placement.} 
The fill-mask task involves two modules of PLMs: a transformer-based module that models context information to obtain representations of masked tokens and a linear transformation that decodes the representation to output the probable IDs of the masked token.
To avoid compromising the contextual modeling ability of PLMs, the Graph Adapter only perform on the last layer of PLMs.
More specifically, the graph adapter is a GNN structure combing with the pre-trained final layer of the PLMs.
Given a specific masked token $\hat{s}_{i,k}$, The inputs of the Graph Adapter are the masked token $\hat{h}_{i,k}$, sentence representations of node $i$ and its neighbors and output is the prediction of the IDs' of the masked token. 
This process aligns with intuition — inferring a possible token based on context first and then determining the final token based on graph information. Formally,
\begin{equation}
    \hat{y}_{i,k} =  \textbf{GraphAdapter} \{f_{\rm{LM}}, \hat{h}_{i, k}, z_i, \{z_j \in \mathcal{N}_i\}, \Theta\},
\end{equation}
where the $z_i$ and $z_j$ denote the sentence embedding of node $i$ and $j$. Note, sentence embedding is task-free and only used to model nodes' influence on their neighbor.
In this paper, we utilize sentence embedding of nodes' textual features as their node feature. 
$\Theta$ is all trainable parameters of the Graph Adapter. 

\textbf{Prompting-friendly network structure}.
% The hidden state of the masked token contains contextual information extracted through the transformer in PLMs. 
% Therefore, directly manipulating it may also affect the contextual information it contains.
The parameters of the adapter are only trained on the fill-mask task based on the textual data contained by the target TAG. 
But the adapter will be used for combining prompts to generate task-related node features in various subsequent downstream tasks.
So the generalization ability of the adapter is crucial. 
On the one hand, the distribution of hidden states responding to masked tokens in prompts may be different from the hidden states used to train the adapter. 
On the other hand, the candidate tokens for task-specific prompts may not appear in the tokens of the TAG. 
Therefore, we carefully design the network structure of the graph adapter and utilize the pre-trained prediction layer of PLM to improve its generalization ability of it.

When it comes to the graph adapter's training stage, it's possible that the hidden states associated with certain prompts may not be present. This means that directly manipulating those hidden states could result in overfitting the tokens already present in the TAGs.
Therefore, the graph adapter models the influence of each modeled node on the distribution of surrounding neighbor tokens based on node feature, which remains unchanged when prompts are added. Considering that some tokens can be predicted well based solely on their context and that different neighbors may have different influences on the same node, the impact of a neighbor on a token is determined jointly by a gate mechanism and the token's context. Give a specific node $i$, it's neighbor $j$, and hidden states of a masked token $\hat{h}_{i,j}$,
\begin{equation}
    \tilde{h}_{i, k, j} = a_{ij}\hat{h}_{i,k} + (1-a_{ij})g(z_j,\Theta_g)
\end{equation}
where $a_{ij} = \mathrm{sigmoid}((z_iW_q)(z_jW_k)^T)$. Here, $g(\cdot)$ represents multi-layer perceptions (MLPs) with parameters $\Theta_g$ that model the influence of node $j$.
It is worth noting that when considering the entire graph, $g(z_j, \Theta_g)$ will be combined with many marked tokens of node $j$'s neighbors, which can help to prevent $g(z_j, \Theta_g)$ from being overfitted on a few tokens.

Subsequently, the graph adapter combines all neighbor influence to infer the final prediction result. Since the prediction layer of PLM (i.e., $f_{LM}(\cdot)$) is well-trained on massive tokens, it also contains an amount of knowledge. Therefore, the graph adapter reuses this layer to predict the final result. 
\begin{equation}
    \tilde{y}_{i,k} =  \mathbf{Pool}\{f_{\rm{LM}}(\tilde{h}_{i, k, j}) | j\in \mathcal{N}_i\},
\end{equation}
In this equation, the $\mathbf{Pool}(\cdot)$ used in this paper is mean-pooling. 
It is worth noting that the position of $f_{\rm{LM}}(\cdot)$ can be interchanged with pooling since it is just a linear transformation. All trainable parameters in the graph adapter are denoted by $\Theta = \{\Theta_g, W_q, W_k\}$.


\subsection{Model optimization of G-Prompt}

The graph adapter is optimized by the original fill-mask loss, $\mathcal{L}_{i,k} = \mathrm{CE} (\tilde{y}_{i,k}, y_{i,k})$, where $\hat{y}_{i,k}$ is the predicted probability of the $k$-th masked token for the node $i$ and $y_{i,k}$ is the true label. We aim to minimize $\mathcal{L}_{i,k}$ with respect to $\Theta$. 

However, in actual optimization, the prediction results of $\tilde{y}_{i,k,j} = f_{\rm{LM}}(\tilde{h}_{i, k, j})$ consist of many small values because of the large vocabulary size of the language model. 
Therefore, using mean-pooling presents a significant problem as it is insensitive to these small values. For example, during some stages of the optimization process, a node may have mostly $0.9$ predictions for the ground truth based on each edge, with only a few being $0.1$. 
Averaging them together would result in a very smooth loss, making it difficult to train the influence of neighbors with temporarily predicted values of 0.1. 
To address this issue, we use geometric mean instead of mean-pooling in the finetuning stage of the graph adapter, which is more sensitive to small values. 
It is easy to prove that the geometric mean of positive numbers is smaller than the arithmetic means, making it harder to smooth and helping the model converge faster. formally, in finetuning stage, the loss function is:
\begin{equation}
    \mathcal{L}_{i,k} = - y_{i,k} \odot \log\{(\prod_{j\in \mathcal{N}_i}{\tilde{y}_{i,k,j}})^{1/|\mathcal{N}_i|}\}
    = -\sum_{j\in \mathcal{N}_i}{ \frac{1}{|\mathcal{N}_i|}y_{i,k}\odot \log(\tilde{y}_{i,k,j})} 
\end{equation}
On the right-hand side of the equation, we are essentially minimizing $\tilde{y}_{i,k,j}$ through the cross-entropy loss $\mathcal{L}_{i,k,j}= \frac{1}{|\mathcal{N}_i|}\mathrm{CE}(\tilde{y}_{i,k,j},y_{i,k})$. It is worth noting that the graph adapter is only performed on the last layer of PLMs. As a result, we can sample a set of masked tokens and preserve their hidden states inferred by the PLMs before training. This implies that training of graph adapters can be achieved with very few computing resources.
\subsection{Prompt-based Node Representations}
After training the graph adapter, it can be combined with task-specific prompts to generate task-specific and graph-aware node representations. Similar to other prompt-based approaches, we simply add task-specific prompts directly into the textual feature. For example, we might use the prompt ``This is a [MASK] user, consider their profile: [textual feature].'' Formally, this process can be expressed as $\hat{h}_{i|p} = \mathbf{PLM}(\{[P_0],[P_1]...[MASK],S_i\})$.
Where, $\hat{h}_{i|p}$ represents the hidden state of the inserted [MASK], while $[P_0],[P_1]...$ refers to the task-specific prompts. The resulting hidden state is then fed into the graph encoder to decode the most probable token.
\begin{equation}
    \hat{y}_{i|p} = \mathbf{Pool}\{{f_{\rm{LM}}(a_{i,j}\hat{h}_{i|p}+(1-a_{i,j})g(z_j,\Theta_g))} | j\in \mathcal{N}_i\}
\end{equation}
$\hat{y}_{i|p}$ is a $|D|$-dimensional vector, where $|D|$ is the size of the PLM vocabulary. Therefore, directly using this prediction result for node representation is not conducive to downstream tasks and storage. Thus, we use the filtered results as node features, denoted by 
$
    x_{i|p} = \mathrm{Filter}(\hat{y}_{i|p})
$. 
Note, each dimension represents the probability of a token being inferred by PLMs with the graph adapter based on node textual features, neighbors' information, and task-respected prompts. Intuitively, tokens that are unrelated to downstream tasks are almost the same for all nodes. 
Therefore, for $Y_{p} \in \mathbb{N}^{|V|\times|D|}$, which denotes prediction results of all nodes. This paper sorts all columns of $Y_p$ in descending order of standard deviation and keeps the top $M$ columns as the node features. Note, we can also manually select task-relevant tokens based on prior knowledge of the task and use them as node features. 
% \subsection{Model optimization of G-Prompt}
% The optimization objective of G-Prompt is straightforward: (1) model the impact of each neighbor on a node's word distribution and (2) infer masked words based on the influence of all neighbors. 
% However, in actual optimization, the large vocabulary size of the language model leads to the prediction results of $\hat{y}_{i,j,k}$ that consist of many small values after softmax. 
% Therefore, using mean-pooling during optimization presents a significant problem as it is insensitive to these small values. For example, during some stages of the optimization process, a node may have mostly 0.9 predictions for the ground truth based on each edge, with only a few being 0.1. 
% Averaging them together would result in a very smooth loss, making it difficult to train the neighbors with temporarily predicted values of 0.1. 
% To address this issue, we use geometric mean instead of mean-pooling in the finetuning stage of the Graph Adapter, which is more sensitive to small values. 
% It is easy to prove that the geometric mean of positive numbers is smaller than the arithmetic means, making it harder to smooth and helping the model converge faster. formally, in finetuning stage, the pooling function is:
% \begin{equation}
%     \mathbf{Pool}^{tr}\{\hat{y}_{i,j,k} | j\in \mathcal{N}_i\} = (\prod_{j\in \mathcal{N}_i}{\hat{y}_{i,j,k}})^{1/|\mathcal{N}_i|}
% \end{equation}

% Considering that multiplication may easily exceed the precision of calculations, we have expanded the loss function based on geometric mean aggregation during optimization. The formula is as follows:
% \begin{equation}
%     \mathcal{L}_{i,k} = - y_{i,k} \odot \log\{(\prod_{j\in \mathcal{N}_i}{\hat{y}_{i,j,k}})^{1/|\mathcal{N}_i|}\}
%     = -\sum_{j\in \mathcal{N}_i}{ \frac{1}{\mathcal{N}_i}y_{i,k}\odot \log(\hat{y}_{i,j,k})}
% \end{equation}

% Therefore, the whole training pipeline is: 
%----------------------------------------------------------------------------------

%-----------------------------------EVAL-------------------------------------------
\section{Evaluation}

In this section, we evaluate multiple aspects of \tool on several datasets of Solana programs.
We start by demonstrating the soundness and completeness precision of \tool with the \emph{Neodyme Breakpoint Workshop} dataset~\cite{Neodyme2021-vw} in~\Cref{sec:eval_valid}.
This dataset is a collection of prevalent Solana program vulnerabilities and is used in previous work~\cite{Cui2022-nm}.
Furthermore, we compare \tool with VRust~\cite{Cui2022-nm}, which is currently the only other approach addressing Solana program security.
Second, we test \tool's vulnerability discovery effectiveness on real-world programs directly taken from the Solana mainnet blockchain. We present our findings and discuss newly discovered bugs in~\Cref{sec:eval_bugs}.
Lastly, in~\Cref{sec:eval_perf}, we demonstrate \tool's performance with bug bounty programs~\cite{immunefi}.
We focus on \tool's test case throughput and achieved code coverage.

\paragraph{Experimental Setup} 
We ran our evaluation on an AMD EPYC 7302P CPU with 16 cores clocked at \SI{3}{GHz} with \SI{256}{GB} RAM.
The experiments are executed in parallel, keeping all physical CPU cores fully occupied.
Each fuzzing experiment uses a single core and uses the same initial seed for all fuzzing runs.

\subsection{Bug Detection Capabilities}
\label{sec:eval_valid}

To validate our design, we test \tool with the Neodyme Breakpoint Workshop dataset~\cite{Neodyme2021-vw}.
The dataset contains common Solana vulnerabilities (cf.~\cref{sub:sol-vulnerabilities}).
This dataset is organized into 5 different levels, where each level consists of a Solana program with a specific vulnerability. 
For this experiment, we fuzz each program with a timeout of 10 minutes.
Our results are depicted in~\Cref{tbl:validity}:
\tool is able to find all the bugs in this dataset within less than 5 seconds.
\tool does \emph{not report any false alarms}, and is able to precisely detect each vulnerability.

We do not include the \emph{Level}~\emph{3} program, because the program has an \emph{account confusion vulnerability}.
Detecting account confusions requires knowledge of the underlying data layout that represents the expected data structure in memory.
This information requires access to the source code of the program.
However, \tool's goal is to detect bugs in Solana programs, without relying on source code, and thus we skip this program.

\paragraph{Comparison with VRust}
In contrast to VRust~\cite{Cui2022-nm}, \tool reliably detects bugs in smart contracts, without source code.
Thus, a full comparison of every metric (e.g., performance) with VRust is impossible.
While VRust is able to detect the same bugs as \tool, it reports false alarms regarding the integer bug in the \emph{Level}~\emph{2} program.
Meanwhile, \tool does not report a single false alarm in this dataset.
In addition, VRust only indicates a missing key check in the \emph{Level}~\emph{0} program and \emph{Level}~\emph{1} program.
\tool, on the other hand, can trace the vulnerability to a missing owner check in the \emph{Level}~\emph{0} program and a missing signer check in the \emph{Level}~\emph{1} program, resulting in \tool being more precise compared to VRust.

%!TEX root = ../main.tex

% \begin{table}[]
% \centering
% \begin{tabular}{@{}l|ccccc|c@{}}
% \toprule
% \multirow{2}{*}{Program} & \multicolumn{5}{c|}{Vulnerabilities} & Time to first \\ 
%                   & MOC     & MSC   & IB    & ACPI & MKC &  Bug (s)      \\
% \midrule
% \texttt{Level 0}  & \tyes   &       & \tyes &     &  & \num{4}                   \\
% \texttt{Level 1}  &         & \tyes & \tyes &     &  & \num{2}                     \\
% \texttt{Level 2}  &         &       & \tyes &     &  & \num{2}                     \\
% %\texttt{Level 3}  &         &       &       &    &   &                       \\
% \texttt{Level 4}  &         &       &       & \tyes & &  \num{1}                     \\
% \midrule
% \texttt{Wormhole}$^{\ast}$  &         &       &       & & \tyes &  \num{37}                     \\
% \midrule


% \end{tabular}%
% \caption{Results of our validity measurement. We mark true bugs with \tyes.}
% \label{tbl:validity}
% \end{table}



\begin{table}[t]
\centering
\resizebox{\columnwidth}{!}{%
\begin{tabular}{@{}l|ccccc|c@{}}
\toprule 
\multirow{2}{*}{Program}    & \multicolumn{5}{c|}{Vulnerabilities}                                                                                     & Time to first        \\
                            & IB                         & MOC   & MSC                      & ACPI                      & MKC                         & Bug (s)              \\
\midrule 
\texttt{Level 0}            & \tyes                      & \tyes &                          &                           &                             & \num{4}              \\
\texttt{Level 1}            & \tyes                      &       & \tyes                    &                           &                             & \num{2}              \\
\texttt{Level 2}            & \tyes                      &       &                          &                           &                             & \num{2}              \\
\texttt{Level 4}            &                            &       &                          & \tyes                     &                             & \num{1}              \\
\midrule 
\texttt{Wormhole}$^{\ast}$  &                            &       &                          &                           & \tyes                       & \num{37}             \\
\midrule 
False Alarms (\tool{} : VRust) & \multicolumn{1}{c}{0 : 62} & 0 : 0 & \multicolumn{1}{c}{0 : 0} & \multicolumn{1}{c}{0 : 3} & \multicolumn{1}{c|}{0 : 107} & \multicolumn{1}{c}{N/A}
\end{tabular}%
}
\caption{Results of our validity measurement. We mark true bugs with \tyes.}
\label{tbl:validity}
% \vspace{-9ex}

\end{table}


\paragraph{The Infamous Wormhole Bug}
In February 2022, wrapped Ether (wETH) with a value of 323 million USD has been stolen from the Wormhole program, which implements a bridge between Ethereum and Solana.
%
The underlying bug is a missing key check in a Solana program.
\tool implements a bug detection oracle that detects this bug.
Furthermore, \tool works on a binary-only level and does not need to understand a program's semantics to detect vulnerabilities.
However, by design, the original Wormhole program and the underlying bug requires this level of context information.
The context is provided by off-chain guardians that check and verify each transaction.
However, we challenged \tool to detect this bug \emph{without} any context information.
Therefore, we created an emulation of this program which shares the same vulnerability as the Wormhole bug.
Here, \tool was able to detect the bug in less than 40 seconds.

\subsection{Discovering New Bugs}
\label{sec:eval_bugs}
To evaluate \tool's effectiveness in discovering unknown vulnerabilities, we assembled a dataset of \noc real-world programs deployed on the Solana mainnet on March 27, 2023.
We take the following steps to ensure that we have the most current and complete collection of Solana programs:
First, we query an RPC node of the Solana network for all programs that belong to the most recent loader program\footnote{At the time of writing, this is \emph{BPFLoaderUpgradeab1e11111111111111111111111}.}.
Second, we use the Solana toolchain to dump each of the programs into an ELF file.
VRust~\cite{Cui2022-nm} is not able to analyze any of these programs, which emphasizes the gap that \tool fills. 
Given the large data set of contracts, we set the timeout to 5 minutes.
In total, \tool reports \num{92} potential security vulnerabilities in \num{52} out of the \noc programs, including \num{30} \emph{missing signer checks}, \num{12} \emph{arbitrary CPIs}, and \num{30} \emph{integer bugs}.
Moreover, \num{20} reports indicate potential vulnerabilities to lamports theft without possessing the vulnerabilities listed before.

\begin{table}[t]
  \centering
  \begin{tabular}{@{}c|c@{}}
    \toprule
    Abbreviation & Full Program ID \\
    \midrule
    \texttt{3nJ2...5erP} & \texttt{3nJ2MWbnS3bW8rnWhejAgnLQTyiqoA5fMq5Z7jRv5erP} \\
    \texttt{3od3...jnsW} & \texttt{3od3X7QN84FTonkyXbQiT1ydxcT9P1jBcA9mbgD3jnsW} \\
    \texttt{3Vtj...4q8v} & \texttt{3VtjHnDuDD1QreJiYNziDsdkeALMT6b2F9j3AXdL4q8v} \\
    \texttt{3w57...obPW} & \texttt{3w57iMhv5Zk5VDuTe5dspm2FzE9zQhscCb9CZpAKobPW} \\
    \texttt{4hPk...JP5N} & \texttt{4hPkNV2WsgPW1wHHcQebvV7GLyLgdDDLEx3Pu6LzJP5N} \\
    \texttt{4M2f...jStx} & \texttt{4M2fancicHbUtMLcMNmbi97YngFoqBcnFk5D31JjjStx} \\
    \texttt{6Lan...szqi} & \texttt{6LanqAFCbucXWSG35ssij4kFDTWJ25BY7d6hbR2szqi} \\
    \texttt{7FWE...9p7p} & \texttt{7FWEcVG1YRW7evGR3bXgu47ge8m6Je7BQuvTzMbn9p7p} \\
    \texttt{9a5d...jZP9} & \texttt{9a5dihgNgBhWnjmRDJ8rUy4ihetvgMmjaPk7NGdsjZP9} \\
    \texttt{9tSW...11yy} & \texttt{9tSWsKwtDL6YseLuh1haGFJk312uu9HGyrnVa5XH11yy} \\
    \texttt{GQ6q...1u6K} & \texttt{GQ6qchUsofiK7rzeFg5jbvpHcJ7pNnfL4yfwaYrB1u6K} \\
    \texttt{9WoL...849B} & \texttt{9WoLnfjLKk1EBtkABhe3vcA8CLogsbs3XBoddn8h849B} \\
    \texttt{H5rp...nPSG} & \texttt{H5rpfCD6hLFCPCfxxqjGg94Gqoigqfk7afhqGLu1nPSG} \\
    \bottomrule
  \end{tabular}
  \caption{Program IDs and abbreviations from~\Cref{tbl:eval_bugs}.}%
  \label{tbl:full-addresses}
\end{table}

Confirming vulnerabilities is challenging due to the absence of source code. Hence, we opted for the following approach. We first generate instructions based on the payload information contained in the vulnerability report generated by \tool.
Next, we analyze the program logs as well as the disassembled eBPF bytecode executed at the runtime of the instructions.
Afterward, we craft transactions and observe if the transactions create an erroneous state in the blockchain. Note that this is a tedious validation process, but a common issue when developing smart contract fuzzers~\cite{echidna,sfuzz,rodler2023efcf}.

At the time of writing, we are able to validate the existence of 14 exploitable bugs, and 2 non-exploitable bugs.
Accordingly, \tool currently has a false alarm rate of \SI{12.5}\%.
\cref{tbl:eval_bugs} shows the 16 discovered bugs and public key abbreviations of the vulnerable programs.
For the sake of reproducibility, we list the full public keys in~\Cref{tbl:eval_bugs}.
%
In the following, we present five interesting vulnerabilities detected by our approach.
To minimize potential damage, we ensured that none of these programs are actively managing valuable assets and that no token accounts exist to which the programs are assigned as authority.

\paragraph{Responsible Disclosure and Ethical Concerns}
Since we conduct this experiment on all Solana programs present on the Solana blockchain, this includes many programs of unknown origin, i.e., the authors are anonymous.
We tried our best to reach out to the program authors of this experiment.
Due to the lack of contact information, we could not disclose our findings directly to the authors of vulnerable Solana programs.
Hence, we decided to disclose \emph{all} of our findings to the Solana foundation\footnote{\url{https://solana.org/}} and offered collaboration to fix the vulnerabilities.

\begin{table}[t]
\centering
\begin{tabular}{@{}l|cccc@{}}
\toprule
\multirow{2}{*}{Program ID} & \multicolumn{4}{c}{Vulnerabilities} \\
                         & MSC   & Integer Bug & ACPI  & Lamport \\
\midrule
% 3nJ2MWbnS3bW8rnWhejAgnLQTyiqoA5fMq5Z7jRv5erP: MSC FP
\texttt{3nJ2...5erP}     & \tno  &       &       &       \\
% 3od3X7QN84FTonkyXbQiT1ydxcT9P1jBcA9mbgD3jnsW: ACPI
\texttt{3od3...jnsW}     &       &       & \tyes &       \\
% 3VtjHnDuDD1QreJiYNziDsdkeALMT6b2F9j3AXdL4q8v: IB
\texttt{3Vtj...4q8v}     &       & \tyes &       &       \\
% 3w57iMhv5Zk5VDuTe5dspm2FzE9zQhscCb9CZpAKobPW: ACPI
\texttt{3w57...obPW}     &       &       & \tyes &       \\
% 4hPkNV2WsgPW1wHHcQebvV7GLyLgdDDLEx3Pu6LzJP5N: ACPI
\texttt{4hPk...JP5N}     &       &       & \tyes &       \\
% 4M2fancicHbUtMLcMNmbi97YngFoqBcnFk5D31JjjStx: IB, MSC
\texttt{4M2f...jStx}     & \tyes & \tyes &       &       \\
% 6LanqAFCbucXWSG35ssij4kFDTWJ25BY7d6hbR2szqi: MSC
\texttt{6Lan...szqi}     & \tyes &       &       &       \\
% 7FWEcVG1YRW7evGR3bXgu47ge8m6Je7BQuvTzMbn9p7p: ACPI
\texttt{7FWE...9p7p}     &       &       & \tyes &       \\
% 9a5dihgNgBhWnjmRDJ8rUy4ihetvgMmjaPk7NGdsjZP9: IB
\texttt{9a5d...jZP9}     &       & \tno &       & \tyes \\
% 9tSWsKwtDL6YseLuh1haGFJk312uu9HGyrnVa5XH11yy: IB, MSC
\texttt{9tSW...11yy}     & \tyes & \tyes &       &       \\
% GQ6qchUsofiK7rzeFg5jbvpHcJ7pNnfL4yfwaYrB1u6K: IB
\texttt{GQ6q...1u6K}     &       & \tyes &       &       \\
% 9WoLnfjLKk1EBtkABhe3vcA8CLogsbs3XBoddn8h849B
\texttt{9WoL...849B}     &  &       &   \tyes    &       \\
% H5rpfCD6hLFCPCfxxqjGg94Gqoigqfk7afhqGLu1nPSG: IB
\texttt{H5rp...nPSG}     &       & \tyes &       &       \\
% X3NongcyQZDhmvFYWbUASHXurHTTyPXYfdDEci99uuR: MSC (2)
%\texttt{X3No...9uuR}     & \tyes$^{\ast}$ &       &       &       \\
\midrule
\end{tabular}%
\caption{Results of our bug discovery experiment. We mark true bugs with \tyes and false alarms with \tno.}%\\$^{\ast}$} %Two distinct bugs of this type were investigated and are actual bugs.}
\label{tbl:eval_bugs}
\vspace{-2 em}
\end{table}
% \begin{table}[h]
% \centering
% \begin{tabular}{c|c|c}

% \textbf{Bug Class} & \textbf{Public Key} & \textbf{Reported} \\
% \toprule
%     % 3VtjHnDuDD1QreJiYNziDsdkeALMT6b2F9j3AXdL4q8v
%     Integer Bugs & \texttt{3Vtj...4q8v} & \tyes \\
%     % GQ6qchUsofiK7rzeFg5jbvpHcJ7pNnfL4yfwaYrB1u6K
%     Integer Bugs & \texttt{GQ6q...1u6K} & \tyes \\
%     % H5rpfCD6hLFCPCfxxqjGg94Gqoigqfk7afhqGLu1nPSG
%     Integer Bugs & \texttt{H5rp...nPSG} & \tyes \\
%     % 9tSWsKwtDL6YseLuh1haGFJk312uu9HGyrnVa5XH11yy
%     Integer Bugs & \texttt{9tSW...11yy} & \tyes \\
%     % 9a5dihgNgBhWnjmRDJ8rUy4ihetvgMmjaPk7NGdsjZP9
%     Integer Bugs & \texttt{9a5d...jZP9} & \tpossible \\
%     % 4M2fancicHbUtMLcMNmbi97YngFoqBcnFk5D31JjjStx
%     Integer Bugs & \texttt{4M2f...jStx} & \tyes \\
% \hline
%     % 3w57iMhv5Zk5VDuTe5dspm2FzE9zQhscCb9CZpAKobPW
%     Arbitrary CPI & \texttt{3w57...obPW} & \tyes \\
%     % 7FWEcVG1YRW7evGR3bXgu47ge8m6Je7BQuvTzMbn9p7p
%     Arbitrary CPI & \texttt{7FWE...9p7p} & \tyes \\
%     % 3od3X7QN84FTonkyXbQiT1ydxcT9P1jBcA9mbgD3jnsW
%     Arbitrary CPI & \texttt{3od3...jnsW} & \tyes \\
%     % 4hPkNV2WsgPW1wHHcQebvV7GLyLgdDDLEx3Pu6LzJP5N
%     Arbitrary CPI & \texttt{4hPk...JP5N} & \tyes \\
% \hline
%     % 3nJ2MWbnS3bW8rnWhejAgnLQTyiqoA5fMq5Z7jRv5erP
%     Missing Signer Check & \texttt{3nJ2...5erP} & \tno \\
%     % 6LanqAFCbucXWSG35ssij4kFDTWJ25BY7d6hbR2szqi
%     Missing Signer Check & \texttt{6Lan...szqi} & \tyes \\
%     % X3NongcyQZDhmvFYWbUASHXurHTTyPXYfdDEci99uuR
%     Missing Signer Check & \texttt{X3No...9uuR} & \tdouble \\
%     % X3NongcyQZDhmvFYWbUASHXurHTTyPXYfdDEci99uuR
%     %Missing Signer Check & \texttt{X3No...9uuR} & \tyes \\
% \hline
%     % 9a5dihgNgBhWnjmRDJ8rUy4ihetvgMmjaPk7NGdsjZP9
%     Lamports-based & \texttt{9a5d...jZP9} & \tyes \\
% \end{tabular}
% \caption{\tyes: Exploitable, \tpossible:   }
% %\label{tab:eval_bugs}
% \end{table}


\finding{Integer bugs in 3Vtj...4q8v} This program contains an \emph{integer bug} that an attacker can use to steal lamports from a program-controlled account. 
The vulnerable instruction requires four accounts, and subtracts one SOL from the fourth account while adding it to the first account.
However, when reducing and crediting, the program does not check whether an overflow or underflow of lamports has occurred. 
Given that the first account owns too much SOL and the fourth account owns too little, an attacker could exploit the integer bug to add lamports to the fourth account and subtract lamports from the first account.

\finding{Arbitrary CPI in 3w57...obPW}
\tool found an \emph{arbitrary CPI} vulnerability.
We confirmed the bug by sending a malicious instruction.
The instruction accepts five accounts, where the last account signs the transaction.
The program then invokes the first account supplied without checking the account.

\finding{Arbitrary CPI in 9WoL...849B}
\tool detected an \emph{arbitrary CPI} vulnerability which grants additional privileges to the invoked program by signing a PDA in the CPI instruction.
The program \texttt{9WoL...849B} manages accounts whose public key follows a PDA seed structure consisting of one seed corresponding to a passable public key (e.g., a wallet account). 
Similarly to \texttt{3w57...obPW}, we have also created an instruction for \texttt{9WoL...849B} to confirm the bug:
The instruction expects \emph{13} accounts, where account \emph{13} is the program an attacker can invoke arbitrarily, and accounts \emph{3}, \emph{8}, and \emph{11} are accounts that have the same public key matching the PDA seed structure of \texttt{9WoL...849B}, i.e., a PDA associated to \texttt{9WoL...849B}.
The instruction results in \texttt{9WoL...849B} invoking the arbitrary invocable program and signing the PDA in the CPI instruction.
This leads to the invoked program having additional privileges than originally included in the transaction. 
Since the program states in the program log that it generates an instruction for calling the function \emph{borrow\_obligation\_liquidity} of the Serum Swap program before performing CPI, we assume that it is supposed to manage accounts of the Serum Swap program as an authority.

\finding{Multiple Vulnerabilities in 4M2f...jStx} 
Here, \tool discovered both a \emph{missing signer check} and an \emph{integer bug} in this program. 
The program expects an instruction containing four accounts and allows playing a gambling game in which a user can win or lose. 
The number of lamports of the second account multiplied by an odd of \emph{1.9934} determines the total payout of the game. 
In case the user wins the game, the program credits the payout to the first account and subtracts
\begin{inparaenum}[1)]
    \item from the fourth account the lamports worth \emph{0,9334} multiplied by the lamports of the second account, and 
    \item from the second account, its total lamport balance.
\end{inparaenum}
When crediting and subtracting the lamports, the program does not check whether overflows or underflows have occurred.
Hence, an attacker can exploit the integer bug to credit the fourth account with lamports instead of subtracting lamports in the case that the user won the game. 
%
In addition, the program does not check which account signed the transaction. 
Thus, an attacker can submit arbitrary program-controlled accounts and start gambling without the program ever checking whether the attacker is authorized to gamble with the submitted accounts. 
We note that the programs \texttt{6Lan...szqi} and \texttt{9tSW...11yy} also allow gambling similar to \texttt{4M2f...jStx}, and also do not verify that gambling with the submitted accounts is authorized. 

\finding{Integer Bug in 9a5d...jZP9}
Besides suffering from an integer bug, this program also enables an attacker to transfer lamports from a program-controlled account to an arbitrary, attacker-controlled account. 
The program expects two accounts and subtracts all lamports of the first account and credits them to the second account without checking for overflow or underflow of lamports. 
Thus, the integer bug is not exploitable, as only the lamports field of the second account can overflow but not of the first account. 
However, the program allows passing an arbitrary program-controlled account as a first account, transferring its lamports to the second account. 
In general, such a behavior is undesirable, as it allows an attacker to drain funds of all accounts belonging to this program.
This is a clear indication of an access control bug:
the intended program behavior would surely only allow an \emph{authorized} account to transfer lamports from program-controlled accounts to carefully selected accounts.  
To address this bug, the program must verify that the authorized account is included in the instruction and that it signed the transaction to prevent exploitation. 

\subsection{Performance Analysis}
\label{sec:eval_perf}

Given that \tool is the first fuzzer for Solana programs, there exists no qualitative baseline or dataset to measure common fuzzing metrics like coverage and execution speed.
Thus, we also aim at establishing a baseline allowing the community to compare future fuzzers with \tool.

We assembled a dataset from the Immunefi bug bounty list~\cite{immunefi}. 
This dataset includes a diverse set of Solana programs used in production and offers a higher grade of code quality and complexity, compared to the average mainnet programs. 
Hence, we assess the execution speed and code coverage of \tool based on this dataset.
For this experiment, we use a representative timeout of 24 hours and collect metrics on execution speed and code coverage.

\paragraph{Binary-only Approach Baseline}
Unlike the experiments in \Cref{sec:eval_bugs}, source code for the bug-bounty dataset is available.
The performance and coverage of \tool could potentially be optimized by analyzing the source code to extract context information about authority or configuration accounts, solving assertions, and uncovering new code paths. 
However, we refrain from doing so because (1)~we aim to provide representative measurements and (2) for the large majority of Solana programs no source code is available. 

\paragraph{Challenge: Measuring Code Complexity of Solana programs}
There is currently no tool support to measure the complexity of Solana smart contracts.
Previous work, like VRust~\cite{Cui2022-nm}, relied on lines of code (LOC) to estimate the complexity of a program. However, this metric is insufficient since it includes unreachable code.
Furthermore, \tool's coverage works on traversed edges in a program's control flow graph (CFG) and is therefore incomparable to LOC.
To tackle this challenge, we developed a static analysis approach which measures the complexity based on traversing the eBPF code and counting every control flow instruction that eBPF supports.
We use this number to over-approximate the number of edges in the CFG.

Note that this ensures that we include every eBPF \texttt{JMP},
\texttt{CALL}, and \texttt{RET} instruction.
This includes any edges that are by design not reachable, since these may represent dead code or Solana-specific error handling routines.
For example, one routine is the handling of an incorrect serialized program id at the program input, which is never executed due to the valid instructions generated by the transaction generator.

We argue that this is sufficient to estimate the complexity of Solana programs because it provides a better insight into the complexity of a program than LOC.
We analyze the target contracts with our CFG-based approach, and compare the results with the covered code paths by \tool.


%!TEX root = ../main.tex

\begin{table}[t]
\centering
\begin{adjustbox}{max width=\linewidth}
\begin{tabular}{@{}l|r|rr|r@{}}%|rrr@{}}
\toprule
\multicolumn{1}{c|}{Program} & \multicolumn{1}{c|}{Bounty}   & \multicolumn{1}{r}{\#CFG} & \multicolumn{1}{r|}{Covered} & \multicolumn{1}{r}{Mean} \\ %& \multirow{2}{*}{Alarms} & \multicolumn{1}{c}{Verified} & Time to first \\
\multicolumn{1}{c|}{Name}    & \multicolumn{1}{c|}{(\$)}     & \multicolumn{1}{r}{Edges}  & \multicolumn{1}{r|}{Edges} &  \multicolumn{1}{r}{Tx/s} \\%     &           & \multicolumn{1}{c}{Alarms} & Bug (s) \\
\midrule
\texttt{Drift Protocol}           & \num{500000}  & \num{67552} & \num{2336} & \num{5211} \\%& 0 & --- & ---\\
\texttt{Jet Airspace}             & \num{100000}  & \num{7559}  & \num{1398} & \num{1951} \\%& 0 & --- & ---\\
\texttt{Jet Control}              & \num{100000}  & \num{9506}  & \num{2716} & \num{1500} \\%& 0 & --- & ---\\
\tno \texttt{Jet Fixed Term}   & \num{100000}  & \num{27246} & \num{2375} & \num{1256}   \\%& 1 & 0   & 115\\         % FP
\texttt{Jet Margin}               & \num{100000}  & \num{18216} & \num{1332} & \num{1771} \\%& 0 & --- & ---\\
\texttt{Jet Margin Swap}          & \num{100000}  & \num{12843} & \num{2626} & \num{1332} \\%& 0 & --- & ---\\
\texttt{Jet Metadata}             & \num{100000}  & \num{5164}  & \num{1310} & \num{786}  \\% & 0 & --- & ---\\
\tyes \texttt{Jet Test Service}   & \num{100000}  & \num{17972} & \num{2706} & \num{2894} \\%& 1 & 1   & 1  \\
\texttt{Lido}                     & \num{2000000} & \num{8731}  & \num{1305} & \num{274}  \\%& 0 & --- & ---\\           % Anker: same
\texttt{Marinade Finance}         & \num{250000}  & \num{22424} & \num{2648} & \num{1023} \\%& 0 & --- & ---\\
\texttt{Port Finance VRL}         & \num{500000}  & \num{10704} & \num{1643} & \num{1066} \\%& 0 & --- & ---\\
\texttt{Pyth}                     & \num{500000}  & \num{4438}  & \num{2984} & \num{576}  \\%& 0 & --- & ---\\
\texttt{Solend Program}           & \num{1000000} & \num{10818} & \num{1681} & \num{1129} \\%& 0 & --- & ---\\
\texttt{Sundial}                  & \num{500000}  & \num{16792} & \num{3171} & \num{1896} \\%& 0 & --- & ---\\
\texttt{Token Faucet}             & \num{500000}  & \num{6500}  & \num{1299} & \num{1217} \\%& 0 & --- & ---\\
\texttt{Whirlpool}                & \num{500000}  & \num{20593} & \num{2726} & \num{1229} \\%& 0 & --- & ---\\
\midrule
\emph{$n=16$}                    &                & $\overline{x} ={16691}$  & $\overline{x}={2141}$ & $\overline{x}={1569}$\\
%\emph{$n=16$}                    &                & $\sum ={16691}$  & $\overline{x}={ }???$ & $\overline{x}={ }???$\\
\bottomrule
\end{tabular}%
\end{adjustbox}
\caption{Results of our performance experiment.}% We mark programs where \tool reports a valid bug with \tyes. We use the \tno marker for reported bugs that we cannot validate.}
\label{tbl:performance}
\vspace{-2 em}
\end{table}

\paragraph{Coverage}
\Cref{tbl:performance} shows the results of this experiment.
First, we observe that the estimated complexity of the dataset varies widely, ranging from \num{4438} edges in the control flow graph to \num{67552}.
This confirms that the size and complexity of the programs in the dataset is diverse.
The number of covered edges by \tool ranges from \num{1299} to \num{3171}.
This provides two important insights: First, \tool is able to consistently generate meaningful transactions to uncover new program paths.
Second, the binary-only analysis approach leads to a number of programs, where the number of covered edges does not increase over time.
By further investigation, we learned that certain barriers or assertions prevent \tool from reaching deeper nested code.
Hence, there is room for optimization for future work. For example, extending \tool with symbolic execution~\cite{fuzz-symex, Mossberg2019-xp}, using Redqueen~\cite{aflpp, Aschermann2019-ha}, or, as mentioned before, incorporating the available source code (\Cref{sec:eval_bugs}), to overcome these roadblocks.

\paragraph{On Fuzzing Throughput}
Another interesting insight is that \tool is capable to generate on average more than 1569 transactions per second for every Solana program.
Furthermore, \tool has an average of over \num{1000} transactions per second for 13 out of 16 programs. 
However, even in these three outliers, \tool is able to generate at minimum \num{274} transactions per second.
We understand that \tool regularly extracts new runtime semantics for these programs, causing the blockchain emulator to initiate updating the \ledgersnap as well as deriving new PDAs.
We measured that initializing the snapshot takes about \num{60}ms, which results in fewer transactions per second being generated from the input bytes.

\paragraph{Vulnerability Reports}
\tool reported 2 bugs in the bug dataset both belonging to the Jet Protocol. 
We investigate the bugs while also consulting the source code of each program. As we will see, one of the bugs is a false positive while we believe the other one is a true positive which is currently under review by the developers.

\finding{False Alarm in Jet Fixed Term}
\tool reports a missing signer check in the \texttt{Jet Fixed Term} program, which is a program for fixed-term lending and borrowing.
The missing signer check exists in a function that cancels orders, which requires two accounts as its input.
While the first account strictly belongs to a user of this program, the second account is a public order book containing the orders.
By signing the transaction, a user is granted authority to remove an order from the marketplace.
As a reminder, the oracle of \tool (c.f.~\Cref{sub:oracle}) considers transitive signer checks if accounts are linked in some way, i.e., an account $a$ may refer to another account $b$ if $a$'s data field contains the public key of $b$.
Additionally, the public key stored in $a$ must be compared to the public key of $b$.
However, in this particular case, the order book's account data may contain the public key of the authority account, but the public key of the authority account---which signed the transaction---is never compared to it.
As a result, \tool reports a missing signer check for this program.
However, we consider this a false alarm, as the bug is not exploitable, because the program checks whether the user account owns the canceled order.

\finding{True Positive Bug in Jet Test Service}
\tool reports an arbitrary cross-program invocation for the \texttt{Jet Test Service} program from the Jet Protocol.
The arbitrary cross-program invocation exists in a function that accepts an arbitrary amount of accounts as long as a minimum of two accounts is provided:
the first account is a potentially uninitialized account, and the second account can be any other program of the Solana blockchain.

The function then checks whether the first account provided is initialized on the Solana blockchain, and if this is \emph{not} the case, the function invokes the second program using CPI.
Since there are no restrictions on the choice of account to invoke, we consider this a true bug in the \texttt{Jet Test Service} program.
This bug can be overcome by having the \texttt{Jet Test Service} program verify the public key of the second account before invoking the second account using CPI. 

We are now in contact with the vendor to fix these issues and confirm the bugs.
In conclusion, we can confirm the ability to fuzz complex targets with high transaction throughput and coverage.
%----------------------------------------------------------------------------------

\section{Empirical Study}
\label{sec:Measure}

In this section, \X1 We perform a more detailed measurement of LLM framework vulnerabilities detected in Section \ref{sec:eval:1}. \X2 We categorize the apps tested during prompt attacks in Section \ref{sec:eval:4} based on their capabilities and delve into the reasons behind attack failures. \X3 We conduct a detailed hazard analysis of these RCE vulnerabilities and propose new practical real-world attacks.


\subsection{Vulnerabilities in LLM-Integrated Frameworks}

\begin{table*}
	\centering
	\scriptsize
\caption{Vulnerabilities found by \tool{}. (CVEs with ``*'' mean that we are not the first discovering these vulnerabilities, and non-* represents the ones credited to us. ``RCE'' is the remote code execution and ``R/W'' represents the vulnerability type of arbitrary file read and write)}
\label{tab:vulns}
\vspace{-10pt}
\begin{tabular}{llllllll}
\toprule
\textbf{Framework} & \textbf{User-level API}                & \textbf{Type}             & \textbf{Trigger}  & \textbf{CVE}     & \textbf{CVSS}     & \textbf{Description}                                                             \\ \midrule

LangChain          & create\_csv\_agent               &RCE     & Prompt                            & CVE-2023-39659   & 9.8               & Execute code without checking                        \\


LangChain          & create\_spark\_dataframe\_agent  &RCE     & Prompt                            & CVE-2023-39659   & 9.8               & Execute code without checking                        \\


LangChain          & create\_pandas\_dataframe\_agent &RCE     & Prompt                            & CVE-2023-39659   & 9.8               & Execute code without checking                        \\
LangChain          & PALChain.run                     &RCE     & Prompt                            & CVE-2023-36095   & 9.8               & Execute code without checking                        \\
LangChain          & load\_prompt                     &RCE     &Loaded File                         & CVE-2023-34541*   & 9.8*               & Use dangerous ``eval'' while loading prompt from file                            \\
LlamaIndex       & PandasQueryEngine.query          &RCE     & Prompt                            & CVE-2023-39662   & 9.8 & Execute code without checking (need LLM escape) \\
Langflow           & api/v1/validate/code             &RCE     & API Post                      & CVE-2023-40977 & Pending  & Limited trigger condition of exec can be bypassed via API post       \\
Langflow           & load\_from\_json                 &RCE     &Loaded File                         & CVE-2023-42287 & Pending  & Limited trigger condition of exec can be bypassed via loading file   \\
PandasAI          & PandasAI.\_\_call\_\_\_          &RCE     & Prompt                            & CVE-2023-39660 & 9.8  & Sandbox can be bypassed (need LLM escape \& code escape)  \\
PandasAI          & PandasAI.\_\_call\_\_\_          &RCE     & Prompt                            & CVE-2023-39661   & 9.8 & Sandbox can be bypassed (need LLM escape \& code escape)  \\
PandasAI          & PandasAI.\_\_call\_\_\_          & R/W & Prompt                            & CVE-2023-40976 & Pending  & Sandbox allows file read and write (need LLM escape)         \\
Pandas-llm         & PandasLLM.prompt                 &RCE     & Prompt                            & CVE-2023-42288 & Pending  & Sandbox does not work as expected                                  \\
Pandas-llm         & PandasLLM.prompt                 &RCE     & Prompt                            & CVE-2023-42288 & Pending  & Sandbox does not work as expected (need LLM escape)      \\    
Griptape         & griptape.tools.Calculator                &RCE     & Prompt                            & CVE-2024-25835 & Pending  & Execute code without checking (need LLM escape) \\
Lagent         & lagent.actions.PythonInterpreter                &RCE     & Prompt                            & CVE-2024-25834 & Pending  & Execute code without checking \\
langroid & TableChatAgent.run & RCE & Prompt & Reporting & - & Execute code without checking (need LLM escape) \\
LlamaIndex         & PandasQueryEngine.query                &RCE     & Prompt                            & - & -  & Bypass the fix via third-party library (need LLM escape \& code escape) \\
MetaGPT & metagpt.strategy.tot.TreeofThought & RCE & Prompt & CVE-2024-5454 & 8.4 & Execute code without checking (need LLM escape)\\ 
MetaGPT & DataInterpreter & RCE & Prompt & - & - & Execute code without checking (need LLM escape)\\
vanna & vanna.ask & RCE & Prompt & CVE-2024-5826 & 9.8 &  Execute code without checking (need LLM escape)
\\ \bottomrule
\vspace{-10pt}
\end{tabular}
\end{table*}

As shown in Table \ref{tab:vulns}, we have discovered a total of 20 vulnerabilities across 11 frameworks and obtained 13 CVEs. 
There are mainly three types of attack triggers: \emph{prompt}, meaning that RCE can be achieved via user prompts to the target app; \emph{API post}, where users send a post via APIs to the app, and \emph{loaded file} is a type of files that are uploaded by users and then loaded by apps, triggering RCE vulnerabilities. 

Here, ``prompt'' is the primary triggering entry point to these vulnerabilities. Therefore, we dive deeper into these vulnerabilities triggered via prompts as follows.

\vspace{3pt}
\noindent\textbf{Vulnerability Type.} These vulnerabilities can be categorized into two types, \ie, remote code execution and arbitrary file read/write. 
In particular, RCE allows remote execution of arbitrary code, leaking sensitive data (\eg developers' OpenAI API key, azure key), even granting control over the server. Arbitrary file read indicates that the attacker gains unauthorized access to some files on a system, and arbitrary file write enables an attacker to modify and create files on the system without proper authorization. 

\vspace{3pt}
\noindent\textbf{Vulnerability Triggering.} The root causes of these critical vulnerabilities are straightforward and intuitive: using hazardous functions to execute untrusted code generated by LLMs. 
However, it requires different prompts to trigger vulnerabilities across frameworks. 
Taking LangChain as an example, an attacker can merely send the request of executing one piece of code, leading to the RCE vulnerabilities. 
For the remaining frameworks like PandasAI, prompts from users will be rewritten or transformed to become more detailed and complex before being passed to the LLM, where the trigger may cease to effect.
For example, when a user sends a prompt \textit{``How many items are there in the dataframe?''}, PandasAI first embeds the input prompt into a template, \eg, \textit{``You are provided with a pandas dataframe (df) with \{num\_rows\} rows and \{num\_columns\} columns, ..., return the python code exactly to get the answer to the following question: How many items are there in the dataframe?''} and then passes it to LLMs.
The additional content is a system prompt, designed initially for providing LLMs with more information about specific tasks (\eg input/output format, detailed description of tasks). 
Interestingly, our attack payloads are always significantly shorter than the templates. Therefore, when these attack prompts are embedded within the templates, the semantics of the payloads become diluted and appear incongruous. Thus, the LLM's attention to the payload is consequently diverted during inference. As a result, the LLM frequently fails to assist effectively in generating malicious code as demanded, either due to safety alignment mechanisms or attention diversion.
Thus, these detailed and complex templates unintentionally grant the framework security ability by offsetting malicious prompts' semantic and the corresponding attention.
However, it can be bypassed through LLM escape.
Additionally, exploitation varies across different frameworks, highlighting discrepancies in security awareness among framework developers.
Some developers (e.g., developers from PandasAI) exhibit a good security awareness, evident in their implementation of a custom sandbox rather than directly code execution. Even if attackers bypass prompt template interference and safety alignment to generate malicious code, this sandbox restricts allowed keywords, functions, and execution environments to prevent arbitrary code execution. However, it is not robust enough, as experienced attackers may escape the sandbox using Python's builtin features (e.g., inheritance chain).
Thus, to successfully exploit vulnerabilities in PandasAI, it necessitates not only LLM escape to eliminate the interference from system prompts, but also code escape to circumvent the custom sandbox implemented by the developers. Figure \ref{fig:pandasai_attack} shows how to exploit PandasAI with LLM escape and code escape working together. 
\begin{figure}
	\centering
    \vspace{-8pt}
	%\setlength{\abovecaptionskip}{0pt}
	\setlength{\belowcaptionskip}{0pt}
	\includegraphics[width=.9\columnwidth]{figures/pandasai_attack.pdf}
	\caption{LLM escape and Python sandbox escape to RCE in PandasAI. Attack session 1 stands for attack prompt with only code escape; attack session 2 stands for attack prompt with only LLM escape; and attack session 3 stands for attack prompt with LLM escape and code escape.} 
	\label{fig:pandasai_attack}
	\vspace{-7mm}
\end{figure}

Although AutoGPT uses a separate Docker container for each code execution behavior to ensure environment isolation, this approach results in significant efficiency loss. Furthermore, this seemingly foolproof solution also has security vulnerabilities. Before this study, some researchers discovered security issues in AutoGPT (CVE-2023-37273~\cite{cve-2023-37273}), allowing attackers to achieve Docker escape by overwriting the \texttt{docker-compose.yml} file. Thus, in the era of LLMs, RCE vulnerabilities have been somewhat overlooked even by well-known frameworks during the rapid development, becoming both tricky and difficult to mitigate due to the trade-off between usability and security.

\subsection{Analysis of LLM-Integrated Apps} \label{sec:Measure:class}
In this section, we intend to systematically and comprehensively understand LLM-integrated apps and the exploitability of their vulnerabilities, as well as to extract insightful information from them.
Based on the experiment in Section~\ref{sec:eval:4}, we first conduct an investigation on the reasons of exploitation failures during prompt attacks. Then we further explore the exploitability level for successful attacks, \ie, what we can achieve through the exploitations. 

\vspace {3pt}\noindent\textbf{Failure Reasons.}
There are 5 types of failure reasons leading one app to be not exploitable (where CE represents for ``code execution'').

\begin{itemize} [leftmargin=*, topsep=0pt,parsep=0pt]
    \item \textbf{Runtime Exceptions.} One app may be dysfunctional due to internal issues and cannot be interacted with properly. Prompt attacks are unsuccessful upon it crashes.
    \item \textbf{Restricted Prompts.} Some apps have restrictions on user provided prompts. As a result, prompt injection, which requires crafting arbitrary prompts, cannot work anymore.
    \item \textbf{Without CE Ability.} Some apps may not possess the ability to execute code, which is common in the apps collected in a black-box manner.
    \item \textbf{Protection from CE.} In such cases, code execution is feasible. But protective measures or limitations are deployed, which can protect apps from prompt attacks.
    \item \textbf{Others.} The remaining is unidentified, especially when LLM-integrated apps exert unique and undisclosed measures like setting query limits and user permission.
\end{itemize} 

As shown in Table~\ref{tab:app_class}, ``Runtime Exception'' and ``Without CE Ability'' account for the largest portions among these failure reasons, with a percent of 38.2\% and 29.4\%, respectively.
However, the most interesting and research-worthy aspect is ``Protection from CE''. Unlike the conventional approaches of executing LLM-generated code on the server and returning results, these apps use Pyodide~\cite{pyodide}, a Python distribution for browsers and Node.js based on WebAssembly, to run the code directly in the browser. Therefore, the code is executed on client-side rather than the server. It fundamentally resolves the RCE vulnerability. However, we observed that such apps are relatively rare for two reasons: \X1 Developers may not have strong security awareness; \X2 Developers are reluctant to sacrifice app functionality and efficiency for security. We observed that technologies like Pyodide only support a limited number of third-party libraries which may not satisfy the needs of LLM-generated code. Additionally, loading the app for the first time can be extremely slow, as the browser may need to download an entire Python interpreter and third-party libraries.

\vspace {3pt}\noindent\textbf{Exploitability Levels.} As for the successful prompt attacks in Section~\ref{sec:eval:4}, we categorize the severity of exploitations with 4 levels.

\begin{itemize} [leftmargin=*, topsep=0pt,parsep=0pt]
    \item \textbf{SQL Injection.} Attackers can perform SQL injection attack against the database via the prompt. Different from conventional SQL injection~\cite{halfond2006classification}, the database manager executes one command that is generated by LLMs without security sanitization. 
    \item \textbf{Limited RCE.}
    Attackers can achieve limited RCE through crafted prompts, meaning only a specific set of code or commands can be executed successfully.
    \item \textbf{Reverse Shell.} Attackers can leverage RCE to gain whole and persistent control over the remote host using reverse shell techniques, allowing them to launch multiple attacks subsequently.
    \item \textbf{Root.} Upon receiving a reversed shell, some apps allow attackers to escalate their privileges to root on the remote host without using complex kernel exploitation.
\end{itemize}



\begin{table*}[!htbp]
\caption{Statistics of (non-)exploitable LLM-integrated apps.}
\label{tab:app_class}
\centering
\scriptsize
\vspace{-10pt}
\begin{tabular}{c|ccccc|cccc}
\toprule
 \multirow{2}{*}{\textbf{Type}}  & \multicolumn{5}{c|}{\textbf{Not Exploitable (34): Failure Reasons}}                                                                         & \multicolumn{4}{c}{\textbf{Exploitable (17): Exploitability Levels}} \\ \cline{2-10}
            & \multicolumn{1}{c}{\textbf{Runtime Ex.}} & \multicolumn{1}{c}{\textbf{Restricted Prompts}} & \multicolumn{1}{c}{\textbf{w/o CE Ability}} & \multicolumn{1}{c}{\textbf{Protection from CE}} & \textbf{Others} & \multicolumn{1}{c}{\textbf{SQL Inj.}} & \multicolumn{1}{c}{\textbf{Limited RCE}} & \multicolumn{1}{c}{\textbf{Reverse Shell}} & \textbf{Root} \\ \midrule

\textbf{\#White-Box}      & \multicolumn{1}{c}{7}               & \multicolumn{1}{c}{1}                     & \multicolumn{1}{c}{1}             & \multicolumn{1}{c}{0}                      & 1              & \multicolumn{1}{c}{1}                      & \multicolumn{1}{c}{13}            & \multicolumn{1}{c}{11}                      & 2             \\
% \textbf{Gray-Box\#} & \multicolumn{1}{c|}{2}               & \multicolumn{1}{c|}{0}                     & \multicolumn{1}{c|}{1}              & \multicolumn{1}{c|}{0}                      & 0              & \multicolumn{1}{c|}{0}                      & \multicolumn{1}{c|}{5}            & \multicolumn{1}{c|}{5}                      & 1             \\
\textbf{\#Black-Box}      & \multicolumn{1}{c}{6}               & \multicolumn{1}{c}{2}                     & \multicolumn{1}{c}{9}              & \multicolumn{1}{c}{2}                      & 5              & \multicolumn{1}{c}{0}                      & \multicolumn{1}{c}{3}            & \multicolumn{1}{c}{3}                      & 2             \\ \midrule


\textbf{\#Total}          & \multicolumn{1}{c}{\textbf{13}}              & \multicolumn{1}{c}{\textbf{3}}                     & \multicolumn{1}{c}{\textbf{10}}             & \multicolumn{1}{c}{\textbf{2}}                      & \textbf{6}              & \multicolumn{1}{c}{\textbf{1}}                      & \multicolumn{1}{c}{\textbf{16}}           & \multicolumn{1}{c}{\textbf{14}}                     & \textbf{4}             \\ \bottomrule
\end{tabular}
\vspace{-8pt}
\end{table*}

Here, we analyze the data in Table~\ref{tab:app_class} from the vertical and horizontal views.

From a vertical perspective, it is observed that 17 of them can be successfully exploited, accounting for 33.3\% of the total (51). Out of these 17 apps, 16 of them suffer from limited remote code execution (limited RCE), making up 31.4\% of the total. Among the exploitable apps, 14 of them allow the attackers to obtain a reversed shell, representing 27.5\% of the total and 87.5\% of the apps with RCE vulnerability. Furthermore, 4 of these reverse shell-exploitable apps can attain root privileges without using complex kernel exploitation after the attacker gains the shell, constituting 7.8\% of the total and 28.6\% of the reverse shell-exploitable apps.

From a horizontal perspective, it is observed that from 51 LLM apps above, there are 24 white-box apps and 27 black-box apps. We calculate their exploitable ratio respectively. The exploitable rate of white-box apps is 58.3\% and 11.1\% for black-box apps. 

These statistics provide us with the following insights: 
\X1 A significant portion of apps can be successfully attacked, confirming the existence, feasibility, and even prevalence of real-world attacks. 
\X2 White-box app has much higher exploitable rates than black-box app. This disparity comes from the fact that attackers can access the code within white-box apps, allowing us to judge if there is a vulnerability and providing insights into potential exploits and escape approaches and so increasing the likelihood of successful exploitation. Black-box apps, on the other hand, lack code visibility, making vulnerabilities and their exploitation mostly unknown, resulting in inherent difficulty and, as a result, lower rates of successful exploitation. 
\X3
A notable number of app developers exhibit insufficient security awareness. Only two apps incorporate some form of security protection, 
whereas four of the successfully exploited apps can be escalate to root privileges (2 are originally rooted, and 2 can escalate privileges to root through improper SUID~\cite{9936713} settings). This indicates that, amidst rapid development, the security of LLM-integrated apps has been somewhat neglected and needs improvement.
\X4 Such apps are in a phase of rapid development, and some are merely experimental. For instance, the ``Runtime Exception'' column in the table reflects the developers' negligence toward the app's usability and maintenance. This indirectly indicates a lack of emphasis on security by app developers as well.

\subsection{Hazard Analysis of RCE Vulnerabilities} \label{sec:Measure:attack}

In this section, we conduct a comprehensive analysis of the hazards caused by these RCE vulnerabilities.

\subsubsection{Hazards to App Hosts}
When an attacker successfully achieves RCE on the app host through prompt injection, it signifies that the attacker gains the ability to execute arbitrary code on the app host, opening the door to various attack vectors. In the following, we present several practical attack vectors for consideration.

\vspace {3pt}\noindent\textbf{Privacy leakage.} 
There is a lot of sensitive information stored in app host servers that should not be visible to the public, but attackers can use RCE to access this sensitive information. In the era of LLMs, the forms of sensitive information have become more diverse. In addition to traditional sensitive information such as SSH configuration, \texttt{/etc/passwd}, kernel version, network topology, and source code of black-box applications, new types of sensitive information have also emerged. For instance, most of apps keep their OpenAI API keys in the environment variables of the host server. Thus, attackers can execute the \texttt{env} command to extract these variables and steal the keys for free. Furthermore, prompts embedded in the source code might also contain sensitive information protected by copyright, e.g., intellectual property.

\vspace {3pt}\noindent\textbf{Backdoor injection.} 
After the attacker gains the capability to execute arbitrary commands remotely via prompts, it can inject backdoors into the app host server, thus gaining and keeping control over the server. For example, the attacker can create a reverse shell script on their VPS, using prompt injection to let the server execute the \texttt{curl} command and download the backdoor script from the VPS. Afterward, by leveraging prompt injection once more, the attacker can execute the backdoor script, thereby attaining a reversed shell to get full control over the server.

\vspace {3pt}\noindent\textbf{Privilege escalation.} 
After successfully using the reverse shell technique to take over the host server, the attacker can potentially change SUID or SGID to escalate privilege. Alternatively, it can exploit kernel vulnerabilities corresponding to the leaked kernel version mentioned above, thus achieving higher execution privilege.
Additionally, the attacker may modify sensitive files that are usually only available to root users. 

\subsubsection{Hazards to Benign App Users}
Since these web apps provide services to the public, the hazards of RCE vulnerabilities can further extend to benign app users. 
Hence we propose several practical attacks, threatening benign app users but without their awareness.

\vspace {3pt}\noindent\textbf{Output hijacking attack.} Previous attacks on chatbots aiming to manipulate the model's output, i.e., jailbreaking, were limited to single sessions and could not affect other users. However, with the RCE, cross-session attacks have become feasible, enabling attackers to compromise other user sessions.
Attackers exploiting RCE vulnerabilities can manipulate the model's output, compromising service availability and disseminating disinformation or phishing attacks. As illustrated in Figure 10, attackers can hijack the app's original output, which is intended to provide details about a CSV file, and replace it with a message like ``I don’t know!''
This undermines user trust and compromises the app's functionality.
We propose proof-of-concept attacks by setting up an app locally. Upon achieving RCE, the attacker changes the output of the app by modifying the main file of the app (``original\_app.py'') as shown in Figure \ref{fig:attack1_diff}. This allows it to entirely control the app's output, inserting offensive words, disinformation or even phishing links, significantly misleading app users. 

\begin{figure}
	\centering
 % \vspace{-10pt}
	%\setlength{\abovecaptionskip}{0pt}
	\setlength{\belowcaptionskip}{0pt}
	\includegraphics[width=0.9\columnwidth]{figures/attack1_diff_new.pdf}
	\vspace{-1mm}
	\caption{Output Hijacking Attack: Diff between malicious and original file.} 
	\label{fig:attack1_diff}
\end{figure}

\vspace {3pt}\noindent\textbf{User data stealing attack.} 
Upon achieving RCE, attackers can exfiltrate users' private data by modifying the source code, including stealing LLM API keys, user-provided prompts, and user-uploaded files. These data may encompass sensitive information, intellectual property, and personal assets. For instance, we illustrate how to steal a user's API key. Numerous applications necessitate users to supply their own LLM API keys to access services. This undoubtedly provides attackers with a new and hard-to-detect attack surface.
In Figure \ref{fig:new_attack2}, 
the attacker modifies the code such that once the app receives an API key entered by the user, it logs and transmits the key to the attacker. Alarmingly, this attack remains undetected from the victim's perspective, as the app performs normal functionalities as expected. This enables the attacker to covertly transform a benign app into a malicious one.
To avoid disrupting the functionalities of public apps, we deploy a real-world white-box app locally and successfully implement this attack. Once an attacker achieves RCE, it modifies the main code of the app (``original\_app.py'') as shown in Figure \ref{fig:attack2_diff}. 
This attack can be extended to steal other privacy. 
\begin{figure}
	\centering
	%\setlength{\abovecaptionskip}{0pt}
	\setlength{\belowcaptionskip}{0pt}
	\includegraphics[width=0.9\columnwidth]{figures/attack2_diff_new.pdf}
	\caption{API Key Stealing Attack: Diff between malicious and original file.}
	\label{fig:attack2_diff}
	\vspace{-5mm}
\end{figure}

\vspace {3pt}\noindent\textbf{Phishing attacks.} 
The phishing attack is a classic attack that can be conducted after achieving RCE. Typically, phishing attack allows attackers to trick users into exposing themselves or their organizations to cybercrime (\eg sensitive information leakage, malware distribution)~\cite{phishing_ibm}. Attackers can manipulate app pages by modifying the code to include phishing attack entry points, exploiting users based on their trust in the app. This enables attackers to launch phishing attacks on benign app users. 
Figure \ref{fig:phishing_attack} illustrates a phishing attack. In this scenario, the attacker modifies the app's functionality and web page, adding persuasive text to induce users to download and open a file purportedly containing a ``secret token'' (which is actually malware). Users cannot use the app normally unless they comply with the attacker's demands. Given their trust in the app, users are likely to download and open the malware in search of the secret token. Once opened, the malware compromises the user's system.
We won't include code samples here because there are many ways to create phishing pages and the serious potential harm caused by such attacks. 
Other phishing attack types are feasible, such as forging websites' login pages to trick people into logging in with their private credentials. 

\section{Related Works}

In this section, we list works on the same topic as ours. \cref{sec:preliminary} contains works on different topics that our explanation depends on, we omit their details for simplicity here.

In our point of view, the research of activation sparsity in MLP modules starts from the discovery of the relation between MLP and knowledge gained during training. \citet{mlp_as_database} first rewrite MLPs in Transformers into an unnormalized attention mechanism where queries are inputs to the MLP block while keys and values are provided by the first and second weight matrices instead of inputs. So MLP blocks are key-value memories. 
\citet{knowledge_neurons} push forward by detecting how each key-value pair is related to each question exploiting activation magnitudes as well as their gradients, and providing a method to surgically manipulate answers for individual questions in Q\&A tasks. These works reorient research attention back to MLPs, which are previously shadowed by self-attention.

Recently, comprehensive experiments conducted by \citet{observation} demonstrate activation sparsity in MLPs is a prevailing phenomenon in various architectures and on various CV and NLP tasks. 
\citet{observation} also eliminate alternative explanations and attribute activation sparsity solely to training dynamics. 
The authors explain the sparsity theoretically with initialization and by calculating gradients, but their explanation is restricted to the last layer and the first step because in later steps the independence between weights and samples required by the explanation is broken. 
They also discover that some activation functions, such as $\tanh$, hinder the sparsity \citep[see][Fig B.3(c)]{observation}, but did not elaborate on it. 
Compared to their explanations, our explanation applies to all layers and large steps, and accounts for the activation functions' critical role in activation sparsity.

Following empirical discoveries by \citet{observation}, \citet{sharpness_aware} show that sharpness-aware (SA) optimization has a stronger bias toward activation sparsity. 
They explain theoretically by calculating gradients and finding that SA optimization imposes in gradients a component toward reducing norms of activations. However, their explanation is still conducted on shallow 2-layer pure MLPs and requires SA optimization, which is not included in standard training practice. Nevertheless, this explanation hints at the role of flatness in the emergence of activation sparsity. Inspired by them, we explain \emph{deep} networks trained by standard SGD or other stochastic trainers by substituting flat minima for SA optimization.

A more recent work by \citet{from_noises} holds a point that sparsity is a resistance to noises. However, noises are manually imposed and not included in standard data augmentations. We substitute gradient noise from SGD or other stochastic optimizers for them.
\citet{large_step} prove sparsity on 2-layer diagonal MLPs and conjecture similar things to happen in more general networks. Both works hint at the relation between noises (Gaussian sample noises and stochastic gradient noises) and activation sparsity, also leading to the flatness bias of stochastic optimization.

\citet{adversarial_of_moe} study the adversarial robustness of Mixture of Experts (MoE) models brought by architecture-imposed sparsity. They inspire us to relate sparsity with adversarial robustness, although we do it reversely. It is the major inspiration for our results.

To sum up, existing discoveries hint at the relation between activation sparsity and noises, flatness and activation functions but they are still restricted to shallow layers, small steps and special training. Inspired by them and filling their gaps, our explanation applies to deep networks and large training steps, and sticks to standard training procedures.

Although not devoting much to explaining the emergence of activation sparsity in CNNs, \citet{exploit_sparsity_in_CNN} boost activation sparsity through Hoyer regularization\citep{hoyer} and a new activation function FATReLU that uses dynamic thresholds between activation and deactivation. They also design algorithms to exploit this sparsity, leading to $\ge 1.75\mathrm{x}$ speedup in CNN's inference. Compared to their sparsity encouragement method that requires well-designed procedures to select thresholds, hyperparameters for our theoretically induced modifications can be easily selected. The discontinuity of FATReLU also bothers training from scratch\citep{exploit_sparsity_in_CNN}, while we recommend applying our modifications from scratch to enjoy better sparsity and additionally smaller \emph{training} costs. Regarding exploitation, we consider it out of the manuscript's scope.
\citet{L1_sparsity} encourages activation sparsity in CNN by explicit $L_1$ regularization. We intend to investigate the emergence of activation sparsity from implicit regularization as demonstrated by \citet{observation}, so we solely rely on implicit regularization boosted by modifications. Nevertheless, our methods are architecturally orthogonal and we believe applying both together can further boost activation sparsity.

There are other works that are not devoted to activation sparsity but are related. \citet{sparse_symbol} formulate, with Shapley value, and prove that there are sparse ``symbols'' as groups of patches that are the only major contributors to the output of any well-trained and masking-robust AIs. They provide a sparsity independent of training dynamics. Their theory focuses on symbols and sparsity in inputs, which is inherently different from ours.

In Primer \citep{primer}, several architectural changes given by architecture searching include a new activation function Squared-ReLU. In this work, we induce a similar squared $\relu$ activation but with the non-zero part shifted left and use it to guide the search for flat minima and gradient/activation sparsity. \cite{primer} demonstrate impressive improvements of Squared-ReLU in both ablation and addition experiments, and our work provides a potential explanation for this improvement.

\section{Discussion}
\label{sec:discuss}

\vspace {3pt}\noindent\textbf{Response from developers.} We have reported all vulnerabilities to the framework maintainers and app developers. After multiple rounds of communication, we have received acknowledgments and bug bounty from several developers or vendors and have summarized the current attitudes of developers toward these vulnerabilities within the LLM ecosystem. 

There are 8 out of 11 vulnerable frameworks (\eg PandasAI) that promptly respond to the issues we raise on GitHub ($\approx$1-2 days). After confirming the vulnerabilities, 
although developers pledge to address vulnerabilities promptly, the patching cycle often proves to be long. This underscores developers' attention to RCE vulnerabilities while highlighting the inherent complexity in achieving comprehensive resolutions for these issues.
Therefore, it can be anticipated that this type of RCE vulnerability may continue to persist in the short-term future.
In contrast, the response of app developers is relatively slow considering the number of participants and activity. Seven vulnerability reports we submitted have not received response yet.
Regarding the vulnerability reports with responses, the average response time is within two to three days. It is worth mentioning that some app developers responded and implemented mitigation measures within two hours.

After our disclosure, these kind of RCE vulnerabilities receive sufficient attention from LLM framework developers. Some frameworks (e.g., LangChain, LlamaIndex) and app deployment platforms (e.g., Streamlit) have raised alarms for users being cautious to use these code execution APIs.

\vspace {3pt}\noindent\textbf{Potential mitigation.} Based on the analysis results, we propose three measures to mitigate the risks.
\X1 Permission Management. Framework and app developers should follow the \emph{principle of least privilege}~\cite{saltzer1975protection}, setting users' privileges to the lowest possible level. For example, disable the permission to read and write the app and its system files or partitions. The execution of privileged programs with SUID and other sensitive commands should also be disabled.
\X2 Environment Isolation. 
Developers can put appropriate limitations on the processes executing LLM code by using tools like ``seccomp'' and ``setrlimit'' for process isolation and resource isolation. Alternatively, they can utilize secure-enhanced versions of Python interpreters like Pypy and IronPython, which provide process-level sandboxing capabilities. Meanwhile, following the exposure of such RCE vulnerabilities, some LLM ecosystem-specific cloud sandboxes (\eg, e2b~\cite{e2b}) have also been developed. These sandboxes host the code execution functionality in a cloud environment, thereby preventing malicious code directly affect the server. Finally, as mentioned previously, app developers can utilize tools like Pyodide to embed the code execution into browsers, allowing the code execution to run on the client-side rather than the server-side. \X3 Prompt Analysis. Some research has also attempted possible defenses at the prompt level. For example, Liu et al.~\cite{liu2023prompt} introduced detection-based defenses to check if the original functionality of prompts has been compromised. Other work proposed methods to inspect the intention of prompts, aiming to filter out malicious prompts~\cite{zeng2024autodefense}. 
Regardless of the mitigation, developers have to balance usability, efficiency, and security to choose the most suitable solution.
Thus, ensuring security without compromising functionality integrity remains a challenge.

\vspace {3pt}\noindent\textbf{Future work.}
\X1 Multiple language support. Currently, \tool{} is only available for detecting RCE vulnerabilities within LLM-integrated frameworks written in Python. However, there are some open-source frameworks built in other languages, such as Chidori~\cite{Chidori} in Rust and Axflow~\cite{Axflow} in TypeScript. In the future, we intend to make \tool{} cover more languages, revealing more vulnerabilities within multi-language LLM-integrated frameworks.
\X2 Multiple vulnerability type support. Currently, \tool{} is only built to detect RCE vulnerabilities within LLM-integrated frameworks, and explore the hazards caused by RCE. 
In the future, we are interested in expanding our detection capabilities to cover a broader range of vulnerability types and to test in real-world scenarios.

\section{Conclusion}
We propose an efficient approach \tool{} to detect and validate RCE vulnerabilities in LLM-integrated frameworks and apps. 
\tool proceeds in three steps, where it first employs static analysis to detect RCE vulnerabilities existing in frameworks, then collects public LLM-integrated apps via white-box and black-box methods, and last launches a novel prompt attack to achieve RCE in these apps. 
\tool{} successfully identifies 20 vulnerabilities across 11 frameworks, obtaining 13 CVEs. 
In the context of automated app testing, \tool{} detects 17 vulnerable apps, with 16 instances achieving RCE. We provide detailed measurements for the mentioned vulnerabilities. 
Moreover, we perform a detailed hazard analysis of RCE vulnerabilities from the perspective of app hosts and benign app users. 
By exploiting these RCE vulnerabilities, we further develop new practical attacks that endanger both app hosts and benign app users. Additionally, we introduce practical mitigations for these RCE attacks.

\section*{acknowledgements}
We thank all the anonymous reviewers for their constructive feedback. The IIE authors are supported in part by NSFC (92270204), CAS Project for Young Scientists in Basic Research (Grant No. YSBR-118), Youth Innovation Promotion Association CAS and Beijing Nova Program.
% \newpage


\bibliographystyle{ACM-Reference-Format}
\bibliography{ref}

\section*{Appendix}\label{sec:appendix}
\renewcommand{\thesubsection}{\Alph{subsection}}

\subsection{LLM Hallucination in the Real World}
During our testing of this app, we discovered a hallucination issue as shown in Figure~\ref{fig:hallucination}. We can observe that when we requested the app to output lines 5-10 from ``test.py'', the output was very peculiar, which raised our alertness. Further communication with the developer and code review confirmed that the issue was indeed caused by hallucination. The app hallucinated when we asked it to perform code execution, generating seemingly correct outputs without actually having the ability to execute code.
\begin{figure}[ht]
	\centering
	%\setlength{\abovecaptionskip}{0pt}
	\setlength{\belowcaptionskip}{0pt}
	\includegraphics[width=1.0\columnwidth]{figures/hallucination.png}
	\caption{LLM hallucination in a real-world app.} 
	\label{fig:hallucination}
	\vspace{-3mm}
\end{figure}


\subsection{Practical Real-World Attacks Against Other App Users}

% asdfasdlfjasdlfja \ref{fig:new_attack1}
% \lipsum[1-5]
Figure~\ref{fig:new_attack1} illustrates an instance of output hijacking in an real-world scenario. The attacker initiates by tampering with the application's source code, compelling the application to generate a specific output. This manipulation disrupts the normal experience of benign users, causing interference and potential harm.
\begin{figure}[ht]
	\centering
	%\setlength{\abovecaptionskip}{0pt}
	\setlength{\belowcaptionskip}{0pt}
	\includegraphics[width=1.0\linewidth]{figures/new_attack1.pdf}
	\caption{Output hijacking attack} 
	\label{fig:new_attack1}
	\vspace{-3mm}
\end{figure}

Figure~\ref{fig:new_attack2} illustrates an instance of OpenAI API Key stealing attack in an real-world scenario. 
The attacker first modifies the app's source code to enable automatic recording of users' OpenAI API keys after they input their keys. When an app user enters their OpenAI API key while using the app, the key is captured by the attacker without the victim being aware of the attack. This poses a significant threat. Additionally, the attacker can also steal other user information such as uploaded files and user prompts.
\begin{figure}[ht]
	\centering
	%\setlength{\abovecaptionskip}{0pt}
	\setlength{\belowcaptionskip}{0pt}
	\includegraphics[width=1.0\columnwidth]{figures/new_attack2.pdf}
	\caption{API key stealing attack} 
	\label{fig:new_attack2}
	\vspace{-3mm}
\end{figure}

Figure~\ref{fig:phishing_attack} illustrates an instance of phishing attack in an real-world scenario. For example, the attacker wants to trick the user to download and open its malware, it modifies the code first. Now here comes an app user, the modified app says every user should enter a secret token first to start using this app. and the secret token can be obtained by downloading the provided files (and actually the file is attacker’s malware). If the user trust the app, he will download the file and try to open it. Thus, the attacker tricks the user into opening its malware.
\begin{figure}[H]
	\centering
	\setlength{\belowcaptionskip}{0pt}
	\includegraphics[width=1.0\columnwidth]{figures/phishing2.pdf}
	\caption{Phishing attack} 
	\label{fig:phishing_attack}
	\vspace{-3mm}
\end{figure}


\subsection{App Search Details}
\label{sec:appendix:C}
\textbf{Vulnerable APIs used in app searching.} 
\begin{itemize}
    \item LangChain: \texttt{create\_csv\_agent, create\_pandas\_dataframe\_agent, create\_spark\_dataframe\_agent, PALChain}.
    \item PandasAI: \texttt{PandasAI}.
    \item LlamaIndex: \texttt{PandasQueryEngine}.
\end{itemize} 
\begin{table}[H]
\caption{Characteristics used to search black-box apps (number contains overlap between each characteristic.)}
\label{tab:cha}
\begin{tabular}{ll}
\toprule
\textbf{Characteristic (keywords)} & \textbf{\#Tested App} \\ \midrule
data analysis           & 16              \\
chat with ...           & 5               \\
csv                     & 6               \\
interperter             & 2               \\
math                    & 1               \\
data science            & 14              \\
langchain               & 5               \\
agent                   & 7               \\ \bottomrule
\end{tabular}
\end{table}

\end{document}
\endinput
%%
%% End of file `sample-sigconf.tex'.
