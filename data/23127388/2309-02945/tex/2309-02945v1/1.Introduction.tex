\section{Introduction}
Data is ubiquitous in the physical world around us. A person may desire to understand more about some passive referent, or to keep informed of the state of some actively updating referent \cite{fleckRagRugToolkitSituated2022}. With augmented reality (AR) head-mounted displays (HMD), it is possible to decrease the level of spatial indirection between the referent and its data, such that it is displayed close to or even embedded on top of each other \cite{willettEmbeddedDataRepresentations2017}. Thus, many works on situated visualization have sought to minimize this indirection, whether it be to overlay AR visualizations directly on top of grocery store products \cite{elsayedSituatedAnalytics2015}, display information about a building next to it \cite{reitmayrCollaborativeAugmentedReality2004}, or show temporal data next to temperature sensors in a building \cite{fleckRagRugToolkitSituated2022}.

In the physical world, we are bound by physical constraints. In particular, the design of situated visualization is influenced by its navigational requirements  \cite{leeDesignPatternsSituated2023}. For example, if the physical referents are spread across a large area, the use of embedded visualizations may be problematic due to the physical and mental effort required to locate and navigate to the referents. Thus, using a \textit{many-to-one} view may help consolidate such spatially distributed information into a singular visual representation \cite{leeDesignPatternsSituated2023}.

In traditional desktop computing, visualization dashboards are vital in their ability to also consolidate large amounts of disparate information into a format that provides an overview of the data. Dashboards are particularly useful for hiding the complexities of the logical world from end-users, making data easily accessible without the end-user needing to know where and how the data comes from.

Therefore, we propose the concept of AR-based \textit{situated dashboards}. While viewing the data in its exact physical context can be useful, situated dashboards may accommodate situations where needing to be in said physical context is too cumbersome or impractical. For example, an AR-situated dashboard might provide a factory manager with the status of all operations on the floor at any moment's notice. The application could then help the manager navigate to a problematic areas of the factory (e.g., using situated AR navigational instructions \cite{reitmayrCollaborativeAugmentedReality2004,mulloniHandheldAugmentedReality2011}). The dashboard may then transition into an embedded view \cite{willettEmbeddedDataRepresentations2017} for the manager to engage in problem-solving within the physical context.

While traditional visualization dashboards are commonplace, and are technically already in use in many situated contexts, there has been little to no exploration on the use of situated dashboards in AR, which is now arguably the de-facto standard for situated visualization \cite{bressaWhatSituationSituated2022,shinRealitySituationSurvey2023}. The possibilities of situated dashboards are vast, and there is no clear definition or approach for how they can be designed, created, or even evaluated. In this position paper, we establish a preliminary understanding of situated dashboards through a set of six interviews with researchers in both situated visualization and AR. Our interviews focus on understanding experts' perception about situated dashboards, design considerations, and potential challenges of designing and authoring situated dashboards.


% Ben's story:
% - Situated visualisation involves placing data in spatial proximity to its physical context
% - Depending on the information shown in the visualisation, it may aid in some form of analytical task, i.e. situated analytics
% - Much research has explored use cases for situated analytics (elsayed's original paper on supermarkets, fleck's ragrug with mopop, basically look at shin's survey paper for references on this)
% - Such research typically associates data views with physical referents in a one-to-one cardinality (i.e. each referent has its own view associated with it)
%     - e.g. visualisations of a particular grocery store product directly on it
% - However, as is common for analytical tasks, obtaining an overview of the data may be necessary, either to identify key measures at a glance (e.g. in a management context), or to provide a "one stop shop" of the data without needing to cross-reference multiple views (useful when referents/views are physically spread apart, necessitating movement)
% - We refer to such overviews of the data as "situated dashboards", drawing upon the same term as commonly used in 2D environments
% - However, the possibilities of situated dashboards are vast, with no clear existing definition nor approach to designing and creating them
% - To build a preliminary understanding of this, we interview n researchers in both AR and situated analytics to identify key challenges and opportunities for situated dashboards
% - We contribute:
%     - Insights from interviews with researchers
%     - Discussion on key challenges of situated dashboards
%     - Immediate research directions for where the concept can be taken