\section{Interview Methodology}%Study Design
A main contribution of our work is the results from semi-structured expert interviews of six AR and/or visualization researchers. The interviews aim to characterize challenges and opportunities for situated dashboard design. A focus group approach was not considered due to timeline constraints.

The participants were recruited through convenience sampling, and had varying levels of expertise in AR, data visualization, and situated visualization. Four of the six participants have published at least one paper on situated visualization/analytics.
Three participants were interviewed in person, and the other three were interviewed remotely. The session started with the participants describing their perception of situated dashboards. They were then tasked with ideating an AR HMD based situated dashboard for their typical workday at the office. At this time, the in-person participants were provided with pen and paper and the remote participants with Excalidraw board. While most participants used these, one remote participant chose not to sketch during the session but later emailed us a sketch. Another remote participant only described their ideas verbally. Throughout this process, participants were encouraged to articulate their thoughts regarding various aspects of their designs, including features, context, interactions, user experience, and potential implementation challenges. During this time the participants were also asked to reflect on their past experiences and talk about workflows for implementing situated visualizations. 
Following a reflexive thematic analysis method \cite{braun2019reflecting}, we collected, transcribed, and analyzed the interview data.

%Three participants were interviewed in person, and the other three were interviewed remotely. After collecting demographic information, participants were asked to design and sketch what they consider to be an AR situated dashboard, which would be used for a regular workday at their office. This design activity however was conducted in a limited manner for remote participants who refused to do so as they felt more comfortable simply talking about their ideas. We then asked the participants about their experiences, workflows, and challenges when working with situated visualization and/or AR in general, and asked what features they might expect from a situated dashboard tool. Each interview took between 50--60 minutes. 

%We identified and extracted all interesting discussion topics from the interviews. After internal discussions between the authors, the topics were then grouped into five main considerations of situated dashboards, presented as follows.

% \subsection{Procedure}
% %The study consisted of a semi-structured interview session and a design activity.
% The interviews were performed in-person (3) and remotely (3) via video/audio conferencing tools, audio-recorded, and transcribed for analysis.  After collecting  demographic information, participants were asked to freely sketch situated dashboard experiences facilitated by AR HMD for a regular workday in the office. Such design activity was conducted in a limited manner for remote sessions where the participants felt comfortable talking about their ideas instead of sketching over conferencing tool. In addition, one of the participants sent us a sketch after the study. We also asked them about their background experiences, workflows, challenges, problems during the development, and expected features of a dashboard tool. Each interview took about 50-60 minutes.  %received no compensation.
%For the design activity, the participants were asked to ideate and sketch situated dashboard experience facilitated by AR HMD for a regular workday in the office. During this time the participants were also encouraged to talk about the various interactions, design considerations and potential challenges associated with the designs they came up with.
%In the follow up interview session, we also encouraged the participants to reflect on their past experiences and relate it to situated dashboards. 

%Out of the 6 studies, 3 were conducted online and 3 in-person. We acknowledge that the design activity was conducted in a limited manner for 2 online sessions where the participants felt comfortable talking about their ideas instead of sketching over zoom. Though one of the participants sent us a sketch after the study was concluded.

% \subsection{Data Analysis}
% Following a reflexive thematic analysis method \cite{braun2019reflecting}, we collect and analyze the interview data. We discussed and identified considerations, which were then grouped into five considerations discussed as follows.
%Constructivist Grounded Theory method~\cite{charmaz2014constructing}, 
%The data consisted of interview transcripts and observation data of how participants acted out the interactions with situated dashboards. We used reflexive thematic analysis method for data analysis.
