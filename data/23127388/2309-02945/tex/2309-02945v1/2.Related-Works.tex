\section{Related Work}

%Many authoring AR studies have been proposed in the HCI community focusing on simulating large-scale subjects (e.g. \cite{CSpace2020, constructingBuildingMixedReality2023}), context-aware experiences (e.g. ), portable interfaces, and proxemic-gestural interactions.
%Our scope in this work is mainly to explore prototyping dashboard design, situated visualization, and authoring immersive visualization tools.
Since our study is focused on assessing dashboard design for situated visualization, we surveyed the literature on: 
(1)  dashboard design considerations and (2) situated visualizations. 
%and (3) authoring tools using augmented reality in situated scenarios.

\subsection{Dashboard Design}
\label{sec:dashboardDesign}
Dashboards are broadly used in business intelligence to support users in analyzing complex data sets through multiple views~\cite{multipleViewsBaldonado2000} and the coordination between them~\cite{scherr2008multiple}.  Dashboard design guidelines emerged to advise visual perception, information load, and interactions (e.g.~\cite{few2006information,few2007dashboard,rasmussen2009business,kitchin2015knowing,BUGWANDEEN2019,QualDash2020,AnsweingChallengesEmergencyResponses2022}). Popular visualization tools like Tableau and PowerBI contain huge galleries of templates in order to generate dashboards. However, such systems are challenging to use for non-experts. On the other hand, researchers make efforts to provide authoring and visualization recommendations tools (e.g.~\cite{GuidedMultiView2021,multivision2022,medley2023}). 
Recently Bach et al.~\cite{Bach2023DashboardDesignPatterns} surveyed dashboard designs and detailed 48 design patterns. They mapped solutions: \textit{data abstraction, screenspace organizing, grouping of elements, relations encoding and the interaction or personalization} in the dashboard design process. Despite those studies being focused on conventional displays, we considered those solutions to prepare the questions.


\subsection{Situated Visualization}

Willett et al.~\cite{willettEmbeddedDataRepresentations2017} defined \textit{situated visualizations} as a situated data representation in a relevant location where the representations are connected to physical referents. When referents are not accessible, referents can be represented using scaled 3D models (\textit{proxies})~\cite{proxSituated2023}. 
%The research community considers the importance of \textit{situatedness} and \textit{visualization} in its studies and has a common interest in bringing closer visualizations into people's everyday environments~\cite{bressaWhatSituationSituated2022}. 
Bressa et al.~\cite{bressaWhatSituationSituated2022} surveyed studies and proposed perspectives to categorize the concept of situatedness: (1) \textit{space} puts emphasis on the spatial organization and relationship between the physical environment and visualizations; (2) \textit{time} focuses on the distance in time between the gathered data and its presentation; (3) \textit{place} considers the meaningful location where users act; (4) \textit{activity} refers to the human activities that designers need to consider with visualizations being appropriated to contexts; and (5) \textit{community} emphasizes in the audience, i.e., designers and developers.
%who are designers of visualizations and look for sharing and supporting issues. 
Each perspective opens challenges and motives our intention to design dashboards. 

Recently, \textit{active proxy dashboard} was proposed to analyze abstract visualizations from \textit{proxies} through tangible interactions~\cite{ActiveProxy2023}. The main idea is to build binding events between proxies and data representations, allowing analysts to interact directly with proxies and visualizations that are displayed on conventional screens. Although the advantages of analyzing inaccessible referents and using powerful known displays, limitations about \textit{place} and \textit{activity} perspectives emerged. The context-dependent from human activities relies on context recreation difficulties. We believe that authoring tools will be closer to creating dashboards context-independent.

Furthermore, multiple studies seek to standardize properties and to establish guidelines that mitigate the challenges of multiple situated views. Batch et al.~\cite{ViewManagament2023} evaluated different ways of view management and identified properties to consider in future implementations. 
%studied the use of a shadowbox, world-in-miniature, and guided-tour to evaluate different ways of view management. In addition, their study identified properties to consider in future implementations.
More formally, Lee et al.~\cite{leeDesignPatternsSituated2023} identified patterns, dimensions, and guidelines on how to investigate situated visualization.  


%(Benjamin) To add: Why use situated dashboards when proxsituated visualisations could work better?
%Reviewer 2 might say that while a situated dashboard can provide an overview of the data which minimises the need for physical navigation to said referents, a proxsituated visualisation can transport the user to a virtual proxy representation of said referent without any physical locomotion.
%The answer might possibly be that situated dashboards don't rely on an exact digital twin of the physical environment, as it is simply an adjustment of the level of spatial indirection rather than strictly a recreation of the physical context. Thus, it is potentially easier to author and, in some ways, easier to use.
%Another answer is that proxsituated dashboards, without some level of scaling, still limit the user into seeing information about one physical context at a time, whereas a dashboard still provides an overview of multiple physical contexts. An interesting observation (I think mentioned in the interviews) is that a shrunk proxy of a situated visualisation can be interpreted as a form of dashboard. This would perhaps go in the discussion regarding the dimensionality of dashboards (i.e. 2D dashboards with traditional visualisations, or 3D dashboards using world-in-minuature like representations).

%\subsection{Situated Authoring Tools}

%Authoring tools have been proposed to facilitate the creation of visualizations. To achieve good authoring systems, the HCI community has proposed programmatic toolkits focused on immersive analytics (e.g.~\cite{DXR2019,cordeilIATKImmersiveAnalytics2019,butcherVRIAWebBasedFramework2021,xvcollab2022}). Although these tools do not envision extending to situated scenarios, many are used like a foundation to build more sophisticated author features. For example, AR authoring toolkits like MARVisT~\cite{marvist2020} can serve as a a mobile tool to create glyph visualizations for non-experts users. Similarly, PapARVis~\cite{paparvis2020} focused on authoring static and virtual visualizations, where designers can create the representations in a UI tool, and users can view such visualizations. It reduces the switching between platforms. Likewise, MIRIA~\cite{miria2021} supports spatio-temporal visualizations and interactions using  HoloLens. The authors report requirements and considerations to in-situ scenarios from expert interviews. In contrast, RagRug~\cite{fleckRagRugToolkitSituated2022}  was developed to support situated scenarios explicitly. It offers an interesting architecture based on IoT streaming and reactive programming, and provides several use cases for illustration. One of its case studies was to reproduce a previous study~\cite{stream2021}, supplanting its architecture entirely and providing a more intuitive implementation. A more recent study, iARVis~\cite{iarvis2023}, proposed a tool focused on non-experts authoring based on automatic positioning and interactions of widgets, using 2D visualizations. 


%Here, we expand previous studies toward dashboard design for situated visualization. We follow the exploration of dashboard design,  analyzing whether those patterns are a viable design choice for studies in situated visualizations using augmented reality.