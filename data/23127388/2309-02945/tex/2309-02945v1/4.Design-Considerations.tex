\section{Design Considerations of Situated Dashboards}
Our participants' perceptions of situated dashboards and their design considerations were fragmented, with diverse and sometimes conflicting views. Motivated by solutions proposed in the literature (Section~\ref{sec:dashboardDesign}), we discuss five main considerations:
(1) What content do situated dashboards display;
(2) What do they look like;
(3) Where are they situated;
(4) What interactions do they facilitate; and
(5) How can they be customized?

% Participants’ perception about situated dashboards appear to be fragmented, with diverse and sometimes conflicting views about - (1) What should be displayed? \textit{(content)}, (2) What should it look like? \textit{(visual features)}, 3) Where should it be situated? \textit{(situatedness)}, 4) What sort of interaction should it have \textit{(interaction)}, and 5) What sort of customization should it support to facilitate personalization? \textit{(customization)}

% In this section, we discuss these aspects of situated dashboards as design considerations.
%In this section, we show the findings found as design considerations.
\subsection{Content of Situated Dashboards}
When asked what they thought ``dashboards'' meant, most participants associated the term with data visualization. According to \beck{}, \fleck{}, \sr{}, and \kadek{}, a dashboard contains information from multiple sources that serves as an overview or summary. \aimee{} identified this overview as the differentiating factor between situated visualization and a situated dashboard.
However, they also defined a dashboard not by its content but by its ability to control something. Thus, a dashboard is \textit{``a surface where we will be able to control something.''} By this definition, seeing the data alone is not sufficient to be considered a dashboard.

For ``situated dashboards'', \ap{} preferred a narrower definition. In their words: \textit{``a dashboard needs to have more than one visualization about the same group of elements of physical objects. So when I hear a situated dashboard, it would mean that there are at least two visualizations about a specific object or a specific part of the physical workspace.''}
\sr{} provided some considerations about the selection of content for a situated dashboard. According to them, the information shown on a situated dashboard depends on: (1) where the dashboard is placed in the environment; (2) who the user is; and (3) what information is important to the user in a target situation. Another important consideration, as pointed out by \sr{}, is the privacy of the information presented on the dashboard, particularly if the situated dashboard is placed in a shared context.

\subsection{Appearance of Situated Dashboards}
When asked to design a situated dashboard,  all participants laid out their visualizations on a rectangular 2D space (Figure~\ref{fig:teaser}). \fleck{}, \aimee{}, and \ap{} used a single panel to display all information. \sr{} instead used multiple panels, with each containing a group of relevant information. These panels were then scattered near their referents. \beck{} designed for both scenarios. When using a single panel for displaying multiple data sources, \beck{} designed their dashboard in such a way that the layout of the dashboard changes dynamically to focus on the information that is most relevant to the location or task of the user.

\ap{} went even further by creating associations between individual visualizations on their dashboard and physical referents through visual links (red lines on \ap{}'s sketch in Figure~\ref{fig:teaser}). Their intention with this was to \textit{``to see the data streaming from the physical referent to the dashboard to the visualization itself.''}

%\begin{figure}[t]
%\centering
%\includegraphics[width=\linewidth]{figures/APdrawing.pdf}
%\vspace{-9mm}
%\caption{Sketch of situated dashboards by \ap: (a) using the slider on the energy consumption visualization of the situated dashboard to move in time and see the level of energy consumption through colors of the situated visualization [in the picture represented in red]. The red line links the visualization of the dashboard to situated visualization. (b) moving slider to see coffee consumption represented as number of cups (brown circles) next to the person (c) use dashboard to slide through camera footage and watch it play out as situated visualization and also in the video screen of the camera. 
%}
%\label{fig:Prouzeau1}
%\end{figure}

\subsection{Situatedness of the Dashboard}
According to \ap{}, \textit{``you don't situate the dashboard, you situate the information from the dashboard.''} All participants agreed that a situated dashboard should be in a specific context where it provides relevant and actionable information to users. As an example, \ap{} suggested that situated dashboards could be placed in the environment in order to playback time series data as situated visualizations.

\kadek{} pointed out that the contents of a dashboard could be situated either near a physical referent, or near a tangible or virtual proxy of said referent \cite{proxSituated2023}. They gave the example of a factory manager having a situated dashboard which was positioned near a tangible or virtual model of factory machinery in their office.
\aimee{} cautioned against the embedding of dashboards directly onto referents, stating \textit{``I would have a hard time situating a dashboard into a very specific object, because if a dashboard is a lot of info why would I want to stick that to one object, unless that object was related to all the information.''} As a possible workaround, \beck{} suggested that the dashboard's layout could dynamically change according to the user's context in order to minimize clutter. For example, during lunchtime, food-related data would mainly be shown on the dashboard with everything else being minimized. During work hours however, the availability of co-workers would be shown instead.

Additionally, as \ap{} pointed out, not all data inherently has a spatial relationship with a physical referent, and thus it is not always straightforward to decide where to situate such data. In these cases, participants situated their dashboard design somewhere which they said would be most convenient for them to access. For example, \fleck{}, \beck{}, and \sr{} all indicated they would choose to place a dashboard near their work desk. For \fleck{} and \sr{} in particular, they emphasized the importance of the dashboard being within arm's reach for easy interaction, particularly if the dashboard contained many interactive controls. For similar reasons, \aimee{} suggested placing a dashboard on top of their students' desks to facilitate interactive teaching activities.

% In general, participants took the following aspects into account when situating the dashboard in context: relevance to physical referent, virtual or tangible proxy of physical referent, time of use, place of use, and convenience of interaction. 

\subsection{Interacting with the Dashboard}
When asked what modalities they would expect to use with situated dashboards, all participants suggested using mid-air hand, eye gaze, and tangible interactions. Voice input was not a preferred modality. \ap{} cited that it ``might be hard to use in noisy environments'', and \beck{} and \sr{} stated that it would likely be uncomfortable to use in public.

As previously mentioned, most participants would rather interact with dashboards that are within arm's reach, and would therefore avoid interacting with dashboards that were far away. In such a scenario, \sr{} said they would use the dashboard to only look at information, not interact with it. \beck{} instead said that gaze interaction on a distant dashboard could be a way to perform certain tasks. They proposed gazing at a calendar on the dashboard, which would then open it on their personal computer for them to make changes on it.

%Participants mostly did not prefer interacting with dashboards that are situated at a distance. However, \sr said that he might only use such dashboard to look at information or use as a shortcut or do simple tasks such as checking off boxes (\beck, \sr). For example, he mentioned looking at the calendar on the dashboard for a while as a way to open it up on his computer where he can make changes on the calendar.

Other participants described alternative methods for interacting with the dashboard. \fleck{} mentioned their dislike of mid-air interaction, suggesting that a mobile application or tangible slider could be used to manipulate the data and/or referent instead. \aimee{} similarly suggested that the dashboard could be aligned against a tabletop surface, with physical objects being used to interact with the dashboard.

When asked how to perform basic visualization tasks, \kadek{} proposed that data could automatically be filtered based on the physical proximity of the user to the referent. In contrast, \beck{} suggested that data could be manually filtered using checkboxes. They also envisioned using a pinch gesture to zoom into specific visualizations for more details, or by grab and dropping a visualization onto a secondary panel to expand it.
Following the same overview first, zoom and filter mantra \cite{shneidermanEyesHaveIt1996}, \fleck{} described a ``reactive situated dashboard'' which changes its level of detail via proxemics \cite{hallProxemicsCommentsReplies1968}. Alternatively, \fleck{} suggested that the user could \textit{``focus [their gaze] on something for an extended period of time, [...], it gets the information and you get more details on your dashboard.''}


% \beck, \fleck, and \aimee---all of them placed dashboards with interactive components within arm's reach and talked about interacting with them through hand interactions. However, \aimee and \fleck{} mentioned that they usually do not like mid-air hand interaction and so \fleck{} talked about using mobile app or physical slider to control temperature through dashboard and \aimee~ talked about positioning the dashboard on tabletop and using physical objects such as flashcards to interact with it.
% \subsubsection{Data filtering}
% The techniques for data filtering used by participants include automatic filtering of information based on proximity of user to the referent (\kadek), and manual filtering of information by tapping on checkboxes (\beck).
% \subsubsection{Overview-Zoom}
% \beck{} used pinching gesture to zoom in on specific visualizations. Other alternative approaches he suggested for overview-zoom were using a second panel where he can grab and drop a visualization from the dashboard to reveal details. 

% \fleck{} suggested using "reactive situated dashboard" which does overview-detail when user comes closer to it. Additionally, he mentioned a strategy where if users "focus on something for an extended period of time, which is unusual to [...] routine while wearing AR glasses, [...] it gets the information and you get more details on your dashboard".

\subsection{Customizing the Dashboard}
\kadek{} and \beck{} emphasized the need to provide customization support for end-users to personalize their experiences with their situated dashboard. \beck{} suggested providing ``building blocks'' so that end-users \textit{``can build a solution that they need.''}
However, \beck{} also acknowledged the limitations of using building blocks to author entire situated dashboards. While it may be relatively easy to provide simple means to customize the layout of the dashboard, for example, they noted that ``\textit{[considering the] whole situated thing and like the context switching and so on [...] it becomes a lot more complicated.}''