\section{Embroidering Data of a Family Survey}

To evaluate the embroidery workflow and to investigate its use for visualizing personal data, we embroidered the results of a family survey using visualization with black-and-white textures.

\textbf{Data Collection.} The last two authors conducted a survey on their extended family members' preferences for seven vegetables (carrots, celery, corn, eggplant, mushrooms, olives, and tomatos). We asked each family member to rate each of the seven vegetables on a scale of 1 to 5 (1 = ``I don't eat it at all,'' 2 = ``I can eat it if necessary,'' 3 = ``it's ok, I neither like nor dislike it, ``4 = ``I like it,'' 5 = ``it's among my most favorite vegetables''). 11 family members participated in this survey. We calculated the average rating score for each vegetable, and present the result in \autoref{tab:family_survey_scores}.

\begin{table}[b]
\caption{The average scores of the family members' preference towards each of the seven vegetables.}
\label{tab:family_survey_scores}
\centering
\begin{tabular}{c@{\hspace{8pt}}c@{\hspace{8pt}}c@{\hspace{8pt}}c@{\hspace{8pt}}c@{\hspace{8pt}}c@{\hspace{8pt}}c}
\toprule
 carrots                   & celery                   & corn                     & eggplant                 & mushrooms                 & olives                    & tomatos                   \\
\midrule
4.33 & 2.33 & 4.11 & 2.78 & 4.00 & 2.56 & 3.89 \\
\bottomrule
\end{tabular}
\end{table}

\textbf{Visual Representation.} We created a bar chart titled ``How much does my family like vegetables'' to display the results. The $x$-axis represents the type of vegetable, while the $y$-axis indicates the degree of liking according to the mentioned scale. We selected the texture design from a collection of visualization designs with black-and-white textures, gathered from visualization design experts. We chose this particular design not just because it was composed of lines---making it well-suited for the embroidery process---but also because it appealed to a particular family member. In addition, iconic textures can have semantic association, thereby freeing the chart from the need for labels and thus making it extremely suitable for data embroidery. 

\textbf{Embroidery.} To convert the chart into an embroidery, we followed the embroidery process as detailed in \autoref{sec:machine-embroidery-process}. The entire embroidery process took approximately 2 hours, including troubleshooting. \autoref{fig:teaser}(a) shows the data embroidery result. We sewed the embroidered piece onto a canvas bag (see \autoref{fig:teaser}(b)). 
%
%Iconic textures have semantic association, therefore liberate the chart from the need for labels, making it extremely suitable for data embroidery. Machine embroidery tends to struggle with text, particularly the small text often used for labels or legends in visualizations. Therefore, the use of textures simplifies the incorporation of data into personal belongings through data embroidery.
%
By integrating family data into such an everyday item that is also used for grocery shopping, this data embroidery can act as a daily reminder of the family's preferences.
