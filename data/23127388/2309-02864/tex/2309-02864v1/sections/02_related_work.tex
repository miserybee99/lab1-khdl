\section{related work}
\label{sec:related_work}
Many researchers and artists have experimented with embroidery as a means to create visualization artwork. For instance, Liz Bravo \cite{LizBravo} manually embroidered charts of the distribution of U.S. cotton from 1942 to 1948, a visualization originally created by Mary Eleanor Spear, a pioneer in data visualization. Olivia Johnson \cite{OliviaJohnson} used cross-stitch techniques to create charts about gender inequality and discrimination at workplaces \cite{cabric:2023:eleven}. Hand embroidery has also been used to explore personal data. For example, Jane Zhang \cite{JaneZhang} logged her anxiety over 21 days and created an embroidered chart to visualize it---akin to other efforts in personal data visualization \cite{Posavec:2016:DD}. In the visualization research literature, Smit \cite{Smit:2021:DataKnitualization} has expanded the fabric-based data visualization landscape by exploring hand knitting as a potential medium for data physicalization, resulting in several \emph{data knitualization} works. All these efforts, however, have employed manual methods to represent data on fabric, while we explore a more automated process.

Machine embroidery has gradually started to draw researchers' attention as well. Wannamaker et al. \cite{wannamaker:2019:data} explored the use of CNC embroidery machines for expressing personal data and embroidered a personal data physicalization representing text message data on a blanket. Schneider \cite{DataVisualizationWithMachineEmbroidery} provided a tutorial on data visualization with machine embroidery using Ink/Stitch \cite{inkstitch}, in which he outlined a general workflow of computerized embroidery. This workflow notably includes an essential step of reducing colors of the drawing to adapt it to the embroidery constraints, which underscores the suitability of monochromatic charts for machine embroidery. We used this process as an inspiration but specifically focused on textured visualizations.


