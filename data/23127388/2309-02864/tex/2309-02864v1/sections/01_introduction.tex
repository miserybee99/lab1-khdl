\section{Introduction}
\label{sec:introduction}

Data embroidery \cite{wannamaker:2019:data} is an innovative technique for data physicalization \cite{Jansen:2015:opportunities}. Machine embroidery as a computer-numerically controlled (CNC) technology makes it  possible to produce complex data embroideries (relatively) quickly and integrate them into fabric-based personal belongings \cite{wannamaker:2019:data}. Data embroidery of personal data has potential because a less conventional approach to visualization may stimulate people to explore their own data more intensively \cite{wang:2015:design}. It can also serve as an ambient visualization within a home setting, thereby initiating dialogues with curious visitors \cite{Pousman:2007:casual}. Data embroidery can, like in our case, be accessible to a broad set of the population through local Fablabs. 

A promising yet so far unexplored avenue within data embroidery involves the use of black-and-white textures. Before the ubiquity of color printing, these monochromatic textures served as a powerful visual channel for data visualization (\eg, see the OldVisOnline collection \cite{Zhang:2024:OCD}). Their inherent simplicity facilitates the conversion of images to embroidery files, overcoming challenges associated with the lower color resolution of embroidery machines compared to color screens. Moreover, they eliminate the need for multiple color changes during the embroidery process, enhancing efficiency.

In our work we explore data embroidery with black-and-white textures and contribute the following: (1) a detailed and hands-on workflow for creating data embroidery from an existing black-and-white textured chart image, (2) preliminary evaluations of which textures can be most effectively translated into data embroidery, and (3) a showcase of data embroidery of personal data with black-and-white textures---a canvas bag with an embroidered chart visualizing a within-family survey.