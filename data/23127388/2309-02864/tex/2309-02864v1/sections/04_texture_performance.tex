\section{Which textures perform better for embroidery?}

The quality of data embroideried is influenced by several factors. Aside from the machine setup and the used materials, the pattern to be embroidered---in our case, the texture used in the design---can also impact the results. Despite having an optimal set-up, some textures are inherently more challenging to embroider than others, leading to more issues during the embroidery process. 

To identify which textures work better for embroidery we conducted an experimental exploration built upon our previous work \cite{He:2024:DCB,zhong:2020:BWT}. There we had categorized textures into two types: geometric textures (with basic shapes as primitives \inlinevis{-1pt}{1em}{figures/geometrictextures.pdf}) and iconic textures (with more figurative icons as primitives \inlinevis{-1pt}{1em}{figures/iconictextures.pdf}), as well as developed a simple design characterization across different dimensions of textures. We also had collected a set of textured visualization designs from visualization design experts. \autoref{fig:test_results} shows some embroidery samples during our experimentation. While our findings are primarily based on the setup we described in \autoref{sec:machine-embroidery-process}, the fundamental principles are similar across various embroidery machines, thus the conclusions we outline next likely offer broader insights. 

\begin{figure}
    \centering
		\includegraphics[width=0.47\columnwidth]{figures/test01.pdf}\hfill%
		\includegraphics[width=0.47\columnwidth]{figures/test02.pdf}
             \includegraphics[width=0.47\columnwidth]{figures/test03.pdf}\hfill%
		\includegraphics[width=0.47\columnwidth]{figures/test04.pdf}		
    \caption{Some embroidery samples during our experimentation. }%\vspace{-1ex}
    \label{fig:test_results}
\end{figure}


\textbf{Continuous vs. Scattered Elements.} Embroidery machines perform more efficiently with continuous elements---either as continuous areas or continuous lines (\eg, the left bars in \autoref{fig:texture_performance}(a)). Even detailed line-based icons (\eg, left bars in \autoref{fig:texture_performance}(b)) can be embroidered clearly and accurately. On the other hand, designs with small, scattered areas  (\eg, the right-most bars in \autoref{fig:texture_performance}(a) and (b), respectively) proved problematic for the embroidery process. For example, small dotted textures are prone to cause issues during the embroidery process: their final appearance is not clean (\autoref{fig:texture_performance}(a), right-most bar) and the process often leads to problems as evident in the complex threads on the back side due to the bobbin thread becoming entangled (\autoref{fig:reverse_texture_performance}, left-most bar), while the textures with fewer problems have leaner back sides (visible for the rest of \autoref{fig:reverse_texture_performance}).

\textbf{Simplicity vs. Complexity.} The resolution of the embroidery affects the results of complex patterns. Designs with numerous small details can be difficult to embroider, leading to a loss in clarity and accuracy (\eg, the tomato of the right-most bar in \autoref{fig:texture_performance} lacks the needed detail that separates the main body from the stem, which leads to it being shown essentially as a large dot). Therefore, we observed that simpler patterns lend themselves better to the embroidery process. In our experiments, simple icon designs were translated more effectively into embroidered designs compared to their more complex counterparts.

To illustrate which textures perform better and to facilitate comparisons, we selected some typical examples and visualized their performance using bar charts. The previously mentioned examples in \autoref{fig:texture_performance} compare the performance of the different designs, one for geometric textures and one for iconic textures. In addition, machine embroidery tends to struggle with text, in particular with the small text that is often used for labels or legends in visualizations. In fact, when we look at the reverse side of the embroidery result (\autoref{fig:reverse_texture_performance}), we see that text exhibits similar problems as the textures with scattered elements we mentioned before. Please note that the data shown in \autoref{fig:texture_performance} and~\ref{fig:reverse_texture_performance} are our own subjectively perceived/assessed values of ease of reproduction, they are not precise measurements.

\begin{figure}
    \centering
		(a)~\includegraphics[width=0.9\columnwidth]{figures/texture_performance_geomatric.pdf}\hfill%
		(b)~\includegraphics[width=0.9\columnwidth]{figures/texture_performance_iconic.pdf}		
    \caption{Two embroidered charts showing the performance of different textures in machine embroidery: (a) for geometric textures, and (b) for iconic textures.}%\vspace{-1ex}
    \label{fig:texture_performance}
\end{figure}

\begin{figure}
    \centering
		\includegraphics[width=0.9\columnwidth]{figures/reverse_texture_performance_geomatric.pdf}		
    \caption{The reverse side of \autoref{fig:texture_performance}(a). The messy threads indicate that these textures or text elements are problematic to embroider.}%\vspace{-1ex}
    \label{fig:reverse_texture_performance}
\end{figure}