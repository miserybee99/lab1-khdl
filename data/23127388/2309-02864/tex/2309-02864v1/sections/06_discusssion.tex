\section{Discussion and Future Work}
Given the monochromatic nature of black-and-white textures and, in particular, the semantic association of iconic textures, we envision that they have the potential to become an important visual channel for data embroidery. Our study represents a first step in exploring this technique and we provide our experiences for future embroidery work. Such future work could focus more systematically on evaluating the impact of data embroidery with black-and-white textures in terms of both efficiency and aesthetics \cite{He:2023:BVS}.

While our work primarily focuses on embroidery, visualizations featuring black-and-white textures can also be easily created physically through a variety of other methods such as 3D printing, laser engraving, vinyl cutting, or embossing. 
We experimented with 3D printing of textured visualizations and produced two charts—one with geometric textures and another with iconic textures (see \autoref{fig:3D_printed_charts}). Compared to embroidery, setting up a 3D printer is simpler and less prone to errors. The process follows basic 3D printing steps: converting the image file (\texttt{PNG} or \texttt{SVG}) to \texttt{STL} using stand-alone or online converters, slicing the STL model to \texttt{GCODE} using a slicer, and, ultimately, importing the \texttt{GCODE} to the 3D printer for printing. The printing process was error-free---textured pieces are not different from any other 3D print.

\begin{figure}
    \centering
		\includegraphics[width=1\columnwidth]{figures/3D_print.jpg}\hfill%	
    \caption{Two 3D printed textured charts, one with geometric textures, and another with iconic textures.}%\vspace{-1ex}
    \label{fig:3D_printed_charts}
\end{figure}


Our exploration of physicalization using monochrome charts can also open up an interesting additional avenue for future investigation: studying their potential used by visually impaired individuals. Given their tactile nature, these charts could potentially serve as valuable tools for data representation for this population.
