
%% bare_jrnl_compsoc.tex
%% V1.4b
%% 2015/08/26
%% by Michael Shell
%% See:
%% http://www.michaelshell.org/
%% for current contact information.
%%
%% This is a skeleton file demonstrating the use of IEEEtran.cls
%% (requires IEEEtran.cls version 1.8b or later) with an IEEE
%% Computer Society journal paper.
%%
%% Support sites:
%% http://www.michaelshell.org/tex/ieeetran/
%% http://www.ctan.org/pkg/ieeetran
%% and
%% http://www.ieee.org/

%%*************************************************************************
%% Legal Notice:
%% This code is offered as-is without any warranty either expressed or
%% implied; without even the implied warranty of MERCHANTABILITY or
%% FITNESS FOR A PARTICULAR PURPOSE! 
%% User assumes all risk.
%% In no event shall the IEEE or any contributor to this code be liable for
%% any damages or losses, including, but not limited to, incidental,
%% consequential, or any other damages, resulting from the use or misuse
%% of any information contained here.
%%
%% All comments are the opinions of their respective authors and are not
%% necessarily endorsed by the IEEE.
%%
%% This work is distributed under the LaTeX Project Public License (LPPL)
%% ( http://www.latex-project.org/ ) version 1.3, and may be freely used,
%% distributed and modified. A copy of the LPPL, version 1.3, is included
%% in the base LaTeX documentation of all distributions of LaTeX released
%% 2003/12/01 or later.
%% Retain all contribution notices and credits.
%% ** Modified files should be clearly indicated as such, including  **
%% ** renaming them and changing author support contact information. **
%%*************************************************************************


% *** Authors should verify (and, if needed, correct) their LaTeX system  ***
% *** with the testflow diagnostic prior to trusting their LaTeX platform ***
% *** with production work. The IEEE's font choices and paper sizes can   ***
% *** trigger bugs that do not appear when using other class files.       ***                          ***
% The testflow support page is at:
% http://www.michaelshell.org/tex/testflow/


\documentclass[10pt,journal,compsoc]{IEEEtran}
%
% If IEEEtran.cls has not been installed into the LaTeX system files,
% manually specify the path to it like:
% \documentclass[10pt,journal,compsoc]{../sty/IEEEtran}





% Some very useful LaTeX packages include:
% (uncomment the ones you want to load)


% *** MISC UTILITY PACKAGES ***
%
\usepackage{ifpdf}
% Heiko Oberdiek's ifpdf.sty is very useful if you need conditional
% compilation based on whether the output is pdf or dvi.
% usage:
% \ifpdf
%   % pdf code
% \else
%   % dvi code
% \fi
% The latest version of ifpdf.sty can be obtained from:
% http://www.ctan.org/pkg/ifpdf
% Also, note that IEEEtran.cls V1.7 and later provides a builtin
% \ifCLASSINFOpdf conditional that works the same way.
% When switching from latex to pdflatex and vice-versa, the compiler may
% have to be run twice to clear warning/error messages.






% *** CITATION PACKAGES ***
%
\ifCLASSOPTIONcompsoc
  % IEEE Computer Society needs nocompress option
  % requires cite.sty v4.0 or later (November 2003)
  \usepackage[nocompress]{cite}
\else
  % normal IEEE
  \usepackage{cite}
\fi
% cite.sty was written by Donald Arseneau
% V1.6 and later of IEEEtran pre-defines the format of the cite.sty package
% \cite{} output to follow that of the IEEE. Loading the cite package will
% result in citation numbers being automatically sorted and properly
% "compressed/ranged". e.g., [1], [9], [2], [7], [5], [6] without using
% cite.sty will become [1], [2], [5]--[7], [9] using cite.sty. cite.sty's
% \cite will automatically add leading space, if needed. Use cite.sty's
% noadjust option (cite.sty V3.8 and later) if you want to turn this off
% such as if a citation ever needs to be enclosed in parenthesis.
% cite.sty is already installed on most LaTeX systems. Be sure and use
% version 5.0 (2009-03-20) and later if using hyperref.sty.
% The latest version can be obtained at:
% http://www.ctan.org/pkg/cite
% The documentation is contained in the cite.sty file itself.
%
% Note that some packages require special options to format as the Computer
% Society requires. In particular, Computer Society  papers do not use
% compressed citation ranges as is done in typical IEEE papers
% (e.g., [1]-[4]). Instead, they list every citation separately in order
% (e.g., [1], [2], [3], [4]). To get the latter we need to load the cite
% package with the nocompress option which is supported by cite.sty v4.0
% and later. Note also the use of a CLASSOPTION conditional provided by
% IEEEtran.cls V1.7 and later.





% *** GRAPHICS RELATED PACKAGES ***
%
\ifCLASSINFOpdf
  \usepackage[pdftex]{graphicx}
  \graphicspath{{figures/}{pictures/}{images/}{./}}
\else
  % or other class option (dvipsone, dvipdf, if not using dvips). graphicx
  % will default to the driver specified in the system graphics.cfg if no
  % driver is specified.
  \usepackage[dvips]{graphicx}
  % declare the path(s) where your graphic files are
  % \graphicspath{{../eps/}}
  % and their extensions so you won't have to specify these with
  % every instance of \includegraphics
  % \DeclareGraphicsExtensions{.eps}
\fi

\usepackage{hyperref}
\usepackage{tabu}                      
\usepackage{booktabs}                  
\usepackage{lipsum}                   
\usepackage{mwe}                       
\usepackage{amsfonts}                   
\usepackage{enumitem}           
\PassOptionsToPackage{warn}{textcomp}  
\usepackage{textcomp}                 
\usepackage{cite}                     
\usepackage{multirow}
\usepackage{makecell}
\usepackage{graphicx}
\usepackage{caption}
\usepackage{xcolor}
\newcommand{\gpt}{{GPT-3}}
\newcommand{\live}{{Live Chart}}
\newcommand{\lives}{{Live Charts}}
\newcommand{\quo}[1]{\textit{``#1''}}
\newcommand{\changed}[1]{{\color{black} {#1}}}
\newcommand{\added}[1]{{#1}}
\newcommand{\minor}[1]{{\color{black} {#1}}}
\newcommand{\last}[1]{{\color{blue} {#1}}}
\renewcommand{\figureautorefname}{Fig.}
\renewcommand{\sectionautorefname}{Sec.}
\renewcommand{\subsectionautorefname}{Sec.}
\renewcommand{\subsubsectionautorefname}{Sec.}
\hyphenation{op-tical net-works semi-conduc-tor}


\begin{document}
\title{Reviving Static Charts into Live Charts}
\author{
    Lu Ying, Yun Wang, Haotian Li, Shuguang Dou,\\ Haidong Zhang, Xinyang Jiang, Huamin Qu, and Yingcai Wu
        \IEEEcompsocitemizethanks{
        \IEEEcompsocthanksitem L. Ying, Y. Wu are with the State Key Lab of CAD\&CG, Zhejiang University, Hangzhou, China. 
        L. Ying is also with Microsoft Research Asia (MSRA), China. 
        E-mail: \{yingluu, ycwu\}@zju.edu.cn. 
        \IEEEcompsocthanksitem  Y. Wang, H. Zhang, X. Jiang is with Microsoft Research Asia, China. 
        E-mail: \{wangyun, haizhang, xinyangjiang\}@microsoft.com.
        \IEEEcompsocthanksitem  H. Li and H. Qu are with The Hong Kong University of Science
        and Technology, Hong Kong, China. 
        E-mail: haotian.li@connect.ust.hk, huamin@cse.ust.hk.
        \IEEEcompsocthanksitem  S. Dou is with the Department of Computer Science and Technology, Tongji University, Shanghai, China.
        He is also with Microsoft Research Asia, Beijing, China. 
        Email: 2010504@tongji.edu.cn.
        \IEEEcompsocthanksitem  Yun Wang is the corresponding author.
        \IEEEcompsocthanksitem This work was done when L. Ying and S. Dou were interns at MSRA.}
    \thanks{Manuscript received xxx xx, 2023; revised xxx xx, 2023.}
}

% The paper headers
\markboth{Journal of \LaTeX\ Class Files,~Vol.~x, No.~x, xx~2023}%
{Lu \MakeLowercase{\textit{et al.}}: Reviving Static Charts into Live Charts}


\IEEEtitleabstractindextext{%
\begin{abstract}
    \begin{abstract}
% 
3D dense captioning requires a model to translate its understanding of an input 3D scene into several captions associated with different object regions.
% 
Existing methods adopt a sophisticated “detect-then-describe” pipeline, which builds explicit relation modules upon a 3D detector with numerous hand-crafted components.
% 
While these methods have achieved initial success, the cascade pipeline tends to accumulate errors because of duplicated and inaccurate box estimations and messy 3D scenes.
% 
In this paper, we first propose Vote2Cap-DETR, a simple-yet-effective transformer framework that decouples the decoding process of caption generation and object localization through parallel decoding.
% 
% We show that the sophisticated and explicit relation reasoning modules can be replaced by the attention mechanism to capture both object-object and object-scene relations.
% 
\whatsnew{
Moreover, we argue that object localization and description generation require different levels of scene understanding, which could be challenging for a shared set of queries to capture.
% 
To this end, we propose an advanced version, Vote2Cap-DETR++, which decouples the queries into localization and caption queries to capture task-specific features.
% 
Additionally, we introduce the iterative spatial refinement strategy to vote queries for faster convergence and better localization performance.
% 
We also insert additional spatial information to the caption head for more accurate descriptions.
% 
Without bells and whistles, extensive experiments on two commonly used datasets, ScanRefer and Nr3D, demonstrate Vote2Cap-DETR and Vote2Cap-DETR++ surpass conventional ``detect-then-describe'' methods by a large margin.
}
% 
Codes will be made available at \href{https://github.com/ch3cook-fdu/Vote2Cap-DETR}{https://github.com/ch3cook-fdu/Vote2Cap-DETR}.
\end{abstract}

\begin{IEEEkeywords}
Multi-modal Learning, 3D Scene Understanding, 3D Dense Captioning, Transformers.
\end{IEEEkeywords}

\end{abstract}


\begin{IEEEkeywords}
Charts, storytelling, machine learning, automatic visualization
\end{IEEEkeywords}}



\maketitle


\IEEEdisplaynontitleabstractindextext
\IEEEpeerreviewmaketitle

\input{sections/1_intro.tex}
\input{sections/2_relatedwork}
\section{Live Charts}
\label{sec: livecharts}
Since making Live Charts from static vector-based charts requires considerable effort, we aim to revive them into Live Charts automatically.

To enhance the chart expression, previous studies have emphasized the significance of annotations or textual cues to enhance user comprehension of charts~\cite{ren2017chartaccent}.
In contrast to traditional text cues near charts that divert visual attention, we utilize voiceover narration—freeing up visual focus while enhancing chart comprehension.
Moreover, prior research~\cite{wang2018narvis} highlights the effectiveness of breaking down complex information into smaller, more digestible chunks when presenting charts.
In this regard, we choose to integrate animations into Live Charts, which can assist users in visually processing intricate data and comprehending complex chart representations.
Furthermore, a multi-sensory experience by synchronizing narrations and animations within the Live Charts can complement each other by providing supplementary details, or indirectly link to each other by presenting data from different perspectives~\cite{cheng2022investigating}.
Therefore, we propose an approach that involves three steps:
\begin{enumerate}[noitemsep]
    \item[(1)] Using GNN-based techniques, we identify the data together with the visual encodings from the input SVG.
    \item[(2)] We derive the audio narration with insights and its corresponding data with LLMs.
    \item[(3)] For some expressions in narration, we design the corresponding animations and synchronize them to the audio.
\end{enumerate}

\begin{figure}[htbp]
    \centering
    \includegraphics{figures/pipeline.pdf}
    \caption{Our automatic approach for generating Live Charts.}
    \label{fig:pipeline}
\end{figure}

\changed{
We then describe each step in detail accompanied by an example depicted in \autoref{fig:ChartUnder}(a), which we refer to as the ``airport chart.''
}




\input{sections/3.1_chart_understanding}
\input{sections/3.2_narration}
\subsection{Animation}
\label{sec:animation}
Previous research has demonstrated that animations synchronized with narration can stimulate both visual and auditory senses, leading to improved data representation~\cite{cheng2022investigating, huth2023studies}.
Therefore, we describe how we incorporate animation and synchronize it with narration in this section.
We design the animations based on the two-part narration generated in \autoref{sec:narration}. 
We aim to identify the data related to specific phrases in the narration and add appropriate animations to the corresponding time intervals.
Based on prior research~\cite{ge2020canis}, we use three key aspects to define an animation:
\begin{itemize}
    \item \textbf{Target}: The target elements to be animated.
    As we have added relevant classes and IDs into the input SVG as the labeled SVG (as discussed in \autoref{sec: understanding}), targeting elements is not challenging.
    Technically, we utilize the W3C Selectors API.
    
    \item \textbf{Effects}: The animation effects.
    Similar to \cite{wang2021animated}, we adopt three principal animation types in presentation software (e.g., Keynote and PowerPoint): entrance, emphasis, and exit. 
    Then, we choose multiple prevalent animation styles from past research~\cite{wang2021animated, ge2020canis} and devise styles to fulfill our requirements.
    Part of the animation effects is depicted in \autoref{fig:cases} with dotted blue boxes.
    
    \item \textbf{Interval}: A suitable time interval for elements to enter and exit the screen.
    Instead of using time seconds to represent the entry and exit times, we adopt a different approach. 
    We use the start and end word indices in the narration as a reference.
    For instance, the interval $[10, 20]$ denotes the animation starting from the 10th word and ending at the 20th word.
    Since we use a text-to-speech technique to generate the audio, the API can conveniently generate the timestamp of a specific word in the narration.
    Thus, with all the words having timestamps, we can convert the word index interval into a normal time interval.
    
\end{itemize}

Defining the three variables is the key to completing the task of adding animation.
For each animation, we need to identify the relevant phrases first and convert them into a word index interval. 
Then, we select the corresponding elements as targets and apply an appropriate animation effect to create a polished and visually engaging Live Chart. 
We process the two-part narration separately.

\subsubsection{Animation for the contextual narration}
We first analyze the contextual narration to locate the interval and identify the relevant targets. 
For the first sentence related to the chart title, we set the target as \textit{``title''} and locate the entire first sentence as its interval. 
Moving on to the second sentence about encodings, we locate the position of the words ``x-axis'' and ``y-axis'' for bar and line charts. 
The sentence can be divided into three parts, each representing a distinct interval.
The first and second parts target \textit{``x-axis''} and \textit{``y-axis''}, respectively, and the last part target \textit{``legend''}. 
\changed{For instance, as shown in \autoref{fig:narration}(a6), the first part is represented by ``displays months while'', the second part is represented by ``shows the average... and''.}
For pie charts, we set the interval as the whole sentence and the target as \textit{``sector''} and \textit{``legend''}. 
To display corresponding elements in the front part of live charts, we utilize the entrance animation among the three available animation types.
Thus, animation effects like fade-in and wipe are selected for the contextual narration.

\begin{figure*}[ht]
    \centering
    \includegraphics{figures/cases.pdf}
    \caption{Two use cases. (a1-a5)(b1-b5) The image flow illustrates the keyframes of the Live Chart. Animations are drawn with dotted blue boxes.
    The following text describes the corresponding audio narration, with the first tag indicating the chart component or the insight type.}
    \label{fig:cases}
\end{figure*}


\subsubsection{Animation for the insights}
We create animations that correspond to each insight in order to enhance the storytelling.
Having identified the pertinent data for each sub-insight, we utilized the ``ac-data'' attribute labeled in \autoref{sec: understanding} to target the appropriate elements.
SVG paths and texts can be the candidates.
Concerning the word interval, we continue to rely on \gpt{} to interpret the insightful narration, as it is responsible for generating the narration itself. 
Our approach is progressive, consisting of two prompts. 
Given the table, a list of the distilled insights, and insightful narration, the first prompt is employed to assign sentences to each insight (\autoref{fig:narration}(a7)).
In the output, sentences are included in the insight structure (\autoref{fig:narration}(b)).
Furthermore, with the same inputs (table and insightful narration) and the distilled insights accompanied by their corresponding sentences, the second prompt is utilized to allocate phrases within each insight sentence to each sub-insight (\autoref{fig:narration}(a8)).
\changed{Phrases are added in the subinsights structure after this step (\autoref{fig:narration}(b)).}
Through this process, we identify the appropriate word interval for the animation.
Then we proceed to select the animation effects for the above elements.
We opt for emphasis animations while choosing animation effects. 
We use the animation effects ``show box'', ``show text'', ``change color'' and ``highlight'' for the elements in three chart types.
Additionally, we include the ``bar bounce'' effect for bar charts, as well as ``show arrow'' and ``show circle'' for line and bar charts.
Sometimes certain animation effects may not have specific elements to target within a chart.
For instance, the ``show circle'' effect cannot be used for line charts that lack points.
In such scenarios, we alter the emphasis effect into a list of entrance-exit animations. To be precise, we create such elements and allow them to enter at the beginning of the animation, followed by their exit at the end of the animation.
We select different animation effects for different insights in each chart to ensure the richness of the generated charts.
All animation effects are chosen based on the chart type and insight type.



\section{Evaluation}
In this section, we first present a set of use cases for reviving real-world charts into \lives{}.
We describe our evaluation process, which involved testing the performance of the understanding model and conducting a user study and expert interview to evaluate \lives{}.
Our evaluation results further confirm the effectiveness of our workflow and demonstrate the potential of the Live Chart format.

\subsection{Use Cases}
\label{sec: usecase}
To demonstrate the capabilities of our workflow in \autoref{sec: livecharts}, we present a set of Live Charts, including bar, line, and pie (donut) charts.
We collected chart SVGs from the Internet and ensured that they covered different topics and styles.
\autoref{fig:teaser} and \autoref{fig:cases} show part of the examples in our gallery~\footnote{The whole examples can be found in the supplementary materials.}.
Each chart is presented in a step-by-step image sequence from index 1 to 5, beginning with contextual information (title, axis, and legend) and followed by various insights.
We use an animation box for each insight depicted in the chart, using a dotted stroke (as shown in \autoref{fig:cases}), to indicate both the type of animation effect and the process by which the corresponding element is altered.
Below the image sequence, we provide the corresponding narration with a tag to indicate contextual or insightful information, followed by a specific description.



\subsection{Performance of Chart Element Recognition}
\label{sec: performance}

\begin{table*}[ht]
\centering
\caption{The table presents the effectiveness of pixel-based methods and our dual-stream GNN specifically designed for vector graphics recognition on Vega-lite, Plotly, and D3 datasets. $E_e$ denotes only using the element-wise encoder, $E_s$ denotes only using the stroke-wise encoder, and $E_s + E_e$ denotes using the two encoder.}
\begin{tabular}{l|ccc|ccc|ccc}
\toprule
& \multicolumn{3}{c|}{Vega-lite} & \multicolumn{3}{c|}{Plotly} & \multicolumn{3}{c}{D3}\\
        \cmidrule(r){2-4} \cmidrule(l){5-7} \cmidrule(l){8-10}
Methods & $AP_{50}$(\%)  & $AP_{75}$(\%) & mAP (\%) & $AP_{50}$(\%)  & $AP_{75}$(\%) & mAP (\%) & $AP_{50}$(\%)  & $AP_{75}$(\%) & mAP (\%) \\\midrule
YoloV8-m       & 77.30 & 68.90 & 64.60 & 77.60 & 70.60 & \textbf{66.00} & 64.30 & 50.50 & 49.30\\
Ours ($E_e$) & 71.75 & 68.65 & 63.14 & 69.25 & 64.90 & 54.93 & 61.43& 59.77 &  51.98\\
Ours ($E_s$)  & 84.81 & 80.75 & 74.28 & 78.04 & 74.06 & 62.87 & 62.47  & 60.27  & 52.66  \\
Ours ($E_s + E_e$) & \textbf{85.71} & \textbf{81.85} & \textbf{75.21} & \textbf{80.04} & \textbf{76.57} & 64.55 & \textbf{64.77} & \textbf{61.23} & \textbf{53.40} \\
\bottomrule
\end{tabular}

\label{table:AblationStudy}
\end{table*}   

We conducted a series of quantitative evaluations to assess our understanding method in comparison with baselines.

We selected pixel-based methods as our baselines, utilizing mmyolo~\cite{mmyolo2022} for implementation while keeping all parameter settings remaining the same.
Our model is implemented using PyTorch Geometric~\cite{pyg} built upon PyTorch~\cite{PyTorch}. 
We construct a two-layer GNN for the stroke and element-wise encoder. 
We evaluated different machine-learning model parameters and settled on the following sets.
The hidden node representation dimension is set to 64, as deeper GNNs tend to suffer from over-smoothing. 
To avoid over-smoothing, mean aggregation is only conducted on the final layer of the GNN, and each node requires only one transformation and one mean-pooling operation. 
Moreover, a three-layer MLP is employed as the classifier, with the output dimensions of the middle layers set to 512 and 256.
The Adam optimizer is utilized with a learning rate of 0.001 and a batch size of 128. 
The training procedure is performed from scratch for 200 epochs on an Nvidia V100 GPU card.

\textbf{Evaluation Metric.} 
We adopt the commonly used metrics of AP50, AP75, and mAP. Specifically, AP* denotes the average precision at the intersection over the union (IOU) threshold of 0.5 or 0.75 for object detection tasks. Moreover, we calculate the mAP as the mean of the average precision over the IOU thresholds ranging from 0.50 to 0.95.

\textbf{Results.} 
We use the dataset described in \autoref{sec: recognition}.
\minor{For pixel-based methods, we compare our methods with popular one-stage object detection method YoloV8~\cite{yolov8} based on mmyolo.} For YoloV8, the -m variant is scaled YoloV8 with more parameters and better performance. 
As shown in \autoref{table:AblationStudy}, we evaluate the effectiveness of two different vector encoders for chart element recognition. When solely relying on the stroke-wise encoder, our model demonstrates commendable performance. Through our experimentation, we observed that using both stroke-wise and element-wise encoders in conjunction, via the GNN, results in optimal classification performance.
\subsection{User Study}

We further assessed the effectiveness of Live Charts generated by our approach through a user study with 90 participants. 
We conducted our study on Prolific~\footnote{https://www.prolific.co/}, a popular platform for research participation. 
This study aims to assess the following aspects:
\begin{itemize}
    \item Can \lives{} assist users in understanding charts?
    \item In what ways do animation and narration in \lives{} enhance the chart representation?
\end{itemize}
We describe the procedure and results in this section. 


\subsubsection{Participants} 
We used the Qualtrics survey platform to set up a crowd-sourcing experiment.
90 participants (29 females) were recruited from Prolific in our study. 
The majority of participants, comprising 42\%, fell into the age range of 25 to 34 years, followed by 27\% in the age range of 35 to 44 years.
As the task required writing, participants' proficiency in writing and English vocabulary could influence the outcomes. Thus, we limited the study to only English-speaking users who had previously received at least a 95\% approval rate on their results. We compensated each participant with \$4.9 for completing the 40-minute task. 


\subsubsection{Procedure} 
The study consisted of five parts. 
The first part involved explaining the study procedure and obtaining consent and demographic information from the participants. 
To ensure that our participants had a basic understanding of insights and visualization, we presented eight insight types that we used in our generation process, along with their definitions and examples.
Subsequently, we asked three questions, in which participants needed to observe a chart and answer a question related to three randomly selected insights out of a total of eight.
For example, one of the questions could be, ``What trend best describes the line?''
Only participants who answered these questions correctly were eligible to participate in the study.

To investigate the role of animations and narrations, we carried out a between-subject study. Participants were randomly divided into three groups and presented with varying chart formats: static, animated, and \lives{}.
Each participant was only shown one type of chart format.
For the third part, we demonstrated a bar, pie, and line chart, providing sample insights for each. 
For the fourth part, participants were tasked to provide insights for charts. The chart format used was the same as in the previous step. The six charts are presented randomly.
Participants were asked to fully explore the chart insights and then write down insights.
Moreover, participants with animated charts and Live Charts were required to remain on the page longer than the video length to ensure they viewed it at least once.
Additionally, for participants with Live Charts, we included task questions at random intervals to test the audio functionality, to verify that they can hear the accompanying narration.
we recorded the insights written by users and further calculated the sentence number of insights.

After completing the writing tasks, participants were asked to score the chart. 
They were required to rate the chart in five dimensions: insightful, memorable, focused attention, understandability, and enjoyment.
We used a seven-point Likert scale, with options ranging from strongly disagree (1) to strongly agree (7). 
Finally, participants were asked to provide a brief explanation for their rating.

\subsubsection{Stimuli Preparation}
We collected SVG charts from use cases (\autoref{sec: usecase}), covering a wide range of topics. 
Considering the potential for user fatigue during prolonged experimentation, we prepared a set of 9 charts for each user, with 3 charts for each chart type (bar, pie, and line). Among these, one chart of each type (3 in total) was used for demonstrating examples, and the remaining 6 charts (2 for each type) were used in the formal study.  
For each chart in the study, we prepared three different formats: a static chart, an animated chart, and a \live{}. The original SVG was used as the static version, while the \lives{} were generated using the techniques discussed in \autoref{sec: livecharts}. For the animated charts, we used the same animations as in the \lives{}.
We further shortened the intervals between each animation to one second because sorely removing the narrations from the live version may result in long intervals between animations, which could have bored users.

\subsubsection{Results}
\changed{This study gathered the Likert ratings from five perspectives.}
As all data are not normally distributed, we use the Kruskal-Walli test for each factor.
Additionally, we utilized the Mann-Whitney tests with Bonferroni correction to compare pairs within each factor and accounted for multiple comparisons. The sample means, and their corresponding confidence intervals are depicted in \autoref{fig:quatitative} and \autoref{table:quatitative}.

The Likert ratings, measured on a scale from 1 to 7, showed significant differences across all dimensions (p $<$ 0.05). 
Post-hoc testing revealed significantly higher scores for all dimensions in the live format compared to the static charts (p $<$ 0.05, $\left| r \right| > 0.3$). 
Furthermore, we found that Live Charts scored marginally significantly higher than animated charts in the \textit{Insightful} and \textit{Focused Attention} dimensions. Animated charts scored marginally significantly higher than static charts in the \textit{Understandability} and \textit{Enjoyment} dimensions.
The results indicate that Live Charts offer a compelling and memorable approach and effectively direct users' attention toward important information.


\begin{figure}[htbp]
\includegraphics{figures/quantitative_v2.pdf}
\centering
\caption{
The statistical analysis results for the overall evaluation include the corresponding Likert ratings. 
The means are depicted using shapes, and their confidence intervals are shown with error bars. On the right side are the specific numerical mean values.}
\label{fig:quatitative}
\end{figure}

\begin{table}[h]
\captionsetup[figure]{labelfont={bf},labelformat={default},labelsep=period,name={Fig.}}
\caption{\minor{Overall statistical evaluation result for each dimension, including the Chi-square values, p-values, degrees of freedom, and effect size. }}
\begin{tabular}{l|l|ll}
\toprule
\multicolumn{1}{c|}{Dimensions}   & Chart Format Pair   & p-value & \thead{effect\\size ($r$)} \\ 
\midrule
\multirow{3}{*}{\thead{Insightful \\ $\chi^2 (2)=8.05$, p=0.017}}      
                                   & Static - Animated & 0.182    & 0.17 \\
                                   & Static - Live     & \textbf{0.008}   & \textbf{0.34}        \\
                                   & Animated - Live   & 0.069    & -0.23       \\
\midrule
\multirow{3}{*}{\thead{Memorable \\ $\chi^2(2)=6.17$, p=0.045}}       
                                   & Static - Animated & 0.284    & 0.13 \\
                                   & Static - Live     & \textbf{0.018}   & \textbf{0.31}        \\
                                   & Animated - Live   & 0.120    & -0.20       \\
\midrule
\multirow{3}{*}{\thead{Focused attention \\ $\chi^2(2)=8.85$, p=0.012}} 
                                   & Static - Animated & 0.206    & 0.16\\
                                   & Static - Live     & \textbf{0.003}   & \textbf{0.38 }       \\
                                   & Animated - Live   & 0.087   & -0.22       \\
\midrule
\multirow{3}{*}{\thead{Understandability \\ $\chi^2(2)=6.36$, p=0.042}} 
                                   & Static - Animated & 0.060    & 0.24\\
                                   & Static - Live     & \textbf{0.017}   & \textbf{0.31}        \\
                                   & Animated - Live   & 0.733    & -0.04       \\
\midrule
\multirow{3}{*}{\thead{Enjoyment \\ $\chi^2(2)=6.18$, p=0.046}}         
                                   & Static - Animated & 0.086    & 0.22\\
                                   & Static - Live     & \textbf{0.013}   & \textbf{0.32}        \\
                                   & Animated - Live   & 0.758    & -0.04      \\
\bottomrule
\end{tabular}
\label{table:quatitative}
\end{table}

\subsubsection{Feedbacks}
Based on our observations, most participants in Live Charts provided positive feedback about Live Charts. 

Both participants in animated charts and Live Charts acknowledged the effectiveness of animation in conveying data effectively.
First, participants using the Live Charts format rated the charts significantly higher in terms of \textit{Understandability} than participants in the static charts format. 
Animated chart users thought that animations allow for breaking down complex information into smaller, more manageable parts, which in return increases comprehension.
Live Chart users also mentioned that \quo{(\lives{}) provide a way to communicate quantitative information in a way that is easier to
 understand for many people.}
In contrast, some participants in static charts found it challenging to absorb all the information presented in a static chart.
For example, one reported that \quo{A chart might be too complicated that I can't recognize all the patterns.}
Moreover, animations provide extra value in the form of accompanying text by extracting information from the static chart using a model. 
This feature was frequently mentioned as a drawback of static charts, with users reporting difficulty in predicting precise numbers.
Second, animated charts and Live Charts users highlighted that animation helps emphasize specific chart elements, transforming data into engaging and memorable stories. 
For example, one participant in Live Charts mentioned that \quo{The animation is also great for highlighting trends and turning the data into a memorable story,} while another participant in static charts wrote that \quo{(Static charts) do not allow highlighting of specific sections.}
Furthermore, users with animated and Live formats agreed that the formats were interesting, while participants with static charts found it dull to look at for a long period. This pattern was also supported in the Likert score for the \textit{Enjoyment} dimension.

Narration emerged as another crucial component, as users noted its ability to complement the animation by directing auditory attention to specific elements.
This pattern can be reflected in the higher \textit{Focused Attention} score in Live Charts.
Moreover, with narration, Live Charts can present specific insights more clearly and effectively, as a participant noted, \quo{(Live Charts) clearly explain what the chart shows.}
Users with animated charts also reinforced this fact by suggesting that \quo{Animated charts can sometimes make it more difficult to understand what is being shown, particularly when only using highlighting to emphasize certain aspects.}
This observation may explain the disparity in the higher score for the \textit{Insightful} dimension for Live Charts.
Furthermore, integrated with the \gpt{} model, the narrator can analyze the data table and combine it with relevant news or context. 
As noted by one user with Live Charts, \quo{There's more context with narration allowing for greater insights into the data and increasing understanding.}

In addition to the benefits of animation and narration on their own, we have also received feedback highlighting the advantages of combining the two.
Users with Live Charts reported that the interplay between animation and narration could enhance the overall experience.
One participant in Live Charts noted that \quo{It is interesting hearing voice while looking at the chart equally.} 
They also suggested that ensuring that the information presented in both the voice-over and the chart matches accurately is crucial.
Furthermore, they rated the \textit{Memorable} dimension significantly higher than participants with the static format, while there was no significant difference between those with animated charts and static charts. 

Although \lives{} offer a more engaging experience by incorporating animations and narrations, they have certain limitations. 
Some users with animated charts and Live Charts commented that \lives{} take up more file space than a static image.
One user noted that \quo{Internet connection and speeds could be a potential drawback,} while another one reported that \quo{(\lives{}) need to have the computer power to display it and cannot just print on paper to display.}
The main reasons for these difficulties can be traced to the characteristics of videos, which consist of a sequence of images.
Additionally, some users with animated charts and Live Charts felt it was not easy to pinpoint the precise moment when a particular insight was discussed, while static charts can be understood at a glance.
This can result in the need to replay the \lives{} multiple times to locate the desired information.
Currently, commercial video players can add small tags to videos' timelines, which can assist users in locating specific content. Such players would be helpful in the playback of Live charts as well.


We also found varying individual preferences for data communication among users.
Although Live Charts was appreciated by a large number of participants for its ability to emphasize particular data points and help absorb the information, some participants pointed out that \lives{} may have limitations in highlighting only certain information, which could be viewed as a drawback, as one noted that \quo{You are more likely to only focus on the information that the narrator is pointing out rather than looking at the chart as a whole for yourself.}
Regarding the method of receiving information, individuals may also exhibit distinct preferences. While some participants enjoy taking an information journey through Live Charts, others prefer a less guided approach to understanding the data by themselves, as noted by one user who stated that \quo{Live Charts don't give people as much time to think on their own.}
The pace of \lives{} can also be divergent. Some participants feel that \quo{(\lives{}) are being explained too slowly.}, while others find that \quo{\lives{} may also go through the information too quickly, and people may miss information.}
Based on the feedback received, we anticipate the development of an authoring tool that would allow users to customize the creation of \lives{} by controlling the information division and the pace for different scenarios.
However, as our research primarily deals with the automatic generation of \lives{}, these concerns are not within the scope of our paper. We will delve deeper into these insightful remarks in the \autoref{sec: discussion_creation}.
\subsection{Expert Interview}
We conducted a series of interviews with three expert users from diverse fields to evaluate the effectiveness of our automatic generation workflow. The first expert (E1) has over three years of experience as a data journalist for a digital-news firm. 
\changed{The second expert (E2), possessing a master's degree in visual communication, has over five years of experience. She demonstrates her prowess as a UI/UX designer in a globally acclaimed software firm.}
The third expert (E3) is a senior researcher with over seven years of experience specializing in animation for data visualization.

We used an online meeting system to conduct the interviews, which began with an introduction to Live Charts followed by three examples. We then probed the experts on the quality of Live Charts and their methods for creating such charts. By introducing our automatic workflow, we solicited feedback on its effectiveness. Finally, we asked how it could potentially benefit their work and future endeavors.

% quality
All three experts agreed that our Live Charts were of high quality. 
They agreed that the narration logic and overall animation design in Live Charts are appropriate. Specifically, E3 believed that the progressive introduction of information is well-suited for public users, and E2 thought that the animation effects effectively complement the visual design.
E1 stated that \quo{We have previously utilized video formats in our articles, and Live Charts have achieved a similar level of quality to our past efforts.}
The experts also offered suggestions for improving Live Charts, with E1 suggesting that the narration could be more expressive and conversational in tone, rather than sounding too formal and robotic.

% workflow
Regarding our workflow for automatically generating Live Charts, all the experts appreciated it and found it helpful. They currently use different tools for creating such charts, with E1 adopting existing authoring tools in her company to generate videos, E2 relying on Adobe PR or AE for video editing, and E3 using D3 through coding. 
E1 mentioned that their company's tool generates videos from data, but it may have limitations in terms of chart style. In contrast, our method can handle charts with different styles, including those for online use.  
Additionally, E1 stressed the significance of data accuracy, which our method achieves through the use of models for extraction.
The experts acknowledged the value of our workflow but also offered suggestions for improving it. 
While E1 and E2 suggested incorporating an extension into familiar tools to avoid switching between applications, E1 also raised concerns about data privacy within the company and the risks associated with using external extensions.
Additionally, E1 and E3 recognized the potential for our workflow to transform data news into videos in the future, enabling them to consume data articles more efficiently while performing other tasks.
As an expert in visualization, E3 expressed a preference for a human-AI collaborative approach, in which the user inputs simple insight hints and animation types, and AI assists in making the charts more compact. 
\input{sections/5_discussion}
\section{Limitations and Open Questions}
\label{sec:limitations}
Though we have proposed two effective non-``detect-then-describe'' methods for 3D dense captioning, the captions do not have much diversity because of the limited text annotations, beam search, and self-critical sequence training with the CiDEr reward.
% 
We believe that multi-modal pre-training on 3D vision-language tasks with more training data and the utilization of \textbf{L}arge \textbf{L}anguage \textbf{M}odels(LLM) trained on large corpus would increase the diversity of the generated captions.
% 
Additionally, other reward functions designed for 3D dense captioning will increase the diversity among object descriptions in the same scene.
% 
We will leave these topics for future study.


\section{Conclusions}
\label{sec:conclusion}
%
\whatsnew{
In this work, we decouple the caption generation from caption generation, and propose a set of two transformer-based approaches, namely Vote2Cap-DETR and Vote2Cap-DETR++, for 3D dense captioning.
%
Comparing with the sophisticated and explicit relation modules in conventional ``detect-then-describe'' pipelines, our proposed methods efficiently capture the object-object and object-scene relation through the attention mechanism.
%
The preliminary model, Vote2Cap-DETR, decouples the decoding process to generate captions and box estimations in parallel.
% 
We also propose vote queries for fast convergence, and develop a novel lightweight query-driven caption head for informative caption generation.
% 
In the advanced model, Vote2Cap-DETR++, we further decouple the queries to capture task-specific features for object localization and description generation.
% 
Additionally, we introduce an iterative spatial refinement strategy for vote queries, and insert 3D spatial information for more accurate captions.
%
Extensive experiments on two widely used datasets validate that both the proposed methods surpass prior ``detect-then-describe'' pipelines by a large margin.
}

\ifCLASSOPTIONcompsoc
  \section*{Acknowledgments}
  The work was supported by National Key R\&D Program of China (2022YFE0137800), Key ``Pioneer'' R\&D Projects of Zhejiang Province (2023C01120), NSFC (U22A2032), and the Collaborative Innovation Center of Artificial Intelligence by MOE and Zhejiang Provincial Government (ZJU).
  The work was also partially supported by HK RGC GRF grant 16210722.
\else
  \section*{Acknowledgment}
\fi


\ifCLASSOPTIONcaptionsoff
  \newpage
\fi


\bibliographystyle{IEEEtran}
% argument is your BibTeX string definitions and bibliography database(s)
\bibliography{template}

\begin{IEEEbiography}[{\includegraphics[width=1.0in,height=1.25in, clip,keepaspectratio]{figures/bio/luying.jpg}}]{Lu Ying} is currently a Ph.D. candidate at the State Key Lab of CAD\&CG, Zhejiang University. Her main research interests are on data storytelling, glyph-based visualization. She is dedicated to integrating the AI technique into visualization to ease the creation of visualization. She received her BEng in Digital Media Technology from Zhejiang University. For more details, please refer to \url{https://yiyinyinguu.github.io/}.
\end{IEEEbiography}

\begin{IEEEbiography}
[{\includegraphics[width=1in,height=1.25in, clip,keepaspectratio]{figures/bio/yunwang.jpg}}]{Yun Wang} is a senior researcher in the Data, Knowledge, Intelligence (DKI) Area at Microsoft Research Asia. Her research lies in the intersection of Human-Computer Interaction (HCI), Information Visualization (VIS), Artificial Intelligence (AI), and Data Science (DS). Her work facilitates Human-Data Interaction, Human-AI Collaboration, and Data Storytelling through an interdisciplinary approach. For more details, please refer to \url{https://www.microsoft.com/en-us/research/people/wangyun/}.
\end{IEEEbiography}

\begin{IEEEbiography}[{\includegraphics[width=1.0in,height=1.25in, clip,keepaspectratio]{figures/bio/haotian.jpg}}]{Haotian Li} is currently a Ph.D. candidate in Computer Science and Engineering at the Hong Kong University of Science and Technology (HKUST). His main research interests are data visualization, visual analytics, human-computer interaction and online education. He received his BEng in Computer Engineering from HKUST. For more details, please refer to \url{https://haotian-li.com/}.
\end{IEEEbiography}

\begin{IEEEbiography}[{\includegraphics[width=1in,height=1.25in,clip,keepaspectratio]{figures/bio/shuguangdou.jpg}}]{Shuguang Dou}
is currently pursuing a Ph.D. degree with the College of Electronics and Information Engineering, at Tongji University, Shanghai, China. His research interests include Infographic understanding, person re-identification, NAS benchmark, X-ray, and remote sensing. He has published papers in ICLR, AAAI, TIP, TIFS, and TCSVT. He is a program member of ICML, NIPS, and ICLR. For more information, please visit \url{https://shuguang-52.github.io/}.
\end{IEEEbiography}

\begin{IEEEbiography}[{\includegraphics[width=1in,height=1.25in,clip,keepaspectratio]{figures/bio/jiang_xinyang.jpg}}]{Xinyang Jiang} is now a researcher from Microsoft Research Asia. Before joining MSRA, he was a researcher from Tencent Youtu Lab. He received B.E. from Zhejiang University in 2012 and a Ph.D. from Zhejiang University in 2017. His main research areas include cross-modal retrieval, computer vision, and pedestrian re-identification. He has published more than ten papers in CVPR, ICML, NIPS, ICLR,  ECCV, AAAI, ACMMM, TIP and other top conferences and journals on computer vision and artificial intelligence. He is a program member of AAAI, CVPR, MM and other conferences, and a reviewer for TCSVT, TIP and other journals. 
\end{IEEEbiography}

\begin{IEEEbiography}[{\includegraphics[width=1in,height=1.25in,clip,keepaspectratio]{figures/bio/haidongzhang.jpg}}]{Haidong Zhang} is currently a principal architect with Microsoft Research Asia. He received a Ph.D. degree in computer science from Peking University, China. His research interests include visualization and human-computer interaction.
\end{IEEEbiography}

\begin{IEEEbiography}[{\includegraphics[width=1.0in,height=1.25in, clip,keepaspectratio]{figures/bio/huamin.jpg}}]{Huamin Qu} is a chair professor in the Department of Computer Science and Engineering (CSE) at the Hong Kong University of Science and Technology (HKUST) and also the director of the interdisciplinary program office (IPO) of HKUST. He obtained a BS in Mathematics from Xi'an Jiaotong University, China, an MS and a PhD in Computer Science from the Stony Brook University. His main research interests are in visualization and human-computer interaction, with focuses on urban informatics, social network analysis, E-learning, text visualization, and explainable artificial intelligence (XAI). For more information, please visit \url{http://huamin.org/}.
\end{IEEEbiography}

\begin{IEEEbiography}[{\includegraphics[width=1in,height=1.25in,clip,keepaspectratio]{figures/bio/yingcaiwu.jpg}}]{Dr. Yingcai Wu} is a Professor at the State Key Lab of CAD\&CG, Zhejiang University.
His main research interests are information visualization and visual analytics, with focuses on urban computing, sports science, immersive visualization, and social media analysis. 
He received his Ph.D. degree in Computer Science from The Hong Kong University of Science and Technology. 
Prior to his current position, Dr. Wu was a postdoctoral researcher in the University of California, Davis from 2010 to 2012, and a researcher in Microsoft Research Asia from 2012 to 2015. 
For more information, please visit \url{http://www.ycwu.org}.
\end{IEEEbiography}

\end{document}


