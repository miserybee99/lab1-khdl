\section{Live Charts}
\label{sec: livecharts}
Since making Live Charts from static vector-based charts requires considerable effort, we aim to revive them into Live Charts automatically.

To enhance the chart expression, previous studies have emphasized the significance of annotations or textual cues to enhance user comprehension of charts~\cite{ren2017chartaccent}.
In contrast to traditional text cues near charts that divert visual attention, we utilize voiceover narration—freeing up visual focus while enhancing chart comprehension.
Moreover, prior research~\cite{wang2018narvis} highlights the effectiveness of breaking down complex information into smaller, more digestible chunks when presenting charts.
In this regard, we choose to integrate animations into Live Charts, which can assist users in visually processing intricate data and comprehending complex chart representations.
Furthermore, a multi-sensory experience by synchronizing narrations and animations within the Live Charts can complement each other by providing supplementary details, or indirectly link to each other by presenting data from different perspectives~\cite{cheng2022investigating}.
Therefore, we propose an approach that involves three steps:
\begin{enumerate}[noitemsep]
    \item[(1)] Using GNN-based techniques, we identify the data together with the visual encodings from the input SVG.
    \item[(2)] We derive the audio narration with insights and its corresponding data with LLMs.
    \item[(3)] For some expressions in narration, we design the corresponding animations and synchronize them to the audio.
\end{enumerate}

\begin{figure}[htbp]
    \centering
    \includegraphics{figures/pipeline.pdf}
    \caption{Our automatic approach for generating Live Charts.}
    \label{fig:pipeline}
\end{figure}

\changed{
We then describe each step in detail accompanied by an example depicted in \autoref{fig:ChartUnder}(a), which we refer to as the ``airport chart.''
}



