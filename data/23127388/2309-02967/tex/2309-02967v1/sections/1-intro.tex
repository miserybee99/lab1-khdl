\section{Introduction}

\IEEEPARstart{S}{tatistical} charts are widely used to deliver visually appealing and informative presentations of data.
They have been used in various fields, such as urban~\cite{zhao2022uncertainty}, education~\cite{zhang2022better}, and science~\cite{li2023causality}, to communicate data patterns~\cite{shirato2023identifying} and insights. 

However, static charts have limitations due to their unchanging nature. 
The absence of a clear reading order can confuse as all information is presented simultaneously, and users do not know where to start or how to locate the corresponding elements~\cite{amini2018hooked}.
Further, static charts have a limited capacity for conveying information~\cite{ren2017chartaccent}.
A chart containing excessive visual components or encodings may overwhelm users, leading to a visual burden. 
Moreover, static charts may not be engaging enough to maintain users' interest~\cite{chevalier2016animations}, reducing their effectiveness as a communication tool.
These phenomena emphasize the need for strategies to enhance data communication via charts.

There have been several studies aimed at enhancing static charts with various solutions. 
One approach is to break down the complex original chart information and introduce pieces of information gradually, making it easier for users to comprehend~\cite{deng2022revisiting}. 
For instance, recent research~\cite{wang2018narvis, wang2021animated} hierarchically decomposes a visualization or infographic design and introduces the compositions progressively.
Alternatively, some studies have attempted to augment static charts' expressiveness by including available information, such as adding question-answering~\cite{kahou2018figureqa}, annotations \cite{lai2020automatic} or captions \cite{mahmood2014automated, balajicharttexta}. 


\begin{figure*}[htbp]
    \centering
    \includegraphics{figures/teaser.pdf}
    \caption{An example of a Live Chart.
        (a1-a5) The sequential process of the Live Chart.
        (b1-b5) The animations in the chart.
        The texts below are the audio narration for the corresponding frames, with the first tag indicating the chart component and the insight type.}
    \label{fig:teaser}
\end{figure*}

In this paper, we introduce Live Charts, a new format that revitalizes static charts by integrating the advantages of simplifying complex information and incorporating explanatory details. 
Our approach employs a multi-sensory technique, including audio as an additional source of information for users. 
Live Charts present data sequentially, accompanied by audio narration that introduces visual encodings and elucidates the messages conveyed by the charts. 
For instance, when examining a static bar chart such as the one depicted in Fig.~\ref{fig:teaser}(a3), a Live Chart presentation would involve increasing the height of the red bars synchronously while narrating stockpile data about the United States. 
Adopting this format for data visualization, we explore how Live Charts may enable users to comprehend the presented information more easily and accurately.

Nevertheless, if manual creation is required for Live Charts, this new format of presentation may be deemed less useful due to the numerous steps involved, such as manually extracting data and visual encodings, designing animations, and crafting a chart narration that synchronizes with the animation. 
To address this concern, we further explore methods to automate the creation of Live Charts from static ones.
We break down the procedure into a series of computational methods that can be automated, thereby enhancing the efficiency of this new format.
\changed{We trained a Graph Neural Network (GNN) model to recover the embedded data and visual encodings of static charts on our own generated chart dataset.}
Subsequently, we aim to create the audio narration.
Given that data insights are frequently used to create narratives~\cite{srinivasan2019augmenting}, we expect our narration to contain a compelling data story showcased together with the Live Chart.
We employ large language models (LLMs) to uncover intriguing insights within the data and then generate the corresponding narration.
Ultimately, we integrate the narration with the static chart to create a Live Chart by animating relevant visual elements based on the extracted insights and corresponding narration.

We evaluate the effectiveness of our approach in different aspects. We first present a variety of real-world use cases that highlight the practical applications of our methods. 
Following that, we showcase the performance of our chart understanding model, which serves as the foundation for our automated technique.
Furthermore, we conduct a crowd-sourced user study to investigate our generated Live Charts and conduct expert interviews to further evaluate our automated approach.
Finally, we discuss the insights gained from our research and propose potential avenues for future exploration.


Our primary contribution is an automatic pipeline for reviving static charts into Live Charts.
The approach consists of 1) a method including a dual-stream GNN that recovers data in SVG charts and 2) an automatic enhancement from recovered data by incorporating narrations and animations, resulting in the generation of Live Charts.
To validate the efficacy of our approach, we conducted comprehensive evaluations.
The results provide evidence supporting the effectiveness of our chart revival approach and the resulting Live Chart format.