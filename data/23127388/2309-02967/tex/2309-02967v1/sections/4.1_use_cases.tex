\section{Evaluation}
In this section, we first present a set of use cases for reviving real-world charts into \lives{}.
We describe our evaluation process, which involved testing the performance of the understanding model and conducting a user study and expert interview to evaluate \lives{}.
Our evaluation results further confirm the effectiveness of our workflow and demonstrate the potential of the Live Chart format.

\subsection{Use Cases}
\label{sec: usecase}
To demonstrate the capabilities of our workflow in \autoref{sec: livecharts}, we present a set of Live Charts, including bar, line, and pie (donut) charts.
We collected chart SVGs from the Internet and ensured that they covered different topics and styles.
\autoref{fig:teaser} and \autoref{fig:cases} show part of the examples in our gallery~\footnote{The whole examples can be found in the supplementary materials.}.
Each chart is presented in a step-by-step image sequence from index 1 to 5, beginning with contextual information (title, axis, and legend) and followed by various insights.
We use an animation box for each insight depicted in the chart, using a dotted stroke (as shown in \autoref{fig:cases}), to indicate both the type of animation effect and the process by which the corresponding element is altered.
Below the image sequence, we provide the corresponding narration with a tag to indicate contextual or insightful information, followed by a specific description.


