\subsection{Expert Interview}
We conducted a series of interviews with three expert users from diverse fields to evaluate the effectiveness of our automatic generation workflow. The first expert (E1) has over three years of experience as a data journalist for a digital-news firm. 
\changed{The second expert (E2), possessing a master's degree in visual communication, has over five years of experience. She demonstrates her prowess as a UI/UX designer in a globally acclaimed software firm.}
The third expert (E3) is a senior researcher with over seven years of experience specializing in animation for data visualization.

We used an online meeting system to conduct the interviews, which began with an introduction to Live Charts followed by three examples. We then probed the experts on the quality of Live Charts and their methods for creating such charts. By introducing our automatic workflow, we solicited feedback on its effectiveness. Finally, we asked how it could potentially benefit their work and future endeavors.

% quality
All three experts agreed that our Live Charts were of high quality. 
They agreed that the narration logic and overall animation design in Live Charts are appropriate. Specifically, E3 believed that the progressive introduction of information is well-suited for public users, and E2 thought that the animation effects effectively complement the visual design.
E1 stated that \quo{We have previously utilized video formats in our articles, and Live Charts have achieved a similar level of quality to our past efforts.}
The experts also offered suggestions for improving Live Charts, with E1 suggesting that the narration could be more expressive and conversational in tone, rather than sounding too formal and robotic.

% workflow
Regarding our workflow for automatically generating Live Charts, all the experts appreciated it and found it helpful. They currently use different tools for creating such charts, with E1 adopting existing authoring tools in her company to generate videos, E2 relying on Adobe PR or AE for video editing, and E3 using D3 through coding. 
E1 mentioned that their company's tool generates videos from data, but it may have limitations in terms of chart style. In contrast, our method can handle charts with different styles, including those for online use.  
Additionally, E1 stressed the significance of data accuracy, which our method achieves through the use of models for extraction.
The experts acknowledged the value of our workflow but also offered suggestions for improving it. 
While E1 and E2 suggested incorporating an extension into familiar tools to avoid switching between applications, E1 also raised concerns about data privacy within the company and the risks associated with using external extensions.
Additionally, E1 and E3 recognized the potential for our workflow to transform data news into videos in the future, enabling them to consume data articles more efficiently while performing other tasks.
As an expert in visualization, E3 expressed a preference for a human-AI collaborative approach, in which the user inputs simple insight hints and animation types, and AI assists in making the charts more compact. 