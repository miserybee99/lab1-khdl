\subsection{User Study}

We further assessed the effectiveness of Live Charts generated by our approach through a user study with 90 participants. 
We conducted our study on Prolific~\footnote{https://www.prolific.co/}, a popular platform for research participation. 
This study aims to assess the following aspects:
\begin{itemize}
    \item Can \lives{} assist users in understanding charts?
    \item In what ways do animation and narration in \lives{} enhance the chart representation?
\end{itemize}
We describe the procedure and results in this section. 


\subsubsection{Participants} 
We used the Qualtrics survey platform to set up a crowd-sourcing experiment.
90 participants (29 females) were recruited from Prolific in our study. 
The majority of participants, comprising 42\%, fell into the age range of 25 to 34 years, followed by 27\% in the age range of 35 to 44 years.
As the task required writing, participants' proficiency in writing and English vocabulary could influence the outcomes. Thus, we limited the study to only English-speaking users who had previously received at least a 95\% approval rate on their results. We compensated each participant with \$4.9 for completing the 40-minute task. 


\subsubsection{Procedure} 
The study consisted of five parts. 
The first part involved explaining the study procedure and obtaining consent and demographic information from the participants. 
To ensure that our participants had a basic understanding of insights and visualization, we presented eight insight types that we used in our generation process, along with their definitions and examples.
Subsequently, we asked three questions, in which participants needed to observe a chart and answer a question related to three randomly selected insights out of a total of eight.
For example, one of the questions could be, ``What trend best describes the line?''
Only participants who answered these questions correctly were eligible to participate in the study.

To investigate the role of animations and narrations, we carried out a between-subject study. Participants were randomly divided into three groups and presented with varying chart formats: static, animated, and \lives{}.
Each participant was only shown one type of chart format.
For the third part, we demonstrated a bar, pie, and line chart, providing sample insights for each. 
For the fourth part, participants were tasked to provide insights for charts. The chart format used was the same as in the previous step. The six charts are presented randomly.
Participants were asked to fully explore the chart insights and then write down insights.
Moreover, participants with animated charts and Live Charts were required to remain on the page longer than the video length to ensure they viewed it at least once.
Additionally, for participants with Live Charts, we included task questions at random intervals to test the audio functionality, to verify that they can hear the accompanying narration.
we recorded the insights written by users and further calculated the sentence number of insights.

After completing the writing tasks, participants were asked to score the chart. 
They were required to rate the chart in five dimensions: insightful, memorable, focused attention, understandability, and enjoyment.
We used a seven-point Likert scale, with options ranging from strongly disagree (1) to strongly agree (7). 
Finally, participants were asked to provide a brief explanation for their rating.

\subsubsection{Stimuli Preparation}
We collected SVG charts from use cases (\autoref{sec: usecase}), covering a wide range of topics. 
Considering the potential for user fatigue during prolonged experimentation, we prepared a set of 9 charts for each user, with 3 charts for each chart type (bar, pie, and line). Among these, one chart of each type (3 in total) was used for demonstrating examples, and the remaining 6 charts (2 for each type) were used in the formal study.  
For each chart in the study, we prepared three different formats: a static chart, an animated chart, and a \live{}. The original SVG was used as the static version, while the \lives{} were generated using the techniques discussed in \autoref{sec: livecharts}. For the animated charts, we used the same animations as in the \lives{}.
We further shortened the intervals between each animation to one second because sorely removing the narrations from the live version may result in long intervals between animations, which could have bored users.

\subsubsection{Results}
\changed{This study gathered the Likert ratings from five perspectives.}
As all data are not normally distributed, we use the Kruskal-Walli test for each factor.
Additionally, we utilized the Mann-Whitney tests with Bonferroni correction to compare pairs within each factor and accounted for multiple comparisons. The sample means, and their corresponding confidence intervals are depicted in \autoref{fig:quatitative} and \autoref{table:quatitative}.

The Likert ratings, measured on a scale from 1 to 7, showed significant differences across all dimensions (p $<$ 0.05). 
Post-hoc testing revealed significantly higher scores for all dimensions in the live format compared to the static charts (p $<$ 0.05, $\left| r \right| > 0.3$). 
Furthermore, we found that Live Charts scored marginally significantly higher than animated charts in the \textit{Insightful} and \textit{Focused Attention} dimensions. Animated charts scored marginally significantly higher than static charts in the \textit{Understandability} and \textit{Enjoyment} dimensions.
The results indicate that Live Charts offer a compelling and memorable approach and effectively direct users' attention toward important information.


\begin{figure}[htbp]
\includegraphics{figures/quantitative_v2.pdf}
\centering
\caption{
The statistical analysis results for the overall evaluation include the corresponding Likert ratings. 
The means are depicted using shapes, and their confidence intervals are shown with error bars. On the right side are the specific numerical mean values.}
\label{fig:quatitative}
\end{figure}

\begin{table}[h]
\captionsetup[figure]{labelfont={bf},labelformat={default},labelsep=period,name={Fig.}}
\caption{\minor{Overall statistical evaluation result for each dimension, including the Chi-square values, p-values, degrees of freedom, and effect size. }}
\begin{tabular}{l|l|ll}
\toprule
\multicolumn{1}{c|}{Dimensions}   & Chart Format Pair   & p-value & \thead{effect\\size ($r$)} \\ 
\midrule
\multirow{3}{*}{\thead{Insightful \\ $\chi^2 (2)=8.05$, p=0.017}}      
                                   & Static - Animated & 0.182    & 0.17 \\
                                   & Static - Live     & \textbf{0.008}   & \textbf{0.34}        \\
                                   & Animated - Live   & 0.069    & -0.23       \\
\midrule
\multirow{3}{*}{\thead{Memorable \\ $\chi^2(2)=6.17$, p=0.045}}       
                                   & Static - Animated & 0.284    & 0.13 \\
                                   & Static - Live     & \textbf{0.018}   & \textbf{0.31}        \\
                                   & Animated - Live   & 0.120    & -0.20       \\
\midrule
\multirow{3}{*}{\thead{Focused attention \\ $\chi^2(2)=8.85$, p=0.012}} 
                                   & Static - Animated & 0.206    & 0.16\\
                                   & Static - Live     & \textbf{0.003}   & \textbf{0.38 }       \\
                                   & Animated - Live   & 0.087   & -0.22       \\
\midrule
\multirow{3}{*}{\thead{Understandability \\ $\chi^2(2)=6.36$, p=0.042}} 
                                   & Static - Animated & 0.060    & 0.24\\
                                   & Static - Live     & \textbf{0.017}   & \textbf{0.31}        \\
                                   & Animated - Live   & 0.733    & -0.04       \\
\midrule
\multirow{3}{*}{\thead{Enjoyment \\ $\chi^2(2)=6.18$, p=0.046}}         
                                   & Static - Animated & 0.086    & 0.22\\
                                   & Static - Live     & \textbf{0.013}   & \textbf{0.32}        \\
                                   & Animated - Live   & 0.758    & -0.04      \\
\bottomrule
\end{tabular}
\label{table:quatitative}
\end{table}

\subsubsection{Feedbacks}
Based on our observations, most participants in Live Charts provided positive feedback about Live Charts. 

Both participants in animated charts and Live Charts acknowledged the effectiveness of animation in conveying data effectively.
First, participants using the Live Charts format rated the charts significantly higher in terms of \textit{Understandability} than participants in the static charts format. 
Animated chart users thought that animations allow for breaking down complex information into smaller, more manageable parts, which in return increases comprehension.
Live Chart users also mentioned that \quo{(\lives{}) provide a way to communicate quantitative information in a way that is easier to
 understand for many people.}
In contrast, some participants in static charts found it challenging to absorb all the information presented in a static chart.
For example, one reported that \quo{A chart might be too complicated that I can't recognize all the patterns.}
Moreover, animations provide extra value in the form of accompanying text by extracting information from the static chart using a model. 
This feature was frequently mentioned as a drawback of static charts, with users reporting difficulty in predicting precise numbers.
Second, animated charts and Live Charts users highlighted that animation helps emphasize specific chart elements, transforming data into engaging and memorable stories. 
For example, one participant in Live Charts mentioned that \quo{The animation is also great for highlighting trends and turning the data into a memorable story,} while another participant in static charts wrote that \quo{(Static charts) do not allow highlighting of specific sections.}
Furthermore, users with animated and Live formats agreed that the formats were interesting, while participants with static charts found it dull to look at for a long period. This pattern was also supported in the Likert score for the \textit{Enjoyment} dimension.

Narration emerged as another crucial component, as users noted its ability to complement the animation by directing auditory attention to specific elements.
This pattern can be reflected in the higher \textit{Focused Attention} score in Live Charts.
Moreover, with narration, Live Charts can present specific insights more clearly and effectively, as a participant noted, \quo{(Live Charts) clearly explain what the chart shows.}
Users with animated charts also reinforced this fact by suggesting that \quo{Animated charts can sometimes make it more difficult to understand what is being shown, particularly when only using highlighting to emphasize certain aspects.}
This observation may explain the disparity in the higher score for the \textit{Insightful} dimension for Live Charts.
Furthermore, integrated with the \gpt{} model, the narrator can analyze the data table and combine it with relevant news or context. 
As noted by one user with Live Charts, \quo{There's more context with narration allowing for greater insights into the data and increasing understanding.}

In addition to the benefits of animation and narration on their own, we have also received feedback highlighting the advantages of combining the two.
Users with Live Charts reported that the interplay between animation and narration could enhance the overall experience.
One participant in Live Charts noted that \quo{It is interesting hearing voice while looking at the chart equally.} 
They also suggested that ensuring that the information presented in both the voice-over and the chart matches accurately is crucial.
Furthermore, they rated the \textit{Memorable} dimension significantly higher than participants with the static format, while there was no significant difference between those with animated charts and static charts. 

Although \lives{} offer a more engaging experience by incorporating animations and narrations, they have certain limitations. 
Some users with animated charts and Live Charts commented that \lives{} take up more file space than a static image.
One user noted that \quo{Internet connection and speeds could be a potential drawback,} while another one reported that \quo{(\lives{}) need to have the computer power to display it and cannot just print on paper to display.}
The main reasons for these difficulties can be traced to the characteristics of videos, which consist of a sequence of images.
Additionally, some users with animated charts and Live Charts felt it was not easy to pinpoint the precise moment when a particular insight was discussed, while static charts can be understood at a glance.
This can result in the need to replay the \lives{} multiple times to locate the desired information.
Currently, commercial video players can add small tags to videos' timelines, which can assist users in locating specific content. Such players would be helpful in the playback of Live charts as well.


We also found varying individual preferences for data communication among users.
Although Live Charts was appreciated by a large number of participants for its ability to emphasize particular data points and help absorb the information, some participants pointed out that \lives{} may have limitations in highlighting only certain information, which could be viewed as a drawback, as one noted that \quo{You are more likely to only focus on the information that the narrator is pointing out rather than looking at the chart as a whole for yourself.}
Regarding the method of receiving information, individuals may also exhibit distinct preferences. While some participants enjoy taking an information journey through Live Charts, others prefer a less guided approach to understanding the data by themselves, as noted by one user who stated that \quo{Live Charts don't give people as much time to think on their own.}
The pace of \lives{} can also be divergent. Some participants feel that \quo{(\lives{}) are being explained too slowly.}, while others find that \quo{\lives{} may also go through the information too quickly, and people may miss information.}
Based on the feedback received, we anticipate the development of an authoring tool that would allow users to customize the creation of \lives{} by controlling the information division and the pace for different scenarios.
However, as our research primarily deals with the automatic generation of \lives{}, these concerns are not within the scope of our paper. We will delve deeper into these insightful remarks in the \autoref{sec: discussion_creation}.