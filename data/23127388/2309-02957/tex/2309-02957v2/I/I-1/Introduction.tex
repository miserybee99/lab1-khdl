%    \comm{The introduction should focus on the importance of foregrounds in the context of CMB B-mode experiments and, in particular, on the issue of dust decorrelation. Review the most recent observational evidence regarding this effect and convince the reader about its crucial importance for experiments that aim at detecting $r$ at the level of 0.001 and below (S4, Litebird). Then introduce QUBIC and its ability to discriminate frequencies in-band and its unique potential in foregrounds characterization.}

This paper addresses one of the burning questions currently concerning the CMB community: Are there reliable strategies to validate or invalidate a detection of primordial \textit{B}-modes in the presence of complex, polarized Galactic foregrounds? The scope of our paper is to investigate a possible solution that exploits the spectral imaging capability of an unconventional technique for CMB polarimetry, called bolometric interferometry (BI), applied to control interstellar dust foreground emission residuals. 

Indeed, the next generation of satellites, like Litebird \citep{Hazumi_2019} and PICO \citep{hanany2019pico}, and ground-based experiments, like Simons Observatory \citep{Ade_2019} and CMB-S4 \citep{Abazajian_2022}, aim at improving the constraint on the tensor-to-scalar ratio, $r$, at the level of 0.001 and below. The accurate removal of foreground and instrumental systematic effects is already the main limiting factor.

To improve foreground removal, modern experiments are relying on multi-frequency observations and improved models of astrophysical emissions. For example, there are many PySM\footnote{\url{https://pysm3.readthedocs.io/en/latest/}} \citep{Thorne_2017} models that have been developed with the goal of simulating the effects of deviations from the single modified blackbody (MBB) emission conventionally assumed for the Galactic dust thermal emission.  The models \textbf{d5} and \textbf{d7} take into account different dust grain compositions \citep{Hensley_2017}, while the models \textbf{d4} and \textbf{d12} describe the dust emission as a sum of two or up to six single MBBs along each line-of-sight (LOS) \citep{Finkbeiner_1999, Martinez_Solaeche_2018}.

This article focuses on the \textbf{d6} model \citep{Vansyngel_2018}, which introduces LOS frequency decorrelation due to a frequency-varying polarization angle, which in turn is caused by a change both in the spectral energy distribution (SED) and in the magnetic field orientation along the LOS~\citep{tassis2015searching}.

This effect is usually quantified at the power spectra level by means of the correlation ratio, $R_{\ell}$, between two frequency maps \citep{planck_2017_decorrelation}. The most recent observational evidence regarding this effect comes from \citet{planck_2017_decorrelation, planck_2020_decorrelation, Pelgrims_2021, Ritacco_2023} and could affect polarimetric and spectral calibration in the case of wide beam instruments \citep{Masi_2021} as well as cause a bias on the tensor-to-scalar ratio \citep{McBride_2023, Hensley_2018}.

However, the \textbf{d6} model mimics the effect of a frequency-varying polarization angle, without making any physical assumptions on the misalignment of the underlying magnetic field, by randomly sampling a frequency-varying multiplication factor from a gaussian distribution, that is later applied to the single MBB emission, using the parametric expression of the correlation ratio derived in \citet{Vansyngel_2018}.

%{\color{red} It could be worth mentioning that, in practice, some level of in-band frequency resolution is available also to traditional imagers as typically the individual detectors that make up a given band do not have identical properties and this allows for a some level of a BI-like approach. Clearly, BI allows for a greater control and flexibility, and it's difficult to forecast the impact of this effect for future experiments without having access to the actual detectors.}

%If dust is indeed not featureless as it is usually assumed but exhibits spectral features, the possibility of reanalysing the data with a set of different spectral resolution and the ability to reanalyze the data are a key asset to detect biases coming from residuals caused by wrong or over simplified foreground models. 
%This \st{latter point} could be achieved by comparing results from different sky patches, as proposed by \citet{aurlien2022foreground}, \ELENIA{or by cross-checking with different component separation methods, such as parametric ones \citep{eriksen_2006, Stompor}, blind one \citep{Aumont_2007} or the moment expansion \citep{Chluba_2017, Vacher_2022}, some of which might be less sensitive to foreground mismodelling.} \ELENIA{\st{Add reference to papers suggested by Silvia about how to mitigate contamination with imagers}}

If dust does not behave as a simple MBB, as is usually assumed, but exhibits more complex spectral features, like frequency decorrelation, we need a method to detect the presence of foreground residuals in our results.
This could be achieved by comparing results from different sky patches, as proposed by \citet{aurlien2022foreground}, or by cross-checking with different component separation methods, such as parametric codes \citep{eriksen_2006, Stompor}, blind algorithms \citep{Aumont_2007} or codes based on moment expansion \citep{Chluba_2017, Vacher_2022}, some of which might be less sensitive to foreground incorrect modeling.

Another possibility, which we illustrate in this paper, is to use BI and its ability to discriminate frequencies in-band during data analysis. This allows us to achieve a spectral resolution of a few~GHz\footnote{Some level of in-band frequency sensitivity is actually achievable to traditional imagers by using the small variations in the spectral properties of different detectors. This was successfully applied to map the CO emission line~\citep{Planck_CO}.} and reanalyze the same data with different spectral configurations.  A variation in the constraint on $r$ between configurations suggests contamination in the tensor-to-scalar ratio due to component separation residuals.

%   \comm{State the paper objective: compare the performance of the most advanced experiment to come in detecting dust decorrelation with a similar experiment based on bolometric interferometry.} \\

In this paper, we investigate the advantage of BI for foreground removal and characterization by comparing the performance in detecting dust frequency decorrelation of one of the most advanced experiments to come, CMB-S4, with a similar, hypothetical experiment based on bolometric interferometry, that we name CMB-S4/BI. We perform a Monte-Carlo simulation starting from frequency maps, with or without band-splitting, and then we apply parametric component separation using two different component separation codes: FGBuster \citep{Stompor} and Commander \citep{eriksen_2006,eriksen_2008}. In the main body of this paper, we focus on FGBuster simulations, while we discuss the results obtained with Commander in Appendix~\ref{app:commander}.

Because the aim of this article is to propose a new methodology, we did not perform an actual map-making process from the Time-Ordered-Data, but we simulated the noise properties directly onto the reconstructed frequency maps, and we neglected the impact of instrumental systematic effects, such as: an imperfect knowledge of the spectral response of the instrument, an uncertainty about the Half-Wave Plate angle, or the feed-horn positions. 
Such effects could reduce the ability to perform band-splitting during data analysis. However, BI offers a specific approach to control instrumental systematic effects, the \textit{self-calibration} technique, that is inherited from radio-interferometers ~\citep{Bigot-Sazy2013}.


%   \comm{Summarize the main structure of the paper with some comments.} \\
The paper is organized as follows. In Sect.~\ref{sec_bolometric_interferometry_and_qubic} we provide a brief introduction to BI \citep[and references therein]{2020.QUBIC.PAPER1}. %For a complete and in-depth description we remind the reader these series of papers {\color{red} [cite Qubic papers]}.
Sect.~\ref{sec_methods} is dedicated to the description of the simulated sky models, instrumental configurations, and the Monte-Carlo pipeline based on the FGBuster \citep{Stompor} component separation code. In Sect.~\ref{sec_results} we compare the results in terms of tensor-to-scalar ratio reconstruction from simulations with conventional foreground models and with unaccounted Galactic dust LOS frequency decorrelation. Here we also describe a machine learning classification used to assess the ability to detect residuals from foreground emissions in a single realization. In Appendix~\ref{app:reconstruction_foregrouds} we present the results obtained with FGBuster regarding the estimation of foreground parameters, while in Appendix~\ref{app:commander} we discuss all the results obtained with Commander.  %In section \ref{sec_discussion} we further discuss our assumptions and compare BI to other spectral imaging techniques. Our conclusions and future developments are summarized in section \ref{sec_conclusions}.\hl{To be completed after the Results, Discussion and Conclusion sections are closed}


%   \comm{Summarize the main structure of the paper with some comments.} \\
%The paper is organized as follows. In section \ref{sec_bolometric_interferometry_and_qubic} we provide a brief introduction to Bolometric Interferometry and to the state-of-the-art, represented by the QUBIC experiment (Q\&U Bolometric Interferometer for Cosmology). For a complete and in-depth description we remind the reader these series of papers {\color{red} [cite Qubic papers]}. Section \ref{sec_methods} is dedicated to the description of the simulated sky models, instrumental configurations and the Monte-Carlo pipeline. In section \ref{sec_results} we compare the results obtained in the assumption of conventional foreground models with the ones in the case of unaccounted Galactic dust frequency decorrelations. Our results, conclusions and future developments are summarized in section \ref{sec_conclusions}.
