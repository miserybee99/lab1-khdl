In this paper, we have shown how bolometric interferometry (BI) has the potential to detect systematic effects caused by interstellar dust in CMB polarization measurements when LOS frequency decorrelation is present in dust emission and it is not accounted for in parametric component separation algorithms. 

We know that there are ways for imagers to mitigate the problem of not precisely knowing the foreground emission, for example, through cross-checking with different component separation methods, such as blind ones \citep{Aumont_2007}, or codes based on the moment expansion \citep{Chluba_2017, Vacher_2022} which might be less sensitive to incorrect foreground modeling. However, in this paper, we propose a new approach based on a different instrument architecture called bolometric interferometry (BI).  An instrument based on BI can be used as an independent verification to future claims of a \mbox{$B$-mode} detection by exploiting the superior purity of the $r$ measurement made possible by the increased spectral resolution.

We have carried out simulations with two component separation codes (FGBuster, discussed in the main text, and Commander, discussed in Appendix~\ref{app:commander}), reconstructing the tensor-to-scalar ratio, $r$, from simulated skies containing CMB, synchrotron and dust emission, and instrumental noise. For the dust emission, we used three models of increasing complexity, one of which contains frequency decorrelation.

We compared two instrument models, CMB-S4 and \mbox{CMB-S4/BI}, the latter being a modified version of CMB-S4 that accounts for the possibility of splitting each physical frequency band in a variable number of sub-bands that can be chosen during data analysis. This feature, which is unique to BI, allows us to assess whether a measurement of $r$ is biased by dust emission residuals or not. While a Fourier-transform spectrometer can be used to increase spectral resolution, it would suffer from a noise penalty compared to BI because it cannot observe all frequencies simultaneously.

Our results are consistent for the two codes and show that with no frequency decorrelation, both instruments yield the same performance (the final precision and systematic uncertainty on $r$ are similar). If decorrelation is present, and it is not accounted for in component separation, then an imager like CMB-S4 would measure a biased value of $r$. This bias can be reduced with \mbox{CMB-S4/BI} by reanalyzing the same data after splitting the band into an increasing number of sub-bands. 

The decrease of the measured $r$ with the number of sub-bands, $n_\mathrm{sub}$, clearly indicates the presence of a dust-induced systematic effect, given that without dust residuals the detected $r$ does not change with $n_\mathrm{sub}$. In such a situation, a classical imager would have no means of classifying the measurement as ``clean'' or ``biased''.

We tested the ability to detect biased $r$ measurements also using a machine learning approach, and we verified that assessing the variation of the $r$ measurement versus $n_\mathrm{sub}$ allowed us to classify clean and biased measurements with a rate $\gtrsim 97\%$.

%This paper focused on a new methodology, we did not consider instrumental systematic effects.

Future developments will test this technique in more realistic situations (representative noise, inclusion of optical effects, and uncertainty on the knowledge of the instrumental spectral response), assess the performance with various dust models, and explore new techniques of component separation, allowing us to separate signals taking into account instrumental effects in a more comprehensive and representative way.

%\optinSilvia{Paolo's comments follow.
%1.The main general issue for the average reader will be to understand why a BI is favored wrt an imager, despite of the fact that the spectrum of dust emission is basically featureless, so spectral resolution should not help significantly. For those who work with spectrometers, it is very evident that, if there are no lines/features in the spectrum, in order to measure the continuum it is good practice to reduce the spectral resolution, since high resolution comes with a penalty in terms of noise. This paper focuses on systematic effects rather than on SNR, but small systematics are usually hidden in the noise, so SNR is also important. From the point of view of the average scientist, the result of the paper is a surprise. He will want to understand in an intuitive way, beyond the quantitative results of the analysis of the simulations, why this happens. A simple, convincing, intuitive explanation of the result is lacking in the present form of the paper, and should be added, if there is one. \newline
%2. Also, the average reader will want to understand better how much this result is general, and how much it depends on the details of the dust model. Showing that the advantage of BI is correlated to the presence of spectral features in the emission spectrum (if true) would be a good way to convince the average reader that there is no bias in the comparison, and no flaws in the analysis. At least a comment on this should be added. \newline
%3. Another issue is: how much does the error in the knowledge of the spectral response of the system (for both BIs and Imagers) affect the results of the simulations? In this paper it is assumed that the bands and sub-bands profiles are known perfectly. In the real instruments this is not true, for both the imagers and the BI. Will this error be larger for the bands of the imagers or for the sub-bands of the BI? If the errors are larger for the BI, which is possible due to the stronger link between beam knowledge and spectral response, will this affect significantly the main conclusion of the paper ? A comment on this is important !}


%decorrelation In the case of model 1 and 2 which are quite simple models of foregrounds, the biases and precision on $r$ are similar in the case of a classical imager and a bolometric interferometer. On the other hand, in the case of a complex model with an effect of dust frequency, the bolometric interferometer allows to go further and check the coherence of the results obtained which a classical imager can not do. For these simple models, we see that we find a bias greater than that published in the paper CMB-S4. 


% \textbf{Proposition by Mathias :} \\ \\
% In this paper, we have shown the challenge of the component separation step and the need to have models consistent with the reality of the data. These models must be improved in order to better understand the astrophysical foregrounds and to better subtract them.
% 
% We have performed simulations to reproduce the results of CMB-S4 and we could see how bolometric interferometry could add value to these experiments and to future results. We use here a complete pipeline while the official paper describes a Fisher analysis. In the same spirit, several parameters such as apodization or E-to-B leakage are not mentioned which leads us to make a choice towards realist models which increase the bias on $r$.
% 
% Future studies will focus on new techniques of component separation, in particular through component map-making. This new technique allows to separate signals and to take into account the instrumental effects on the measurements and the reconstruction of the maps.
