The first instrument considered in our analysis is CMB-S4 \citep{Abazajian_2022}, which will observe at 9 different frequencies in the 20--280\,GHz range to constrain both synchrotron and thermal dust emissions. The goal of CMB-S4 will be the detection of $r$ at the level $r > 0.003$ with more than 5$\sigma$. %This precision is notably due to the delensing step, which will leave only $10$\% residuals which will considerably reduce $\sigma(r)$. This is shown in \cite{Abazajian_2022}.

The second instrument is a version of CMB-S4 based on bolometric interferometry (CMB-S4/BI), where each of the bolometer-based frequency bands, $\Delta\nu_i$ (i.e. above 85 GHz), can be subdivided into $n_\mathrm{sub}$ sub-bands of width: 

%This king of analysis allows to compare classical imager and bolometric interferometer for the main current CMB experiments. In this paper, we will focus only on CMB-S4 instrument. To do this, bolometric interferometry takes into account the relative bandwidth of the instrument and allows the reconstruction of sub-bands within the broad bands. 

\begin{equation}
    \Delta \nu^\mathrm{BI}_i = \frac{\Delta \nu_i}{n_\mathrm{sub}}.
    \label{bandwidth}
\end{equation}

If we now consider $m$ frequency bands of CMB-S4, each one subdivided in $n_\mathrm{sub}$ sub-bands in CMB-S4/BI we can calculate the sensitivity in each sub-band as:

\begin{equation}
    \sigma^\mathrm{BI}_{j,i} = \sigma_j \times \sqrt{n_\mathrm{sub}} \times \varepsilon,
    \label{sigma}
\end{equation}
%
where $\sigma_j$ is the CMB-S4 sensitivity in the $j$-th sub-band within $i$-th physical band, $n_\mathrm{sub}$ is the number of sub-bands and $\varepsilon$ is a parameter introduced to account for the sub-optimality of bolometric interferometry \cite[for further details about BI sub-optimality see][]{2020.QUBIC.PAPER2}.

Two approximations have been done regarding the instrument models:
\begin{enumerate}
 \item The noise is always assumed to be white, although, in \mbox{CMB-S4/BI}, we have added the multiplicative term $\varepsilon$ to account for the sub-optimality of bolometric interferometry. We know that the noise of a bolometric interferometer is not entirely white, and this calls for specific component separation techniques able to deal with correlated noise. These techniques are currently under development within the QUBIC collaboration;
 \item We have neglected the angular resolution of the optical beam to be consistent with the CMB-S4 reference paper. The angular resolution of a traditional imager, such as CMB-S4, is set by the aperture of the telescope, whereas in the BI case this is set by the largest distance between 
 horns. Although the contribution of the physical beam affects the final sensitivity of both instruments, it should not impact the generality of our results.
\end{enumerate}

Figure~\ref{fig:Polarization_depth} shows the bandwidths and sensitivities of some of the tested experimental configurations. For each CMB-S4 frequency interval above 85\,GHz, we have studied seven configurations of CMB-S4/BI, with $n_\mathrm{sub}$ ranging from 2 to 8. Increasing the number of sub-bands results in a sensitivity degradation, as indicated in Eq.~(\ref{sigma}), with $\varepsilon$ ranging between 20\% and 60\%, according to \cite{2020.QUBIC.PAPER2}. Since we focus on dust decorrelation, we have not subdivided the synchrotron frequency bands, so that the first three intervals of the various configurations overlap. Note that because the simulated CMB-S4 sky patch is centered far away from the Galactic plane, we expect the correlations between dust and synchrotron to be negligible for the scope of our study by following \cite{Krachmalnicoff_2018} and~\cite{planck2017}.

%{\color{red} Even if you are mainly interested in dust properties, there may be correlations between dust and synchrotron, in which case applying spectral imaging also to lower frequencies could be relevant. You are looking at regions far from the plane, so synch will be low, but maybe it's worth commenting on this?}


We emphasize that because this band-splitting is performed at the data analysis level, one can explore various values of the number of sub-bands $n_\mathrm{sub}$ with the same dataset. Studying the evolution of the resulting constraints as a function of $n_\mathrm{sub}$ is the core of this study.

% An observation made by a conventional imager and a bolometric interferometer of equivalent sensitivity is the same. The equation \ref{bandwidth} is the equality between shows the relationship between the wide bandwidth and each sub-band width for a given $i$ sub-band. The addition of all sub-band widths forms the total width of the wide band. The equation \ref{sigma} shows how the noise becomes more important in the case of a bolometric interferometer. Because each band receives less photon, the noise grows as $\sqrt{n}$ with $n$ the number of sub-bands. We added an $\varepsilon$ term which correspond to the sub-optimality of bolometric interferometry demonstrate by \cite{2020.QUBIC.PAPER2}. For this study, we considered white noise scaled with the right level of the experiment. In the futur, we will explore the results of this paper in the case of the QUBIC instrument with all its complexities.

\begin{figure}
    \centering
    \includegraphics[width=9cm]{figure/S4-BI_experimental_configuration_v2.pdf}
    \caption{Polarization sensitivity of CMB-S4 and three examples of \mbox{CMB-S4/BI}, with $n_\mathrm{sub}=3,5,7$ respectively. Note that the bands of the three lowest frequency channels are identical for all the instruments. Because our study focuses on dust decorrelation we have chosen not to split the bandwidths of the synchrotron channels.}
    \label{fig:Polarization_depth}
\end{figure}



