\begin{figure*}[h!]
    \centering
    \includegraphics[width=\textwidth]{figure/figlike.png}
    \caption{{\bf a)} Histograms of the reconstructed $r$ in the case of r=0 and the {\bf d1s1} model (smoothed using Kernel Density Estimator) for a number of sub-dbands ranging from 1 (classical imager - CMB-S4 current design) to 8 using bolometric interferometry.  {\bf b)} Same but with r=0.01 and the {\bf d1s1} model. {\bf c)} Same but with r=0 and the {\bf d6s1} model where Dust decorrelation (with $\ell_{corr}=10$) induces a significant bias that reduces with the number of sub-bands.
    {\bf d)} Evolution of the mean and standard-deviation of the histograms for {\bf d1s1} and {\bf d6s1}.}
    \label{fig:like_d1}
\end{figure*}

 The first model we explore here is the simplest one where spectral indices for thermal dust and synchrotron radiation are constant across the sky. This corresponds to {\bf d0s0}, model 1 of the section~\ref{sec_simulated_sky_model}. The simplicity of this model result makes it computationally efficient. \\

\begin{table}[H]
	\centering
	\caption{Recovered parameters for {\bf d0s0}:}
	\label{tab:params_simple}
	\begin{tabular}{lccr} % 2 columns, alignment for each
		\hline
		$r$ bias & $3.0 \times 10^{-5}$ \\ 
		$\sigma(r)$ & $4.998 \times 10^{-4}$ \\ 
		Dust spectral index $\beta_d$ & $1.540 \pm 2.13 \times 10^{-3}$ \\
		Synchrotron spectral index $\beta_s$ & $-3.000 \pm 7.43 \times 10^{-4}$ \\
		\hline
	\end{tabular}
\end{table}

We performed simulations with 500 realizations of CMB and noise. Table~\ref{tab:params_simple} summarizes the main results of the pipeline in this simple case with constant spectral indices across the sky. We obtain an unbiased estimate of the spectral indices, resulting in low foreground residuals in our maps.
As a result, we find a very small bias on $r$ with a standard deviation very close to that found in \cite{Abazajian_2022} with the same configuration. 

