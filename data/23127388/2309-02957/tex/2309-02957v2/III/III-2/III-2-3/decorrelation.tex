Let us now consider a more complex dust model that exhibits frequency decorrelation for the dust component ({\bf d6s1}) as in Eq.~\ref{Rl}. This is a more realistic situation than in the previous section and the non-trivial frequency behavior of the dust SED is expected to induce dust residuals in the clean CMB maps when using a component separation parametrization corresponding to {\bf d1s1} that assumes no decorrelation for the dust component (a simple modified black-body).

We have applied the pipeline described above on 500 realizations of model {\bf d6s1}  with $\ell_{\text{corr}} = 10$ (with frequency correlation matrix shown in figure~\ref{fig:Dust_d6_corrl10_correlation_matrix_over_frequency}). As in the previous section, we use our clean CMB maps to reconstruct a value of $r$ for each realization and for each number of sub-bands ($N_{sub}=1$ for a classical imager and $N_{sub}>1$ for a BI). The results are displayed in panel {\bf c)} from figure~\ref{fig:like_d1}. We observe that the range of reconstructed $r$ is dramatically increased by the dust frequency decorrelation, even though we have used a conservative amount of decorrelation with respect to the best current upper-limits from Planck \citep{planck_2020_decorrelation, Planck_2020}. 
For comparison, we show in panel {\bf b)} from figure~\ref{fig:like_d1} the reconstructed vaules of $r$ for the {\bf d1s1} but with an input $r=0.01$.

The {\bf d)}  panel of figure~\ref{fig:like_d1} shows a comparison of the averages and standard deviations of our reconstructed r as a function of the number of sub-bands $N_{sub}$ for the three cases considered here ({\bf d1s1}, $r=0$), ({\bf d1s1}, $r=0.01$), and ({\bf d6s1}, $r=0$). 
For a classical imager (CMB-S4, $N_{sub}=1$), the range of reconstructed $r$, when the input is zero, goes from $7.1\times 10^{-4}\pm 3.2\times 10^{-4}$ ({\bf d1s1}) to $1.3\times 10^{-2}\pm 2\times 10^{-2}$ when unaccounted dust decorrelation is present preventing any detection of primordial B-modes with r below $3\times 10^{-2}$, close to the current upper-limit~\citep{2021PhRvL.127o1301A}.

However, we observe that when using a BI with the same data containing unaccounted foreground decorrelation, the possibility of splitting into several sub-bands at the data analysis level reveals a significant decrease of the average bias and standard deviation of the reconstructed $r$ as one increases the number of reconstructed sub-bands. This reduction of the reconstructed $r$ is not seen with a simpler dust model, be it with $r=0$ or $r=0.01$ where one sees no evolution with the number of reconstructed sub-bands. A similar behavior is shown in figure~\ref{fig:result_r_vs_dust_model} where we vary the dust correlation length. We observe that the reduction of the bias on r due to uncontrolled foreground residuals always decreases when increasing the number of sub-bands. Of course, as expected, the overall amount of bias reduces when increasing the dust correlation length, reaching a similar situation as for {\bf d1s1} for $\ell_{corr} \sim 100$.

The possibility of analysing the same data with a varying number of reconstruction sub-bands, and the observed reduction of the bias when doing so, open the door to using Bolometric Interferometry  for 
controlling foreground residuals contamination in the case of complex dust models, for which a single, deterministic SED is a too crude approximation. However, the behavior observed in figures~\ref{fig:like_d1} and~\ref{fig:result_r_vs_dust_model} is 
found to be significant on the average over a large number of noisy CMB and dust realisations, which is not accessible in the real world. One therefore needs to assess whether the distinct evolution of the reconstructed $r$ with the number of sub-bands a case with no dust residuals (flat evolution) and unaccounted dust residuals (decrease) is significant on a single realization, which the what an experiment has access to.


%  The results of these simulations are shown in the figure~\ref{fig:result_r_vs_dust_model} and shows the results on the recovered best fit value of r for all the simulated instrumental configurations and dust models. Whereas in the d0 and d1 case the results are compatible between all instruments and show no bias, unaccounted dust frequency decorrelation contaminates the CMB signal and can cause a bias on r. However, the increased spectral resolution provided by BI helps reducing the bias and for correlation length $\ell_{corr} > 10$ it even removes it.

% Moreover, when adding spectral imaging, the recovered value of r changes significantly with respect to the single band case (the blue point of BI7 are not compatible with the red ones of S4 for correlation length of the order of 10), thus hinting that the recovered r is not a cosmological signal but the residual due to improper foreground removal.

% Finally, when increasing the correlation length, thus decreasing the deviation from a single-modified black-body, one can see that the results are consistent with the d1 case, as expected in the limit of an infinite $\ell_{corr}$.


% \begin{figure}
%     \centering
%     \includegraphics[scale=0.5]{figure/d6_recon_r.png}
%     \caption{Likelihoods on $r$ for $N_{\text{sub}}$ = 1, 3, 5 and 8. \comm{Elenia - Why is this likelihood not gaussian (is it obtained from the average Dls)? Are we able to justify it? I suggest to substitute this plot with the one of the average log-like instead, which is fully gaussian}}

%     \comm{Answer - I'm ploting the distribution of each max liklelihood for each reals. I tought that we decided to plot this kind of results.}
%     \label{fig:like_d6_corr15}
% \end{figure}


% \begin{figure}
%     \centering
%     \includegraphics[scale=0.30]{figure/rmax_d6_corr15.pdf}
%     \caption{Positions of maximum likelihoods on $r$ (dots) and their $\sigma(r)$ (arrow towards up) for $N_{\text{sub}}$ = 1, 3, 5 and 8.}
%     \label{fig:like_d6_corr15}
% \end{figure}

\begin{figure}
    \centering
    \includegraphics[width=0.5\textwidth]{figure/Results_on_r_vs_dust_model.pdf}
    \caption{Results on the tensor-to-scalar ratio as a function of the simulated dust model for CMB-S4 and the three BI versions of CMB-S4.  These results indicate that the increased spectral resolution of BI allows one to identify the presence of dust residuals due to frequency decorrelation effects in a way unattenable for a traditional imager. \JCH{redo this plot with histo of the r (Elenia). Also we need to use $\ell$ instead of $l$ + increase the size of the characters. Additionnaly, for now the point do not seem consistent with thos from figure~\ref{fig:like_d1}. SO we need to check this in details.}}
    \label{fig:result_r_vs_dust_model}
\end{figure}

