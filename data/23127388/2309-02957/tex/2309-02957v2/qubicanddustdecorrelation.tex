%                                                                 aa.dem
% AA vers. 9.1, LaTeX class for Astronomy & Astrophysics
% demonstration file
%                                                       (c) EDP Sciences
%-----------------------------------------------------------------------
%
%\documentclass[referee]{aa} % for a referee version
%\documentclass[onecolumn]{aa} % for a paper on 1 column  
%\documentclass[longauth]{aa} % for the long lists of affiliations 
%\documentclass[letter]{aa} % for the letters 
%\documentclass[bibyear]{aa} % if the references are not structured 
%                              according to the author-year natbib style

%
\documentclass{aa}  

\newcommand{\JCH}[1]{{\textcolor[rgb]{0, 0, 1}{#1 (JCH)}}}
\newcommand{\MATHIAS}[1]{{\textcolor[rgb]{0, 1, 0}{#1 (Mathias)}}}
\newcommand{\ELENIA}[1]{{\textcolor[rgb]{1, 0, 1}{#1 (Elenia)}}}
\newcommand{\optinMDP}[1]{{\textcolor[rgb]{1, 0.5, 0}{\textbf{#1 (Marco De Petris)}}}}
\newcommand{\optinLM}[1]{{\textcolor[rgb]{1, 0.5, 0}{\textbf{#1 (Louise Mousset)}}}}
\newcommand{\optinCOS}[1]{{\textcolor[rgb]{1, 0.5, 0}{\textbf{#1 (Creidhe O'Sullivan)}}}}
\newcommand{\optinN}[1]{{\textcolor[rgb]{1, 0.5, 0}{\textbf{#1 (Nahuel)}}}}
\newcommand{\optinSilvia}[1]{{\textcolor[rgb]{1, 0.5, 0}{\textbf{#1 (Silvia Masi)}}}}
\newcommand{\commentsreferee}[1]{{\textbf{#1}}}
\usepackage{ulem}


\usepackage{times}
\usepackage[utf8]{inputenc} %
\usepackage[T1]{fontenc}    %
\usepackage{url}            %
\usepackage{booktabs}       %
\usepackage{nicefrac}       %
\usepackage{microtype}      %
\usepackage{graphics,color}

\usepackage{algorithm}
\usepackage{enumitem}

\usepackage{amsfonts}       %
\usepackage{amsmath}       %
\usepackage{amssymb}


\newcommand{\Figref}[1]{Figure~\ref{#1}}  %
\newcommand{\figref}[1]{Fig.~\ref{#1}}    %
\newcommand{\Algoref}[1]{Algorithm~\ref{#1}}  %
\newcommand{\algoref}[1]{Alg.~\ref{#1}}    %
\newcommand{\Tabref}[1]{Table~\ref{#1}}
\newcommand{\tabref}[1]{Tab.~\ref{#1}}
\newcommand{\Eqnref}[1]{Equation~\ref{#1}}
\newcommand{\eqnref}[1]{Eq.~\ref{#1}} %
\newcommand{\eqnpref}[1]{(Eq.~\ref{#1})} %
\newcommand{\Secref}[1]{Sec.~\ref{#1}} %
\newcommand{\secref}[1]{Sec.~\ref{#1}} %
\newcommand{\suppref}[1]{Suppl.~\ref{#1}}
\newcommand{\fs}[1]{{\bf #1}}

\usepackage{xspace}
\makeatletter
\DeclareRobustCommand\onedot{\futurelet\@let@token\@onedot}
\def\@onedot{\ifx\@let@token.\else.\null\fi\xspace}
\makeatother

\newcommand{\eg}{e.g\onedot}
\newcommand{\ie}{i.e\onedot}
\newcommand{\cf}{cf\onedot}
\newcommand{\etc}{etc\onedot}
\newcommand{\wrt}{w.r.t\onedot}

\newcommand\Tstrut{\rule{0pt}{2.4ex}}
\newcommand\hlinespace{\hline\Tstrut}    %

\usepackage{xr-hyper}
\makeatletter
\newcommand*{\addFileDependency}[1]{%
  \typeout{(#1)}
  \@addtofilelist{#1}
  \IfFileExists{#1}{}{\typeout{No file #1.}}
}
\makeatother
\newcommand*{\myexternaldocument}[1]{%
    \externaldocument{#1}%
    \addFileDependency{#1.tex}%
    \addFileDependency{#1.aux}%
  }

\usepackage{xcolor}
\definecolor{ourblue}{rgb}{0.368,0.507,0.71}
\definecolor{ourorange}{rgb}{0.881,0.611,0.142}
\definecolor{ourgreen}{rgb}{0.56,0.692,0.195}
\definecolor{ourred}{rgb}{0.923,0.386,0.209}
\definecolor{ourviolet}{rgb}{0.528,0.471,0.701}
\definecolor{ourbrown}{rgb}{0.772,0.432,0.102}
\definecolor{ourlightblue}{rgb}{0.364,0.619,0.782}
\definecolor{ourdarkgreen}{rgb}{0.572,0.586,0.}

\definecolor{ourred2}{rgb}{0.84,0.15,0.16}
\definecolor{ourorange2}{rgb}{1,0.5,0.05}
\definecolor{ourblue2}{rgb}{0.12,0.47,0.71}
\definecolor{ourgreen2}{rgb}{0.17,0.63,0.17}
\definecolor{ourgray2}{rgb}{0.82,0.82,0.82}
\definecolor{ourgreen2}{rgb}{0.29,0.62,0.224}




\usepackage{subcaption}
\graphicspath{../figs/}
\newcommand{\myparagraph}[1]{%
    \par\noindent\hspace*{1em}\textbf{#1}\hspace{0.1em}%
}

\newcommand{\limittorque}{\ensuremath{\tau_{i}^{\mathrm{lim}}}\xspace}
\newcommand{\smooth}{\ensuremath{(u - u_{\mathrm{prev}})^{2}}\xspace}
\newcommand{\weight}{\ensuremath{w}}
\newcommand{\cost}{\ensuremath{c}}

\newcommand{\fgrf}{\ensuremath{F_{j}^{\mathrm{GRF}}}\xspace}
\newcommand{\numbermus}{\ensuremath{N_{\mathrm{active}}}\xspace}

\newcommand{\hyfydy}{Hyfydy\xspace}
\newcommand{\mujoco}{MuJoCo\xspace}
\newcommand{\planarmodel}{\textit{H0918}\xspace}
\newcommand{\threedmodel}{\textit{H1622}\xspace}
\newcommand{\complexmodel}{\textit{H2190}\xspace}
\newcommand{\myoleg}{\textit{MyoLeg}\xspace}

\usepackage{pifont}%
\newcommand{\cmark}{\ding{51}}%
\newcommand{\xmark}{\ding{55}}%

%
%%%%%%%%%%%%%%%%%%%%%%%%%%%%%%%%%%%%%%%%
%\usepackage[options]{hyperref}
% To add links in your PDF file, use the package "hyperref"
% with options according to your LaTeX or PDFLaTeX drivers.
%
\begin{document} 

   \title{Identifying frequency decorrelated dust residuals in B-mode maps by exploiting the spectral capability of bolometric interferometry}
   \titlerunning{Identifying frequency decorrelated dust residuals in B-mode maps}

    \author{M.~Regnier\inst{1}
        \and
        E.~Manzan\inst{2,3}
        \and
        J-Ch.~Hamilton\inst{1}
        \and
        A.~Mennella\inst{2,3}
        \and
        J.~Errard\inst{1}
        \and
        L.~Zapelli\inst{2,3}
        \and
        S.A.~Torchinsky\inst{1,4}
        \and
        S.~Paradiso\inst{5,6}
        \and
        E.~Battistelli \inst{8}
        \and
        P.~De Bernardis\inst{8}
        \and
        L.~Colombo \inst{2}
        \and
        M.~De Petris \inst{8}
        \and
        G.~D’Alessandro \inst{8}
        \and
        B.~Garcia \inst{11}
        \and
        M.~Gervasi \inst{10}
        \and
        S.~Masi \inst{8}
        \and
        L.~Mousset \inst{7}
        \and
        N.~Miron Granese \inst{13, 14, 15}
        \and
        C.~O’Sullivan \inst{9}
        \and
        M.~Piat \inst{1}
        \and
        E.~Rasztocky \inst{12}
        \and
        G.E~Romero \inst{12}
        \and
        C.G.~Scoccola \inst{13, 14}
        \and
        M.~Zannoni \inst{10}}

\institute{Université Paris Cité, CNRS, Astroparticule et Cosmologie, F-75013 Paris, France
        \and
            Università degli studi di Milano, Italy
        \and 
            INFN sezione di Milano, 20133 Milano, Italy
        \and
            Université PSL, Observatoire de Paris, AstroParticule et Cosmologie, F-75013 Paris, France
        \and
            Waterloo Centre for Astrophysics, University of Waterloo, Waterloo, ON N2L 3G1, Canada
        \and
            Department of Physics and Astronomy, University of Waterloo, Waterloo, ON N2L 3G1, Canada
        \and
            Institut de Recherche en Astrophysique et Planetologie, Toulouse (CNRS-INSU), France
        \and
            Universita di Roma - La Sapienza, Italy
        \and
            National University of Ireland, Maynooth, Ireland
        \and
            Università di Milano – Bicocca and INFN Milano-Bicocca
        \and
            ITeDA-Mza.(CNEA, CONICET, UNSAM)
        \and
            Instituto Argentino de Radioastronomía (CCT La Plata, CONICET; CICPBA; UNLP), Buenos Aires, Argentina
        \and
            Consejo Nacional de Investigaciones Científicas y Técnicas (CONICET), Godoy Cruz 2290, Ciudad de Buenos Aires C1425FQB, Argentina
        \and
            Facultad de Ciencias Astronómicas y Geofísicas, Universidad Nacional de La Plata, Paseo del Bosque, La Plata B1900FWA, Buenos Aires, Argentina
        \and 
            Universidad de Buenos Aires, Facultad de Ciencias Exactas y Naturales, Departamento de Física, Intendente Güiraldes 2160, Ciudad Universitaria, Ciudad de Buenos Aires C1428EGA, Argentina
}


   \date{Received September 15, 1896; accepted March 16, 2197}

 
  \abstract
    {Astrophysical polarized foregrounds represent the most critical challenge in Cosmic Microwave Background (CMB) $B$-mode experiments, requiring multi-frequency observations to constrain astrophysical foregrounds and isolate the CMB signal. However, recent observations indicate that foreground emission may be more complex than anticipated. Not properly accounting for these complexities during component separation can lead to a bias in the recovered tensor-to-scalar ratio.}  
    {In this paper we investigate how the increased spectral resolution provided by band-splitting in bolometric interferometry (BI) through a technique called \textit{spectral imaging} can help control the foreground contamination in the case of unaccounted Galactic dust frequency decorrelation along the line-of-sight (LOS).} 
    {We focus on the next-generation ground-based CMB experiment CMB-S4 and compare its anticipated sensitivity, frequency, and sky coverage with a hypothetical version of the same experiment based on bolometric interferometry (CMB-S4/BI). We perform a Monte-Carlo analysis based on parametric component separation methods (FGBuster and Commander) and compute the likelihood of the recovered tensor-to-scalar ratio, $r$.}
    {The main result is that spectral imaging allows us to detect systematic uncertainties on $r$ from frequency decorrelation when this effect is not accounted for in component separation. Conversely, an imager like CMB-S4 would detect a biased value of $r$ and would be unable to spot the presence of a systematic effect. We find a similar result in the reconstruction of the dust spectral index, where we show that with BI we can measure more precisely the dust spectral index also when frequency decorrelation is present and not accounted for in component separation.} 
    {The in-band frequency resolution provided by BI allows us to identify dust LOS frequency decorrelation residuals where an imager of similar performance would fail. This opens the prospect of exploiting this potential in the context of future CMB polarization experiments that will be challenged by complex foregrounds in their quest for $B$-modes detection.}

    

   \keywords{cosmic microwave background --
             inflation -- ISM -- data analysis}

   \maketitle
%
%-------------------------------------------------------------------

\section{Introduction}
\label{sec_introduction}

    %IoT systems are subject to different kinds of failures, either being internally generated %from the system 
%or caused by the surrounding environment. Such failures may affect not only the correctness of the system but also the environment in which they operate. 

%Assigned to DAVIDE 
IoT systems may experience different failures, either internally generated or caused by the surrounding environment~\cite{KIRCHHOF2022111087}. Such failures may affect not only the correctness of the system, but also the environment in which it operates. Consider, for instance, Smart Irrigation Systems: they monitor parameters related to weather and soil to irrigate crop fields based on the data collected automatically. A failure affecting the behaviour of those IoT systems may cause a waste of water or loss to the farm's production. Since IoT systems are composed of components of different natures (e.g., temperature/humidity sensors, cloud servers, and irrigation units), studying how failures (e.g., caused by a malfunctioning component) may propagate within a system and impact its behaviour can be highly challenging, further than being of high importance \cite{thesisFelicien,challengesIOTFLA}.

Developing IoT systems is complex due to several reasons. The integration of diverse components, the need to handle real-time data, and the distributed nature of IoT systems are just a few factors contributing to this complexity. Several modeling approaches have been proposed over the last few years to raise the level of abstraction (e.g., \cite{MDE4IoT, CAPS, KIRCHHOF2022111087, MonitorIoT}), promoting the adoption of models for increasing automation and easing analysis. These models help understanding systems behaviour, performance, and potential failure scenarios~\cite{ihirwe2021towards}.

Failure-Logic Analysis (FLA)~\cite{Gallina} is one of the analyses that can be applied to IoT systems. By using FLA, it is possible to define how a component's failure logic shall behave, which can help analyze how failures could potentially spread throughout a system and predict any potential issue. For FLA to work correctly, it is important to have accurate information about how failures may occur within each  component and propagate between components.
FLA relies on the manual specification of rules, that rigorously indicate the different kinds of failures that might occur and how they can propagate throughout the components. Specifying such rules is a strenuous and error-prone process, as identifying all possible fault scenarios and formulating accurate rules is challenging, possibly leading to incomplete or incorrect specifications.

In this paper, we propose to adopt testing methodologies to mitigate the issues related to the \emph{completeness} and \emph{correctness} of the manually specified FLA rules. By systematically introducing failures into an IoT system and running test cases, it is possible to observe how failures propagate. The collected evidence is then used to add, refine, and eliminate FLA rules, better capturing the behavior of the system in failure scenarios.

The paper is organized as follows: In Section \ref{sec:motivatingExample} we provide motivation for this work and present an explanatory example. We describe our approach in Section~\ref{sec:proprosedApproach}. Section~\ref{sec:evaluation} reports a preliminary evaluation of the proposed approach. In Section~\ref{sec:relatedWork}, we discuss related work. Finally, Section~\ref{sec:conclusion} concludes the paper and outlines future work.




\section{Bolometric interferometry in a nutshell}
\label{sec_bolometric_interferometry_and_qubic}

    In this section we briefly describe the principles of BI, focusing on a specific feature of this technique, called \textit{spectral imaging}, which is at the heart of our study. 
    The interested reader can find more details on BI and \textit{spectral imaging} in \cite{2020.QUBIC.PAPER1, 2020.QUBIC.PAPER2}, whereas more information about the QUBIC experiment, currently the only one based on BI, and on its laboratory characterization can be found in \cite{2020.QUBIC.PAPER3, 2020.QUBIC.PAPER4, 2020.QUBIC.PAPER5, 2020.QUBIC.PAPER6, 2020.QUBIC.PAPER7, 2020.QUBIC.PAPER8}.

    \subsection{Principles of bolometric interferometry}
    \label{sec_bolometric_interferometry}
    
        Bolometric interferometry is a technique that combines the use of bolometers, which are state-of-the-art wide-band cryogenic detectors providing high sensitivity, with the advantage of precision control of systematic effects provided by the \textit{self-calibration} technique, commonly used in radio-interferometry \citep{1981MNRAS.196.1067C}. The application to BI is detailed in \citet{Bigot-Sazy2013}.

Figure~\ref{fig:BI_instrumental_concept} shows a schematic of the QUBIC instrument, highlighting the fundamentals of BI. The sky signal enters the cryostat through an aperture window and propagates through a series of filters, a step-rotating half-wave plate, a polarizing grid, and an array of paired back-to-back feed-horn antennas. The back horns directly illuminate an optical combiner, which focuses the radiation onto two focal planes through a dichroic plate. 

When the instrument observes a distant point source along the optical axis an interference pattern forms on the two focal planes (see the top panel of Fig.~\ref{fig:Theo_SB}). As a result, each focal plane element measures the sky signal convolved by a specific beam pattern, called the \textit{synthesized beam}, shown in the bottom panel of Fig.~\ref{fig:Theo_SB}. The constructive or destructive interference of the incoming signal defines a series of peaks and nulls, with properties that depend on the signal wavelength, $\lambda$, on the number of horns along the maximum axis of the antennas array, $P$, and on the separation between two consecutive horns, $\Delta h$, as follows \citep{2020.QUBIC.PAPER2}:

\begin{equation}
    \label{eq_synth_beam_properties_dependece}
    \theta_\mathrm{FWHM}  = \frac{\lambda}{(P-1)\Delta h},\,\,\,\,\,\Theta = \frac{\lambda}{\Delta h},
\end{equation}
% 
where $\theta_\mathrm{FWHM}$ is the half power width of the peaks and $\Theta$ is the angular distance between the main peak and the first secondary peak.

Equation~\ref{eq_synth_beam_properties_dependece} demonstrates that the positions of the secondary peaks depend on $\lambda$. As an example, in the bottom panel of Fig.~\ref{fig:Theo_SB} we show a cut of the synthesized beam at a fixed azimuthal angle for two frequencies: 140\,GHz and 160\,GHz. Knowing how the multiple-peaked shape of the synthesized beam evolves with frequency allows us to recover the sky signal during data analysis at various frequencies within the physical band. This is possible as long as the two frequencies, $\nu_1$ and $\nu_2$, are far enough apart that the secondary peaks are well-resolved.  That is, we require $\Theta(\nu_2) -  \Theta(\nu_1) > \theta_\mathrm{FWHM} (\sqrt{\nu_1\nu_2})$, which occurs for $\frac{\Delta \nu}{\nu} \geq \frac{1}{P-1}$. We call this technique \textit{spectral imaging}.

Our goal is to reconstruct maps of an extended source in polarization thereby computing the three Stokes parameters I, Q, and U at the same time. Because an extended source is a linear combination of point-sources, this reconstruction is possible but requires deconvolving from the multiple peaks of the synthesized beam, as well as relying on a half-wave plate modulation for polarization reconstruction.
This problem can be solved thanks to a scanning strategy which allows information to be captured several times with various geometrical configurations\footnote{Similar to grism spectroscopy that benefits from different orientations of the field of view.}, and through an inverse problem approach that reconstructs unbiased maps of the three Stokes parameters in sub-bands within the physical band of the instrument ~\citep{2020.QUBIC.PAPER2}.

%\textbf{\st{Cosmic fluctuations in the sky form an extended source, which can be interpreted as a linear combination of point sources. In addition, scanning strategies allow information to be captured several times with various geometrical configurations to constrain observations\footnote{Similar to grism spectroscopy benefit of different field of view}.}}

%Because each detector on the focal plane is illuminated by all the feed-horns simultaneously, the received signal is the sky convolved with the angular response of all the antennas, which is called \textit{synthesized beam}.


%As a result, the position of the secondary peaks evolves with frequency and allows us to recover the sky signal during data analysis for frequencies far enough that the same secondary peak at two different frequency is well-resolved, namely 

%A unique feature of bolometric interferometry is thus \textit{spectral imaging}, which is the ability to recover the sky signal at different frequencies within the physical band through a precise knowledge of how the multiple-peaked shape of the synthesized beam evolves with frequency.

Consequently, the frequency dependence of the secondary peaks enables us to achieve a spectral resolution of a few GHz within the physical band. Furthermore, since spectral imaging occurs at the data analysis level, it allows us to re-analyze the same data with different spectral configurations, which can help us detect biases in the obtained results. This is a unique asset compared to traditional imagers, which would need several focal planes coupled to multichroic filters to achieve the same spectral performance, or to Fourier-transform spectrometers, which would suffer from a noise penalty related to not observing all frequencies simultaneously.

% The dependence of secondary peaks on frequency brings two advantages: on the one hand, it allows us to achieve a spectral resolution of a few GHz, which is unfeasible \JCH{[\bf Here, I don't really like this formulation: it is not that it is unfeasible for a traditional imager. One can do it, but the number of photons will be significantly reduced for this set of detectors... So we need to find a better formulation.]} for a traditional imager; on the other hand, 

In this context, our aim is to investigate how the increased spectral resolution provided by BI helps in controlling the contamination from Galactic foregrounds in the quest for primordial $B$-modes detection, with a special focus on the Galactic dust emission.

\begin{figure}[ht]
	\includegraphics[width=9cm]{II/II-1/QUBIC_scheme.pdf}
        \caption{\label{fig:BI_instrumental_concept}Schematic of the QUBIC instrument showing the principle of bolometric interferometry. The sky signal is received by an array of back-to-back horns and re-imaged onto the bolometric focal planes where the field interferes additively. A polarizer and a rotating half-wave plate make the instrument sensitive to linear polarization.}
\end{figure}

\begin{figure}[ht]
    \centering
    \includegraphics[width=7cm]{figure/Interference_uv_plane.pdf} \\
    \includegraphics[width=9cm]{figure/Synthesized_beam_theoretical.pdf}
	%\includegraphics[width=9cm]{II/II-1/tmp_Synthesized_beam_theoretical.pdf}
        \caption{\label{fig:Theo_SB} \textit{\textbf{Top} panel}: simulation of the interference pattern on the focal plane generated by a monochromatic point source. \textit{\textbf{Bottom} panel}: azimuth cut of the theoretical synthesized beam (solid lines) at 140 GHz (blue line) and at 160 GHz (green line) for a detector at the center of the focal plane. Dashed lines represent the beam pattern of a single feed horn. The frequency-dependent position of the secondary peaks is clearly visible.}
\end{figure}



    
    %\subsection{The QUBIC instrument}
    %\label{sec_qubic}
    
        %\st{QUBIC is the first CMB B-mode experiment based on bolometric interferometry. The instrument observes the sky in two frequency bands, centered at 150~GHz and 220~GHz, respectively, with a 25$\%$ bandwidth.} \st{Each feed-horn array is made of 400+400 back-to-back antennas and each focal plane is equipped with 992 TES bolometers.} \st{QUBIC will observe from the Alto Chorrillos site, in the Argentinian Andes, at about 5000 meters a.s.l. and will provide an upper limit on $r<0.015$ at $68\%$ C.L. after three years of observations}% \citep{2020.QUBIC.PAPER1}.

\st{The first QUBIC prototype is referred to as the  \textit{technological demonstrator} (TD), a reduced version of the instrument to demonstrate bolometric interferometry with laboratory measurements and sky observations. The QUBIC-TD uses the same cryostat} \citep{Masi_2022}, HWP \citep{D_Alessandro_2022} \st{and polarizing grid as the final instrument, but will observe only in the 150 GHz channel. It is equipped with smaller diameter mirrors, a smaller feed-horn array made of 64+64 back-to-back feed-horns, and a smaller focal plane, made of 248 bolometers.}

\st{After extensive laboratory testing the QUBIC-TD was installed at the observation site during November 2022} (see Fig.~\ref{fig:QUBIC_Rendering}). \st{Routine observations will start after commissioning.} \st{The interested reader can find the details of the laboratory tests and their results in:}% \citet{2020.QUBIC.PAPER1, 2020.QUBIC.PAPER2, 2020.QUBIC.PAPER3, 2020.QUBIC.PAPER4, 2020.QUBIC.PAPER5, 2020.QUBIC.PAPER6, 2020.QUBIC.PAPER7, 2020.QUBIC.PAPER8}.

\st{The top-left panel of figure}~\ref{fig:SB_meas_vs_theo} \citep[taken from][]{2020.QUBIC.PAPER1} \st{shows the synthesized beam measured by one of the focal plane TES detectors compared to a simulation of the same beam (top-right panel). The bottom panel of the same figure} \citep[taken from][]{2020.QUBIC.PAPER3} \st{shows the synthesized beams measured at various frequencies. These data show the multiple-peaked shape of the synthesized beam with good agreement with the theoretical predictions and the expected frequency dependence of the secondary peak positions. The measured instrumental noise (detectors and read-out) is $2.06\times 10^{-16}\,\mathrm{W}/\sqrt{\mathrm{Hz}}$} \citep{2020.QUBIC.PAPER4} \st{and the median cross-polarization of the detectors is $0.12\%$ }\citep{2020.QUBIC.PAPER6}.


%\begin{figure}[ht]
%	\includegraphics[width=9cm]{II/II-2/Qubic_on_site.png} %II/II-2/
%        \caption{QUBIC-TD in the Alto Chorrillos site. The cryostat is placed on a movable mount to scan in azimuth and elevation. The system is placed inside a motor-controlled dome that can be opened and closed.}
%     \label{fig:QUBIC_Rendering}
%\end{figure}
% 
\begin{figure}[ht]
	\includegraphics[width=9cm]{II/II-2/SB_data_vs_theory_new.png}\\
	\mbox{}\\
	\includegraphics[width=9cm]{figure/SB_vs_frequency_mod.pdf}
        \caption{\label{fig:SB_meas_vs_theo}\textit{Top panel}: synthesized beam measured for one detector at 150\,GHz (left) compared with the  simulated 150\,GHz synthesized beam for the same detector \citep[right, from][]{2020.QUBIC.PAPER1}. \textit{Bottom panel}: synthesized beam for one detector measured at various frequencies \citep[from][]{2020.QUBIC.PAPER3}.}
     
\end{figure}



 
\section{Dust decorrelation with bolometric interferometry and direct imaging}

\label{sec_detecting_dust_decorrelation}

    This paper aims to quantify the effect of various dust models with increasing complexity on the component separation results and demonstrate the benefits of spectral imaging in this regard. We focus, in particular, on the LOS frequency decorrelation of thermal dust, a phenomenon already observed in Planck data \citep{Pelgrims_2021}. 
    
    To quantify dust decorrelation, we follow \citet{planck_2020_decorrelation}, and use the quantity $\mathcal{R}_\ell$, defined in Eq.~(\ref{Rl}): 
    
    \begin{equation}
        \mathcal{R}_{\ell}^{\nu_1 \times \nu_2} = \frac{\mathcal{C}_{\ell}^{\nu_1 \times \nu_2}}{\sqrt{\mathcal{C}_{\ell}^{\nu_1 \times \nu_1} \times \mathcal{C}_{\ell}^{\nu_2 \times \nu_2}}}.
        \label{Rl}
    \end{equation}
    
    $\mathcal{R}_\ell$ is the ratio between the crossed spectrum between two frequencies, $\nu_1$ and $\nu_2$, and the square root of the product of the auto-spectra at these same frequencies. This ratio is close to 1 for completely correlated thermal dust. In our sky simulations, we can increase or decrease the level of complexity in the thermal dust spectral energy density (SED) by tuning $\mathcal{R}_{\ell}$ to a value farther or closer to one, thanks to the parametric expression of $\mathcal{R}_{\ell}$ derived in Eq.~(14) of \citet{Vansyngel_2018}.
    
    To assess the potential of BI, we compare the component separation performance of CMB-S4 to a BI version of the same experiment having the same sensitivity per unit bandwidth, but allowing for a higher spectral resolution through band-splitting using spectral imaging. In the following subsections, we present the methods used for this comparison.


    \subsection{Methods}
    \label{sec_methods}

        \subsubsection{Simulated sky}
        \label{sec_simulated_sky_model}
            Our sky model contains the CMB plus synchrotron and dust emission foregrounds. %\ELENIA{Is it ok to use to past here?}

We simulated the CMB using angular power spectra provided by the \texttt{fgbuster} package that are based on the latest Planck 2018 results\footnote{Spectra can be accessed at \protect\url{https://github.com/fgbuster/fgbuster/tree/master/fgbuster/templates}}. We used the following two FITS files: 
\begin{enumerate}[(i)]
\item \texttt{\small{Cls\_Planck2018\_lensed\_scalar.fits}} in which $B$-modes are considered with $r=0$ and lensing, 
\item \texttt{\small{Cls\_Planck2018\_unlensed\_scalar\_and\_tensor\_r1.fits}} in which $B$-modes are considered with $r=1$ and no lensing. 
\end{enumerate}


In our simulations, we used $TT$, $EE$, and $TE$ spectra taken directly from the file (i), while the $BB$ spectrum was obtained by summing the $BB$ spectrum from the file (i) multiplied by a lensing residual of 0.1 with the $BB$ spectrum from the file (ii) multiplied by the value of $r$ (either 0 or 0.006). Note that such a simplified approach neglects the additional tensor contribution to the $TT$, $TE$, and $EE$ spectra, but is sufficient in our case, as we only perform the likelihood analysis on the $BB$ spectrum.

%{\color{red} This is not full self-consistent when r /= 0, as you should also add the corresponding tensor mode contribution to TT, TE, and EE. Given that you perform the likelihood analysis only on B modes, this is not relevant, but it'd be useful to point it out.}

For the foregrounds we considered the following models\footnote{See \protect\url{https://pysm3.readthedocs.io/en/latest/#models} for more details about the models.}:
% We simulated the CMB with \texttt{camb} \citep{Lewis:1999bs} and \texttt{synfast} \citep{Gorski_2005} using the set of cosmological parameters defined in \hl{Put a table with the used parameters}. The foregrounds were simulated using the PySM package \citep{Thorne_2017}, exploring three different foreground models\footnote{See \url{https://pysm3.readthedocs.io/en/latest/#models} for more details about the models.}:


% \begin{table}
%     \caption{\label{tab:cosmopar}}
%     \begin{center}
%         \begin{tabular}{l c}
%          \hline
%          $H_0$ &\\
%          $\Omega_\mathrm{b}h^2$ &\\
%          $\Omega_\mathrm{c}h^2$ &\\
%          $m_\nu$ &\\
%          $\Omega_k$ &\\
%          $\tau$ & \\
%          $A_\mathrm{s}$ &\\
%          $n_\mathrm{s}$ &\\
%          
%         \end{tabular}
% 
%     \end{center}
% 
% \end{table}



\begin{enumerate}
    \item model \textbf{d0s0}, which assumes a single modified black-body (MBB) emission for the thermal dust and a power-law emission for the synchrotron with no curvature, with constant dust spectral index across the sky, $\beta_\mathrm{d} = 1.54$, dust temperature, $T_\mathrm{d} = 20$\,K, and synchrotron spectral index, $\beta_\mathrm{s} = -3$;

    %{\color{red} From pysm webpage, it's not clear to me that s1 is also based on the post-processing of Commander results? }
    \item model \textbf{d1s1}, derived from the Planck data post-processed with the Commander code \citep{Planck_15} for the dust emission, while the synchrotron emission is taken from the Haslam data at 408 MHz in \cite{remazeilles2015improved}, \cite{Haslam82}. The thermal dust emission is modeled as a modified black body with spatially varying temperature and spectral index projected on the sky, while the synchrotron emission is modeled as a power-law with spatially varying spectral index with no curvature;
    \item model \textbf{d6s1}. This model is derived from \textbf{d1s1} with the introduction of LOS frequency decorrelation in the dust emission following the statistical approach described in Eq.~(14) of \citet{Vansyngel_2018}. 
 %and tune it in ranges consistent with current observations \citep{planck_2020_decorrelation,Pelgrims_2021}.
\end{enumerate}

Whereas models \textbf{d0s0} and \textbf{d1s1} are fixed realizations, the model \textbf{d6s1} results in a random realization of the SED. For each simulated frequency, the MBB emission is multiplied by a randomly sampled decorrelation factor that mimics the effect of a frequency-varying polarization angle without making any physical assumptions on the underlying Galactic magnetic field. The magnitude of the decorrelation factor is governed by the correlation length, $\ell_\mathrm{corr}$, a parameter that can be set in PySM. Figure~\ref{fig:Dust_d6_SED_dispersion} displays the dispersion of various SED realizations as a function of $\ell_\mathrm{corr}$, showing that the dispersion increases with a shorter correlation length.

% An example of observations is done in figure \ref{fig:SED} where we show the twos last physical bands of CMB-S4. We show here how mainly the thermal dust is distributed along frequency. This figure show mainly how bolometric interferometry can convert raw sensitivity to spectral resolution. \\

%Whereas models \textbf{d0s0} and \textbf{d1s1} are fixed realizations, model \textbf{d6s1} results in a random realization of the SED : for each simulated frequency, the MBB emission is multiplied by a randomly sampled decorrelation factor that mimics the effect of a frequency-varying polarization angle without making any physical assumption on the underlying Galactic magnetic field. The magnitude of the decorrelation factor is determined by the correlation length $\ell_\mathrm{corr}$. Figure~\ref{fig:Dust_d6_SED_dispersion} displays the dispersion of various SED realizations as a function of $\ell_\mathrm{corr}$, showing that the dispersion increases with a shorter correlation length.

In our simulations, we explore the effect of dust LOS frequency decorrelation with a level of decorrelation consistent with current observations. Specifically, the range of correlation lengths used in our study is $\ell_\mathrm{corr}\geq 10$, which corresponds to a decorrelation level below $5\%$ for all the simulated frequencies. This configuration represents a conservative scenario with respect to the decorrelation level measured by Planck \citep{planck_2017_decorrelation,planck_2020_decorrelation} in the same multipole range considered in our work ($\ell \leq 300$ $-$ see Fig.~\ref{fig:Correlation_ratio_Planck_vs_pysm_models} for a comparison with Planck estimates).


% \begin{figure*}
%     \centering
%     \includegraphics[scale=0.8]{figure/SED_d0.pdf}
%     \caption{Spectral Energy Distribution for Q components with physical observed bands.}
%     \label{fig:SED}
% \end{figure*}

% Model description
\begin{figure}
    \centering
    % OLD IMAGE
    %\includegraphics[width=9cm]{figure/Dust_frequency_decorrelation_SED_statistical_dispersion_500_realizations_v4.pdf}
    
    % NEW IMAGE (LOUISE"S SUGGESTION)
    \includegraphics[width=9cm]{figure/Dust_frequency_decorrelation_SED_statistical_dispersion_500_realizations_v4_normalized.pdf}
    \caption{Dispersion of the dust SED for different correlation lengths of the PySM \textbf{d6} model normalized by the single MBB emission (\textbf{d1} model).
    The colored areas represent the statistical deviation from an MBB for a given correlation length, evaluated over 500 realizations.
    }
    \label{fig:Dust_d6_SED_dispersion}
\end{figure}

% \begin{figure}
%     \centering
%     \includegraphics[width=0.5\textwidth]{figure/Correlation_matrix_d6_corrl10.pdf}
%     \caption{Correlation matrix of the \textbf{d6} model for $\ell_\mathrm{corr} = 10$, corresponding to a level of decorrelation below $5\%$ for all the simulated frequencies.}
%     \label{fig:Dust_d6_corrl10_correlation_matrix_over_frequency}
% \end{figure}

\begin{figure}
    \centering
    \includegraphics[width=0.5\textwidth]{figure/Correlation_ratio_Planck_pysm_models_500iter_no_title_v3.pdf}
    \caption{Correlation ratio measured by Planck from the Half Mission (HM) maps at 217 GHz and 353 GHz, compared to the simulated ratio using PySM dust and CMB templates at the same frequencies. The dots in blue and orange represent the expected $R_{\ell}$ for the CMB and a single MBB dust emission, with constant (\textbf{d0}) or varying (\textbf{d1}) spectral indices pixel-by-pixel. Note that the dots are so close that they overlap in the figure. The green envelope shows the range of $R_{\ell}$ obtained from 500 realizations of dust LOS frequency decorrelation with $\ell_\mathrm{corr} = 10$. The black dots are from Fig.~2 of \citet{planck_2017_decorrelation}, the gray dots are from Fig.~B.2 of \citet{planck_2020_decorrelation}, and the red point has been obtained from the values in the middle plot of the second row in Fig.~18 of \citet{planck_2020_decorrelation}.}
    \label{fig:Correlation_ratio_Planck_vs_pysm_models}
\end{figure}


    %\caption{Correlation ratio measured by Planck from the Half Mission (HM) maps at 217 GHz and 353 GHz, compared to the same ratio obtained from a simulation with the PySM CMB template plus the dust models \textbf{d0}, \textbf{d1} and various realizations of the \textbf{d6} with $\ell_\mathrm{corr} = 10$. \textbf{Note that the blue and orange dots are so close that they overlap in the figure.} Black points are from Fig.~2 of \citet{planck_2017_decorrelation}, gray points are from Fig.~B.2 of \citet{planck_2020_decorrelation}, the red point has been obtained from the values in the middle plot of the second row in Fig.~18 of \citet{planck_2020_decorrelation}.}
    
%The CMB part is generated from a template of the B-mode spectrum, the levels of r and lensing can be variable. The main configuration is $r=0$ and $A_L = 0.1$.
%{\color{red} [Here I think we should be more precise: if we only show the case $r=0$ and $A_L = 0.1$, then just say it, otherwise the reader would expect to alse see results from other configurations. Unless we decide to also show results with different values of $r$ and $A_L$. In that case a table could summarize the info pretty well].}

            
        \subsubsection{Instrument models}
        \label{sec_instrument_models}
            The first instrument considered in our analysis is CMB-S4 \citep{Abazajian_2022}, which will observe at 9 different frequencies in the 20--280\,GHz range to constrain both synchrotron and thermal dust emissions. The goal of CMB-S4 will be the detection of $r$ at the level $r > 0.003$ with more than 5$\sigma$. %This precision is notably due to the delensing step, which will leave only $10$\% residuals which will considerably reduce $\sigma(r)$. This is shown in \cite{Abazajian_2022}.

The second instrument is a version of CMB-S4 based on bolometric interferometry (CMB-S4/BI), where each of the bolometer-based frequency bands, $\Delta\nu_i$ (i.e. above 85 GHz), can be subdivided into $n_\mathrm{sub}$ sub-bands of width: 

%This king of analysis allows to compare classical imager and bolometric interferometer for the main current CMB experiments. In this paper, we will focus only on CMB-S4 instrument. To do this, bolometric interferometry takes into account the relative bandwidth of the instrument and allows the reconstruction of sub-bands within the broad bands. 

\begin{equation}
    \Delta \nu^\mathrm{BI}_i = \frac{\Delta \nu_i}{n_\mathrm{sub}}.
    \label{bandwidth}
\end{equation}

If we now consider $m$ frequency bands of CMB-S4, each one subdivided in $n_\mathrm{sub}$ sub-bands in CMB-S4/BI we can calculate the sensitivity in each sub-band as:

\begin{equation}
    \sigma^\mathrm{BI}_{j,i} = \sigma_j \times \sqrt{n_\mathrm{sub}} \times \varepsilon,
    \label{sigma}
\end{equation}
%
where $\sigma_j$ is the CMB-S4 sensitivity in the $j$-th sub-band within $i$-th physical band, $n_\mathrm{sub}$ is the number of sub-bands and $\varepsilon$ is a parameter introduced to account for the sub-optimality of bolometric interferometry \cite[for further details about BI sub-optimality see][]{2020.QUBIC.PAPER2}.

Two approximations have been done regarding the instrument models:
\begin{enumerate}
 \item The noise is always assumed to be white, although, in \mbox{CMB-S4/BI}, we have added the multiplicative term $\varepsilon$ to account for the sub-optimality of bolometric interferometry. We know that the noise of a bolometric interferometer is not entirely white, and this calls for specific component separation techniques able to deal with correlated noise. These techniques are currently under development within the QUBIC collaboration;
 \item We have neglected the angular resolution of the optical beam to be consistent with the CMB-S4 reference paper. The angular resolution of a traditional imager, such as CMB-S4, is set by the aperture of the telescope, whereas in the BI case this is set by the largest distance between 
 horns. Although the contribution of the physical beam affects the final sensitivity of both instruments, it should not impact the generality of our results.
\end{enumerate}

Figure~\ref{fig:Polarization_depth} shows the bandwidths and sensitivities of some of the tested experimental configurations. For each CMB-S4 frequency interval above 85\,GHz, we have studied seven configurations of CMB-S4/BI, with $n_\mathrm{sub}$ ranging from 2 to 8. Increasing the number of sub-bands results in a sensitivity degradation, as indicated in Eq.~(\ref{sigma}), with $\varepsilon$ ranging between 20\% and 60\%, according to \cite{2020.QUBIC.PAPER2}. Since we focus on dust decorrelation, we have not subdivided the synchrotron frequency bands, so that the first three intervals of the various configurations overlap. Note that because the simulated CMB-S4 sky patch is centered far away from the Galactic plane, we expect the correlations between dust and synchrotron to be negligible for the scope of our study by following \cite{Krachmalnicoff_2018} and~\cite{planck2017}.

%{\color{red} Even if you are mainly interested in dust properties, there may be correlations between dust and synchrotron, in which case applying spectral imaging also to lower frequencies could be relevant. You are looking at regions far from the plane, so synch will be low, but maybe it's worth commenting on this?}


We emphasize that because this band-splitting is performed at the data analysis level, one can explore various values of the number of sub-bands $n_\mathrm{sub}$ with the same dataset. Studying the evolution of the resulting constraints as a function of $n_\mathrm{sub}$ is the core of this study.

% An observation made by a conventional imager and a bolometric interferometer of equivalent sensitivity is the same. The equation \ref{bandwidth} is the equality between shows the relationship between the wide bandwidth and each sub-band width for a given $i$ sub-band. The addition of all sub-band widths forms the total width of the wide band. The equation \ref{sigma} shows how the noise becomes more important in the case of a bolometric interferometer. Because each band receives less photon, the noise grows as $\sqrt{n}$ with $n$ the number of sub-bands. We added an $\varepsilon$ term which correspond to the sub-optimality of bolometric interferometry demonstrate by \cite{2020.QUBIC.PAPER2}. For this study, we considered white noise scaled with the right level of the experiment. In the futur, we will explore the results of this paper in the case of the QUBIC instrument with all its complexities.

\begin{figure}
    \centering
    \includegraphics[width=9cm]{figure/S4-BI_experimental_configuration_v2.pdf}
    \caption{Polarization sensitivity of CMB-S4 and three examples of \mbox{CMB-S4/BI}, with $n_\mathrm{sub}=3,5,7$ respectively. Note that the bands of the three lowest frequency channels are identical for all the instruments. Because our study focuses on dust decorrelation we have chosen not to split the bandwidths of the synchrotron channels.}
    \label{fig:Polarization_depth}
\end{figure}



 
        
        \subsubsection{Simulation pipeline}
        \label{sec_simulation_pipeline}

        We describe here the simulation pipeline for the Monte-Carlo analysis that we performed using the FGBuster parametric component separation code. In Appendix~\ref{app:commander_pipeline} we report the same information regarding the simulations performed with Commander.

In the FGBuster analysis, we simulated the anticipated CMB-S4 patch, which is a 3$\%$, circular sky patch centered in RA, DEC = $(0^{\circ},-45^{\circ})$.

% We performed a Monte-Carlo analysis using two parametric component separation codes: FGBuster \citep{Stompor} and Commander 2 \citep{eriksen_2006,eriksen_2008}. We exploited the two codes using different sets of parameters and simulating two distinct sky patches, one corresponding to the sky patch observed by QUBIC (dark region in Fig.~\ref{fig:sky_patches}) and the other corresponding to a region closer to the Galactic plane and, therefore, potentially more contaminated by foregrounds (light region in Fig.~\ref{fig:sky_patches}). This allowed us to verify that the same conclusions could be obtained using different tools and conditions. Table~\ref{tab:fgbuster_vs_commander} lists the parameters used in the two simulations.

%\begin{figure}[h!]
    %\begin{center}
        %\includegraphics[width=9cm]{figure/s4_patch.pdf}
        %\caption{\label{fig:sky_patches}The sky patch observed by CMB-S4, also used in our simulations. The patch (highlighted in white) is centered in a relatively clean sky area, as one can see from the dust map at 150 GHz plotted underneath.}
    %\end{center}

%\end{figure}

\begin{table}[h!]
    \caption{\label{tab:fgbuster_vs_commander} Parameters used for analyzing simulations with FGBuster for all dust models}
    \renewcommand{\arraystretch}{2}
    \begin{center}
        \begin{tabular}{p{5cm} m{3cm}}
%             \hline
%             & \makecell[c]{FGBuster} & \makecell[c]{Commander} \\ 
            \rule{0.47\textwidth}{0.02cm}\\
%             \makecell[l]{N. realizations\\ \mbox{}[CMB and noise]\dotfill}	& \makecell[c]{500} \\
            Map $N_\mathrm{side}$\dotfill	& \makecell[c]{256} \\
            Multipole range\dotfill	&\makecell[c]{21--335} \\
            $\Delta\ell$\dotfill &\makecell[c]{\phantom{0}{35}} \\
            Input $r$\dotfill & \makecell[c]{0, 0.006} \\
            Residual lensing fraction\tablefootmark{a}\dotfill & \makecell[c]{\phantom{000}10\%} \\
            Sky fraction [\%]\dotfill & \makecell[c]{\phantom{0000}3\%} \\
            \makecell[l]{Sky patch center\tablefootmark{b}\\ \mbox{}[Equatorial coord.]\dotfill}\dotfill & 
                \makecell[c]{$\text{RA}=0^\circ$\\\phantom{1}$\text{DEC}=-45^\circ$\rule[-2.2ex]{0pt}{0pt}}\\
            \rule{0.47\textwidth}{0.01cm}\\
            \multicolumn{2}{l}{\makecell[l]}{    
              %\footnotesize{$^1$Limited by computational time.}\\
              %\footnotesize{$^2$Limited by $N_\mathrm{side}=64$} \\
              
%               \footnotesize{\phantom{$^3$}The value of 100\% means that all the lensing signal was left }\\
%             \footnotesize{\phantom{$^3$}in the map, in the other case only the 10\% of the lensing remains.}\\
              
              \tablefoot{\\
                \tablefoottext{a}{This is the fraction of the lensing signal left in the CMB map.}\\
                \tablefoottext{b}{Center of the CMB-S4 sky patch.} }
            }
%             \footnotesize{$^5$Center of sky patch observed by QUBIC (dark region in Fig.~\ref{fig:sky_patches}).} 
%              }
        \end{tabular}
    \end{center}
\end{table}

% \hl{We generated 500 CMB and instrumental noise realizations and simulated the observation of a circular $3\%$ sky patch centered at high latitudes, corresponding to the anticipated CMB-S4 sky patch, using the CMB-S4 configuration and three BI versions, obtained by dividing each frequency band in 3, 5 or 7 sub-bands. After component separation, we performed our cross-spectra analysis using a uniform binning, spanning a multipole range $21 \leq \ell \leq 335$ separated by $\Delta \ell = 35$, as assumed in the CMB-S4 forecast paper} \cite{Abazajian_2022}.

% \textbf{Component Separation.} In the whole paper, we consider that the components of our signal, namely the foregrounds, can be modeled by a parametric model. The cosmological signal is defined by a set of parameters that are fixed here to the value of the Planck results. The only parameter which will be aimed is the tensor-to-scalar ratio $r$ which will be estimated at the end of the process. Here, we consider that our data can be modeled as follows :

% \hl{We first generated 500 CMB and instrumental noise realizations and then simulated the observation of a circular $3\%$ sky patch centered at $\alpha=-30^\circ,\,\beta=-30^\circ$\footnote{Equatorial coordinates} (corresponding to the patch observed by CMB-S4) in the simulations performed with FGBuster and $\alpha=0^\circ,\,\beta=-57^\circ$ (corresponding to the patch observed by QUBIC) in the simulations performed with Commander. }

We considered eight instrument configurations (see also Fig.~\ref{fig:Polarization_depth}): The CMB-S4 configuration (parametrized following~\citet{Abazajian_2022}) and seven versions of CMB-S4/BI, obtained by subdividing each frequency band. We then applied component separation and analyzed the cross-spectra of the resulting maps using a uniform binning (see Table~\ref{tab:fgbuster_vs_commander} for a summary of the simulation set-up).

For each instrument configuration, the overall analysis chain followed these steps:
\begin{enumerate}
 \item Generate a CMB realization as described at the beginning of Sect.~\ref{sec_simulated_sky_model};
 \item Generate a noise realization for each frequency channel in the considered instrument configuration; 
%  \item smooth the maps with a Gaussian beam of 1$^\circ$ FWHM (only in the Commander simulations); \red{mettere in appendice Commander}
 
 \item Add the CMB and the noise to the foreground maps generated as described in Sect.~\ref{sec_simulated_sky_model};
\item Apply component separation to the input maps. In some cases we assumed the same model used to generate the input case.  In others, we assumed a different model in order to mimic a realistic situation in which the actual foregrounds are not 100\% known and one might assume a model that does not completely describe reality;
%{\color{red} Cross spectra between what maps? I feel the description of your procedure is not clear enough to allow a reader to independently reproduce it, even though the general idea of what you did gets through.}
 \item Perform a cross-spectra analysis between two noise realizations (each with half the exposure time) to recover the tensor-to-scalar ratio, $r$. We calculated angular power spectra using the NaMaster\footnote{\protect\url{https://namaster.readthedocs.io/en/latest/}} code \citep{Alonso_2019} with an apodization radius of 4$^\circ$.
\end{enumerate}

In Table~\ref{tab:performed_simulations} we list the various cases studied in this work. Each case was simulated with all the instrument configurations described in Sect.~\ref{sec_instrument_models} and Fig.~\ref{fig:Polarization_depth}. 

\begin{table}[h!]
    \begin{center}
        \renewcommand{\arraystretch}{1.3}
        \caption{\label{tab:performed_simulations}Cases analyzed in this work.}
        \begin{tabular}{c c}
            \hline
            {Input foreground model} & \makecell[c]{{Model assumed in} \\ {component separation}}\\
            \hline
            \hline
            \textbf{d0s0}\phantom{ ($\ell_\mathrm{corr} = \phantom{1}10$)} &\textbf{d0s0}\\
            \hline
            \textbf{d1s1}\phantom{ ($\ell_\mathrm{corr} = \phantom{1}10$)} &\textbf{d1s1}\\
%             \hline
%             \makecell[c]{\textbf{d6s1} ($\ell_\mathrm{corr} = \phantom{1}10$)} &\textbf{d1s1}\\
            \hline
              \begin{tabular}{c r}
                 \multirow{5}{*}{\textbf{d6s1}} &\hfill $\ell_\mathrm{corr} = \phantom{1}10$ \\
                                                &\hfill $\ell_\mathrm{corr} = \phantom{1}13$ \\
                                                &\hfill $\ell_\mathrm{corr} = \phantom{1}16$ \\
                                                &\hfill $\ell_\mathrm{corr} = \phantom{1}19$ \\
                                                &\hfill $\ell_\mathrm{corr} = 100$ \\
             \end{tabular} & \textbf{d1s1}\\
            
%             \textbf{d1s1} &\\
%             \textbf{d6s1}         &\\
%             \hfill $\ell_\mathrm{corr} = 10$ &\\
%             \hfill $\ell_\mathrm{corr} = 13$ &\\
%             \hfill $\ell_\mathrm{corr} = 16$ &\\
%             \hfill $\ell_\mathrm{corr} = 18$ &\\
%             \hfill $\ell_\mathrm{corr} = 100$ &\\
            \hline
        \end{tabular}

    \end{center}    
\end{table}

\paragraph{Component Separation.} We performed parametric component separation modelling on our data as follows:
% consider that the components of our signal, namely the foregrounds, can be modeled by a parametric model. The cosmological signal is defined by a set of parameters that are fixed here to the value of the Planck results~\citep{planck-paprameters}. The only parameter which will be studied is the tensor-to-scalar ratio $r$ which will be estimated at the end of the process. Here, we consider 

\begin{equation}
    \vec{d}_p = \textbf{A}\cdot \vec{s}_p + \vec{n}_p,
    \label{eq:data_model}
\end{equation}

where $p$ is the pixel index, $\vec d_p$ and $\vec{n}_p$ are vectors representing the data and noise measured by the instrument frequency channels, $\vec{s}_p$ is a vector containing the ``true'' sky values at the same frequencies, and $\textbf{A}$ is a mixing matrix that contains information about the sky components (CMB, synchrotron and interstellar dust). In our simulations, we considered the dust temperature as a known parameter, $T_\mathrm{d}=20$\,K. Thus, the only unknown parameters for synchrotron and dust emissions were their spectral indices, $\beta_\mathrm{s}$ and $\beta_\mathrm{d}$.

%{\color{red} I'm assuming here you are describing how FGbuster works? In that case it may be clearer to rephrase it like "FGBuster solves for the best spectral indexes [...]" }

FGBuster solves for the best spectral indices, $\beta_\mathrm{s}$ and $\beta_\mathrm{d}$, given the data, $\vec d_p$, and the noise covariance matrix, $\textbf{N}$, following the spectral likelihood approach of \citet{Stompor}. In order to cope with computational constraints (processing time and computer memory) and keep the same parameters as in~\citet{Abazajian_2022}, we used a double pixelization scheme in our component separation: A fine resolution of $N_\mathrm{side}=256$ for the pixels of the reconstructed maps, and a coarse resolution of $N_\mathrm{side}=8$ (corresponding to a super-pixel resolution of about 7$^\circ$) for the spectral indices. In other words, the spectral indices are kept constant on larger pixels compared to those of the reconstructed maps. This approach introduces a slight bias on $r$, as demonstrated and addressed in Sect.~\ref{sec:reconstruction_r_fgbuster}. However, this bias does not alter the general validity of our results.

% It is worth mentioning that to cope with computational constraints (processing time and computer memory) and keep the same parameters as in~\citet{Abazajian_2022}, in the case of {\bf d1s1} and {\bf d6s1} we reconstructed the spectral indices on maps with $N_\mathrm{side}=8$, corresponding to a pixel resolution of about 7$^\circ$. \textbf{Note that the map pixelization of the reconstructed spectral indices do not affect the pixelization of CMB after component separation.}

%{\color{red} Is it possible to somehow assess the impact of this choice? E.g., if it is feasible to run just 1 simulation at Nside=256,  you could compare the FG residual power spectra for that case with what you get with your standard setup.}

%\MATHIAS{We decided Nside = 8 for fitting according to Josquin says to me (as far I remember), not so useful to do that because increasing the Nside will decrease the signal-to-noise ratio and bring other effects}

%\ELENIA{Here might be worth specifying that the nside=8 is in the d1s1 and d6s1 case, not in the d0s0 case where instead the parameters are just scalar values.}

% \begin{equation}
%     -2 \ln \mathcal{L}\left(\beta\right) = - \sum_p \left( \textbf{A} \textbf{N}^{-1} \vec d_p \right)^T 
%     \left( \textbf{A}^T \textbf{N}^{-1} \textbf{A} \right)^{-1} \left( \textbf{A} \textbf{N}^{-1} \vec d_p \right).
%     \label{spectral_likelihood}
% \end{equation}
% 
% The best-fit sky component maps, $\hat{\vec s}$, are then obtained from:
% \begin{equation}
%     \hat{\vec s} = \left( \textbf{A}^T \textbf{N}^{-1} \textbf{A} \right)^{-1} \textbf{A}^T \textbf{N}^{-1} \vec d,
%     \label{components}
% \end{equation}
% where the CMB component is our best estimate of the ``clean'' CMB maps and will be used to compute the primordial $BB$ spectrum. 
% Because the whole analysis depends on parameters characterizing the foregrounds, biased estimates of those will induce foreground residuals in the clean CMB maps and will lead to a bias on the cosmological parameters. 

\paragraph{Tensor-to-scalar ratio estimation.} The main goal of our study is to assess how residuals caused by biased estimates of foreground parameters impact the reconstruction of the tensor-to-scalar ratio, $r$, which is the main parameter characterizing the primordial CMB $B$-modes. %\textbf{Along this study, we let vary the tensor-to-scalar ratio $r$ with flat prior in the range [-1, 1]. We decide to leave the $A_{\text{lens}}$ parameter fixed to 0.1}. \ELENIA{Improve the English (grammar, verb tense) of the previous sentence.}

We write the likelihood on $r$ using a Gaussian approximation~\citep{Hamimeche_2008}:

%{\color{red} I'm confused by this likelihood. Shouldn't you include also the CMB contribution to the covariance? Even if you fiducial model has r = 0, shouldn't you include the contributions of your assumed lensing residuals? Also, you are using the same symbol for the noise covariance used in FGBuster compsep, and the noise covariance used in the likelihood, which is confusing. }

%\MATHIAS{We already included the CMB contribution because we used different seed for each iteration. Do we have to include also lensing residuals ?}

%\JCH{I think this is a good remark here: we need to be sure we did the things correctly => discuss it together in a meeting: does $N_{\ell \ell}$ include both noise and sample variance ? If so we need to say it, otherwise Loris is right, this is confusing (well incorrect in fact).}

%\ELENIA{We do include the lensing residual in our model ${D}^{BB}_{\ell, \text{model}}$. Regarding the sample variance, my understanding is that if we change the CMB seed at each iteration that's already included in the noise cov matrix. Am I wrong?}


\begin{equation}
    -2 \ln \mathcal{L}(r) = \left( \textbf{D}^{BB}_{\ell, \text{exp}} - \textbf{D}^{BB}_{\ell, \text{model}} \right)^T \textbf{N}_{\ell, \ell}^{-1} \left( \textbf{D}^{BB}_{\ell, \text{exp}} - \textbf{D}^{BB}_{\ell, \text{model}} \right),
    \label{chi2}
\end{equation}

where $\textbf{D}^{BB}_{\ell, \text{exp}}$ and $\textbf{D}^{BB}_{\ell, \text{model}}$ are the measured and theoretical angular power spectra, $\textbf{N}_{\ell, \ell}^{-1}$ is the inverse of the sum of the noise and sample variance-covariance matrices, and $\mathcal{L}(r)$ is the likelihood on $r$. The theoretical angular power spectrum, $\textbf{D}^{BB}_{\ell, \text{model}}$, includes the contribution of the $10\%$ lensing residual that we assumed throughout the study. Therefore, the only free parameter is the tensor-to-scalar ratio, $r$, which we vary with a flat prior in the range [-1, 1]. Although allowing for negative values of $r$ is unphysical, we opt for this more general approach because it has the benefit of highlighting potential biases due only to differing observational methodologies.

This work explores what happens when dust is more complex than anticipated. In order to do so, we perform component separation assuming a simple model for dust, namely \textbf{d1s1}, but applied on data simulated with the \textbf{d6s1} model. In such cases, incorrect dust modeling leads to residuals in the clean CMB maps. 

We then construct the log-likelihood for $r$ assuming dust to be well modeled by \textbf{d1s1} using the noise covariance matrix in Eq.~(\ref{chi2}), $\textbf{N}_{\ell, \ell}$, obtained from simulations without frequency decorrelation in the dust emission. Such a covariance matrix does not incorporate the variance arising from the dust SED decorrelation so that the bias on $r$ appears with a high significance, which is precisely the effect we want to study. 

After a large number of realizations of \textbf{d6s1}, we see a distribution that shows the large spread in the possible values of $r$ which would be incorrectly considered a measure of high significance because we assumed a simple model for dust. We use the same scheme for all of our instrument configurations (from a classical imager to a bolometric interferometer with a number of sub-bands) so that we can explore if the extra spectral information provided by BI allows us to identify if the ``clean'' CMB maps after component separation are indeed clean or are contaminated by dust residuals.

% In the cases where we used the \textbf{d6s1} model as the sky input, we assumed a different and simpler model in the component separation, namely the \textbf{d1s1} model. This approach allowed us to evaluate a realistic situation in which the sky emissions are more complex than we think, and the use of a parametric model may result in a biased estimation of $r$. Therefore, in these cases, we performed the log-likelihood evaluation using the noise covariance matrix in Eq.~(\ref{chi2}), $\textbf{N}$, obtained from simulations without frequency decorrelation in the dust emission.

% calculate the best-fit $r$ for each of our sky realizations by minimizing the cost-function in Eq.~\ref{chi2}. The aim of this study is to assess the impact of unaccounted foreground complexities on the reconstructed tensor-to-scalar ratio. In another word, we perform the analysis assuming a simple foreground model, such as {\bf d1s1} (without dust decorrelation) while the simulated sky is more complex. As a result, we performed the log-likelihood evaluation assuming the noise covariance matrix $N$ obtained from simulations without frequency decorrelation in the dust emission.


            
        %\subsubsection{Detecting dust decorrelation on a single realization}
        %\label{sec:machine_learning}
        
        

%\begin{enumerate}
%    \item produce 500 sky realizations with $r=0.006$\footnote{The value of $r=0.006$ was chosen so that the average reconstructed $r$ matched the bias that would be obtained from a map with CMB with $r=0$ and \textbf{d6s1} foregrounds removed assuming a \textbf{d1s1} model with a single reconstructed sub-band(see Fig.~\ref{fig:results_d6s1_r})} in which the sky is generated with \textbf{d1s1} and fitted with the same model (we call this dataset \textbf{d1-d1}), this dataset is labeled as "clean"; 
%    \item produce 500 simulations with $r=0$, in which the sky is generated with \textbf{d6s1} ($\ell_\mathrm{corr}=10$) and fitted with \textbf{d1s1} (we call this dataset \textbf{d6-d1}), this dataset is labelled as "contaminated"; 
%    \item \label{item:train_set} for each simulation and for each value of $n_\mathrm{sub}$ calculate the following two quantities: $\rho(n_\mathrm{sub}) = r(n_\mathrm{sub}) / r(n_\mathrm{sub}=1)$ and $\sigma_\rho(n_\mathrm{sub}) = \sigma(r(n_\mathrm{sub})) / \sigma(r(n_\mathrm{sub}=1))$  (``training'' dataset), both with "clean" or "contaminated" label, depending on the model used as an input. These quantities are those that discriminate whether we have foreground residuals or not: if $\rho \neq 1$ it means that the detection depends on the number of sub-bands and, therefore, is likely to be affected by foreground residuals;
%    \item train the network with 250 \textbf{(d1s1, $r=0.006$)} and 250 \textbf{(d6s1, $r=0$)} randomly selected realizations from the training dataset (using 100 cross-validation subsets); 
%    \item \label{item:predict_set} calculate $\rho(n_\mathrm{sub})$ and $\sigma_\rho(n_\mathrm{sub})$ for the remaining 250 \textbf{(d1s1, $r=0.006$)} and 250 \textbf{(d6s1, $r=0$)} simulations (``test'' dataset);
%    \item feed the trained network with the values calculated in step \ref{item:predict_set} to test its ability to classify the simulations as ``clean'' (constant $\rho(n_\mathrm{sub})$) or ``contaminated'' (variable $\rho(n_\mathrm{sub})$.
%\end{enumerate}




% We used machine learning based classification using our simulated data and explore if one can use, on a individual realization basis, the evolution of the reconstructed $r$ as a function of the number of sub-bands to distinguish between skies with or without frequency decorrelation. 

% We trained . In order not to be sensitive to a possible non-zero input tensor-to-scalar ratio $r$, for each of our realization, we divide the reconstructed $r(N_{sub})$ and their uncertainties $\sigma_r(N_{sub})$ for all sub-bands by the value found by the classical imager with $N_{sub}=1$. The features used for training are therefore all the values of $r(N_{sub})/r(N_{sub}=1)$ and uncertainties $\sigma_r(N_{sub}) / r(N_{sub}=1)$ for all the realizations from {\bf d1s1} and {\bf d6s1} both with an input $r$ equal to zero, as well as {\bf d1s1} with input $r=0.01$. We have used a {\em cross-validation scheme} in order to optimize the hyper-parameters of the decision tree and to avoid over-fitting in our training.


    \subsection{Results}    
    \label{sec_results}   

    %{\color{red} I was wondering whether it could be useful to adding a plot showing how, for a single realization, increasing the number of sub-bands helps recovering the actual SED for d6? E.g., you could show how the power spectra of the residuals gets smaller, or how the reconstructed SED gets closer to input one as you increase $n_{sub}$. I don't think this is necessary, but it seems like a nice thing to show.}

\subsubsection{Reconstruction of the tensor-to-scalar ratio, $r$}
\label{sec:reconstruction_r_fgbuster}
    Here we discuss the results of our FGBuster simulations in terms of the reconstruction of the tensor-to-scalar ratio, $r$. The performance in terms of foreground reconstruction is discussed in Appendix~\ref{app:reconstruction_foregrouds} (FGBuster simulations), whereas the Commander simulations are presented in Appendix~\ref{app:commander_results}.
    \\
    \\
    %{\color{red} I feel something like "the histograms of the maximum likelihood estimates" would be clearer.}
    The four panels in Fig.~{\ref{fig:results_d0s0_r}} show the histograms of the maximum likelihood values of $r$ computed from Eq.~(\ref{chi2}) for each iteration of the Monte-Carlo chain. Each panel shows the result for one of the simulated sky models as a function of $n_\mathrm{sub}$. The \textit{top-left panel} shows the CMB with $r_\mathrm{input}=0$ and \textbf{d0s0} foregrounds.  The \textit{top-right panel} shows the CMB with $r_\mathrm{input}=0$ and \textbf{d1s1} foregrounds. The \textit{bottom-left panel} shows the CMB with $r_\mathrm{input}=0.006$ and \textbf{d1s1} foregrounds. The \textit{bottom-right panel} shows the CMB with $r_\mathrm{input}=0$ and \textbf{d6s1} foregrounds with $\ell_\mathrm{corr}=10$.

    The histograms are normalized to the maximum count value and smoothed with a kernel density estimator (KDE) of width equal to one-fourth of the standard deviation of the histogram. The histograms extend to negative $r$ because we compute the posterior likelihood over a range of $r$ that includes negative values in order to avoid a sharp truncation of the likelihood at $r=0$.
    
    A more detailed discussion of each of the four cases follows below.
    
%     We present here the Fig.~{\ref{fig:results_d0s0_r}} which shows the maximum likelihood histograms for the reconstruction of r in different cases. We reconstruct for different values of subbands a set of realizations considering a certain foreground model, and more specifically thermal dust. We show here that the reconstructions of r actually depend on the model considered, especially when there is frequency decorrelation with $\ell_\mathrm{corr}=10$. We note that the histograms have been smoothed out using a Kernel Density Estimation (KDE).

%     The first three cases are simple dust models which are respectively with a constant index on the sky without B-modes, a variation of its indices always without B-modes and an addition of primordial fluctuation corresponding to $r = 0.006$. These simulations are mainly used as a control to show that the biases are small in these specific cases. The reconstruction of the tensor-to-scalar ratio is compatible with the value of between a 2$\sigma(r)$.
    %{\color{red} From the plots it seems that for the d0s1 and d1s1 cases the bias is more than ~1 $\sigma$, which I think most people would not consider small. Also, why is this  leakage not captured in the empirical covariance?}

    
    \paragraph{Top-left panel.} Here we have the CMB with $r_\mathrm{input}=0$ and \textbf{d0s0} foregrounds. In this case, the reconstructed $r$ does not depend on $n_\mathrm{sub}$ and there is a small bias due to an $E\rightarrow B$ modes leakage caused by the power spectra computation on a sky patch, where the spherical harmonics are no longer orthogonal. This bias could be mitigated by increasing the apodization radius of the mask at the expense of a smaller effective sky fraction ($<3\%$). This optimization, however, is outside the scope of the paper.

    
    \paragraph{Top-right panel.} Here we have the CMB with $r_\mathrm{input}=0$ and \textbf{d1s1} foregrounds. Also in this case we see that the reconstructed $r$ does not depend on $n_\mathrm{sub}$, even if the complexity of the dust emission is higher (the dust spectral index varies in the sky). However, here we observe a slightly larger bias in $r$ with respect to the \textbf{d0s0} case, caused by the aforementioned leakage and also by the difference in pixel size of the reconstructed spectral indices maps ($N_\mathrm{side}=8$) compared to the input sky ($N_\mathrm{side}=256$). %{\color{red} Could this issue also affect the d0s1 results? Even if the input spectral properties are uniform over the sky, by fitting it on Ns=8 pixels you introduce patches in which the assumed, e.g., dust spectral index is coherently above or below the actual value, leading to residuals with a characteristic scale corresponding to the Ns=8 pixel size.}

    %\MATHIAS{Having d0s0 data fitted with d1s1 model (varying spectral indexes) will not include bias on r, we will just fit many time the same values, a sentence to say that ?}

    %\JCH{No, I don't think it is necessary}

    %\ELENIA{In the d0s0 case don't we fit for scalar parameters (sort of nside=0) and not for nside=8? At least this is what I'm doing, which means that we are fitting for the exact same model. If this is the case, it should be worth specifying it in the text to avoid confusion.}

    \paragraph{Bottom-left panel.} Here we have the CMB with $r_\mathrm{input}=0.006$ and \textbf{d1s1} foregrounds. This case is similar to the previous one, the only difference being the value of $r_\mathrm{input}$.
    
    \paragraph{Bottom-right panel.} Here we have the CMB with $r_\mathrm{input}=0$ and \textbf{d6s1} foregrounds fitted with the \textbf{d1s1} model. The histograms show that fitting with a model that does not account for frequency decorrelation produces distributions that are larger for smaller values of $n_\mathrm{sub}$. Also, the mean value of the reconstructed $r$ obtained from such distributions varies and becomes smaller as $n_\mathrm{sub}$ increases.
    \\

%     In the \textbf{d1s1} case, the spectral indices of the astrophysical foregrounds are supposed to vary in the sky. The model includes some pixelization with Nside = 8 which is lower than our maps. This difference is understandable for computational reasons but also to reproduce the real case where the pixelization of the indices is infinite unlike the frequency maps. \hl{highlight the bias in the reconstruction of r and explain that it is due to the reconstruction of foregrounds with Nside = 8}

%     The presence of dust frequency decorrelation has a strong effect on the reconstruction of the tensor-scalar relationship with the \textbf{d6s1} case. The distribution becomes much larger in the case of the classical imager. By considering a bolometric interferometer, it is now possible to repeat this analysis by varying the number of sub-bands to study the variation of the bias. The increase in the number of frequency maps allows a stronger constraint on the spectral indices, and thus a bias that is reduced by increasing the number of sub-bands. This technique shows that the first value of $r$ is an artifact due to the presence of residual dust.


    \begin{figure*}[h!]
%         {\includegraphics[width=0.495\columnwidth]{figure/fig_d0_0.000.pdf}} 
%         {\includegraphics[width=0.495\columnwidth]{figure/fig_d1_0.000.pdf}}\\
%         {\includegraphics[width=0.495\columnwidth]{figure/fig_d1_0.006.pdf}} 
%         {\includegraphics[width=0.495\columnwidth]{figure/fig_d6_0.000.pdf}}
        \centering  
        \includegraphics[width=11cm]{figure/histograms_top.pdf} \\
        \includegraphics[width=11cm]{figure/histograms_bot.pdf}
        \caption{\label{fig:results_d0s0_r}Normalized histograms of the maximum likelihood values of $r$ as a function of the number of sub-bands. \textit{Top-left}: model \textbf{d0s0} with $r_\mathrm{input}=0$. \textit{Top-right}: model \textbf{d1s1} with $r_\mathrm{input}=0$. \textit{Bottom-left}: model \textbf{d1s1} with $r_\mathrm{input}=0.006$. \textit{Bottom-right}: model \textbf{d6s1} with $\ell_\mathrm{corr}=10$ and $r_\mathrm{input}=0$.}
    \end{figure*}
    

    Fig.~\ref{fig:r_vs_nsub} shows the average $r$ and standard deviation computed from the histograms of Fig.~\ref{fig:results_d0s0_r} as a function of $n_\mathrm{sub}$. This result represents the range of $r$ from which we expect to sample our measurement when performing CMB observations.

    Note that since the error bar is the standard deviation, we assume it to be symmetrical. Moreover, in the \textbf{d6s1} case the histogram is unsymmetrical, and therefore the average $r$ is not centered with the distribution.
    
    The blue, orange, and green curves refer to the case in which we fit the same dust model used to simulate the input sky. In these three cases, the recovered $r$ does not depend on $n_\mathrm{sub}$, as one would expect for a detection not contaminated by foregrounds. The difference between the recovered $r$ with respect to $r_\mathrm{input}$ that we see in all three cases is caused by the $E\rightarrow B$ leakage and pixel size effects discussed above.

    The red curve refers to the case in which the input sky contains dust emission with frequency decorrelation while component separation was performed ignoring this feature, assuming the {\bf d1s1} model. In this case, the increase in the number of frequency maps provided by BI allows us to better constrain the spectral indices, thus reducing the bias as the number of sub-bands increases. On average, a classical imager (represented by $n_\mathrm{sub}=1$) would measure $r\sim 0.008$ while a bolometric interferometer would see this estimate reducing by increasing $n_\mathrm{sub}$. This indicates that the first value of $r$ is an artifact due to the presence of residual dust emission.
    
    \begin{figure}[h!]
        \includegraphics[width=\columnwidth]{figure/new_r_vs_nsub.pdf} \\
        \caption{\label{fig:r_vs_nsub} Average maximum likelihood value of $r$ and standard deviation as a function of the number of sub-bands in the case of unaccounted dust frequency decorrelation (model \textbf{d6s1} with  $\ell_\mathrm{corr}=10$  and $r=0$) compared to two cases of no decorrelation (model \textbf{d1s1}): $r=0$ and $r=0.006$. On top of the average $r$ values and their standard deviation, we have overplotted the shape of the distribution as a ``violin plot''. Note that for the \textbf{d6} case the distribution is asymmetric for small $n_\mathrm{sub}$, so that the average is not centered on the distribution.}
    \end{figure}

    %{\color{red} What is the typical uncertainty on r for a single simulation? I would naively expect it to be of the same order of the width of the distribution of the maximum likelihood values, if r was truly Gaussian distributed. If the above is correct, it seems to me that, even for $r = 0.006$ you'd expect a 1-2 sigma measurements of r, while Abazajian et al. claim S4 should achieve a 5sigma detection of $r > 0.003$. Do you have any idea of the origin of this discrepancy?Am I missing something here?}

    %\MATHIAS{This discrepancy could comes from the different method used in the pipeline, also the lack of information we have on the simulation pipleine they used. To be mentionned here ?}

    %\JCH{I personally do not get the point from Loris}
    
    
%     We show an overview of the capabilities of bolometric interferometry in several cases of astrophysical foregrounds is shown in Fig.~{\ref{fig:results_d6s1_r}}. The reconstructed mean value of r and the corresponding standard deviation as a function of the number of subbands are shown in the upper panel of Fig.~{\ref{fig:results_d6s1_r}} for the case of decorrelation not taking into account the dust frequency (model \textbf{d6s1}, $\ell_\mathrm{corr}=10$) compared to the case of correct foreground modeling (model \textbf{d1s1}) with $r=0$ and $r=0.006$.

    %The bottom panel shows a summary for different correlation length values that parameterize the frequency decorrelation. The longer the correlation length, the more we find a simple model such as \textbf{d1s1} and as expected we recovered the simple foregrounds case. The simulations have been done with $\ell_{\mathrm{corr.}}=10,13,16,19,100$. 
    
    Finally, Fig.~{\ref{fig:results_d6s1_r}} shows a summary of the average $r$ and standard deviation for all the simulated dust models with $r_\mathrm{input}=0$, including various correlation lengths for the \textbf{d6s1} case: $\ell_{\mathrm{corr.}}=10,13,16,19,100$. For the sake of simplicity, we only show four instrument configurations: CMB-S4 and CMB-S4/BI with 3, 5, and 7 sub-bands. As one can see, the advantage of BI in diagnosing foreground residuals, and therefore decreasing the bias on $r$, is maintained even in the case of smaller levels of dust frequency decorrelation. As expected, in the limit of $\ell_{\mathrm{corr.}}=100$ the result is compatible with the case of a single modified black-body (model \textbf{d1s1}).

    \begin{figure}[h!]
    \begin{center}
        \includegraphics[width=0.9\columnwidth]{figure/Results_on_r_vs_dust_model_mod_2.pdf} 
        \caption{\label{fig:results_d6s1_r}Summary of the average maximum likelihood value of $r$ and standard deviation for an input $r=0$ and all the simulated foreground models (\textbf{d0s0}, \textbf{d1s1} and several $\ell_\mathrm{corr}$ cases of \textbf{d6s1}). Note that we assume symmetric error bars.}
    \end{center}
    \end{figure}

%     As one can see in the top panel of Fig.~{\ref{fig:results_d6s1_r}}, unaccounted dust frequency decorrelation would lead to a bias of the order of $r=0.006$ in the classical imager configuration. This is the only result that a classical imager would be able to recover. However, the addition of bolometric interferometry allows us to strongly reduce the bias when increasing the spectral resolution. Moreover, its possibility of reanalizing the data with different spectral resolution allow us to diagnose the presence of foreground residuals: indeed, in the case of true primordial \textit{B}-modes we would recover a slightly increasing value of r as a function of the number of sub-bands due to the increase in the instrumental noise; the fact that the result on r decreases as a function of the number of sub-bands is thus a strong hint of the presence of foreground residuals in the recovered CMB signal.

    %\begin{figure}[h!]
    %    \begin{tabular}{c | c}
    %        \multicolumn{1}{c}{\textbf{d6s1}, $r=0$} & \multicolumn{1}{c}{\textbf{Comparison}}\\
    %        \makecell[c]{\includegraphics[width=4.1cm]{example-image-a}} &
    %        \makecell[c]{\includegraphics[width=4.1cm]{example-image-b}}
    %    \end{tabular}
    %    \centering
    %    \begin{tabular}{c}
    %        \multicolumn{1}{c}{\textbf{Summary}} \\
    %        \makecell[c]{\includegraphics[width=4.1cm]{figure/Results_on_r_vs_dust_model_def.pdf}}
    %    \end{tabular}

    %    \caption{\label{fig:results_d6s1_r}\hl{ \textit{Top-left panel}: Histogram of reconstructed $r$ for the model \textbf{d6s1} with $\ell_\mathrm{corr}=10$ and $r_\mathrm{input}=0$ as a function of the number of sub-bands. \textit{Top-right panel}: Comparison of the result on r from the \textbf{d6} case with $r=0$ and the \textbf{d1} case with $r=0.01$. \textit{Bottom panel}: Summary of the result on r for all the simulated dust models (\textbf{d0s0}, \textbf{d1s1} and several $\ell_\mathrm{corr}$ cases of \textbf{d6s1}).}}
        %\caption{\label{fig:results_d6s1_r}\hl{Model \textbf{d6s1}: histograms of reconstructed $r$ as a function of the number of sub-bands. \textit{Left}: $r_\mathrm{input}=0$ . \textit{Right}: $r_\mathrm{input}=0.01$.}}
    %\end{figure}
  
%     \red{Semplificare la sezione, rimuovere i sottocasi, mettere una prima riga con istogrammi con d0s0 e d1s1 (si fitta con il modello giusto) e una seconda riga con il d6s1 contenente anche il caso con r diverso da zero}
    
\subsubsection{Identifying foreground residuals on a single realization}
\label{classifier}
    We used machine learning to test the ability of BI to detect foreground residuals that may be present when the assumed foreground model is different from that describing the actual sky emission. That might occur, for example, if one assumes a \textbf{d1s1} model when the sky is described by a \textbf{d6s1} model. Therefore, we explore the possibility of classifying between ``contaminated'' and ``not contaminated'' cases that both end up producing the same average reconstructed $r$ for an imager (described by the case in which we do not split the physical band in sub-bands).

This ability is a key issue when an experiment detects a tensor-to-scalar ratio that is significantly different from zero. In this case, there is only one realization (i.e., the actual measurement) to understand whether there are unknown systematic effects biasing the value beyond the uncertainty set by the noise plus the known systematic effects.

We carried out this test by performing a machine learning classification based on a simple gradient-boosted decision tree (a \texttt{GradientBoostingClassifier} from the \texttt{scikit-learn} Python library\footnote{\protect\url{https://scikit-learn.org/}}) according to these steps:  

\begin{enumerate}
    \item Produce 500 sky realizations with $r=0.006$\footnote{The value of $r=0.006$ was chosen so that the average reconstructed $r$ matched the bias that would be obtained from a map with CMB with $r=0$ and \textbf{d6s1} foregrounds removed assuming a \textbf{d1s1} model with a single reconstructed sub-band(see Fig.~\ref{fig:r_vs_nsub})} in which the sky is generated with \textbf{d1s1} and fitted with the same model (we call this dataset \textbf{d1-d1}). This dataset is labeled as ``clean''; 
    \item Produce 500 simulations with $r=0$, in which the sky is generated with \textbf{d6s1} ($\ell_\mathrm{corr}=10$) and fitted with \textbf{d1s1} (we call this dataset \textbf{d6-d1}). This dataset is labelled as ``contaminated''; 
    \item \label{item:train_set} For each simulation, and for each value of $n_\mathrm{sub}$, calculate a normalized reconstructed $r$ and its uncertainty normalized by what is found with $n_\mathrm{sub}=1$, expressed by 
    the following two quantities: $\rho(n_\mathrm{sub}) = r(n_\mathrm{sub}) / r(n_\mathrm{sub}=1)$ and $\sigma_\rho(n_\mathrm{sub}) = \sigma(r(n_\mathrm{sub})) / r(n_\mathrm{sub}=1)$  (``training'' dataset), both with ``clean'' or ``contaminated'' label, depending on the model used as an input. These quantities are those that discriminate whether we have foreground residuals or not. If $\rho \neq 1$, it means that the detection depends on the number of sub-bands and, therefore, is likely to be affected by foreground residuals;
    \item Train the network with 250 \textbf{(d1s1, $r=0.006$)} and 250 \textbf{(d6s1, $r=0$)} randomly selected realizations from the training dataset (using 100 cross-validation subsets); 
    \item \label{item:predict_set} Calculate $\rho(n_\mathrm{sub})$ and $\sigma_\rho(n_\mathrm{sub})$ for the remaining 250 \textbf{(d1s1, $r=0.006$)} and 250 \textbf{(d6s1, $r=0$)} simulations (``test'' dataset);
    \item Feed the trained network with the values calculated in step \ref{item:predict_set} to test its ability to classify the simulations as ``clean'' (constant $\rho(n_\mathrm{sub})$) or ``contaminated'' (variable $\rho(n_\mathrm{sub})$\textbf{)}.
\end{enumerate}

The result of this procedure is the so-called ``confusion matrix'', i.e., a matrix that compares the results from the classification predicted by the algorithm with the true one as shown in Fig.~\ref{fig:ML}. The performance of our classifier is as follows (we adopted the convention ``clean=negative'' and ``contaminated=positive''):
\begin{itemize}
    \item True negative rate very close to 1, indicating that the realizations with no dust residuals (dataset {\bf d1-d1} with $r=0$ and $r=0.006$) displayed a constant ratio $\rho(n_\mathrm{sub})$ and were correctly classified as ``clean'';
    \item True positive rate very close to 1, indicating that the realizations with dust residuals (dataset {\bf d6-d1} with $r=0$), displayed a variable ratio $\rho(n_\mathrm{sub})$ and were correctly classified as ``contaminated'';
    \item Low false negative rate of $2.9\%\pm 1.6\%$, indicating a very low percentage of realizations with dust residuals that were wrongly classified as ``clean''. This is a very important figure of merit that we want to minimize;
    \item Low false positive of $1.2\% \pm 0.3\%$, indicating a very low percentage of realizations without dust residuals that were wrongly classified as ``contaminated''.
\end{itemize}

Such high classification performance demonstrates that BI, with its capability to measure $r$ in several sub-bands, is a promising solution to identify residuals in the clean CMB maps arising from LOS frequency decorrelation in the dust emission. In such a case, a classical imager lacks the frequency resolution to identify this contamination, leading to a systematic uncertainty in the reconstructed $r$ that is well above the target sensitivity of \mbox{CMB-S4}.

\begin{figure}
    \centering
    \resizebox{\hsize}{!}
    {\includegraphics{figure/confusion_matrix.png}}
    \caption{Confusion matrix representing our ability to classify between our simulated data sets with dust frequency decorrelation (contaminated) or without (clean) using the measurements of $r$ as a function of $n_{sub}$. We observe that the fraction of false negatives (``contaminated'' data set incorrectly classified as ``clean'') is close to zero.}
    \label{fig:ML}
\end{figure}



% \JCH{Went up to here}
   

     

\section{Conclusions}
\label{sec_conclusions}
In this paper, we have shown how bolometric interferometry (BI) has the potential to detect systematic effects caused by interstellar dust in CMB polarization measurements when LOS frequency decorrelation is present in dust emission and it is not accounted for in parametric component separation algorithms. 

We know that there are ways for imagers to mitigate the problem of not precisely knowing the foreground emission, for example, through cross-checking with different component separation methods, such as blind ones \citep{Aumont_2007}, or codes based on the moment expansion \citep{Chluba_2017, Vacher_2022} which might be less sensitive to incorrect foreground modeling. However, in this paper, we propose a new approach based on a different instrument architecture called bolometric interferometry (BI).  An instrument based on BI can be used as an independent verification to future claims of a \mbox{$B$-mode} detection by exploiting the superior purity of the $r$ measurement made possible by the increased spectral resolution.

We have carried out simulations with two component separation codes (FGBuster, discussed in the main text, and Commander, discussed in Appendix~\ref{app:commander}), reconstructing the tensor-to-scalar ratio, $r$, from simulated skies containing CMB, synchrotron and dust emission, and instrumental noise. For the dust emission, we used three models of increasing complexity, one of which contains frequency decorrelation.

We compared two instrument models, CMB-S4 and \mbox{CMB-S4/BI}, the latter being a modified version of CMB-S4 that accounts for the possibility of splitting each physical frequency band in a variable number of sub-bands that can be chosen during data analysis. This feature, which is unique to BI, allows us to assess whether a measurement of $r$ is biased by dust emission residuals or not. While a Fourier-transform spectrometer can be used to increase spectral resolution, it would suffer from a noise penalty compared to BI because it cannot observe all frequencies simultaneously.

Our results are consistent for the two codes and show that with no frequency decorrelation, both instruments yield the same performance (the final precision and systematic uncertainty on $r$ are similar). If decorrelation is present, and it is not accounted for in component separation, then an imager like CMB-S4 would measure a biased value of $r$. This bias can be reduced with \mbox{CMB-S4/BI} by reanalyzing the same data after splitting the band into an increasing number of sub-bands. 

The decrease of the measured $r$ with the number of sub-bands, $n_\mathrm{sub}$, clearly indicates the presence of a dust-induced systematic effect, given that without dust residuals the detected $r$ does not change with $n_\mathrm{sub}$. In such a situation, a classical imager would have no means of classifying the measurement as ``clean'' or ``biased''.

We tested the ability to detect biased $r$ measurements also using a machine learning approach, and we verified that assessing the variation of the $r$ measurement versus $n_\mathrm{sub}$ allowed us to classify clean and biased measurements with a rate $\gtrsim 97\%$.

%This paper focused on a new methodology, we did not consider instrumental systematic effects.

Future developments will test this technique in more realistic situations (representative noise, inclusion of optical effects, and uncertainty on the knowledge of the instrumental spectral response), assess the performance with various dust models, and explore new techniques of component separation, allowing us to separate signals taking into account instrumental effects in a more comprehensive and representative way.

%\optinSilvia{Paolo's comments follow.
%1.The main general issue for the average reader will be to understand why a BI is favored wrt an imager, despite of the fact that the spectrum of dust emission is basically featureless, so spectral resolution should not help significantly. For those who work with spectrometers, it is very evident that, if there are no lines/features in the spectrum, in order to measure the continuum it is good practice to reduce the spectral resolution, since high resolution comes with a penalty in terms of noise. This paper focuses on systematic effects rather than on SNR, but small systematics are usually hidden in the noise, so SNR is also important. From the point of view of the average scientist, the result of the paper is a surprise. He will want to understand in an intuitive way, beyond the quantitative results of the analysis of the simulations, why this happens. A simple, convincing, intuitive explanation of the result is lacking in the present form of the paper, and should be added, if there is one. \newline
%2. Also, the average reader will want to understand better how much this result is general, and how much it depends on the details of the dust model. Showing that the advantage of BI is correlated to the presence of spectral features in the emission spectrum (if true) would be a good way to convince the average reader that there is no bias in the comparison, and no flaws in the analysis. At least a comment on this should be added. \newline
%3. Another issue is: how much does the error in the knowledge of the spectral response of the system (for both BIs and Imagers) affect the results of the simulations? In this paper it is assumed that the bands and sub-bands profiles are known perfectly. In the real instruments this is not true, for both the imagers and the BI. Will this error be larger for the bands of the imagers or for the sub-bands of the BI? If the errors are larger for the BI, which is possible due to the stronger link between beam knowledge and spectral response, will this affect significantly the main conclusion of the paper ? A comment on this is important !}


%decorrelation In the case of model 1 and 2 which are quite simple models of foregrounds, the biases and precision on $r$ are similar in the case of a classical imager and a bolometric interferometer. On the other hand, in the case of a complex model with an effect of dust frequency, the bolometric interferometer allows to go further and check the coherence of the results obtained which a classical imager can not do. For these simple models, we see that we find a bias greater than that published in the paper CMB-S4. 


% \textbf{Proposition by Mathias :} \\ \\
% In this paper, we have shown the challenge of the component separation step and the need to have models consistent with the reality of the data. These models must be improved in order to better understand the astrophysical foregrounds and to better subtract them.
% 
% We have performed simulations to reproduce the results of CMB-S4 and we could see how bolometric interferometry could add value to these experiments and to future results. We use here a complete pipeline while the official paper describes a Fisher analysis. In the same spirit, several parameters such as apodization or E-to-B leakage are not mentioned which leads us to make a choice towards realist models which increase the bias on $r$.
% 
% Future studies will focus on new techniques of component separation, in particular through component map-making. This new technique allows to separate signals and to take into account the instrumental effects on the measurements and the reconstruction of the maps.

    

\begin{acknowledgements}
QUBIC is funded by the following agencies. France: ANR (Agence Nationale de la
Recherche) contract ANR-22-CE31-0016, DIM-ACAV (Domaine d’Intérêt Majeur-Astronomie et Conditions d’Apparition de la Vie), CNRS/IN2P3 (Centre national de la recherche scientifique/Institut national de physique nucléaire et de physique des particules), CNRS/INSU (Centre national de la recherche scientifique/Institut national et al de sciences de l’univers). Italy: CNR/PNRA (Consiglio Nazionale delle Ricerche/Programma Nazionale Ricerche in Antartide) until 2016, INFN (Istituto Nazionale di Fisica Nucleare) since 2017.  Argentina: MINCyT (Ministerio de Ciencia, Tecnología e Innovación), CNEA (Comisión Nacional de Energía Atómica), CONICET (Consejo Nacional de Investigaciones Científicas y Técnicas).
 
 S. Paradiso acknowledges support from the Government of Canada's New Frontiers in Research Fund, through grant NFRFE-2021-00595.

 The authors want to thank Alexandre Boucaud for valuable advices about Machine Learning.
\end{acknowledgements}


\bibliographystyle{aa} % style aa.bst
\bibliography{biblio,qubic} % your references Yourfile.bib

\appendix

\section{Reconstruction of foregrounds parameters}
\label{app:reconstruction_foregrouds}

%\hl{I've made two version of the plot: one with the results of S4,BI3,BI5,BI7 and one with all the nsubs cases. If we decide to only plot S4,BI3,BI5,BI7, I think we can write down the average and std within the plot of the histogram. If we want to show all the nsubs, I think it's better to leave the plot of the histogram plain and summarize the average and sigma in a separate table for each foreground model that we've used.
%If anyone has a better idea, please feel free to share any suggestion.}

%\comm{Mathias : I guess we can keep the plot with all nsubs and the colorbar on the bottom.}


In our paper, we focused on the reconstructed tensor-to-scalar ratio, $r$, as it is the main quantity of interest. The level of systematic uncertainties in the reconstructed $r$, however, depends on the reconstructed foreground spectral parameters and distributions. Thus, in this appendix we focus on the distribution of the foregrounds spectral indices after component separation. 

%\begin{figure}[ht]
%	\includegraphics[width=9cm]{figure/Param_distribution_5_FinalConfNoiseNew2_Comp_42_['d0', 's0']_0.0_0.1_None_['S4', 'BI3', 'BI5', 'BI7']_nsidefit0_Nbpintegr100_iter500.pdf}
%        \caption{\label{fig:param_reconstruction_d0s0}Reconstruction of foregrounds parameters in the \textbf{d0s0} case.\hl{version with S4,BI3,BI5,BI7}}
%\end{figure}

In Fig.~\ref{fig:histograms_foregrounds} we show the normalized histograms of the difference between the reconstructed and input dust and synchrotron spectral indices, $\Delta\beta_\mathrm{d}$, $\Delta\beta_\mathrm{s}$ for the following three models: \textbf{d0s0} (top row), \textbf{d1s1} (middle row), \textbf{d6s1} (bottom row), all with \mbox{$r_\mathrm{input}=0$}. Each histogram does not correspond to a particular pixel but contains values from the sky patch. 

\begin{figure}[h!]
    \resizebox{\hsize}{!}
	{\includegraphics[width=9cm]{figure/histograms_foregrounds.pdf}}
    \caption{Reconstruction of foregrounds spectral indices. \textit{Top}: model \textbf{d0s0}, $r_\mathrm{input}=0$. \textit{Middle}: model \textbf{d1s1}, $r_\mathrm{input}=0$. \textit{Bottom}: model \textbf{d6s1}, $r_\mathrm{input}=0$.}
    \label{fig:histograms_foregrounds}
\end{figure}

In the case of \textbf{d0s0}, the model assumes a constant spectral index all over the sky. Therefore we expect unbiased estimates with a standard deviation related to the noise in the input frequency maps. The results shown in the top row of Fig.~\ref{fig:histograms_foregrounds} confirm this expectation as we observe no bias on the reconstructed spectral indices. We note that the standard deviation slightly increases with the number of sub-bands, $n_\mathrm{sub}$, because of the slight sub-optimality inherent to spectral-imaging \cite[parametrized by $\varepsilon$ in Eq.~\ref{sigma}, see][]{2020.QUBIC.PAPER2}. 

%{\color{red} It feels you are mixing up two different effects here. If the FG spectral properties change across the sky, you'd expect "decorrelation" regardless of the $N_{side}$ at which you fit for the FG spectral indexes. You can easily verify this if you use pysm to generate a  pure d1 map at e.g. 353 and 150GHz and  compute the corresponding $R_{\ell}$. You will notice a ~ 0.005 level decorrelation. The fact that you evaluate the spectral indexes at $Ns=8$ rather than $Ns=256$ leaves biases and residuals in your recovered FG and CMB parameters, but does not affect decorrelation, which is a feature of the input dust maps.}

%\ELENIA{I think that we should just remove "spatial decorrelation" in the sentence below. We should keep the reason why we get the bias (which is because we fit the spectral params for Ns=8 instead Ns=256) but without calling this a "spatial decorrelation".}

When spectral indices vary across the sky, as in \textbf{d1s1}, we expect biases in the reconstructed spectral indices because we only reconstruct the spectral indices on relatively large sky pixels ($N_\mathrm{side}=8$), while the input sky was simulated with spectral indices that vary among smaller pixels ($N_\mathrm{side}=256$). Consequently, averaging multiple spectral indices in large pixels introduces a bias to the reconstructed spectral index. This bias is responsible for foreground residuals in the CMB maps obtained after component separation and produces the bias on $r$ observed in Figs.~\ref{fig:results_d0s0_r} and ~\ref{fig:r_vs_nsub}.

This is shown in the middle row of Fig.~\ref{fig:histograms_foregrounds}. The bias due to spatial decorrelation appears as an enlarged spread of the distribution with respect to the \textbf{d0s0} case (notice the increased scale of the $x$-axis in the middle row compared to the top row). Also in this case we observe an increase in standard deviation with $n_\mathrm{sub}$ caused by the sub-optimality related to spectral imaging.

Finally, in the case of frequency decorrelation in the dust emission ({\bf d6s1} model), spectral indices are no longer an accurate description of the dust spectral behavior. As a result, if we reconstruct $\beta_\mathrm{d}$ using a {\bf d1s1} model we expect a much larger bias. Note that the comparison between input and reconstructed spectral indices is done using the template map of $\beta_{\mathrm{d}}$ \citep{Planck_15} that was used as an input for the {\bf d6s1} model. It is clear however that in the case of {\bf d6s1}, the comparison between the input and recovered spectral indices is less meaningful than in the simpler models. In this case, one is more interested in the residuals found in the "clean" CMB maps, discussed in the main text of this article. In this case, the increase in spectral resolution provided by spectral imaging supplies extra information, allowing us to reduce this bias. 
This is confirmed by the results shown in the bottom row of Fig.~\ref{fig:histograms_foregrounds}. First, we see a much larger spread in the histograms compared to the other two cases.  Second, we see that the spread reduces significantly by increasing $n_\mathrm{sub}$. In this case the benefit from spectral imaging more than balances the sub-optimality effect and allows us to reduce the bias on the reconstructed spectral index which then reduces the bias on $r$, as shown in Fig.~\ref{fig:r_vs_nsub}.


\section{Simulations with Commander}
\label{app:commander}

    \subsection{Simulation pipeline}
    \label{app:commander_pipeline}
    
We describe here the simulation pipeline for the analysis performed using the Commander code \citep{eriksen_2006,eriksen_2008}.
We generated 100 CMB power spectra using CAMB \citep{Lewis:1999bs} from the set of cosmological parameters shown in Table \ref{tab:1}.

\begin{table}[h!]
    \caption{\label{tab:1}Set of cosmological parameters from the \textit{CAMB Python example notebook} in the CAMB documentation (\protect\url{https://camb.readthedocs.io/en/latest/CAMBdemo.html}).
    %\footnote{\url{https://camb.readthedocs.io/en/latest/CAMBdemo.html}}.
    }
    \renewcommand{\arraystretch}{2}
    %\setlength{\tabcolsep}{0.1pt}
    \begin{center}
        \begin{tabular}{p{2cm} m{2cm}}
\hline
\hline
\makecell[c]{$\mathrm{H}_0$} & \makecell[c]{67.5}\\
\makecell[c]{$\Omega_b h^2$} & \makecell[c]{0.022}\\
\makecell[c]{$\Omega_c h^2$} & \makecell[c]{0.122}\\
\makecell[c]{$\Omega_K$} & \makecell[c]{0}\\
\makecell[c]{$m_\nu$} & \makecell[c]{0.06}\\
\makecell[c]{$\tau$} & \makecell[c]{0.06}\\
\makecell[c]{$A_s \times 10^{-9}$} & \makecell[c]{2}\\
\makecell[c]{$n_s$} & \makecell[c]{0.965}\\
\hline

\end{tabular}
%\footnotesize{\st{$^1$\url{https://camb.readthedocs.io/en/latest/CAMBdemo.html}}}
\end{center}
\end{table}


\begin{table}[h!]
    \caption{\label{tab:2}Parameters used for analyzing simulations with Commander.}
    \renewcommand{\arraystretch}{2}
    \setlength{\tabcolsep}{1pt}
    \begin{center}
        \begin{tabular}{p{5cm} m{2cm}}
            \hline
            \hline
            Number of CMB realizations\dotfill & \makecell[c]{100}\\
            Map $N_\mathrm{side}$\tablefootmark{a} \dotfill	& \makecell[c]{64} \\
            Multipole range\tablefootmark{b} \dotfill	&\makecell[c]{21--128} \\
            $\Delta\ell$\dotfill &\makecell[c]{\phantom{0}{35}} \\
            Input $r$\dotfill & \makecell[c]{\phantom{00}{0}} \\
            Residual lensing fraction\tablefootmark{c} \dotfill & \makecell[c]{\phantom{000}100\%} \\
            Sky fraction [\%]\dotfill & \makecell[c]{\phantom{0000}3\%} \\
            \makecell[l]{Sky patch center\\ \mbox{}[Equatorial coord.]\dotfill}\dotfill & 
                \makecell[c]{$\text{RA}=0^\circ$\\\phantom{1}$\text{DEC}=-57^\circ$\rule[-2.2ex]{0pt}{0pt}}\\
            FWHM\dotfill & \makecell[c]{1$^\circ$}\\
            \hline
            \multicolumn{2}{l}{\makecell[l]{
              \tablefoot{\\
               \tablefoottext{a}{Limited by computational time.}
               \\
               \tablefoottext{b}{Limited by $N_\mathrm{side}=64$.}
               \\
               \tablefoottext{c}{The value of 100\% means that all the lensing signal was left.}}
              }}
        \end{tabular}
    \end{center}
\end{table}

%The accuracy for lensing B-modes is defined by setting $\ell_\mathrm{max} = 2500$ and 
%lens\_potential\_accuracy = 0
%in the code \textbf{(write it or not?)}. 
We smoothed both the CMB and foreground signals with a Gaussian beam with FWHM of 1$^\circ$ and applied the HEALPix pixel window function at $N_\mathrm{side}=64$.
The only model used to generate the foreground is the $\mathbf{d6s1}$ described in Sect. \ref{sec_simulated_sky_model}, in particular setting the dust correlation length to $\ell_\mathrm{corr}=10$.

For this test we considered a circular patch covering 3\% of the sky, centered on the QUBIC observation field ---corresponding to RA = $0^\circ$ and DEC = $-57^\circ$. 
We made this choice in order to be consistent with an already-existing BI experimental set-up, after observing that such a sky region is reasonably close to the CMB-S4 one
---considered throughout the analysis in the main text. We note that the foreground contamination is going to differ in the two pipelines, as well as the CMB realization,
leading to a slightly different estimate of $r$. Nevertheless, we still expect the final posterior distribution to be compatible within the instrumental uncertainty, the component separation residual 
contamination, and the additional statistical uncertainty arising from the different CMB realizations. However, the latter contribution is reduced by $\sqrt{N}$, where
$\mathrm{N}=100$ is the number of CMB realizations considered in the analysis.

%We considered a 3\% sky patch around the center of the one observed by QUBIC. The coordinates of the center of the patch are RA = $0^\circ$ and DEC = $-57^\circ$. 
For computational reasons only four configurations have been studied.  They are \mbox{CMB-S4/BI} with 1, 3, 5, and 7 sub-bands.

For each simulated sky map, we generated a second version by taking the same CMB, synchrotron, and dust realization, and a different Gaussian noise realization.
The analysis chain is the same as outlined in Sect. \ref{sec_simulation_pipeline} except that we use an apodization radius of 4.6$^\circ$ instead of 4$^\circ$.
We performed the component separation sampling the amplitudes $\mathrm{a}_{\mathrm{CMB}}, \mathrm{a}_{\mathrm{s}}, \mathrm{a}_{\mathrm{d}}$ and the spectral indices $\beta_{\mathrm{s}}, \beta_{\mathrm{d}}$ by means of the following Gibbs chain:
\begin{align}
        \{\mathrm{a}_{\mathrm{CMB}},\mathrm{a}_\mathrm{s}, \mathrm{a}_\mathrm{d}\}^{i+1} &\leftarrow P(\mathrm{a}_{\mathrm{CMB}},\mathrm{a}_\mathrm{s}, \mathrm{a}_\mathrm{d} \,|\, \beta_\mathrm{s}^i,\beta_\mathrm{d}^i,\mathrm{d})\tag{7a}\label{eq:gibbs_chain_1}\\
        \notag\\
        \beta_\mathrm{s}^{i+1} &\leftarrow P(\beta_\mathrm{s} \,|\, \mathrm{a}_{\mathrm{CMB}}^{i+1},\mathrm{a}_\mathrm{s}^{i+1}, \mathrm{a}_\mathrm{d}^{i+1},\beta_\mathrm{d}^i,\mathrm{d})\tag{7b}\label{eq:gibbs_chain_2}\\
        \notag\\
        \beta_\mathrm{d}^{i+1} &\leftarrow P(\beta_\mathrm{d} \,|\, \mathrm{a}_{\mathrm{CMB}}^{i+1},\mathrm{a}_\mathrm{s}^{i+1}, \mathrm{a}_\mathrm{d}^{i+1},\beta_\mathrm{s}^{i+1},\mathrm{d})\tag{7c}\label{eq:gibbs_chain_3}\,.
\end{align}

%\noindent
The spectral indices are sampled at $N_\mathrm{side}=8$ as for the \mbox{FGBuster} pipeline.
We generated 1000 MCMC samples for each input sky realization and discarded the first 100 samples as burn-in.
The two noise uncorrelated versions of the same sky realization are associated with two parallel sampling chains.
We compute the cross-spectra between these two parallel chains, iteration by iteration, to collect a set of 900 spectra for each CMB realization considered in the analysis. We also average all of the sampled maps produced in a single chain into a mean map, and for every couple of parallel chains we compute the cross-spectrum between the two mean maps.

After the component separation, we compute the likelihood function for the cross spectrum of each mean map, exploiting the sample-based noise covariance matrix obtained by all the power spectra from the corresponding sampling chain.


    
    \subsection{Results}
    \label{app:commander_results}

    From the probability density functions of the model parameters obtained with Commander, we find that the upper limit to the estimation of a single realization of $r$ is reduced with the number of sub-bands, as shown in Figure \ref{fig:commander_r0}. The $r$ bias and $\sigma(r)$ are greater than the FGBuster results due to the marginalization over the foreground components.

\begin{figure}[ht]
	\includegraphics[width=9cm]{figure/appendix_commander_r0.png}
        \caption{\label{fig:commander_r0}Mean and standard deviation of the best fit distributions
        obtained with Commander, using the \textbf{d6s1} model with $\ell_\mathrm{corr}=10$ and $r=0$.}
\end{figure}

Increasing the number of sub-bands also reduces the standard deviation of the marginalized posterior distributions of the standard deviation of the spectral indices for all pixels. Figure \ref{fig:commander_indices_all_pixels} shows the comparison between the reconstructed dust and synchrotron spectral indices. As in Appendix~\ref{app:reconstruction_foregrouds} we compare reconstructed spectral indices with the input ones, from~\cite{Planck_15} for 1, 3, and 5 sub-bands on all pixels. Because of frequency decorrelation, the spectral indices residuals for dust are not as meaningful as the distribution of reconstructed $r$ shown in Fig.~\ref{fig:commander_r0}. This analysis has not been performed for the 7 sub-band configuration results because of data storage issues. Here a single $\Delta \beta$ from the plotted distributions represents the difference between the mean value of the marginal distribution on a single pixel for a given sky realization and the template value in the same pixel from the model. These results are in agreement with the FGBuster simulations.

%\begin{figure}[ht]
%	\includegraphics[width=9cm]{figure/appendix_commander_indices_d6_one_pixel.pdf}
%        \caption{\label{fig:commander_indices_one_pixel}: Reconstruction of foregrounds parameters in the \textbf{d6s1} case for one pixel with the Commander pipeline.}
%\end{figure}

\begin{figure}[ht]
    %\centering
    \hspace{-0.1cm}
	\includegraphics[scale=0.2]{figure/appendix_commander_indices_d6_all_pixels.pdf}
        \caption{\label{fig:commander_indices_all_pixels}: Reconstruction of foreground spectral indices for the \textbf{d6s1} model with the Commander pipeline.}
\end{figure}

\end{document}
%



%%%% End of aa.dem
