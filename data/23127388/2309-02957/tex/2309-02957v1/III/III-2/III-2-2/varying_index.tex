
In the previous section, considered the case of a simple model with constant indices across the sky. This is clearly an oversimplified case~\citep{Planck_15}. Let's now consider a model with foreground spectral indices that varies across the sky. Our simulated foregrounds have inputs $\beta$ maps with a pixelization $N_{\text{side}} = 256$, based on the results from  \cite{Planck_15}. For computational reasons and in order to be more realistic, the fit of our spectral indices during component separation is performed with a coarser pixelization $N_{\text{fit}} = 8$. We expect this to introduce a slight bias on the spectral indices recovered during component separation, resulting in foreground residuals in our maps and therefore induce a small bias on $r$. 
%\\ {\color{red}[I suggest to expand on this a bit: this gives a bias if we look at the product of the likelihoods, but on the average likelihood we don't see a bias (see fig \ref{fig:result_r_vs_dust_model}: the model d1 is is consistent with the input r)]}.

% \begin{figure}
%     \centering
%     \includegraphics[scale=0.45]{figure/d1_recon_r.png}
%     \caption{{\color{red}Clear this fig}Likelihood for the case of a classical imager ($N_{\text{sub}} = 1$) and a bolometric interferometer ($N_{\text{sub}} > 1$).}
%     \label{fig:like_d1}
% \end{figure}



The whole pipeline is applied first in the standard CMB-S4 (classical imager) case with no sub-band splitting ($Nsub=1$) and then for a BI version of the same instrument where each band above 50 GHz is split into a number of sub-bands ranging from 1 to 8 as in Fig~\ref{fig:Polarization_depth}. We account for BI sub-optimality (introduced in section~\ref{sec_instrument_models}) as in~\citep{2020.QUBIC.PAPER2} and Appendix~\ref{appendixA}. We performed an estimation of $r$ for each of our sky realization and we show the result in figure \ref{fig:like_d1} panel {\bf a)}. The increase of $\sigma(r)$ is caused by BI sub-optimality. We see that adding more sub-bands does not add information in this case and, in particular, does not reduce the residual bias on $r$ as it is due to the coarse pixelization on the reconstructed spectral indices that leaves some foreground residuals on the clean CMB maps.
