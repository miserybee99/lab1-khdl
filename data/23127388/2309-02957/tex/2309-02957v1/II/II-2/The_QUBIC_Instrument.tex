\st{QUBIC is the first CMB B-mode experiment based on bolometric interferometry. The instrument observes the sky in two frequency bands, centered at 150~GHz and 220~GHz, respectively, with a 25$\%$ bandwidth.} \st{Each feed-horn array is made of 400+400 back-to-back antennas and each focal plane is equipped with 992 TES bolometers.} \st{QUBIC will observe from the Alto Chorrillos site, in the Argentinian Andes, at about 5000 meters a.s.l. and will provide an upper limit on $r<0.015$ at $68\%$ C.L. after three years of observations}% \citep{2020.QUBIC.PAPER1}.

\st{The first QUBIC prototype is referred to as the  \textit{technological demonstrator} (TD), a reduced version of the instrument to demonstrate bolometric interferometry with laboratory measurements and sky observations. The QUBIC-TD uses the same cryostat} \citep{Masi_2022}, HWP \citep{D_Alessandro_2022} \st{and polarizing grid as the final instrument, but will observe only in the 150 GHz channel. It is equipped with smaller diameter mirrors, a smaller feed-horn array made of 64+64 back-to-back feed-horns, and a smaller focal plane, made of 248 bolometers.}

\st{After extensive laboratory testing the QUBIC-TD was installed at the observation site during November 2022} (see Fig.~\ref{fig:QUBIC_Rendering}). \st{Routine observations will start after commissioning.} \st{The interested reader can find the details of the laboratory tests and their results in:}% \citet{2020.QUBIC.PAPER1, 2020.QUBIC.PAPER2, 2020.QUBIC.PAPER3, 2020.QUBIC.PAPER4, 2020.QUBIC.PAPER5, 2020.QUBIC.PAPER6, 2020.QUBIC.PAPER7, 2020.QUBIC.PAPER8}.

\st{The top-left panel of figure}~\ref{fig:SB_meas_vs_theo} \citep[taken from][]{2020.QUBIC.PAPER1} \st{shows the synthesized beam measured by one of the focal plane TES detectors compared to a simulation of the same beam (top-right panel). The bottom panel of the same figure} \citep[taken from][]{2020.QUBIC.PAPER3} \st{shows the synthesized beams measured at various frequencies. These data show the multiple-peaked shape of the synthesized beam with good agreement with the theoretical predictions and the expected frequency dependence of the secondary peak positions. The measured instrumental noise (detectors and read-out) is $2.06\times 10^{-16}\,\mathrm{W}/\sqrt{\mathrm{Hz}}$} \citep{2020.QUBIC.PAPER4} \st{and the median cross-polarization of the detectors is $0.12\%$ }\citep{2020.QUBIC.PAPER6}.


%\begin{figure}[ht]
%	\includegraphics[width=9cm]{II/II-2/Qubic_on_site.png} %II/II-2/
%        \caption{QUBIC-TD in the Alto Chorrillos site. The cryostat is placed on a movable mount to scan in azimuth and elevation. The system is placed inside a motor-controlled dome that can be opened and closed.}
%     \label{fig:QUBIC_Rendering}
%\end{figure}
% 
\begin{figure}[ht]
	\includegraphics[width=9cm]{II/II-2/SB_data_vs_theory_new.png}\\
	\mbox{}\\
	\includegraphics[width=9cm]{figure/SB_vs_frequency_mod.pdf}
        \caption{\label{fig:SB_meas_vs_theo}\textit{Top panel}: synthesized beam measured for one detector at 150\,GHz (left) compared with the  simulated 150\,GHz synthesized beam for the same detector \citep[right, from][]{2020.QUBIC.PAPER1}. \textit{Bottom panel}: synthesized beam for one detector measured at various frequencies \citep[from][]{2020.QUBIC.PAPER3}.}
     
\end{figure}
