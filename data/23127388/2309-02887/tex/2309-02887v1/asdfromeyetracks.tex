% This is samplepaper.tex, a sample chapter demonstrating the
% LLNCS macro package for Springer Computer Science proceedings;
% Version 2.21 of 2022/01/12
%
\documentclass[runningheads]{llncs}
\usepackage[shortlabels]{enumitem}
\usepackage[T1]{fontenc}
\usepackage{graphicx}
\usepackage{tabularx}
\PassOptionsToPackage{hyphens}{url}
\usepackage{hyperref}
\usepackage{multirow}
\usepackage{float}
\usepackage{fancyvrb}
\graphicspath{{./images/}}
% If you use the hyperref package, please uncomment the following two lines
% to display URLs in blue roman font according to Springer's eBook style:
%\usepackage{color}
%\renewcommand\UrlFont{\color{blue}\rmfamily}

% Add a period to the end of an abbreviation unless there's one
% already, then \xspace.
\usepackage{xspace}
\makeatletter
\DeclareRobustCommand\onedot{\futurelet\@let@token\@onedot}
\def\@onedot{\ifx\@let@token.\else.\null\fi\xspace}
%
\def\eg{\textit{e.g}\onedot} \def\Eg{\textit{E.g}\onedot}
\def\ie{\textit{i.e}\onedot} \def\Ie{\textit{I.e}\onedot}
\def\cf{\textit{c.f}\onedot} \def\Cf{\textit{C.f}\onedot}
\def\etc{\textit{etc}\onedot} \def\vs{\textit{vs}\onedot}
\def\wrt{w.r.t\onedot} \def\dof{d.o.f\onedot}
\def\etal{\textit{et al}\onedot}
\makeatother

\usepackage[dvipsnames]{xcolor}
\newcommand{\todo}[1]{\textcolor{red}{[TO DO: #1]}}
\newcommand{\franz}[1]{\textcolor{ForestGreen}{[FRANZ: #1]}}
\newcommand{\marco}[1]{\textcolor{blue}{[LORENZO: #1]}}
\newcommand{\RQBOX}[1]{\noindent\fbox{\parbox{0.97\linewidth}{\bfseries #1}}}

\begin{document}

\title{A deep Natural Language Inference predictor without language-specific training data}
\titlerunning{A deep NLI predictor without language-specific training data}
% If the paper title is too long for the running head, you can set
% an abbreviated paper title here
%
% \author{Anonymous ICIAP23 submission}
\author{Lorenzo Corradi\inst{1} \and
Alessandro Manenti\inst{1} \and
Francesca Del Bonifro\inst{1} \and
Francesco Setti\inst{2}\orcidID{0000-0002-0015-5534} \and
Dario Del Sorbo\inst{1} \orcidID{0000-0003-0390-7540}}

\authorrunning{Lorenzo Corradi \etal}
% %
\institute{Data Science Team, Lutech S.p.A., Cinisello Balsamo, Italy
\and
Department of Engineering for Innovation Medicine, University of Verona, Verona, Italy\\
\vspace{.3em}\email{l.corradi@lutech.it}}



\maketitle              % typeset the header of the contribution

% Abstract
\begin{abstract}
In this paper we present a technique of NLP to tackle the problem of inference relation (NLI) between pairs of sentences in a target language of choice without a language-specific training dataset.
We exploit a generic translation dataset, manually translated, along with two instances of the same pre-trained model --- the first to generate sentence embeddings for the source language, and the second fine-tuned over the target language to mimic the first. This technique is known as Knowledge Distillation.
%A full summary of the performances over different tasks in the target language is provided. 
The model has been evaluated over machine translated Stanford NLI test dataset, machine translated Multi-Genre NLI test dataset, and manually translated RTE3-ITA test dataset.
%, respectively achieving the performances in terms of accuracy of \textbf{76.04\%}, \textbf{72.74\%}, and \textbf{67.50\%}.
We also test the proposed architecture over different tasks to empirically demonstrate the generality of the NLI task. The model has been evaluated over the native Italian ABSITA dataset, on the tasks of Sentiment Analysis, Aspect-Based Sentiment Analysis, and Topic Recognition. 
%We balanced the dataset applied to each task to obtain a 50/50 division, and we respectively achieve the performances in terms of accuracy of \textbf{88.12\%}, \textbf{94.03\%}, and \textbf{71.19\%}.
We emphasise the generality and exploitability of the Knowledge Distillation technique that outperforms other methodologies based on machine translation, even though the former was not directly trained on the data it was tested over.
%While the model developed has still room for improvement, it is suitable to operate unsupervised free-text information extraction in a production environment.
\keywords{Natural Language Inference \and Knowledge Distillation \and Domain adaptation}
\end{abstract}
\section{Introduction}
\label{Introduction}
Natural Language Processing (NLP) has gained huge improvements and importance in the last years. It has many different applications as it helps in many ways human language productions understanding and analysis in an automated manner.
Natural Language Inference (NLI) is one of these applications: it is the task of determining the inference relation between two short texts written in natural language, usually defined as \emph{premise} and \emph{hypothesis}~\cite{SNLI,MNLI}. 
This implies the extraction of the meaning of the two texts and then evaluating if the \emph{Premise} (P) entails the \emph{Hypothesis} (H) (\textit{entailment} situation), if the \emph{premise} and the \emph{hypothesis} are in contradiction between each other (\textit{contradiction} situation), or if none of these two situations happen and there is no inference relation among the two texts (\textit{neutral} situation). 
This is a challenging task that requires understanding the nuances of language and context, as well as the ability to reason and make logical implications.
%
The relevance of this task can be easily understood by highlighting some of its possible applications. 
Common tasks based on NLI are Aspect-Based Sentiment Analysis (ABSA), Sentiment Analysis (SA), and Topic Recognition (TR) described in Sec.\ref{sec:exp}.
% ABSA is the task of extracting positivity or negativity about a certain aspect among others in a given text e.g., negativity with respect to the position of a restaurant given a review about that restaurant.
% SA is just ABSA in general form in which the model capture positivity or negativity of a certain text.
% TR is the task of extracting topic subject in a given text.
All these tasks, when approached with NLI strategy, are tackled by comparing an input text (e.g., ``We really enjoyed the food, it is tasty and cheap, the staff was very nice and kind. However the restaurant is very hard to reach.'') and an hypothesis about the input text (e.g., ``The position of the restaurant is difficult to be reached'') and predicting if the input text either entails, contradicts, or is not related to the hypothesis (``Entailment'' is the correct prediction in the previous example). 
%From these possible applications it is easy to see the power of the underlying NLI models.
A common problem for many NLP tasks is the fact that the developed models usually require a big amount of natural language productions data, and usually they are made available in the English language. 
There are many languages that are underrepresented in these NLP dataset contexts and this made interesting tools hard to develop in these other languages, and there is the need to solve this to make these advance in tech available for low represented languages too. 
Data scarcity may be tacked with different strategies and this work describe some of them relatively to the NLI task in the Italian language. 
%Let us consider a large corpus of sentences in the target language, with the objective to extract information from each sentence. 
The goal of this research is to build a model with the following traits:
\begin{enumerate}[(a)]
\item %the ability to understand the inference relation between sentence pairs in a specific language. Informally, this means to train a model on a NLI dataset;
it can perform the NLI task in a specific language;
\item based on the sentence embedding operation, such as in~\cite{SENTENCEBERT}; %consider having a dataset of $N$ reviews and for each of them, we want to query for $Q$ pieces of information. Without the sentence embedding operation, we would need $N \cdot Q$ calls to the model --- we feed two input sentences in parallel, perform the embedding operation on the sentence pair in an independent manner, and compare them together to obtain the result. With this method, we would need to compare all sentences in our $N$ dataset, resulting in $\sim(N^2)/2$ comparisons. By embedding all the sentences first, then compare them together by means of another artifact, the encoder model -- i.e. the most computationally expensive part of the model -- is invoked $\sim (N + Q)$ times.
\item it is able to understand another language, in this case Italian; %The majority of the published models are in English. This reflects the fact that research is published in English, and relevant NLI datasets are all in English. 
;
\item it is able of being general and not requiring any re-train for each specific industrial task (ABSA, SA, TR); % as Companies' needs evolve over time; %We propose a method that addresses this challenge and obtains acceptable results.
\end{enumerate}
% To the best of our knowledge, we did not find any comprehensive NLI datasets in the Italian language, large enough to support a Deep Learning model training.
We can state the Research Question (RQ) which drive our effort is

\RQBOX{RQ: It is possible to build a NLI model with acceptable performances on NLI related downstream tasks in Italian language compliant with \textit{(a)}, \textit{(b)}, \textit{(c)}, \textit{(d)} constraints, without requiring a language-specific dataset?}
% \vspace{5pt}
% \begin{itemize} \itshape
    % \item [] \textbf{RQ:} It is possible to build a NLI model with acceptable performances on NLI related downstream tasks in Italian language compliant with \textit{(a)}, \textit{(b)}, \textit{(c)}, \textit{(d)} constraints, without requiring a language-specific dataset?
% \end{itemize}

% To achieve these aims we explored two different approaches, both of them based on the Transformers architecture to embed the sentences which bring state-of-the-art results in English-based NLP and NLI tasks.
The main differences among the proposed models to achieve these aims is the training approach: one is based on Knowledge Distillation (KD)\cite{KD} a technique which aim to transfer knowledge from a \emph{Teacher} model (English-based NLI model) to \emph{Student} model (that will handle the Italian language).% and which require a corpus of parallel English-Italian sentences. 
The other model includes a step for the dataset translation from English to Italian by means of a Machine Translation model \cite{NLLB}. % in order to obtain a NLI Italian dataset from an English one and train the NLI model directly on the translated data.
%%%%%%%%%%
% The lack of a comprehensive NLI dataset in Italian has been the key driver to adopt the architecture described in Sec.~\ref{Method}. 
% This procedure can be generalised to any task involving a language-specific dataset. 
% Another plus of this methodology is that we require no dataset labeling costs. 
The approach exploiting KD has been demonstrated to have NLI capability in the target language, namely Italian, without being exposed to a NLI training dataset in Italian. This model has been named \textbf{I-SPIn} (\textbf{I}talian-\textbf{S}entence \textbf{P}air \textbf{In}ference) and is available at this \href{https://huggingface.co/Lutech-AI/I-SPIn}{link} along with all instructions for usage. 

The remainder of this paper is structured as follows: in Sec.~\ref{RelatedWork} we report literature and datasets discussion, Sec.~\ref{Method} describes the two implemented approaches, Sec.~\ref{sec:exp} reports the settings and results of the performed experiments, Sec.~\ref{sec:discussion} reports discussion and conclusions about this work.
\section{Related Work} 
\label{RelatedWork}
%Inferring in an automated way whether a text representing the \emph{hypothesis} is an entailment, a contradiction, or undetermined given another text known as the \emph{premise} is a complex task which implies both language understanding and a form of logic reasoning. 
%This task is known as Natural Language Inference (NLI) and several works have been done to tackle it, many of them reaching very impressive results on English benchmark datasets (most used are SNLI~\cite{SNLI}, and MNLI~\cite{MNLI}). 
Common approaches to tackle NLI include Neural Networks, such as Recurrent Neural Networks or Transformer based methods~\cite{BILSTM,BERT}.
\cite{BILSTM} presents an architecture based both on learning \emph{Hypothesis} and \emph{Premise} in a dependent way using bidirectional LSTM and Attention mechanism~\cite{attention-original} to extract the text pair representation needed for final classification. The obtained results on SNLI~\cite{SNLI} validation set gives a $89\%$ accuracy.
\cite{BERT} describes  language representation model (BERT) building a model based on Transformers~\cite{ATTENTION,TRANSFORMERS} and pre-training it in a bidirectional way, this has the aim to serve as a pre-trained model that can be fine-tuned on several different tasks including NLI. BERT is fine-tuned and tested on MNLI dataset~\cite{MNLI} and reaches around $86\%$ accuracy.
Recent researches~\cite{T5,XLNET} demonstrate that Transformers models~\cite{ATTENTION} are more suited for the NLI task, consistently surpassing neural models~\cite{TRANSFORMERS}. %Also, the architecture scales with training data and model size, facilitates efficient parallel training, and captures long-range sequence features. 
%The authors in~\cite{T5} build T5 model exploit Large Language Models that uses corrupted input text to reconstruct the original text to account for noise, while~\cite{XLNET} learns conditional distributions for all permutations of tokens in a sequence to build up XLNet model.
%In~\cite{DBLP:journals/corr/abs-2002-04815} there is also some NLI experimentation on downstream tasks such as ABSA. The authors uses BERT-based architecture~\cite{BERT} and fine-tuned it on the desired task. The obtained performance reaches $91\%$ accuracy for NLI on SNLI~\cite{SNLI} dataset and a $73-85\%$ accuracy range on different domains for ABSA task on SemEval 2014 Task 4~\cite{pontiki-etal-2014-semeval} and ACL 14 Twitter~\cite{dong-etal-2014-adaptive} datasets.
All of these high performance approaches mainly hold for English language as it is the language in which there is the higher data availability. 
%This is a known issue also for other NLP tasks as these require natural language datasets and many languages are too underrepresented in data availability to build competitive results. 
Some multi-language NLI approaches are proposed in~\cite{hypernymy}, where cross-lingual training, multilingual training, and meta learning are attempted using a dataset extracted from Open Multilingual WordNet. The best model resulted to be the one exploiting meta learning and reached $76\%$ accuracy on True/False classification task of text pairs for the Italian language.
\cite{magnini-etal-2014-excitement} represents another work on multi-language NLI where the Excitement Open Platform is presented as open source software for experimenting in NLI related tasks. It has many linguistics and entailment components based on transformations between \emph{Premise} (P) and \emph{Hypothesis} (H), edit distance algorithms, and a classification using features extracted from P and H. Italian Language is tested on a manually translated RTE-3 dataset~\cite{giampiccolo-etal-2007-third} and the best model has $63\%$ accuracy. 
In the context of an Italian Textual Entailment competition the task Recognizing Textual Entailment (RTE) is proposed. It is similar to NLI task but it only contains two Entailment Yes/No classes. The competition's winner model is described in~\cite{Bos2009TextualEA} and it is based on a open source software EDITS based on edit distance reaching $71\%$ accuracy on the convention EVALITA 2009 dataset which is extracted from Wikipedia.
\cite{Pakray2012RecognizingTE} presents a model based on translation. The input texts can be in any language and are translated into English using a standalone machine translation system. The authors show that machine translation can be used to successfully perform the NLI related tasks or when P and H are provided in different languages. For Italian, it uses Bing translation and it is tested on EVALITA 2009 dataset reaching $66\%$ accuracy.
% In our approach we used a BERT~\cite{BERT} architecture fine-tuned for the NLI task and applied to the Italian case by using Knowledge Distillation~\cite{KD}, a powerful technique to transfer knowledge among models. 
% Another approach uses a BERT~\cite{BERT} architecture and trains it on a translated dataset using the machine translation model No Language Left Behind (NLLB) \cite{NLLB}, however the model based on Knowledge Distillation reaches the best results as can be seen in Sec.\ref{sec:exp}.
%%%%%%%%%%
% The industrial applications include, among others: Product or service review analysis, our primary target; Brand reputation analysis; Social media sentiment analysis; Improving confidence and coherence of Large Language Models in Italian~\cite{CONCORD}; etc.
%%%%%
\subsubsection{Datasets}
\label{datasets}
The datasets used in this work are described in this paragraph and examples can be found in Appendix \ref{appendix:data}.
% section will provide a comprehensive overview of the datasets used for training and testing of the different architectures.
%\subsubsection{Stanford NLI} \label{Stanford NLI}

The Stanford NLI (SNLI)~\cite{SNLI} corpus is a collection of 570k human-written English sentence pairs manually labeled for balanced classification with the labels ``entailment'', ``contradiction'', and ``neutral'', supporting the task of NLI. %It was developed with the objective to serve both as a benchmark for evaluating representational systems for text, as well as a resource for developing NLP models of any kind.
The SNLI dataset presents the canonical dataset split --- consisting of train, validation, and test sets.

%\subsubsection{Multi-Genre NLI} \label{Multi-Genre NLI}
Multi-Genre NLI (MNLI)~\cite{MNLI} corpus is a crowd-sourced collection of 433k sentence pairs annotated with textual entailment information. The corpus is modeled on the SNLI corpus, but differs in that covers a range of \emph{genres} of spoken and written text. 
%A genre can be seen as a peculiar data source. For instance, a sentence extracted from a letter will appear in the corpus with the genre ``Letters''. 
The train set is composed of sentences with the same genres: ``Telephone'', ``Fiction'', ``Government'', ``Slate'' and ``Travel''. 
The MNLI dataset supports a distinctive cross-genre generalisation evaluation. There is a matched validation set which is derived from the same source as those in the training set, and a mismatched validation set which do not closely resemble any genres seen at training time.
%The mismatched validation set is composed of sentences with the following genres: ``Face-to-face'', ``9/11'', ``Letters'', ``Oxford University Press'', ``Verbatim''. Currently, no test sets are available for this dataset.

%\subsubsection{RTE3-ITA and RTE2009} \label{RTE}
RTE datasets (RTE3-ITA and RTE2009)  are English-native NLI datasets, manually translated by a community of researchers. 
%The Italian variants chosen are RTE3-ITA and RTE-2009. 
The Italian version RTE3-ITA refers to the third refinement of this dataset%.
%Unfortunately the official website is down, but 
\footnote{The validation and test datasets can be downloaded at this \href{https://github.com/gilnoh/RTEFormatWork/tree/master}{link}.}.
Instead, RTE2009 was submitted for the EVALITA 2009 Italian campaign \cite{EVALITA}\footnote{The validation and test datasets can be found at this \href{https://www.evalita.it/campaigns/evalita-2009/data-distribution/}{link}.}.
These datasets are only used for testing, since they contain too few observations to be suitable for training. 
The RTE3-ITA dataset contains 1600 observations, whereas RTE-2009 contains 800 observations.
Unlike classical NLI, these datasets present only two labels: ``Entailment'' and ``No-Entailment''.

%\subsubsection{TED2020} \label{TED2020}
TED2020~\cite{TED2020} is a generic translation dataset. The option (English--Italian) has been selected for training among more than a hundred possible languages. The dataset consists of more than 400k parallel sentences. The transcripts have been translated by a community of volunteers. This dataset is used to make a model understand different languages~\cite{KD}, starting from a language known to the model. %The TED2020 dataset presents the canonical train-validation-test split.
% An example from TED2020 translation dataset is:
% \begin{enumerate}[align=left]
%     \item [EN:] ``I gave my speech, then went back
% to the airport to fly back home.''
%     \item  [IT:] ``Io feci il mio discorso, poi
% andai all’aeroporto per tornare.''
% \end{enumerate}

\section{Method}
\label{Method}
Three different architectures will be detailed throughout the section.
In Sec.~\ref{NLI training in the source language} the objective is to obtain the a model that is able to perform NLI in English. Starting from this model, we propose two parallel approaches to perform NLI in the target language. One is detailed in Sec.~\ref{Knowledge Distillation in the target language}, and the other is detailed in Sec.~\ref{Machine Translation in the target language}. Both approaches attempt a domain adaptation and generalisation in the target language --- namely, Italian, while lacking a language-specific dataset.
The models' parameters were selected among few different possibilities suggested by online informal documentation and literature. No cross-validation or grid-search analyses have been performed for computational constraints. Therefore, no guarantees on the optimality of the parameters can be made.
To reduce computational complexity during the inference phase for the models described in Sec.~\ref{Knowledge Distillation in the target language} and Sec.~\ref{Machine Translation in the target language}, we recommend to split the model to obtain independent instances of encoder and classifier. The proposed methodology is the following: transform all the sentence pairs in vectorial forms --- with the encoder --- first; in a second phase, the classifier will receive the embeddings to return an inference relation.
\subsubsection{NLI training in the source language} \label{NLI training in the source language}
The proposed solution makes use of a Transformer~\cite{ATTENTION}. The Transformer lately has become the state-of-the-art architecture for NLP, as detailed in \ref{RelatedWork}.
The first step of our methodology is to retrieve a sentence encoder model, based on Transformers. This sentence encoder model is already fine-tuned for general purposes over different languages. 
The encoder of choice to transform sentences in vectors was Sentence-BERT~\cite{SENTENCEBERT}. It is a fine-tuning of BERT~\cite{BERT}, that is a word embedding Transformer model, tailored for the task of sentence embedding. 
It has the ability to perform sentence embedding faster than BERT as detailed in~\cite{SENTENCEBERT}\footnote{Sentence-BERT was downloaded from this \href{https://huggingface.co/sentence-transformers/paraphrase-multilingual-mpnet-base-v2}{link}.}, by means of a Siamese training approach~\cite{SIAMESE}. 
Referring to this model with the term Sentence-BERT is inappropriate, since it has been fine-tuned on RoBERTa~\cite{ROBERTA}, that is a larger counterpart of BERT. Hence, the name Sentence-RoBERTa would be more appropriate. 
In this paper we will adopt the name Sentence-BERT to refer to any siamese structure accepting a sentence pair as input, including the instance of Sentence-RoBERTa to be fine-tuned.
Since Transformers are computationally expensive to train from scratch, we decided to test a multilingual version of Sentence-BERT and fine-tune it on SNLI and MNLI merged together to create a single NLI dataset.
After a fine-tuning session over the merged NLI dataset, the result is a model based on Transformers, that can proficiently address the NLI task --- only in English though, despite being originally trained on multiple languages. More information about this work available at~\cite{MULTILINGUAL}.

The output of the fine-tuned Sentence-BERT is composed of an embeddings pair, containing a vectorial representation of the premise and the hypothesis. Note that the sentence encoder model is invoked two separate times for this operation, for complexity optimisation reasons.
The Sentence-BERT output embeddings have been further transformed to maximise and emphasise the relevant information for our task. In detail, the following operations have been applied:
\begin{itemize}
    \item Element-wise product. Captures similarity of the two embeddings, and highlights components of the embeddings that are more relevant than others.
    \item Difference. Asymmetric operation; captures the direction of implication. We want the hypothesis to imply the premise, and not vice-versa.
\end{itemize}

The two transformed embeddings were concatenated and passed as input to a fully-connected Feed Forward architecture of six (6) layers, detailed in Appendix \ref{appendix:param}, with three (3) outputs (``Entailment'', ``Neutral'', ``Contradiction''), to predict the probability of the sentence pair to belong to each NLI class.
Finally, a softmax function was applied to the three-dimensional vector to obtain the class probabilities (Fig.~\ref{Figure 1}).

\begin{figure}%[ht] 
  \centering
\includegraphics[width=0.75\textwidth, keepaspectratio]{images/model_structure_new.png}
  \caption{\label{Figure 1} Model structure. Two sentences are transformed in embeddings. The embeddings are compared with a classifier to get the prediction for the sentence pair.}
\end{figure}

Execution-wise, the NLI fine-tuning task on a Tesla P100--PCIE--16GB GPU was completed in approximately six (6) hours on the merged NLI training dataset composed of an ensemble of SNLI and MNLI datasets, accounting for more than 1M observations. The main parameters can be found in Appendix \ref{appendix:param}.
% \begin{itemize}
% \item \texttt{batch\_size = 8}
% \item \texttt{max\_sentence\_length = 256}
% \item \texttt{max\_tokens\_length = 128}
% \item \texttt{epochs = 1}
% \item \texttt{learning\_rate = 2e-5}
% \item \texttt{epsilon = 1e-8}
% \item \texttt{weight\_decay = 0}
% \item \texttt{accumulation\_step = 8}
% \end{itemize}
In our work, we want to enable a multilingual Transformers-based model, previously fine-tuned for a specific task only in one specific language, to proficiently address that specific task in another language.

\subsubsection{Knowledge Distillation in the target language} \label{Knowledge Distillation in the target language}
The second step of our methodology is to employ a training without language-specific NLI training data and we selected the Knowledge Distillation (KD)~\cite{KD} approach.

KD was born as a model compression technique~\cite{HINTON}, where knowledge is transferred from the teacher model to the student by minimizing a loss function, in which the target is the distribution of class probabilities predicted by the teacher model. 
%A parameter called ``temperature'' was introduced to provide more information as to which classes the teacher model found more similar to the predicted class.
KD is a powerful technique since it can be used for a variety of multiple tasks. In our experiments, we employed KD to perform NLI in the target language, with the objective of forcing a translated sentence to have the same embedding --- i.e. location in the vector space --- as the original sentence. 
The soft targets of the teacher model constitute the labels to be compared with the predictions returned by the student model. 
%The ``temperature'' parameter is not employed since this is not a classification task. 
The task at hand may fall in the domain adaptation problem sphere.

We require a teacher model (encoder) $T$, that maps sentences in the source language to a vectorial representation. 
Further, we need parallel (translated) sentences $D = ((source_1, target_1), ...,(source_n, target_n))$ with $source_j$ being a sentence in the source language and $target_j$ being a sentence in the target language. We train a student encoder model $S$ such that $T(source_j) \approx S(target_j)$. 
For a given mini-batch $B$, we minimise the Mean Squared Error loss function:

\begin{equation}
MSE_{(S, T, D = (source_j, target_j))} = \frac{1}{|B|}\sum_{j \in |B|} ( T(source_j) - S(target_j) )^2 \\
\end{equation}

Two instances of the encoder described in Sec.~\ref{NLI training in the source language} have been taken for the experiment. One acts as teacher encoder model $T$, the other as a student encoder model $S$.
The application of KD has the objective to share the domain knowledge of the teacher encoder model to the student encoder model, and at the same time learn a new vectorial representation for the target language. 
A schematic representation is provided in Fig.~\ref{Figure 2}. 

\begin{figure}%[H]
  \centering
\includegraphics[width = 0.75\linewidth]{images/kd.png}
  \caption{\label{Figure 2} Knowledge Distillation. Teacher encoder model receives source sentences, student model receives target sentences. Student encoder model is updated with new information from the teacher.}
\end{figure}

The obtained NLI classifier, able to understand Italian, accepts a sentence pair to output a NLI label.
Execution-wise, the KD task on a Tesla P100--PCIE--16GB GPU was completed in approximately five (5) hours on the TED2020 (English--Italian) dataset consisting of more than 400k parallel sentences. The main parameters can be found in Appendix \ref{appendix:param}.%, with the following main parameters:
% \begin{itemize}
% \item \texttt{batch\_size = 24}
% \item \texttt{max\_sentence\_length = 256}
% \item \texttt{max\_tokens\_length = 128}
% \item \texttt{epochs = 6}
% \item \texttt{learning\_rate = 2e-5}
% \item \texttt{epsilon = 1e-6}
% \item \texttt{weight\_decay = 1e-2}
% \item \texttt{accumulation\_step = 4}
% \end{itemize}
\subsubsection {Machine Translation in the target language} \label{Machine Translation in the target language}
As an alternative method for our second step, we employ a Large Language Model named No Language Left Behind (NLLB)~\cite{NLLB} to address the lack of language-specific NLI training data. To the best of our knowledge, it was not possible to find a comprehensive NLI dataset in Italian. The RTE3-ITA and RTE-2009 datasets, both detailed in Sec.~\ref{datasets}, together present about 2500 observations, too few to train a Deep Learning model. Therefore, the dataset used to fine-tune this architecture is the same as in  Sec.~\ref{NLI training in the source language}, with an alteration: we perform a translation of the dataset. In fact, the simplest way to perform NLI in a language other than English is to machine translate the ensemble NLI dataset, consisting in SNLI and MNLI merged together.
Note that, for memory and performance optimisation, the ensemble NLI training dataset was dynamically translated during execution by invoking the NLLB model for each mini-batch.
Execution-wise, this fine-tuning task over the target language on a Tesla P100--PCIE--16GB GPU was completed in approximately ten (10) hours on the translated ensemble NLI dataset, consisting of more than 1M sentence-pairs.  The main parameters can be found in Appendix \ref{appendix:param}.% with the following main parameters:
% \begin{itemize}
% \item \texttt{batch\_size = 8}
% \item \texttt{max\_sentence\_length = 256}
% \item \texttt{max\_tokens\_length = 256}
% \item \texttt{epochs = 5}
% \item \texttt{learning\_rate = 4e-5}
% \item \texttt{epsilon = 1e-16}
% \item \texttt{weight\_decay = 1e-4}
% \item \texttt{accumulation\_step = 4}
% \end{itemize}
%The same as for Sec.~\ref{Knowledge Distillation in the target language} can be said about the choice of the parameters.
%The same as for Sec.~\ref{Knowledge Distillation in the target language} can be said about the recommendation for the inference phase to split the model to independent instances of encoder and classifier.

\section{Experiments}
\label{sec:exp}
\subsubsection{NLI results in the source language} \label{NLI results in the source language}
The architecture discussed as in  Sec.~\ref{NLI training in the source language} has been tested over the standard NLI task in English.
For SNLI the accuracy reached $80.69\%$ while for MNLI a $77.00\%$ accuracy is reached. 
% Follow accuracies for the original Stanford NLI and Multi-Genre NLI test sets, detailed in Tab.~\ref{snli mnli acc}.
% \begin{table}%[H]
% \centering
% \caption{\label{snli mnli acc} Stanford NLI and Multi-Genre NLI accuracies.}
% %\resizebox{\columnwidth}{!}{%
% \begin{tabular}{|c|c|c|c|}
% \hline
% Dataset & Task  & Accuracy \\ \hline
% \begin{tabular}[c]{@{}c@{}}Stanford NLI\end{tabular} &
%   NLI &
%   80.69\% \\ \hline
%   \begin{tabular}[c]{@{}c@{}}Multi-Genre NLI\end{tabular} &
%   NLI &
%   77.00\% \\ \hline
% \end{tabular}%
% %}
% \end{table}
\subsubsection{NLI results in the target language} \label{NLI results in the target language}
The architecture discussed as in Sec.~\ref{Knowledge Distillation in the target language} --- that is the main focus of this paper --- has been tested over the standard NLI task in Italian, and compared with the alternative architecture based on Machine Translation. The underlying model, an open-source machine translation model, developed by Facebook, named No Language Left Behind~\cite{NLLB}, was also exploited to obtain a comprehensive Italian NLI dataset, suitable for testing.
Results for the SNLI and the MNLI test sets (both translated in Italian) are detailed in Tab.~\ref{nli_ita_res}.
\begin{table}%[H]
\centering
\label{nli_ita_res}
 \caption{ NLI (IT) results.}
%\resizebox{\columnwidth}{!}{%
\begin{tabular}{|c|c|c|c|c|}
\hline
Dataset  &Task& Acc. & Min F1 & Macro-Avg F1 \\ \hline
SNLI (IT) & NLI & 74.21 (-1.83\% ) &  67.19\% (-4.34\% ) & 74.08\% (-4.94\%) \\ \hline
MNLI-Mismatch (IT) &
   NLI &
72.74\% (\textbf{+1.09\%})&
 64.53\% (\textbf{+0.55\%})
 & 72.78\% (\textbf{+1.37\%}) \\ \hline

 RTE3-ITA & RTE & 67.50\% (\textbf{+4.75\%})
  & 60.12\% (\textbf{+5.55\%}) & 66.35\% (\textbf{+4.85\%}) \\ \hline
RTE-2009 & RTE & 59.00\% (-0.75\%) 
& 31.09\% (-2.65\%) 
& 50.96\% (-1.46\%)\\ \hline
\end{tabular}%
%}
\end{table}
% \begin{table}[H]
% \centering
% \caption{\label{nli ita res} Stanford NLI (IT) results.}
% %\resizebox{\columnwidth}{!}{%
% \begin{tabular}{|c|c|c|c|c|}
% \hline
% Dataset & Task & Metric & Result & Delta\\ \hline
% \begin{tabular}[c]{@{}c@{}}Stanford NLI (IT)\end{tabular} &
%   NLI &
%   Accuracy &
%   74.21\% &
%   -1.83\% \\ \hline
% \begin{tabular}[c]{@{}c@{}}Stanford NLI (IT)\end{tabular} &
%   NLI &
%   Min F1-Score &
%   67.19\% &
%   -4.34\% \\ \hline
%   \begin{tabular}[c]{@{}c@{}}Stanford NLI (IT)\end{tabular} &
%   NLI &
%   Macro-Avg F1-Score &
%   74.08\% &
%   -4.94\% \\ \hline
% \end{tabular}%
% %}
% \end{table}

SNLI results in Tab.~\ref{nli_ita_res} are not far from the theoretical accuracy cap these models have --- presented for NLI in source language. 
This could be interpreted as a success for the training of both architectures.
The Min F1-Score metric captures the most misclassified class. The Neutral class, in general, has been the most challenging to classify, as translation biases may slightly change the connotation of a sentence.
Note that this test is biased towards the Machine Translation-based architecture. Remember that this architecture has been fine-tuned over the translated NLI dataset in the target language; % hence, this test is the obvious standard way to infer the capabilities of the model in question. 
the KD-based architecture, instead, had never seen the NLI dataset in the target language. This suggests that the KD-based architecture may have relevant learning capabilities over this task.
% Follow results for the Multi-Genre NLI test set (translated in Italian), similar to the Stanford NLI dataset.
% \begin{table}[H]
% \centering
% \caption{\label{mnli res} Multi-Genre NLI (IT) results.}
% %\resizebox{\columnwidth}{!}{%
% \begin{tabular}{|c|c|c|c|c|}
% \hline
% Dataset & Task & Metric & Result & Delta\\ \hline
% \begin{tabular}[c]{@{}c@{}}Multi NLI-Mismatched (IT)\end{tabular} &
%   NLI &
%   Accuracy &
%   72.74\% &
%   \textbf{+1.09\%} \\ \hline
% \begin{tabular}[c]{@{}c@{}}Multi NLI-Mismatched (IT)\end{tabular} &
%   NLI &
%   Min F1-Score &
%   64.53\% &
%   \textbf{+0.55\%} \\ \hline
%   \begin{tabular}[c]{@{}c@{}}Multi NLI-Mismatched (IT)\end{tabular} &
%   NLI &
%   Macro-Avg F1-Score &
%   72.78\% &
%   \textbf{+1.37\%} \\ \hline
% \end{tabular}%
% %}
% \end{table}
Differently from the SNLI dataset, we briefly remark that MNLI datasets are divided into genres, and supports a distinctive cross-genre generalisation evaluation by means of the mismatched validation set. A higher accuracy on the mismatched validation set corresponds to a better generalisation of the model.
In the same way as before, also for this test the Machine Translation-based architecture had an objective advantage, by being trained on the same dataset it was tested over. Nonetheless, the KD-based architecture performed better in this test. This dataset tests the generalisation capability and the ability to understand a wide range of contexts of a model, as it contains multiple genres. This could be a motivation to consider the KD-based architecture the most powerful architecture of the two.

In addition to the tests above, the architecture has been tested over the RTE datasets.
We briefly remind that, unlike classical NLI, these datasets present only two labels: ``Entailment'' and ``No-Entailment''. Both our models produce three labels --- ``Entailment'', ``Neutral'', ``Contradiction'' --- as they were trained on SNLI and MNLI. 
% A mapping is required:
% \begin{center}
% Entailment \textrightarrow{} Entailment \\
% Neutral \textrightarrow{} No-Entailment \\
% Contradiction \textrightarrow{} No-Entailment \\
% \end{center}
% This mapping is the one that maximises the accuracy on the validation set.
The two-label mapping for this task maps both \emph{Neutral} and \emph{Contradiction} to \emph{No-Entailment}%has been chosen to be:
% \begin{center}
% Entailment \textrightarrow{} Entailment \\
% Neutral \textrightarrow{} Entailment \\
% Contradiction \textrightarrow{} Contradiction \\
% \end{center}
%This mapping is the one that 
 as this maximises the accuracy on the validation set.
Results for the RTE3-ITA and RTE-2009 test sets are reported in Tab.\ref{nli_ita_res} too.
% \begin{table}[H]
% \label{rte res} 
% \centering
% \caption{RTE results.}
% %\resizebox{\columnwidth}{!}{%
% \begin{tabular}{|c|c|c|c|c|}
% \hline
% Dataset & Task & Metric  & Result & Delta\\ \hline
% RTE3-ITA & NLI & Accuracy     & 67.50\% & \textbf{+4.75\%} \\ \hline
% RTE3-ITA & NLI & Min F1-Score  & 60.12\% & \textbf{+5.55\%}\\ \hline
% RTE3-ITA & NLI & Macro-Avg F1-Score  & 66.35\% & \textbf{+4.85\%} \\ \hline
% RTE-2009 & NLI & Accuracy      & 59.00\% & -0.75\%        \\ \hline
% RTE-2009 & NLI & Min F1-Score  & 31.09\% & -2.65\%        \\ \hline
% RTE-2009 & NLI & Macro-Avg F1-Score  & 50.96\% & -1.46\% \\ \hline
% \end{tabular}%
% %}
% \end{table}
The performance difference between the two architectures may be explained by the difference in quality of the target language the two architectures have been exposed to during training. In fact, Machine Translation-based architecture has been trained over a machine translated dataset, whereas the KD-based architecture was trained over a manually translated dataset. This supposition can be made because this dataset has been manually translated in Italian, hence presents a better language quality than the NLI datasets translated in the target language.

\subsubsection{ABSA results} \label{ABSA results}
Aspect-Based Sentiment Analysis at EVALITA (ABSITA), detailed in~\cite{EVALITA}, is an ABSA dataset. 
Contains Italian hotel reviews that may touch different topics (such as price, location, cleanliness, etc.) and a sentiment associated to each topic (knowing that sentiments for different topics may be contrasting). 
By choosing arbitrary NLI hypotheses, this dataset may emulate a total of three (3) different tasks, namely SA, TR, and ABSA. 
The core idea behind this setting comes from the desire to query a text --- in NLI, a set of premises (e.g. a set of reviews), in an unsupervised way, to receive specific answers from a predefined list of answers (e.g. the presence of a topic from a list of topics). 
In the case of open answers, a question-answer architecture would have been more suitable.
\begin{table}
\centering
\caption{\label{absita res} ABSITA results, over the Sentiment Analysis and Topic Recognition tasks}
%\resizebox{\columnwidth}{!}{%
\begin{tabular}{|c|c|c|c|c|c|}
\hline
Dataset & Balancing &Task& Acc. & Min F1 & Macro-Avg F1 \\ \hline
ABSITA  & 1:1 & SA&  88.12\% (\textbf{+3.08\%} )  & 86.89\% (\textbf{+3.19\%})    &  
88.02\% (\textbf{+3.09\%}) \\ \hline
ABSITA  & 1:1 & TR&  68.09\% (-3.1\% )  & 65.75\% (-3.01\%)    &  
67.97\%  (-3.16\% ) \\ \hline
ABSITA  & 1:7 & TR&  71.11\%  (\textbf{+5.27\%} )  & 37.94\% (-0.77\%)    &  
59.56\%   (\textbf{+2.04\%} ) \\ \hline
ABSITA & 1:1  & ABSA &  94.03\% (\textbf{+6.24\%}) &  93.90\% (\textbf{+6.65\%})& 94.02\% (\textbf{+6.35\%}) \\ \hline
ABSITA & 1:15 & ABSA & 78.42\% (\textbf{+11.39\%})  & 37.66\% (\textbf{+8.3\%}) & 62.30\% (\textbf{+8.37\%}) \\ \hline
\end{tabular}%
%}
\end{table}
% \begin{table}[H]
% \centering
% \caption{\label{absita res} ABSITA results, over the Sentiment Analysis task.}
% %\resizebox{\columnwidth}{!}{%
% \begin{tabular}{|c|c|c|c|c|c|}
% \hline
% Dataset & Balancing & Task & Metric & Result & Delta \\ \hline
% ABSITA  & 1:1                            & Sentiment Analysis                        & Accuracy                                              & 88.12\% & \textbf{+3.08\%}               \\ \hline
% ABSITA  & 1:1                            & Sentiment Analysis                        & Min F1-Score                                             & 86.89\% & \textbf{+3.19\%}                  \\ \hline
% ABSITA  & 1:1                            & Sentiment Analysis                        & Macro-Avg F1-Score                                             & 88.02\% & \textbf{+3.09\%}                  \\ \hline
% \end{tabular}%
% %}
% \end{table}

%\paragraph {Sentiment Analysis} \label {Sentiment Analysis}
\textit{Sentiment Analysis}
 (SA) is the task to recognise the overall sentiment of a sentence. As detailed above, we would like to exploit the models to apply SA in an unsupervised manner --- to do this, we fix a hypothesis arbitrarily. We assume that the hypothesis we have chosen captures the logical implication that is the core of NLI.
 Results for the ABSITA dataset are detailed in Tab.~\ref{absita res}. Note that the hypothesis has been arbitrarily set to ``Sono soddisfatto'' (``I feel satisfied''), hence ``Entailment'' refers to the model predicting positive sentiment.
The two-label mapping for this task maps \emph{Neutral} to \emph{Entailment}.%has been chosen to be:
% \begin{center}
% Entailment \textrightarrow{} Entailment \\
% Neutral \textrightarrow{} Entailment \\
% Contradiction \textrightarrow{} Contradiction \\
% \end{center}
%This mapping is the one that 
%as this maximises the accuracy on the validation set.

%\paragraph{Topic Recognition} \label {Topic Recognition}
\textit{Topic Recognition}
 (TR) is the task to recognise whether or not a sentence is about a topic. As detailed above, we would like to exploit the models to apply TR in an unsupervised manner --- to do this, we fix a hypothesis arbitrarily. We assume that the hypothesis we have chosen captures the logical implication that is the core of NLI.
 Results for the ABSITA dataset are detailed in Tab.~\ref{absita res}. %{tr res}. 
The seven (7) in the ``Balancing'' column stands for the number of different topics in the dataset. The 1:1 balancing has been obtained by randomly sampling sentences from the seven (7) classes that do not compose the target. The two scenarios have been proposed to extensively test the generalisation capability of the models.
Note that the hypothesis has been arbitrarily set to ``Parlo di pulizia'' (``I'm talking about cleanliness''), hence ``Entailment'' refers to the model predicting the label ``cleanliness''.
The two-label mapping for this task maps \emph{Neutral} to \emph{Entailment}.%has been chosen to be:
% \begin{center}
% Entailment \textrightarrow{} Entailment \\
% Neutral \textrightarrow{} Entailment \\
% Contradiction \textrightarrow{} Contradiction \\
% \end{center}
%This mapping is the one that 
%as this maximises the accuracy on the validation set.
% \begin{table}[H]
% \centering
% \caption{\label{tr res} ABSITA results, over the Topic Recognition task.}
% %\resizebox{\columnwidth}{!}{%
% \begin{tabular}{|c|c|c|c|c|c|}
% \hline
% Dataset & Balancing & Task & Metric &  Result & Delta\\ \hline
% ABSITA & 1:1 & Topic Recognition & Accuracy           & 68.09\% & -3.1\% \\ \hline
% ABSITA & 1:1 & Topic Recognition & Min F1-Score  & 65.75\% & -3.01\% \\ \hline
% ABSITA & 1:1 & Topic Recognition & Macro-Avg F1-Score  & 67.97\% & -3.16\% \\ \hline
% ABSITA & 1:7 & Topic Recognition & Accuracy            & 71.11\% & \textbf{+5.27\%} \\ \hline
% ABSITA & 1:7 & Topic Recognition & Min F1-Score  & 37.94\%  & -0.77\%       \\ \hline   
% ABSITA & 1:7 & Topic Recognition & Macro-Avg F1-Score  & 59.56\% & \textbf{+2.04\%} \\ \hline
% \end{tabular}%
% %}
% \end{table}
% Where the seven (7) in the ``Balancing'' column stands for the number of different topics in the dataset. The 1:1 balancing has been obtained by randomly sampling sentences from the seven (7) classes that do not compose the target. The two scenarios have been proposed to extensively test the generalisation capability of the models.
% The two-label mapping for this task has been chosen to be:
% \begin{center}
% Entailment \textrightarrow{} Entailment \\
% Neutral \textrightarrow{} Entailment \\
% Contradiction \textrightarrow{} Contradiction \\
% \end{center}
% This mapping is the one that maximises the accuracy on the validation set.

%\paragraph{Aspect-Based Sentiment Analysis} \label{Aspect-Based Sentiment Analysis}
\textit{Aspect-Based Sentiment Analysis}
 (ABSA) is the task to recognise the sentiment about each sub-topic in a sentence. As detailed above, we would like to exploit the models to apply ABSA in an unsupervised manner --- to do this, we fix a hypothesis arbitrarily. We assume that the hypothesis we have chosen captures the logical implication that is the core of NLI.
Results for the ABSITA dataset are detailed in Tab.~\ref{absita res}. %absa res}. 
% The fifteen (15) in the ``Balancing'' column stands for the number of different tuples (topic, sentiment) in the dataset. The 1:1 balancing has been obtained by randomly sampling sentences from the fifteen (15) classes that do not compose the target. The two scenarios have been proposed to extensively test the generalisation capability of the models.
Note that the hypothesis has been arbitrarily set to ``La camera \'e pulita'' (``The room is clean''), hence ``Entailment'' refers to the model predicting positive sentiment and ``cleanliness'' label.
The two-label mapping for this task maps \emph{Neutral} to \emph{Contradiction}.%has been chosen to be:
% \begin{center}
% Entailment \textrightarrow{} Entailment \\
% Neutral \textrightarrow{} Entailment \\
% Contradiction \textrightarrow{} Contradiction \\
% \end{center}
%This mapping is the one that 
%as this maximises the accuracy on the validation set.
% \begin{table}[H]
% \centering
% \caption{\label{absita absa res} ABSITA results, over the Aspect-Based Sentiment Analysis task.}
% %\resizebox{\columnwidth}{!}{%
% \begin{tabular}{|c|c|c|c|c|c|}
% \hline
% Dataset & Balancing & Task & Metric  & Result & Delta \\ \hline
% ABSITA & 1:1  & ABSA & Accuracy     &  94.03\% & \textbf{+6.24\%}\\ \hline
% ABSITA & 1:1  & ABSA & Min F1-Score &  93.90\% & \textbf{+6.65\%}  \\ \hline
% ABSITA & 1:1  & ABSA & Macro-Avg F1-Score &  94.02\% &  \textbf{+6.35\%} \\ \hline
% ABSITA & 1:15 & ABSA & Accuracy      & 78.42\% & \textbf{+11.39\%} \\ \hline
% ABSITA & 1:15 & ABSA & Min F1-Score     & 37.66\% & \textbf{+8.3\%}  \\ \hline
% ABSITA & 1:15  & ABSA & Macro-Avg F1-Score     & 62.30\% & \textbf{+8.37\%}  \\ \hline
% \end{tabular}%
% %}
% \end{table}
% Where the fifteen (15) in the ``Balancing'' column stands for the number of different tuples (topic, sentiment) in the dataset. The 1:1 balancing has been obtained by randomly sampling sentences from the fifteen (15) classes that do not compose the target. The two scenarios have been proposed to extensively test the generalisation capability of the models.
% The two-label mapping for this task has been chosen to be:
% \begin{center}
% Entailment \textrightarrow{} Entailment \\
% Neutral \textrightarrow{} Contradiction \\
% Contradiction \textrightarrow{} Contradiction \\
% \end{center}
% This mapping is the one that maximises the accuracy on the validation set.

\section{Conclusions and Discussions}
\label{sec:discussion}
To interpret the apparently decent results for NLI in the source language, listed in  Sec.~\ref{NLI results in the source language}, we need to consider the fact that, during training, sentence encoders do not look at both inputs simultanously, hence generating good but not top-tier performances. 
Potentially, we could have obtained slightly better results by making use of a word encoder instead of a sentence encoder, at the cost of a large computational overhead.
To address various industrial tasks, we decided to prioritise scalability and responsiveness.
The discussed architecture, based on KD, demonstrated to perform better than the other architecture --- that was directly trained over machine translated NLI datasets --- despite having an objective disadvantage. 
We stress the fact that the proposed architecture was never directly trained over any kind of Italian NLI data.
Compared to the other methodology, the KD presents the following advantages: 
\begin{enumerate}
    \item Easier to extend models: we just require few samples for the new languages.
    \item Lower hardware requirements: machine translation --- that is an expensive task --- is not needed as an intermediate step.
\end{enumerate}

To test our model performances over SA, TR, and ABSA, we employed arbitrary hypotheses. 
We tried our best to avoid any biases (e.g. hypotheses were chosen by colleagues that had never taken a look at the datasets), but we acknowledge that some bias may have been introduced. This is currently considered an open problem.
% \section{Conclusions}
% \label{sec:concl}

Different architectures have been tested showing that it is possible to obtain reasonable accuracies over different NLP tasks by fine-tuning a single architecture based on sentence embeddings over the NLI task. 
We showed that various NLP problems may be mapped into a NLI task --- in this way, we empirically proved the generality of the NLI task. 
We would like to stress over the lack of need to re-train any models to obtain the results over each specific task.
Moreover, lately NLI models find an important academic usage for boosting the consistency and accuracy of NLP
models without fine-tuning or re-training~\cite{CONCORD}. This is because models should demonstrate internal self-consistency, in the sense that their predictions across inputs should imply logically compatible beliefs about the world --- NLI models are trained to achieve that understanding.
%\subsubsection{Acknowledgements} Please place your acknowledgments at the end of the paper, preceded by an unnumbered run-in heading (i.e. 3rd-level heading).
\bibliographystyle{splncs04}
\bibliography{asdfromeyetracks}
\newpage
\appendix
\section{Dataset examples}
\label{appendix:data}
Examples from the benchmark Stanford NLI dataset are shown in Tab.~\ref{snli table} to show its standard structure.
% An example for \emph{Entailment} from the SNLI dataset is:
% \begin{enumerate}[align=left]
%     \item [Premise:] ``A soccer game with multiple males playing''
%     \item  [Hypothesis:] `Some men are playing a sport''
% \end{enumerate}
\begin{table}%
[H]
  \centering
  %\resizebox{\columnwidth}{!}{%
  \label{snli table} 
  \caption{Stanford NLI dataset. A label is produced based on the logical interaction between two short texts.}
  \begin{tabular}{|c|c|c|}
    \hline
    Premise & Hypothesis & Label \\ \hline
    \begin{tabular}[c]{@{}c@{}}``A soccer game with \\ multiple males playing''\end{tabular} &
    \begin{tabular}[c]{@{}c@{}}``Some men are \\ playing a sport''\end{tabular} & Entailment \\ \hline
    \begin{tabular}[c]{@{}c@{}}``An older and younger \\ man smiling''\end{tabular} &
    \begin{tabular}[c]{@{}c@{}}``Two men are smiling\\  and laughing at the cats\\  playing on the floor''\end{tabular} & Neutral \\ \hline
    \begin{tabular}[c]{@{}c@{}}``A man inspects the\\  uniform of a figure \\ in some East Asian country''\end{tabular} & ``The man is sleeping'' & Contradiction \\ \hline
  \end{tabular}%
  %}
\end{table}

% An example for \emph{Neutral} from Multi-Genre NLI dataset is:
Examples from the benchmark Multi-Genre NLI dataset are shown in Tab.~\ref{mnli table} to show its standard structure.
% \begin{enumerate}[align=left]
%     \item [Premise:] ``Theoretically scale economies in delivery are not firm specific.''
%     \item  [Hypothesis:] ``Scale economies are flexible.''
% \end{enumerate}
\begin{table}%[H]
  \centering
  %\resizebox{\columnwidth}{!}{%
  \label{mnli table} 
  \caption{Multi-Genre NLI dataset. A label is produced based on the logical interaction between two short texts.}
  \begin{tabular}{|c|c|c|}
    \hline
    Premise & Hypothesis & Label \\ \hline
    \begin{tabular}[c]{@{}c@{}}``It has a staff of about 100 employees, \\ including attorneys and support staff, \\ in 10 branch offices.''\end{tabular} &
    \begin{tabular}[c]{@{}c@{}}``The 10 branches had close \\ to 100 employees.''\end{tabular} & Entailment \\ \hline
    \begin{tabular}[c]{@{}c@{}}``Theoretically scale economies in \\ delivery are not firm specific.''\end{tabular} &
    \begin{tabular}[c]{@{}c@{}}``Scale economies are flexible.''\end{tabular} & Neutral \\ \hline
    \begin{tabular}[c]{@{}c@{}}``Mrs. Cavendish is in her \\ mother-in-law's room. ''\end{tabular} &
    \begin{tabular}[c]{@{}c@{}}``Mrs. Cavendish has left \\ the building.''\end{tabular} & Contradiction \\ \hline
  \end{tabular}%
  %}
\end{table}
% If the hypothesis may be inferred from the premise, the NLI task may be reinterpreted as a task of information extrapolation. We could query a dataset of reviews to extract any custom information, like the topic or sentiment.
% To make an example, let us consider the following statements (in Italian) as in Tab.~\ref{Table 2}:
% \begin{table}[H]
% \centering
% \caption{\label{Table 2} Information extraction (in Italian). The label indicates whether the review has a relation with the query.}
% %\resizebox{\columnwidth}{!}{%
% \begin{tabular}{|c|c|c|}
% \hline
% Dataset &
%   Balancing &
%   Task \\ \hline
% \multirow{2}{*}{\begin{tabular}[c]{@{}c@{}}``Un po' fuori mano, proprio come il nome.\\ Per\'o \'e un bel locale. Cibo normale, e prezzi un po' alti.''\end{tabular}} &
%   ``Prezzo alto'' &
%   Entailment \\ \cline{2-3} 
%  &
%   ``Prezzo basso'' &
%   \multicolumn{1}{l|}{Contradiction} \\ \hline
% \end{tabular}%
% %}
% \end{table}
% \todo{sistemare formattazione Table 3}
% The English translated table may be found in Tab.~\ref{Table 3}.
% \begin{table}[H]
% \centering
% \caption{\label{Table 3} Tab.~\ref{Table 2} translated in English.}
% %\resizebox{\columnwidth}{!}{%
% \begin{tabular}{|c|c|c|}
% \hline
% Dataset &
%   Balancing &
%   Task \\ \hline
% \multirow{2}{*}{\begin{tabular}[c]{@{}c@{}}``A tad out-of-the-way, just like its name.\\ The location is nice though. Average food, \\ prices a bit on the high-end.''\end{tabular}} &
%   ``High price'' &
%   Entailment \\ \cline{2-3} 
%  &
%   ``Low price'' &
%   \multicolumn{1}{l|}{Contradiction} \\ \hline
% \end{tabular}%
% %}
% \end{table}

Examples from the RTE3-ITA NLI dataset are shown in Tab.~\ref{rte tab}. Note the table will be proposed in the Italian language.
% \begin{enumerate}[align=left]
%     \item [Premise:] ``La signora Minton lasciò
% l’Australia nel 1961 per proseguire i suoi studi a Londra.''
%     \item  [Hypothesis:] ``La signora Minton è
% nata in Australia.''
% \end{enumerate}
\begin{table}%[H]
  \centering
  %\resizebox{\columnwidth}{!}{%
  %\label{mnli table} 
  \label{rte tab} 
  \caption{RTE3-ITA dataset. A label is produced based on the logical interaction between two short texts.}
  \begin{tabular}{|c|c|c|}
    \hline
    Premise & Hypothesis & Label \\ \hline
    \begin{tabular}[c]{@{}c@{}}``All'uscita del gioco \\ Final Fantasy III nella versione \\ per la console Super Nintendo, \\ il nome di Bigg era Vicks.''\end{tabular} &
    \begin{tabular}[c]{@{}c@{}}``Final Fantasy III venne \\ prodotto per la console \\ Super Nintendo.''\end{tabular} & Entailment \\ \hline
    \begin{tabular}[c]{@{}c@{}}``La signora Minton lasciò \\ l'Australia nel 1961 per \\ proseguire i suoi studi a Londra.''\end{tabular} &
    \begin{tabular}[c]{@{}c@{}}``La signora Minton è \\ nata in Australia.''\end{tabular} & Contradiction \\ \hline
  \end{tabular}%
  %}
\end{table}

Examples from the TED2020 translation dataset are shown in Tab.~\ref{ted}.
\begin{table}%[H]
\centering
\resizebox{\columnwidth}{!}{%
\begin{tabular}{|c|c|}
\hline
sentence$_{en}$ & sentence$_{it}$ \\ \hline
\begin{tabular}[c]{@{}c@{}}``I gave my speech, then went back \\ to the airport to fly back home.''\end{tabular} &
  \begin{tabular}[c]{@{}c@{}}``Io feci il mio discorso, poi \\ andai all'aeroporto per tornare.''\end{tabular} \\ \hline
\begin{tabular}[c]{@{}c@{}}``He romanticised the idea \\ they were star-crossed lovers.''\end{tabular} &
  \begin{tabular}[c]{@{}c@{}}``Lui fantasticava sull'idea di \\ loro come amanti sfortunati.''\end{tabular} \\ \hline
\begin{tabular}[c]{@{}c@{}}``In Japan, a game of ping-pong \\ is really like an act of love.''\end{tabular} &
  \begin{tabular}[c]{@{}c@{}}``In Giappone, una partita di ping-pong\\  \'e come un atto d'amore.''\end{tabular} \\ \hline
\end{tabular}%
}
\caption{\label{ted} TED2020 dataset (English--Italian version).}
\end{table}

\section{Models parameters}
\label{appendix:param}
Fully-connected Feed Forward architecture used for classification in Sec.~\ref{NLI training in the source language} is reported here:

\noindent
\begin{Verbatim}[tabsize=4]
(layers): ModuleList(
     (0): Linear(in=1536, out=1024, activation=GELU())
     (1): Linear(in=1024, out=512, activation=GELU())
     (2): Linear(in=512, out=256, activation=GELU())
     (3): Linear(in=256, out=128, activation=GELU())
     (4): Linear(in=128, out=64, activation=GELU())
     (5): Linear(in=64, out=3, activation=GELU())
)
\end{Verbatim}

Hyper-parameters for NLI model training in the source language Sec.~\ref{NLI training in the source language} are listed here:

\begin{itemize}
\item \texttt{batch\_size = 8}
\item \texttt{max\_sentence\_length = 256}
\item \texttt{max\_tokens\_length = 128}
\item \texttt{epochs = 1}
\item \texttt{learning\_rate = 2e-5}
\item \texttt{epsilon = 1e-8}
\item \texttt{weight\_decay = 0}
\item \texttt{accumulation\_step = 8}
\end{itemize}

Hyper-parameters for NLI model training in the source language Sec.~\ref{Knowledge Distillation in the target language} are listed here:

\begin{itemize}
\item \texttt{batch\_size = 24}
\item \texttt{max\_sentence\_length = 256}
\item \texttt{max\_tokens\_length = 128}
\item \texttt{epochs = 6}
\item \texttt{learning\_rate = 2e-5}
\item \texttt{epsilon = 1e-6}
\item \texttt{weight\_decay = 1e-2}
\item \texttt{accumulation\_step = 4}
\end{itemize}

Hyper-parameters for NLI model training in the source language Sec.~\ref{Machine Translation in the target language} are listed here:

\begin{itemize}
\item \texttt{batch\_size = 8}
\item \texttt{max\_sentence\_length = 256}
\item \texttt{max\_tokens\_length = 256}
\item \texttt{epochs = 5}
\item \texttt{learning\_rate = 4e-5}
\item \texttt{epsilon = 1e-16}
\item \texttt{weight\_decay = 1e-4}
\item \texttt{accumulation\_step = 4}
\end{itemize}

\end{document}
