\section{Conclusion}
\label{sec:6}

This article presents a new VNA calibration method based on partially defined standards. The proposed SRM method uses one-port symmetric standards, a two-port reciprocal device, a combination of the reciprocal device with the one-port device, and a match standard. Only the match standard must be characterized among all standards, defining the calibration’s reference impedance.

We have extended our proposed method to the particular case of an on-wafer setup, where the probes are fixed in distance. To do this, we restricted the two-port reciprocal device to be symmetric, allowing us to use half of it to define the network-load standards.

To validate the effectiveness of the proposed SRM method, we conducted coaxial measurements, demonstrating its capability to achieve results comparable to metrology-level accuracy while relying solely on the provided S-parameters of the match standard. Additionally, an MC analysis of the SRM method was performed using synthetic data based on CPW model and error boxes derived from an actual on-wafer measurement setup. This numerical analysis aimed to evaluate various sources of uncertainty impacting the calibration process. Our findings highlighted the significance of variations in symmetrical standards and the influence of the reciprocal network, which plays a crucial role as part of the network-load standards. Interestingly, crosstalk showed minor influence compared to other sources of uncertainties.

