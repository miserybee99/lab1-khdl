\section{Introduction}
\label{sec:1}

\IEEEPARstart{T}{he} most commonly used method for calibrating vector network analyzers (VNAs) is the short-open-load-thru (SOLT) method \cite{Kruppa1971}, which requires that all four standards to be fully characterized or modeled. In the past, many VNAs used a three-sampler architecture with three receivers.To account for the non-driving port's termination mismatches (switch terms), the VNA is modeled with the well-known 12-term model \cite{Rumiantsev2008}. This model forms the foundation of the SOLT calibration.

Nowadays, modern VNAs use a full-reflectometry architecture that allows for sampling all waves, thus directly measuring the switch terms of a VNA by simply connecting a transmissive device between the ports \cite{Jargon2018}. This upgraded architecture enabled the use of the simplified error box model of VNAs \cite{Marks1997}, which has led to many new advanced calibration methods that surpass the accuracy of SOLT \cite{Rumiantsev2008}. Furthermore, even with the three-sampler VNA architecture, it is possible to indirectly measure the switch terms of the VNA using a set of reciprocal devices, which enable the application of the error box model \cite{Hatab2023c}. A well-known family of calibration methods based on the error box model is the self-calibration methods \cite{Rumiantsev2008}, which do not require full characterization of some of the standards. One of the most used self-calibration methods nowadays is the short-open-load-reciprocal (SOLR) method \cite{Ferrero1992}, which is the same as SOLT, but with any transmissive reciprocal device instead of the thru standard. SOLR has proven useful in scenarios where a direct connection is unavailable. However, the drawback of the SOLR method is the requirement of the full definition of the short-open-load (SOL) standards, which bounds the accuracy of SOLR to the SOL standards.

Other self-calibration methods include thru-reflect-line (TRL) and multiline TRL \cite{Engen1979,Marks1991,Hatab2022,Hatab2023}, which use line standards of different lengths, thru connection, and symmetric unknown reflect standard. The thru standard in TRL is fully defined. However, there is an implementation that eliminates the requirement of the thru standard for any transmissive device with an additional reflect standard \cite{Hatab2023b}. While multiline TRL is a very accurate calibration method, especially at millimeter-wave frequencies, it cannot be applied at lower frequencies, as it results in using an extremely long line standard. A common replacement for the multiline TRL method for on-wafer application is the line-reflect-match (LRM), thru-match-reflect-reflect (TMRR), and line-reflect-reflect-match (LRRM) methods \cite{Eul1988,Zhao2017,Rumiantsev2018,Hayden2006}. These methods use unknown symmetric reflect standards and one known match standard. However, these methods suffer from some impracticality, especially in defining the line standard and shifting the reference plane, as opposed to the TRL method. These methods can also be extended to account for crosstalk \cite{Williams2014,Dahlberg2014,Wang2023}. Additionally, due to the requirement of defining the thru/line standard, such methods can be challenging to use in on-wafer measurement scenarios where the probes are orthogonal or at an angle \cite{Basu1997}.

In this paper, we propose a new approach to self-calibration of VNAs using multiple symmetric one-port loads, a two-port reciprocal device, and a matched load. The multi-load one-port standards are two-port symmetric loads, and at least three unique loads must be used. The values of the loads themselves are not specified. For example, a short, an open, and any finite impedance load would be suitable. The reciprocal device can be any transmissive device. In fact, if we only care about the one-port error boxes of the VNA, then the two-port device can be any transmissive device, even if it is non-reciprocal. Lastly, the matched load is fully defined but can be asymmetric. The match standard can be implemented as part of the symmetric one-port loads to reduce the number of standards. We refer to this calibration method as the symmetric-reciprocal-match (SRM) method. All standards are generally partially defined, except for the match standard. We demonstrate the method using synthetic data of coplanar waveguide (CPW) structures, as well as measurements with commercial SOLR coaxial standards.

A significant benefit of the proposed approach is that all the standards are partially defined, except for the match standard. This is in contrast to LRRM/LRM/TMRR approaches, which necessitate fully defined thru/line standards. As a result, such techniques can be challenging in the case of on-wafer setups where the probes are positioned at an orthogonal angle. Equivalently, the SOLR calibration addresses the problem of the thru/line connection by using any two-port reciprocal device instead but necessitates the specification of the remaining standards. In brief, our SRM technique combines the benefits of LRRM/LRM/TMRR techniques in utilizing undefined symmetric standards, as well as the SOLR technique in utilizing a two-port reciprocal device. This revised definition of the standards enables accurate calibration by limiting the definition to solely the match standard.

The remainder of this article is structured as follows. In Section~\ref{sec:2}, we discuss our SRM method when using a thru standard instead of any reciprocal device, highlighting the method's fundamentals. Afterward, in Section~\ref{sec:3}, we extend the mathematics of the calibration to consider any transmissive reciprocal device. Section~\ref{sec:4} introduces a special case of the SRM method when considering a fixed distance between measuring ports, which is often the case in on-wafer applications. Lastly, in Section~\ref{sec:5}, we provide experimental measurements using commercial METAS traceable coaxial $2.92\,\mathrm{mm}$ calibration and verification standards, as well as numerical Monte Carlo analysis using synthetic data. Finally, we conclude in Section~\ref{sec:6}.

% EOF